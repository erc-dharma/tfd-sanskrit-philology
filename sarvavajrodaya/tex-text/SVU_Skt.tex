%%% XeLaTeX-article%%%
%!TEX encoding = UTF-8 Unicode
%!TEX TS-program = xelatex 
%\documentclass[a4paper,11pt]{bxjsarticle_ver2}
\documentclass[a4paper,11pt]{book}

\usepackage{4bxjspreamble2}
\usepackage{graphicx}
\usepackage{tanemura2}
\usepackage[noeledmac]{ledmac}
\usepackage{color}
\usepackage{endnotes}
\usepackage{SVU}
% \usepackage{udline}
% \usepackage{ulem}
% change of lemma sign
%\renewcommand*{\rbracket}{\textbf{\thinspace]\kern -0.1667em ]}}
\renewcommand*{\rbracket}{}

\lineation{page}
\footparagraph{A}
\footparagraph{B}
\footparagraph{D}
\footparagraph{E}
% \linenummargin{inner}

% \def\verse{\pstart \hsize 378pt \rightskip 40pt \noindent}% for bxjsarticle
\def\verse{\pstart \hsize 330pt \rightskip 40pt \noindent}
\def\pquote{\pstart\hsize 350pt \rightskip 40pt \noindent}
\def\lverse{\pstart \hsize 310pt \noindent}
\def\mantra{\pstart \hsize 350pt \rightskip 40pt \noindent}
\def\smantra{\pstart \hsize 310pt \noindent}

\def\ac{$^{ac}$}
\def\pc{$^{pc}$}

\newcommand{\siddham}{\textcolor{red}{〜}}
\newcommand{\flower}{◎}

%to make proper m with candrabindu/anunāsika
\newcommand{\anunasikar}{\XeTeXglyph2303}%in roman
\newcommand{\anunasikai}{\XeTeXglyph1505}%in italic

\renewcommand\endashchar{\textnormal{\mbox{}\textendash\mbox{}}} % 異読註の行番号が複数にわたる場合長いハイフンを使用

%\newfontfamily\sanskritfont[Script=Devanagari,Mapping=romantodevanagari,Scale=1.3]{Sanskrit 2003}
%\newfontfamily\tibetanfont[Script=Tibetan,Mapping=romantotibetan,Scale=1.2,WordSpace=0.4]{Jomolhari}

\title{\textit{Sarvavajrodayā}}

\author{Arlo \textsc{Griffiths}, Ryugen \textsc{Tanemura} and Shiori \textsc{Ijuin}}
\date{\today}
\pagestyle{myheadings}
%\pagestyle{plain}

%%%%%%    TEXT START    %%%%%%
\begin{document}
\maketitle
\markboth{Griffiths and Tanemura}{SVU Preliminary Edition}

\beginnumbering

\noindent
{\large 1. Maṅgala}

\bigskip

\noindent
{\large 2. Trisamādhi (Pūrvasevā)}

\bigskip

\noindent
{\large 2.1. Ādiyoga}

\bigskip

% maṇḍalarājāgrī


\pstart
% \linebreak{21r1}
\Skt{\edtext{\dots\ \edtext{cintanīyā}{\lemma{%
	{\rm cintanīyā ity\lem}
}\Dfootnote{%
	\emn;
	\textit{cintanīyaty} \cod
}\lemma{}\Efootnote{%
	(folio missing) [21r1] cintanīyety
}\lemma{%
	{\rm 21r1\lem}
}\Cfootnote{%
	The scribe makes no clear interruption for the string hole in line 1, where the elaborate punctuation sequence occupies precisely this space.
}} ityādiyogo nāma samādhiḥ  //  //}{\lemma{%
	{\rm cintanīyā \dots\ samādhiḥ\lem}
}\Cfootnote{%
%
\MSK\ omits this part.
%
}} }
\pend

\bigskip

\pstart\noindent
{\large 2.2. Maṇḍalarājāgrī}
\pend

\bigskip

\pstart\noindent
{\large 2.2.1. Emanation of the Sixteen Bodhisattvas}
\pend

\medskip

\pstart\noindent
2.2.1.1. Vajrasattva
\pend

\medskip

\pstart
%($\rightarrow$KSP)
\Skt{\edtext{tadanu \edtext{vajradhātvīśvaryā}{\lemma{%
	{\rm vajradhātvīśvaryā\lem}
}\Dfootnote{%
	\emn\ \MSK\ ($\leftarrow$ KSP, Tib.);
	\textit{vajradhātvīśvaryādi} \cod
%	\MSK\ emends \textit{vajradhātvīśvaryā}, with Tib.
}} caturṣu sthāneṣvātmānamadhiṣṭhāya bodhyagrīṃ baddhvā \edtext{hṛdgatavajrasattvaṃ}{\lemma{%
	{\rm hṛdgataṁ vajrasatvañ\lem}
}\Dfootnote{%
	\emn\ \MSK;
	\textit{hṛdgatāvajrasatvañ} \cod;
	\textit{hṛdgatavajrasattvaṁ} KSP (\Ied)
%	\textcolor{red}{or emend hṛdgatavajrasatvañ}, with MSK?
}} codayedvajrasattveti//}{\lemma{%
	{\rm tad anu \dots\ vajrasattveti\lem}
}\Bfootnote{%
%
%KSP, chapter 6
%\textit{tad anu vajradhātvīśvaryā catuḥsthāneṣv ātmānaṁ adhiṣṭhāya bodhyagrīmudrāṁ baddhvā hṛdgatavajrasattvaṁ codayet vajrasattva iti}. 
%(\textsc{Inui} 1994: 110) KSP reads \textit{hṛdgatavajrasattvaṁ} for \textit{hṛdgataṁ vajrasattvaṁ};
%(From this passage onwards, the KSP has parallels. most probably the KSP borrows them from the SVU.)
%
Cf.\ STTS 35
%\textit{sarvatathāgatasamantabhadramahābodhisattvasamayasambhavasattvādhiṣṭhānavajraṁ nāma samādhiṁ samāpadyedaṁ sarvatathāgatamahāyānābhisamayaṁ nāma sarvatathāgatahṛdayaṁ svahṛdayān niścacāra. vajrasattva}.
%
}} }
	%cintanīyaty: em. cintanīyety.
	%The scribe makes no clear interruption for the string hole in line 1, where the elaborate punctuation sequence occupies precisely this space.
	%vajradhātvīśvaryādi: MSK emends vajradhātvīśvaryā, with Tib.
	%hṛdgatāvajrasatvañ: em. hṛdgataṁ vajrasatvañ; or hṛdgatavajrasatvañ, with MSK.
\pend

\pstart
\Skt{\edtext{tataḥ sarvatathāgatahṛdayebhyaḥ candramaṇḍalāni bhūtvā viniḥsṛtya yāvatsarvalokadhātuṣu sarvasattvānāṃ nairātmyaprativedhaṃ kārayitvā \edtext{candramaṇḍalākāraṃ cittaikāgratāṃ}{\lemma{
	{\rm candramaṇḍalākāraṁ\lem}
}\Dfootnote{%
	\emn\ ($\leftarrow$KSP (\Ied));
	\textit{candramaṇḍalākārāṁ cittaikāgratāṁ} \cod;
	\textit{candramaṇḍalākāraṁ cittaikāgratāṁ} \emn\ \MSK\ \sil
}} niṣpādyāgatya sarvadevatāsthāneṣu candramaṇḍalānye\edtext{va}{\lemma{%
	{\rm eva\lem}
}\Dfootnote{%
	\cod\ KSP (\Ied);
	\textit{evaṁ} \emn\ \MSK
%	\MSK\ emends to \textit{evaṁ}.
}} bhūtvāvasthitaṃ cintayet //}{\lemma{%
	{\rm tataḥ \dots\ cintayet\lem}
}\Bfootnote{%
%
KSP, chapter 6
% \textit{tataḥ sarvatathāgatahṛdayebhyaś candramaṇḍalāni bhūtvā viniḥsṛtya yāvat sarvalokadhātuṣu sarvasattvānāṁ nairātmyaprativedhaṁ kārayitvā candramaṇḍalākāraṁ cittaikāgratāṁ niṣpādyāgatya sarvadevatāsthāneṣu candramaṇḍalāny eva bhūtvāvasthitaṁ cintayet}. 
(\textsc{Inui} 1994: 110);
Cf.\ STTS 36
%\textit{athāsmin viniḥsṛtamātre sarvatathāgatahṛdayebhyaḥ sa eva bhagavān samantabhadraś candramaṇḍalāni bhūtvā viniḥsṛtya sarvasattvānāṁ mahābodhicittāni saṁśodhya sarvatathāgatānāṁ sarvapārśveṣv avasthitaḥ.}
%
}} }
	%STTS quotation?
	%candramaṇḍalākārāṁ: em. candramaṇḍalākārāṁ, with MSK (sil. em.).
	%evaṁ: MSK reads eva which it emends to evaṁ.
\pend

\pstart
\Skt{\edtext{tatastebhyo \edtext{jñānaraśmayo}{\lemma{%
	{\rm jñānaraśmayo\lem}
}\Dfootnote{%
	\emn\ \MSK\ \sil;
	\textit{jñānarasmayo} \cod
}} viniḥsṛtya svahṛdgatavajre praviśya tena \edtext{sahaikībhūya}{\lemma{%
	{\rm sahaikībhūya\lem}
}\Dfootnote{%
	\emn;
	\textit{sahaikībhūyaḥ} \cod\ \MSK
}} / sarvatathāgatādhiṣṭhānena mahān sarvākāśasamavasaraṇapramāṇapañcasūcikavajravigraho bhūtvā~/
\edtext{punaḥ pūrvavaddhṛdgata}{\lemma{%
	{\rm punaḥ pūrvavad dhṛdgata-\lem}
}\Dfootnote{%
	\emn;
	\textit{punaḫ pūrvvahṛdgata-} \cod;
	\textit{punaḥ pūrvahṛdgata-} \emn\ \MSK\ \sil
}}vajrapramāṇameva bhūtvā \edtext{tasmā}{\lemma{%
	{\rm tasmād\lem}
}\Dfootnote{%
	\emn\ \MSK\ \sil;
	\textit{tasmā} \cod
}}dviniḥsṛtya svahastasthitaṃ cintayet / 
tasmā\edtext{cca}{\lemma{%
	{\rm ca\lem}
}\Dfootnote{%
	\emn\ \MSK\ \sil;
	\textit{caḥ} \cod
}} punaḥ \edtext{vajrākāraraśmayo}{\lemma{%
	{\rm vajrākāraraśmayo\lem}
}\Dfootnote{%
	\emn\ \MSK\ \sil;
	\textit{vajrākārarasmayo} \cod
}}'nekavarṇānekasaṃsthānā viniḥsṛtya sarvākāśadhātuṃ vyāpya / tebhyaḥ \edtext{puna}{\lemma{%
	{\rm punar\lem}
}\Dfootnote{%
	\emn\ \MSK\ \sil;
	\textit{puna} \cod
}}rvajrasattvādirūpeṇa sarvasattvadhātuṃ yāvatsarvatathāgatasamatājñānā\-bhi\edtext{saṃbo\-dhyādau}{\lemma{%
	{\rm -saṁbodhyādau\lem}
}\Dfootnote{%
	\emn\ \MSK\ \sil;
	\textit{-samvoddhyādau} \cod
}} niyojya / punaḥ saṃhārayogenaikasattvakāyo bhūtvā \edtext{svahṛdgatavajre}{\lemma{}\Efootnote{%
	svahṛdgata[21v1]vajre
}} praviśya / tatrāva\edtext{sthitenodānamu}{\lemma{%
	{\rm -sthitenodānam\lem}
}\Dfootnote{%
	\emn\ \MSK\ \sil;
	\textit{-sthitetodānam} \cod
}}\edtext{dānayamānaṃ}{\lemma{%
	{\rm udānayamānaṃ\lem}
}\Dfootnote{%
	\emn;
	\textit{udāyaṁtāṁ} \cod;
	\textit{udānayantaṁ} \emn\ \MSK;
	\textcolor{red}{CROSS REFERENCE. The same wording is found in the 2.3. Karmarājāgrī.}
}} cintayet //}{\lemma{%
	{\rm tatas \dots\ cintayet\lem}
}\Bfootnote{%
%
KSP, chapter 6
%\textit{tatas tebhyo jñānaraśmayo viniḥsṛtya hṛdgatavajre praviśya tena sahaikībhūya sarvatathāgatādhiṣṭhānena mahat sarvākāśadhātusamavasaraṇapramāṇaṁ pañcasūcikavajravigraho bhūtvā 
%*oṁ vajrātmako 'ham ity uccārayet} (\emn;  \textit{ity uccārayet oṁ vajrātmako 'ham} ed.).
%\textit{punar api hṛdgatavajrapramāṇam eva bhūtvā tasmād viniḥsṛtya svahaste sthitaṁ cintayet.
%tasmāc ca punar vajrākāraraśmayo anekavarṇā anekasaṁsthānā viniḥsṛtya yāvat sarvākāśadhātuṁ vyāpya tebhyaḥ punar vajrasattvādirūpeṇa sarvasattvadhātuṁ sarvatathāgatāṁś ca samatājñānābhisaṁbodyādau niyojya, punaḥ saṁhārayogenaikasattvakāyo bhūtvā svahṛdgatavajre praviśya tatrāvasthitenodānam udānayām āsa}. 
(\textsc{Inui} 1994: 110–109);\ 
Cf.\ STTS 36–40.
%\textit{atha tebhyaś candramaṇḍalebhyaḥ sarvatathāgatajñānavajrāṇi viniḥsṛtya bhagavato vairocanasya tathāgatasya hṛdaye praviṣṭāni.
%sudṛḍhatvāc ca vajrasattvasamādheḥ sarvatathāgatādhiṣṭhānena caikaghanaḥ sakalākāśadhātusamavasaraṇapramāṇo raśmimālī pañcamūrdhā sarvatathāgatakāyavākcittavajramayo vajravigrahaḥ prādur bhūya, sarvatathāgatahṛdayān niṣkramya pāṇau pratiṣṭhitaḥ.
%atha tato vajrād vajr'ākārāḥ raśmayo vicitravarṇarūpaḥ sarvalokadhātvābhāsanaspharaṇāh viniścaritāḥ.
%tebhyaś ca vajraraśmimukhebhyaḥ sarvalokadhātuparamāṇurajaḥsamās tathāgatavigrahāḥ viniḥsṛtya sakaladharmadhātusamavasaraṇeṣu sarvākāśadhātuparyavasāneṣu sarvalokadhātuprasarameghasamudreṣu sarvatathāgatasamatājñānābhijñāsv abhisaṁbodhāt
%sarvatathāgatamahābodhicittotpādana,samantabhadravividhacaryāniṣpādana-sarvatathāgatakulārāgaṇa-mahābodhimaṇḍopasaṁkramaṇa-saddharmacakrapravartanādīni yāvadaśeṣānavaśeṣasattvadhātu-paritrāṇa-sarvahitasukha-sarvatathāgatajñānābhijñottamasiddhiniṣpādanādīni sarvatathāgatarddhivikurvitāni saṁdarśya samantabhadratvād vajrasattvasamādheḥ sudṛdhatvāc caikaghanaḥ samantabhadramahābodhisattvakāyaḥ saṁbhūya, bhagavato vairocanasya hṛdaye sthitvedam udānam udānayām āsa}.
%
}} }
	%jñānarasmayo: em. jñānaraśmayo, with MSK (sil. em.).
	%sahaikībhūyaḥ: thus MSK; em. sahaikībhūya.
	%punaḫ pūrvvahṛdgata-: punaḥ pūrvahṛdgata- MSK (sil. em.); em. punaḥ pūrvavad dhṛdgata-?
	%tasmā: em. tasmā, with MSK (sil. em.).
	%vajrākārarasmayo: em. vajrākāraraśmayo, with MSK (sil. em.).
	%puna: em. punar, with MSK (sil. em.).
	%vajrasatvādirūpeṇa: em. vajrasatvarūpeṇa?
	%-samvoddhyādau:  em. saṁbodhyādau, with MSK (sil. em.).
	%-sthitetodānam: em. -sthitenodānam, with MSK (sil. em.).
	%udāyaṁtāṁ: em. udānayantaṁ, with MSK.
\pend

\verse
\Skt{\edtext{aho samantabhadro'haṃ dṛḍhasattvaḥ svayaṃbhuvām /\\
yaddṛḍhatvādakāyo'pi sattvakāyatvamāgata iti //}{\lemma{%
	{\rm aho \dots\ āgata\lem}
}\Bfootnote{%
%
STTS 40
%\textit{aho samantabhadro 'ham dṛḍhasattvaḥ svayaṃbhuvām |
%yad dṛḍhatvād akāyo 'pi sattvakāyatvam āgataḥ ||}
= KSP, chapter 6 (\textcolor{red}{LOCATION!!}).
%
}} }
	%STTS §40.
		%yad dṛḍhatvād: the reading ddṛ is unclear.
\pend

\pstart
\Skt{\edtext{tato \edtext{hṛdayā}{\lemma{%
	{\rm hṛdayād\lem}
}\Dfootnote{%
	\emn\ \MSK\ \sil;
	\textit{hṛdayod} \cod
}}davatīryākṣobhyasyāgrato \edtext{sthitvājñāṃ}{\lemma{%
	{\rm sthitvājñāṁ\lem}
}\Dfootnote{%
	\emn\ \MSK\ \sil;
	\textit{sthitvājñā} \cod
}} \edtext{mārgayamānaṃ}{\lemma{%
	{\rm mārga\-yamānaṁ\lem}
}\Dfootnote{%
	\emn\ \MSK;
	\textit{mārgayamānā} \cod;
	\MSK's footnote gives the impression that they read  \textit{rgayamānā}.
}} cintayet /
\edtext{tato}{\lemma{%
	{\rm tato\lem}
}\Dfootnote{%
	\emn\ \MSK\ \sil;
	\textit{tabho} \cod
}} sarvatathāgatakulacakravartitve \edtext{pañcabuddha}{\lemma{%
	{\rm pañcabuddha-\lem}
}\Dfootnote{%
	\emn;
	\textit{pañcavuddham} \cod
}}\edtext{makuṭa}{\lemma{%
	{\rm makuṭa-\lem}
}\Dfootnote{%
	\cod;
	\textit{mukuṭa-} \emn\ \MSK\ \sil
}}paṭṭābhi\edtext{ṣekenābhiṣicyānuttara}{\lemma{%
	{\rm -ṣekenābhiṣicyānuttara-\lem}
}\Dfootnote{%
	\emn;
	\textit{-ṣekenanābhiṣiñcyānuttara-} \cod;
	\MSK\ silently emends \textit{-ṣekena cābhiṣicyānuttara-}.
}}\edtext{śīlādikaṃ}{\lemma{%
	{\rm -śīlādikaṃ\lem}
}\Cfootnote{%
	KSP reads \textit{-śīlādikayā}, which is supported by SVU Tib (\textcolor{red}{reading of Tib.!!}).
}} yāvatsarvatathāgatasamatājñānā\edtext{bhisaṃbodhiniṣpādakamā}{\lemma{%
	{\rm -bhisaṁbodhiniṣpādakam\lem}
}\Cfootnote{%
	KSP reads \textit{-bhisaṁbodhiṁ niṣpādyaikam}.
}}dyavajramādya\edtext{vajrāṅkita}{\lemma{%
	{\rm -vajrāṅkita-\lem}
}\Dfootnote{%
	\emn;
	\textit{-vajrāṅkitaṁ} \cod\ \MSK
}}\edtext{ghaṇṭāsahitama\-śe\-ṣānavaśeṣasattva}{\lemma{%
	{\rm -ghaṇṭāsahitamaśeṣānavaśeṣasattva-\lem}
}\Dfootnote{%
	\emn;
	\textit{ghaṇṭasahita aśeṣānavaśeṣā-} \cod;
	\textit{ghaṇṭāsahitam aśeṣānavaśeṣa-} \emn\ \MSK 
	(\textit{ghaṇṭāsahitam} is a silent emendation)
}}dhātūnāṃ niṣpādanāya samantabhadrāya dadyāt/
	%hṛdayod: em. hṛdayād, with MSK (sil. em.).
	%sthitvājñā: em. sthitvājñāṁ, with MSK (sil. em.).
	%mārgayamānā: em. mārgayamānaṁ, with MSK; MSK's footnote gives the impression that they read  rgayamānā.
	%tabho: em. tato, with MSK (sil. em.).
	%makuṭa-: mukuṭa- MSK (sil. em.).
	%-ṣekenanābhiṣiñcyānuttara-: em. -ṣekenābhiṣicyānuttara-; MSK silently emends -ṣekena cābhiṣicyānuttara-.
	%ghaṇṭasahita: em. ghaṇṭāsahitaṁ; ghaṇṭāsahitaṁ MSK (sil. em.).
	%-*śeṣāsatva-: em. -śeṣasattva-, with MSK.
%	
tato \edtext{nāmābhiṣekādi}{\lemma{%
	{\rm nāmābhiṣekādi\lem}
}\Dfootnote{%
	\cod;
	\textit{nāmābhiṣekādiṁ} \emn\ \MSK\ \sil
}} dadyāt//}{\lemma{%
	{\rm tato \dots\ dadyāt\lem}
}\Bfootnote{%
%
KSP, chapter 6
\textit{tato hṛdayād avatīrya \dots\ pañcabuddhaṃ makuṭapaṭṭābhiṣekena cābhiṣicyānuttaraśīlādikayā \dots\ -bhisaṃbodhiṃ niṣpādyaikam ādyavajraṃ vajrāṇkitaṃ ghaṇṭāsahitam \dots} (\textsc{Inui} 1994: 109);
Cf.\ STTS 41–42:
\textit{\dots\
atha bhagavān \dots\ sarvatathāgata-\textbf{śīla}-samādhi-\dots mahājñāna-samayam, \dots\ yāvat-sarvatathāgata-samatā-jñānābhijñānuttara-mahāyānābhisamay\textbf{ottama-siddhy}avāpti-phala-hetos, \dots}
%
}} } 
\pend

\pstart
\Skt{tadanu \edtext{vajrapāṇyahaṃkāreṇa}{\lemma{%
	{\rm vajrapāṇyahaṁkāreṇa\lem}
}\Dfootnote{%
	\emn\ \MSK\ \sil;
	\textit{vajrapāṇyāhaṅkāreṇa} \cod
}} udānamudānayet / }
\pend

\verse
\Skt{\edtext{idaṃ tatsarvabuddhānāṃ siddhivajramanuttaram /\\ 
ahaṃ mama kare dattaṃ vajre vajraṃ pratiṣṭhitamiti}{\lemma{%
	{\rm idaṃ \dots\ pratiṣṭhitam\lem}
}\Bfootnote{%
	STTS 43
%	\textit{idaṃ tat sarvabuddhānāṃ siddhivajram anuttaram |
%	ahaṃ mama kare dattaṃ vajraṃ vajrapratiṣṭhitam ||}
}} //}
\pend
	%nāmābhiṣekādi: nāmābhiṣekādiṁ MSK (sil. em.). 
	%vajrapāṇyāhaṅkāreṇa: em. vajrapāṇyahaṁkāreṇa, with MSK.
	%stanza quoted from STTS §43.

\bigskip

\pstart\noindent
{\large 2.2.1.2. The Other Fifteen Bodhisattvas}
\pend

\bigskip

\pstart
\Skt{\edtext{\edtext{evaṃ}{\lemma{%
	{\rm evam\lem}
}\Dfootnote{%
	\cod;
	\textit{evam} \emn\ \MSK\ \sil
}} utpattispharaṇasaṃhāra\edtext{nilaya}{\lemma{%
	{\rm -nilaya-\lem}
}\Dfootnote{%
	\emn\ \MSK\ \sil;
	\textit{-ṇilaya-} \cod
}}\edtext{dṛḍhībhāva}{\lemma{%
	{\rm -dṛḍhībhāva-\lem}
}\Dfootnote{%
	\cod;
	\textit{-dṛḍhībhava-} \MSK
}}vairocanahṛdgatavajra\edtext{madhyāvasthitena udānamudānayamāna}{\lemma{%
	{\rm -madhyāvasthitena udānamudānayamānam\lem}
}\Dfootnote{%
	\emn;
	\textit{-madhyovasthitonodānam udānaṁ} \cod
}}mabhiṣekānantarodānaṃ ca \edtext{vajramuṣṭi}{\lemma{%
	{\rm vajramuṣṭi-\lem}
}\Dfootnote{%
	\emn\ \MSK\ \sil;
	\textit{vajramuṣṭhi-} \cod
}}paryantena draṣṭavyamiti}{\lemma{%
	{\rm evam \dots\ draṣṭavya iti\lem}
}\Bfootnote{%
	KSP: \textit{evam utpattispharaṇasaṃhārayogena vairocanahṛdgatamadhyāvasthitenodānam udānayanto 'bhiṣekānantare vajrasattvād ārabhya yāvad vajramuṣṭiparyantaṃ svasvacihnāni pūrvavat spharaṇasaṃhārayogenodānam udānayanti}.
}} /} 
%\pend

%\medskip
%
%\pstart\noindent
%[udāna]
%\pend
%
%\medskip
%
%\pstart
\Skt{vajrarājādīnā\edtext{mudānāni}{\lemma{%
	{\rm udānāni\lem}
}\Dfootnote{%
	\emn\ \MSK\ \sil;
	\textit{uddānāni} \cod
}} \edtext{bhavanti}{\lemma{%
	{\rm bhavanti\lem}
}\Dfootnote{%
	\MSK\ reads \textit{bhavati} which it emends to \textit{bhavanti}.
}} /}
\pend

\verse
\textbf{[Vajrarāja]}\\%
\Skt{\edtext{aho hya\edtext{mogharājo'haṃ}{\lemma{%
	{\rm amogharājo 'haṁ\lem}
}\Dfootnote{%
	\cod;
	\textit{amogharājāhaṁ} \MSK\ STTS(\Hed) STTS (\Yed)
	% STTT 46 as edited by Horiuchi reads \textit{amogharājāhaṁ}.%
}} \edtext{vajrasaṃbhavamaṅkuśaḥ}{\lemma{%
	{\rm vajrasaṃbhava-m-aṅkuśaḥ\lem}
}\Dfootnote{%
	\emn\ \MSK\ \sil;
	\textit{vajrāsambhavamaṅkuśa} \cod
}} /\\ 
yatsarvavyāpino \edtext{buddhāḥ}{\lemma{%
	\emn\ \MSK\ \sil;
	\textit{v\unclear{uddha}ḥ} \cod
}} \edtext{samākṛṣyanti %\newfolio{22r1}  
siddhaye}{\lemma{}\Efootnote{%
	samākṛṣyanti [22r1] siddhaye
}}}{\lemma{%
	{\rm aho \dots\ siddhaye\lem}
}\Bfootnote{%
	STTS 46. % Horiuchi reads \textit{amogharājāhaṁ} for \textit{amogharājo 'haṁ}.
}}  //}\\
%
	%KSP evam utpattispharaṇasaṃhārayogena vairocanahṛdgatamadhyāvasthitenodānam udānayanto 'bhiṣekānantare vajrasattvād ārabhya yāvad vajramuṣṭiparyantaṃ svasvacihnāni pūrvavat spharaṇasaṃhārayogenodānam udānayanti.
	%°evaṁ○mut-: em. °evamut-, with MSK (sil. em.).
	%-ṇilaya-: em. -nilaya-, with MSK (sil. em.).
	%-bhāva-: -bhava- MSK.
	%-madhyovasthitonodānam udānaṁ: em. -madhyāvasthita udānam udānayan
	%vajramuṣṭhi- em. vajramuṣṭi-.
	%draṣṭavyam iti: em. draṣṭavya iti.
	%An amazingly ill-constructed sentence? KSP seems even worse. Provisional  translation: '... (Vajrasattva) who is present in the Vajra inside the heart of Vairocana who is ...., while doing the udāna and (also doing) the udāna immediately after the unction, should be visualized (and so should the other Bodhisattvas) up to Vajramuṣṭi'.
	%uddānāni: em. udānāni, with MSK (sil. em.).
	%bhavan\unclear{t}i: MSK reads bhavati which it emends to bhavanti.
	%amogharājo haṁ: amogharājāhaṁ MSK. STTT §46 as edited by Horiuchi reads amogharājāhaṁ.
	%vajrāsambhavamaṅkuśa: em. vajrasambhava-m-aṅkuśaḥ, with MSK (sil. em.).
	%v\unclear{uddha}ḥ: em. buddhāḥ, with MSK (sil. em.).%
%
\Skt{\edtext{\mbox{}\edtext{idaṃ tatsarvabuddhānāṃ}{\lemma{%
	{\rm idaṁ tat sarvabuddhānāṁ\lem}
}\Dfootnote{%
	\emn\ \MSK\ \sil;
	\textit{idan ta savavuddhānā} \cod
}} \edtext{vajrajñāna}{\lemma{%
	{\rm vajrajñānam\lem}
}\Dfootnote{%
	\emn\ \MSK\ \sil;
	\textit{vajrānam} \cod
}}manuttaram / \\
\edtext{yatsarvabuddhārthasiddhyarthaṃ}{\lemma{%
	{\rm yat sarvabuddhārthasiddhyarthaṁ\lem}
}\Dfootnote{%
	\emn\ with STTS;
	\textit{yatsarvavuddhārthasiddhārtha} \cod;
	\MSK\ emends \textit{sarvavuddhārthasiddhyarthaṁ}, mentioning Horiuchi’s different reading for STTS but ignoring that the SVU ms. agrees with the STTS ms.
}} samākarṣaṇamuttamamiti}{\lemma{%
	{\rm idaṁ \dots\ uttamam\lem}
}\Bfootnote{%
	STTS 49
%	\textit{idaṃ tat sarvabuddhānāṃ vajrajñānam anuttaram |
%	yat sarvabuddhārthasiddhyarthaṃ samākarṣaṇam uttamam iti ||}
}} //} \\
	%°idan ta savavuddhānā: em. idaṁ tat sarvabuddhānāṁ, with MSK (sil. em.).
	%vajrānam: em. vajrajñānam, with MSK (sil. em.).
	%yatsarvavuddhārthasiddhārtha: em. yatsarvabuddhārthasiddhyarthaṁ, with STTS; MSK emends sarvavuddhārthasiddhyarthaṁ, mentioninh Horiuchi's different reading for STTS but ignoring our ms. readings.
%
%
\textbf{[Vajrarāga]}\\%
\Skt{\edtext{aho svabhāvaśuddho'hamanurāgaḥ \edtext{svayambhuvām}{\lemma{%
	{\rm svayambhuvām\lem}
}\Dfootnote{%
	\emn\ \MSK\ \sil;
	\edtext{svayabhuvam} \cod
}} /\\ 
\edtext{yacchuddhyarthaṃ}{\lemma{%
	{\rm yac chuddhyarthaṁ\lem}
}\Dfootnote{%
	restored with \MSK\ and STTS;
	\textit{\unclear{ya} + + +} \cod
}} viraktānāṃ \edtext{rāgeṇa}{\lemma{%
	{\rm rāgeṇa\lem}
}\Dfootnote{%
	\emn\ \MSK\ \sil;
	\textit{sagaṇa} \cod
}} vinayanti hi}{\lemma{%
	{\rm aho \dots\ vinayanti hi\lem}
}\Bfootnote{%
	STTS 52
	%: \textit{aho svabhāvaśuddho 'ham anurāgaḥ svayaṃbhuvām | yac chuddhyarthaṃ viraktānāṃ rāgeṇa vinayanti hi ||}
}} //}\\
	%svayabhuvam·: em. svayambhuvām, with MSK (sil. em.).
	%\unclear{ya} + + +: restore yac chuddhyarthaṁ, with MSK (sil. em.) and STTS.
	%sagaṇa: em. rāgeṇa, with MSK (sil. em.).
%
\Skt{\edtext{idaṃ tatsarvabuddhānāṃ rāgajñānamanāvilam /\\ 
hatvā \edtext{virāgaṃ}{\lemma{%
	{\rm virāgaṁ\lem}
}\Dfootnote{%
	\emn\ \MSK\ \sil;
	\textit{virāga} \cod
}} rāgeṇa \edtext{sarva}{\lemma{%
	{\rm sarva-\lem}
}\Dfootnote{%
	\emn\ \MSK\ \sil;
	\textit{tsarva-} \cod
}}saukhyaṃ dadanti hīti}{\lemma{%
	{\rm idaṁ \dots\ dadanti hīti\lem}
}\Bfootnote{%
	STTS 55
	%: \textit{idaṃ tat sarvabuddhānāṃ rāgajñānam anāvilam |
	%hatvā virāgaṃ rāgeṇa sarvasaukhyaṃ dadanti hi ||}
}} //} \\
	%virāga: em. virāgaṁ, with MSK (sil. em.).
	%tsarva-: em. sarva-, with MSK (sil. em.).
%
\textbf{[Vajrasādhu]}\\%
\Skt{\edtext{aho hi sādhukāro'haṃ sarvaḥ sarvavidāṃ varaḥ /\\ 
yadvikalpa\edtext{prahīṇānāṃ}{\lemma{%
	{\rm -prahīṇānāṁ\lem}
}\Dfootnote{%
	\emn\ \MSK;
	\textit{-prahīṇyano} \cod;
	\MSK\ reads \textit{-prahīṇyanāṁ}.
}} tuṣṭiṃ janayate dhruvam}{\lemma{%
	{\rm aho \dots\ dhruvam\lem}
}\Bfootnote{%
	STTS 58
	%\textit{aho hi sādhukāro 'haṃ sarvaḥ sarvavidāṃ varaḥ |
	%yad vikalpaprahīṇānāṃ tuṣṭiṃ janayate dhruvam ||}
}} //}\\
	%-prahīṇyano: em. -prahīṇānāṁ, with MSK, wich reads -prahīṇyanāṁ.
%
\Skt{\edtext{idaṃ tatsarvabuddhānāṃ sādhukārapravartakam /\\ 
sarvatuṣṭikaraṃ vajraṃ divyaṃ prāmodyavardhanamiti}{\lemma{%
	{\rm idaṁ \dots\ vardhanam iti\lem}
}\Bfootnote{%
	STTS 61
%	\textit{idaṁ tat sarvabuddhānāṃ sādhukārapravartakam |
%	sarvatuṣṭikaraṁ vajraṁ divyaṁ prāmodyavardhanam iti ||}
}} //} \\
%
% Vajraratna
\textbf{[Vajraratna]}\\%
\Skt{\edtext{aho hi svabhiṣeko\edtext{'haṃ}{\lemma{%
	{\rm 'haṁ\lem}
}\Dfootnote{%
	\emn\ \MSK\ \sil;
	\textit{haṁm} \cod
}} vajraratnamanuttaram /\\ 
yanniḥsaṅgā api jināstridhātupatayaḥ smṛtāḥ}{\lemma{%
	{\rm aho \dots\ smṛtāḥ\lem}
}\Bfootnote{%
	STTS 66
%	\textit{aho hi svabhiṣeko 'haṁ vajraratnam anuttaram |
%	yan niḥsaṅgā api jinās tridhātupatayaḥ smṛtāḥ ||}
}} //}\\
	%haṁm: em. 'haṁ, with MSK (sil. em.).
%
\Skt{\edtext{idaṃ tatsarvabuddhānāṃ sattvadhātvabhiṣecanam /\\ 
ahaṃ mama kare dattaṃ \edtext{ratnaṃ ratne}{\lemma{%
	{\rm ratnaṁ ratne\lem}
}\Dfootnote{%
	\emn\ \MSK\ \sil;
	\textit{ratna ratne} \cod;
	\textit{ratne ratnaṁ} STTS(\Hed);
	\textit{ratne ratnan} STTS(\Yed)
}} \edtext{niyojita}{\lemma{%
	{\rm niyojitam\lem}
}\Dfootnote{%
	\emn\ \MSK\ \sil;
	\textit{tiyojitam} \cod  
}}miti}{\lemma{%
	{\rm idaṁ \dots\ niyojitam iti\lem}
}\Bfootnote{%
	STTS 69
%Horiuchi reads \textit{ratne ratnaṁ} for \textit{ratnaṁ ratne}.
%	\textit{idaṁ tat sarvabuddhānāṁ sattvadhātvabhiṣecanam |
%	ahaṁ mama kare dattaṁ ratne ratnaṁ niyojitam iti ||} 
}} //} \\
	%ratna: em. ratnaṁ, with MSK (sil. em.).
	%tiyojitam: em. niyojitam, with MSK (sil. em.).
%
% Vajrateja
\textbf{[Vajrateja]}\\%
\Skt{\edtext{aho'nupamatejo'haṃ \edtext{sattva}{\lemma{%
	{\rm sattva-\lem}
}\Dfootnote{%
	\emn\ ($\leftarrow$STTS, SVU Tib., \textit{Tattvālokakarī});
	\textit{sarva-} \cod\ \MSK
}}dhātvavabhāsanam / \\
ya\edtext{cchodhayati}{\lemma{%
	{\rm chodhayati\lem}
}\Dfootnote{%
	\cod;
	\textit{chodayati} \MSK
}} \edtext{śuddhānāṃ}{\lemma{%
	{\rm śuddhānāṁ\lem}
}\Dfootnote{%
	\emn\ \MSK\ \sil;
	\textit{suddhānāṁ} \cod
}} buddhānāmapi tāyi\supplied{nā}m}{\lemma{%
	{\rm aho \dots\ tāyi\supplied{nā}m\lem}
}\Bfootnote{%
	STTS 72. 
	The MS and the editions read the first half as follows:
	\textit{aho hy anupaman tejaḥ satvadhātvavabhāsanam} \cod\ \Yed;
	\textit{aho hy anupamaṁ tejaḥ sattvadhātvavabhāsanam} \Hed
%	yac chodhayati śuddhānāṁ buddhānām api tāyinām ||}.
%	STTS has variants here.
}} //}\\ 
	%STTS has variants here.
	%suddhānāṁ: em. śuddhānāṁ, with MSK (sil. em.).
%
\Skt{\edtext{idaṃ tatsarvabuddhānāmajñāna\edtext{dhvānta}{\lemma{%
	{\rm -dhvānta-\lem}
}\Dfootnote{%
	\emn\ \MSK\ \sil;
	\textit{-dhvanta-} \cod
}}nāśanam /\\ 
paramāṇurajaḥ\edtext{saṃkhyā}{\lemma{%
	{\rm -saṁkhyā- \lem}
}\Dfootnote{%
	\cod\ \MSK;
	\textit{-saṁkhya-} STTS(\cod\ \Hed\ \Yed)
}}sūryā\edtext{dhikataraprabham}{\lemma{%
	{\rm -dhikataraprabham\lem}
}\Dfootnote{%
	\emn\ \MSK;
	\textit{-dhikataprabham} \cod
}}}{\lemma{%
	{\rm idaṁ \dots\ -prabham\lem}
}\Bfootnote{%
	STTS 74 
	%Horiuchi reads \textit{-saṁkhya-} for \textit{-saṁkhyā-}.
%	\textit{idaṁ tat sarvabuddhānām ajñānadhvāntanāśanam |
%	paramāṇurajaḥsaṁkhyasūryādhikataraprabham iti ||}%
}} //} \\
	%-dhvanta-: em. -dhvānta-, with MSK (sil. em.).
	%-dhikataprabham: em. -dhikataraprabham, with MSK.
%
% Vajraketu
\textbf{[Vajraketu]}\\%
\Skt{\edtext{aho \edtext{hyasadṛśaḥ}{\lemma{%
	{\rm hy asadṛśaḥ\lem}
}\Dfootnote{%
	\emn\ \MSK;
	\textit{hy isadṛśaḥ} \cod
}} ketuḥ ahaṃ sarvārthasiddhinām /\\ 
yatsarvāśā\edtext{prapūrṇānāṃ}{\lemma{%
	{\rm -prapūrṇānāṁ\lem}
}\Dfootnote{%
	\emn\ \MSK\ \sil;
	\textit{-prapūṇṇānāṁ} \cod;
	\textit{-paripūrṇānāṁ} STTS(\cod\ \Hed\ \Yed)
}} sarvārtha\edtext{paripūraṇa}{\lemma{%
	{\rm paripūraṇam\lem}
}\Dfootnote{%
	\emn\ \MSK\ \sil;
	\textit{paripūrṇṇām} \cod;
	\textit{-pratipūraṇam} STTS(\cod\ \Hed\ \Yed)
}}miti}{\lemma{%
	{\rm aho \dots\ paripūraṇam iti\lem}
}\Bfootnote{%
	STTS 77 
	%STTS reads the latter half as follows:
	%\textit{yat sarvāśāparipūrṇṇānāṁ sarvārthapratipūraṇam} \cod;
	%\textit{yat sarvāśāparipūrṇānāṁ sarvārthapratipūraṇam} \Hed\ \Yed\ (unmetrical).
}} //}\\
%
\Skt{\edtext{idaṃ ta\edtext{tsarvavuddhānāṃ sarvāśāparipūraṇam}{\lemma{}\Efootnote{%
	sarvabuddhānāṁ [22v1] sarvāśāparipūraṇam
}} /\\ 
cintāmaṇidhvajaṃ nāma dānapāramitānayam}{\lemma{%
	{\rm idaṁ \dots\ dānapāramitānayam\lem}
}\Bfootnote{%
	STTS 80
%	\textit{idaṁ tat sarvabuddhānāṁ sarvāśāparipūraṇam |
%	cintāmaṇidhvajaṁ nāma dānapāramitānayam iti ||}
}} //} \\
%
% Vajrahāsa
\textbf{[Vajrahāsa]}\\%
\Skt{\edtext{\mbox{}\edtext{aho mahāhāsa}{\lemma{
	{\rm aho mahāhāsam\lem}
}\Dfootnote{%
	\emn\ \MSK\ STTS(\cod\ \Hed\ \Yed);
	\textit{aho hi mahāhāsam} \cod
}}mahaṃ \edtext{sarvāgryāṇāṃ}{\lemma{%
	{\rm sarvāgryāṇāṁ\lem}
}\Dfootnote{%
	\cod\ \MSK\ STTS(\Hed\ \Yed);
	\textit{sarvāgryāṇāṁ} STTS(\cod)
}} mahādbhutam /\\
\edtext{yatprayuñjanti}{\lemma{%
	{\rm prayuñjanti\lem}
}\Dfootnote{%
	\emn\ following STTS(\Hed\ \Yed);
	\textit{prayujyanti} \cod;
	\textit{prayuṁjaṁ*ya(cancelled ?)nti} STTS(\cod)
}} \edtext{buddhārthe}{\lemma{%
	{\rm buddhārthe\lem}
}\Dfootnote{%
	\cod\ \MSK\ STTS(\cod\ \Yed);
	\textit{buddhārthaṁ} STTS(\Hed)
}} sadaiva \edtext{susamāhitāḥ}{\lemma{%
	{\rm susamāhitāḥ\lem}
}\Dfootnote{%
	\cod;
	\textit{susamāhitaḥ} \MSK
}}}{\lemma{%
	{\rm aho \dots\ susamāhitāḥ\lem}
}\Bfootnote{%
	STTS 83
%	\textit{aho mahāhāsam ahaṁ sarvāgryāṇāṁ mahādbhutam |
%	yat prayuñjanti buddhārthaṁ sadaiva susamāhitāḥ ||}
}} //}\\
	%°aho hi mahāhāsam: em. °aho mahāhāsam, with MSK and STTS.
	%susamāhitāḥ: susamāhitaḥ MSK.
%
\Skt{\edtext{idaṃ tatsarva\edtext{buddhānā}{\lemma{%
	{\rm -buddhānām\lem}
}\Dfootnote{%
	\emn\ \MSK\ \sil;
	\textit{-vuddhānāṁm} \cod
}}\edtext{madbhutotpādadarśakam}{\lemma{%
	{\rm adbhutotpādadarśakam\lem}
}\Dfootnote{%
	\emn\ \MSK;
	\textit{adbhut·pādadarśakam} \cod
}} /\\ 
\edtext{mahāharṣakaraṃ}{\lemma{%
	{\rm mahāharṣakaraṁ\lem}
}\Dfootnote{%
	\emn\ \MSK\ STTS;
	\textit{mahākarṣakaraṁ} \cod
}} jñānamajñātaṃ \edtext{paraśāsibhi}{\lemma{%
	{\rm paraśāsibhir\lem}
}\Dfootnote{
	\emn\ ($\leftarrow$STTS \cod\ \Yed);
	\textit{paraparaśāsibhir} \cod
	\textit{paraśāstribhiḥ} \emn\ \MSK\ ($\leftarrow$STTS Horiuchi's note);
	\textit{parasāmitir} STTS(\Hed)
}\Cfootnote{%
	STTS(\Hed) here reads \textit{parasāmitir}, but Horiuchi mentions a possibility to emend to \textit{paraśāstribhir} on the basis of Tib. (\textit{gtsug lag gzhan gyis}). See STTS §86, note 1.
	But agreement between SVU and STTS mss.\ suggest we may not need such a violent emendation. 
	We should accept the reading \textit{paraśāsibhir} as equivalent to \textit{paraśāstribhir}. 
}}riti}{\lemma{%
	{\rm idaṁ \dots\ paraśāstribhir iti\lem}
}\Bfootnote{%
	STTS 86
%	\textit{idaṁ tat sarvabuddhānām adbhutotpādadarśakam |
%	mahāharṣakaraṁ jñānam ajñātaṁ parasāmitir śāstribhir iti ||}
}} //} \\
	%-vuddhānāṁm: em. -buddhānām, with MSK (sil. em.).
	%adbhut·pādadarśakam·: em. adbhutotpādadarśakam·, with MSK.
	%mahāka○rṣakaraṁ: em. mahāharṣakaraṁ, with MSK and STTS.
	%paraparaśāsibhir: em. paraśāstribhir, with MSK; STTS here read paraśāmitir, emended to paraśāstribhir by Horiuchi. But agreement between SVU and STTS mss. suggest we may not need such a violent emendation. Emend paraśāsibhir as equivalent to paraśāstribhir? Or paraśāsitair
%
%
% Vajradharma
\textbf{[Vajradharma]}\\%
\Skt{\edtext{aho hi paramārtho'hamādiśuddhaḥ \edtext{svayambhuvām}{\lemma{%
	{\rm svayambhuvām\lem}
}\Dfootnote{%
	\emn\ \MSK\ \sil\ STTS(\cod\ \Hed);
	\textit{svayadbhuvam} \cod;
	\textit{svayambhuvān} STTS(\Yed)
}} / \\
ya\edtext{tkolopamadharmāṇāṃ}{\lemma{%
	{\rm kolopamadharmāṇāṁ\lem}
}\Dfootnote{%
	\emn\ \MSK\ ($\leftarrow$STTS);
	\textit{lokopamadharmmā\unclear{ṁ}} \cod
}} \edtext{viśuddhi}{\lemma{%
	{\rm viśuddhir\lem}
}\Dfootnote{%
	\emn\ \MSK\ ($\leftarrow$STTS);
	\textit{viśuddhim} \cod
}}rupalabhyate}{\lemma{%
	{\rm aho \dots\ upalabhyate\lem}
}\Bfootnote{%
	STTS 91
%	\textit{aho hi paramārtho 'haṁ ādiśuddhaḥ svayaṁbhuvām |
%	yat kolopamadharmāṇāṁ viśuddhir upalabhyate ||}
}} //}\\
	%svayadbhuvam·: em. svayambhuvām·, with MSK (sil. em.).
	%lokopamadharmmā\unclear{ṁ}: em. kolopamadharmāṇāṁ, with MSK.
	%viśuddhim: em. viśuddhir, with MSK.
%
%
\Skt{\edtext{idaṃ tatsarvabuddhānāṃ rāgatattvāvabodhanam /\\ 
ahaṃ mama kare dattaṃ \edtext{dharmaṃ dharme}{\lemma{%
	{\rm dharmaṁ\lem}
}\Dfootnote{%
	\emn\ \MSK\ ($\leftarrow$STTS(\cod\ \Yed));
	\textit{dharmma dharme} \cod;
	\textit{dharmaṁ dharma-} STTS(\Hed)
}} pratiṣṭhitam}{\lemma{%
	{\rm idaṁ \dots\ pratiṣṭhitam\lem}
}\Bfootnote{%
	STTS 94
%	\textit{idaṁ tat sarvabuddhānāṁ rāgatattvāvabodhanam |
%	ahaṁ mama kare dattaṁ dharmaṁ dharmapratiṣṭhitam iti}
}} //} \\
	%dharmma: em. dharmaṁ, with MSK.
%
% Vajratīkṣṇa
\textbf{[Vajratīkṣṇa]}\\%
\Skt{\edtext{aho hi sarvabuddhānāṃ mañjughoṣamahaṃ smṛtaḥ /\\ 
\edtext{ya}{\lemma{%
	{\rm yat\lem}
}\Dfootnote{
	\emn\ \MSK;
	\textit{ya} \cod
}}\edtext{tprajñāyā}{\lemma{%
	{\rm prajñāyā\lem}
}\Dfootnote{%
	\emn\ following STTS;
	\textit{prajñayā} \cod
}} \edtext{arūpiṇyā}{\lemma{%
	{\rm arūpiṇyā\lem}
}\Dfootnote{%
	\emn\ (STTS(\cod\ \Hed\ \Yed));
	\textit{rūpināḥ} \cod;
	\textit{-rūpiṇyā} \MSK
}} ghoṣatvamupalabhyate}{\lemma{%
	{\rm aho \dots\ upalabhyate\lem}
}\Bfootnote{%
	STTS 97%:
%	\textit{aho hi sarvabuddhānāṁ mañjughoṣa-m-ahaṁ smṛtaḥ |
%	yat prajñāyā arūpiṇyā ghoṣatvam upalabhyate ||}
}} //}\\
	%ya pra-: em. yat pra-, with MSK.
	%rūpiṇāḥ: em. -rūpiṇyā, with MSK.
%
\Skt{\edtext{idaṃ tatsarvabuddhānāṃ prajñāpāramitānayam / \\
\edtext{chettāraṃ}{\lemma{%
	{\rm chettāraṁ\lem}
}\Dfootnote{%
	\emn\ \MSK;
	\textit{cchetāraṁ} \cod
}} \edtext{sarvaśatrūṇāṃ}{\lemma{%
	{\rm sarvaśatrūṇāṁ\lem}
}\Dfootnote{%
	\emn\ \MSK\ \sil;
	\textit{sarvasatrūṇāṁ} \cod
}} sarvapāpaharaṃ param}{\lemma{%
	{\rm idaṁ\dots param\lem}
}\Bfootnote{%
	STTS 100%:
%	\textit{idaṁ tat sarvabuddhānāṁ prajñāpāramitānayam |
%	chettāraṁ sarvaśatrūṇāṁ sarvapāpaharaṁ param iti ||}
}} //} \\
	%cchetāraṁ: em. chettāraṁ, with MSK.
	%sarvasatrūṇāṁ: em. sarvaśatrūṇāṁ, with MSK (sil. em.).
%	
% Vajrahetu
\textbf{[Vajrahetu]}\\%
\Skt{\edtext{aho vajramayaṃ cakramahaṃ vajrāgradharmiṇām /\\ 
ya\edtext{ccittotpāda}{\lemma{%
	{\rm cittotpāda-\lem}
}\Dfootnote{%
	\emn\ \MSK;
	\textit{citotpāda-} \cod
}}mātreṇa dharmacakraṃ pravartate}{\lemma{%
	{\rm aho \dots\ pravartate\lem}
}\Bfootnote{%
	STTS 103
%	\textit{aho vajramayaṁ cakram ahaṁ vajrāgradharmiṇām |
%	yac cittotpādamātreṇa dharmacakraṁ pravartate ||}
}} //}\\
	%citotpāda-: em. cittotpāda-, with MSK.
%
\Skt{\edtext{idaṃ tatsarvabuddhānāṃ sarvadharma\edtext{viśodhanam}{\lemma{%
	{\rm -viśodhanam\lem}
}\Dfootnote{%
	\emn\ \MSK\ \sil;
	\textit{-visodhanam} \cod;
	\textit{-viśodhakaṁ} STTS(\cod\ \Hed\ \Yed)
}} /\\ 
\edtext{avaivartika}{\lemma{%
	{\rm avaivartika-\lem}
}\Dfootnote{%
	\emn\ \MSK\ \sil;
	\textit{avaivarttika-} \cod
}}cakraṃ tu bodhimaṇḍamiti smṛtam}{\lemma{%
	{\rm idaṁ\dots smṛtam\lem}
}\Bfootnote{%
	STTS 106
%	\textit{idaṁ tat sarvabuddhānāṁ sarvadharmaviśodhakam |
%	avaivartikacakraṁ tu bodhimaṇḍam iti smṛtam iti ||}
}} //} \\
	%-visodhanam: em. -viśodhanam, with MSK (sil. em.).
	%avaivarttika-: em. avaivartika, with MSK (sil. em.).
%
% Vajrabhāṣa
\textbf{[Vajrabhāṣa]}\\%
\Skt{\edtext{aho \edtext{svayambhuvāṃ}{\lemma{%
	{\rm svayambhuvāṁ\lem}
}\Dfootnote{%
	\emn\ \MSK\ \sil;
	\textit{svayaṁdbhuvāṁ} \cod
}} guhyaṃ saṃdhābhāṣamahaṃ smṛtaḥ /\\ 
yaddeśayanti \edtext{saddharmaṃ}{\lemma{%
	{\rm saddharmaṁ\lem}
}\Dfootnote{%
	\emn\ \MSK;
	\textit{sadharmma} \cod;
	\MSK\ reports only the variant \textit{sadha-}.
}} vā\edtext{kprapañca}{\lemma{%
	{\rm -prapañca-\lem}
}\Dfootnote{%
	\emn\ \MSK;
	\textit{-prapañcā-} \cod
}}vivarjitam}{\lemma{%
	{\rm aho \dots\ -vivarjitam\lem}
}\Bfootnote{%
	STTS 109
%	\textit{aho svayaṁbhuvāṁ guhyaṁ saṁdhābhāṣa-m-ahaṁ smṛtaḥ |
%	yad deśayanti saddharmaṁ vākprapañcavivarjitam ||}
}} //}\\
	%svayaṁdbhuvāṁ: em. svayambhuvāṁ, with MSK (sil. em.).
	%sadharmma: em. saddharmaṁ, with MSK; MSK reports only the variant sadha-.
	%-prapañcāvi-: em. -prapañcavi-, with MSK.
%
\Skt{\edtext{idaṃ tatsarvabuddhā\supplied{nāṃ} vajrajāpa\edtext{\textcolor{red}{manantaram}}{\lemma{%
	{\rm anantaram\lem}
}\Dfootnote{%
	\cod\ \MSK;
	\textit{udāhṛtaṁ} STTS
}} /\\ 
sarvatathāgatānāṃ tu mantrāṇāmā\supplied{śu} \edtext{sā\supplied{dha}na}{\lemma{%
	{\rm sā\supplied{dha}nam\lem}
}\Dfootnote{%
	\emn\ \MSK\ \sil;
	\textit{sānam} \cod
}\lemma{}\Efootnote{%
	sā[23r1]\supplied{dha}nam
}}miti}{\lemma{%
	{\rm idaṁ\dots sā\supplied{dha}nam iti\lem}
}\Bfootnote{%
	STTS 112
%	\textit{idaṁ tat sarvabuddhānāṁ vajrajāpau udāhṛtam |
%	sarvatathāgatānāṁ tu mantrāṇām āśu sādhanam iti ||}
}} //} \\
	% sarvabuddhānā\supplied{ṃ}
	% vajrajāpamanantara\cancelled{ṁ}m
	%sānam: em. sādhanam, with MSK (sil. em.).
	%āś\supplied{u}
%
% Vajrakarma
\textbf{[Vajrakarma]}\\%
\Skt{\edtext{aho hya\edtext{moghaṃ}{\lemma{%
	{\rm amoghaṁ\lem}
}\Dfootnote{%
	\emn\ after STTS:
	\textit{amogha} \cod;
	\MSK\ emends \textit{amogho}.
}} buddhānāṃ \edtext{sarvakarmamahaṃ bahu}{\lemma{%
	{\rm sarvakarma-m-ahaṁ bahu\lem}
}\Dfootnote{%
	\emn\ ($\leftarrow$STTS);
	\textit{sarvakarmāvaham ahaṁ} \cod;
	\textit{sarvakarmā ahaṁ bahuḥ} \MSK;
	\textit{sarvakarma-m-ahaṃ bahu} STTS(\Hed)
	\textcolor{red}{STTS CHECK MS AND YAMADA!!}
%
%	Possibly sarvakarmāvahaṁm aham (Aiśa sandhi, analogous to sandhi -nn a-) in meaning of sarvakarmāvaham aham?
}} /\\ 
yadanābhogabuddhārthaṃ vajrakarma \edtext{pravartate}{\lemma{%
	{\rm pravartate\lem}
}\Dfootnote{%
	\emn\ ($\leftarrow$STTS);
	\textit{pravartakaḥ} \cod;
	\textit{pravartate} STTS (\textcolor{red}{CHECK MS AND YAMADA!!})
}}}{\lemma{%
	{\rm aho \dots\ pravartate\lem}
}\Bfootnote{%
	STTS 116
%	\textit{aho hy amoghaṃ buddhānāṃ sarvakarma-m-ahaṃ bahu | 
%	yadanābhogabuddhārthaṃ vajrakarma pravartate ||};
%	we reject any emendation based upon the SVU variant for pāda b, because the last verse of §2.2.2 (= STTS 150) too clearly argues in favor of a reading with \textit{bahum}.
%	\textcolor{red}{Move this note to the third layer!!}
}} //}\\
	%STTS §116 aho hy amoghaṃ buddhānāṃ sarvakarmam ahaṃ bahu | yad anābhogavuddhārthaṃ vajrakarma pravartate ||
	%Note interesting variant in pāda c.
	%amogha: em. amoghaṁ, after STTS; MSK emends amogho.
	 % sarvakarmā ahaṁ bahuḥ MSK
	%sarvakarmā: STTS reads sarvakarmam.
	%vajraṁkarmma: MST vajrakarma
	%vajra\cancelled{ṁ}
%
\Skt{\edtext{idaṃ tatsarvabuddhānāṃ viśvakarmakaraṃ param /\\ 
ahaṃ mama ka\supplied{re} da\supplied{ttaṃ} \edtext{viśve}{\lemma{%
	{\rm viśve\lem}
}\Dfootnote{%
	\emn\ \MSK\ \sil;
	\textit{visva} \cod
}} viśvaṃ niyojitam}{\lemma{%
	{\rm idaṁ\dots niyojitam\lem}
}\Bfootnote{%
	STTS 119
%	\textit{idaṁ tat sarvabuddhānāṁ viśvakarmakaraṁ param |
%	ahaṁ mama kare dattaṁ viśve viśvaṁ niyojitam iti ||}
}} //} \\
	%visva: em. viśve, with MSK (sil. em.).
	%kar\supplied{e}
	%datta\supplied{ṁ}
%
% Vajrarakṣa
\textbf{[Vajrarakṣa]}\\%
\Skt{\edtext{aho vīryamayo \edtext{varma}{\lemma{%
	{\rm varma\lem}
}\Dfootnote{%
	\cod\ \textcolor{red}{\MSK};
	\textit{varmaḥ} STTS (\textcolor{red}{CHECK MS AND YAMADA!!})
}} sudṛḍho'haṃ dṛḍhātmanām /\\ 
\edtext{ya}{\lemma{%
	{\rm yad\lem}
}\Dfootnote{%
	\emn\ \MSK\ \sil;
	\textit{ya} \cod
}}ddṛḍhatvādakāyānāṃ \edtext{vajrakāyakaraṃ}{\lemma{%
	{\rm vajrakāyakaram\lem}
}\Dfootnote{%
	\emn\ \MSK\ \sil;
	\textit{vajrākāyakaram} \cod
}} param}{\lemma{%
	{\rm aho \dots\ param\lem}
}\Bfootnote{%
	STTS 122
%	\textit{aho vīryamayo varmaḥ sudṛḍho 'haṁ dṛḍhātmanām |
%	yad dṛḍhatvād akāyānāṁ vajrakāyakaraṁ param iti ||}
}} //}\\
	%varma: STTS reads varmaḥ.
	%ya dṛḍhatvād: em. yad dṛḍhatvād, with MSK (sil. em.).
	%vajrākāyakaram: em. vajrakāyakaram, with MSK (sil. em.).
%
\Skt{\edtext{idaṃ tatsarvabuddhānāṃ maitrīkavacamuttamam /\\ 
dṛḍhavīryamahārakṣaṃ mahāmitramudāhṛtam}{\lemma{%
	{\rm idaṁ\dots udāhṛtam\lem}
}\Bfootnote{%
	STTS 125
%	\textit{idaṁ tat sarvabuddhānāṁ maitrīkavacam uttamam |
%	dṛḍhavīrya\-mahārakṣaṁ mahāmitram udāhṛtam iti ||}
}} //} \\
	%\unclear{u}ttamam·
%
% Vajrayakṣa
\textbf{[Vajrayakṣa]}\\%
\Skt{\edtext{aho mahopāyamahaṃ buddhānāṃ karuṇātmanām /\\ 
ya\edtext{tsattvārthatayā}{\lemma{%
	{\rm sattvārthatayā\lem}
}\Dfootnote{%
	\emn\ \MSK\ \sil\ after STTS;
	\textit{sarvārthatayā} \cod
}} śāntā raudratvamapi \edtext{kurvate}{\lemma{%
	{\rm kurvate\lem}
}\Dfootnote{%
	\emn\ ($\leftarrow$STTS);
	\textit{kurvataḥ} \cod;
	\textit{kuruvataḥ} \MSK\ (unintentional error)
	\textcolor{red}{CHECK STTS MS AND YAMADA!!}
	% kurvataḥ: nominative plural
}}}{\lemma{%
	{\rm aho \dots\ kurvate\lem}
}\Bfootnote{%
	STTS 128
%	\textit{aho mahopāya-m-ahaṁ buddhānāṁ karuṇātmanām |
%	yat sattvā\-rthatayā śāntā raudratvam api kurvate ||}
}} //}\\
	%sarvārthatayā: em. sattvārthatayā, with MSK (sil. em.) after STTS.
	%kurvataḥ: kuruvataḥ MSK (unintentional error).
%
\Skt{\edtext{idaṃ tatsarvavuddhānāṃ sarva\edtext{duṣṭāgrya}{\lemma{%
	{\rm -duṣṭāgrya-\lem}
}\Dfootnote{%
	\emn\ ($\leftarrow$KSP \textcolor{red}{CHECK!!});
	\textit{-\unclear{d}uṣṭāpya-} \cod
	\textit{-duṣṭāgra-} STTS \textcolor{red}{CHECK STTS MS AND YAMADA!!}
	%\MSK\ reads \textit{-draṣṭāpya}, emended to \textit{-duṣṭāgra-} after STTS. \textcolor{red}{Emend rather \textit{-duṣṭāpya-, -duṣṭādya-, -duṣṭasya}}?
}}\edtext{dāmakam}{\lemma{%
	{\rm -dāmakam\lem}
}\Dfootnote{%
	\emn\ \MSK\ \sil;
	\textit{-dāyakam} \cod
}} /\\ 
\edtext{vajradaṃṣṭrāyudhaṃ}{\lemma{%
	{\rm vajradaṁṣṭrāyudhaṁ\lem}
}\Dfootnote{%
	\emn\ \MSK\ \sil;
	\textit{vajradraṣṭrāyudha} \cod
}} tīkṣṇamupāyaḥ \edtext{karuṇātmanām}{\lemma{%
	{\rm karuṇātmanām\lem}
}\Dfootnote{%
	\emn\ \MSK\ \sil;
	\textit{karuṇātmanā} \cod
}}}{\lemma{%
	{\rm idaṁ\dots karuṇātmanām\lem}
}\Bfootnote{%
	STTS 131
%	\textit{idaṁ tat sarvabuddhānāṁ sarvaduṣṭāgradāmakam |
%	vajradaṁṣṭrāyudhaṁ tīkṣṇam upāyaḥ karuṇātmanām iti ||}
}} //} \\
	%-\unclear{d}uṣṭāpya-: MSK -draṣṭāpya, emended to -duṣṭāgra- after STTS. Emend rather -duṣṭāpya-, -duṣṭādya-, -duṣṭasya?
	%-dāyakam: em. -dāmakam, with MSK (sil. em.)?
	%vajradraṣṭrāyudha-: em. vajradaṁṣṭrāyudhaṁ, with MSK (sil. em.).
	%karuṇātmanā: em. karuṇātmanām, with MSK (sil. em.).
%
% Vajramuṣṭi
\textbf{[Vajramuṣṭi]}\\%
\Skt{\edtext{aho hi \edtext{sudṛḍho bandhaḥ}{\lemma{%
	{\rm sudṛḍho bandhaḥ\lem}
}\Dfootnote{%
	\cod\ \MSK;
	\textit{sudṛḍhabandhaḥ}STTS (\textcolor{red}{CHECK MS AND YAMADA!!})
}} samayo'haṃ \edtext{dṛḍhātmanām}{\lemma{%
	{\rm dṛḍhātmanām\lem}
}\Dfootnote{%
	\emn\ \MSK\ \sil;
	\textit{dṛḍhātmānāṁ} \cod
}} /\\ 
yatsarvāśā\edtext{prasiddhyarthaṃ}{\lemma{%
	{\rm prasiddhyarthaṁ\lem}
}\Dfootnote{%
	\emn\ \MSK;
	\textit{pratasiddhyarthaṁ} \cod
}} muktānāmapi bandhanam}{\lemma{%
	{\rm aho \dots\ bandhanam\lem}
}\Bfootnote{%
	STTS 134
%	\textit{aho hi sudṛḍhabandhaḥ samayo 'haṁ dṛḍhātmanām |
%	yat sarvāśāprasidhyarthaṁ muktānām api bandhanam ||}
}} //}\\
	%sudṛḍho vandhaḥ: STTS reads sudṛḍhavandhaḥ.
	%dṛḍhātmānāṁ: em. dṛḍhātmanām, with MSK (sil. em.).
	%-pratasiddhyarthaṁ: em. -prasiddhyarthaṁ, with MSK 
%
\Skt{\edtext{idaṃ tatsarvabuddhānāṃ mudrābandhaṃ mahādṛḍham /\\ 
sarva\edtext{buddhāśusiddhyarthaṃ}{\lemma{%
	{\rm -buddhāśusiddhyarthaṁ\lem}
}\Dfootnote{%
	\emn\ \MSK;
	\textit{vuddhāstusiddhyarthaṁ} \cod
%	\textit{-buddhāśusiddhyarthaṁ} \emn\ \MSK
}} samayo duratikrama iti}{\lemma{%
	{\rm idaṃ \dots\ duratikrama\lem}
}\Bfootnote{%
	STTS 137:
	\textit{idaṁ tat sarvabuddhānāṁ muṣṭibandhaṁ mahādṛḍham |
	yat sarvabuddhās tu siddhyarthaṁ samayo duratikrama iti ||}.
	SVU has a variant in \textit{pāda} c.
	\textcolor{red}{REPORT MS AND YAMADA!!}
%	The STTS reading of \textit{pāda} c is unmetrical; the SVU reading is metrical but still unsatisfactory, so we propose an emendation. 
}} //} 
	%STTS begins pāda c with yat.
	%-vuddhāstusiddhyarthaṁ: MSK emends -buddhāśusiddhyarthaṁ. We propose -buddhājñāsiddhyarthaṁ.
	%\unclear{sa}rvavuddhānāṁ
\pend

\bigskip

\pstart\noindent
{\large 2.2.2. Emanation of the Four Goddesses Surrounding Vairocana}
\pend

\bigskip

\pstart
\Skt{\edtext{tato'kṣobhyā}{\lemma{%
	{\rm tato 'kṣobhyā-\lem}
}\Dfootnote{%
	\emn\ \MSK\ \sil;
	\textit{tata akṣobhyā-} \cod
}}haṃkāreṇa sattvavajrīṃ niṣpādayet / ratnasambhavāhaṃkāreṇa ratnavajrīm / amitābhāhaṃkāreṇa dharmavajrīm / amogha\edtext{siddhyahaṃkāreṇa}{\lemma{%
	{\rm -siddhyahaṁkāreṇa\lem}
}\Dfootnote{%
	\emn\ \MSK\ \sil;
	\textit{-siddhyāhaṅkāreṇa} \cod
}} karmavajrīm / āsā\edtext{mudānāni}{\lemma{%
	{\rm udānāni\lem}
}\Dfootnote{%
	\emn\ \MSK\ \sil;
	\textit{uddānāni} \cod
	%Same variant appeared earlier.
}} bhavanti //}
	%tata °akṣo-: em. tato 'kṣo-, with MSK (sil. em.).
	%-siddhyāhaṅkāreṇa: em. -siddhyahaṁkāreṇa, with MSK (sil. em.).
	%uddānāni: em. udānāni, with MSK (sil. em.). Same variant appeared earlier.
\pend

\pstart\noindent
\textbf{[Sattvavajrī]}%
\pend

\verse
\Skt{\edtext{aho hi sarvabuddhānāṃ sattvavajramahaṃ \edtext{dṛḍhaḥ}{\lemma{%
	{\rm dṛḍhaḥ\lem}
}\Dfootnote{%
	\emn\ \MSK\ ($\leftarrow$STTS);
	\textit{dṛḍhaṁ} \cod.
	See STTS 141, where Horiuchi reads \textit{dṛḍhaḥ} but admits in a note that he cannot justify why the text does not show \textit{dṛḍham}. We believe that \textit{satvavajram aham} is \textit{metri causa} for \textit{satvavajro ’ham} and thus that \textit{dṛḍhaḥ} is correct. This explains the masculine adjective forms in the following hemistich.
	\textcolor{red}{CHECK MS AND YAMADA!!}
}} / \\
\edtext{yaddṛḍhatvā}{\lemma{}\Efootnote{%
	dṛḍham [siddham sign] [23v1] yad dṛḍhatvād
}}dakāyo'pi vajrakāyatvamāgataḥ}{\lemma{%
	{\rm aho \dots\ āgataḥ\lem}
}\Bfootnote{%
	STTS 141
%	\textit{aho hi sarvabuddhānāṁ sattvavajra-m-ahaṁ dṛḍhaḥ |
%	yaddṛdhatvād akāyo vajrakāyatvam āgata iti ||}
}} //} \\
	%dṛḍha*ṁ: em. dṛḍhaḥ, with MSK. See STTS §141, with note Horiuchi on his reading dṛḍhaḥ. We disagree with Horiuchi; it is conceivable that satvavajram aham stands for satvavajro 'ham in which case dṛḍhaḥ is correct, and this is confirmed by the masculine adjective forms in the following stanzas.
	%yadṛḍhatvād: em. yad dṛḍhatvād, with MSK (sil. em.). Same variant appeared earlier.
%
\textbf{[Ratnavajrī]}\\%
\Skt{\edtext{aho hi sarvabuddhānāṃ \edtext{ratnavajra}{\lemma{%
	{\rm ratnavajram\lem}
}\Dfootnote{%
	\emn\ \MSK\ \sil\ ($\leftarrow$STTS \cod\ \Hed\ \textcolor{red}{CHECK YAMADA!!});
	\textit{ratnavajrām} \cod
}}mahaṃ \edtext{smṛtaḥ}{\lemma{%
	{\rm smṛtaḥ\lem}
}\Dfootnote{%
	\emn\ \MSK;
	\textit{smṛtam} \cod\ STTS (\cod\ \Hed) \textcolor{red}{YAMADA!!}
}} /\\ 
yanmudrāṇāṃ hi \edtext{sarvāsā}{\lemma{%
	{\rm sarvāsām\lem}
}\Dfootnote{%
	\emn\ \MSK\ \sil\ ($\leftarrow$STTS \cod\ \Hed\ \textcolor{red}{CHECK YAMADA!!});
	\textit{sarvāśām} \cod
}}\edtext{mabhiṣekanayaṃ}{\lemma{%
	{\rm abhiṣekanayaṁ\lem}
}\Dfootnote{%
	\emn\ \MSK\ ($\leftarrow$STTS \cod\ \Hed\ \textcolor{red}{CHECK YAMADA!!});
	\textit{abhi\unclear{ṣ}ekataḥyaṁ} \cod
}} \edtext{dṛḍham}{\lemma{%
	{\rm dṛḍham\lem}
}\Dfootnote{%
	\emn\ \MSK\ ($\leftarrow$STTS (\cod\ \Hed) \textcolor{red}{CHECK YAMADA!!});
	\textit{dṛḍhaṁḥ} \cod;
	\MSK\ reads \textit{dṛḍhaḥ}.
}}}{\lemma{%
	{\rm aho \dots\ dṛḍham\lem}
}\Bfootnote{%
	STTS 144
%	\textit{aho hi sarvabuddhānāṁ ratnavajram ahaṁ smṛtam |
%	yan mudrāṇāṁ hi sarvāsām a\-bhi\-ṣekanayaṁ dṛḍham iti ||}
}} //} \\
	%ratnavajrām: em. ratnavajram, with MSK (sil. em.).
	%smṛtam·: perhaps em. smṛtaḥ, with MSK.
	%sarvāśām: em. sarvāsām, with MSK (sil. em.).
	%abhi\unclear{ṣ}ekataḥyaṁ: em. abhiṣekanayaṁ, with MSK.
	%dṛḍhaṁḥ: dṛḍhaḥ MSK. Em. dṛḍham, with MSK.
%
\textbf{[Dharmavajrī]}\\%
\Skt{\edtext{aho hi sarvabuddhānāṃ dharmavajramahaṃ śuciḥ /\\ 
yatsvabhāvaviśuddhyā vai \edtext{rāgo'pi}{\lemma{%
	{\rm rāgo 'pi\lem}
}\Dfootnote{%
	\emn\ \MSK\ $\leftarrow$STTS \cod\ \Hed\ \textcolor{red}{CHECK YAMADA!!};
	\textit{rāṇyepi} \cod
}} hi \edtext{sunirmalaḥ}{\lemma{%
	{\rm sunirmalaḥ\lem}
}\Dfootnote{%
	\emn\ \MSK\ $\leftarrow$STTS \cod\ \Hed\ \textcolor{red}{CHECK YAMADA!!};
	\textit{sunirmmalaṁ} \cod
}}}{\lemma{%
	{\rm aho \dots\ sunirmalaḥ\lem}
}\Bfootnote{%
	STTS 147
%	\textit{aho hi sarvabuddhānāṁ dharmavajram ahaṁ śuciḥ |
%	yatsvabhāvaviśuddhyā vai rāgo 'pi hi sunirmala iti ||}
}} //} \\
	%rāṇyapi: em. rāgo 'pi, with MSK.
	%sunirmmalaṁ: em. sunirmalaḥ, with MSK.
%
\textbf{[Karmavajrī]}\\%
\Skt{\edtext{aho hi sarvabuddhānāṃ karmavajramahaṃ \edtext{bahuḥ}{\lemma{%
	{\rm bahuḥ\lem}
}\Dfootnote{
	\cod\ \MSK;
	\textit{bahu} STTS (\cod\ \Hed\ \textcolor{red}{CHECK YAMADA!!})
}} /\\
yadekaḥ sannaśeṣasya \edtext{sattvadhātoḥ}{\lemma{%
	{\rm sattvadhātoḥ\lem}
}\Dfootnote{%
	\emn\ \MSK\ \sil\ $\leftarrow$(STTS \cod\ \Hed\ \textcolor{red}{CHECK YAMADA!!});
	\textit{satvadhāto} \cod
}} sukarmakṛditi}{\lemma{%
	{\rm aho \dots\ sukarmakṛd iti\lem}
}\Bfootnote{%
	STTS 150
%	\textit{aho hi sarvabuddhānāṁ karmavajram ahaṁ bahu |
%	ya-d-ekaḥ sann aśeṣasya sattvadhātoḥ sukarmakṛd iti ||}
}} //} 
	%satvadhāto |: em. sattvadhātoḥ, with MSK (sil. em.).
	%ya\unclear{d}
%
\pend

\bigskip

\pstart\noindent
{\large 2.2.2.3. The Four Goddesses of Worship in the Inner Circle}
\pend

\bigskip

\pstart
\Skt{punaḥ vairocanāhaṃkāreṇa lāsyādi\edtext{catuṣṭayam}{\lemma{%
	{\rm -catuṣṭayam\lem}
}\Dfootnote{%
	\emn\ \MSK;
	\textit{-catuṣṭhayam} \cod
}} / \edtext{udānānyāsām}{\lemma{%
	{\rm udānāny āsām\lem}
}\Dfootnote{%
	\emn\ \MSK;
	\textit{uddānayāmāsām} \cod;
	MSK reads \textit{udānayāmāsām} and emends \textit{udānāny āsām}; we accept the emendation. %The corruption is influenced by STTS §154 \textit{udānayām āsa}.
	% ; MSK reads udānayāmāsām and emends udānāny āsām. We accept the emendation.
}} //}
	%-catuṣṭhayam·: em. -catuṣṭayam, with MSK.
	%°uddānayāmāsām·: MSK reads udānayāmāsām and emends udānāny āsām. We accept the emendation. The corruption is influenced by STTS §154 udānayām āsa.
\pend

\verse
\textbf{[Vajralāsyā]}\\%
\Skt{\edtext{aho na \edtext{sadṛśī}{\lemma{%
	{\rm sadṛśī\lem}
}\Dfootnote{%
	\emn\ \MSK\ \sil\ ($\leftarrow$ \textcolor{red}{STTS});
	\textit{ṣadṛsī} \cod
}} me'sti pūjā \edtext{hyanyā}{\lemma{%
	{\rm anyā\lem}
}\Dfootnote{%
	\emn\ \MSK\ \sil\ ($\leftarrow$ \textcolor{red}{STTS});
	\textit{at\unclear{ya}ḥ} \cod
}} svayaṃbhuvām /\\ 
yatkāmaratipūjābhiḥ sarvapūjā pravartate}{\lemma{%
	{\rm aho \dots\ pravartate\lem}
}\Bfootnote{%
	STTS 154
%	\textit{aho na sadṛśī me 'sti pūjā hy anyā svayaṁbhuvām |
%	yat kāmaratipūjābhiḥ sarvapūjā pravartata iti ||}
}} //} \\
	%ṣadṛsī: em. sadṛśī, with MSK (sil. em.).
	%at\unclear{ya}ḥ: em. anyā, with MSK (sil. em.).
%	
\textbf{[Vajramālā]}\\%
\Skt{\edtext{aho hyasadṛśāhaṃ vai ratnapūjeti kīrtitā /\\ 
yattraidhātuka\edtext{rājyāgryaṃ}{\lemma{%
	{\rm -rājyāgryaṁ\lem}
}\Dfootnote{%
	\emn\ \MSK\ ($\leftarrow$ \textcolor{red}{STTS});
	\textit{-rājāgryaṁ} \cod
}} śāsayanti \edtext{prapūjitāḥ}{\lemma{%
	{\rm prapūjitāḥ\lem}
}\Dfootnote{%
	\emn\ \MSK\ ($\leftarrow$ \textcolor{red}{STTS});
	\textit{prapūjitā} \cod
}}}{\lemma{%
	{\rm aho \dots\ prapūjitāḥ\lem}
}\Bfootnote{%
	STTS 157
%	\textit{aho hy asadṛśāhaṁ vai ratnapūjeti kīrtitā |
%	yat traidhātukarājyāgryaṁ śāsayanti prapūjitā iti ||}
}} //} \\
	%prapūjitā: em. prapūjitāḥ, with MSK.
	%-rājāgryaṁ: em. -rājyāgryaṁ, with MSK.
%
\textbf{[Vajragītā]}\\%
\Skt{\edtext{aho hi saṃgītimayī pūjāhaṃ \edtext{sarvadarśinām}{\lemma{%
	{\rm sarvadarśinām\lem}
}\Dfootnote{%
	\emn\ \MSK\ ($\leftarrow$ \textcolor{red}{STTS});
	\textit{sarvadarsinam} \cod;
	\MSK\ reads \textit{sarvadarśinam}.
}} /\\ 
ya\edtext{ttoṣayanti}{\lemma{%
	{\rm toṣayanti\lem}
}\Dfootnote{%
	\emn\ \MSK\ \sil\ ($\leftarrow$ \textcolor{red}{STTS});
	\textit{tośayanti} \cod
}} pūjābhiḥ \edtext{pratiśrutkopameṣva}{\lemma{%
	{\rm pratiśrutkopameṣv\lem}
}\Dfootnote{%
	\emn\ \MSK\ ($\leftarrow$ \textcolor{red}{STTS});
	\textit{pratiśrutakopameṣv} \cod
}}pi}{\lemma{%
	{\rm aho \dots\ pratiśrutkopameṣv api\lem}
}\Bfootnote{%
	STTS 160
%	\textit{aho hi saṁgītimayī pūjāhaṁ sarvadarśinām |
%	yat toṣayanti pūjābhiḥ pratiśrutkopameṣv apīti ||}
}} //} \\
	%sarvadarsinam·:  sarvadarśinam MSK; em. sarvadarśinām, with MSK.
	%tośayanti: em. toṣayanti, with MSK (sil. em.).
	%pratiśrutakopameṣv: em. pratiśrutkopameṣv, with MSK.
%
\textbf{[Vajranṛtyā]}\\%
\Skt{\edtext{aho hyudārapūjāhaṃ sarvapūjārthakāriṇām /\\ 
yadvajra\edtext{nṛtya}{\lemma{%
	{\rm -nṛtya-\lem}
}\Dfootnote{%
	\cod\ \MSK;
	\textit{-nṛtta-} STTS (\Hed) \textcolor{red}{CHECK MS AND YAMADA!!}
}}vidhinā \edtext{buddhapūjā}{\lemma{%
	{\rm buddhapūjā\lem}
}\Dfootnote{%
	\emn\ ($\leftarrow$ \textcolor{red}{STTS});
	\textit{buddhapūjāḥ} \cod
}} \edtext{prakalpyata iti}{\lemma{%
	{\rm prakalpyata iti\lem}
}\Dfootnote{%
	\emn\ ($\leftarrow$ \textcolor{red}{STTS});
	\textcolor{red}{prakalpyateti} \cod;
	\MSK\ as usual separates the \textit{iti} and edits \textit{prakalpyate || iti}.
}}}{\lemma{%
	{\rm aho \dots\ prakalpyata iti\lem}
}\Bfootnote{%
	STTS 163
%	\textit{aho hy udārapūjāhaṁ sarvapūjārthakāriṇām |
%	yad vajra\textcolor{red}{nṛtta}vidhinā buddhapūjā prakalpyata iti ||}
}} //} 
	%prakalpyateti: em. prakalpyata iti; MSK as usual separates the iti and edits prakalpyate || iti.
\pend

\bigskip

\pstart\noindent
{\large 2.2.2.5. The Four Goddesses of Worship at the Corners}
\pend

\bigskip

\pstart
\Skt{punarakṣobhyādyahaṃkāreṇa vajradhūpādicatuṣṭayam / \edtext{udānamāsām}{\lemma{%
	{\rm udānāny āsām\lem}
}\Dfootnote{%
	\emn;
	\textit{udānānyāsām} \emn\ \MSK;
	\textit{uddānayāmāsām} \cod
	%\MSK\ again reads \textit{udānayāmāsām} and emends \textit{udānāny āsām}. We accept the emendation. 
}} //}
	%°uddānayāmāsām·: MSK again reads udānayāmāsām and emends udānāny āsām. We accept the emendation. 
\pend

\verse
\textbf{[Vajradhūpā]}\\%
\Skt{\edtext{aho hyahaṃ mahāpūjā prahlādanavatī śubhā /\\ %\newfolio{24r1} 
\edtext{ya}{\lemma{}\Efootnote{%
	śubhā | [24r1] yat
}}\edtext{tsattvāveśayogā}{\lemma{%
	{\rm -sattvāveśayogād\lem}
}\Dfootnote{%
	\emn\ \MSK\ \sil;
	\textit{-satvāveśayogod} \cod
}}ddhi kṣipraṃ bodhiravāpyate}{\lemma{%
	{\rm aho \dots\ avāpyate\lem}
}\Bfootnote{%
	STTS 167
%	\textit{aho hy ahaṁ mahāpūjā prahlādanavatī śubhā |
%	yatsattvāveśayogād dhi kṣipraṁ bodhir avāpyata iti ||}
}} //} \\
	%satvāveśayogod: em. sattvāveśayogād, with MSK (sil. em.).
%
\textbf{[Vajrapuṣpā]}\\%
\Skt{\edtext{aho hi puṣpapūjāhaṃ \edtext{sarvālaṃkāra}{\lemma{%
	{\rm sarvālaṁkāra-\lem}
}\Dfootnote{%
	\emn\ \MSK\ \sil;
	\textit{sarvālāṅkāra-} \cod
}}kārikā / \\
\edtext{yattathāgata}{\lemma{%
	{\rm yat tathāgata-\lem}
}\Dfootnote{%
	\emn\ \MSK\ \sil\ ($\leftarrow$STTS);
	\textit{yat sarvatathāgata-} \cod
}}ratnatvaṃ \edtext{pūjya}{\lemma{%
	{\rm pūjya\lem}
}\Dfootnote{%
	\emn\ \MSK\ \sil;
	\textit{pūjā} \cod
%	emend \textit{pūjya}, with \MSK\ and STTS? Or \textit{pūjyaṁ}?
}} kṣipramavāpyate}{\lemma{%
	{\rm aho \dots\ avāpyate\lem}
}\Bfootnote{%
	STTS 170
%	\textit{aho hi puṣpapūjāhaṁ sarvālaṁkārakārikā |
%	yat tathāgataratnatvaṁ pūjya kṣipram avāpyata iti ||}
}} //} \\
	%sarvālāṅkāra-: em. sarvālaṁkāra-, with MSK (sil. em.).
	% yat sarvatathāgata-: em.  yat tathāgata-, with MSK.
	%pūjā: em. pūjya, with MSK and STSS? Or pūjyaṁ?
%
\textbf{[Vajradīpā]}\\%
\Skt{\edtext{aho hya\supplied{haṃ} mahodārā pūjā dīpamayī śubhā /\\ 
yadālokavatī kṣipraṃ sarvabuddha\edtext{dṛśo}{\lemma{%
	{\rm -dṛśo\lem}
}\Dfootnote{%
	\emn\ \MSK\ \sil\ ($\leftarrow$STTS);
	\textit{-dṛso} \cod
}} labhet}{\lemma{%
	{\rm aho \dots\ labhet\lem}
}\Bfootnote{%
	STTS 173.
%	\textit{aho hy ahaṁ mahodārā pūjā dīpamayī śubhā |
%	yad ālokavatī kṣipraṁ sarvabuddhadṛśo labhed iti ||};
	It seems that \textit{ālokavatī} should be understood as or emended to \textit{ālokavatīḥ}.
%	\textcolor{red}{CHECK MS AND YAMADA!!}
}} //} \\
	%-dṛso: em. -dṛśo, with MSK (sil. em.).
	%aha\supplied{ṁ}
%
\textbf{[Vajragandhā]}\\%
\Skt{\edtext{aho gandhamayī pūjā divyāhaṃ hi manoramā /\\ 
yattathāgata\edtext{gandhaṃ}{\lemma{%
	{\rm -gandhaṃ\lem}
}\Dfootnote{%
	\emn\ \MSK;
	\textit{-gandho} \cod
%	emend \textit{-gandhaṁ}, with MSK?
}} vai \edtext{sarvakāye}{\lemma{%
	{\rm sarvakāye\lem}
}\Dfootnote{%
	\emn;
	\textit{sarvakāma} \cod;
	\MSK\ reads \textit{sarvakāme} and emend \textit{sarvakāye}.
}} \edtext{da\supplied{dā}ti}{\lemma{%
	{\rm dadāti\lem}
}\Dfootnote{%
	\emn\ \MSK\ \sil;
	\textit{da.āti} \cod
}} hīti}{\lemma{%
	{\rm aho \dots\ dadāti hīti\lem}
}\Bfootnote{%
	STTS 176
%	\textit{aho gandhamayī pūjā divyāhaṁ hi manoramā |
%	yat tathāgatagandhaṁ vai sarvakāye dadāti hīti ||}
}} //} 
	%-gandho: em. -gandhaṁ, with MSK?
	%sarvakāma: sarvakāme MSK; em. sarvakāye, with MSK?
	%dada\supplied{d}āti: em. dadāti, with MSK (sil. em.).
\pend

\bigskip

\pstart\noindent
{\large 2.2.2.6. The Four Gate-keepers}
\pend

\bigskip

\pstart
\Skt{\edtext{ato}{\lemma{%
	{\rm ato\lem}
}\Dfootnote{%
	\emn\ \MSK\ \sil;
	\textit{ata} \cod
}} vairocanāhaṃkāreṇa vajrāṅkuśādicatuṣṭayam / \edtext{udāna}{\lemma{%
	{\rm udānam\lem}
}\Dfootnote{%
	\emn\ \MSK\ \sil;
	\textit{uddānam} \cod
}}meṣām //}
	%°ata: em. °ato, with MSK (sil. em.).
	%°uddānam: em. udānam, with MSK (sil. em.).
\pend

\verse
\textbf{[Vajrāṅkuśa]}\\%
\Skt{\edtext{aho hi sarvabuddhānāṃ samākarṣamahaṃ \edtext{dṛḍhaḥ}{\lemma{%
	{\rm dṛḍhaḥ\lem}
}\Dfootnote{%
	\emn\ \MSK\ ($\leftarrow$STTS);
	\textit{dṛḍhaṁ} \cod
%	emend \textit{dṛḍhaḥ}, with MSK?
}} /\\ 
yanmayā hi \edtext{samākṛṣṭā}{\lemma{%
	{\rm samākṛṣṭā\lem}
}\Dfootnote{%
	\emn\ \MSK\ \sil;
	\textit{samākṛṣṭhā} \cod
}} bhajante sarvamaṇḍalam}{\lemma{%
	{\rm aho \dots\ sarvamaṇḍalam\lem}
}\Bfootnote{%
	STTS 180
%	\textit{aho hi sarvabuddhānāṁ samākarṣa-m-aham dṛḍhaḥ |
%	yan mayā hi samākṛṣṭā bhajante sarvamaṇḍalam iti ||}
}} //} \\
	%dṛḍhaṁ: em. dṛḍhaḥ, with MSK?
	%samākṛṣṭhā: em. samākṛṣṭā, with MSK (sil. em.).
	%\unclear{sa}rvamaṇḍalaṁ
%
\textbf{[Vajrapāśa]}\\%
\Skt{\edtext{aho hi sarvabuddhānāṃ vajrapāśamahaṃ \edtext{dṛḍhaḥ}{\lemma{%
	{\rm dṛḍhaḥ\lem}
}\Dfootnote{%
	\emn\ \MSK;
	\textit{dṛḍhaṃ} \cod
}} /\\ 
ya\edtext{tsarvatra}{\lemma{%
	{\rm sarvatra\lem}
}\Dfootnote{%
	\cod\ \MSK;
	\textit{sarvāṇu-} STTS (\Hed) \textcolor{red}{CHECK MS AND YAMADA!!}
}} \edtext{praviṣṭāpi}{\lemma{%
	{\rm praviṣṭāpi\lem}
}\Dfootnote{%
	\emn\ \MSK\ \sil;
	\textit{praviṣṭhāpi} \cod
}} praveśyante mayā punaḥ}{\lemma{%
	{\rm aho \dots\ punaḥ\lem}
}\Bfootnote{%
	STTS 183
%	\textit{aho hi sarvabuddhānāṁ vajrapāśa-m-ahaṁ dṛḍhaḥ |
%	yat sarvāṇupraviṣṭāpi praveśyante mayā punar iti ||}
}} //} \\
	%dṛḍhaṁ: em. dṛḍhaḥ, with MSK?
	%praviṣṭhāpi: em. praviṣṭāpi, with MSK (sil. em.).
	%STTS §183 reads sarvāṇupraviṣṭhāpi.
%
\textbf{[Vajrasphoṭa]}\\%
\Skt{\edtext{aho hi sarvabuddhānāṃ vajrasphoṭamahaṃ \edtext{dṛḍhaḥ}{\lemma{%
	{\rm dṛḍhaḥ\lem}
}\Dfootnote{%
	\emn\ \MSK;
	\textit{dṛḍhaṃ} \cod
}} / \\
yatsa\supplied{rva}\edtext{bandhamuktānāṃ}{\lemma{%
	{\rm -bandhamuktānāṁ\lem}
}\Dfootnote{%
	\emn\ \MSK;
	\textit{-vandhanamuktānāṁ} \cod
}} \edtext{sattvārthādbandha}{\lemma{%
	{\rm sattvārthād bandha\lem}
}\Dfootnote{%
	\emn\ \MSK\ \sil;
	\textit{satvārthān vandha} \cod
}} iṣyate}{\lemma{%
	{\rm aho \dots\ iṣyate\lem}
}\Bfootnote{%
	STTS 186
%	\textit{aho hi sarvabuddhānāṁ vajrasphoṭa-m-ahaṁ dṛḍhaḥ |
%	yat sarvabandhamuktānāṁ sattvārthād bandha iṣyata iti ||}
}} //} \\
	%dṛḍhaṁ: em. dṛḍhaḥ, with MSK?
	%-vandhanamuktānāṁ: em. -vandhamuktānāṁ, with MSK.
	%satvārthān vandha: em. sattvārthād bandha, with MSK (sil. em.).
	%sa\supplied{r}\unclear{va}vandhanamuktānāṁ
%
\textbf{[Vajrāveśa]}\\%
\Skt{\edtext{aho hi sarvabuddhānāṃ vajrāveśamahaṃ \edtext{dṛḍhaḥ}{\lemma{%
	{\rm dṛḍhaḥ\lem}
}\Dfootnote{%
	\emn\ \MSK;
	\textit{dṛḍhaṃ} \cod
}} /\\ 
ya\edtext{tsarvapatayo}{\lemma{%
	{\rm sarvapatayo\lem}
}\Dfootnote{%
	\emn\ \MSK;
	\textit{sarvāpatayo} \cod
}} bhūtvā ceṭā api bhavanti hīti}{\lemma{%
	{\rm aho \dots\ bhavanti hīti\lem}
}\Bfootnote{%
	STTS 189
%	\textit{aho hi sarvabuddhānāṁ vajrāveśa-m-ahaṁ dṛḍhaḥ |
%	yat sarvapatayo bhūtvā ceṭā api bhavanti hīti ||}
}} //} 
\pend
\medskip
\pstart
\mbox{}\hfill \Skt{// iyatā maṇḍalarājāgrī nāma samādhiḥ //}\hfill \mbox{}
	%dṛḍhaṁ: em. dṛḍhaḥ, with MSK?
	%sarvāpatayo: em. sarvapatayo, with MSK.
\pend




\bigskip

% karmarājāgrī

\pstart\noindent
{\large 2.3. Karmarājāgrī}
\pend

\bigskip

\pstart
\edtext{}{\lemma{%
	Section 2.3\lem
}\Bfootnote{%
	$\rightarrow$ KSP (\Ied\ sections 157–159)
}}%
\Skt{tato vairocanena sahābhinnamātmānaṃ vicintya samājaṃ kuryāt / tata\edtext{stānsa\-mājaga\-tānsarvatathāgatā}{\lemma{%
	{\rm tān samājagatān sarvatathāgatān\lem} 
}\Dfootnote{%
	\emn\ \MSK;
	\textit{tāṁ samājāgatāṁ sarvatathāgatāṁ} \cod;
	\textit{tān samājagatān sarvabuddha-} KSP (\Ied)
}}\edtext{nsabodhisattvaparṣa\edtext{nmaṇḍalān}{\lemma{%
	{\rm -maṇḍalān\lem}
}\Dfootnote{
	\emn\ \MSK\ KSP (\Ied);
	\textit{-maṇḍalam} \cod
}}}{\lemma{}\Efootnote{%
	sabodhisattva[24v1]{\siddham}parṣanmaṇḍalān
}}  \edtext{{\om} \edtext{sarva}{\lemma{%
	{\rm sarva-\lem}
}\Dfootnote{%
	\emn\ \MSK\ \sil\ KSP (\Ied);
	\textit{sava-} \cod
}}tathāgatapāda\edtext{vanda\textcolor{red}{naṃ}}{\lemma{%
	{\rm -vandanaṁ\lem}
}\Dfootnote{%
	\emn\ \MSK\ (which reports as their reading \textit{-vandhanaṁ}, with an \textit{anusuvāra} that is not actually written);
	\textit{-vandhana} \cod;
	\textit{-vandanāṃ} STTS (\Hed) \textcolor{red}{CHECK MS AND YAMADA!!}; 
	\textit{-vandanāṅ} KSP (\Ied)
}} karomītyu}{\lemma{%
	{\rm oṁ \dots\ karomīty\lem}
}\Bfootnote{%
	STTS 192
%	\textit{oṁ sarvatathāgatapādavandanāṁ karomīty anena prakṛtisiddhena mantre \dots}
}}\-dīrya /}
\pend

\verse
\Skt{\edtext{aho samantabhadrasya bodhisattvasya satkriyā / \\
yattathāgatacakrasya madhye bhāti tathāgataḥ}{\lemma{%
	{\rm aho \dots\ tathāgataḥ\lem}
}\Bfootnote{%
	STTS 193
%	\textit{aho samantabhadrasya bodhisattvasya satkriyā |
%	yat tathāgatacakrasya madhye bhāti tathāgataḥ ||}
%	quoted in KSP (\Ied\ section 157)
}} //}
\pend
% 20210713: satkriyā: aiśa verb form?

\pstart\noindent
\Skt{ityudānamudānayamānāṃścintayet //}
	%tāṁ samājāgatāṁ sarvatathāgatāṁ: em. tān samājagatān sarvatathāgatān, with MSK.
	%-maṇḍalam·: em. maṇḍalān·, with MSK.
	%-vandhana: em. -vandanaṁ, with MSK (which reports as their reading -vandhanaṁ, with an ansuvāra that is not actually written).
	%sava-: em. sarva, with MSK (sil. em.).
	%udānaya○mānāṁś: udānayāmānāṁś MSK (misprint?).
	%cittayet·: em. cintayet·, with MSK (sil. em.)
	%the mantra is STTS §192; the verse is STTS §193.
	%from this section onward, there is a long parallel in KSP.
\pend

\pstart
\Skt{tato \edtext{vairocanahṛdaye}{\lemma{%
	{\rm vairocanahṛdaye\lem}
}\Dfootnote{%
	\emn\ \MSK\ \sil;
	\textit{vairocanahṛdaya} \cod;
	\textit{vairocanasya hṛdaye} KSP (\Ied)
}} praviśya niṣkramya sattvavajrādibhirekībhūya punarapyudāna\edtext{mudānaya\-mānān}{\lemma{%
	{\rm udānayamānān\lem}
}\Dfootnote{%
	\emn\ \MSK;
	\textit{udānayām\supplied{ā}nām} \cod;
	\textit{udānayamānaḥ} KSP (\Ied)
}} /}
\pend

\verse
\Skt{\edtext{aho hi sarvabuddhānāṃ mahaudāryamanādijam /\\ 
\edtext{yatsarvāṇuprasaṃkhyā}{\lemma{%
	{\rm yat sarvāṇuprasaṁkhyā\lem}
}\Dfootnote{%
	\emn\ \MSK\ following STTS;
	\textit{sarvāṇuprasaṁkhyā} \cod;
	\textit{sarvānuprasaṁkhyā} KSP (\Ied)
}} vai buddhā hyekatvamāgatā}{\lemma{%
	{\rm aho \dots\ āgatā\lem}
}\Bfootnote{%
	STTS 195;
%	\textit{aho hi sarvabuddhānāṁ mahaudāryam anādijam |
%	yat sarvāṇu\-prasaṁkhyā vai buddhā hy ekatvam āgatā iti ||}
	quoted in KSP (\Ied\ section 158)
}} iti //}
	%vairocanahṛdaya: em. vairocanahṛdaye, with MSK (sil. em.).
	%udānayām\supplied{ā}nām·: em. udānayāmānān, scil. cintayet, as in prec. section, with MSK.
	%sarvāṇuprasaṁkhyā: inserts yat before this, with MSK.
	%the verse is STTS §195.
\pend

\pstart
\Skt{tataḥ śrīvajrasattvasthānastho yogī sarvatathāgatebhyo \edtext{mālābhiṣekādika}{\lemma{%
	{\rm mālābhiṣekādikam\lem}
}\Dfootnote{%
	\emn\ \MSK;
	\textit{mālābhikākādikam} \cod;
	\textit{mālābhiṣekādim} KSP (\Ied)
}}mādāya \edtext{vajrāṅkuśādibhi}{\lemma{%
	{\rm vajrāṅkuśādibhir\lem}
}\Dfootnote{%
	\emn\ \MSK\ (with wrong reading \textit{-śādīr} in the footnote) KSP (\Ied);
	\textit{vajrāṅkuśādir} \cod
}}\edtext{rākṛṣya}{\lemma{%
	{\rm ākṛṣya\lem}
}\Dfootnote{%
	\emn\ \MSK\ \sil\ KSP (\Ied);
	\textit{ākṛśya} \cod
}} praveśya baddhvā vaśīkṛtya caturmudrā\edtext{mudritān vairocanādīn}{\lemma{%
	{\rm -mudritān vairocanādīn\lem}
}\Dfootnote{%
	\emn\ \MSK\ KSP (\Ied);
	\textit{-mudritāṁ vairocanādī} \cod
}} kṛtvā sarvabuddhaika\edtext{saṃgrahe}{\lemma{%
	{\rm -saṁgrahe\lem}
}\Dfootnote{%
	\emn\ \MSK\ KSP (\Ied);
	\textit{-saṁgrahai} \cod
}} sarvakulāmudraṇe / bodhicittotpādane / sarvatathāgatākarṣaṇe / \edtext{anurāge}{\lemma{%
	{\rm anurāge\lem}
}\Dfootnote{%
	\emn\ \MSK\ \sil;
	\textit{anurāgai} \cod;
	\textit{anurāgaṇe} KSP (\Ied)
}} / toṣaṇe / abhiṣeke / \edtext{prabhāsā}{\lemma{%
	{\rm prabhāsā-\lem}
}\Dfootnote{%
	\emn\ ($\leftarrow$KSP (\Ied));
	\textit{prabhayā-} \cod\ \MSK
}}vabhāsane / dānapāramitāniyojane / smitādbhuta\edtext{pratiṣṭhāpane}{\lemma{%
	{\rm -pratiṣṭhāpane\lem}
}\Dfootnote{%
	\emn\ \MSK\ \sil\ KSP (\Ied);
	\textit{-pratiṣṭhāpaṇe} \cod
}} / suviśuddhisamādhiniṣpādane / \edtext{kleśopa}{\lemma{%
	{\rm kleśopa-\lem}
}\Dfootnote{%
	\emn\ \MSK\ \sil\ KSP (\Ied);
	\textit{klesopa-} \cod
}}kleśacchedane / mahāmaṇḍalapraveśane / niṣprapañcadharmatāniyojane / aśeṣānavaśeṣa\edtext{pūjayā sarvatathāgatapūjane}{\lemma{}\Efootnote{
	-pūjayā [25r1] sarvatathāgatapūjane
}} / anyayānaspṛhā\edtext{cittā}{\lemma{%
	{\rm -cittāt\lem}
}\Dfootnote{%
	\emn\ \MSK\ \sil;
	\textit{-cittā} \cod\ KSP (\Ied)
}}\edtext{tkleśopa}{\lemma{%
	{\rm kleśopa-\lem}
}\Dfootnote{%
	\emn\ \MSK\ \sil\ KSP (\Ied);
	\textit{klesopa} \cod
}}kleśādibhayebhya\edtext{ścārakṣaṇe}{\lemma{%
	{\rm cārakṣaṇe\lem}
}\Dfootnote{%
	\cod\ KSP (\Ied);
	\MSK\ silently emends \textit{ca rakṣaṇe} \emn\ \MSK\ \sil\ (the emendation is unnecessary.)
}} / \edtext{sarvarakṣāparipālane}{\lemma{%
	{\rm sarvarakṣāparipālane\lem}
}\Dfootnote{%
	\cod\ KSP (\Ied);
	\MSK\ needlessly emends \textit{sarvarakṣāt paripālane}.
}} / kāyavākcittaikīkaraṇabandha\edtext{tathāgata}{\lemma{%
	{\rm -tathāgata-\lem}
}\Dfootnote{%
	\emn\ ($\leftarrow$STTS 132, KSP);
	\textit{-tathatā-} \cod\ \MSK
%	KSP seems to read \textit{-tathāgata-}.
}}\edtext{muṣṭyā}{\lemma{%
	{\rm -muṣṭyā\lem}
}\Dfootnote{%
	\emn\ \MSK\ \sil;
	\textit{-muṣṭhyā \unclear{|}} \cod
}} \edtext{sarvabuddhatvaniṣpādane}{\lemma{%
	{\rm sarvabuddhatvaniṣpādane\lem}
}\Dfootnote{%
	\cod;
	\MSK\ misreads \textit{-buddhaddha-} and emends \textit{sarvabuddhaniṣpādane};
	n.e.\ KSP (\Ied)
}} / \edtext{dānaśīlakṣāntivīryaprajñādhyānapraṇidhyupāyaniṣpādane}{\lemma{%
	{\rm dāna\dots niṣpādane\lem}
}\Cfootnote{%
	why only 8 pāramitās, and not 10?
	KSP too mentions only the same 8.
}} / bodhicittāṅkuśena mahāmokṣapure sarvasattvākarṣaṇe / daśapāramitācaryayā praveśane / anyayānapṛhācittasphoṭane / prakṛtiprabhāsvarānutpādāveśe / saddharmanagarapālane ca niyojayediti //}
($\leftarrow$KSP)
\pend
\medskip
\pstart
\mbox{}\hfill \Skt{// karmarājāgrī nāma samādhiḥ //}\hfill\mbox{}
	%mālābhikākādikam: mālābhiṣekādikam, with MSK.
	%vajrāṅkuśādir: em. vajrāṅkuśādibhir, with MSK (with wrong reading -śādīr in the footnote).
	%ākṛśya: em. ākṛṣya, with MSK (sil. em.).
	%vadhvā: em. baddhvā, with MSK (sil. em.). BUT NO NEED TO REPORT THIS.
	%-mudritāṁ vairocanādī: em. -mudritān vairocanādīn, with MSK.
	%-saṁgrahai: em. -saṁgrahe, with MSK.
	%°anurāgai: em. °anurāge, with MSK (sil. em.).
	%-ṣṭhāpaṇe: em. -ṣṭhāpane, with MSK (sil. em.).
	%kle§○sopa-: em. kleśopa-, with MSK (sil. em.).
	%-ci§○ttā: em. -cittāt, with MSK (sil. em.).
	%klesopa-: em. kleśopa-, with MSK (sil. em.).
	%cārakṣaṇe: MSK silently emends ca rakṣaṇe, but the emendation is unnecessary.
	%sarvarakṣāparipālane:  MSK emends sarvarakṣāt paripālane, but the emendation is unwanted.
	%-tathatā-: KSP seems to read -tathāgata-.
	%-muṣṭhyā \unclear{|}: em. -muṣṭyā (without |), with MSK (sil. em.).
	%sarvavuddhatvaniṣpādane: MSK misreads -buddhaddha- and emends sarvabuddhaniṣpādane.
	%why only 8 pāramitās, and not 10?
\pend

\bigskip

\pstart\noindent
{\large 2.4. Supplement \textcolor{red}{(Reconsider the section title!!)}}
\pend

\bigskip

\pstart
%RT&AG
\Skt{tato vajrayakṣaparijapta\edtext{gandhodakena}{\lemma{%
	{\rm -gandhodakena\lem}
}\Dfootnote{%
	\emn\ \MSK;
	\textit{-gandhodake} \cod
}} sarvapūjāṅgāni prokṣya / \edtext{vajrānalamudreṇa}{\lemma{%
	{\rm vajrānalena mudreṇa\lem}
}\Dfootnote{%
	\cod;
	\emn\ \textit{vajrānalena} or \textit{vajrānalena mantreṇa}? \MSK\ emends \textit{vajrānalamudrayā}.
	\textcolor{red}{Probably \MSK's\ ememdation is based on Tib.: \textit{rdo rje me'i phyag rgya dang bcas pas}}
	\textcolor{red}{Mañjuśriyamūlakalpa, Hevajratantra}
}} parijapya /  \edtext{{\om}}{\lemma{%
	{\rm oṃ\lem}
}\Dfootnote{%
	\emn;
	n.e. \cod
}} \edtext{vajrapuṣpe}{\lemma{%
	{\rm vajrapuṣpe\lem}
}\Dfootnote{%
	\emn\ \MSK;
	\textit{vajrapuṣpa} \cod
}} \edtext{hū{\cb} iti}{\lemma{%
	{\rm hū;ṁ iti\lem}
}\Dfootnote{%
	\emn;
	\textit{hū;ṁmiti} \cod;
	\textit{hūṁ || iti} \emn\ \MSK\ \sil
}} puṣpamudrayā \edtext{puṣpāṇi}{\lemma{%
	{\rm puṣpāṇi\lem}
}\Dfootnote{%
	\cod;
	\textit{puṣpāni} \MSK
}} / {\om} vajragandhe \edtext{hū{\cb}}{\lemma{%
	{\rm hū;ṁ\lem}
}\Dfootnote{%
	\emn;
	n.e. \cod
}} iti gandhamudrayā \edtext{gandham}{\lemma{%
	{\rm gandham\lem}
}\Dfootnote{%
	\emn\ \MSK;
	\textit{gandhat} \cod
}} / {\om} vajradhūpe \edtext{hū{\cb} iti}{\lemma{%
	{\rm hū;ṁ iti\lem}
}\Dfootnote{%
	\emn;
	\textit{hū\textcolor{red}{ṁ}miti} \cod;
	\textit{hūṁ || iti} \emn\ \MSK\ \sil
}} dhūpamudrayā dhūpam / \edtext{\mbox{}\edtext{{\om}}{\lemma{%
	{\rm oṁ\lem}
}\Dfootnote{%
	\emn;
	n.e. \cod
}} akāro mukhaṃ sarvadharmāṇāmādyanutpannatvāditi}{\lemma{%
	{\rm oṁ \dots\ -anutpannatvād\lem}
}\Bfootnote{%
%
\textit{Pañcaviṃśatisāhasrikā Prajñāpāramitā} I-2: 85
%
}} khaḍgamudrayā balim / vajrabandhena \edtext{khaḍgākarṣābhinayā}{\lemma{%
	{\rm khaḍgākarṣābhinayā\lem}
}\Dfootnote{%
	\emn;
	\textit{khaḍgātkarṣābhinayā} \cod;
	\MSK\ accepts the text as it is.
}} khaḍgamudrā / \edtext{{\om} vajrāloke hū{\cb}}{\lemma{%
	{\rm oṁ vajrāloke hū;ṁ\lem}
}\Dfootnote{%
	\emn;
	\textit{hūṁ vajrāloke} \cod\ \MSK
}} iti dīpamudrayā dīpam/ {\om} vajrasattva \edtext{hū{\cb} iti}{\lemma{%
	{\rm hū;ṁ iti\lem}
}\Dfootnote{%
	\emn;
	\textit{hūṁmiti} \cod;
	\textit{hūṁ || iti} \emn\ \MSK\ \sil
}} / tilayavakuśalājāsitasugandhikusuma\edtext{candanodakāni}{\lemma{}\Efootnote{%
	candanoda[25v1]kāni
}} śaṅkhabhājanādau prakṣipya sphuṭitavajrāñjalilakṣaṇayārghamudrayā triḥ parijapya saptavāra\edtext{mekaviśati}{\lemma{%
	{\rm ekaviśati-\lem}
}\Dfootnote{%
	\emn\ \MSK\ \sil;
	\textit{ekaviśaṁti-} \cod
}}vāraṃ vā parijapya sthāpayet / 
tato yathāva\edtext{ddvārodghāṭanaṃ}{\lemma{%
	{\rm dvārodghāṭanaṁ\lem}
}\Dfootnote{%
	\emn\ \MSK\ \sil;
	\textit{dvārodghaṭānaṁ} \cod
}} kṛtvā \edtext{śrīvajrasattvamahāmudrāṃ}{\lemma{%
	{\rm śrīvajrasattvamahāmudrāṁ\lem}
}\Dfootnote{%
	\emn\ \MSK;
	\textit{śrīvajrasatvāmahāmudrām} \cod
}} baddhvā/ {\om} vajrasattva \edtext{hū{\cb} iti}{\lemma{%
	{\rm hū;ṁ iti\lem}
}\Dfootnote{%
	\emn;
	\textit{hū;ṁmiti} \cod;
	\textit{hūṁ || iti} \emn\ \MSK\ \sil
}} parijapya/ }
\pend

\verse
\Skt{\edtext{bāhubhyāṃ \edtext{vajrabandhena}{\lemma{%
	{\rm vajrabandhena\lem}
}\Dfootnote{%
	\emn;
	\textit{vajrabandhe} \cod\pc\ \MSK;
	\textit{vajrena} \cod\ac
}} vajrācchaṭāvimokṣaṇaiḥ /\\
śrīvajrasattvayogātmā \edtext{sarvabuddhān}{\lemma{%
	{\rm sarvabuddhān\lem}
}\Dfootnote{%
	\emn\ \MSK;
	\textit{sarvavuddhāṁ} \cod
}} samājayet //\\ 
vāmācchaṭakatālena \edtext{\textcolor{red}{savyatāleti}}{\lemma{%
	{\rm savyatāleti\lem}
}\Dfootnote{%
	\emn\ ($\leftarrow$SVU Tib.: \textit{g-yas pa thal mo zhes bya ste});
	\textit{samyaktāleti} \cod\ \MSK
}} siddhyati /\\ 
dakṣiṇena tu tāloktā saṃnipātāvubhā\edtext{vapīti}{\lemma{%
	{\rm apīti\lem}
}\Cfootnote{%
	\cod;
	\MSK\ note indicates that they read \textit{ayīti}.
}}}{\lemma{%
	{\rm bāhubhyāṁ\lem}
}\Bfootnote{%
%
\textit{Samāyoga} 9.320, 322:
\textit{bāhubhyāṃ vajrabandhena vajrācchaṭāvimokṣaṇaiḥ |
śrīvajrasattvayogātmā sarvabuddhān samājayet}  || 9.320 || (= \textit{Samāyoga} 5.49)
\textit{vāmācchaṭakatālena śamyatāleti sidhyati |
dakṣiṇena tu tāloktā saṃnipātāv ubhāv api}  || 9.322 ||;
%\textit{Samāyoga} 5.49
%\textit{bāhubhyāṃ vajrabandhena vajrācchaṭāvimokṣaṇaiḥ |
%śrīvajrasattvayogātmā sarvabuddhān samājayet} ||;
%\textit{sarvabuddhasamājaṃ tu vajrācchaṭāvimokṣaṇaiḥ |
% śamyatālasamājais tu tritālaṃ viniyojayet} ||5.32||
SDPT:
\textit{bāhubhyāṃ vajrabandhena vajra\textcolor{red}{chaṭa}kavimokṣaṇe |
śrīśākyarājayogātmā sarvabuddhān samājayet ||
vāme chaṭakatālena samatālena siddhyati |
dakṣiṇena tu tāloktaṃ saṃnipātāv ubhāv api ||} (p.172)
(CHECK THE READING OF THE EDITION);
\textcolor{red}{Possibly quoted from the Paramādya. CHECK!!};
Ānandagarbha's \textit{Sarvadurgatipariśodhanamaṇḍalavidhi}:
\textit{lag ngar gnyis ni rdo rje bcings || 
rdo rje se gol bcas pa yis || 
shā kya'i rgyal po sbyor bdag nyid || 
sangs rgyas thams cad bsdu bar bya ||
g-yon pa'i se gol brdabs pa dang || 
g-yas pa yis ni 'grub par 'gyur || 
g-yas pa thal mo zhes bya ste || 
'dus pa yang ni gnyis ka yin ||} (D f.193v6–7);
\textit{Abhidhānottara} (Tib.):
rdo rje sems dpa'  %@286a *| |
bdag sbyor bas || lag gnyis rdo rjer bcings pa yi || 
rdo rje se gol brdabs pa yis || sangs rgyas thams cad bsdu bar bya || 
g-yon pa'i se gol brdabs pa dang || thal mo \textbf{dal bus} brdabs pas 'grub || 
g-yas ba'i thal mo yin par gsungs || gnyis ka kun tu sdud pa yin ||
(Lhasa 285b-286a \textcolor{red}{D P Skt!!};
\textit{Abhidhānottara} (Tib.)
\textit{rdo rje sems dpa'i sbyor ba yis || dpung pa gnyis ni rdo rjer bcings || 
rdo rje se gol brdabs pa yis || sangs rgyas thams cad bsdu mdzad na ||}
(Lhasa 255a)
}} //}
\pend

\pstart\noindent
\Skt{samājamudrālakṣaṇam / {\om} vajrasamāja \edtext{\textcolor{red}{jjaḥ}}{\lemma{%
	{\rm jjaḥ\lem}
}\Cfootnote{%
	emend \textit{jaḥ} with \MSK\ \sil?
}} hū{\cb} va{\cb} hoḥ iti samāja\edtext{mudrāhṛdayam}{\lemma{%
	{\rm -mudrāhṛdayam\lem}
}\Dfootnote{%
	\emn\ \MSK;
	\textit{-mudrāhṛdayat} \cod
}} / }
\pend

\verse
\Skt{\edtext{asyājñāmātracakitāḥ \edtext{saparṣaccakra}{\lemma{%
	{\rm saparṣaccakra-\lem}
}\Dfootnote{%
	\emn\ \MSK;
	\textit{saparṣacakra-} \cod
}}saṃcayāḥ /\\
sarvabuddhāḥ samāyānti kā kathānyeṣu vartate //}{\lemma{%
	{\rm asyā\dots\ vartate\lem}
}\Bfootnote{%
%
\textit{Samāyoga} 9.321=5.50:
\textit{asyājñāmātracakitāḥ saparṣaccakrasaṃcayāḥ |
sarvabuddhāḥ samāyānti kā kathānyeṣu vartate ||};
SDPT p.172:
\textit{asyā ājñāyā mātreṇa saparṣaccakrasaṃcayaḥ |
sarvabuddhāḥ samāyānti kā kathānyeṣu varttate ||};
Ānandagarbha's \textit{Sarvadurgatipariśodhanamaṇḍalavidhi}:
\textit{'di'i bka' tsam gyis bsgrags pas || 
'khor lo'i tshogs ni 'khor bcas pa'i || 
sangs rgyas thams cad gshegs 'gyur na || 
gzhan lta smos kyang ci zhig dgos ||} (D ff.193v7–194r1);
\textit{Abhidhānottara} (Tib.)
\textit{de yi bka' tsam gyis skrag nas ||
'khor dang 'khor lo'i tshogs bcings pa ||
sangs rgyas thams cad gshegs 'gyur na ||
gzhan lta smos kyang ci zhig dgos ||}
(Lhasa f.255a \textcolor{red}{D P!! SKT!!});
\textit{de yi bka' tsam gyis skrag nas || 
'khor gyi dkyil 'khor dang bcas pa'i || 
sangs rgyas kun kyang 'du 'gyur na ||
gzhan lta smos kyang ci zhig smos ||}
(Lhasa 286a \textcolor{red}{D P!! SKT!!})
%
}}\\ 
\edtext{tataḥ \edtext{śīghraṃ}{\lemma{%
	{\rm śīghraṁ\lem}
}\Dfootnote{%
	\emn\ \MSK\ \sil;
	\textit{śrīghra} \cod
}} mahāmudrāṃ vajrasattvasya \edtext{sevayan}{\lemma{%
	{\rm sevayan\lem}
}\Dfootnote{%
	\emn\ \MSK;
	\textit{sevayaṁ} \cod
}} /\\ 
\edtext{uccāraye}{\lemma{%
	{\rm uccārayet\lem}
}\Dfootnote{%
	\emn\ \textit{uccāraye} \cod
}}tsakṛdvāraṃ \edtext{nāmāṣṭaśata}{\lemma{%
	{\rm nāmāṣṭaśatam\lem}
}\Dfootnote{%
	\emn\ \MSK;
	\textit{nāmāṣṭhagatam} \cod;
	\MSK's note reports the misreading \textit{nāmāṣṭāśatam}.
}}muttamam}{\lemma{%
	{\rm tataḥ \dots\ uttamam\lem}
}\Bfootnote{%
	To be traced.
}} //\\
\edtext{vajrasattva \edtext{mahāsattvetyādi}{\lemma{%
	{\rm mahāsattvetyādi\lem}
}\Dfootnote{%
	\MSK\ \sil;
	\textit{mahāsattvetyādi} \cod\pc;
	\textit{mahāsatvetyidi} \cod\ac
}}}{\lemma{%
	{\rm The \textit{nāmāṣṭaśata} beginning with \textit{vajrasattva}\lem}
}\Bfootnote{%
	STTS §§197–201.
}} /}
	%-gandhodake: em. -gandhodakena, with MSK.
	%vajrānalena mudreṇa: em. vajrānalena or vajrānalena mantreṇa? MSK emends vajrānalamudrayā.
	%vajrapuṣpa: em. vajrapuṣpe, with MSK.
	%puṣpā\unclear{ṇ}i : puṣpāni MSK. puṣpā\unclear{ṇ}i \cod
	%gandhat·: em. gandham·, with MSK.
	%°ekaviśaṁti-: em. °ekaṁviśati-, with MSK (sil. em.).
	%khaḍgātkarṣābhinayā: em. khaḍgākarṣābhinayā; MSK accepts the text as it is.
	%dvārodghaṭānaṁ: em. dvārodghāṭanaṁ, with MSK (sil. em.).
	%śrīvajrasatvāmahāmudrām: em. śrīvajrasattvamahāmudrām, with MSK.
	%vajravandhe: thus pc; vajrena ac.
	%apīti: MSK note indicates that they read ayīti.
	%-mudrāhṛdayat·: em. mudrāhṛdayam·, with MSK.
	%saparṣacakra-: em. saparṣaccakra-, with MSK.
	%śrīghraṁ: em. śīghraṁ, with MSK (sil. em.).
	%°uccāraye: em. °uccārayet, with MSK? Or keep as Ārṣa form?
	%nāmāṣṭhagatam: em. nāmāṣṭaśatam, with MSK; MSK's note reports the misreading nāmāṣṭāśatam.
	%mahāsatvetyādi: tyā is pc, corrected from tyi.
	%the verses are found in SDPT p. 172.
\pend

\verse
\Skt{\edtext{tato dvāreṣu sarveṣu \edtext{karma}{\lemma{%
	{\rm karma\lem}
}\Dfootnote{%
	\emn\ \MSK\ \sil;
	\textit{karmmaṁ} \cod
}} kṛtvāṅkuśādibhiḥ /\\ 
\edtext{mahākarmāgyramudrābhiḥ}{\lemma{%
	{\rm mahākarmāgyramudrābhiḥ\lem}
}\Dfootnote{%
	\emn\ \MSK; %\textit{mahākarmāgyramudrābhiḥ}%, with SDPT and MSK?
	\textit{mahākarmmābhramudrābhiḥ} \cod
}} \edtext{samayāṃstu}{\lemma{%
	{\rm samayāṁs\lem}
}\Dfootnote{%
	\emn\ \MSK;
	\textit{samayās} \cod;
	\MSK\'s note reports the misreading \textit{samayas}.
}} niveśayet //\\
%
mudrābhiḥ \edtext{samayāgryābhiḥ}{\lemma{%
	{\rm samayāgryābhiḥ\lem}
}\Dfootnote{%
	\emn\ \MSK;
	\textit{samāgryābhiḥ} \cod
}} sattvavajrādibhistathā /\\ 
\edtext{sādhayeta mahāsattvān}{\lemma{%
	{\rm sādhayeta mahāsattvān\lem}
}\Dfootnote{%
	\emn\ \MSK;
	\textit{sādhayet mahāsatvāṁ} \cod
}} \edtext{\textcolor{red}{jaḥ}}{\lemma{%
	{\rm jaḥ\lem}
}\Dfootnote{%
	\emn\ \MSK\ \sil;
	\textit{jāḥ} \cod
}} hū{\cb} va{\cb} hoḥ pravartayet}{\lemma{%
	{\rm tato \dots\ pravartayet\lem}
}\Bfootnote{%
	STTS §209.22–23. STTS (\Hed) reads \textit{pravartayan} for \textit{pravartayet}.
%	\textit{tato dvāreṣu sarveṣu karma kṛtvāṅkuśādibhiḥ |
%	mahākarmāgryamudrābhiḥ samayāṁs tu niveśayet ||
%	mudrābhiḥ samayāgryābhiḥ sattvavajrādibhis tathā |
%	sādhyeta mahāsattvāñ jaḥ hūṁ vaṁ hoḥ pravartayan ||}
}} //}
	%karmmaṁ: em. karma, with MSK (sil. em.).
	%mahākarmmābhramudrābhiḥ: em. mahākarmāgyramudrābhiḥ, with SDPT and MSK?
	%samayās: em. samayāṁs, with MSK; MSK in note reports misreading samayas.
	%samāgryābhiḥ: em. samayāgryābhiḥ, with MSK.
	%sādhayet· mahāsatvāṁ: em. sādhayeta mahāsattvān, with MSK.
	%jāḥ: em. jaḥ, with MSK (sil. em.).
	%STTS §209.22–23.
\pend

\pstart
\Skt{etaduktaṃ \edtext{bhavati}{\lemma{%
	{\rm bhavati\lem}
}\Dfootnote{%
	\emn\ \MSK;
	\textit{bhavanti} \cod
}} / \edtext{vajrayakṣeṇa vighnotsāraṇaṃ}{\lemma{}\Efootnote{%
	vajrayakṣeṇa [26r1] vighnotsāraṇaṁ
}} \edtext{\textcolor{red}{rakṣāṃśca}}{\lemma{%
	{\rm rakṣāṁś\lem}
}\Dfootnote{%
	\emn;
	\textit{rakṣāṃś} \cod\ \MSK.
	Cf. SVU §140 \textit{... tathaiva vajrāṅkuśādibhiḥ ākṛṣya praveśya baddhvā vaśīkṛtya vajrayakṣeṇa vighṇotsāraṇaṃ prākāraṃ pañjaraṃ kṛtvā samayavajramuṣṭinā maṇḍaladvārāṇi baddhvā dvayakṣarakavacena sarvarakṣāḥ saṃrakṣyārghadānapūrvikābhiḥ svasamayamudrābhir dṛśyaṃ kṛtvā}; Bhūtaḍāmara manual:  \textit{... śāntikaṃ homaṃ kṛtvā duṣṭotsāraṇamantreṇa duṣṭān utsārya pūrvavad digbandhādibhiḥ sthānarakṣāṃ kṛtvā ...}
}} kṛtvā / \edtext{vajramuṣṭinā}{\lemma{%
	{\rm vajramuṣṭinā\lem}
}\Dfootnote{%
	\emn\ \MSK\ \sil;
	\textit{vajramuṣṭhinā} \cod
}} dvārabandham / vajrasattvenārghaṃ dattvā vajradhātvādisamayamudrāṃ baddhvā / vajradhātu dṛśyetyādinā \edtext{sarvān dṛśyā}{\lemma{%
	{\rm sarvān dṛśyān\lem}
}\Dfootnote{%
	\emn\ \MSK;
	\textit{savā dṛśyā} \cod;
	\MSK's reports the reading as \textit{sarva x x x}.
%	or should one read sa\supplied{r}vā\supplied{ṁ} dṛśyā\supplied{ṁ}?
}}nkṛtvā jaḥ hū{\cb} va{\cb} hoḥ samayastvam / samayastvamaham / svahṛdayāni \edtext{mantrāṃśca}{\lemma{%
	{\rm mantrāṁś\lem}
}\Dfootnote{%
	\emn;
	\textit{mantrāś} \cod;
	\textit{mantrāñ} \emn\ \MSK\ \sil
}} śrīvairocanādīnāṃ triruccārya / dharmakarmamahāmudrābhi\edtext{\textcolor{red}{ścainānāmudryābhi}}{\lemma{%
	{\rm cainān āmudryābhi\lem}
}\Dfootnote{%
	\emn\ ($\leftarrow$Tib.);
	\textit{ca nānāmudrābhi} \cod\ \MSK;
	\textit{chos dang las dang phyag rgya chen po rnams kyis kyang | de rnams la rgyas btab ste | dbang bskur
ba’i phyag rgya rnams kyis} Tib.
% see f.53r4–5. 
}}ṣeka\edtext{mudrābhi}{\lemma{%
	{\rm -mudrābhir\lem}
}\Dfootnote{%
	\emn \MSK;
	\textit{-mudrābhi} \cod
}}rbuddhādī\edtext{nabhi\-ṣicyārghaṃ}{\lemma{%
	{\rm abhiṣicyārghaṁ\lem}
}\Dfootnote{%
	\emn;
	\textit{abhiśiścārghaṁ} \cod;
	\textit{abhiṣiñcārghaṁ} \MSK 
}} dattvā /
%
%
{\om} sarvatathāgatapuṣpapūjāmeghasamudraspharaṇa\edtext{samaye}{\lemma{%
	{\rm -samaye\lem}
}\Dfootnote{%
	\emn\ \MSK;
	\textit{-samaya} \cod
}} hū{\cb} iti \edtext{puṣpaiḥ}{\lemma{%
	{\rm puṣpaiḥ\lem}
}\Dfootnote{%
	\emn\ \MSK\ \sil;
	\textit{puṣaiḥ} \cod
}} / 
{\om} sarvatathāgatagandhapūjāmeghasamudraspharaṇasamaye hū{\cb} iti gandhaiḥ /
{\om} sarvatathāgatadhūpapūjāmeghasamudraspharaṇasamaye \edtext{hū{\cb} iti}{\lemma{%
	{\rm hū;ṁ iti\lem}
}\Dfootnote{%
	\emn
	\textit{hū;ṁm iti} \cod
	\textit{hūṁ || iti} \emn\ \MSK\ \sil
}} dhūpaiḥ /
\edtext{\textcolor{red}{{\om}}}{\lemma{%
	{\rm oṁ\lem}
}\Dfootnote{%
	\emn\ \MSK;
	n.e. \cod
}} akāro mukhaṃ sarvadharmānāmityādinā balipūjayā / 
{\om} sarvatathāgata\edtext{dīpapūjā}{\lemma{%
	{\rm -dīpapūjā-\lem}
}\Dfootnote{%
	\emn\ \MSK;
	\textit{-dīpaṁpūjā-} \cod
}}meghasamudraspharaṇasamaye \edtext{hū{\cb} iti}{\lemma{%
	{\rm hū;ṁ iti\lem}
}\Dfootnote{%
	\emn
	\textit{hū\textcolor{red}{;ṁ}m iti} \cod;
	\textit{hūṁ || iti} \emn\ \MSK\ \sil
}} dīpaiḥ /}
	%SDPT p. 174
	%-samaya: em. -samaye, with MSK.
	%puṣaiḥ: em. puṣpaiḥ, with MSK (sil. em.).
	%°akāro: em. °oṁ °akāro, with MSK? This mantra is missing in SDPT.
	%-dīpaṁpūjā-: em.-dīpapūjā-, with MSK.
%
\Skt{lāsyādyaṣṭavidhapūjayā ca saṃpūjya /}
\pend

\verse
\textbf{[Vajrasattva]}\\%
\Skt{\edtext{{\om} sarvatathāgatasarvātmaniryātana\edtext{pūjāspharaṇakarmavajrī}{\lemma{%
	{\rm -pūjāspharaṇakarmavajrī\lem}
}\Dfootnote{%
	\emn\ \MSK;
	\textit{-pūjāmeghasamudraspharaṇakarmmavajra} \cod
}} āḥ / \\
%
\ledsidenote{{\rm [Vajrarāja]}}%
{\om} sarvatathāgatasarvātmaniryātanapūjāspharaṇakarmāgri \edtext{\textcolor{red}{jāḥ}}{\lemma{%
	{\rm jāḥ\lem}
}\Cfootnote{%
	\emn\ \textit{jaḥ} or \textit{jjaḥ}.
}} / \\
%
\ledsidenote{{\rm [Vajrarāga]}}%
{\om} sarvatathāgatasarvātmaniryātanānurāgaṇa\edtext{pūjāspharaṇakarmabāṇe}{\lemma{%
	{\rm -pūjāspharaṇakarmavāṇe\lem}
}\Dfootnote{%
	\emn\ \MSK\ \sil;
	\textit{-pūspharaṇakarmmavāṇe} \cod
}\lemma{}\Efootnote{%
	pū\supplied{jā}[26v1]spharaṇakarmmavāṇe
}} hū{\cb} hoḥ~/ \\
%
\ledsidenote{{\rm [Vajrasādhu]}}%
{\om} sarvatathāgatasarvātmaniryātanasādhukārapūjāspharaṇakarmatuṣṭi āḥ~/ \\
%
\ledsidenote{{\rm [Vajraratna]}}%
{\om} namaḥ sarvatathāgatakāyābhiṣekaratnebhyo vajramaṇi {\om} / \\
%
\ledsidenote{{\rm [Vajrateja]}}%
\edtext{{\om} namaḥ sarvatathāgatasūryebhyo}{\lemma{%
	{\rm oṁ namaḥ sarvatathāgatasūryebhyo\lem}
}\Dfootnote{%
	\emn\ \MSK;
	\textit{oṁ sarvatathāgatasūryebhyo} \cod
}} vajratejini jvala hrīḥ / \\
%
\ledsidenote{{\rm [Vajraketu]}}%
{\om} namaḥ sarvatathāgatāśāparipūraṇacintāmaṇidhvajāgrebhyo \edtext{vajradhvajāgri}{\lemma{%
	{\rm vajradhvajāgri\lem}
}\Cfootnote{%
	STTS read \textit{vajradhvajāgre}.
}} \edtext{tra{\cb}}{\lemma{%
	{\rm tram\lem}
}\Cfootnote{%
	\MSK\ emends \textit{trāṁ}, but STTS read \textit{tram}.
}} / \\
%
\ledsidenote{{\rm [Vajrahāsa]}}%
{\om} namaḥ sarvatathāgatamahāprītiprāmodyakarebhyo vajrahāse haḥ / \\
%
\ledsidenote{{\rm [Vajradharma]}}%
\edtext{{\om} sarva}{\lemma{%
	{\rm oṁ sarva-\lem}
}\Dfootnote{%
	\emn\ following STTS
	\textit{oṁ namaḥ sarva-} \cod
}}tathāgatavajra\edtext{dharmasamatā}{\lemma{%
	{\rm -dharmasamatā-\lem}
}\Cfootnote{%
	\emn\ \textit{-vajradharmatā-}, with STTS?
}}\edtext{samādhibhiḥ stunomi}{\lemma{%
	{\rm samādhibhiḥ stunomi\lem}
}\Dfootnote{%
	\emn\ with STTS;
	\textit{samādhibhi stutomi} \cod;
	\MSK\ keeps \textit{stutomi}.
}} mahādharmāgri hrīḥ~/ \\
%
\ledsidenote{{\rm [Vajratīkṣṇa]}}%
{\om} sarvatathāgataprajñāpāramitā\edtext{nirhāraiḥ stunomi}{\lemma{%
	{\rm -nirhāraiḥ stunomi\lem}
}\Dfootnote{%
	\emn\ with STTS;
	\textit{-nirhārai stutomi} \cod;
	\MSK\ keeps \textit{stutomi}.
}} mahāghoṣā\edtext{nuge dha{\cb}}{\lemma{%
	{\rm -nuge dham\lem}
}\Dfootnote{%
	\emn\ \MSK\ (with STTS);
	\textit{-nurodham} \cod
}} / \\
%
\ledsidenote{{\rm [Vajrahetu]}}%
{\om} sarvatathāgatacakrākṣaraparivartādisarvasūtrāntanayaiḥ \edtext{stunomi}{\lemma{%
	{\rm stunomi\lem}
}\Dfootnote{%
	\emn\ with STTS;
	\textit{stutopi} \cod\ (\MSK\ reads \textit{stunoyi});
	\textit{stutomi} \emn\ \MSK
}} sarvamaṇḍale hū{\cb} / \\
%
\ledsidenote{{\rm [Vajrabhāṣa]}}%
{\om} sarvatathāgatasaṃdhābhāṣabuddhasaṅgītibhir \edtext{gāyan stunomi}{\lemma{%
	{\rm gāyan stunomi\lem}
}\Dfootnote{%
	\emn\ with STTS;
	\textit{gāya stutopi} \cod;
	\MSK\ emends \textit{gāyan stutomi}.
}} \edtext{vajravāce}{\lemma{%
	{\rm vajravāce\lem}
}\Dfootnote{%
	\emn\ \MSK\ \sil\ with STTS;
	\textit{vajrāvāce} \cod
}} va{\cb}~/ \\ 
%
\ledsidenote{{\rm [Vajrakarma]}}%
{\om} sarvatathāgatadhūpameghaspharaṇapūjākarme kara kara / \\
%
\ledsidenote{{\rm [Vajrarakṣa]}}%
{\om} sarvatathāgatapuṣpaprasaraspharaṇapūjākarme kiri kiri / \\
%
\ledsidenote{{\rm [Vajrayakṣa]}}%
{\om} sarvatathāgatāloka\edtext{jvālāspharaṇapūjākarme bhara bhara}{\lemma{%
	{\rm -jvālāspharaṇapūjākarme bhara bhara\lem}
}\Dfootnote{%
	\emn\ with STTS;
	\textit{-jvālospharaṇapūjākarmme | kara kara} \cod;
	MSK reads \textit{-jvālā}- (sil. em.) but keeps \textit{kara kara}.
}} / \\
%
\ledsidenote{{\rm [Vajrasaṃdhi]}}%
{\om} sarvatathāgatagandhasamudraspharaṇapūjākarme kuru kuru / }{\lemma{%
	{\rm oṃ \dots\ kuru kuru\lem}
}\Bfootnote{%
	STTS §§506–507, 509–510, 512–513, 515–516:
	\textit{oṁ sarvatathāgatasarvātmaniryātanapūjāspharaṇakarmavajri āḥ.
oṁ sarvatathāgatasarvātmaniryātanapūjāspharaṇakarmāgri jjaḥ.
oṁ sarvatathāgatasarvātmaniryātanānurāgaṇapūjāspharaṇakarmavāṇe hūṁ hoḥ.
oṁ sarvatathāgatasarvātmaniryātanasādhukārapūjāspharaṇakarmatuṣṭi aḥ.
\dots\ 
oṁ namaḥ sarvatathāgatakāyābhiṣekaratnebhyo vajramaṇi oṁ.
oṁ namaḥ sarvatathāgatasūryebhyo vajratejini jvala hrīḥ.
oṁ namaḥ sarvatathāgatāśāparipūraṇacintāmaṇidhvajāgrebhyo vajradhvajāgre traṁ.
oṁ namaḥ sarvatathāgatamahāprītiprāmodyakarebhyo vajrahāse haḥ.
\dots\
oṁ sarvatathāgatavajradharmatāsamādhibhiḥ stunomi mahādharmāgri hrīḥ.
oṁ sarvatathāgataprajñāpāramitānirhāraiḥ stunomi mahāghoṣānuge dhaṁ.
oṁ sarvatathāgatacakrākṣaraparivartādisarvasūtrāntanayaiḥ stunomi sarvamaṇḍale hūṁ.
oṁ sarvatathāgatasaṁdhābhāṣabuddhasaṁgītibhir gāyan stunomi vajravāce vaṁ.
\dots\
oṁ sarvatathāgatadhūpameghaspharaṇapūjākarme kara kara.
oṁ sarvatathāgatapuṣpaprasaraspharaṇapūjākarme kiri kiri.
oṁ sarvatathāgatālokajvālāspharaṇapūjākarme bhara bhara.
oṁ sarvatathāgatagandhasamudraspharaṇapūjākarme kurr kuru.}
}} }
\pend
\pstart\noindent
\Skt{ityābhirvidyābhiḥ %\newfolio{27r1}  
\edtext{ṣoḍaśa}{\lemma{%
	{\rm ṣoḍaśa-\lem}
}\Dfootnote{%
	\emn\ \MSK\ \sil;
	\textit{ṣoḍaṣa-} \cod
}\lemma{}\Efootnote{%
	vidyābhiḥ [27r1] ṣoḍaśa-
}}sattvasadṛśībhiḥ karmamudrāyuktābhiḥ \edtext{pūjāṃ kuryāt}{\lemma{%
	{\rm pūjāṁ kuryāt\lem}
}\Dfootnote{%
	\emn\ \MSK;
	\textit{pūjā kuryāt} \cod
}} //}
	%STTS §506-516.
	%-pūjāmeghasamudraspharaṇakarmmavajra: em. -pūjāspharaṇakarmavajrī, with MSK.
	%jāḥ: em. jaḥ or jjaḥ.
	%-pūspharaṇakarmmavāṇe: em. -pūjāspharaṇakarmavāṇe, with MSK (sil. em.).
	%°o;ṁ sarvatathāgatasūryebhyo: em. °o;ṁ namaḥ sarvatathāgatasūryebhyo, with MSK. 
	%vajradhvajāgri: STTS read vajradhvajāgre.
	%tram·: MSK emends trāṁ, but STTS read tram.
	%°o;ṁ namaḥ sarva-: STTS shows no namaḥ here, and MSK omits it too.
	%-vajradharmmasamatā-: em. -vajradharmatā-, with STTS?
	%-samādhibhi stutomi:  em. -samādhibhiḥ stunomi, with STTS. MSK keeps stutomi.
	%-nirhārai stutomi: em. nirhāraiḥ stunomi, with STTS. MSK keeps stutomi.
	%-nurodham·: em. -nuge dham·, with MSK and STTS.
	%stutopi: stutoyi MSK, with em. stutomi; em. stunomi, with STTS.
	%gāya stutopi: em. gāyan stunomi, with STTS. MSK emends gāyan stutomi.
	%vajrāvāce: em. vajravāce, with STTS and MSK (sil. em.).
	%-jvālospharaṇapūjākarmme | kara kara |: em. -jvālāspharaṇapūjākarme bhara bhara |, with STTS. MSK reads -jvālā- (sil. em.) but keeps kara kara.
	%ṣoḍaṣa-: em. ṣoḍaśa-, with MSK (sil. em.).
	%pūjā kuryāt·: em. pūjāṁ kuryāt·, with MSK.

\Skt{tatremā mudrā bhavanti / kāyasthaṃ vajrabandhaṃ saṃpīḍya dvidhīkṛtya muṣṭidvayena }
\pend

\verse
\Skt{\edtext{\edtext{\textcolor{red}{sagarvotkarṣaṇād} dvābhyā}{\lemma{%
	{\rm sagarvot\-karṣaṇād dvābhyām\lem}
}\Dfootnote{%
	\emn\ \MSK;
	\textit{sagarvvotkarṣa\unclear{ṇādvā}bhyām} \cod;
	The relevant \textit{akṣara} is damaged in the ms., but our impression is that the scribe wrote \textit{-ṇādvā-} rather than \textit{-ṇāddvā-}.
}}maṅkuśagrahasaṃsthitā /\\ 
\edtext{bāṇa}{\lemma{%
	{\rm bāṇa-\lem}
}\Dfootnote{%
	\emn;
	\textit{vāma-} \cod;
	MSK reads and retains \textit{vāna-}.
}}\edtext{ghaṭṭana}{\lemma{%
	{\rm -ghaṭṭana-\lem}
}\Dfootnote{%
	\emn\ \MSK\ \sil;
	\textit{-ghattana-} \cod
}}yogā ca sādhukārā  hṛdi sthitā //\\
abhiṣeke dvivajraṃ tu hṛdi sūrya\edtext{pradarśanam}{\lemma{%
	{\rm -pradarśanam\lem}
}\Dfootnote{%
	\emn\ \MSK;
	\textit{-padarśanam} \cod
}} /\\ 
vāmasthabāhudaṇḍā ca tathāsye \edtext{parivartitā}{\lemma{%
	{\rm parivartitā\lem}
}\Dfootnote{%
	\emn\ \MSK;
	\textit{parivartitā} \cod
}} //\\ 
savyāpasavyavikacā hṛdvāmakhaḍga\textcolor{red}{māraṇī} /\\ 
alātacakrabhramitā vajradvayamukhotthitā //\\
vajranṛtya\edtext{bhra\textcolor{red}{monmukta}}{\lemma{%
	{\rm -bhramonmukta-\lem}
}\Dfootnote{%
	\emn\ \MSK;
	\textit{-bhramomukta-} \cod
}}\edtext{\textcolor{red}{kapolo}ṣṇīṣa}{\lemma{%
	{\rm -kapoloṣṇīṣa-\lem}
}\Dfootnote{%
	\emn\ \MSK;
	\textit{-kapoloṣṇiṁṣa-} \cod;
	\MSK\ reports as reading \textit{-kapoloṣṇiṣa-}.
}}saṃsthitā /\\ 
kavacā kaniṣṭha\edtext{daṃṣṭrāgrā}{\lemma{%
	{\rm -daṁṣṭrāgrā\lem}
}\Dfootnote{%
	\emn\ \MSK;
	\textit{-draṁṣṭrāgrā} \cod
}} \edtext{muṣṭi}{\lemma{%
	{\rm muṣṭi-\lem}
}\Dfootnote{%
	\emn\ \MSK\ \sil;
	\textit{-muṣṭhi-} \cod
}}dvayanipīḍiteti}{\lemma{%
	{\rm sagarvo-\ \dots\ -nipīḍiteti\lem}
}\Bfootnote{%
	STTS §287–288:
	\textit{sagarvotkarṣaṇaṁ dvābhyām aṅkuśagrahasaṁsthitā |
vāṇaghaṭṭanayogāc ca sādhukārā hṛdi sthitā ||
abhiṣeke dvivajraṁ tu hṛdi sūryapradarśanam |
vāmasthabāhudaṇḍā ca tathāsye parivartitā ||
savyāpasavyavikacā hṛdvāmā khaḍgadhāraṇā |
alātacakrabhramitā vajradvayamukhotthitā ||
vajranṛtyabhramonmuktakapoloṣṇīṣasaṁsthitā |
kavacā kaniṣṭhadaṁṣṭrā(gryā) muṣṭidvayanipīḍitā ||}
}} //}
	%verses from STTS §287-288
	% sagarvvotkarṣa\unclear{ṇādvā}bhyām:  sagarvotkarṣaṇād dvābhyām MSK; em. with MSK. The relevant akṣara is damaged in the ms., but our impression is that the scribe wrote -ṇādvā- rather than -ṇāddvā-.
	%vāma-: em. bāṇa-; MSK reads and retains vāna-.
	%-ghattana-: em. -ghaṭṭana-, with MSK (sil. em.).
	%-padarśanam·: em. -pradarśanam·, with MSK .
	%pavarttitā: em. parivartitā, with MSK.
	%-bhramomukta-: em. -bhramonmukta-, with MSK.
	%-kapoloṣṇiṁṣa-: em. -kapoloṣṇīṣa-, with MSK; MSK reports as reading -kapoloṣṇiṣa-.
	%-draṁṣṭṛāgrā: em. -daṁṣṭṛāgrā, with MSK
	%muṣṭhi-: em. muṣṭi-, with MSK (sil. em.).
\pend

\pstart
\Skt{tataḥ catuḥpraṇāmaṃ kṛtvā śrīvajrasattvamahāmudrā\edtext{vyavasthitaḥ}{\lemma{%
	{\rm -vyavasthitaḥ\lem}
}\Dfootnote{%
	\emn\ \MSK;
	\textit{-vyavasthitāḥ} \cod
}} sarvatathāgatakāya\edtext{vā}{\lemma{%
	{\rm -vāk-\lem}
}\Dfootnote{%
	\emn\ \MSK;
	\textit{-kāk-} \cod
}}kci\-ttavajramātmānaṃ bhāvayedanena / {\om} vajrātmako'hamiti / tataḥ svabhāvaśuddhaṃ / {\om} svabhāvaśuddho'hamiti / nairātmyasamatayā ca~/ vairocanādisarvadevatāsvabhāvam / {\om} sarvasamo'hamiti //}
	%MSK refers to Amoghavajra as source for these mantras.
	%-vyavasthitāḥ: em. -vyavasthitaḥ, with MSK.
	%-kāk-: em. -vāk-, with MSK.

\Skt{tataḥ śrīvajrasattva\edtext{śatākṣaraṃ}{\lemma{%
	{\rm -śatākṣaraṁ\lem}
}\Dfootnote{%
	\emn\ \MSK;
	\textit{-ṣaḍākṣaraṁ} \cod
}} vajravācodīrayan \edtext{manasā vā}{\lemma{%
	{\rm manasā vā\lem}
}\Dfootnote{%
	\emn\ \MSK;
	\textit{manavā} \cod
}} sarvamevāhamiti bhāvayet / sarvadevatāmukhebhyaśca mantradhvanirabhiraṇatītyadhyavasāyaḥ kāryaḥ / evaṃ \edtext{sarve}{\lemma{%
	{\rm sarve\lem}
}\Dfootnote{%
	\cod;
	\textit{sarvair} \MSK
}} japtā bhavanti / tāva\edtext{dbhāvaye}{\lemma{}\Efootnote{%
	bhāvaye[27v1]d 
}}\edtext{dyāvatkhedo}{\lemma{%
	{\rm yāvat khedo\lem}
}\Dfootnote{%
	\emn;
	\textit{yāvad khedo} \cod 
	(thus also \MSK; the \textit{sandhi} is impossible.)
}} na jāyate /
khede sati \edtext{puna}{\lemma{%
	{\rm punar\lem}
}\Dfootnote{%
	\emn\ \MSK\ \sil;
	\textit{puna} \cod
}}rnāmāṣṭaśata\edtext{stuti}{\lemma{%
	{\rm -stutim\lem}
}\Dfootnote{%
	\emn\ \MSK\ \sil\ \textcolor{red}{CHECK!!};
	\textit{stutiṁm} \cod
}}marghaṃ dattvā pūjāṃ catuḥpraṇāmaṃ ca kṛtvā yatoyataḥ samutpannā mudrāstāstatratatraiva muñcet / sattvavajrādimudrāmokṣam / \edtext{\mbox{}\edtext{vajra}{\lemma{%
	{\rm vajra\lem}
}\Dfootnote{%
	\emn\ with STTS §309;
	\textit{vajrasatva} \cod;
	\MSK\ emends \textit{vajrasattva}.
}} muriti}{\lemma{%
	{\rm vajra mur iti\lem}
}\Bfootnote{%
	STTS §309: \textit{atha sarvamudrāṇāṁ sāmānyo mokṣavidhivistaro bhavati. tatrādita eva yato yataḥ samutpannā mudrā tāṁ tatra tatraiva muñced anena hṛdayena, vajra muḥ}.
}} /}
	%-ṣaḍākṣaraṁ: em. -śatākṣaraṁ, with MSK.
	%manavā: em. manasā vā, with MSK.
	%sarve: sarvair MSK.
	%yāvad khedo: thus also MSK; the sandhi is impossible; em. yāvat khedo.
	%puna: em. punar, with MSK (sil. em.).
	%vajrasatva: em. vajra, with STTS §309; MSK emends vajrasattva.
% \pend
%
% \pstart
\Skt{śrīvajrasattvādīnāṃ sthānaniyama uktaḥ / akṣobhyādīnāṃ vajrasattvavajraratnavajradharmavajrakarmasthānameva \edtext{sthānam}{\lemma{%
	{\rm -sthānam\lem}
}\Dfootnote{%
	\emn\ \MSK
	\textit{sthānām} \cod
}} / \edtext{vairocanasyoṣṇīṣaḥ}{\lemma{%
	{\rm vairocanasyoṣṇīṣaḥ\lem}
}\Dfootnote{%
	\textit{vairocanasyoṣṇīṣa-} \emn\ \MSK\ \sil;
	\textit{vairocanasyoṣṇīṁṣa-} \cod
}} sthānam //}
	%sthānām·: em. sthānam, with MSK?
	%vairocanasyoṣṇīṁṣa-: em. vairocanasyoṣṇīṣa-, with MSK (sil. em.).
\pend

\pstart
\Skt{\edtext{tato vajraratnasamayamudrayā hṛdayotthitayā svābhiṣekasthānasthitayā sarvamudrābhiṣekaṃ kuryāt/ {\om} vajraratnābhiṣiñceti/ tadanu \edtext{pūrvava}{\lemma{%
	{\rm pūrvavat\lem}
}\Dfootnote{%
	\emn\ \MSK;
	\textit{pūrvat} \cod
}}tkavacabandhaṃ kuryāttarjanībhyām / \edtext{{\om}}{\lemma{
	{\rm oṁ\lem}
}\Dfootnote{%
	\emn;
	n.e. \cod\ \MSK
}} sarvamudrāṃ me dṛḍhīkuru varakavacena \edtext{vamiti}{\lemma{%
	{\rm vam iti\lem}
}\Dfootnote{%
	\emn\ \MSK\ (whose note "Ms om. vam" is misleading, because the ms. does not omit the m);
	\textit{miti} \cod
}}}{\lemma{%
	{\rm tato \dots\ vam iti\lem}
}\Bfootnote{%
	STTS 310: \textit{tato hṛdayotthitayā ratnavajrimudrayā svābhiṣekasthānasthitayābhiṣcyāgrāṅgulibhyāṁ mālāṁ veṣṭayan baddhvā, tathaiva kavacaṁ bandhayed anena hṛdayena, oṁ ratnavajrābhiṣiñca sarvamudrāṁ me dṛḍhīkuru varakavacena vaṁ}.
}}/ ante samatālayā toṣayet \edtext{pūrvavat}{\lemma{%
	{\rm pūrvavat\lem}
}\Dfootnote{%
	\emn\ \MSK;
	\textit{pūrvat} \cod
}}/ \edtext{vajrasattvaśatākṣaraṃ}{\lemma{%
	{\rm vajrasattvaśatākṣaraṁ\lem}
}\Dfootnote{%
	\emn;
	\textit{vajrasatvaṁ śatākṣaraṁ} \cod;
	\textit{vajrasattvaṁ śatākṣaraṁ} \MSK
}} \edtext{coccāryābhi}{\lemma{%
	{\rm coccāryābhi-\lem}
}\Dfootnote{%
	\emn;
	\textit{coccāpyābhi-} \cod;
	\MSK\ emends ca \textit{japyābhi-}.
}}pretasiddhaye kuśalaṃ \edtext{\textcolor{red}{pariṇāmayya}}{\lemma{%
	{\rm pariṇamayya\lem}
}\Dfootnote{%
	\emn;
	\textit{pariṇamayya} \emn\ \MSK\ \sil;
	\textit{parinamayya} \cod
}}  yanmayā \edtext{vidhinyūnaṃ}{\lemma{%
	{\rm vidhinyūnaṁ\lem}
}\Dfootnote{%
	\emn\ \MSK\ \sil;
	\textit{viddhinyūnaṁ} \cod
}} kṛtaṃ tat \edtext{kṣāntumarhatheti}{\lemma{%
	{\rm kṣāntum arhatheti\lem}
}\Dfootnote{%
	\emn\ \MSK\ \sil;
	\textit{kṣaṁtum arhartheti} \cod
}} \edtext{vijñāpyārghaṃ}{\lemma{%
	{\rm vijñāpyārghaṁ\lem}
}\Dfootnote{%
	\emn\ \MSK\ \sil;
	\textit{vijñānyārghañ} \cod
}} ca dattvā gamanāya saṃcodayet sarvabuddhabodhi\edtext{sattvān}{\lemma{%
	{\rm -sattvān\lem}
}\Dfootnote{%
	\emn\ \MSK;
	\textit{-satvat} \cod
}}/}
\pend

\verse
\Skt{\edtext{{\om} kṛto vaḥ sarvasattvārthaḥ siddhiṃ dattvā yathānugā /\\
gacchadhvaṃ buddhaviṣayaṃ punarāgamanāya tviti}{\lemma{%
	{\rm oṃ \dots\ punarāgamanāya tv\lem}
}\Bfootnote{%
	STTS 317:
	\textit{oṁ kṛto vaḥ sarvasattvārthaḥ siddhir dattā yathānugā |
	gacchadhvaṁ buddhaviṣayaṁ punar āgamanāya tu ||}
}} //} 
\pend

\pstart\noindent
\Skt{sattvavajrīṃ \edtext{cordhvato}{\lemma{%
	{\rm cordhvato\lem}
}\Dfootnote{%
	\emn\ \MSK\ \sil;
	\textit{corddhato} \cod
}} muñcedanena \edtext{hṛdayena}{\lemma{%
	{\rm hṛdayena\lem}
}\Dfootnote{%
	\emn\ \MSK;
	\textit{hṛdayedanena} \cod
}} / \edtext{vajrasattva muḥ}{\lemma{%
	{\rm vajrasattva muḥ\lem}
}\Bfootnote{%
	STTS 317: \textit{vajrasattva muḥ}.
}} iti / evaṃ sarve \edtext{visarjitā bhavanti}{\lemma{}\Efootnote{%
	visarjitā [28r1] bhavanti
}} //}
	%STTS §310, §317.
	%pūrvat·: em. pūrvavat·, with MSK.
	%va§○rakavacena: em. vajrakavacena, with MSK (sil. em.).
	%miti: em. vam iti, with MSK (whose note "Ms om. vam" is misleading, because the ms. does not  omit the m).
	%pūrvat·: em. pūrvavat·, with MSK (sil. em.).
	%coccāpyābhi-: em. coccāryābhi-; MSK emends ca japyābhi-.
	%parinamayya: em. pariṇamayya, with MSK (sil. em.).
	%viddhinyūnaṁ: em. vidhinyūnaṁ, with MSK (sil. em.).
	%kṣaṁtu○m arhartheti: em. kṣāntum arhatheti, with MSK (sil. em.).
	%vijñānyārghañ: em. vijñāpyārghaṁ, with MSK (sil. em.).
	%-satvat·: em. -sattvān, with MSK.
	%siddhin datvā: STTS has siddhir dattā. Emend thus?
	%corddhato: em. cordhvato, with MSK (sil. em.).
	%hṛdayedanena: em. hṛdayena, with MSK.

\Skt{yathā nirmitaṃ ca \edtext{saparṣaccakra}{\lemma{%
	{\rm saparṣaccakra-\lem}
}\Dfootnote{%
	\emn\ \MSK\ \sil;
	\textit{saparṣacakra-} \cod
}}saṃcayaṃ tathaiva manasā sādhuyogena svakāye praveśya vajrarakṣayakṣasaṃdhikarmamudrābhirātmānaṃ \edtext{saṃrakṣyotthāya}{\lemma{%
	{\rm saṁrakṣyotthāya\lem}
}\Dfootnote{%
	\emn\ \MSK;
	\textit{saṁrakṣotthāya} \cod
}} \edtext{\textcolor{red}{maṇḍalakarma pustakavācikādi}}{\lemma{%
	{\rm maṇḍalakarma pustakavācikādi\lem}
}\Dfootnote{%
	\emn;
	\textit{maṇḍalakapustakavācikādīṁ} \cod;
	\textit{maṇḍalakarmapustakavācikādīṁ} \emn\ \MSK;
	or \textcolor{red}{\textit{maṇḍalakoṇa-} !!}
}} kuryāt / evaṃ pratyahaṃ \edtext{catuḥsaṃdhyaṃ}{\lemma{%
	{\rm catuḥsaṁdhyaṁ\lem}
}\Dfootnote{%
	\emn\ (see end of next paragraph);
	\textit{catusadhyaṁ} \cod;
	\MSK\ emends \textit{catuṣṭayaṁ}.
}} \edtext{kuryānmāsaṃ}{\lemma{%
	{\rm kuryān māsaṃ\lem}
}\Dfootnote{%
	\emn;
	\textit{kuryātmāsaṃ} \cod
	\textit{kuryād māsaṃ} \emn\ \MSK\ \sil
}} ṣaḍmāsaṃ \edtext{saṃvatsaraṃ}{\lemma{%
	{\rm saṃvatsaraṃ\lem}
}\Dfootnote{%
	\emn\ \MSK;
	\textit{saṃvaccharaṁ} \cod
}} yāvatā vā kālena devatābhiranujñāto bhavati / athavā \edtext{pūrvava}{\lemma{%
	{\rm pūrvavan\lem}
}\Dfootnote{%
	\emn;
	\textit{pūrvan} \cod;
	\MSK\ silently emends \textit{pūrvavat}.
}}nmahāyogaṃ kṛtvā pūjādikaṃ ca / }
\pend

\verse
\Skt{\edtext{\edtext{vajraṃ tattvena}{\lemma{%
	{\rm vajraṁ tattvena\lem}
}\Dfootnote{%
	\emn\ \MSK;
	\textit{vajra tatvena} \cod
}} saṃgṛhya \edtext{ghaṇṭāṃ}{\lemma{%
	{\rm ghaṇṭāṁ\lem}
}\Dfootnote{%
	\emn\ \MSK\ \sil;
	\textit{ghaṭāṁ} \cod
}} dharmeṇa vādya ca/\\ 
samayena mahāmudrāmadhiṣṭhāya hṛdayaṃ \edtext{japet}{\lemma{%
	{\rm japet\lem}
}\Dfootnote{%
	\emn\ \MSK;
	\textit{jopet} \cod
}}}{\lemma{%
	{\rm vajraṁ \dots\ japet\lem}
}\Bfootnote{%
	\textit{Vajrāvalī} mentions the source of the verse: \textit{tad uktaṁ paramādyatantre \textemdash\ vajraṃ tattvana saṃgrāhya ghaṇṭāṃ dharmeṇa vādya ca | samayena mahāmudrāṃ adhiṣṭhāya hṛd ājapet ||} (Mori's edition 29.3).
	Paramādyamantrakalpakhaṇḍa (Ota.\ No.\ 120, Toh.\ 488):
	\textit{de nyid kyis ni rdo rje gzung || chos kyi dril bu dkrol bar bya ||
	dam tshig gis ni phyag rgya cher || byin gyis brlabs nas snying po bzlas ||}
	(P vol.5 156.1.5–6 \textcolor{red}{folio number!! D LOCATION!!} ;
	\textcolor{red}{Taisho No.244, vol.8, 812c TEXT!!}
	See note 29.3 *2 (Mori's edition, vol.2, pp.607–608)
	\textcolor{red}{mention Hevajrasekaprakriyā}
	\textcolor{red}{CROSS REFERENCE}!!
	The same verse is quoted again below. See p.\pageref{paramadyaverse1-2}.
}} //}\label{paramadyaverse1}
	%saparṣacakra-: em. saparṣaccakra-, with MSK (sil. em.).
	%saṁrakṣotthāya: em. saṁrakṣyotthāya, with MSK.
	%maṇḍalakapustaka-: MSK emends maṇḍalakarmapustaka-, but this is perhaps not necessary.
	%catusadhyaṁ: em. catuḥsaṁdhyaṁ (see end of next paragraph); MSK em. catuṣṭayaṁ.
	%kuryāt: em. kuryād, with MSK (sil. em.).
	%pūrvan: em. pūrvavan; MSK silently emends pūrvavat.
	%vajra tatvena: em. vajraṁ tattvena, with MSK.
	%gha§ṭāṁ: em. ghaṇṭāṁ.\
	%jopet: em. japet, with MSK
	%Vajrāvalī mentions the source of the verse: tad uktaṁ paramādyatantre (Mori's edition 29.3).
\pend

\pstart
\Skt{tato vajrātmakādi\edtext{mantratraya}{\lemma{%
	{\rm -mantratrayam\lem}
}\Dfootnote{%
	\textit{-mantraṁ traya} \cod\ \MSK
}}mudīrayaṃstadarthaṃ trayaṃ vibhāvya svahṛdi \edtext{caturmudrā}{\lemma{%
	{\rm caturmudrā-\lem}
}\Dfootnote{%
	\emn\ \MSK;
	\textit{catumudrā-} \cod
}}\edtext{maṇḍalo\-ktahṛdaya}{\lemma{%
	{\rm -maṇḍaloktahṛdaya-\lem}
}\Dfootnote{%
	\cod;
	\textit{-maṇḍale rakṣahṛdaya-} \MSK
}}pañcakena vajradhātumahāmaṇḍalaṃ nirmāya \edtext{pūrvoktai}{\lemma{%
	{\rm pūrvoktair\lem}
}\Dfootnote{%
	\emn\ \MSK;
	{\rm pūrvvoktai\lem}
}}rhṛdayaiḥ śrīvairocanādīn yathāsthāne niveśyākāśasamatāyoge\textcolor{red}{ṇa} praviśya \edtext{taiḥ saha}{\lemma{%
	{\rm taiḥ saha\lem}
}\Dfootnote{%
	\emn;
	\textit{tair saha} \cod;
	\MSK\ accepts wrong \textit{sandhi} without note.
}} vajrātmakādimantrairekībhūya vajradhāturahaṃ svayamiti vajradhātumahāmaṇḍalamātmānaṃ vicintya sarvamevāhamiti \edtext{bhāvayan}{\lemma{%
	{\rm bhāvayan\lem}
}\Dfootnote{%
	\emn\ \MSK\ \sil;
	\textit{bhāvayam} \cod
}} śrīvajrasattva\edtext{}{\lemma{%
	{\rm -hṛdayamantraṃ\lem}
}\Dfootnote{%
	\MSK\ \emn\ \sil;
	\textit{-hṛdayaṃ mantraṁ} \cod
}} japet / sarvadevatāmukhebhyaśca mantradhvanirabhiraṇatītyatrāpi cintanīyam / \edtext{evamāśveva}{\lemma{%
	{\rm evam āśv eva\lem}
}\Dfootnote{%
	\emn;
	\textit{evam āśceva} \cod;
	\MSK\ reads \textit{evam ādyeva} and emends \textit{evamādinaiva}.
}} sarve vairocanādayaḥ siddhā bhavanti / \edtext{śrāntaḥśrāntaśca}{\lemma{%
	{\rm śrāntaḥśrāntaś ca\lem}
}\Cfootnote{%
%
\textit{Manusmṛti}:
\textit{ārabhetaiva karmāṇi \textbf{śrāntaḥ śrāntaḥ} punaḥ punaḥ |
karmāṇy ārabhamāṇaṃ hi puruṣaṃ śrīr niṣevate} || 9.300 ||
%
}\lemma{}\Efootnote{%
	śrāntaḥśrānta[28v1]ś ca
}} \edtext{nāmāṣṭaśatastutiṃ}{\lemma{%
	{\rm nāmāṣṭaśatastutiṁ\lem}
}\Dfootnote{%
	\emn\ \MSK;
	\textit{nāmāṣṭaśatutim} \cod
}} pūjāṃ catuḥpraṇāmaṃ ca kuryāt / \edtext{catuḥsaṃdhyāvasāne}{\lemma{%
	{\rm catuḥsaṁdhyāvasāne\lem}
}\Dfootnote{%
	\emn\ \MSK\ \sil;
	\textit{catursaṁdhyāvasāne} \cod
}} ca nāmāṣṭaśatastutipūrvakaṃ vidhiṃ kṛtvā visarjayediti //}
	%catumudrā-: em. caturmudrā-, with MSK.
	%-maṇḍaloktahṛdaya-: -maṇḍale rakṣahṛdaya- MSK.
	%pūrvvoktai: em. pūrvoktair, with MSK.
	%niveśyākāśasa-: niveśyāśāsa- MSK.
	%-yogeṇa: em. -yogena, with MSK (sil. em.).
	%tair saha: em. taiḥ saha. MSK accepts wrong sandhi without note.
	%bhāvayam·: em. bhāvayan·, with MSK (sil. em.).
	%-hṛdayaṁ mantraṁ: em. -hṛdayamantraṁ, with MSK (sil. em.).
	%evam ā*śceva: em. evam āśv eva; MSK reads evam ādyeva and emends evamādinaiva.
	%nāmāṣṭaśatutim·: em. nāmāṣṭaśatastutiṁ, with MSK.
	%catursaṁdhyāvasāne: em. catuḥsaṁdhyāvasāne, with MSK (sil. em.).
\pend

\pstart
\Skt{\edtext{sarvalaukikalokottarāśca}{\lemma{%
	{\rm sarvalaukikalokottarāś\lem}
}\Dfootnote{%
	\emn;
	\textit{sarvalokikalokotarāś} \cod;
	\MSK\ silently emends \textit{sarvalokikalokottarāś}.
}} mantrā mahā\edtext{yogena}{\lemma{%
	{\rm -yogena\lem}
}\Dfootnote{%
	\emn\ \MSK\ \sil;
	\textit{-yogeṇa} \cod
}} \edtext{sādhyāḥ}{\lemma{%
	{\rm sādhyāḥ\lem}
}\Dfootnote{%
	\emn\ \MSK;
	\textit{sādhyā} \cod
}}/ yastu maṇḍalaṃ bhāvayitu\edtext{maśaktaḥ}{\lemma{%
	{\rm aśaktaḥ\lem}
}\Dfootnote{%
	\emn\ \MSK;
	\textit{aśakta} \cod
}} sa mahāyogaṃ kṛtvā pratyekaṃ sarvamantrāṇāṃ lakṣajāpaṃ kuryāt / vajrottiṣṭheti lakṣajāpaṃ kṛtvā \edtext{vajrottiṣṭhamu}{\lemma{%
	{\rm vajrottiṣṭhamu-\lem}
}\Dfootnote{%
	\emn\ \MSK\ \sil;
	\textit{vajrotiṣṭhamu-} \cod
}}drāṃ baddhvā \edtext{sakalāṃ rātriṃ}{\lemma{%
	{\rm sakalāṁ rātriṁ\lem}
}\Dfootnote{%
	\emn\ \MSK;
	\textit{sakalotrāttriñ} \cod
}} japet / siddho bhavati //}
	%sarvalokikalokotarāś: em. sarvalaukikalokottarāś; MSK silently emends sarvalokikalokottarāś.
	%-yogeṇa: em. -yogena, with MSK (sil. em.).
	%sādhyā: em. sādhyāḥ, with MSK.
	%aśakta: em. aśaktaḥ, with MSK.
	%vajrotiṣṭhamu-: vajrottiṣṭhamu-, with MSK (sil. em.).
	%sakalotrāttriñ: em. sakalāṁ rātriṁ, with MSK.
\pend	

\pstart
\Skt{tataḥprabhṛti tayā kalaśādikamākāśe \edtext{sthāpayet}{\lemma{%
	{\rm sthāpayet\lem}
}\Dfootnote{%
	\emn\ \MSK;
	\textit{sthāpayeti} \cod
}} / }
\pend

\verse
\Skt{\edtext{\edtext{dvivajrāgryāṅgulī}{\lemma{%
	{\rm dvivajrāgryāṅgulī\lem}
}\Dfootnote{%
	\emn\ following STTS;
	\textit{dvivajrāṅgulī} \cod\ \MSK
}} samya\edtext{ksaṃdhāyottānato}{\lemma{%
	{\rm saṁdhāyo\-ttānato\lem}
}\Dfootnote{%
	\emn\ \MSK;
	\textit{sandhāyottanato} \cod;
	but STTS reads differently.
}} dṛḍham /\\ 
utthāpayen mṛtaṃ \edtext{sarvaṃ}{\lemma{%
	{\rm sarvaṁ\lem}
}\Dfootnote{%
	\emn\ \MSK;
	\textit{sarva}
}} \edtext{vajrottiṣṭheti}{\lemma{%
	{\rm vajrottiṣṭheti\lem}
}\Dfootnote{%
	\emn\ \MSK;
	\textit{vajrotiṣṭheti} \cod
}} saṃjñiteti}{\lemma{%
	{\rm dvivajrāṅgulī \dots\ saṃjñiteti\lem}
}\Bfootnote{%
	STTS 963:
	\textit{dvivajrāgryāṅgulī samyak saṁdhāya susamāhitaḥ |
	utthāpayen mṛtaṁ sarvaṁ vajrottiṣṭheti saṁjñitā ||}
}} // }
\pend

\pstart\noindent
\Skt{vajrāveśa {\ah} iti ca lakṣaṃ parijapya \edtext{vajraghaṇṭāṃ}{\lemma{%
	{\rm vajraghaṇṭāṁ\lem}
}\Dfootnote{%
	\emn\ \MSK;
	\textit{vajraghaṇṭā} \cod
}} ghṛtasaṃpātāhutisahasreṇābhi\edtext{saṃskṛtya}{\lemma{%
	{\rm -saṁskṛta\lem}
}\Dfootnote{%
	\textit{-saṃskṛta} \cod;
	\textit{-saṁskṛtāṁ} \emn\ \MSK
}} vajrāveśasamaya\edtext{mudrayāvaṣṭabhya}{\lemma{%
	{\rm -mudrayāvaṣṭabhya\lem}
}\Dfootnote{%
	\emn\ \MSK\ \sil;
	\textit{mudrayāvaṣṭhabhya} \cod
}} sakalāṃ rātriṃ japet / tayā siddhayā hastasthayā sarvamāveśayatīti //  //}
	%STTS §963.
	%sthāpayeti: em. sthāpayet·, with MSK.
	%sandhāyottanato: em. saṁdhāyottānato, with MSK. But STTS is different.
	%sarva: em. sarvaṁ, with MSK.
	%vajrotiṣṭheti: em. vajrottiṣṭheti, with MSK.
	%vajraghaṇṭā: em. vajraghaṇṭāṁ, with MSK. (There is a confusion in their note 6.)
	%-saṁskṛta: em. -saṁskṛtāṁ, with MSK?
	%-mudrayāvaṣṭhabhya: em. -mudrayāvaṣṭabhya, with MSK (sil. em.).
	

\pend

\bigskip

% 
% \input{04_Bhumi.tex}

%
\pstart\noindent
{\large 3. Maṇḍala rituals}
\pend

\bigskip

\pstart\noindent
\noindent
{\large 3.1. Bhūmisaṁśodhanaparigrahavidhiḥ}
\pend

\bigskip

\pstart
\quad
\edtext{}{\lemma{%
	Section 3.1\lem
}\Bfootnote{%
	parallel to Tattvālokakarī (P LOCATION!!, D f.108v1–109r1)
	% Endo pp.192–193
}}%
\Skt{%
evaṃ kṛtvā \edtext{pūrvasevāṃ}{\lemma{
	{\rm pūrvasevāṁ\lem}
}\Dfootnote{
	\corr\ \MSK\ \sil;  
	\textit{pūrvasevā} \cod
}} maṇḍalamālikhedvihārārāma\edtext{grāma}{\lemma{
	{\rm -grāma-\lem}
}\Dfootnote{
	\corr\ \MSK\ \sil;  
	\textit{-grāmā-} \cod
}}nagarāṇāṃ pūrvottaradigbhāge~/
% grāma] em.; grāmā \cod
\edtext{yatra}{\lemma{
	{\rm yatra\lem}
}\Dfootnote{
	\emn\ \MSK; 
	\textit{yattra} \cod;
	\textit{yantra} \MSK
}} vā \edtext{manaso}{\lemma{
	{\rm manaso\lem}
}\Dfootnote{
	\emn\ \MSK;
	\textit{maneso} \cod;
	\textit{manaiso} \MSK
}}'nukūlaṃ bhavati tatra~/ susamasnigdhasuplava\edtext{supramāṇānūṣare}{\lemma{
	{\rm -supramāṇānūṣare\lem}
}\Dfootnote{
	\emn;
	{\rm -supramāṇābhūṣare} \cod\ \MSK
}%
%\Efootnote{
%	\textit{tshad dang ldan zhing tsho sgo can ma yin pa'i} D}%
\Bfootnote{
	Padmaśrīmitra's \textit{Maṇḍalopāyikā} 2.17 (CHECK!!):
	supramāṇe same snigdhe 
	'nūṣare doṣavarjite |
% ṣavarjite] f.1v9
	pūrvottaraplave ramye 
	bhūbhāge śodhayet punaḥ ||
}} bhūbhāge~/
% yatra] em.; yattra \cod
rājño hastaśataṃ pañcāśaddhastaṃ vā~/
sāmantamahāsāmantānāṃ 
pañcāśa\edtext{tpañcaviṃśatihastaṃ}{\lemma{
	{\rm pañcaviṁśatihastaṁ\lem}
}\Dfootnote{
	\emn\ \MSK; 
	\textit{pañcaviṃśatirhastam} \cod
}} vā~/
% pañcaviṃśatihastaṃ] em.; pañcaviṃśatirhastam \cod
\edtext{śreṣṭhinaḥ}{\lemma{}\Efootnote{
	vā | [29r1] śreṣṭhinaḥ
}} sārthavāhasya vā \edtext{pañcaviṃśatiṃ}{\lemma{
	{\rm pañcaviṃśatiṁ\lem}
}\Dfootnote{
	\emn\ \MSK\ \sil; 
	\textit{pañcaviṁsatis} \cod
}} tadardhaṃ vā /
% pañcaviṃśatiṃ] em.; pañcaviṃsatis \cod
sādhakānāṃ dvādaśahastaṃ \edtext{ṣaḍḍhastaṃ}{\lemma{
	{\rm ṣaḍḍhastaṁ\lem}
}\Dfootnote{
	\emn\ \MSK\ \sil; 
	\textit{śaḍḍhastam} \cod
}} vā~/}
% ṣaḍḍhastaṃ] \emn\ \MSK; śanda(?)staṣṭhā \cod
\Skt{%
tatrādau tāvadabhimatamaṇḍalabhūmimadhye 
mānuṣāsthi\edtext{cūrṇahome\supplied{na}}{\lemma{
	{\rm -cūrṇahomena\lem}
}\Dfootnote{
	\emn;
	\textit{-cūrṇah\unclear{ome}+} \cod\ \MSK
}} \supplied{rakta}\edtext{\supplied{vi}ṣasahitena}{\lemma{
	{\rm -viṣasahitena\lem}
}\Dfootnote{
	\emn\ \MSK; 
	\textit{+ṣasahitena} \cod
}\Cfootnote{
	According is Harunaga Isaacson, Alexis Sanderson's restoration is \textit{asṛgviṣasahitena}.%
}} 
% 	\textit{mi rus kyi phye ma khrag dang dug dang bcas pa dang} D
%	(powder of human bone mixed with blood and poison)
maṇḍala\-\edtext{vighnān}{\lemma{
	{\rm -vighnān\lem}
}\Dfootnote{
	\emn;
	\textit{-vighnaṃ} \cod\ \MSK
}} \edtext{nivāryā}{\lemma{
	{\rm nivāryā\lem-}
}\Dfootnote{
	\corr\ \MSK\ \sil; 
	\textit{nivāyā} \cod
}}tmaśiṣya\-bhūpālādiśāntikahomaṃ kuryāt //}
\pend

\pstart
\quad
\Skt{%
tato \edtext{bhūmiṃ}{\lemma{
	{\rm bhūmiṁ\lem}
}\Dfootnote{
	\corr\ \MSK;
	\textit{bhūmi} \cod
}} \edtext{
śodhayet}{\lemma{
	{\rm śodhayet\lem}
}\Dfootnote{
	\emn; 
	\textit{soḥdhāpayet} \cod;
	\textit{śodhāpayet} \MSK\ \silemn
}}~/
\edtext{vyāmamātraṃ}{\lemma{
	{\rm vyāmamātraṁ\lem}
}\Dfootnote{
	\emn\ \MSK;
	\textit{vyāmamantraṁ} \cod
}} kaṇṭhanābhi\edtext{jānu}{\lemma{
	{\rm -jānu-\lem}
}\Dfootnote{
	\cod;
	\textit{-janu-} \emn\ \MSK
}}mātraṃ vādhaḥ khanitvā
\edtext{sugandhābhyakta}{\lemma{
	{\rm sugandhābhyakta-\lem}
}\Dfootnote{
	\cod;
	\textit{sugandhātyakta-} \MSK
}}\edtext{mṛdāpūrya}{\lemma{
	{\rm -mṛdāpūrya\lem}
}\Dfootnote{
	\corr\ \MSK\ \sil; 
	\textit{-mṛdāpūya} \cod
}}
\edtext{vajraśikharā}{\lemma{
	{\rm vajraśikharā-\lem}
}\Dfootnote{
	\cod; 
	\textit{vajraśikhara-} \emn\ \MSK
}}parijaptagandhodaken\edtext{āsicyā}{\lemma{%
	{\rm -āsicyā-\lem}
}\Dfootnote{%
	\emn;
	\textit{-āsicyāsicyā-} \cod\ \MSK
}\Cfootnote{
	This emendation is supported by the Bhūtaḍāmara initiation manual and \textit{Vajrāvalī}.
}}koṭayet~/
\edtext{susamāṃ}{\lemma{
	{\rm susamāṁ\lem}
}\Dfootnote{
	\cod;
	\textit{susamaṁ} \emn\ \MSK
}} \edtext{kūṭāgāramantargatāṃ}{\lemma{
	{\rm kūṭāgārāntargatāṁ\lem}
}\Dfootnote{
	\cod;
	\textit{kūṭāgārāntargataṁ} \emn\ \MSK
	% \textcolor{red}{(Tib supports the emendation. NOTE)}
}} \edtext{catustoraṇaśobhitāṃ}{\lemma{
	{\rm catustoraṇaśobhitāṁ\lem}
}\Dfootnote{
	\cod;
	catustoraṇaśobhitaṁ \emn\ \MSK\ \sil;
	\textit{catustoraṇasobhitām} \cod
}}
\edtext{caturdvārākṣavāṭaparivṛtaparyantāṃ}{\lemma{
	{\rm caturdvārākṣavāṭaparivṛtaparyantāṁ\lem}
}\Dfootnote{
	\emn;
	\textit{caturdvārātkavāṭaparivṛtaṁ paryante} \cod;
	\textit{caturdvārākṣavāṭaparivṛtaṁ paryanta-} \MSK
}}
\edtext{\mbox{}\edtext{sucitrita}{\lemma{
	{\rm sucitrita-\lem}
}\Dfootnote{
	\emn;
	\textit{sucittritam} \cod;
	\textit{sucitritam} \MSK\ \silemn
}}saghaṇṭāvasa\-ktasatketu\edtext{vitānavitatottamāṃ}{\lemma{
	{\rm -vitānavitatottamāṁ\lem}
}\Dfootnote{
	\emn;
	\textit{vitānavitanottamam} \cod\ \MSK
}}}{\lemma{%
	{\rm sucitrita\dots vitatottamāṁ\lem}
}\Bfootnote{%
	Cf.\ \textit{Samāyoga} 6.3cd:
	\textit{ghaṇṭāvasaktasatketuvitānavitatottame}
}}
buddharatnādi\edtext{paṭapratimābhirupaśobhitāṃ}{\lemma{
	{\rm -paṭapratimābhir upaśobhitāṁ\lem}
}\Dfootnote{
	\corr;
	\textit{-paṭapratimābhirūpasobhitām} \cod;
	\textit{-paṭṭapratimābhir upaśobhitam} \MSK
}}
\edtext{catuṣkoṇāva}{\lemma{
	{\rm catuṣkoṇāva-\lem}
}\Dfootnote{
	\emn;
	\textit{catukoṇāva-} \cod
}}sthita\-dhūpa\edtext{ghaṭikāṃ}{\lemma{
	{\rm -ghaṭikāṁ\lem}
}\Dfootnote{
	\cod;
	\textit{-ghaṭikaṁ} \emn\ \MSK;
	\textit{-ghaṭṭikām} \MSK
}}
puṣpadīpavastrādibhi\edtext{ścopaśobhya}{\lemma{5
	{\rm copaśobhya\lem}
}\Dfootnote{
	\emn\ \MSK;
	\textit{cascopasobhya} \cod;
	\textit{casvopaśobhya} \MSK 
}} gandhenopalipya \edtext{vajrayakṣa}{\lemma{
	{\rm vajrayakṣa-\lem}
}\Dfootnote{
	\emn\ \MSK;
	\textit{vajrayarakṣa-} \cod
}}parijaptagandhodakena prokṣya bhūmau hastaṃ dattvā
vajrasattvaṃ śatākṣaraṃ ca saptaśa āvartayediti//}
\pend

\medskip

\pstart
\mbox{}\hfill
\Skt{%
// bhūmisaṃśodhanaparigrahavidhiḥ //}\hfill\mbox{}
\pend



\bigskip

% \input{05_Adhivasanahoma.tex}



\pstart\noindent
{\large 3.2. Adhivāsanahomavidhiḥ}
\pend

\bigskip

\pstart
\quad
\edtext{}{\lemma{%
	tataḥ \dots\ adhyeṣaṇīyaḥ\lem
}\Bfootnote{%
	Parallel to \textit{Tattvālokakarī} (P LOCATION!!, D f.109r1–109v1)
}}%
\Skt{tataḥ} 
\pend
\verse
\Skt{%
\edtext{\mbox{}\edtext{svayaṃ}{\lemma{
	{\rm svayaṁ\lem}
}\Dfootnote{%
	\corr\ \MSK; 
	\textit{svaya} \cod
}} \edtext{prātaḥ}{\lemma{
	{\rm prātaḥ\lem}
}\Dfootnote{
	\emn;
	\textit{snātaḥ} \cod\ \MSK
}} sugandhāṅgo
%	ov; sugandhaṁgo \cod
\edtext{yathāptyābharaṇāmbaraḥ}{\lemma{
	{\rm yathāptyābharaṇāmbaraḥ\lem}
}\Cfootnote{%
	RT proposes \emn\ yathāptā-.
}} /\\
suraktavastra\edtext{saṃvītaḥ}{\lemma{
	{\rm -saṃvītaḥ\lem}
}\Dfootnote{
	\emn\ \MSK; 
	\textit{-sampītaḥ} \cod
}} \edtext{sragvī}{\lemma{
	{\rm sragvī\lem}
}\Dfootnote{
	\emn\ \MSK; 
	\textit{śragdhī} \cod
}\lemma{}\Efootnote{
	-saṁvītaḥ [29v1] sragvī
}} surabhitānanaḥ //}{\lemma{%
	{\rm svayaṃ \dots\ surabhitānanaḥ\lem}
}\Bfootnote{
	\textit{Paramādya} (P LOCATION!!, D. f.189r);
	Tibetan translation of \textit{Paramādya} suggest that the reading of the opening is not \textit{snātaḥ} but \textit{prātaḥ} (in the morning).
	See also \textit{Samāyoga} 6.20:
	\textit{tataḥ snātaḥ sugandhāṅgo vicitrābharaṇāvṛtaḥ |
	vicitravastrasaṃvītaḥ sragvī surabhitānanaḥ ||}
}}}
\pend

\pstart\noindent
\Skt{%
śuklāṣṭamīmārabhya daśamīṃ trayodaśīṃ caturdaśīṃ vārabhya pañcadaśīṃ yāvanmaṇḍalakarma \edtext{kuryāt}{\lemma{
	{\rm kuryāt\lem}
}\Dfootnote{
	\emn\ \MSK\ \sil; 
	\textit{kuyāt} \cod
}} /
\edtext{prakṛtisthabhūbhāgaṃ}{\lemma{
	{\rm prakṛtisthabhūbhāgaṃ\lem}
}\Dfootnote{
	\emn;
	\textit{prakṛtisthabhūbhāge} \cod\ \MSK
}} tu \edtext{saṃmārjyopalepanaṃ}{\lemma{
	{\rm saṁmārjyopalepanaṁ\lem}
}\Dfootnote{
	\emn; 
	\textit{sanmārjyepalepana} \cod; 
	\textit{saṁmārjya lepanaṁ} \MSK\ \silemn
}} kṛtvā tathaiva \edtext{hastenālabhya}{\lemma{
	{\rm hastenālabhya\lem}
}\Dfootnote{
	\emn\ \MSK\ \sil; 
	\textit{hastenolabhya} \cod
}} vajrasattvamāvartayet /}
\Skt{tataḥ sarvaṃ vidhiṃ \edtext{kuryāt}{\lemma{
	{\rm kuryāt\lem}
}\Dfootnote{
	\emn\ \MSK\ \sil; 
	\textit{kuyāt} \cod
}} /
paurṇamāsyāṃ vā \edtext{pūrvāhnamārabhya}{\lemma{
	{\rm pūrvāhnam\lem}
}\Dfootnote{
	\corr\ \MSK\ \sil; 
	\textit{pūrvvāhṇam} \cod
}\lemma{
	{\rm ārabhya\lem}
}\Dfootnote{
	\emn\ \MSK; 
	\textit{āra} \cod\
	Or to be emended to \textit{ārambhaṁ}?
}} kuryāditi /
maṇḍalapraveśadivase \edtext{tvanāhāreṇā}{\lemma{
	{\rm anāhāreṇā-\lem}
}\Dfootnote{
	\emn\ \MSK\ \sil; 
	\textit{anahāreṇā-}
}}cāryeṇa śiṣyasahitena bhavitavyam //}

\quad
\Skt{%
tatra \edtext{tāvanmaṇḍalabhūmimadhye}{\lemma{
	{\rm maṇḍalabhūmimadhye\lem}
}\Dfootnote{
	\emn;
	\textit{mahālabhūmimadhye} \cod$^{ac}$;
	\textit{malabhūmimadhye }\cod$^{pc}$.
	Cf. Tib. \textcolor{red}{CHECK!}
}}
sthitvātmarakṣāṃ vighnaghātādikaṃ ca kuryāt /
tato \edtext{vajracakrayā}{\lemma{
	{\rm vajracakrayā\lem}
}\Cfootnote{
	 \MSK\ wrongly reads this as \textit{vajracakrayo}, 
	 and needlessly emends to \textit{vajracakreṇa}. 
	 Understand vajracakrayā [mudrayā].
}} maṇḍalaṃ nirmāya
\edtext{praṇāmādi}{\lemma{
	{\rm praṇāmādi-\lem}
}\Dfootnote{
	\emn;
	\textit{praṇāmādika-} \cod\ \MSK
}}pūrvakaṃ saṃvaragrahaṇaṃ \edtext{mahāyogaṃ}{\lemma{
	{\rm mahāyogaṁ\lem}
}\Dfootnote{
	\emn\ \MSK;
	\textit{mayogaṁ} \cod;
	\textit{mahāyogaṁ} \emn\ \MSK;
	Cf. Tib. \textcolor{red}{CHECK!}
	corrupt. We don't accept \MSK/Tib. \textit{mahāyogaṁ}; need a verb/absolutive here.
%	D 2510 (first volume 109a, 2nd vol. 105b, 178b,) Tattvaalokakarii
%	Kosala 2512 31a 230b 56a rnayl 'byor.*khang	
%  Mahaayoga is to empower body, speech, and mind: Bhavyakiirti's commentary on th Pa.jcakrama (D 1793 f.3b)
% Guhyasamaajama.n.dalavidhi vv.320–321
}} \supplied{kṛtvā} \edtext{kūṭāgāram}{\lemma{
	{\rm kūṭāgāram\lem}
}\Dfootnote{
	\emn;
	\textit{kuṭāram} \cod;
	\textit{kūṭāgāram} \emn\ \MSK
}} āsanāni ca niṣpādya
\edtext{sattvaparyaṅkaniṣaṇṇa}{\lemma{
	{\rm sattvaparyaṅkaniṣaṇṇa\lem}
}\Dfootnote{
	\emn;  
	\textit{satvaparyaṅkaniṣarṇṇa} \cod
}} ātmānaṃ \edtext{harṣayet}{\lemma{
	{\rm harṣayet\lem}
}\Dfootnote{
	\emn;
	\textit{karṣayet} \cod\ \MSK
}\Cfootnote{
	This emendation is supported by KSP.
}} //}
\pend

\verse
\Skt{%
\edtext{adya me \edtext{saphalaṃ janma}{\lemma{
	{\rm saphalaṁ janma\lem}
}\Dfootnote{
	\emn;
	\textit{satvamahāsamudrayā vyavasthitaphala janmā} \cod;
	The contamination is due to the eyeskip of the scribe from \textit{sa} of \textit{saphalaṁ} to \textit{sa} of \textit{śrīvajrasattva-}.
}} saphalaṃ jīvitaṃ ca me /\\
samaḥ samayabuddhānāṃ \edtext{bhavitāhaṃ}{\lemma{
	{\rm bhavitāhaṁv\lem}
}\Cfootnote{
	reading supported by testmonia 
	No need to emend.
}\Dfootnote{
	\cod;
	{\rm bhavitāhe} \emn\ \MSK
}} na saṃśayaḥ //\\
\edtext{avaivartyo}{\lemma{
	{\rm avaivartyo\lem}
}\Cfootnote{
	CHECK KSP!
}} bhaviṣyāmi bodhisattvaikacetanaḥ /\\
\edtext{tathāgatakulotpattirmamādya}{\lemma{
	{\rm mamādya\lem}
}\Dfootnote{
	\emn\ \MSK;
	\textit{mmagādya} \cod
	Perhaps there are traces of an attempt to correct the scribal error.
}} syānna saṃśayaḥ //\\
agro me divaso hyadya yajño me'dya niruttaraḥ /\\
saṃnipāto'dya me hyagraḥ sarvabuddhanimantraṇāditi //}{\lemma{
	{\rm adya\ \dots nimantraṇāt\lem}
}\Bfootnote{
%
%
This set of three stanzas are found in the following texts.
Gottingen Bhūtaḍāmara manual:
\textit{adya me saphalaṃ janma saphalaṃ ca jīvitaṃ ca me |
samayasamayadevānāṃ bhavitāhaṃ na saṃśayaḥ | 
avaivarttyo bhaviṣyāmi bodhisattvaikacetanaḥ | 
mahāvajrakulotpattir mamādya syān na saṃśayaḥ | 
agro me divaso hy adya yajño me 'dya niruttaraḥ | 
saṃnipāto 'dya me hy agraḥ sarvabuddhanimantraṇāt |}
(f.\ 1v–2r);
\textit{Viṃśatividhi} 6.3–5:
\textit{adya me saphalaṃ janma saphalaṃ jīvitaṃ ca me | 
samaḥ samayabuddhanāṃ bhavitāhaṃ na saṃśayaḥ || 
avaivarttī bhaviṣyāmi bodhisattvaikacetanaḥ | 
tathāgatakulotpattir mamādyaiva na saṃśayaḥ || 
agro me divaso hy adya yajño me 'dya niruttaraḥ | 
sannipāto 'dya me hy agraḥ sarvabuddhanimantraṇād iti ||};
KSP ch.6 (\textit{pratiṣṭhā}):
\textit{adya me saphalaṃ janma saphalaṃ jīvitaṃ ca me | 
samaḥ samayabuddhanāṃ bhavito 'haṃ na saṃśayaḥ || 
avaivartyo bhaviṣyāmi bodhicittaikacetanaḥ | 
tathāgatakulotpattir mamādyaiva na saṃśayaḥ || 
agro me divaso hy adya yajño me 'dya niruttaraḥ | 
sannipāto 'dya me hy agraḥ sarvabuddhanimantraṇāt ||}
(Tanemura 2004: 176).
%
Only the first and third stanza are found in Padmaśrīmitra's \textit{Maṇḍalopāyikā}:
\textit{adya me saphalaṃ janma saphalaṃ jīvitaṃ ca me |
samaḥ samayadevānāṃ bhavitāhaṃ na saṃśayaḥ ||
agro me divaso hy *adya} (\corr\ \textit{adyaḥ} \cod)\textit{ yajño me (')dya niruttaraḥ |
ājñāṃ karomy ahaṃ nātha sarvabuddhānukaṃpayā ||}
(\textit{pāda}s c and d are different) (f.\ 7v1)
%
}}}
% KSP; Padmaśrīmitra; Gottingen Bhūtaḍāmara manual; Viṃśatividhi
\pend

\pstart
\quad
\Skt{tataḥ \edtext{sarvāṅgena}{\lemma{
	{\rm sarvāṅgena\lem}
}\Dfootnote{
	corr;
	\textit{savāṅgena} \cod
}} praṇamya \edtext{dhūpaghaṭikāhastaḥ}{\lemma{
	{\rm dhūpaghaṭikāhastaḥ\lem}
}\Dfootnote{
	\emn;
	\textit{dhūpavattikāhastaḥ} \cod
	See two other occurrences of \textit{dhūpaghaṭikā} in the text.
}} \edtext{sarva\supplied{buddhānnimantra}yet}{\lemma{
	{\rm sarrva\supplied{buddhān nimantra}yet\lem}
}\Cfootnote{
	the reconstruction is that of \MSK following Tib.:
	\textit{sang rgyas thams cad spyan drang bar bya'o}.
	\textit{buddhān āvāhayet} would also be possible.
}\lemma{}\Efootnote{%
	nimantra[30r1]yet
}} /} 
\pend

\verse
\Skt{%
samanvāharantu māṃ buddhā \edtext{aśeṣadikṣu}{\lemma{
	{\rm aśeṣadikṣu\lem}
}\Dfootnote{
	corr.;
	\textit{aśeṣādikṣu} \cod
}} saṃsthitāḥ /\\
amuko nāmāhaṃ vajrī maṇḍalaṃ kalpayāmyaham /\\
āyāntu sarvabuddhādyāḥ siddhimenāṃ pradāsyatheti //}
\pend

\pstart
\noindent
\Skt{%
\edtext{uktvā}{\lemma{
	{\rm pradāsyathety uktvā\lem}
}\Dfootnote{
	restoration by \MSK;
	\textit{pradāsya[tyetyuktvā} \cod.
	The restoration of \MSK seems metrically unacceptable.
}}~/
ādiyogaṃ maṇḍalarājāgrīṃ karmarājāgrīṃ ca vibhāvya
\edtext{punardvārodghāṭana}{\lemma{
	{\rm punardvārodghāṭana-\lem}
}\Dfootnote{
	\corr;
	\textit{punadvārodghāṭana} \cod;
	\textit{punar dvārodghāṭana} \MSK
}}pūrvakaṃ vajrasattvasthāne 
śrīvajrasattvamahāmudrayā 
\edtext{vyavasthitastanmantrodīraṇa}{\lemma{
	{\rm tanmantrodīraṇa-\lem}
}\Dfootnote{
	\corr\ \MSK;
	\textit{tamantrodīraṇa} \cod
}}tatparo yogī sarvottarasādhakaistathāgatāhaṃkāra\edtext{yuktairnāmāṣṭaśatena}{\lemma{
	{\rm -yuktair\lem}
}\Dfootnote{
	corr. \MSK;
	\textit{-yuktai} \cod
}} mahāmaṇḍalālikhanāyādhyeṣaṇīyaḥ //}
\pend

\pstart
\quad
\edtext{}{\lemma{%
	tata utthāya \dots\ na riñcati\lem
}\Bfootnote{%
	parallel to \textit{Tattvālokakarī} (P LOCATION!!, D f.109v2–4)
}}%
\Skt{%
tata utthāya \edtext{sarvatathāgatairgagaṇa\edtext{māpūrayamāṇaṃ}{\lemma{
	{\rm āpūrayamāṇaṁ\lem}
}\Dfootnote{
	\emn\ as in §59;
	\textit{āpūrayad} \cod
}} \edtext{dṛṣṭvā}{\lemma{
	{\rm dṛṣṭvā\lem}
}\Dfootnote{
	\emn\ \MSK;
	\textit{dṛṣṭā} \cod
}}
sarvatathāgatapāda\edtext{vandanāṃ}{\lemma{
	{\rm -vandanāṁ\lem}
}\Dfootnote{
	\cod;
	\textit{-vandanaṁ} \MSK (probably a typo)
}} kṛtvā}{\lemma{%
	{\rm sarvatathāgatair gagaṇa\dots\ kṛtvā\lem}
}\Cfootnote{%
	In \cod, \textit{sarvatathāgatapādavandanāṃ kṛtvā} precedes \textit{sarvatathāgatair gagaṇam āpūrayamāṇaṃ dṛṣṭvā}, but it is implausible that the officiant makes reverence to the feet of the \textit{tathāgata}s before drawing down them. Probably the metaphasis is caused by the face that the both parts begin with \textit{sarvatathāgata-}. Our emendation is supported by Tib.
}} 
%
}
\pend

\verse
\Skt{%
\edtext{ahameva svayaṃ vajrī vajrasattva ahaṃ svayam /\\
ahaṃ buddho mahārājā ahaṃ vajrī mahābalaḥ //}{\lemma{
	{\rm aham \dots\ mahābalaḥ\lem}
}\Bfootnote{
	VĀ and KSP
}}\\
\edtext{ahaṃ yogīśvaro rājā vajrapāṇirahaṃ dṛḍhaḥ /\\
ahaṃ svāmī mahāvajra \edtext{adhiṣṭhānaṃ}{\lemma{
	{\rm adhiṣṭhānaṁ\lem}
}\Dfootnote{
	\emn;
	\textit{adhiṣṭhānā} \cod\ \MSK
}} na riñcati //}{\lemma{%
	{\rm ahaṁ \dots\ riñcati\lem}
}\Bfootnote{%
	\textcolor{blue}{These two verses and the preceding prose:
	Bhūtaḍāmara manual.}
}}}
\pend
% ここまでlem作業 2018/06/30
\pstart
\quad
\Skt{%
\edtext{tato vajradṛṣṭi maḍiti cakṣuṣorvinyasya~/
{\ah}kāreṇa pādatalayorviśvavajraṃ nirmāya
\edtext{svasamayamudrāṃ baddhvoparyākāśe}{\lemma{
	{\rm svasamayamudrāṁ baddhvoparyākāśe\lem}
}\Dfootnote{
	\emn;
	\textit{svasamayamudrāmvadho[pa]yākāse} \cod
	Cf. §49 and 68 for \textit{uparyākāśa}.
}}  tanmaṇḍalamutthāpya 
\edtext{vajrottiṣṭheti}{\lemma{
	{\rm vajrottiṣṭheti\lem}
}\Dfootnote{
	\emn;
	\textit{vajrasatva vajrotiṣṭheti} \cod.
	Or keep \MSK\ reading, or emend \textit{vajrasattva vajrottiṣṭheti}?
}} /
%
tatastayā svasamayamudrayātmānaṃ \edtext{sakṛdadhiṣṭhāya}{\lemma{
	{\rm sakṛd\lem}
}\Dfootnote{
	\emn\ \MSK (supported by Tib.);
	\textit{kṛd} \cod
}} 
\edtext{punarmahāmudrāṃ}{\lemma{
	{\rm punar\lem}
}\Dfootnote{
	\emn\ \MSK;
	\textit{puna} \cod
}} baddhvā}{\lemma{%
	{\rm tato \dots\ baddhvā\lem}
}\Bfootnote{%
	parallel to \textit{Tattvālokakarī} (P LOCATION!!, D f.109v5–6)
}} /}
\pend

\verse
\Skt{%
\edtext{tathaivotthāya \edtext{mudrāsthaḥ sarvato}{\lemma{}\Efootnote{
	mudrāsthaḥ [30v1] sarvato
}} vyavalokayan /\\
\edtext{parikrametsvagarveṇa}{\lemma{
	{\rm parikramet svagarveṇa\lem}
}\Dfootnote{
	\emn;
	\textit{parikrameta sagarvveṇa} \cod\ \MSK
}} \edtext{vajrasattvamudāharan}{\lemma{
	{\rm vajrasattvam\lem}
}\Dfootnote{
	\textit{vajrasatvam} \cod;
	\textit{vajrasattvety} \MSK. The emendation is unwarranted.
}} }{\lemma{
	{\rm tathaivotthāya \dots\ udāharan\lem}
}\Bfootnote{
	This is a verse quoted from STTS §203:
	\textit{tathaivotthāya mudrāsthaḥ sarvato vyavalokayan | 
	parikramena garveṇa vajrasattvam udāharan ||}. 
	No doubt \textit{parikrameta} is the better reading. 
	\textcolor{red}{Check ed. Yamada and ms.}
	RT confirmed that Yamada reads \textit{parikrameta}.
	\textcolor{red}{CHECK Bhūtaḍāmara manual!!}
	\textit{tathaivotthāya mudrāsthaḥ sarvato vyavalokayan |
	parikramet svagarveṇa svamantraṃ samudāharan ||}
}} //}
\pend

\pstart
\noindent
\Skt{%
\edtext{\edtext{vajradṛṣṭyā}{\lemma{
	{\rm vajradṛṣṭyā\lem}
}\Dfootnote{
	\emn\ \MSK;
	\textit{vajradṛṣṭvā} \cod
}} \edtext{diksīmāmaṇḍalabandha}{\lemma{
	{\rm diksīmāmaṇḍalabandha-\lem}
}\Dfootnote{
	\emn\ \MSK\ \sil;
	\textit{diksīmā\-maṇḍalavadhva-} \cod
%	\textit{diksīmāmaṇḍalabandha-} \MSK
}}prākārapañjaraṃ 
pādatalaparigraheṇa \edtext{ca rasātala}{\lemma{
	{\rm ca rasātalam\lem}
}\Dfootnote{
	\emn;
	\textit{cārasātalam} \cod
}\Cfootnote{
	\textit{-grahe(ṇ)a}: the scribe has accidentally added a diagonal stroke to \textit{ṇa}. Perhaps he meant to cross out the \textit{akṣara}.
	\textit{pādatalaparigraheṇa bhūmitalam} \MSK.
	This emendation is unwarranted.
	cf. Ratnākaraśānti \textit{Bhramaharasādhana} for term \textit{sīmābandha} and \textit{ārasātala}.
}}mupādāya sumeru\edtext{pṛṣṭhaṃ}{\lemma{
	{\rm -pṛṣṭhaṁ\lem}
}\Dfootnote{
	corr.;
	\textit{-pṛṣṭaṁ} \cod
}} yāvadvajramayaṃ \edtext{kurva}{\lemma{
	{\rm kurvan\lem}
}\Dfootnote{
	\emn\ \MSK;
	\textit{kurvvam} \cod
}}npunarvighnaghātādikaṃ kṛtvā guhyarūpaṃ maṇḍalamākarṣayedanena mudrāyuktena
{\om} vajramaṇḍala hū{\cb} jaḥ iti}{\lemma{%
	{\rm vajradṛṣṭyā \dots\ hūṃ jaḥ iti\lem}
}\Bfootnote{%
	parallel to \textit{Tattvālokakarī} (P LOCATION!!, D f.109v7–110r3)
}} //}

\quad
\Skt{%
\edtext{\mbox{}\edtext{vajramuṣṭidvayena}{\lemma{
	{\rm vajramuṣṭidvayena\lem}
}\Dfootnote{
	\emn;
	\textit{vajramuṣṭidvaye} \cod\ \MSK
}} \edtext{tarjanyāṅguṣṭhacakrā}{\lemma{
	{\rm tarjanyāṅguṣṭhacakrā\lem}
}\Dfootnote{
	\corr;
	\textit{tarjanyāṁguṣṭavajra-} \cod;
	\textit{tarjanyāṅguṣṭhavajrā} \MSK
}} sarvamaṇḍalākarṣaṇī \edtext{vajracakrā nāma mudrā}{\lemma{
	{\rm vajracakrā nāma mudrā\lem}
}\Cfootnote{
	For \textit{vajracakrā}, see STTS §275.
}}}{\lemma{%
	{\rm vajramuṣṭidvayena \dots\ mudrā\lem}
}\Bfootnote{%
	KSP (CHECK!!)
}} /}
\pend

\verse
\Skt{
{\om} \edtext{mahāvajrācakre'dhitiṣṭha}{\lemma{
	{\rm mahāvajrācakre 'dhitiṣṭha\lem}
}\Dfootnote{
	\cod;
	\textit{vajracakrādhitiṣṭha} \MSK
}} sidhya hū{\cb} /} 
\pend
\pstart
\noindent
\Skt{ityanayā maṇḍalaṃ punaradhitiṣṭhet /
vajrahetukarmamudraivāsyā mudrā //}
\pend

\pstart
\quad
\Skt{%
tataḥ khadiravajrakīlakā \edtext{maṇḍalakoṇacatuṣṭaye}{\lemma{
	{\rm maṇḍalakoṇacatuṣṭaye\lem}
}\Dfootnote{
	corr.;
	\textit{maṇḍalakoṇe catuṣṭaye} \cod \MSK
}} vajreṇākoṭyāḥ /}
\pend

\verse
\Skt{%
{\om} vajrakīla \edtext{kīlaya}{\lemma{
	{\rm kīlaya\lem}
}\Dfootnote{
	corr. \MSK;
	\textit{kīleya} \cod
}} \edtext{sarvavighnān}{\lemma{
	{\rm sarvavighnān\lem}
}\Dfootnote{
	\emn;
	\textit{sarvavighnām} \cod;
	\textit{sarvavighnaṁ} \MSK
}} bandha hū{\cb} phaṭ /}
\pend
\pstart
\noindent
\Skt{%
ityanena \edtext{hṛdayenāṣṭottaraśataśataparijaptāḥ}{\lemma{
	{\rm hṛdayenā\dots parijaptāḥ\lem}
}\Dfootnote{
	\emn;
	\textit{hṛdayecāṣṭottaraśataparijaptā °i} \cod;
	\textit{hṛdayenāṣṭottaraśataṁ parijapya} \MSK
}\Cfootnote{
	We do not follow \MSK in emending a an adverbial form in \textit{-m/ṁ}, because other examples of \textit{aṣṭottaraśata(pari)japta} are apparently attested in the text; their emendation of an absolutive form (\textit{parijapya}) also seems unnecessary. The error \textit{°i} for ḥ could be explained by assuming a graphic error \textit{°i} for \textit{ḥ} in an ancestor manuscript of our codex. 
}} /}
\Skt{\edtext{vāmavajramuṣṭyā}{\lemma{
	{\rm vāmavajramuṣṭyā\lem}
}\Dfootnote{
	corr. \MSK;
	\textit{vāmavajramuṣṭhyā} \cod
}} vā pañcasūcikaṃ vajraṃ ādāya tena hū{\cb}kāra\edtext{mudīraya}{\lemma{
	{\rm udīrayan\lem}
}\Dfootnote{
	corr. \MSK;
	\textit{udīrayaṁ} \cod
}}\edtext{nmaṇḍalakoṇa\-catuṣṭaye}{\lemma{
	{\rm maṇḍalakoṇa\-catuṣṭaye\lem}
}\Dfootnote{
	\emn;
	\textit{maṇḍalaṁ koṇe catuṣṭayaṁ} \cod;
	\textit{maṇḍalakoṇe catuṣṭaye} \MSK
}} maṇḍalanābhau ca kīlakapañcakaṃ niṣpādya
dakṣiṇakareṇa trisūcikavajrīkṛtenākoṭaye\edtext{dimamudīrayan}{\lemma{
	{\rm imam udīrayan\lem}
}\Dfootnote{
	\emn\ \MSK;
	iyam udīyan \cod
}} \textemdash\ }
\pend

\verse
\Skt{%
{\om} \edtext{gha gha}{\lemma{
	{\rm gha gha\lem}
}\Dfootnote{
	\cod$^{pc}$ \MSK;
	\textit{gha ghā} \cod $^{ac}$
}} ghātaya ghātaya \edtext{sarvaduṣṭān}{\lemma{
	{\rm sarvaduṣṭān\lem}
}\Dfootnote{
	\emn\ \MSK;
	\textit{sarvaduṣṭā} \cod
}} phaṭ, 
kīlaya kīlaya \edtext{sarvapāpān}{\lemma{
	{\rm sarvapāpān\lem}
}\Dfootnote{
	\emn\ \MSK;
	\textit{sarvapāpā} \cod
}} phaṭ, 
vajrakīla \edtext{vajradhara ājñāpayati}{\lemma{
	{\rm vajradhara ājñāpayati\lem}
}\Dfootnote{
	\emn\ \MSK;
	\textit{vajradharo jñāpayati}  \cod
}\Cfootnote{
	the same reading is found in the inscription edited in Griffiths (2014: 169). Nevertheless, perhaps we should emend -\textit{dhara ājñā-} with \MSK.
}} svāheti~//}
\pend

\pstart
\quad
\Skt{%
\edtext{tato}{\lemma{}\Efootnote{
	svāheti || [31r1] tato
}} vajrayakṣaparijaptaprokṣaṇakalaśaṃ maṇḍalagṛhadvāre nyasya \edtext{vajramuṣṭikarmamudrayā}{\lemma{
	{\rm vajramuṣṭikarmamudrayā\lem}
}\Dfootnote{
	\emn\ \MSK;
	\textit{vajramuṣṭikarmmamudrāyā} \cod
}} sarvarakṣāṃ dṛḍhīkṛtya \edtext{vajrakavacena}{\lemma{
	{\rm vajrakavacena\lem}
}\Dfootnote{
	\emn;
	\textit{ca vajravajrakava[jra]na} \cod;
	\textit{ca vajrakavacena} \MSK
}\Cfootnote{
	emend \textit{ca vajrakavacena}, or even \textit{vajrakavacena} without \textit{ca}. The \textit{ca} might be a dittographical error.
}} kavacayet /
tathā cāha \textemdash\ }
\pend

\verse
\Skt{%
\edtext{bandhayogavidhānaṃ ca \edtext{vajramuṣṭiṃ}{\lemma{
	{\rm vajramuṣṭiṃ\lem}
}\Dfootnote{
	\emn\ \MSK
	\textit{vajramuṣṭi} \cod;
	\textit{vajramuṣṭiṁ} \MSK
}\Cfootnote{
	emend \textit{vajramuṣṭiṁ}, or keep Ārṣa/Aiśa form.
}} prakalpayet /\\
kavacaye\edtext{dvajrakavaca}{\lemma{
	{\rm vajrakavaca\lem}
}\Cfootnote{
	emend \textit{vajrakavacaṁ}, or keep Ārṣa/Aiśa form.
}} rakṣāṃ \edtext{sarvāṃ}{\lemma{
	{\rm sarvāṁ\lem}
}\Dfootnote{
	\emn;
	\textit{sarvan} \cod;
	\textit{sarvaṁ} \MSK
}} tu maṇḍala iti //}{\lemma{
	{\rm bandha\dots\ maṇḍala iti\lem}
}\Bfootnote{
	Source? Scripture?
	Pāda c: hypermetrical
}}}
\pend


\pstart
\quad
\Skt{%
punarmaṇḍalamadhye niṣadya manasā sarvamaṇḍalaṃ parikalpya sarvamaṇḍalasthānaṃ gandhenopalipya
pañcatathāgatasthāneṣu \edtext{caturasrāṇi}{\lemma{
	{\rm caturasrāṇi\lem}
}\Dfootnote{
	\emn;
	\textit{caturagrāṇi} \cod;
	\textit{caturaśrāṇi} \MSK
}\Cfootnote{
	The exemplar of our ms. presumably had \textit{caturaśrāṇi}, and tgis is what \MSK read.
}} candana\edtext{kuṃkumādibhi}{\lemma{
	{\rm -kuṁkumādibhir\lem}
}\Dfootnote{
	\emn\ \MSK;
	\textit{kuṁkumādibhi} \cod
}}rmaṇḍalakāni \edtext{kuryāt}{\lemma{
	{\rm kuryāt\lem}
}\Dfootnote{
	\emn\ \MSK;
	\textit{kuyāt} \cod
}} /
\edtext{śeṣeṣu}{\lemma{
	{\rm śeṣeṣu\lem}
}\Cfootnote{
	corr. or read \textit{śeṣeṣu}, as \MSK reads. It may be that there were two types of \textit{ś} in tguis ms., one (as here) archaic, close in shape to \textit{g} and especially \textit{s}, the other somewhat resembling \textit{ṇ}.
}} \edtext{vartulāni}{\lemma{
	{\rm vartulāni\lem}
}\Dfootnote{
	\emn\ \MSK;
	\textit{vaṁtulāni} \cod
}} /
\edtext{svasvamantraiśca}{\lemma{
	{\rm svasvamantraiś ca\lem}
}\Dfootnote{
	\emn;
	\textit{svamantrais ca} \cod;
	\textit{svamantraiś ca} \MSK;
	This emendation is supported by Tib.
}\Cfootnote{
	\textit{svamantrais ca} \cod.
	correct or read \textit{svamantraiś ca}. See note on preceding sentence.
}} saptavāra\edtext{parijaptāni}{\lemma{
	{\rm -parijāptāni\lem}
}\Dfootnote{
	\emn\ \MSK\ \sil;
	\textit{-parijāptāni} \cod
}} //}
\pend

\pstart
\quad
\Skt{%
tato vajrāṅkuśādibhi\edtext{rākṛṣya}{\lemma{
	{\rm ākṛṣya\lem}
}\Dfootnote{
	\emn\ \MSK; 
	\textit{ākṛ} \cod
}} praveśya baddhvā vaśīkṛtyākāśadeśātteṣu \edtext{tathāgatādīn}{\lemma{
	{\rm tathāgatādīn\lem}
}\Dfootnote{
	\emn;
	\textit{tathāgatodiṁ} \cod;
	\textit{tathāgatādiṁ} \MSK
}} \edtext{svahṛdayai}{\lemma{
	{\rm svahṛdayair\lem}
}\Dfootnote{
	\emn;
	\textit{svahṛdayai} \cod;
	\textit{svahṛdaye} \MSK
}}rniveśya pañcabhiru\edtext{pacāraiḥ}{\lemma{%
	{\rm upacāraiḥ\lem}
}\Dfootnote{%
	\emn\ \MSK\ \sil;
	\textit{upācāraiḥ} \cod
}}
\edtext{saṃpūjya/}{\lemma{
	{\rm saṁpūjya/ abhiṣekāya\lem}
}\Dfootnote{
	\textit{saṁpūjyā abhiṣekāya} \cod
}} }
\pend

\verse
\Skt{%
\edtext{\edtext{abhiṣekāya kalaśaṃ sarvavrīhyādisaṃyutam}{\lemma{
	{\rm abhiṣekāya kalaśaṁ sarvavrīhyādisaṃyutam\lem}
}\Cfootnote{
	na vipulā
}} /\\
stokaṃ toyasya prakṣipya vajrasattvābhimantritam // \\
adhivāsaye\edtext{dvidhivaddattvārghaṃ}{\lemma{
	{\rm vidhivad dattvārghaṁ\lem}
}\Dfootnote{
	\emn\ \MSK;
	\textit{vidhiva datvārghaṅ} \cod
}} gandhavāriṇā /\\
kusumāni ca prakṣipya \edtext{dhūpenaivādhi}{\lemma{
	{\rm dhūpenaivādhi-\lem}
}\Dfootnote{
	\emn\ \MSK;
	\textit{dhūpanaivādhi-} \cod
}}vāsayet //\\
\edtext{anyasmintvahani}{\lemma{
	{\rm anyasmin tv ahani\lem}
}\Dfootnote{
	\emn\ \MSK;
	\textit{°anyasmi tv ahani} \cod
}} \textcolor{red}{trisaṃdhyantaṃ} samyakparijapet /\\
tenābhiṣekaṃ kurvīta \edtext{puna}{\lemma{
	{\rm punar-\lem}
}\Dfootnote{
	corr.;
	\textit{puna} \cod
}}rjaptena maṇḍale /\\
lakṣaṇaṃ \edtext{vakṣyamāṇaṃ}{\lemma{
	{\rm vakṣyamāṇaṁ\lem}
}\Dfootnote{
	\emn\ \MSK; 
	\textit{vakṣamānañ} \cod
}} ca tasya jñeyaṃ mahātmabhiḥ //}{\lemma{
	{\rm abhiṣekāya \dots\ mahātmabhiḥ\lem}
}\Bfootnote{
	Durgatipariśodhana (CHECK!!)
}}}
\pend
\pstart
\noindent
\Skt{%
śrīvajrasattvasya purataś cāyaṃ sthāpyaḥ //}
\pend

\pstart
\quad
\Skt{%
tato \edtext{vajrayakṣa}{\lemma{}\Efootnote{%
	vajra[31v1]yakṣa-
}}vajrasattvabuddhalocanābhiḥ pratyekaṃ ghṛtādikamaṣṭottaraśataṃ \edtext{juhuyāditi}{\lemma{
	{\rm juhuyāt\lem}
}\Dfootnote{
	\emn\ \MSK;
	\textit{juhuyān} \cod
}}~//}
\pend

\medskip

\pstart
\mbox{}\hfill \Skt{%
// adhivāsanahomavidhiḥ //}\hfill \mbox{}
\pend

\bigskip

\pstart
\noindent
{\large 3.3. Adhivāsanavidhiḥ}
\pend

\bigskip

%\input{06_Adhivasana.tex}

\pstart
\Skt{%
tato bāhyabaliṃ dattvopaspṛśya nāmāṣṭaśatena saṃstutya \edtext{puṣpādibhirlāsyādibhiśca}{\lemma{
	{\rm puṣpādibhir lāsyādibhiś\lem}
}\Dfootnote{
	\emn\ \MSK;
	\textit{puṣpādibhi lāsyādibhis} \cod
}} sarvatathāgataṃ saṃpūjya praṇamya śiṣyānadhivāsayet~/}
\pend

\pstart
\Skt{%
tatrādau \edtext{tāva}{\lemma{
	{\rm tāvat\lem}
}\Dfootnote{
	\emn\ \sil\ \MSK;
	\textit{tāva} \cod
}}\edtext{tsusnāta}{\lemma{%
	{\rm -susnāta-\lem}
}\Dfootnote{
	\emn;
	\textit{-susnāna-} \cod\ \MSK
}}śucivastraiḥ puṣpakarai\edtext{ścācāryaṃ}{\lemma{
	{\rm cācāryaṁ\lem}
}\Dfootnote{
	\emn\ \MSK;
	\textit{cācāyam} \cod
}} praṇamya śiṣyai\edtext{revaṃ}{\lemma{
	{\rm evaṁ\lem}
}\Dfootnote{
	\emn;
	\textit{eva} \cod \MSK
}} \edtext{vaktavyam}{\lemma{
	{\rm vaktavyam\lem}
}\Dfootnote{
	\cod; 
	\textit{vaktavyaḥ} \MSK
}}~/}
\pend

\verse
\Skt{%
\edtext{tvaṃ me śāstā mahārata~/\\
icchāmyahaṃ mahānātha bodhisattvanayaṃ dṛḍham~/\\
dehi me samayaṃ tattvaṃ bodhicittaṃ ca dehi me~//\\
buddhaṃ dharmaṃ ca saṃghaṃ ca dehi me śaraṇaṃ trayam~/\\
praveśayasva māṃ nātha mahāmokṣapuraṃ varamiti~//}{\lemma{%
	{\rm tvaṁ me \dots\ varam iti\lem}
}\Bfootnote{%
	Nāgabodhi's \textit{Viṃśatividhi} 7.3–4:
	\textit{tvaṃ me śāstā mahārata | % eyeskip?
	icchāmy ahaṃ mahānātha bodhisattvanayaṃ dṛḍhaṃ |
	dehi me samayaṃ tattvaṃ bodhicittaṃ ca dehi me || 
	buddhaṃ dharmaṃ ca saṅghaṃ ca dehi me śaraṇatrayaṃ | 
	praveśayasva māṃ nātha mahāmokṣapuraṃ varam iti ||}
	\textcolor{red}{CHECK TANAKA'S ED!!};
	\textit{Durgatipariśodhana}:
	\textit{tataḥ śiṣyān praveśayet. 
	tatra pañcaśikṣāpadaparigṛhītena śrāmaṇerakabhikṣusaṃvaragṛhītena vā ācāryābhiṣekārhe nācāryapādayoḥ praṇipatyaivaṃ vaktavyam. 
	tvaṃ me śāstā mahārataḥ |
	icchāmy ahaṃ mahānātha mahābodhinayaṃ dṛḍham |
	dehi me samayatattvaṃ saṃvaraṃ ca dadasva me || iti}.
	(p.284)\
	In later texts, the first \textit{pāda} is embedded into a single stanza.
	See, e.g., VA \S\ 20.4:
	\textit{mṛtyujanmajarānakramakarādibhyānakāt |
	bhavābdher eka uddhartā tvaṃ me śāstā mahārataḥ ||
	icchāmy ahaṃ mahānātha mahābodhinayaṃ dṛḍham |
	dehi me samayaṃ tattvaṃ bodhicittaṃ ca dehi me ||
	buddhaṃ dharmaṃ ca saṅghaṃ ca dehi me śaraṇatrayam |
	praveśayasva māṃ nātha mahāmokṣapuraṃ varam ||
	iti pāṭhayan \dots} (p.375)
}}
}
\pend

\pstart
\Skt{%
tataḥ pāpadeśanādikaṃ kārayet~/}
\pend

\verse
\Skt{%
\edtext{\mbox{}}{\lemma{
	{\rm śṛṇu \dots\ vidivehe\lem}
}\Cfootnote{
	The stanzas are in \textit{Āryā} except the 7th one \textit{Gīti} and the 8th \textit{Anuṣṭubh}.
}}śṛṇu \edtext{bhadrāśayanibhṛta}{\lemma{
	{\rm bhadrāśayanibhṛta\lem}
}\Dfootnote{
	\emn\ \MSK;
	bhadrāśayanibhṛtaṃ \cod
}} samyaksaṃhṛtya kalpanāḥ sakalāḥ~/\\
visṛtamatirno sugatairadhiṣṭhyate vajrasattvādyaiḥ~//}
\pend

\verse
\Skt{
kṛtamanumoditamakuśalamavaśena kāritaṃ yacca~/\\
tatsarvamagrabodheḥ purataḥ pratideśayāmyadhunā~//}
\pend

\verse
% \textit{Samantabhadrasādhana} by Jñānapāda, v.60
% CHECK!!
\Skt{saṃbhāradvaya\edtext{maniśaṃ}{\lemma{
	{\rm aniśaṃ\lem}
}\Dfootnote{
	\emn\ \MSK;
	\textit{aniśa} \cod
}} sugatasutānāma\edtext{gāḍha}{\lemma{
	{\rm -agāḍha-\lem}
}\Dfootnote{
	corr.; 
	\textit{agādha} \cod;
	\textit{agādhaṃ} \MSK
}}gambhīram~/\\
sakalajagadarthasādhakamanumode 'haṃ tato 'nyadapi~//}
\pend

\verse
\Skt{%
\edtext{kṛpayā parīttamānasa\edtext{managhamatiṃ}{\lemma{%
	{\rm anaghamatiṁ\lem}
}\Dfootnote{%
	\textit{ānaghamatiṁ} \cod\ \MSK
}} sakalakāyahatamoham~/\\
sugataṃ \edtext{prayāmi śaraṇaṃ}{\lemma{}\Efootnote{%
	prayāmi [32r1] śaraṇaṁ
}} sakala\edtext{kulāmbhoja}{\lemma{
	{\rm -kulāmbhoja-\lem}
}\Dfootnote{
	\corr\ \MSK;
	\textit{-kulāmbhojā-} \cod
}}madhyagatam~//}{\lemma{%
	{\rm kṛpayā \dots\ -madhyagatam\lem}
}\Bfootnote{%
	Quoted in \textit{Vajrajvālodayā}
	\textcolor{red}{(LOCATION!!)}
}} %
}
\pend

\verse
\Skt{%
pravaradhiyāmārāmaṃ hīnabhayaṃkaratayā ca jetṛvanam~/\\
dharmaṃ gato 'smi śaraṇaṃ bhayavibhavavibhāvanācaturam~//}
\pend

\verse
\Skt{%
\edtext{rāgādyuraga}{\lemma{
	{\rm rāgadyuraga-\lem}
}\Dfootnote{
	\emn\ ($\leftarrow$ Tib.: ’dod chags la sogs sbrul) \MSK;
	\textit{rā[gādyu]+ga-} \cod
}}viṣāpahamurukaruṇāmānasaṃ vibuddhadhiyam~/\\
\edtext{vītabhavavandita}{\lemma{
	{\rm vītabhavavanditam\lem}
}\Dfootnote{
	\cod;
	\textit{vītabhavaṁ vanditam} \MSK
}}mahaṃ prayāmi śaraṇaṃ yatīśagaṇam~//}
\pend

\verse
\Skt{
\edtext{hetusamanantarādhipa}{\lemma{
	{\rm hetusamanantarādhipa-\lem}
}\Dfootnote{
	\cod$^{pc}$;
	\textit{hetusamantantarādhipa} \cod$^{ac}$
}}\edtext{viṣayātma}{\lemma{
	{\rm -viṣayātma-\lem}
}\Dfootnote{
	\corr\ \sil\ \MSK (silently);
	\textit{viśayātma} \cod
}}phalaprabhāvajitaśatrum~/\\
\edtext{pratividhya}{\lemma{
	{\rm pratividhya\lem}
}\Dfootnote{
	\emn\ \MSK;
	\textit{pratividdhya} \cod
}\lemma{
	{\rm pratividhya kṛpāmūlaṁ\lem}
}\Cfootnote{
	This should be emended to \textit{pratividdhakṛpāmūlaṃ}
	in a single compound?
}} kṛpāmūlaṃ bodhau saṃvedanaṃ bibharmyasamam~//}
\pend

\verse
\Skt{
sattvānāṃ paripākāya \edtext{paritrāṇāya}{\lemma{
	{\rm paritrāṇāya\lem}
}\Dfootnote{
	\corr\ \MSK;
	\textit{parittrāṇāya} \cod
}} vā \edtext{punaḥ}{\lemma{
	{\rm punaḥ\lem}
}\Dfootnote{
	\corr\ \MSK;
	\textit{yunaḥ} \cod
}}~/\\
svacetaḥśuddhaye caitatsarvaṃ \edtext{dadyāṃ}{\lemma{
	{\rm dadyāṁ\lem}
}\Dfootnote{
	\emn;
	\textit{dadyāt} \cod\ \textcolor{red}{(CHECK!!)} \MSK
}} \edtext{tṛṇādivat}{\lemma{
	{\rm tṛṇādivat\lem}
}\Dfootnote{
	\emn\ \MSK;
	\textit{taṇādivat} \cod
}}~//}
\pend

\verse
\Skt{%
evaṃ śīlaṃ \edtext{kṣamāṃ}{\lemma{
	{\rm kṣamāṃ\lem}
}\Dfootnote{
	\emn\ \MSK;
	\textit{kṣamā} \cod
}} vīryaṃ \edtext{dhyānaṃ}{\lemma{
	{\rm dhyānaṃ\lem}
}\Dfootnote{
	\emn\ \MSK;
	\textit{dhyāna} \cod
}} prajñāmanuttarām~/\\
bhāvayeyaṃ viśuddhyarthī \edtext{svapareṣāṃ}{\lemma{
	{\rm svapareṣāṃ\lem}
}\Dfootnote{
	\emn\ \MSK;
	\textit{svareṣāṃ} \cod
}} pratikṣaṇam~//}
\pend

\verse
\Skt{
saugatamantraviviktaṃ satatamanābhogavāhi \edtext{sāmājyam}{\lemma{
	{\rm sāmājyam\lem}
}\Dfootnote{
	\cod;
	\textit{samājam} \MSK
}\Cfootnote{
	Perhaps no need to be emended.
	This is Buddhist counterpart of Śaiva \textit{sāyujya}, etc.
}}~/\\
prāpayituma\edtext{śeṣajagat}{\lemma{
	{\rm aśeṣajagat\lem}
}\Dfootnote{
	\emn;
	\textit{eteṣāṃ jagatāṃ} \emn\ \MSK
	\textit{eśajatāṁ} \cod
}} sthito 'hama\edtext{dhunāpi}{\lemma{%
	{\rm adhunāpi\lem}
}\Dfootnote{
	\emn;
	\textit{adhunāvi} \cod\ \MSK
}} diveha~//}
% There must be serious corruptions here.

\quad
\edtext{}{\lemma{
	This section\lem
}\Bfootnote{
	Cf.\ \textit{Aṣṭasāharikā Prajñāpāramitā}
}}%
\Skt{anena caivaṃ \edtext{samudānītena}{\lemma{
	{\rm samudānītena\lem}
}\Dfootnote{
	\emn\ \MSK;
	\textit{samudānitena} \cod
}} kuśalamūlena sarvasvaparityāginā mayā bhavitavyam~/
sarvasattvānām antike samacittatā \edtext{mayotpādayitavyā}{\lemma{}\Efootnote{
	ma[32v1]yotpādayitavyā
}\lemma{
	{\rm mayotpādayitavyā\lem}
}\Dfootnote{
	\emn\ \MSK;
	\textit{māyotpādayitavyā} \cod
}}~/
sarvasattvā mayā sarvayānaiḥ \edtext{sarvopāyavisarai}{\lemma{
	{\rm sarvopāyavisarair\lem}
}\Dfootnote{
	\emn\ \cod;
	\textit{sarvopāyavisarai} \cod
}\Cfootnote{
	Perhaps this should be emended further to 
	\textit{sarvopāyavistarair}
}}rvinīyā apratiṣṭhite nirvāṇadhātau pratiṣṭhāpayitavyāḥ~/
sarvasattvānapi ca parinirvāpya na kaścitsattvaḥ parinirvāpito bhavatītyevaṃ mayā cittamutpādayitavyam~/
\edtext{anutpādatā}{\lemma{%
	{\rm anutpādatā\lem}
}\Dfootnote{%
	\emn\ \MSK;
	\textit{anutpādato} \cod
}} mayā \edtext{sarvadharmāṇāma}{\lemma{
	{\rm sarvadharmāṇām\lem}
}\Dfootnote{
	\emn\ \MSK;
	\textit{sarvadharmmaṇām\lem} \cod
}}vaboddhavyā~/
avyavakīrṇena ca \edtext{sarvajñatājñānacittena}{\lemma{
	{\rm sarvajñatājñānacittena\lem}
}\Dfootnote{
	\emn\ \MSK;
	\textit{sarvajñātājñānacittena} \cod
}} ṣaṭsu pāramitāsu mayā śikṣitavyam~/
ekayānanayanirhāro mayāvaboddhavyaḥ~/
saptatriṃśadbodhipakṣyadharma\edtext{nirhāra}{\lemma{
	{\rm -nirhāra-\lem}
}\Dfootnote{
	\emn\ \MSK;
	\textit{nihārra} \cod
}}pratibodhāya mayā \edtext{śikṣitavyam}{\lemma{
	{\rm śikṣitavyam\lem}
}\Dfootnote{
	\cod;
	\textit{śikṣitavyaḥ} \cod
}\Cfootnote{
	No need to emend as is done by \MSK
}}~/
daśabalavaiśāradyapratisaṃvidāveṇikabuddhadharmapratibodhāya mayābhiyogaḥ kartavyaḥ~/
evaṃ yāvadaparimitasakalabuddhadharma\edtext{ni\-rhārāya}{\lemma{
	{\rm -nirhārāya\lem}
}\Dfootnote{
	\emn\ \MSK (silently);
	\textit{nihārāya} \cod
}} mayā pratipattavyam~//}

\quad
\Skt{%
ayamasau bodhisattvānāṃ vajropamo mahābodhicittotpādaḥ sarvatathāgatapitā
sarvatathāgatājñākāraḥ sarvatathāgata\edtext{jyeṣṭhaputraḥ}{\lemma{
	{\rm -jyeṣṭhaputro\lem}
}\Dfootnote{
	\emn; 
	\textit{jyeṣṭaputtraḥ} \cod;
	\textit{jyeṣṭaputro} \MSK
}}~/ 
ayamasau \edtext{bhagavānsamantabhadraḥ}{\lemma{}\Efootnote{
	bhagavā[33r1]n samantabhadraḥ
}}~/
tadyathāsaṃpāditakuśalamūla\edtext{balena}{\lemma{
	{\rm -balena\lem}
}\Dfootnote{
	\emn;
	\textit{-valinā} \cod\ \MSK
}} \edtext{mayātra}{\lemma{
	{\rm mayātra\lem}
}\Dfootnote{
	\emn;
	\textit{mayā tu} \cod;
	The emendation is supported by Tib.
}} cittotpāde vyavasthitenānantamukhanirhārasya \edtext{samantabhadracaryābhidhānasyāparya}{\lemma{
	{\rm samantabhadracaryābhidhānasyāparya\lem}
}\Dfootnote{
	\cod;
	\textit{samantabhadracayābhidhānasyāpi} \MSK
}}\edtext{vasthitaprāntasya niṣpādane}{\lemma{
	{\rm vasthitaprāntasya niṣpādane\lem}
}\Dfootnote{
	\conj;
	\textit{vasthita[prānta] + niṣpādane} \cod;
	\textit{vyavasthite prāntaniṣpādane} \MSK
}} \edtext{nikhilavineyajanarāśi}{\lemma{
	{\rm nikhilavineyajanarāśir\lem}
}\Dfootnote{
	\emn;
	\textit{nikhilavineyajanarāśi} \cod;
	\textit{nikhilavineyajanarāśiṃ} \MSK
}}rbahumukhaiśca\-ryā\edtext{prabheda}{\lemma{
	{\rm -prabheda-\lem}
}\Dfootnote{
	\emn\ \MSK 
}}saṃdarśanairāvarjya saṃpraharṣya \edtext{samuttejya}{\lemma{
	{\rm samuttejya\lem}
}\Dfootnote{
	\emn\ \MSK;
	\textit{samutejya} \cod
}} niyojanīya iti~//}
\pend


\pstart
\Skt{
tataḥ śiṣyahṛdaye samaya {\ah} iti \edtext{paṭhaṃścandramaṇḍalaṃ}{\lemma{
	{\rm candramaṇḍalaṃ\lem}
}\Dfootnote{
	\emn\ \MSK;
	\textit{candramaṇḍala}
}} nirmāya tasyopari pañcasūcikaṃ} 
\pend

\verse
\edtext{\Skt{vajramasya \edtext{pratiṣṭhāpyaṃ}{\lemma{
	{\rm pratiṣṭhāpyaṃ\lem}
}\Dfootnote{
	\cod;
	\textit{pratiṣṭhāpya} \MSK
}} hṛdaye hṛdayena tu~/\\
surate samayastvaṃ hoḥ vajra \edtext{siddhya}{\lemma{
	{\rm siddhya\lem}
}\Dfootnote{
	\emn\ \sil\ \MSK;
	\textit{sidhya} \cod
}} yathāsukhamiti~//}}{\lemma{
	{\rm vajram \dots\ yathāsukham\lem}
}\Bfootnote{
	\textit{Durgatipariśodhana}.
	identified by RT in SDPT, ed. Skorupski p. 288 (and Paramādya, but only in Chinese [but transcribed] and Tibetan)
	CHECK article by Shigetoshi Otsuka!
}}
\pend

\pstart
\noindent
\Skt{%
śirasi ca gandhodakapariplutaṃ \edtext{karaṃ}{\lemma{
	{\rm karaṃ\lem}
}\Dfootnote{
	\corr\ \MSK;
	\textit{kara} \cod
}} nyasya vajrasattvamā\edtext{vartayanvajraṃ}{\lemma{
	{\rm āvartayan vajraṃ\lem}
}\Dfootnote{
	\emn\ \MSK;
	\textit{āvarttayaṁ vajra} \cod
}} cintayet~/
\edtext{lalāṭe}{\lemma{
	{\rm lalāṭe\lem}
}\Dfootnote{
	\emn\ \MSK;
	\textit{lalaṭo} \cod
}} candrasyopari trāṃkāreṇa vajraratnam~/
\edtext{kaṇṭhe}{\lemma{
	{\rm kaṇṭhe\lem}
}\Dfootnote{
	\emn\ \MSK;
	\textit{kaṇṭho} \cod
}} hrīḥkāreṇa vajrapadmam~/
mūrdhni kaṃkāreṇa viśvavajram~/ vajrasattva vajraratna vajradharma vajrakarmeti
\edtext{uccārayet}{\lemma{
	{\rm uccārayet\lem}
}\Cfootnote{
	PDS suggests that this should be \textit{uccārayan}.
}}~/
\edtext{svahṛdayādi}{\lemma{
	{\rm svahṛdayādiṣv\lem}
}\Cfootnote{
	Emend \textit{svasvahṛdayādiṣv}?
}}ṣvadhitiṣṭhet~//}
\pend

\pstart
\quad
\Skt{\edtext{tato vajragandhayā}{\lemma{}\Efootnote{
	tato [33v1] vajragandhayā
}} teṣāṃ \edtext{kareṣu}{\lemma{
	{\rm kareṣu\lem}
}\Dfootnote{
	\corr\ \MSK;
	\textit{kāreṣu} \cod
}} gandhaṃ dadyāt~/
\edtext{puṣpayā}{\lemma{
	{\rm puṣpayā\lem}
}\Dfootnote{
	corr. \MSK;
	\textit{puṣpayāḥ} \cod
}} puṣpāṇi~/
dhūpayā tāndhūpayet~/
dīpayā dīpenābhāsayet~/}
%
\Skt{
vajrahāsa\edtext{parijaptaṃ}{\lemma{
	{\rm -parijaptaṃ\lem}
}\Dfootnote{
	\corr\ \MSK;
	\textit{-jāptam} \cod
}} dvādaśāṅgulapramāṇamu\edtext{dumbaramayamaśva\-ttha\-mayaṃ vā}{\lemma{
	{\rm dumbaramayam aśvatthamayaṃ vā\lem}
}\Dfootnote{
	\emn; 
	\textit{udumvaram aśvatthamayam vā} \cod (dittographical error).
	Or this should be emended to \textit{udumvaram aśvattham vā}.
}} gandhadigdhamagre kusumanibaddhaṃ \edtext{dantadhāvanaṃ}{\lemma{
	{\rm dantadhāvanaṃ\lem}
}\Dfootnote{
	\emn\ \MSK;
	\textit{dantadhāvana} \cod
}} dadyāt~/
taiśca prāṅmukhairu\edtext{daṅmukhai}{\lemma{
	{\rm udaṅmukhair\lem}
}\Dfootnote{
	\emn;
	\textit{udrammukhair} \cod;
	\textit{uttaramukhair} \MSK.
	\MSK emendation \textit{uttaramukhair} seems less likely to be correct. 
	Cf.\ \textit{Vajrāvalī}, ed. Sakura just after verse 20.
}\Cfootnote{
	\textit{drammu} of \textit{udrammukhair}: 
	the scribe corrected the \textit{akṣara}s.
}}rvā\-greṇaiva khādanīyam~//}
\pend

\pstart
\quad
\Skt{%
\edtext{vajratīkṣṇa}{\lemma{
	{\rm -tīkṣṇa-\lem}
}\Cfootnote{
	It seems the scribe started to write a long \textit{-ī} which would have made \textit{tīkṣṇī}, then added a diagonal cancellation stroke for the unwanted \textit{-ī} sign.
}}parijaptā\edtext{npratyagrānkuśāndattvaivaṃ}{\lemma{
	{\rm pratyagrān kuśān dattvaivaṃ\lem}
}\Dfootnote{
	\emn\ \MSK;
	\textit{pratyagroṅkuśān damvaivam} \cod
}} vadet~/}
\pend

\verse
\Skt{
\edtext{ebhi}{\lemma{
	{\rm ebhir\lem}
}\Dfootnote{
	\emn\ \MSK;
	\textit{debhir} \cod
}}\edtext{rāstāraṃ}{\lemma{
	{\rm āstāraṃ\lem}
}\Dfootnote{
	\emn\ \MSK;
	\textit{āstāra} \cod
}} \edtext{kṛtvaikaṃ}{\lemma{
	{\rm kṛtvaikaṃ\lem}
}\Cfootnote{
	\textit{tvai} of \textit{kṛtvaikaṃ} is slightly different from other \textit{akṣara tva}.
}} śirasi dattvā śayyā \edtext{kalpanīyeti}{\lemma{
	{\rm kalpanīyeti\lem}
}\Dfootnote{
	\emn;
	\textit{kalpanīye kalpanīyeti} \cod (dittographical error?)
}}~//}
\pend


\pstart
\quad
\Skt{vajrarakṣābhijaptaṃ} 
\pend

\verse
\edtext{\Skt{tataḥ samya\edtext{ṅnirbadhnīyā}{\lemma{
	{\rm nirbadhnīyād\lem}
}\Dfootnote{
	\emn\ \MSK;
	\textit{nirdhadhnīyād} \cod
}}\edtext{dvāmapāṇau}{\lemma{
	{\rm vāmapāṇau\lem}
}\Dfootnote{
	\emn\ \MSK;
	\textit{vāmapāṇai} \cod
}} tu sūtrakam~/\\
granthibhiḥ samupetaṃ vai \edtext{tisṛbhiḥ}{\lemma{
	{\rm tisṛbhiḥ\lem}
}\Dfootnote{
	\cod;
	\textit{tribhiḥ} \MSK
}} svayameva tu~//}}{\lemma{
	{\rm tataḥ \dots\ svayam eva tu\lem}
}\Bfootnote{
	Not traced.
	Probably Ānandagarbha uses a scriptural source or a predecessor's work silently as seen elsewhere in SVU.
}}
\pend

\pstart
\quad
\Skt{
tato \edtext{yathāśaktyā}{\lemma{
	{\rm yathāśaktyā\lem}
}\Dfootnote{
	\emn\ \MSK;
	\textit{yathāśakyā} \cod
}} gambhīrodāradharmadeśanayā \edtext{sarvā}{\lemma{
	{\rm sarvān\lem}
}\Dfootnote{
	corr. \MSK;
	\textit{sarvāṁ} \cod
}}nsamuttejya saṃpraharṣya vadet~/}
\pend

\verse
\Skt{
\edtext{uttiṣṭhata}{\lemma{
	{\rm uttiṣṭhata\lem}
}\Dfootnote{
	\emn\ \MSK;
	\textit{utiṣṭhata} \cod
}} bhadramukhāḥ \edtext{śvo}{\lemma{
	{\rm śvo\lem}
}\Dfootnote{
	corr. \MSK;
	\textit{svo} \cod
}} mahāmaṇḍalaṃ drakṣyatheti~//}
\pend


\medskip

\pstart
\mbox{}\hfill \Skt{%
// adhivāsanavidhiḥ~//}\hfill \mbox{}
\pend


\bigskip

% \input{07_Sutrana.tex}

\pstart\noindent
{\large 3.4. Sūtraṇavidhiḥ}
\pend

\bigskip

\pstart
\edtext{}{\lemma{%
	tataḥ sūryodayakāle \dots\ aiśānīṃ diśam ārabhya sūtrayet
	(just before the quotation from the STTS etc.)\lem
}\Bfootnote{%
	parallel to \textit{Tattvālokakarī} (P LOCATION!!, D f.111r3–112r4)
	%Endo pp.197–199
}}%
\Skt{tataḥ \edtext{sūryodayakāle}{\lemma{
	{\rm sūryodākāle\lem}
}\Dfootnote{
	\emn\ \MSK\ \sil;
	\textit{sūyodayakāle}  \cod
}} tathaiva tanmaṇḍala\edtext{muparyākāśe}{\lemma{
	{\rm uparyākāśe\lem}
}\Dfootnote{
	\emn\ \MSK;
	\textit{upayākāśe}  \cod
}} nyasya sūtrayet / 
tatrādau tāvannīlapītarakta\edtext{haritasita}{\lemma{}\Efootnote{
	-harita[34r1]sita-
}}sugandhavarṇakaiḥ \edtext{pṛthak pṛthak sūtrāṇi}{\lemma{
	{\rm pṛthak pṛthak sūtrāṇi\lem}
}\Cfootnote{
	First the scribe wrote different \textit{akṣara}s and then corrected them.
}} \edtext{rañjitānya}{\lemma{
	{\rm rañjitāny\lem}
}\Dfootnote{
	\emn\ \MSK;
	\textit{rañjatāny} \cod
}}nayaiva \edtext{paripāṭyāvasthāpya}{\lemma{
	{\rm paripāṭyāvasthāpya\lem}
}\Cfootnote{
	any connection with §50 \textit{avaṣṭabhya}?
}}}
%
\Skt{hū{\cb} \edtext{traḥ}{\lemma{
	{\rm traḥ\lem}
}\Dfootnote{
	\cod;
	\textit{trāḥ} \MSK
}\Cfootnote{
	There is an unusual blotch to the left of the \textit{akṣara}, 
	but it cannot be read as \textit{-ā}. In §68 we seem to have the same mantra. 
	Emend \textit{trāḥ}?
}} hrīḥ {\ah} āḥ
ityebhiḥ svabījairakṣobhyādīn sūtreṣu nyasya
\edtext{dīptadṛṣṭyāṅkuśi}{\lemma{
	{\rm dīptadṛṣṭyāṅkuśi\lem}
}\Dfootnote{
	\emn\ \MSK;
	\textit{dīptadṛṣṭhāṁkuśi} \cod
}} \edtext{jaḥ}{\lemma{
	{\rm jaḥ\lem}
}\Dfootnote{
	\emn\ \MSK;
	\textit{jāḥ} \cod
}}}
\Skt{iti \edtext{cakṣurdvaye}{\lemma{
	{\rm cakṣurdvaye\lem}
}\Dfootnote{
	\emn\ \MSK;
	\textit{cakṣudvaye} \cod
}} nyasya}
\Skt{\edtext{pradruta\edtext{pracara}{\lemma{%
	{\rm -pracalac-\lem}
}\Dfootnote{
	\emn;
	\textit{-prajvalac-} \cod\ \MSK
}}ccakṣuḥpakṣmākarṣaṇalocanādṛṣṭyākṣobhyādī}{\lemma{
	{\rm pradruta\dots kṣobhyādīn\lem}
}\Dfootnote{
	\emn\ \MSK;
	\textit{pradrutaprajvalaccakṣuḥpadmākarṣaṇalocayādṛṣṭhyākṣobhyādī} \cod.
	Cf. STTS ed. Hor. §367: see Tanemura 2004: 141-2, 240-241.
}}\-nsaṃcodayedida\edtext{muccārayan}{\lemma{
	{\rm uccārayan\lem}
}\Dfootnote{
	\emn\ \MSK;
	\textit{uccārayam} \cod
}}}
\pend

\verse
\Skt{vajrasūtraṃ me \edtext{bhagavā}{\lemma{
	{\rm bhagavān\lem}
}\Dfootnote{
	\emn;
	\textit{bhagavāṁta} \cod
}\Cfootnote{
	Or  emend \textit{bhagavan}?
}}\edtext{nprayacchatu}{\lemma{
	{\rm prayacchatu\lem}
}\Cfootnote{
	emend \textit{prayaccha}?
}} mahāmaṇḍalasūtraṇāyeti /}
\pend

\pstart
\Skt{tataḥ \edtext{svabījaraśmisūtraira}{\lemma{
	{\rm svabījaraśmisūtrair\lem}
}\Dfootnote{
	\emn\ \MSK\ \sil;
	\textit{svavījārasmisūtrair} \cod
}}kṣobhyādi\edtext{saṃpreṣitai}{\lemma{
	{\rm -saṁpreṣitair\lem}
}\Dfootnote{
	\emn\ \MSK;
	\textit{sampreṣitai} \cod
}}rdīptāṅkuśākṛṣṭasvahastasaṃsthitaiḥ tāni sūtrāṇi saṃpūrya~/}
\pend

\verse
\Skt{%
\edtext{anyonyānugatāḥ sarvadharmāḥ 
\edtext{parasparānupraviṣṭāḥ}{\lemma{
	{\rm parasparānupraviṣṭāḥ\lem}
}\Dfootnote{
	\corr;
	\textit{parasparanupraviṣṭhāḥ} \cod
}} sarvadharmā
\edtext{atyantānupraviṣṭāḥ}{\lemma{
	{\rm atyantānupraviṣṭāḥ\lem}
}\Dfootnote{
	\corr;
	\textit{atyantānupraviṣṭhāḥ} \cod
}} sarvadharmā}{\lemma{
	{\rm anyonyānugatāḥ \dots\ atyantānupraviṣṭāḥ sarvadharmā\lem}
}\Bfootnote{
	cf. \textit{Sūtrapātanam} section of KSP (3.5.4), ed. Tanemura 2004.
}}}
\Skt{{\om} vajrasattva hū{\cb} iti //}
\pend

\pstart
\noindent
\Skt{%
\edtext{codāharan}{\lemma{
	{\rm caodāharan\lem}
}\Dfootnote{
	\emn\ \MSK;
	\textit{codāharam} \MSK
}} \edtext{\textcolor{red}{valaye saṃvartya}}{\lemma{
	{\rm valaye saṁvartya\lem}
}\Cfootnote{
	perhaps correct \textit{valayet}, as seen in KSP? 
	And then delete \textit{saṁvartya}? 
	Or emend \textit{valayitvā}, after \textit{Padminī}.
	Or emend \textit{valayya}?
	Tib. reads \textit{bsgril bar bya’o// bsgril nas} (\textit{*valayet | saṁvartya}).
}} \edtext{maṭakārābhyāṃ}{\lemma{
	{\rm maṭakārābhyāṁ\lem}
}\Dfootnote{
	\emn\ \MSK\ \sil;
	\textit{maṭkārābhyāṁ} \cod
}} dakṣiṇetaracakṣuṣoḥ \edtext{saurya}{\lemma{
	{\rm saurya-\lem}
}\Dfootnote{
	\emn\ \MSK\ \sil;
	\textit{sauyā-} \cod
}}candramasau niṣpādya /}
\Skt{% 
vajradṛṣṭi \edtext{maḍiti}{\lemma{
	{\rm maṭ iti\lem}
}\Dfootnote{
	\emn\ \MSK;
	\textit{mat iti} \cod
}} sthirīkṛtya gandhopaliptahemabhājanādau tatsūtraṃ saṃsthāpya gandhapuṣpadhūpaiḥ saṃpūjya /}
\Skt{{\om} vajrasamayasūtraṃ mātikrama \edtext{hū{\cb} iti}{\lemma{
	{\rm hūṁ iti\lem}
}\Dfootnote{
	\corr\ \MSK\ \sil;
	\textit{hū;m miti} \cod
}} /}
\Skt{%
\edtext{vajrabandhenāvaṣṭabhyāṣṭottara}{\lemma{
	{\rm vajrabandhenāvaṣṭabhyāṣṭottara-\lem}
}\Dfootnote{
	\cod;
	\textit{vajrabandhena viṣṭabhyāṣṭottara} \MSK
}}śataṃ ca \edtext{parijapet}{\lemma{
	{\rm parijapet\lem}
}\Dfootnote{
	\emn\ \MSK (misreading \textit{parijaptet});
	\textit{parijāptet} \cod
}} //}


\Skt{
tataḥ ṣoḍaśasattvaiḥ puna\edtext{rmaṇḍalabhūmiṃ gandhenopalipya}{\lemma{
	{\rm maṇḍalabhūmiṁ gandhenopalipya\lem}
}\Dfootnote{
	\emn\ \MSK;
	\textit{mmaṇḍalabhūmīnnandenopalipya} \cod
}\lemma{}\Efootnote{
	maṇḍalabhūmiṁ [34v1] gandhenopalipya
}} bahiḥ samantataḥ \edtext{puṣpeṇopaśobhyāntarikṣāvasthita}{\lemma{
	{\rm puṣpeṇopaśobhyāntarikṣāvasthita-\lem}
}\Dfootnote{
	\corr\ \MSK\ (silently emending \textit{s} to \textit{ś}, but wrongly keeping \textit{ṇ});
	\textit{puṣpeṇopasobhyāntarikṣāvasthita-} \cod
}}sarvakulebhyo'rghaṃ dattvā puṣpādibhi\edtext{rlāsyādibhi}{\lemma{
	{\rm lāsyādibhiś\lem}
}\Dfootnote{
	\emn\ \MSK\ \sil;
	\textit{lāsyādis} \cod
}}śca sarvāṇi saṃpūjya praṇamya jaḥkārābhyāṃ vāmetaracakṣuṣo\edtext{ścandrasūryau}{\lemma{
	{\rm candrasūryau\lem}
}\Dfootnote{
	\emn\ \MSK\ \sil;
	\textit{candrasūyau} \cod
}} pītābhau niṣpādya \edtext{gītaśabdopahāraiḥ}{\lemma{
	{\rm gītaśabdopahāraiḥ\lem}
}\Dfootnote{
	\emn\ \MSK\ \sil;
	\textit{gīteśavdopahāraiḥ} \cod
}} saha maṇḍalaṃ kalpayet //}


\Skt{%
tatra tāva\edtext{tpūrvābhimantritavarṇakena}{\lemma{
	{\rm pūrvābhimantritavarṇakena\lem}
}\Dfootnote{
	\cod\ (\textit{pūrvvābhimantritavarṇṇakena});
	\textit{pūrvābhimantritaṃ varṇakena} \emn\ \MSK\ \sil;.
	We suspect it is better to keep the sequence as a single compound.
}} sugandhaśuklena sūtraṃ \edtext{tīmayet}{\lemma{
	{\rm tīmayet\lem}
}\Dfootnote{
	\emn\ \MSK;
	\textit{temayet} \cod
}} /}
\Skt{%
tato manasaiva vajradhātviti \edtext{vajravācoccārayan}{\lemma{
	{\rm vajravācoccārayan\lem}
}\Dfootnote{
	\emn\ \MSK;
	\textit{vajravāccocārayaṁ} \cod
}} vairocanībhūya svahṛdayādvajrakarmetyudāharan vajrakarmarūpamuttarasādhakaṃ nirmāya /}
\Skt{%
tato \edtext{vajrasattvamudāhara}{\lemma{
	{\rm vajrasattvam udāharan\lem}
}\Dfootnote{
	\emn;
	\textit{vajrasatvam udāharan} \cod;
	\textit{vajrasattvodāharan} \MSK\ (emendation not clearly indicated)
}}nvajrasattvamātmānaṃ \edtext{\textcolor{red}{niścintya}}{\lemma{%
	{\rm niścintya\lem}
}\Cfootnote{%
	Emend \textit{niścitya}?
}} \edtext{tatsūtraṃ}{\lemma{
	{\rm tat sūtraṁ\lem}
}\Cfootnote{
	\textit{tatsūtraṁ} \MSK; 
	This should be separate words.
}} vāmavajramuṣṭyādāya \edtext{jaḥkāreṇa}{\lemma{
	{\rm jaḥkāreṇa\lem}
}\Dfootnote{
	\cod$^{pc}$ \MSK;
	\textit{jaḥkāreṇā} \cod$^{ac}$
}\Cfootnote{
	correct \textit{jjaḥ-} as in STTS and KSP? 
	but see spelling \textit{jaḥ-} e.g. in Javanese inscriptions.
}} \edtext{taṃ vajrakarmarūpa}{\lemma{
	{\rm taṃ vajrakarmarūpam\lem}
}\Dfootnote{
	\emn;
	\textit{tadvajrakarmarūpam} \cod\ \MSK
}}muttarasādhakaṃ sūtrahastaṃ \edtext{saṃpreṣya}{\lemma{
	{\rm saṁpreṣya\lem}
}\Dfootnote{
	\emn\ \MSK;
	\textit{sampraveṣya} \cod
}} /}
\Skt{%
sarvadiksamatāṃ \edtext{vāhayan}{\lemma{
	{\rm vāhayan\lem}
}\Dfootnote{
	\conj;
	\textit{bhāvayat} \cod:
	\textit{bhāvayan} \MSK
}} pūrvābhimukho \edtext{brahmasūtraṃ}{\lemma{
	{\rm brahmasūtraṁ\lem}
}\Dfootnote{
	\cod\ (\textit{vrahmasūtraṁ});
	\textit{vajrasūtraṃ} \MSK\ (\emn\ following Tib.)
}} \edtext{pātayet}{\lemma{
	{\rm pātayet\lem}
}\Dfootnote{
	\cod;
	\textit{pāṭayet} \MSK
}} /}
\Skt{%
\edtext{puna}{\lemma{
	{\rm punar\lem}
}\Dfootnote{
	\emn\ \MSK;
	\textit{puna} \cod
}}\edtext{ryāmyāyāṃ}{\lemma{
	{\rm yāmyāyāṁ\lem}
}\Dfootnote{
	\emn\ \MSK;
	\textit{yāmyāṁ} \cod
}} sthita \edtext{ācārya}{\lemma{
	{\rm ācārya\lem}
}\Dfootnote{
	\emn\ \MSK\ \sil;
	\textit{ācāya} \cod
}} uttarābhimukho dvitīyam /}
\Skt{%
\edtext{tata}{\lemma{
	{\rm tata\lem}
}\Dfootnote{
	\emn\ \MSK\ \sil;
	tataḥ \cod
}} āgneyadigbhāge sthitvottarābhimukho \edtext{bāhyamaṇḍalapūrvasūtram}{\lemma{
	{\rm bāhyamaṇḍalapūrvasūtram\lem}
}\Dfootnote{
	\emn\ \MSK;
	\textit{vāhyamaṇḍapūrvvasūtraṁ} \cod
}} /}
\Skt{
vāyavye sthito dakṣiṇābhimukhaḥ \edtext{paścimasūtram}{\lemma{
	{\rm paścimasūtram\lem}
}\Dfootnote{
	\emn\ \MSK;
	\textit{paścimasūtraṁḥ} \cod
}} /}
\Skt{
tatraiva \edtext{sthita}{\lemma{
	{\rm sthita\lem}
}\Dfootnote{
	\emn\ \MSK\ \sil;
	\textit{sthitaṁ} \cod
}} uttarasūtram /}
\Skt{%
punarāgneyabhāge sthito dakṣiṇapārśvasūtram / \edtext{evaṃ}{\lemma{
	{\rm evaṁ\lem}
}\Dfootnote{
	\emn;
	\textit{eva} \cod\ \MSK
}} \edtext{vedikāsūtraṇaṃ}{\lemma{
	{\rm vedikāsūtraṇaṁ\lem}
}\Dfootnote{
	\cod\ (\textit{vedikāsūttraṇañ})
	\textit{vedikāsūtraṁ} \MSK
}} ca /}
\Skt{%
\edtext{punarā}{\lemma{}\Efootnote{%
	ca | [35r1] punar
}}gneyakoṇe sthito vāyavyakoṇaṃ yāvat /}
\Skt{
\edtext{nairṛte}{\lemma{
	{\rm nairṛte\lem}
}\Dfootnote{
	\emn\ \MSK;
	\textit{nairite} \cod
}} sthita aiśānakoṇaṃ yāva\edtext{tpātayet}{\lemma{
	{\rm pātayet\lem}
}\Dfootnote{
	\cod;
	\textit{pāṭayet} \MSK
}} / 
pradakṣiṇayogena \edtext{cācāryottarasādhakayo}{\lemma{
	{\rm cācāryottarasādhakayor\lem}
}\Dfootnote{
	\emn\ \MSK\ \sil;
	\textit{cācāryottarasādhakayor} \cod
}}rgamanam /
\edtext{tatrāyaṃ sūtraṇa}{\lemma{
	{\rm tatrāyaṁ sūtraṇa-\lem}
}\Dfootnote{
	\emn\ \MSK\ \sil;
	\textit{tatrāyatsūtreṇa-} \cod
}}mantraḥ /}
\pend

\verse
\Skt{%
{\om} \edtext{vajrasamayasūtraṃ}{\lemma{
	{\rm vajrasamayasūtraṁ\lem}
}\Dfootnote{
	\emn;
	\textit{vajrasamayasūtra} \cod
	\textit{vajrasamaya sūtraṁ}  \MSK
}} mātikrama hū{\cb} iti //}
\pend

\pstart
\noindent
\Skt{%
jaḥ jaḥ jaḥ 
iti \edtext{tribhi}{\lemma{
	{\rm tribhir\lem}
}\Dfootnote{
	\emn\ \MSK\ \sil;
	\textit{tṛbhi} \cod
}}rjaḥkārairuttarasādhakena sūtrākarṣaṇaṃ \edtext{kāryam /}{\lemma{
	{\rm kāryam |\lem}
}\Dfootnote{
	\emn\ \MSK\ \sil\ (\textit{kāryaṁ});
	\textit{kāyāṁ} \cod
}}}


\Skt{dvārāṇi cāṣṭamabhāgikāni /}
\Skt{dvārapramāṇā \edtext{dvāraniryūhāḥ}{\lemma{
	{\rm dvāraniryūhāḥ\lem}
}\Dfootnote{
	\emn\ \MSK\ \sil;
	\textit{dvāraniryūhāḥ} \cod
}} /}
\Skt{
\edtext{sarvābhayantarā sarvabāhyā ca}{\lemma{
	{\rm sarvābhayantarā sarvabāhyā ca\lem}
}\Dfootnote{
	\emn\ \MSK\ \sil;
	\textit{sarvābhayantaraḥ sarvavāhyaś ca} \cod
}\Cfootnote{
	emend \textit{sarvābhayantarāḥ sarvavāhyāś ca} |? 
	\MSK\ (silently) emends \textit{sarvābhayantarā sarvabāhyā ca}. 
	The latter seems to be supported by \textit{sabāhyābhyantaravedikāyāṁ} towards end of §54.
}} \edtext{dvārārdhapramāṇā}{\lemma{
	{\rm dvārārdhapramāṇā\lem}
}\Dfootnote{
	\emn\ \MSK\ \sil;
	\textit{dvārarddhapramāṇā}
}} vedikā /
vedikārdhapramāṇā pañcaraṅgikarajobhūmiḥ /
toraṇaṃ \edtext{dvāratriguṇaṃ kāryam}{\lemma{
	{\rm dvāratriguṇaṃ kāryam\lem}
}\Dfootnote{
	\emn\ \MSK\ \sil;
	\textit{dvāratṛguṇaṁ kāyam} \cod
}} /}
\Skt{
bāhyamaṇḍalārdhenābhyantaramaṇḍalaṃ} 
\pend
\verse
\Skt{\edtext{caturasraṃ \edtext{caturdvāraṃ vedikāparivāritam}{\lemma{
	{\rm caturdvāraṁ vedikāparivāritam\lem}
}\Dfootnote{
	\cod\ (\textit{caturasrañ caturdvāraṁ});
	\textit{caturdvāravedikāparivārtam} \MSK.
	\MSK\ silently emends \textit{caturdvāravedikā-}, 
	but such a compound is not expected here.
}}}{\lemma{
	{\rm caturasraṁ caturdvāraṁ vedikāparivārtam\lem}
}\Cfootnote{
	taken from some scriptural (?) source.
}}}
\pend

\pstart\noindent
\Skt{%
aṣṭastambhayuktaṃ ca saṃsūtrya \edtext{maṇḍalanābhau \edtext{khādira}{\lemma{%
	{\rm khādiram\lem}
}\Dfootnote{
	\emn;
	\textit{khadiram} \cod\ \MSK
}}\edtext{madhaekasūcikavajrākāra}{\lemma{
	{\rm adhaekasūcikavajrākāram\lem}
}\Dfootnote{
	\emn;
	\textit{adhaekasūcikavajrākārem} \cod;
	\textit{adha ekasūcikavajrākāram} \MSK
}}\edtext{muparipañcasū\-cikavajrākāraṃ}{\lemma{
	{\rm uparipañcasūcikavajrākāraṁ\lem}
}\Cfootnote{
	\MSK\ reads \textit{upari pañcasūcikavajrākāraṁ},
	but this should be a compound.
}} kīlakaṃ}{\lemma{%
	{\rm maṇḍalanābhau \dots\ kīlakaṁ\lem}
}\Bfootnote{
	VA \S\ 12.2.7:
	\textit{maṇḍalanābhau khādiram *adhaekasūcikam} (\emn; \textit{adha ekasūcikam} ed.) \textit{*uparipañcasūcikavajrākāraṃ} (\emn; \textit{upari pañcasūcikavajrākāraṃ} ed.) \textit{kīlakaṃ} \dots
}} pūrvavadvajrakīlamantreṇa parijapya 
\edtext{vajreṇākoṭya}{\lemma{
	{\rm vajreṇākoṭya\lem}
}\Dfootnote{
	\emn\ \MSK;
	\textit{vajreṇākoṭyā} \cod
}} 
\edtext{tadavasakta}{\lemma{
	{\rm tadavasakta-\lem}
}\Dfootnote{
	\emn\ \MSK\ \sil;
	\textit{tadavaśakta-} \cod
}}sūtreṇā\-bhyantaramaṇḍalabāhyato \edtext{vajramālāṃ}{\lemma{
	{\rm vajramālāṁ\lem}
}\Dfootnote{
	\emn \MSK\ \sil;
	\textit{vajramalām} \cod
}} \edtext{pradakṣiṇakrameṇaiśānīṃ}{\lemma{
	{\rm pradakṣiṇakrameṇaiśānīṁ\lem}
}\Dfootnote{
	\emn;
	\textit{prakṣi$\sqcup$ṇakrameṇaiśānīn} \cod;
	\textit{pradakṣiṇakrameṇaiṣānīṁ} \MSK;
	\MSK\ silently and needlessly emend to \textit{krameṇaiṣānīṁ}.
}} \edtext{diśa}{\lemma{
	{\rm diśam\lem}
}\Dfootnote{
	\emn\ \MSK;
	\textit{diśim} \cod
}}mārabhya sūtrayet /
maṇḍalapramāṇena ca dviguṇaṃ \edtext{ṣaḍḍhaste}{\lemma{
	{\rm ṣaḍḍhaste\lem}
}\Dfootnote{
	\emn\ \MSK\ \sil;
	\textit{ṣatṛste} \cod
}} \edtext{\mbox{}\edtext{parimāṇḍalyena}{\lemma{
	{\rm parimāṇḍalyena\lem}
}\Dfootnote{	
	\emn;
	\textit{pārimaṇḍalyena} \cod;
	\textit{pārimāṇḍalyena} \emn\ \MSK\ \sil;
t}} \edtext{kanīyasī}{\lemma{
	{\rm kanīyasī-\lem}
}\Dfootnote{
	\cod;
	\textit{kanīyasā} \emn\ \MSK
}}pramāṇaṃ sūtraṃ \edtext{kārya}{\lemma{
	{\rm kāryam\lem}
}\Dfootnote{
	\emn\ \MSK\ \sil;
	\textit{kāyam} \cod
}\lemma{}\Efootnote{%
	kā[35v1]ryam
}}\edtext{manyatrānurūpam}{\lemma{
	{\rm anyatrānurūpam\lem}
}\Dfootnote{
	\emn\ \MSK\ \sil;
	\textit{anyatrānurūpa} \cod
}}}{\lemma{%
	{\rm parimāṇḍalyena \dots\ anyatrānurūpam\lem}
}\Bfootnote{
	VA \S\ 12.1.3:
	\textit{pārimaṇḍalyena ṣaḍḍhastamaṇḍale kanīyasī\-pramāṇaṃ sūtraṃ kāryam anyatrānurūpata ity ānandagarbha-ādayaḥ} (ed. Mori, \textcolor{red}{p.143}).
}}~//}
\pend

\smallskip

\pstart
% \MSK: (54) 
\Skt{idānīmārṣoktyā \edtext{sūtraṇa}{\lemma{
	{\rm sūtraṇam\lem}
}\Dfootnote{
	\emn;
	\textit{sūtreṇam} \cod;
	\textit{sūtralakṣaṇam} \MSK;
	\MSK's emendation is based on Tib.
}}mucyate /}
\pend

\verse
\Skt{%
\edtext{navena suniyuktena supramāṇena cāruṇā /\\
sūtreṇa sūtrayetprājño yathāśaktyā \edtext{sumaṇḍalam}{\lemma{
	{\rm sumaṇḍalam\lem}
}\Dfootnote{
	\emn\ following STTS;
	\textit{nu maṇḍalam} \cod\ \MSK
}\Cfootnote{
	\textit{nu}: emend \textit{tu}, with \MSK? 
	Or keep \textit{anumaṇḍalam}? 
	Or read with STTS \textit{sumaṇḍalam}?%
}} //  \\
\edtext{caturasraṃ}{\lemma{
	{\rm caturasraṁ\lem}
}\Dfootnote{
	\emn\ \MSK\ \sil;
	\textit{caturaśraṁ} \cod
}} caturdvāraṃ catustoraṇa\edtext{śobhitam}{\lemma{
	{\rm -śobhitam\lem}
}\Dfootnote{
	\emn\ \MSK\ \sil;
	\textit{-sobhitam} \cod
}} / \\
catuḥsūtrasamāyuktaṃ paṭṭa\edtext{sragdāma}{\lemma{
	{\rm -sragdāma-\lem}
}\Dfootnote{
	\emn\ \MSK\ \sil;
	\textit{-śragdāma-} \cod
}}bhūṣitam // \\
%
koṇabhāgeṣu sarveṣu dvāra\edtext{niryūha}{\lemma{
	{\rm -niryūha-\lem}
}\Dfootnote{
	\emn\ \MSK;
	\textit{-niyūha-} \cod
}}saṃdhiṣu /\\
khacitaṃ vajraratnaistu \edtext{sūtraye}{\lemma{
	{\rm sūtrayed\lem}
}\Dfootnote{
	\emn\ \MSK;
	\textit{sūtraye} \cod
}}dbāhyamaṇḍalam //  \\
%
tasya cakrapratīkāśaṃ praviśyābhyantaraṃ puram /\\
vajrasūtra\edtext{parikṣipta}{\lemma{
	{\rm -kṣiptam\lem}
}\Dfootnote{
	\emn\ \MSK;
	\textit{-kṣipta} \cod
}}maṣṭastambhopa\edtext{śobhitam}{\lemma{
	{\rm -śobhitam\lem}
}\Dfootnote{
	\emn\ \MSK\ \sil;
	\textit{-sobhitam} \cod
}} // \\
%
maṇḍalaṃ \edtext{sūtraye}{\lemma{
	{\rm sūtrayet\lem}
}\Dfootnote{
	\emn\ \MSK;
	\textit{sūtrayet} \cod
}}tprājñaḥ sarvadiksamatāṃ \edtext{vahan}{\lemma{
	{\rm vahan\lem}
}\Dfootnote{
	\emn\ \MSK;
	\textit{vaham} \cod
}} /\\
prākpratīcyuttarāvāgdikcatuḥsūtrāṣṭamaṇḍalamiti}{\lemma{%
	{\rm navena \dots\ -maṇḍalam iti\lem}
}\Bfootnote{%
	The first four stanzas = STTS \S\S\ 203(4), 204(5), 204(6), and 204(7).
	The final stanza = \textit{Samāyoga} 7.14
	\textit{maṇḍalaṃ sūtrayet prājñaḥ sarvadiksamatāṃ vahan |
prākpratīcyuttarāvāgdik catuḥsūtrāṣṭamaṇḍalam ||} =
	\textcolor{red}{\textit{Paramādya} D 264a, \textit{Vajramaṇḍālaṃkāra} D80b CHECK!!}
}} // 
}
\pend


\pstart
\Skt{sabāhyābhyantaravedikāyāṃ ca hārārdhahāra\edtext{cāmara}{\lemma{
	{\rm -cāmara-\lem}
}\Dfootnote{%
	\emn;
	\textit{-camara-} \cod\ \MSK
}}candra\edtext{sūrya}{\lemma{
	{\rm -sūrya-\lem}
}\Dfootnote{
	\emn\ \MSK;
	\textit{sūya-} \cod
}}maṇḍalāni lekhyāni /
%
sarvamaṇḍalakoṇeṣu mārutoddhūta\edtext{ghaṇṭā}{\lemma{
	{\rm -ghaṇṭā-\lem}
}\Dfootnote{
	\emn\ \MSK;
	\textit{-ghaṇṭhā-} \cod.
	Or \cod reads \textit{-ghaṇṭā-} ?
}}vasaktacitrāgrapatākāḥ /
%
teṣu dvāra\edtext{niryūha}{\lemma{
	{\rm -niryūha-\lem}
}\Dfootnote{
	\emn\ \MSK;
	\textit{-niyūha-} \cod
}}saṃdhiṣu cārdhacandra\edtext{pratiṣṭhāne vajraratnaṃ karojjvalaṃ}{\lemma{
	{\rm pratiṣṭhāne vajraratnaṁ karojjvalam\lem}
}\Dfootnote{
	\emn;
	\textit{-pratiṣṭhāne vajraratnakarojvāl\{ā\}aṁ} \cod;
	\textit{-pratiṣṭhānavajraratnakarajvālā} \MSK
}\Cfootnote{
	\textit{-pratiṣṭhāne vajra-}: \MSK silently emend \textit{-pratiṣṭhānavajra-}. 
	Emend \textit{-pratiṣṭhāneṣu} or \textit{pratiṣṭhānaṁ}?
	\textit{-karojvāl\{ā\}aṁ}: \MSK emends \textit{-karajvālā}. 
	We propose \textit{-karojjvalam}.
}} \edtext{lekhyami}{\lemma{
	{\rm lekhyam\lem}
}\Dfootnote{
	\cod;
	\textit{lekhyāḥ} \MSK
}}ti //}
\pend

\bigskip

% \input{08_Vighnanivarana}

\pstart\noindent
{\large 3.5. Vighnanivāraṇavidhiḥ}
\pend

\bigskip

\pstart\noindent
{\large 3.5.1. Vighnanivāraṇa}
\pend

\bigskip

\pstart
%\MSK\ (55) 
\Skt{sacedvighnanivāraṇaṃ kartukāmo bhavati~/
lakṣajaptena kīlamantreṇāṣṭottaraśatābhimantritairmaṇḍalanābhistha\edtext{nirdiṣṭa}{\lemma{
	{\rm -nirdiṣṭa-\lem}
}\Dfootnote{
	\emn\ \MSK\ \sil;
	\textit{-nirddiṣṭha-} \cod
}}kīlakasadṛśaiḥ kīlakaiḥ \edtext{sarvavighnā}{\lemma{
	{\rm sarvavighnān\lem}
}\Dfootnote{
	\MSK;
	\textit{sarvavighnā;m |} \cod
}\Cfootnote{
	\textit{sarvavighnā;m |}: understand \textit{-ghnān} (without punctuation); 
	\MSK\ reads this, but wrongly reports the reading as \textit{-ghnāḥ}.
}}\edtext{nkīlaye}{\lemma{}\Efootnote{%
	kīla[36r1]yed
}}\edtext{danayānupūrvyā}{\lemma{
	{\rm anayānupūrvyā\lem}
}\Cfootnote{
	\MSK\ inserts punctuation after this, but it seems unnecessary to us.
}}~/ yathoktamaṇḍalanābhisthakīlakāvasakta\edtext{sūtreṇa}{\lemma{
	{\rm -sūtreṇa\lem}
}\Cfootnote{
	\textit{-sū(tre)ṇa} \cod. 
	the \textit{tra} is written somewhat low and detaches from preceding and following \textit{akṣara}s. 
	Possibly a secondary insertion into a space previously left blank?
}} pradakṣiṇato bāhyamaṇḍalasya vartulaṃ prabhāmaṇḍalaṃ saṃsūtrya tasya \edtext{bahi}{\lemma{
	{\rm bahiś\lem}
}\Dfootnote{
	\emn\ \MSK;
	\textit{vahis} \cod
}}ścakravāḍaṃ tathaiva vartulaṃ saṃsūtrayet /}

%\MSK: 
\Skt{tatra prabhāmaṇḍale valmīkamṛttikayā devādivighnapratikṛtiṃ kṛtvā \edtext{śakrapratikṛtau}{\lemma{
	{\rm śakrapratikṛtau\lem}
}\Dfootnote{
	\corr\ \MSK\ \sil;
	\textit{sakrapratikṛtau} \cod
}} sva\-di\edtext{ksthitāyāṃ}{\lemma{
	{\rm ksthitayāṁ\lem}
}\Dfootnote{
	\emn;
	\textit{sthityāṁ} \cod;
	\textcolor{red}{CHECK} \MSK
}}\edtext{mīkāreṇa}{\lemma{
	{\rm īkāreṇa\lem}
}\Dfootnote{
	\cod;
	\textit{ikāreṇa} \MSK
}} \edtext{śakraṃ}{\lemma{
	{\rm śakraṁ\lem}
}\Dfootnote{
	\corr\ \MSK\ \sil;
	\textit{sakraṁ}  \cod
}} pītam~/
evamagnyādiprakṛtau \edtext{raṃkāreṇa}{\lemma{
	{\rm raṁkāreṇa\lem}
}\Dfootnote{
	\emn\ \MSK;
	\textit{ruṅkāreṇa} \cod
}} raktamagnim~/ 
ahamiti yamaṃ kṛṣṇam~/ 
kramiti rākṣasaṃ kṛṣṇam~/ 
vamiti varuṇaṃ śuklam~/ 
yamiti vāyuṃ dhūmram~/ 
phaḍiti kuberaṃ pītam~/ 
sumbhetīśānaṃ śuklaṃ niṣpādya vajrāṅkuśādibhirākṛṣya praveśya baddhvā vaśīkṛtya kīlayet //}
\pend

\bigskip

\pstart\noindent
{\large 3.5.2. Atyantavighnanivāraṇa}
\pend

\bigskip

\pstart
\Skt{atyantanirvighnaṃ kartukāmena mṛttikayā pracchādanīyāḥ / 
evamapi yadi vighnaṃ kurvanti~/
vajrahūṃkārayogaṃ kṛtvā ṭakkirājenākṛṣya 
\edtext{vajrāṅkuśādibhirvākarṣaṇādi}{\lemma{
	{\rm vajrāṅkuśādibhirvākarṣaṇādi\lem}
}\Dfootnote{
	\emn;
	\textit{vajraṁku*śādibhignākarṣaṇādi} \cod;
	\textit{vajrāṅkuśādibhirapyākarṣaṇādiṃ} \emn\ \MSK
}} kṛtvā 
vajrahūṃkāraṃ baddhvā 
vāmapādena vighnapratikṛtimākramya 
hū{\cb} va{\cb} hū{\cb} ityādi\edtext{vidarbhaṇa}{\lemma{
	{\rm -vidarbhaṇa-\lem}
}\Dfootnote{
	\cod;
	\textit{-vivardhana-} \emn\ \MSK\ (after Tib.);
}\Cfootnote{%
	\textit{vidarbhaṇa} here means that the seed-syllable of Varuṇa is 
	preceded and followed by the syllables \textit{hūṁ}.
}}yogātpratyālīḍha\-pādāvasthito'ntarāntarā ca meghādyabhimukhīṃ tāṃ mudrāṃ \edtext{kṣipan}{\lemma{%
	{\rm kṣipan\lem}
}\Dfootnote{
	\emn;
	\textit{kṣipet} \cod\ \MSK
}} 
\edtext{gaga\textcolor{red}{ṇo}dara}{\lemma{
	{\rm gagaṇodara-\lem}
}\Dfootnote{
	\cod;
	\textit{gagaṇodāra-} \emn\ \MSK;
}}spharaṇadīptajvā\-lā\-kula\edtext{prabhena}{\lemma{
	{\rm prabhena\lem}
}\Dfootnote{
	\corr;
	\textit{prabheṇa} \cod\ \MSK
}\lemma{}\Efootnote{
	-prabhena [36v1] vajrahūṁkāreṇa
}} vajrahūṃkāreṇa pādaprahārābhighātena meghādikaṃ \edtext{bhasmīkriyamāṇaṃ}{\lemma{
	{\rm bhasmīkriyamāṇaṁ\lem}
}\Dfootnote{
	\emn;
	\textit{bhasmīkṛyamāṇaṁ}  \cod \MSK
}} cintayet/ 
\edtext{evaṃ}{\lemma{
	{\rm evaṁ\lem}
}\Dfootnote{
	\emn\ \MSK;
	\textit{aivaṁ} \cod
}} ghātitā bhavanti /}
\pend

\bigskip

\pstart\noindent
{\large 3.5.3. Vāyuvighnanivāraṇa}
\pend

\bigskip

\pstart
\quad
\Skt{%
\edtext{}{\lemma{%
	{\rm atha vāyuvighna\dots\lem}
}\Bfootnote{%
	Parallel to \textit{Saṃpuṭatilaka},
	Hodgson f.115r, Wellcome f.168r, Derge 503b-504a
	\textcolor{red}{CHECK!!}
}}\edtext{atha \uline{vāyuvighnanivāraṇaṃ bhavati} /}{\lemma{
	{\rm atha \dots\ bhavati\lem}
}\Dfootnote{
	\conj;
	\textit{athavā} \cod\ \MSK
}\Cfootnote{
	\textcolor{red}{New command for "no-gap supplied"!}
}} \edtext{śūnyatāsamādhi}{\lemma{
	{\rm śūnyatāsamādhim\lem}
}\Cfootnote{
	\textit{śū} written in archaic form; 
	the same shape also in \textit{śūlāni} in line 6 on this same folio.
}}māmukhīkṛtyānantaramakāreṇa \edtext{vairocanībhūya}{\lemma{%
	{\rm vairocanībhūya\lem}
}\Cfootnote{%
	Emend \textit{vairocanābhisaṃbodhi-}?
	Since the system of this ritual might not be that of STTS, but \textit{Vairocanābhisaṃbodhi}.
}} \edtext{tantrokta}{\lemma{
	{\rm tantrokta-\lem}
}\Dfootnote{
	\emn\ \MSK;
	\textit{tatrokta-} \cod
}}vairocanarūpaṃ \textcolor{red}{niścintya} hāṃkāre\edtext{ṇāryācalaṃ}{\lemma{
	{\rm āryācalaṃ\lem}
}\Dfootnote{
	\emn\ \MSK\ \sil;
	\textit{-āyācalaṁ} \cod
}} manasā \edtext{svahṛdayā}{\lemma{
	{\rm svahṛdayān\lem}
}\Dfootnote{
	\emn;
	\textit{svahṛyān} \cod;
	\textit{svahṛdayād} \MSK
}}\edtext{nniścārya}{\lemma{
	{\rm niścārya\lem}
}\Dfootnote{
	\emn\ \MSK\ \sil;
	\textit{niścāya} 
}} 
vajra \edtext{hāṃ}{\lemma{
	{\rm hāṃ\lem}
}\Dfootnote{
	\emn\ \MSK;
	\textit{haṁ} \cod
}} bandha
iti manasodīrayan bāhyamaṇḍalasya bāhyato vāyavye koṇe gandhena \edtext{bindusaptakaṃ}{\lemma{
	{\rm bindusaptakaṃ\lem}
}\Dfootnote{
	\emn\ \MSK\ \sil;
	\textit{vindusaptaka} \cod
}} kṛtvaikaikaṃ \edtext{binduṃ}{\lemma{
	{\rm binduṁ\lem}
}\Dfootnote{
	\emn\ \MSK;
	\textit{vindu} \cod
}} \edtext{yakāreṇā}{\lemma{
	{\rm yakāreṇā-\lem}
}\Dfootnote{
	\cod;
	\textit{yaṁāreṇā-} \emn\ \MSK
}}bhyantarīkṛtya madhye ca saptamaṃ yaṃkāraṃ binduyuktam / 
ete ca sapta bindavaḥ sapta vāyavo \edtext{mṛgārūḍhāḥ}{\lemma{
	{\rm mṛgārūḍhāḥ\lem}
}\Dfootnote{
	\cod;
	\textit{mṛgārāḍhāḥ} \MSK\ (misprint)
}} kṛṣṇavarṇā āryācalena \edtext{pāśenānīya}{\lemma{
	{\rm pāśenānīya\lem}
}\Dfootnote{
	\emn;
	\textit{pāśenānīyā} \cod\ \MSK
}\Cfootnote{
	note archaic form of \textit{ś}.
}} \edtext{baddhā}{\lemma{
	{\rm baddhāś\lem}
}\Dfootnote{
	\emn;
	\textit{vadhvāś} \cod;
	\textit{bandhyāś} \MSK\ \silemn
}}ścintanīyāḥ~/}
\Skt{%
madhye \edtext{caiṣā}{\lemma{
	{\rm caiṣām\lem}
}\Dfootnote{
	\corr\ \MSK\ \sil;
	\textit{caiṣāṁm} \cod.
	\MSK\ ignores presence of \textit{anusvāra}.
}}meko nāyakaḥ śeṣāḥ ṣaḍbindavaḥ taṃ \edtext{parivārya}{\lemma{
	{\rm parivārya\lem}
}\Dfootnote{
	\emn\ \MSK\ \sil;
	\textit{parivāya} \cod
}} samantato 'vasthitā iti~/
tataḥ śarāvamākāreṇa meruṃ \edtext{vicintya}{\lemma{
	{\rm vicintya\lem}
}\Dfootnote{
	\emn\ \MSK;
	\textit{vicitya} \cod
}} tasyopari \edtext{māhendra}{\lemma{
	{\rm māhendra-\lem}
}\Dfootnote{
	\cod;
	\textit{mahendra-} \MSK
}}maṇḍala\edtext{mākāreṇaiva}{\lemma{
	{\rm ākāreṇaiva\lem}
}\Dfootnote{
	\emn\ \MSK;
	\textit{ākāreṇaivaḥ} \cod
}} sarvatra vajrasaṃcchannam /}
\Skt{%
madhye ca pañcasūcikaṃ vajraṃ \edtext{hūṃkārabījitam}{\lemma{
	{\rm hūṃkārabījitam\lem}
}\Cfootnote{
	it seems that \textit{bījita} is not common in Bauddha tantra, but we find occurrences in \textit{Tantrasadbhāva}, \textit{Brahmayāmala}, \textit{Svacchandatantra}, etc.
}} / 
koṇeṣu śūlāni ca vicintya
tathaiva \edtext{vajra hāṃ}{\lemma{
	{\rm vajra hāṃ\lem}
}\Dfootnote{
	\emn\ \MSK;
	\textit{vajrāhaṁ} \cod
}} bandhetyudrīrayaṃstena śarāveṇa tadyathoktaṃ \edtext{bindusaptakaṃ}{\lemma{%
	{\rm bindusaptakaṁ\lem}
}\Dfootnote{
	\emn;
	\textit{binduṃ saptakaṁ} \cod\ \MSK
}} \edtext{\textcolor{red}{pīḍayet}}{\lemma{%
	{\rm pīḍayet\lem}
}\Dfootnote{%
	\emn;
	\textit{pithayet} \cod\ \MSK
}}~/}
\Skt{%
tataḥ śarāvamerūpa\edtext{ryakāreṇa}{\lemma{
	{\rm akāreṇa\lem}
}\Dfootnote{
	\emn;
	\textit{aṁkāreṇa} \cod\ \MSK
}} \edtext{bhagavantaṃ}{\lemma{
	{\rm bhagavantaṁ\lem}
}\Dfootnote{
	\emn\ \MSK;
	\textit{bhagavanta} \cod
}} pūrvoktaṃ vairocanaṃ \edtext{vicintyāryācalaṃ}{\lemma{
	{\rm vicintyāryācalaṁ\lem}
}\Dfootnote{
	\emn\ \MSK;
	\textit{vicintyāyācalañ} \cod
}} cālīḍhapādena parvatamavaṣṭabhya \edtext{bhagavataḥ}{\lemma{
	{\rm bhagavataḥ\lem}
}\Dfootnote{
	\cod;
	\textit{bhagavantaḥ} \MSK
}\lemma{}\Efootnote{%
	bhagava[37r1]ntaḥ
}} purataḥ khaḍga\edtext{pāśa}{\lemma{
	{\rm -pāśa-\lem}
}\Dfootnote{
	\emn\ \MSK\ \sil;
	\textit{-ṣāṣa-} \cod
}}hastaṃ \edtext{vāyū}{\lemma{
	{\rm vāyūn\lem}
}\Dfootnote{
	\emn;
	\textit{vāyuṃ} \cod\ \MSK
}}ntarjamānaṃ cintayet /
evaṃ sarvavāyubandhaḥ kṛto bhavati //}
\pend

\bigskip

\pstart\noindent
{\large 3.5.4. Udakādivighnanivāraṇa}
\pend

\bigskip

\pstart
\quad
\Skt{%
athodakādivighna\edtext{nivāraṇaṃ}{\lemma{
	{\rm -nivāraṇaṁ\lem}
}\Dfootnote{
	\emn;
	\textit{-nivāraṁ} \cod \MSK
}} bhavati / 
hāṃkāreṇa vāyavyāgneyamaṇḍalastha\edtext{māryācala}{\lemma{
	{\rm āryācalam\lem}
}\Dfootnote{
	\emn\ \MSK\ \sil;
	\textit{āyācalam} \cod
}}\-mātmānaṃ niṣpādya~/ 
valmīkamṛttikayāpatitagomayena vā vighnapratikṛtiṃ kṛtvā~/ 
khaḍgaṃ khaṃkāreṇa \edtext{rephāgnigarbhaṃ}{\lemma{
	{\rm rephāgnigarbhaṁ\lem}
}\Dfootnote{
	\cod;
	\textit{rephenāgnigarbhaṁ} \MSK.
	The emendation does not seem called for.
}} \edtext{nirvartya}{\lemma{
	{\rm nirvartya\lem}
}\Dfootnote{
	\emn\ \MSK\ \sil;
	\textit{nirvattya} \cod
}}~/
\textcolor{red}{kṣmī}kāreṇa ca \dag\edtext{pāśaṃ kuśamayaṃ}{\lemma{%
	{\rm pāśaṃ kuśamayaṁ\lem}
}\Dfootnote{%
	\cod;
	\textit{pāśam aṅkuśamayaṁ} \MSK
}}\dag\ niṣpādya vāmahastenādāya dakṣiṇena khaḍgaṃ pāśenākṛṣya vighnaṃ prakṛtau niveśya~/ 
ālīḍhapādena sthitvā mūrdhni \edtext{vāmapādena}{\lemma{
	{\rm vāmapādena\lem}
}\Dfootnote{
	\emn\ \MSK\ \sil;
	\textit{cāmapādena} \cod
}} pīḍayedida\edtext{mudīrayan}{\lemma{
	{\rm udīrayan\lem}
}\Dfootnote{
	\emn;
	\textit{udīrayam} \cod;
	\textit{udīrayaṁ} \MSK
}} }
\Skt{%
namaḥ samantavajrāṇāṃ caṇḍamahāroṣaṇa sphoṭaya hūṃ \edtext{traṭ}{\lemma{
	{\rm traṭ\lem}
}\Dfootnote{
	\emn;
	\textit{tram} \cod;
	\textit{traṃ} \MSK
}} hāṃ māmiti tatpratikṛtiṃ khaḍgena vā vidārayet / 
viṣāktābhirvā \edtext{rājikābhiḥ}{\lemma{
	{\rm rājikābhiḥ\lem}
}\Dfootnote{
	\emn\ \MSK;
	\textit{rājikābhi} \cod
}} pratikṛtiṃ vilipyāgninā tāpayet /}
\pend

\verse
\Skt{%
\edtext{āpitastvagninā so hi lepitaśca na saṃśayaḥ /\\
api \edtext{brahmāpi}{\lemma{
	{\rm brahmāpi\lem}
}\Dfootnote{%
	\emn\ \MSK;
	\textit{vrāhmāpi} \cod
}} śakro vā kṣipraṃ dahyati tatkṣaṇamiti // \\
āha bhagavānvairocanaḥ //}{\lemma{%
	{\rm tāpitas \dots\ vairocanaḥ\lem}
}\Bfootnote{
	Not traced.
	The phrase "\textit{ity āha bhagavān vairocanaḥ}"
	is a part of quotation.
	Quoted also in the \textit{*Trailokyavijayodayā} (Toh.2519)
	\textcolor{red}{CHECK!!}
}}}
\pend

\pstart\noindent
\Skt{%
\edtext{āryācalayogena}{\lemma{
	{\rm āryācalayogena\lem}
}\Dfootnote{
	\emn\ \MSK\ \sil;
	\textit{ayācalayogena} \cod
}} cāsya \edtext{mantrasyāyutasevāṃ}{\lemma{}\Efootnote{%
	mantra[37v1]syāyutasevāṁ
}} kṛtvā 
\edtext{paścātkarma kuryādi}{\lemma{
	{\rm paścāt karma kuryād\lem}
}\Dfootnote{
	\emn\ \MSK\ \sil;
	\textit{paścāt karma kuryād} \cod
}}ti //}
\pend

\pstart
\quad
\Skt{% 
\edtext{{\om} āḥ hū{\cb} i}{\lemma{
	{\rm oṃ āḥ hūṃ ity\lem}
}\Dfootnote{
	\emn\ \MSK;
	\textit{o;m aḥ hū;mm ity} \cod
}}tyanena \edtext{\textcolor{red}{cā}ryamañjuśrīyamāntakaṃ}{\lemma{
	{\rm cāryamañjuśrīyamāntakaṃ\lem}
}\Dfootnote{
	\emn\ \MSK;
	\textit{cāyamañjuśrīyamantakaṁ} \cod.
	\emn\ with \MSK????
	Or \textit{cāryamañjuśrīyam antakaṁ}????
}} kṛṣṇaṃ \edtext{ṣaṭcaraṇaṃ}{\lemma{
	{\rm ṣaṭcaraṇaṁ\lem}
}\Dfootnote{
	\emn\ \MSK\ \sil;
	\textit{śaṭcaraṇaṁ} \cod
}} caturmukhaṃ \edtext{caturbhujaṃ}{\lemma{
	{\rm caturbhujaṁ\lem}
}\Dfootnote{
	\emn\ \MSK\ \sil;
	\textit{caturbhurjāṁ} \cod
}} dakṣiṇabhujābhyāṃ khaḍgaparaśudharaṃ vāmabhujābhyāṃ \edtext{pāśamusala}{\lemma{
	{\rm pāśamusala-\lem}
}\Dfootnote{
	\emn\ \MSK\ \sil;
	\textit{pāsamuśala-} \cod
}}dharaṃ dakṣiṇābhimukhaṃ krodhagaṇairgagaṇamāpūrayamāṇaṃ vicintya śūlamudrāṃ baddhvā meghādinivāraṇāya \edtext{sākṣepaṃ}{\lemma{
	{\rm sākṣepaṁ\lem}
}\Dfootnote{
	\emn\ \MSK;
	\textit{sākṣepa} \cod
}} prayojayet / 
talavajrabandhe tarjanīdvayasūcī śūlamudrā //}
\pend

\bigskip

\pstart\noindent
{\large 3.5.5. Optional Ritual}
\pend

\bigskip

\pstart
\Skt{%
sacedevamapi \edtext{vighnopaśāntirna bhavati}{\lemma{
	{\rm vighnopaśāntir na bhavati\lem}
}\Dfootnote{
	\emn\ \MSK;
	\textit{vighnopaśānti bhavati} \cod
}}~/
āmaśarāvadvayamadhye rudhiracityaṅgāra\edtext{viṣa}{\lemma{
	{\rm -viṣa-\lem}
}\Dfootnote{
	\emn\ \MSK\ \sil;
	\textit{-viśa-} \cod
}}\-rājikonmattaka\edtext{pattra}{\lemma{
	{\rm -pattra-\lem}
}\Dfootnote{
	\emn\ \MSK\ \sil;
	\textit{-patta-} \cod
}}rasairmantramālikhya siddhārthakā\edtext{naṣṭaśataparijaptāṃstasminśarāvasaṃpuṭe}{\lemma{
	{\rm aṣṭaśataparijaptāṁs tasmin śarāvasaṃpuṭe\lem}
}\Dfootnote{
	\emn\ \MSK;
	\textit{aṣṭaśataparijaptām|s tasmi sarāvasampuṭe} \cod.
	\MSK\ does not report \textit{sarāva-}.
}} \edtext{nyasyānalavighnādiṃ}{\lemma{
	{\rm nyasyānalavighnādiṃ\lem}
}\Dfootnote{
	\emn\ \MSK;
	\textit{nyasyanaravighnādiāṁ} \cod.
	\MSK\ does not report \textit{-diā}ṁ with double vocalization.
}} \edtext{dṛṣṭvāhutiśataṃ}{\lemma{
	{\rm dṛṣṭvāhutiśataṃ\lem}
}\Dfootnote{
	? \emn\ \MSK;
	\textit{dṛṣṭvā āhutiśataṁ} \cod
}} saptakaṃ vā juhuyāt / 
tata\edtext{ste naśyante mriyante}{\lemma{
	{\rm te naśyante mriyante\lem}
}\Dfootnote{
	\emn;
	\textit{tai nasyantai mṛyate} \cod;
	\textit{te naśyante mṛyante} \MSK\ (\emn);
	\MSK\ does not report \textit{-sy-}.
}} \edtext{vāmanuṣyai}{\lemma{
	{\rm vāmanuṣyair\lem}
}\Cfootnote{
	\textit{vā amanuṣyair} \cod.
	\MSK\ reads \textit{vāmamanuṣyair} and emends this to \textit{vāmanuṣyair}. 
	But the text is correct as it stands.
}}rvā gṛhyante /}
\Skt{vajrahūṃkārayogeṇa vā hū{\cb}kāraṃ lakṣajaptaṃ kṛtvā mānuṣāsthimayacaturaṅgulapramāṇakīlakenāṣṭottaraśatajaptena tathaiva \edtext{vajrāṅkuśādibhirākṛṣya}{\lemma{}\Efootnote{%
	vajrāṅkuśā[38r1]dibhir ākṛṣya
}} praveśya baddhvā \supplied{vaśī\-kṛtya ca} pratikṛtim vajreṇa kīlayediti //}
	%-śāntibhavati: emend -śāntir na bhavati, with MSK.
	%-viśa-: emend -viṣa-, with MSK (silent em.).
	%-patta-: emend -pattra-, with MSK (silent em.).
	%-japtām·|s tasmi sarāva-: emend -japtāṁs tasmin śarāva-, with MSK. MSK does not report sarāva-.
	%nyasyanaravighnādiāṁ: emend nyasyānalavighnādiṁ, with MSK. MSK does not report -diāṁ with double vocalization.
	%dṛṣṭvā °āhutiśataṁ: MSK emends to dṛṣṭvāhutiśataṁ.
	%tai nasyantai mṛyate: emend te naśyante mriyante , with MSK. MSK does not report -sy-.
	%vā °amanuṣyair: MSK reads vāmamanuṣyair and emends this to vāmanuṣyair. But the text is correct as it stands.

\pend

\bigskip

\pstart\noindent
{\large 3.5.6. Agnidāhaśamana}
\pend

\bigskip
\pstart
\quad
\Skt{%
athāgnidāhaśamanamabhidhīyate /
agnerupari \edtext{vaṃkāreṇa}{\lemma{
	{\rm vaṃkāreṇa\lem}
}\Dfootnote{
	\emn\ \MSK;
	\textit{vakāreṇa} \cod
}} \edtext{vāruṇamaṇḍalaṃ}{\lemma{
	{\rm vāruṇamaṇḍalaṁ\lem}
}\Dfootnote{
	\cod;
	\textit{varuṇamaṇḍalaṁ} \MSK
}} tasyoparipadmamadhyasthitaṃ cakraṃ tasyopari \edtext{vaṃkāreṇaiva}{\lemma{
	{\rm vaṁkāreṇaiva\lem}
}\Dfootnote{
	\emn\ \MSK;
	\textit{vakāreṇaiva} \cod
}} śaṅkhakundendudhavalaṃ vairocanamupaviṣṭaṃ \edtext{catuḥsamudra}{\lemma{
	{\rm catuḥsamudra-\lem}
}\Dfootnote{
	\emn\ \MSK;
	\textit{catussamudra-} \cod
}}\edtext{caturnadya}{\lemma{
	{\rm -caturnadya-\lem}
}\Dfootnote{
	\emn;
	\textit{-catunardyā-} \cod;
	\textit{-caturnadyā-} \emn\ \MSK\ \sil
}}navataptasadṛśanavavāridhārābhirāpūrayantaṃ daśadiśaścintayediti /
vajranetryā \edtext{vajrajvālānalārkeṇa}{\lemma{
	{\rm vajrajvālānalārkeṇa\lem}
}\Dfootnote{
	\emn\ \MSK\ \sil;
	\textit{vajrājvālānalārkeṇa} \cod
}} ca \edtext{śarāvameruṃ kīlakaṃ ca}{\lemma{
	{\rm śarāvameruṃ kīlakaṃ ca\lem}
}\Dfootnote{
	\emn\ \MSK\ \sil;
	\textit{śarāvamekaṁ§ kīlakaś ca} \cod.
	Emendation supported by Tib. and par. 57.
}} saṃrakṣya vajrayakṣavajrabhairavanetrābhyāṃ \edtext{teṣāmākāśabandhaṃ}{\lemma{
	{\rm teṣām ākāśabandhaṃ\lem}
}\Dfootnote{
	\emn\ \MSK\ \sil;
	\textit{teṣāṁmm ākāśavandhaṁ} \cod$^{ac}$;
	\textit{teṣāmm ākāśavandhaṁ} \cod$^{ac}$
}} kṛtvā vajrabandhena \edtext{vajrapañjaraṃ}{\lemma{%
	{\rm vajrapañjaraṁ\lem}
}\Cfootnote{%
	\textcolor{red}{Emend vajrapañjare?}
}} dadyāt /
abhicārahomena vā vighnanivāraṇaṃ \edtext{kuryādi}{\lemma{
	{\rm kuryād\lem}
}\Dfootnote{
	\emn\ \MSK;
	\textit{kuyād} \cod
}}ti //}
\pend

\bigskip

\pstart\noindent
{\large 3.5.7. Removing the Pegs from the Ground}
\pend

\bigskip

\pstart
\quad
\Skt{%
\edtext{tato maṇḍalanābhisthitakīlakaṃ caturhūṃkāreṇotpāṭya vakṣyamāṇapañcavarṇakaiḥ kīlakagartāṃ \edtext{prapūrya}{\lemma{
	{\rm prapūrya\lem}
}\Dfootnote{
	\emn\ \MSK\ \sil;
	\textit{prapūya} \cod
}} \edtext{samīkuryāt}{\lemma{
	{\rm samīkuryāt\lem}
}\Dfootnote{
	\emn\ \MSK\ \sil;
	\textit{samīkuyāt·} \cod
}} /
tato hrīḥkāreṇa vāmetaracakṣuṣoścandrasūryau niṣpādya \edtext{krodhadṛṣṭyā}{\lemma{
	{\rm krodhadṛṣṭyā\lem}
}\Dfootnote{
	\emn\ \MSK;
	\textit{krodhadṛṣṭvā} \cod
}} parito nibhālya \edtext{dvārāṇi}{\lemma{
	{\rm dvārāṇi\lem}
}\Dfootnote{
	\emn\ \MSK\ \sil;
	\textit{dvārāni} \cod
}} \edtext{kuryāt //}{\lemma{
	{\rm kuryāt ||\lem}
}\Dfootnote{
	\emn\ \MSK\ \sil;
	\textit{kuyāt |} \cod.
	It seems the scribe first wrote \textit{kuyāditi} because traces of \textit{diti} are visible.
}}}{\lemma{%
	{\rm tato \dots\ kuyāt\lem}
}\Bfootnote{%
	Parallel to \textit{Tattvālokakarī} (P LOCATION, D f.112r4–6)
	% Endo p.199
}} }
\pend


\bigskip

% \input{09_Rajah}


\pstart\noindent
{\large 3.6. Rajaḥpātanavidhi}
\pend

\bigskip

\pstart\noindent
{\large 3.6.1. Raṅgābhisaṃskāraḥ}
\pend

\bigskip

\pstart
\Skt{atha \edtext{raṅgābhisaṃskāro}{\lemma{
	{\rm raṅgābhisaṁskāro\lem}
}\Dfootnote{
	\emn\ \MSK;
	\textit{raṅgābhisaskāro} \cod
}} bhavati / 
{\om} vajracitrasamaya hū{\cb} / 
\edtext{ityanena}{\lemma{%
}\Efootnote{%
	hūṁ | [38v1]ity anena
}} mudrāyuktena \edtext{sarvaraṅgā}{\lemma{
	{\rm sarvaraṅgān\lem}
}\Dfootnote{
	\emn\ \MSK;
	\textit{sarvaraṅgāṁ} \cod
}}nsaptavārānabhimantraye\-dvajramayā bhavantītyāha bhagavānmahāvajradharaḥ / 
\edtext{tatreyaṃ}{\lemma{
	{\rm tatreyaṁ\lem}
}\Dfootnote{
	\emn\ \MSK;
	\textit{tattreya} \cod.
	\MSK\ reads \textit{tantreyaṁ}.
}} mudrā /} 
\pend

\verse
\Skt{\edtext{susaṃdhita\edtext{samāgryaṃ}{\lemma{
	{\rm -samāgryaṁ\lem}
}\Dfootnote{
	\emn\ \MSK\ \sil;
	\textit{-samābhrantu} \cod.
	Cf. STTS
}} tu vajramudrādvikasya tu / \\
\edtext{kṛtvā tu}{\lemma{
	{\rm kṛtvā tu\lem}
}\Dfootnote{
	\emn\ \MSK;
	\textit{kṛtvā} \cod;
	Cf. STTS.
}} \edtext{sarvaraṅgāṇi}{\lemma{
	{\rm sarvaraṅgāṇi\lem}
}\Dfootnote{
	\emn\ $\leftarrow$ STTS
	\textit{sarvaraṅgāṇāṁ} \cod;
	\MSK\ does not make this emendation.
}} \edtext{dīptadṛṣṭyā}{\lemma{
	{\rm dīptadṛṣṭyā\lem}
}\Dfootnote{
	\emn\ \MSK;
	\textit{dīptadṛṣṭvā} \cod
}} samāhvayet //}{\lemma{
	{\rm susaṁdhita\dots\ samāhvayet\lem}
}\Bfootnote{
	STTS \S\ 986
}} }
\pend

\pstart
\Skt{hīḥkāreṇa cānte jvālayedvajra\edtext{sūrya}{\lemma{
	{\rm -sūrya\lem}
}\Dfootnote{
	\emn\ \MSK\ \sil;
	\textit{-sūya-} \cod
}}krodhasamayamudrayā vāmavajra\edtext{muṣṭinā}{\lemma{
	{\rm -muṣṭinā\lem}
}\Dfootnote{
	\emn\ \MSK\ \sil;
	\textit{-muṣṭhinā} \cod
}} ca / 
{\om} vajracitrasamaya hū{\cb} /
ityuccārayannaiśānīṃ diśamārabhya pradakṣiṇato \edtext{raṅgaṃ}{\lemma{
	{\rm raṅgaṁ\lem}
}\Dfootnote{
	\emn\ \MSK\ \sil;
	\textit{raṅga} \cod
}} pātayet / }
\Skt{tataḥ paścādyathāsukhamanena krameṇa nīlaṃ pītaṃ raktaṃ śyāmaṃ śuklamiti / 
% tataḥ paścād: here refers back to jvālayet
% yathāsukham here means "in appropriate quantity [of pigments]"
ṣaḍḍhaste \edtext{kanīyasī}{\lemma{
	{\rm kanīyasī\lem}
}\Dfootnote{
	\emn;
	\textit{kaniyasī} \cod.
	\MSK\ incorrectly emends \textit{kanīyasā}.
}}\edtext{pramāṇā}{\lemma{%
	{\rm pramāṇā\lem}
}\Dfootnote{%
	\emn;
	\textit{parimāṇaṁ} \cod\ \MSK
}} \edtext{raṅgaṃ}{\lemma{
	{\rm raṅgaṃ\lem}
}\Dfootnote{
	\emn\ \MSK\ \sil;
	\textit{raṅga} \cod
}} \edtext{raṅgarekhā / ataḥ paraṃ}{\lemma{
	{\rm raṅgarekhā | ataḥ paraṁ\lem}
}\Dfootnote{
	\emn;
	\textit{raṅgarekhāata | paraṁ} \cod
	\MSK\ incorrectly emends \textit{raṅgaṁ rekhāyeta | paraṁ}.
}} \edtext{hastavṛddhyā}{\lemma{
	{\rm hastavṛddhyā\lem}
}\Dfootnote{
	\emn\ \MSK\ \sil;
	\textit{hastavṛddhā} \cod
}} pādavṛddhiḥ / }
\pend

\bigskip

\pstart\noindent
{\large 3.6.2. Colouring the Maṇḍala (Scriptural Souces)}
\pend

\bigskip

\pstart
\Skt{tathā \edtext{coktaṃ}{\lemma{
	{\rm coktaṁ\lem}
}\Dfootnote{
	\emn\ \MSK\ \sil;
	\textit{cokta} \cod
}} vajraśekhare /} 
\pend

\verse
\Skt{%
\edtext{nīlavajramayī \edtext{sūciḥ}{\lemma{
	{\rm sūciḥ\lem}
}\Dfootnote{
	\emn\ \MSK;
	\textit{sūci |} \cod
}} sauvarṇālambanā parā /\\ 
padmarāgamayī sūcistathā mārakatī parā / \\
\edtext{śvetābhyantarato}{\lemma{
	{\rm śvetābhyantarato\lem}
}\Dfootnote{
	\cod\pc;
	\textit{śvetābhyāntarato} \cod\ac
}} jñeyā eṣa raṅgakramaḥ smṛtaḥ //}{\lemma{%
	{\rm nīlavajramayī \dots\ smṛtaḥ\lem}
}\Bfootnote{%
	\textcolor{red}{\textit{Vajraśekhara} LOCATION!!}.
	This is incorporated into Dīpaṅkarabhadra's \textit{Guhyasamājamaṇḍalavidhi}, vv.246–247ab:
	\textit{śvetavajramayī sūcī sauvarṇālambanāparā |
	padmarāgamayī sūcī tathā marakatāparā || 
	kṛṣṇābhyantarato jñeyā eṣa raṅgakramo 'sya tu |.}
	Dīpaṅkarabhadra rewrites \textit{nīlavajramayī} to \textit{śvetavajramayī}, 
	and \textit{śvetābhyantarato} to \textit{kṛṣṇābhyantarato},
	since Vairocana and Akṣobhya changes their positions in the Guhyasamāja system.
	See also the notes below.
}} }
% sūci, parā and ālambana lines of triangle?
% CHECK Guhyasamājamaṇḍalavidhi 246 - 247
\pend
\pstart\noindent
\Skt{ityāha bhagavānmahāvajradharaḥ //}
\pend

\verse
\Skt{\edtext{\edtext{pūrveṇa mahānīlaṃ}{\lemma{%
	{\rm pūrveṇa mahānīlaṁ\lem}
}\Cfootnote{
	\MSK\ emends \textit{pūrveṇa tu mahānīlaṁ}, with ref. to Tib., but this does not actually support restitution of \textit{tu}. 
}} dakṣiṇaṃ pītasaṃyutam /\\
lohitaṃ \edtext{paścimabhāgaṃ}{\lemma{}\Efootnote{%
	paścimabhā[39r1]gaṁ
}} mañjiṣṭottarasaṃyutam //\\
%
madhyato bhūmighāgaṃ ca sphaṭikābhaṃ niruttaram /\\
vajraghaṇṭādharo nityaṃ saṃlikhetsusamāhitaḥ //}{\lemma{%
	{\rm pūrveṇa \dots\ susamāhitaḥ\lem}
}\Bfootnote{%
	These two verses are also found in the \textit{Guhyasamājamaṇḍalavidhi} immediately after the six \textit{pāda}s quoted above, i.e. Dīpaṅkarabhadra does not insert the phrase \textit{ity āha bhagavān mahāvajradharaḥ}.
	\textit{Guhyasamājamaṇḍalavidhi} vv.247cd–249ab:
	\textit{pūrveṇa tu mahāśvetaṃ dakṣiṇe pītasaṃyutam || 
	lohitaṃ paścimabhāgaṃ mañjiṣṭhottarasaṃyutam |
	madhyato bhūmibhāgaṃ tu indranīlaprabhāsvaram ||
	prajñopāyātmako nityaṃ saṃlikhet susamāhitaḥ |}
	Dīpaṅkarabhadra rewrites \textit{mahānīlaṃ} to \textit{mahāśvetaṃ},
	and \textit{sphaṭikābhaṃ niruttaram} to \textit{indranīlaprabhāsvaram}
	for the same reason stated above.
	He also rewrites \textit{vajraghaṇṭādharo} to \textit{prajñopāyātmako}.
}} }\\
%
% STTS
\Skt{\edtext{\mbox{}\edtext{vajrastambhāgra}{\lemma{
	{\rm vajrastambhāgra-\lem}
}\Dfootnote{
	\emn\ \MSK;
	\textit{-vajrastambhogra} \cod
}}\edtext{saṃsthendu}{\lemma{%
	{\rm saṁsthendu\lem}
}\Dfootnote{%
	\emn\ following STTS;
	\textit{saṃstheṣu} \cod 
}}pañcamaṇḍalamaṇḍitam /\\
madhyamaṇḍalamadhye tu buddhabimvaṃ niveśayet //\\
%
buddhasya sarvapārśveṣu \edtext{maṇḍalānāṃ tu}{\lemma{
	{\rm maṇḍalānāṁ tu\lem}
}\Dfootnote{
	\emn\ \MSK;
	\textit{maṇḍalānan tu} \cod
}} madhyataḥ /\\
\edtext{samayāgrya}{\lemma{
	{\rm samayāgryaś\lem}
}\Cfootnote{
	\MSK\ emends \textit{samayāgrīḥ}. No emendation is necessary. 
	STTS, ed. Horiuchi, §205.
}}ścatasro hi saṃlikhedanupūrvaśaḥ //\\
%
\edtext{vajravegeṇa}{\lemma{
	{\rm vajravegeṇa\lem}
}\Cfootnote{
	STTS reads \textit{vajravegena}.
}} cākramya maṇḍalānāṃ catuṣṭaye /\\
\edtext{akṣobhyādyāṃ}{\lemma{
	{\rm akṣobhyādyāṁs\lem}
}\Dfootnote{
	\emn\ \MSK\ $\leftarrow$STTS;
	\textit{akṣobhyādyās} \cod
}}stu caturaḥ \edtext{sarvabuddhān}{\lemma{%
	{\rm sarvabuddhān\lem}
}\Dfootnote{%
	\emn\ \MSK;
	\textit{sarvabuddhāṁ} \cod
}} niveśayet //\\
%
akṣobhyamaṇḍalaṃ ku\textcolor{red}{ryā}t samaṃ vajradharādibhiḥ /\\
vajragarbhādibhiḥ \edtext{pūrṇaṃ}{\lemma{
	{\rm pūrṇaṁ\lem}
}\Dfootnote{
	\emn\ \MSK\ $\leftarrow$STTS;
	\textit{pūrṇṇā} \cod
}} ratnasaṃbhavamaṇḍalam //\\
%
%\MSK
%     vajranetrādibhiḥ śuddhaṁ maṇḍalaṁ tv amitāyuṣaḥ /
%     amoghasiddheḥ saṁlekhyaṁ vajraviśvādimaṇḍalam // 7 // iti //
%
%RT&AG:
vajranetrādibhiḥ śuddhaṃ maṇḍalaṃ tvamitāyuṣaḥ /\\
amoghasiddheḥ saṃlekhyaṃ vajraviśvādimaṇḍalamiti //\\
%
%\MSK
%     cakrasya koṇasaṁstheṣu vajradevīḥ saṁlikhet /
%     bāhyamaṇḍalakoṇeṣu buddhapūjāḥ samālikhet // 8 //
%
%RT&AG
cakrasya koṇasaṃstheṣu \edtext{vajradevyaḥ}{\lemma{
	{\rm vajradevyaḥ\lem}
}\Cfootnote{
	\MSK\ emends \textit{vajradevīḥ}. No emendation is necessary. STTS, ed. Horiuchi, §206.
}} 
%\foliochange{vajradevyaḥ[39r5]samālikhet}
samālikhet /\\
\edtext{bāhyamaṇḍalakoṇeṣu}{\lemma{
	{\rm bāhyamaṇḍalakoṇeṣu\lem}
}\Dfootnote{
	\emn\ \MSK\ \sil;
	\textit{vāhyamaṇḍalakoṇepu} \cod
}} buddhapūjāḥ samālikhet //\\
	%vajradevyaḥ: \MSK\ emends vajradevīḥ. No emendation is necessary. STTS, ed. Horiuchi, §206.
	%-koṇepu
%
%\MSK
%     dvāramadhyeṣu sarveṣu dvārapālacatuṣṭayam /
%     bāhyamaṇḍalasaṁstheṣu mahāsattvān niveśayet // 9 //
%
%RT&AG
dvāramadhyeṣu sarveṣu dvārapāla\edtext{catuṣṭayam}{\lemma{
	{\rm -catuṣṭayam\lem}
}\Dfootnote{
	\emn\ \MSK\ \sil\ following STTS;
	\textit{-catuṣṭhayam} \cod
}} /\\
bāhyamaṇḍalasaṃstheṣu \edtext{mahāsattvā}{\lemma{
	{\rm mahāsattvān\lem}
}\Dfootnote{
	\emn\ \MSK;
	\textit{mahāsatvā} \cod
}}nniveśayet //}{\lemma{%
	{\rm vajrastambhāgra\dots\ niveśayet}
}\Bfootnote{%
	STTS \S\S\ 204, 8)–206: 
	\textit{vajrastambhāgrasaṃsthendupañcamaṇḍalamaṇḍitam |
	madhyamaṇḍalamadhye tu buddhabimbaṃ niveśayet ||
	buddhasya sarvapārśveṣu maṇḍalānāṃ tu madhyataḥ |
	samayāgryaś catasro hi saṃlikhed anupūrvaśaḥ ||
	vajravegena cākramya maṇḍalānāṃ catuṣṭaye |
	akṣobhyamaṇḍalaṃ kuryāt samaṃ vajradhārādibhiḥ |
	vajragarbhādibhiḥ pūrṇaṃ ratnasaṃbhavamaṇḍalam ||
	vajranetrādibhiḥ śuddhaṃ maṇḍalaṃ tv amitāyuṣaḥ |
	amoghasiddheḥ saṃlekhyaṃ vajraviśvādimaṇḍalam iti ||
	cakrasya koṇasaṃstheṣu vajradevyaḥ samālikhet |
	bāhyamaṇḍalakoṇeṣu buddhapūjāḥ samālikhet ||
	dvāramadhyeṣu sarveṣu dvārpālacatuṣṭayam |
	bāhyamaṇḍalasaṃstheṣu mahāsattvān niveśayet ||}
}} }\\
	%-catuṣṭhayam·: emend -catuṣṭayam·, with \MSK\ (sil. \emn) following STTS.
	%mahāsatvā: emend mahāsattvān, with \MSK.
%
%
%\MSK
% tatredaṁ vajravegahṛdayaṁ bhavati / oṁ vajravegākrama hūṁ // iti // 
%
%RT&AG
\Skt{\edtext{tatredaṃ vajravegahṛdayaṃ bhavati / 
{\om} vajravegākrama \edtext{hū{\cb}}{\lemma{
	{\rm hūṁ\lem}
}\Dfootnote{
	\emn\ \MSK;
	\textit{hū;mm} \cod
}} iti //}{\lemma{
	{\rm tatredaṁ \dots\ iti\lem}
}\Bfootnote{
	STTS §864:
	\textit{tatredaṃ vajravegahṛdayaṃ bhavati \textemdash\
	oṃ vajravegākrama hūṃ}.
}} }
\pend

\verse
\Skt{\edtext{athāsya \edtext{mudrā}{\lemma{
	{\rm mudrā\lem}
}\Dfootnote{
	\emn\ \MSK\ following STTS;
	\textit{muttra} \cod
}} bhavati \textemdash\\
\noindent
\edtext{manasotkṣipya}{\lemma{
	{\rm manasotkṣipya\lem}
}\Dfootnote{
	\emn\ \MSK\ \sil\ following STTS;
	\textit{manasokṣipya} \cod
}} \edtext{rekhāṃ}{\lemma{
	{\rm rekhāṁ\lem}
}\Dfootnote{
	\emn\ \MSK\ following STTS;
	\textit{rekhan} \cod
}} tu \edtext{vajrasūtramathāpi}{\lemma{
	{\rm vajrasūtram athāpi\lem}
}\Dfootnote{
	\emn\ following STTS;
	\textit{vajrasūtrayathāpi}  \cod;
	\MSK\ emends \textit{vajrasūtraṁ yathāpi}.
}} vā /\\
\edtext{praviśan}{\lemma{%
	{\rm praviśan\lem}
}\Dfootnote{%
	\emn\ \MSK;
	\textit{praviṣyanti} \cod
}} \edtext{niṣkramanvāpi}{\lemma{%
	{\rm niṣkraman\lem}
}\Dfootnote{
	\emn\ \MSK;
	\textit{niṣkrāman} \cod
}\lemma{}\Efootnote{
	niṣkraman [39v1] vāpi
}} \edtext{bhraśyate}{\lemma{%
	{\rm bhraśyate\lem}
}\Cfootnote{%
	STTS reads \textit{trasyate}, which should be emended to \textit{bhraśyate} following SVU.
}} samāyānna sa iti //}{\lemma{
	{\rm athāsya \dots\ iti\lem}
}\Bfootnote{
	STTS §866:
	\textit{athāsya mudrā bhavati \textemdash\ 
	manasotkṣipya rekhāṃ tu vajrasūtram athāpi vā |
	praviśan niṣkraman vāpi *bhraśyate (\emn\ following SVU; \textit{trasyate} ed.) samayān na sa iti ||}
}} }
\pend

\bigskip

\pstart\noindent
{\large 3.6.3. Colouring: The Five Tathāgatas}
\pend

\bigskip

\pstart
\edtext{}{\lemma{%
	{\rm §§3.6.3–5\lem}
}\Bfootnote{%
	Ānandagarbha gives more detailed iconographical information
	about the deities of the Vajradhātumahāmaṇḍala (except the Bodhisattvas of the Bhadrakalpa) in his \textit{Tattvālokakarī} (P LOCATION!!, D f.113r2–115r2 CHECK!!).
	See also Endo 2018: 177, note 26.
}}%
\Skt{%
\edtext{tatra}{\lemma{
	{\rm tatra\lem}
}\Dfootnote{
	\emn\ \MSK;
	\textit{tattra} \cod
}} \edtext{bhagavā}{\lemma{
	{\rm bhagavān\lem}
}\Dfootnote{
	\emn\ \MSK;
	\textit{bhagavāṁ} \cod
}}\edtext{nvairocanaḥ}{\lemma{
	{\rm vairocanaḥ\lem}
}\Dfootnote{
	\emn\ \MSK;
	\textit{vairocana} \cod
}} sitavarṇaḥ siṃhāsane vajra\edtext{paryaṅka}{\lemma{
	{\rm -paryaṅka-\lem}
}\Dfootnote{
	\emn\ \MSK\ \sil;
	\textit{-payaṅka-} \cod
}}\edtext{niṣaṇṇo}{\lemma{
	{\rm -niṣaṇṇo\lem}
}\Dfootnote{
	\emn\ \MSK\ \sil;
	\textit{niṣarṇṇo} \cod
}} \edtext{bodhyagrīmudrayā}{\lemma{%
	{\rm bodhyagrīmudrayā\lem}
}\Dfootnote{%
	\emn;
	\textit{bodhyagrimudrayā} \cod\ \MSK
}} pañcasūcikavajradhārī \edtext{sūryaprabhamaṇḍalaḥ}{\lemma{%
	{\rm sūryaprabhamaṇḍalaḥ\lem}
}\Dfootnote{%
	\emn;
	\textit{sūryaprabhaḥ} \cod
}} paṭṭaśāṭikānivasanottarīyaścaturmukho ratnamakuṭapaṭṭābhiṣekī pradhānamukhena \edtext{pūrvānanaḥ}{\lemma{
	{\rm pūrvānanaḥ\lem}
}\Dfootnote{
	\emn\ \MSK;
	\textit{pūrvvānana} \cod
}} / 
evamakṣobhyādayo'pi gajāsanā\textcolor{red}{di}ṣu vajraparyaṅka\edtext{niṣaṇṇāḥ}{\lemma{
	{\rm -niṣaṇṇāḥ}
}\Dfootnote{
	\emn\ \MSK;
	\textit{-niṣarṇṇaḥ} \cod
}} sūryaprabhāmaṇḍalā vairocanābhimukhā \edtext{ratnamakuṭapaṭṭābhiekinaḥ}{\lemma{%
	{\rm ratnamakuṭapaṭṭābhiekinaḥ\lem}
}\Dfootnote{%
	\emn\ \MSK\ \sil;
	\textit{ratnamakuṭāpaṭṭābhiṣekiṇaḥ} \cod
}} nīlapītaraktaśyāmāvarṇā yathākrameṇa / 
ekamukhāḥ \edtext{pañcasūcikavajravajraratnavajrapadmaviśvavajradharāḥ}{\lemma{
	{\rm pañca\dots viśvavajradharāḥ\lem}
}\Dfootnote{
	\emn;
	\textcolor{red}{\textit{pañcasūcikavajraḥ vajraratnaḥ padmacadmaḥ viśvavajaradharāḥ} \cod\ac};
	\textcolor{red}{\textit{pañcasūcikavajraḥ vajraratnaḥ padmaḥ viśvavajaradharāḥ} \cod\pc};
	\MSK\ emends \textit{pañcasūcikavajravajraratnapadmacakraviśvavajradharāḥ}
}} svamahāmudrābhiḥ~/ vajradhātviti 
pañcatathāgatahṛdayaudīrayatā sthāpyā lekhyā \edtext{vāntarikṣāvasthitā}{\lemma{
	{\rm vāntarikṣāvasthitāś\lem}
}\Dfootnote{
	\emn\ \MSK\ \sil;
	\textit{vā'ntarikhāvasthitāś}  \cod
}}śca \edtext{tathāvatāryaikīkāryāḥ}{\lemma{
	{\rm tathāvatāryaikīkāryāḥ\lem}
}\Dfootnote{
	\emn\ \MSK\ \sil;
	\textit{tathaivātāyaikīkāyāḥ} \cod
}} /}
\pend

\bigskip

\pstart\noindent
{\large 3.6.4. Colouring: The Four Goddesses Surrounding Vairocana}
\pend

\bigskip

\pstart
%\MSK\ (66) evaṁ sattvavajrī / iti sattvavajrīṁ yāvad vajrāveśa aḥ / iti vajrāveśam / bhagavato vairocanasyāgrataḥ pañcasūcikaṁ raktavajraṁ sattvavajrī / dakṣiṇapārśve pañcasūcikavajraśikhaṁ cintāmaṇiratnaṁ ratnavajrī / pṛṣṭhataḥ ṣḍaśapattaraṁ padmaṁ sitaraktam / aṣṭau pattrāṇy adho vikasitāni / upari cā ........

%RT&AG
\Skt{evaṃ sattvavajrīti \edtext{sattvavajrī}{\lemma{
	{\rm sattvavajrī\lem}
}\Dfootnote{
	\textit{sattvavajriṁ} \cod;
	\textit{sattvavajrīṁ} \MSK(\emn?)
	Cf.\ STTS §139
}} yāva\edtext{dvajrāveśeti vajrāveśaḥ}{\lemma{
	{\rm vajrāveśeti vajrāveśaḥ\lem}
}\Dfootnote{
	\emn;
	\textit{vajrāveśeti dvajrāveśaṁ} \cod.
	\textit{vajrāveśa aḥ / iti vajrāveśam} \MSK;
	 Cf. STTS §187.
}} / 
bhagavato vairocanasyāgrataḥ pañcasūcikaṃ \edtext{raktavajraṃ}{\lemma{
	{\rm raktavajraṁ\lem}
}\Dfootnote{
	\emn\ \MSK;
	\textit{raktavajrāṁ} \cod
}} sattvavajrī / \edtext{dakṣiṇapārśve}{\lemma{
	{\rm dakṣiṇapārśve\lem}
}\Dfootnote{
	f.\ 39v6
}} pañcasūcikavajraśikhaṃ cintāmaṇiratnaṃ ratnavajrī / 
\edtext{pṛṣṭhataḥ}{\lemma{
	{\rm pṛṣṭhataḥ\lem}
}\Dfootnote{
	\emn\ \MSK;
	\textit{pṛṣṭhata} \cod
}} / \edtext{ṣoḍaśapatraṃ}{\lemma{
	{\rm ṣoḍaśapatraṁ}
}\Dfootnote{
	\emn\ \MSK\ \sil;
	\textit{ṣoḍaṣapattram} \cod
}} padmaṃ sitaraktam aṣṭau patrāṇy \edtext{adho}{\lemma{
	{\rm adho\lem}
}\Dfootnote{
	\emn\ \MSK;
	\textit{ardhye} \cod
}} vikasitāni / upari \edtext{cāntargata}{\lemma{}\Efootnote{%
	cā[40r1]ntargata-
}}pañcasūcikavajrapattrāṇi mukulitānya\edtext{ṣṭāv eva}{\lemma{
	{\rm ṣṭāv eva\lem}
}\Dfootnote{
	\emn;
	\textit{aṣṭā caiva} \cod;
	\textit{aṣṭa caiva} \MM
}} \edtext{dharmavajrī}{\lemma{%
	{\rm dharmavajrī\lem}
}\Dfootnote{%
	\emn\ \MM;
	\textit{dharmmavajrīm} \cod
}} / vāmato \edtext{dvādaśasūcikaṃ pañcavarṇakaṃ viśvavajraṃ}{\lemma{
	{\rm dvādaśasūcikaṁ pañcavarṇakaṁ viśvavajraṁ\lem}
}\Dfootnote{
	\emn;
	\textit{dvādaśasūcikaṁ pañcavarṇṇākaṁ viśvavajraṁ} \cod;
	\textit{dvādaśasūcikapañcavarṇakaviśvavajram} \MM\ \sil
}} / karmavajrī / madhye sitavarṇam / purato nīlam / dakṣiṇe pītam / \edtext{pṛṣṭhato}{\lemma{
	{\rm pṛṣṭhato}
}\Dfootnote{
	\emn\ \MM\ \sil;
	\textit{pṛṣṭato} \cod
}} raktam / vāmato \edtext{marakatavarṇa}{\lemma{
	{\rm marakatavarṇam\lem}
}\Dfootnote{
	\emn;
	\textit{maragatavarṇṇam} \cod;
	\MM\ edits \textit{maragatavarṇam} without note.
}}miti //}
	%aṣṭā: emend aṣṭau; MM emends aṣṭa.
	%dharmmavajrīm·: emend dharmavajrī, with MM.
	%dvādaśasūcikaṁ pañcavarṇṇākaṁ viśvavajraṁ: emend pañcavarṇakaṁ; MM reads dvādaśasūcikapañcavarṇakaviśvavajram (sil. em.).
	%pṛṣṭato: emend pṛṣṭhato, with MM (sil. em.).
	%maragatavarṇṇam: emend marakatavarṇam; MM edits maragatavarṇam without note.
%
\pend

\bigskip

\pstart\noindent
{\large 3.6.5. Colouring: The Other Deities}
\pend

\bigskip

\pstart
%RT&AG
\Skt{%
\edtext{vajrasattvādyāḥ}{\lemma{
	{\rm vajrasatvādyā\lem}
}\Dfootnote{
	\emn;
	\textit{vajrasatvādyā} \cod;
	\textit{vajrasatvādy ā} \MM
}} sattvaparyaṅkaniṣaṇṇā \edtext{bhadrakalpikaparyantā}{\lemma{
	{\rm bhadrakalpikaparyantā\lem}
}\Dfootnote{
	\MM\ \emn\ \sil;
	\textit{bhadrakalpikaparyantāṁ} \cod
}} \edtext{svamahāmudrayā}{\lemma{%
	{\rm svamahāmudrayā\lem}
}\Dfootnote{%
	\emn;
	\textit{mahāmudrayā} \cod\ \MM
}} svacihnavajrādidhāriṇaḥ praharṣotphullalocanāśca / 
sarve caite vairocanābhimukhā dharmadhātvālamvanajñānasvabhāvatvāt /vairocanaśca tathatāsvabhāva iti //}
	%-payaṅka-: em. -paryaṅka-, with MM (sil. em.).
	%-payantāṁ: em. -paryantā, with MM (sil em.).
\pend

\pstart
%RT&AG:
\Skt{%
tatra vajrasatvo'kṣobhyasya \edtext{purato}{\lemma{%
	{\rm purato\lem}
}\Dfootnote{
	\cod;
	\textit{prato} \MM
}} dakṣiṇapārśve vajrarājo vāme vajrarāgaḥ pṛṣṭhataḥ sādhuḥ / 
evaṃ \edtext{ratnasambhavādīnāṃ}{\lemma{
	{\rm ratnasambhavādīnāṁ\lem}
}\Dfootnote{
	\emn;
	\textit{ratnasambhavādīnā} \cod;
	\textit{ratnasaṁbhavādinā} \MM
}} vajraratnādayaḥ śeṣā jñāyanta eva / 
bāhyamaṇḍalavedikāyāṃ ca \edtext{rūpaṇaṃ}{\lemma{
	{\rm rūpaṇaṁ\lem}
}\Cfootnote{
	emend rūpiṇaṁ? MM emends to rūpaṁ.
}} bhadrakalpikamaitreyādibodhisattvasahasramālikhet paripāṭyā samabhāgato dikṣu pūrvādiṣu /}
	%purato: prato MM.
	%ratnasambhavādīnā: em. ratnasambhavādīnāṁ; ratnasaṁbhavādinā MM.
	%rūpaṇaṁ: emend rūpiṇaṁ? MM emends to rūpaṁ.
	% paripāṭyā |: delete the |.
%
\Skt{tatra pūrvveṇa pañcasūcika\edtext{vajrāyudhāḥ}{\lemma{%
	{\rm -vajrāyudhāḥ\lem}
}\Cfootnote{
	MM emends \textit{-vajradharā}.
}} dakṣiṇena vajraratnadharāḥ paścimena \edtext{vajrapadmadharā}{\lemma{%
	{\rm vajrapadmadharā\lem}
}\Dfootnote{
	\emn\ \MM\ \sil;
	\textit{vajrapadmadharāḥ} \cod\
}} uttareṇa viśvavajradharā iti / atraiṣāṃ niveśanavidyā nāmāni ca bhavanti /
{\om} sarva\edtext{saṃskāra}{\lemma{%
	{\rm -saṁskāra-\lem}
}\Dfootnote{
	\emn\ \MM\ \sil;
	\textit{saskāra} \cod
}\Efootnote{%
	-saṁskāra[40v1]pariśuddha
}}pariśuddha\edtext{dharmate mahānayaparivāre}{\lemma{%
	{\rm -dharmate mahānayaparivāre\lem}
}\Cfootnote{
	\MM\ supplies gaganasamudgate between these two words.
}} svāheti vidyā / tatra nāmāni //}
\pend

\bigskip

\pstart\noindent
{\large 3.6.6. Bhadrakalpikabodhisattvasahasranāmāni}
\pend

\bigskip

\pstart\noindent
[East]
\pend

\medskip

\pstart
\Skt{maitreyaḥ 1 / siṃhaḥ / 1 / pradyotaḥ 2 / muniḥ 4 / kusumaḥ 5 / punarapikusumaḥ 6 / sunetraḥ 7 / sārthavāhaḥ 8 / mahābāhuḥ 9 / mahāvalaḥ 10 //}
\pend

\pstart
\Skt{nakṣatrarājaḥ 1 / oṣadhiḥ 2 / \edtext{yaśāḥ}{\lemma{%
	{\rm yaśāḥ\lem}
}\Dfootnote{%
	\emn\ \MM\ \sil;
	\textit{ṣaśāḥ} \cod
}} 3 / ketuḥ 4 / mahāprabhaḥ 5 / muktiskandhaḥ 6 / vairocanaḥ 7 / sūryagarbhaḥ 8 / candraḥ 9 / \edtext{arciṣmān}{\lemma{%
	{\rm arciṣmān}
}\Dfootnote{
	\emn\ \MM\ \sil;
	\textit{aviṣmān} \cod
}} 10 //}
\pend

\pstart
\Skt{suprabhaḥ 1 / \edtext{aśokaḥ}{\lemma{%
	{\rm aśokaḥ\lem}
}\Dfootnote{
	\emn\ \MM\ \sil;
	\textit{asokaḥ} \cod
}} 2 / tiṣyaḥ 3 / pradyotaḥ 4 / \edtext{mānadhārī}{\lemma{
	{\rm mānadhārī\lem}
}\Cfootnote{
	\MM\ reads/emends \textit{māladhārī}.
}} 5 / guṇaprabhaḥ 6 /  arthadarśīḥ 7 / pradīpaḥ 8 / prabhūtaḥ 9 / vaidyaḥ 10 //}
\pend

\pstart
\Skt{sūrataḥ 1 / \edtext{ūrṇṇāḥ}{\lemma{%
	{\rm ūrṇṇāḥ\lem}
}\Dfootnote{
	\emn\ \MM\ \sil;
	\textit{ūrṇṇāḥ} \cod
}} 2 / dṛḍhaḥ 3 / śrīdevaḥ 4 / \edtext{duṣpradharṣaḥ}{\lemma{
	{\rm duṣpradharṣaḥ\lem}
}\Dfootnote{
	\emn\ \MM\ \sil;
	\textit{duṣyadharśaḥ} \cod
}} 5 / guṇadhvajaḥ 6 / \edtext{rāhulaḥ}{\lemma{%
	{\rm rāhulaḥ\lem}
}\Dfootnote{%
	\emn\ \MM\ \sil;
	\textit{rāhula} \cod
}} 7 / gaṇī 8 / brahmaghoṣaḥ 9 / dṛḍhasandhiḥ 10 //}
\pend

\pstart
\Skt{anunnataḥ \addition{1} / prabhaṅkaraḥ 2 / mahāmeruḥ 3 / vajraḥ 4 / sañjayī / 5 / nirbhayaḥ 6 / ratnaḥ 7 / padmākṣaḥ 8 / valasenaḥ 9 / kusumaraśmiḥ 10 //}
\pend

\pstart
\Skt{jñānapriyaḥ 1 / mahādevaḥ 2 / brahmā 3 / amitābhaḥ 4 / nāgadattaḥ 5 / dṛḍhakramaḥ 6 / amoghadarśī 7 / \edtext{vīryadattaḥ \supplied{8}}{\lemma{}\Efootnote{%
	vīryadattaḥ [41r1] \supplied{8}
}} \supplied{/ bhadrapālaḥ 9 /} nandaḥ 10 //}
	%vīyadattaḥ: em. vīryadattaḥ, with MM (sil. em.).
\pend

\pstart
\Skt{akṣyataḥ 1 / siṅhadhvajaḥ 2 / jayaḥ 3 / dharmaḥ 4 / \edtext{prāmodyarājaḥ}{\lemma{%
	{\rm prāmodyarājaḥ\lem}
}\Dfootnote{
	\cod;
	\textit{pramodyarājaḥ} \MM
}} 5 / sārathiḥ 6 / priyaṅkaraḥ 7 / varuṇaḥ 8 / guṇa\edtext{vāhuḥ}{\lemma{%
	{\rm -vāhuḥ\lem}
}\Dfootnote{
	\cod;
	\textit{-bāhyaḥ} \MM
}} 9 / \edtext{gandhahastī}{\lemma{%
	{\rm gandhahastī\lem}
}\Dfootnote{
	\emn\ \MM\ \sil;
	\textit{gandhahastīḥ} \cod
}} 10 //}
	%prāmodyarājaḥ: pramodyarājaḥ MM.
	%-vāhuḥ: -bāhyaḥ MM.
	%gandhahastīḥ: em. -hastī, with MM (sil. em.).
\pend

\pstart
\Skt{vilocanaḥ 1 / \edtext{meghasvaraḥ}{\lemma{
	{\rm meghasvaraḥ\lem}
}\Dfootnote{
	\emn;
	\textit{meghaśvaraḥ} \cod;
	\textit{meghesvaraḥ} \MM
}} 2 / sucintitaḥ 3 / sumanāḥ 4 / śaśī 5 / \edtext{mahāyaśāḥ}{\lemma{
	{\rm mahāyaśāḥ\lem}
}\Dfootnote{
	\emn;
	\textit{mahāyaśā} \cod;
	\textit{mahāyaśaḥ} \MM
}} 6 / manicūḍaḥ 7 / ugraḥ 8 / siṅhagatiḥ 9 / drumaḥ 10 /}
	%meghaśvaraḥ: em. meghasvaraḥ; meghesvaraḥ MM.
	%mahāyaśā: em. mahāyaśāḥ; mahāyaśaḥ MM.
\pend

\pstart
%RT&AG:
\Skt{\edtext{vijitāvī}{\lemma{%
	{\rm vijitāvī\lem}
}\Dfootnote{
	\emn\ \MM\ \sil;
	\textit{vijatāvī} \cod
}} 1 / prajñākūṭaḥ 2 / susthitaḥ 3 / matiḥ 4 / aṅgajaḥ 5 / amṛtavuddhiḥ 6 / surūpaḥ 7 / jñānī 8 / raśmiḥ 9 / dṛḍhavrataḥ 10 //}
	%vijatāvī: em. vijitāvī, with MM (sil. em.).
\pend

\pstart
\Skt{maṅgalī 1 / satyaketuḥ 2 / padmaḥ 3 / \edtext{nārāyaṇaḥ}{\lemma{%
	{\rm nārāyaṇaḥ\lem}
}\Dfootnote{
	\emn;
	\textit{nārāyanaḥ} \cod;
	\MM\ accepts \textit{nārāyanaḥ}.
}} 4 / sukhavāhuḥ 5 / \edtext{jñānākarabuddhiḥ}{\lemma{%
	{\rm jñānākarabuddhiḥ\lem}
}\Dfootnote{
	\cod;
	\textit{jñānākaraḥ} \MM
}} 6 / guṇārcciḥ 7 / brahmadattaḥ 8 / ratnākaraḥ 9 / kusumadevaḥ 10 //}
	%nārāyanaḥ: em. nārāyaṇaḥ; MM accepts nārāyanaḥ.
	%jñānākaravuddhiḥ: jñānākaraḥ MM.
	%brahmādattaḥ: em. brahmadattaḥ, with MM (sil. em.).\pend
\pend

\pstart
\Skt{sucintitārthaḥ 1 / dharmmeśvaraḥ 2 / \edtext{yaśomatiḥ}{\lemma{%
	{\rm yaśomatiḥ\lem} 
}\Dfootnote{
	\emn\ \MM\ \sil;
	\textit{yasomatiḥ} \cod
}} 3 / \edtext{pratibhāṇakūṭaḥ}{\lemma{%
	{\rm pratibhāṇakūṭaḥ\lem}
}\Dfootnote{
	\emn\ \MM\ \sil;
	\textit{pratibhāṇakūṭa} \cod
}} 4 / vajradhvajaḥ 5 / \edtext{hitaiṣī}{\lemma{%
	{\rm hitaiṣī\lem}
}\Dfootnote{
	\emn\ \MM\ \sil;
	\textit{hitaiśī} \cod
}} 6 / \edtext{vikriḍitāvī}{\lemma{%
	{\rm vikriḍitāvī\lem}
}\Dfootnote{
	\emn;
	\textit{vikriḍitāvī} \cod
}} 7 / \edtext{vigattottamaḥ}{\lemma{
	{\rm vigattottamaḥ \lem}
}\Dfootnote{
	\cod;
	\textit{vigatāndhamaḥ} \MM. Emend?
}} 8 / bahudevaḥ 9 / merudhvajaḥ \edtext{10}{\lemma{%
	{\rm 10\lem}
}\Dfootnote{
	\emn\ \MM\ \sil;
	5 \cod
}} //}
	%yasomatiḥ: em. yaśomatiḥ, with MM (sil em.).
	%-kūṭa: em. -kūṭaḥ, with MM (sil em.).
	%hitaiśī: em. hitaiṣī, with MM (sil. em.).
	%vikriḍitāvī: em. vikrīḍitāvī.
	%vigattottamaḥ: vigatāndhamaḥ MM. Emend?
	%5: emend 10, with MM (sil. em.).
\pend

\pstart
\Skt{gaṇiprabhaḥ 1 / \edtext{vararatnagarbhaḥ}{\lemma{%
	{\rm vararatnagarbhaḥ\lem}
}\Dfootnote{
	\cod;
	\textit{ratnagarbhaḥ} MM
}} 2 / atyuccagāmī 3 / viṣya 4 / maṇiviṣāṇī 5 / \edtext{guṇakīrtiḥ}{\lemma{%
	{\rm guṇakīrtiḥ\lem}
}\Dfootnote{
	\emn\ \MM\ \sil;
	\textit{guṇakīrtti} \cod
}} 6 / candrārkābhaḥ 7 / \edtext{sūryaprabhaḥ}{\lemma{}\Efootnote{%
	7 | [41v] sūryaprabhaḥ
}} 8 / \edtext{jyotīrasaḥ}{\lemma{%
	{\rm jyotīrasaḥ\lem}
}\Dfootnote{
	\emn\ \MM\ \sil;
	\textit{jotīrasaḥ} \cod;
	Note other case of \textit{jotīrasaḥ} spelt this way below.
}} 9 / \edtext{siṃhaketuḥ}{\lemma{%
	{\rm siṁhaketuḥ\lem}
}\Dfootnote{
	\emn\ \MM\ \sil;
	\textit{siṅhaketu} \cod
}} 10 //}
	%vararatnagarbhaḥ: ratnagarbhaḥ MM.
	%viṣya: em. tiṣyaḥ, with MM (sil. em.).
	%guṇakīrtti: em. guṇakīrtiḥ, with MM (sil. em.).
	%jotī-: em. jyotī-, with MM (sil. em.). Note other case of jotīrasaḥ spelt this way below.
	%siṅhaketu: em. siṁhaketuḥ, with MM (sil. em.).
\pend

\pstart
%RT&AG:
\Skt{\edtext{velāmaśrī}{\lemma{%
	{\rm velāmaśrī\lem}
}\Dfootnote{
	\cod;
	\textit{velāmāgrī} \MM
}} 1 / śrīgarbhaḥ 2 / \edtext{bhavāntadarśī}{\lemma{%
	{\rm bhavāntadarśī\lem}
}\Dfootnote{%
	\emn\ \MM\ \sil;
	\textit{bhāvantadarśī} \cod
}} 3 / vidyuprabhaḥ 4 / kanakaparvvataḥ 5 / \edtext{siṁhadattaḥ}{\lemma{%
	{\rm siṁhadattaḥ\lem}
}\Dfootnote{%
	\emn\ \MM\ \sil;
	\textit{sinhadattaḥ} \cod
}} 6 / aparājitadhvajaḥ  7 / jinaḥ 8 / pradyotakīrtiḥ 9 / dṛḍhavīryaḥ 10 //}
	%velāmaśrī: velāmāgrī MM.
	%dṛḍhavīyaḥ: em. dṛḍhavīryaḥ with MM (sil. em.).
	%bhāvantadarśī: em. bhavāntadarśī, with MM (sil. em.).
	%sinhadattaḥ: em. siṁhadattaḥ, with MM (sil. em.).
\pend

\pstart
\Skt{\edtext{anupamakīrtiḥ}{\lemma{%
	{\rm anupamakīrtiḥ\lem}
}\Dfootnote{%
	\emn;
	\textit{anupakīrttiḥ} \cod;
	\MM\ silently emends \textit{anupamākīrtiḥ}.
}} 1 / vigatabhayaḥ 2 / \edtext{arhadevaḥ}{\lemma{%
	{\rm arhadevaḥ\lem}
}\Cfootnote{
	\MM\ silently emends \textit{arhaddevaḥ}. Emend?
}} 3 / mahāpradīpaḥ 4 / lokaprabhaḥ 5 / surabhigandhaḥ 6 / \edtext{guṇāgradhārī}{\lemma{%
	{\rm guṇāgradhārī\lem}
}\Cfootnote{
	\textit{gunāgradhārī} \MM\ (typo).
}} 7 / \edtext{vigatatamāḥ}{\lemma{%
	{\rm vigatatamāḥ\lem}
}\Dfootnote{%
	\emn;
	\textit{vigatamataḥ} \cod;
	\MM\ silently emends \textit{vigatatamaḥ}.
}} 8 / \edtext{siṃhaketuḥ}{\lemma{%
	{\rm siṁhaketuḥ\lem}
}\Dfootnote{
	\emn;
	\textit{sinhahetu} \cod
}\Cfootnote{
	em. \textit{siṁhaketuḥ} or \textit{siṁhahanuḥ};
	\MM\ silently emend \textit{siṁhahanuḥ}.
}} 9 / ratnakīrtiḥ 10 //}
	%°anupakīrttiḥ: em. °anupamakīrtiḥ; MM silently emends anupamākīrtiḥ.
	%°arhadevaḥ: MM silently emends °arhaddevaḥ. Emend?
	%guṇāgradhārī: gunāgradhārī MM (typo). 
	%vi○gatamataḥ: em. vigatatamāḥ; MM silently emends vigatatamaḥ.
	%sinhahetu: em. siṁhaketuḥ or siṁhahanuḥ; MM silently emend siṁhahanuḥ.
\pend

\pstart
\Skt{\edtext{praśāntadoṣaḥ}{\lemma{
	{\rm praśāntadoṣaḥ\lem}
}\Dfootnote{%
	\emn\ \MM\ \sil;
	\textit{praṣāntadoṣaḥ} \cod
}} 1 / manujacandraḥ 2 / sudarśanaḥ 3 / pratimaṇḍiṭaḥ 4 / \edtext{maṇiprabhaḥ}{\lemma{%
	{\rm maṇiprabhaḥ\lem}
}\Dfootnote{%
	\emn\ \MM\ \sil;
	\textit{maṇibhadraḥ} \cod
}} 5 / \edtext{girikūṭaketuḥ}{\lemma{%
	{\rm girikūṭaketuḥ\lem}
}\Dfootnote{
	\emn\ \MM\ \sil;
	\textit{girikūṭaketu} \cod
}} 6 / dharmākaraḥ 7 / artha\edtext{viniścitajñānaḥ}{\lemma{%
	{\rm -viniścitajñānaḥ\lem}
}\Dfootnote{
	\emn\ \MM\ \sil;
	\textit{viniścatajñāna} \cod
}} 8 / āyurdadaḥ 9 / ratnākaraḥ 10 //}
	%praṣānta-: em. praśānta-, with MM (sil. em.).
	%maṇibhadraḥ: maṇiprabhaḥ MM (sil. em.).
	%-viniścatajñāna: em. -viniścitajñānaḥ, with MM (sil. em.).
	%girikūṭaketu: em. girikūṭaketuḥ, with MM (sil. em.).
\pend

\pstart
\Skt{janendrakalpaḥ 1 / \edtext{siṃhavikrāntagāmī}{\lemma{%
	{\rm siṁhavikrāntagāmī\lem}
}\Dfootnote{
	\emn\ \MM\ \sil;
	sinhavikrāntagāmī \cod
}} 2 / sthitabuddhiḥ 3 / vibhāgacchatraḥ 4 / jyeṣṭhaḥ 5 / \edtext{atyudgataśrīḥ}{\lemma{%
	{\rm atyudgataśrīḥ\lem}
}\Cfootnote{%
	should it be \textit{atyudgataśrī}?
}} 6 / siṃhaghoṣaḥ 7 / vikrīḍitāvī 8 / nāgaprabhāsaḥ 9 / kusumaparvataḥ 10 //}
	%sinha-: em. siṁha-, with MM (sil. em.).
	%atyudgataśrīḥ: should it be atyudgataśrī?
\pend

\pstart
\Skt{nāganandī 1 / \supplied{\edtext{gandheśvaraḥ}{\lemma{%
	{\rm gandheśvaraḥ\lem}
}\Dfootnote{%
	\textit{gandeśvaraḥ} \MM
}} 2 /} \edtext{\supplied{mahāya}śāḥ}{\lemma{
	{\rm \supplied{mahāya}śāḥ\lem}
}\Dfootnote{
	\emn;
	\supplied{mahāya}śaḥ \cod;
	MM accepts \textit{mahāyaśaḥ}.
}} 3 / valadevaḥ 4 / guṇamālī 5 / nāgabhujaḥ 6 / \edtext{pratimaṇḍitaḥ}{\lemma{%
	{\rm pratimaṇḍitaḥ\lem}
}\Dfootnote{%
	\cod;
	\textit{netrapratimaṇḍitaḥ} \MM. Emend?
}} 7 / sucīrṇṇavuddhiḥ 8 / jñānavibhuḥ 9 / \edtext{amitalocanaḥ}{\lemma{%
	{\rm amitalocanaḥ\lem}
}\Dfootnote{%
	\cod;
	\textit{anantanāyanaḥ} \MM
}} 10 //}
	%gandheśvaraḥ: gandeśvaraḥ MM.
	%mahāyaśaḥ: em. mahāyaśāḥ; MM accepts mahāyaśaḥ.
	%pratimaṇḍitaḥ: netrapratimaṇḍitaḥ MM. Emend?
	%°amitalocanaḥ: °anantanāyanaḥ MM.
\pend

\bigskip

\pstart
\mbox{}\hfill (folios 42–44 are missing)\hfill\mbox{}
\pend

\bigskip

\pstart\noindent
[South]
\pend

\medskip

\pstart
%\foliobreak{45r1}%
\Skt{\edtext{timān 8}{\lemma{%
	{\rm -timān 8\lem}
}\Dfootnote{%
	\cod;
	\textit{timān 10} \emn\ \MM\ \sil
}\lemma{}\Efootnote{
	(folio missing) [45r1]timān 8
}}  / \edtext{maraskandhaḥ 9}{\lemma{%
	{\rm maraskandhaḥ 9\lem}
}\Dfootnote{
	\cod;
	\textit{maruskandhaḥ} \emn\ \MM\ \sil
}} / \edtext{guṇaguptaḥ 10}{\lemma{%
	{\rm guṇaguptaḥ 10\lem}
}\Dfootnote{
	\cod;
	\textit{guṇaguptaḥ 2} \emn\ \MM \sil
}} / \edtext{arthaṅgamaḥ 1}{\lemma{%
	{\rm arthaṅgamaḥ 1\lem}
}\Dfootnote{
	\cod;
	\textit{arthamatiḥ 3} \MM\ (with citation of ms. variant!)
}\Cfootnote{
	difference between MM numbering and our ms. no longer noted from here on.
}} / acchambhī 2 / \edtext{sthitamitraḥ}{\lemma{%
	{\rm sthitamitraḥ\lem}
}\Dfootnote{%
	\cod;
	\textit{sthiramitraḥ} \MM\ (with citation of ms. variant!)
}} 3 / prabhasthitakalpaḥ 4 / \unclear{maṇica}raṇaḥ 5 / mokṣatejāḥ 6 / śuddhapārśvaḥ 7 / subuddhiḥ 8 /}
	%timān· 8: timān 10 MM (sil. em.).
	%maraskandhaḥ: maruskandhaḥ 1 MM (sil. em.).
	%guṇaguptaḥ 10: guṇaguptaḥ 2 MM (sil. em.).
	%°a○rthaṅgamaḥ 1: arthamatiḥ 3 MM (with citation of ms. variant!).
%%difference between MM numbering and our ms. no longer noted from here on.
	%sthitamittraḥ: sthiramitraḥ MM (with citation of ms. variant!).
	%prabhasthitaka\unclear{lpa}ḥ: prabhāsthitasadṛśaḥ MM.
	%mokṣatejāḥ: mokṣatejaḥ MM.
%	(sundarapā)rśvaḥ
\pend

\pstart
\Skt{\edtext{samatejāḥ}{\lemma{%
	{\rm samatejāḥ\lem}
}\Dfootnote{%
	\cod;
	\textit{samatejaḥ} \MM
}} 9 / jñānavaraḥ 10 / brahmadattaḥ 1 / satyarutaḥ 2 / subuddhiḥ 3 / baladattaḥ 4 / \edtext{siṃhagatiḥ}{\lemma{%
	{\rm siṁhagatiḥ\lem}
}\Dfootnote{
	\emn\ \MM;
	\textit{sinhagatiḥ} \cod
}} 5 / puṣpaketuḥ 6 / jñānākaraḥ 7 / \edtext{puṇyadattaḥ}{\lemma{%
	{\rm puṇyadattaḥ\lem}
}\Dfootnote{
	\cod;
	\textit{puṣpadantaḥ} \MM
}} 8 /}
	%samatejāḥ: samatejaḥ MM.
	%sinhagatiḥ: em. siṁhagatiḥ, with MM.
	%puṇyadattaḥ: puṣpadantaḥ MM.
\pend

\pstart
\Skt{guṇagarbhaḥ 9 / \edtext{yaśoratnaḥ}{\lemma{%
	{\rm yaśoratnaḥ\lem}
}\Dfootnote{%
	\emn\ \MM;
	\textit{yasoratnaḥ} \cod
}} 10 / \edtext{adbhutayaśāḥ}{\lemma{%
	{\rm adbhutayaśāḥ\lem}
}\Dfootnote{
	\cod;
	\textit{adbhutayaśaḥ} \MM
}} 1 / anihatavrataḥ 2 / abhayaḥ 3 / \edtext{prabhamatiḥ}{\lemma{%
	{\rm prabhamatiḥ\lem}
}\Dfootnote{%
	\cod;
	\textit{prabhāmatiḥ} \MM
}} 4 / \edtext{brahmabāhuḥ}{\lemma{%
	{\rm brahmabāhuḥ\lem}
}\Dfootnote{
	\emn\ \MM;
	\textit{prahmabāhuḥ} \cod
}} 5 / \edtext{vikrāntatejāḥ}{\lemma{%
	{\rm vikrāntatejāḥ\lem}
}\Dfootnote{%
	\cod;
	\textit{vikrāntadevaḥ} \emn\ \MM\ \sil
}} \edtext{9}{\lemma{%
	{\rm 9\lem}
}\Cfootnote{
	emend 6?
}} / jñānapriyaḥ 7 / satyadevaḥ 8 /}
	%yasoratnaḥ: em. yaśoratnaḥ, with MM.
	%-yaśāḥ: -yaśaḥ MM.
	%prabhamatiḥ: prabhāmatiḥ MM.
	%prahmabāhuḥ: em. brahmabāhuḥ, with MM.
	%vikrāntatejāḥ: vikrāntadevaḥ MM (sil. em.).
	% 9: emend 6?
\pend

\pstart
\Skt{maṇigarbhaḥ 9 / \edtext{jñānaśrī}{\lemma{%
	{\rm jñānaśrī\lem}
}\Dfootnote{%
	\cod;
	\textit{jñānaśrīḥ} \MM; 
	Should it be \textit{jñānaśrīḥ}?
}} 10 / asitaḥ 1 / \edtext{dṛḍhavrataḥ}{\lemma{%
	{\rm dṛḍhavrataḥ\lem}
}\Dfootnote{%
	\emn\ \MM;
	\textit{dṛḍhacutaḥ} \cod
}} 2 / \edtext{marutejāḥ}{\lemma{%
	{\rm marutejāḥ\lem}
}\Dfootnote{%
	\cod;
	\textit{marutejaḥ} \MM
}} 3 / brahmamuniḥ 4 / samantagāmī 5 / \edtext{prāptavaraḥ}{\lemma{%
	{\rm prāptavaraḥ\lem}
}\Dfootnote{%
	\cod;
	\textit{prāptabaraḥ} \MM
}} 6 / arciskandhaḥ 7 / \edtext{mahātejāḥ}{\lemma{%
	{\rm mahātejāḥ \lem}
}\Dfootnote{%
	\cod;
	\textit{mahātejaḥ} \MM
}} 8 /}
	%jñāna§○śrī: jñānaśrīḥ MM. Should it be jñānaśrīḥ?
	%dṛḍhacutaḥ: dṛḍhavrataḥ MM. Emend dṛḍhavrataḥ.
	%marutejāḥ: marutejaḥ MM.
	%prāptavaraḥ: prāptabaraḥ MM.
	%mahātejāḥ: mahātejaḥ MM.
\pend

\pstart
\Skt{campakagandhaḥ 9 / \edtext{jñānorkaḥ}{\lemma{%
	{\rm jñānorkaḥ\lem}
}\Dfootnote{%
	\cod;
	\textit{jñānolkaḥ} \MM
}} 10 / sugaṇī 1 / indradhvajaḥ 2 / mahāpriyaḥ 3 / \edtext{sumanaprabhaḥ}{\lemma{%
	{\rm sumanaprabhaḥ\lem}
}\Dfootnote{%
	\cod;
	\textit{sumanāprabhaḥ} \MM
}} 4 / gaṇiprabhaḥ 5 / vodhyaṅgaḥ 6 / \edtext{ojaṅgamaḥ}{\lemma{%
	{\rm ojaṅgamaḥ\lem}
}\Dfootnote{
	\cod;
	\textit{ūrjagamaḥ} \MM
}} 7 / \edtext{aniścitārthaḥ}{\lemma{%
	{\rm aniścitārthaḥ\lem}
}\Dfootnote{%
	\cod;
	\textit{suniścitārthaḥ} \MM
}} 8 /}
	%jñā§○norkkaḥ: jñānolkaḥ MM.
	%sumanaprabhaḥ: sumanāprabhaḥ MM.
	%°ojaṅgamaḥ: ūrjagamaḥ MM.
	%°aniścitārthaḥ: suniścitārthaḥ MM.
\pend

\pstart
\Skt{viṣṇudattaḥ 9 / subāhu 10 / \edtext{mahāraśmiḥ}{\lemma{%
	{\rm mahāraśmiḥ\lem}
}\Dfootnote{
	\cod;
	\textit{mahārathiḥ} \MM
}} 1 / \edtext{āśadadaḥ}{\lemma{%
	{\rm āśadadaḥ\lem}
}\Dfootnote{%
	\cod;
	\textit{āśādadaḥ} \MM
}} 2 / puṇyābhaḥ 3 / ratnaketuḥ 4 / vajrasenaḥ 5 / samṛddhaḥ 6 / %\newfolio{45v1}% 
\edtext{siṁhabalaḥ}{\lemma{%
	{\rm siṁhabalaḥ\lem}
}\Dfootnote{%
	\emn;
	\textit{sinhavalaḥ} \cod
}\lemma{}\Efootnote{%
	6 / [45v1] siṁhabalaḥ
}} 7 / vimalanetraḥ 8 /}
	%mahāraśmiḥ: mahārathiḥ MM.
	%āśadadaḥ: āśādadaḥ MM.
	%sinha\unclear{va}laḥ: em. siṁhabalaḥ 7 | 
\pend

\pstart
\Skt{kāśyapaḥ 9 / prasannabuddhiḥ 10 // jñānakramaḥ 1 / \edtext{ugratejāḥ}{\lemma{%
	{\rm ugratejāḥ\lem}
}\Dfootnote{
	\emn;
	\textit{ugratejaḥ} \cod
}} 2 / mahāraśmiḥ 3 / sūryaprabhaḥ 4 / vimalaprabhaḥ 5 / \edtext{vibhaktatejāḥ}{\lemma{%
	{\rm vibhaktatejāḥ\lem}
}\Dfootnote{%
	\emn;
	\textit{vibhakṣatejaḥ} \cod;
	\MM\ emends \textit{vibhaktatejaḥ}.
}} 6 / anupamaḥ 7 / \edtext{madhupātraḥ}{\lemma{%
	{\rm madhupātraḥ\lem}
}\Dfootnote{%
	\cod;
	\textit{madhuvaktraḥ} \MM
}} 8 /}
	%°ugratejaḥ: em. °ugratejāḥ.
	%sūya-: em. sūrya-.
	%vibhakṣatejaḥ: em. vibhaktatejāḥ; MM emends vibhaktatejaḥ.
	%madhupātraḥ: madhuvaktraḥ MM.
\pend

\pstart
\Skt{candraprabhaḥ 9 / \edtext{vidyadattaḥ}{\lemma{%
	{\rm vidyadattaḥ\lem}
}\Cfootnote{%
	emend \textit{vidyuddattaḥ}?
}} 10 / praśāntagāmī 1 / akṣobhyaḥ 2 / arhakīrtiḥ 3 / \edtext{guṇavarmāḥ}{\lemma{%
	{\rm guṇavarmāḥ\lem}
}\Cfootnote{%
	\emn\ \textit{guṇadharmaḥ}, with \MM?
}} 4 / \edtext{laḍitaksetraḥ}{\lemma{%
	{\rm laḍitaksetraḥ\lem}
}\Cfootnote{%
	\emn\ l\textit{alitaksetraḥ}? But see another case of \textit{laḍita-} below.
}} 5 / \edtext{vyūharājaḥ}{\lemma{%
	{\rm vyūharājaḥ\lem}
}\Dfootnote{%
	\emn\ \MM\ \sil;
	\textit{vyūharājāḥ} \cod
}} 6 / abhyudgataḥ 7 / \edtext{hutārciḥ}{\lemma{%
	{\rm hutārciḥ\lem}
}\Dfootnote{%
	\emn\ \MM;
	\textit{hutārcci} \cod
}} 8 /}
	%vidyadattaḥ: emend vidyuddattaḥ?
	%°arhakīrttiḥ: °arhatkīrtiḥ MM.
	%guṇavarmmāḥ: em. guṇadharmaḥ, with MM?
	%laḍitaksetraḥ: em. lalitaksetraḥ? But see another case of laḍita- below.
	%vyūharājāḥ: em. vyūharājaḥ, with MM (sil. em.).
	%hutārcci: em. hutārciḥ, with MM.
\pend

\pstart
\Skt{\edtext{padmaśrīḥ}{\lemma{%
	{\rm padmaśrīḥ\lem}
}\Cfootnote{%
	should it be \textit{padmaśrī}?
}} 9 / \edtext{ratnabāhuḥ}{\lemma{%
	{\rm ratnabāhuḥ\lem}
}\Dfootnote{%
	\cod;
	\textit{ratnavyūhaḥ} \emn\ \MM
}} 10 / subhadraḥ 1 / ratnottamaḥ 2 / sumeruḥ 3 / amitaprabhaḥ 4 / samudradattaḥ 5 / brahmaketuḥ 6 / \edtext{somakecchaḥ}{\lemma{%
	{\rm somakecchaḥ\lem}
}\Cfootnote{%
	\MM\ reads \textit{soma(katruḥ)} and emends \textit{somacchattraḥ}. We propose emending \textit{somaketuḥ}.
}} 7 / arciṣmān 8 /}
	%padmaśrīḥ: should it be padmaśrī?
	%ratnavāhuḥ: MM em. ratnavyūhaḥ.
	%somakecchaḥ: MM reads soma(katruḥ) and emends somacchattraḥ. We propose emending somaketuḥ.
\pend

\pstart
\Skt{velāmarājaḥ 9 / etāni dakṣiṇasyāṃ diśi \edtext{vedikābhyantare}{\lemma{%
	{\rm vedikābhyantare\lem}
}\Dfootnote{%
	\emn;
	\textit{vedikābhyāntare} \cod
}} lekhyāni // //}
	%vedikābhyāntare: em. vedikābhyantare.
\pend

\medskip

\pstart\noindent
[West]
\pend

\medskip

\pstart
\Skt{jñānakīrtiḥ 1 / \edtext{sañjayī}{\lemma{%
	{\rm sañjayī\lem}
}\Dfootnote{%
	\cod;
	\textit{samjayī} \MM\ (misprint for \textit{saṁjayī})
}} 2 / guṇaprabhaḥ 3 / vighuṣṭaśabdaḥ 4 / pūrṇacandraḥ 5 / \edtext{padmaraśmiḥ}{\lemma{%
	{\rm padmaraśmiḥ\lem}
}\Dfootnote{%
	\emn;
	\textit{padmarasmiḥ} \cod
}} 6 / suvrataḥ 7 / \edtext{pradīparājaḥ}{\lemma{%
	{\rm pradīparājaḥ\lem} 
}\Dfootnote{%
	\emn;
	\textit{pradīparājāḥ} \cod
}} 8 / \textcolor{red}{vidyuketuḥ} 9 / raśmirājaḥ 10 //}
	%sañjayī: samjayī MM (misprint for saṁjayī).
	%padmarasmiḥ: em. padmaraśmiḥ.
	%pradīparājāḥ: em. pradīparājaḥ
\pend

\pstart
\Skt{\edtext{jyotiṣkaḥ}{\lemma{%
	{\rm jyotiṣkaḥ}
}\Dfootnote{%
	\emn\ \MM;
	\textit{jotiṣkaḥ} \cod
}} 1 / anupamakīrtiḥ 2 / padmagarbhaḥ 3 / \edtext{puṣpaḥ}{\lemma{%
	{\rm puṣpaḥ\lem}
}\Dfootnote{%
	\cod;
	\textit{puṣyaḥ} \MM
}} 4 / cārulocanaḥ 5 / anāvilārthaḥ 6 / ugrasenaḥ 7 / \edtext{puṇyatejāḥ}{\lemma{%
	{\rm puṇyatejāḥ\lem}
}\Dfootnote{%
	\emn;
	\textit{puṇyatejaḥ} \cod;
	\MM\ retains the ms. reading.
}} 8 / vikramaḥ 9 / asaṅgabuddhiḥ 10 //}
	%jotiṣkaḥ: em. jyotiṣkaḥ, with MM.
	%puṣpaḥ: puṣyaḥ MM.
	%puṇyatejaḥ: em. puṇyatejāḥ; MM retains the ms. reading.
\pend

\pstart
\Skt{rāhudevaḥ 1 / \edtext{jñānaraśmiḥ 2 / sārathiḥ}{\lemma{}\Efootnote{%
	ñānaraśmiḥ 2 | [46r1] sārathiḥ
}} 3 / \edtext{janendraḥ}{\lemma{%
	{\rm janendraḥ\lem}
}\Cfootnote{%
	\MM\ emends \textit{janendrakalpaḥ}. Emend \textit{jinendra(kalpa)ḥ}?
}} 4 / puṣpaketuḥ 5 / rāhulaḥ 6 / mahauṣadhiḥ 7 / nakṣatrarājaḥ 8 / vaidyarājaḥ 9 / puṇyahastī 10 //}
	%janendraḥ: MM emends janendrakalpaḥ. Emend jinendra(kalpa)ḥ?
\pend

\pstart
\Skt{\edtext{tacchakaḥ}{\lemma{%
	{\rm tacchakaḥ\lem}
}\Cfootnote{%
	\MM\ emends \textit{takṣakaḥ}.
}} 1 / \edtext{vighuṣṭaśabdaḥ}{\lemma{%
	{\rm vighuṣṭaśabdaḥ\lem}
}\Cfootnote{%
	\MM\ emends \textit{vighuṣṭarājaḥ}.
}} 2 / sūryarasmiḥ 3 / dharmakośaḥ 4 / sumatiḥ 5 / \edtext{gaṇendrakalpaḥ 7}{\lemma{%
	{\rm gaṇendrakalpaḥ 7\lem}
}\Cfootnote{%
	\MM\ emends \textit{guṇendrakalpaḥ 6, vajrasenaḥ 7}. The item \textit{vajrasenaḥ} is supplied by \MM\ from Tib.
}} / prajñākūṭaḥ 8 / susthitaḥ 9 / sucīrṇabuddhiḥ 10 //}
	%tacchakaḥ: MM emends takṣakaḥ.
	%vighuṣṭaśavdaḥ: MM emends vighuṣṭarājaḥ.
	%sūyarasmiḥ: emend sūryaraśmiḥ.
	%gaṇendrakalpaḥ 7: MM em. guṇendrakalpaḥ 6, vajrasenaḥ 7. The item vajrasenaḥ is supplied by MM from Tib.
\pend

\pstart
\Skt{\edtext{brahmaghoṣaḥ}{\lemma{%
	{\rm brahmaghoṣaḥ\lem}
}\Dfootnote{%
	\emn\ \MM
	\textit{vrahmāghoṣaḥ} \cod
}} 1 / nāgaḥ 2 / guṇagarjita\edtext{svaraḥ}{\lemma{%
	{\rm -svaraḥ\lem}
}\Dfootnote{%
	\emn\ \MM;
	\textit{-śvaraḥ} \cod
}} 3 / abhijñāketuḥ 4 / \edtext{prabhāketuḥ}{\lemma{%
	{\rm prabhāketuḥ\lem}
}\Dfootnote{%
	\emn\ \MM;
	\textit{prabhaketuḥ} \cod
}} 5 / kṣemaḥ 6 / brahmā 7 / puṅgavaḥ 8 / laḍitanetraḥ 9 / nāgadattaḥ 10 //}
	%vrahmāghoṣaḥ: em. brahmaghoṣaḥ, with MM.
	%-śvaraḥ: em. -svaraḥ, with MM.
	%prabhaketuḥ: em. prabhāketuḥ, with MM.
\pend

\pstart
\Skt{satyaketuḥ 1 / paṇḍitaḥ 2 / ajitaghoṣaḥ 3 / ratnaprabhaḥ 4 / ghoṣadattaḥ 5 / \edtext{keśarī}{\lemma{%
	{\rm keśarī\lem}
}\Dfootnote{%
	\cod;
	\textit{kesarī} \MM
}} 6 / vicitrabhāṇī 7 / jñāna\edtext{śūraḥ}{\lemma{%
	{\rm -śūraḥ\lem}
}\Dfootnote{%
	\emn\ \MM;
	\textit{-sūraḥ} \cod
}} 8 / padmarāśiḥ 9 / puṣpitaḥ 10 //}
	%keśarī: kesarī MM.
	%-sūraḥ: em. -śūraḥ, with MM.
\pend

\pstart
\Skt{vikrāntagāmī 1 / puṇya\edtext{rāśiḥ}{\lemma{%
	{\rm -rāśiḥ\lem}
}\Dfootnote{%
	\emn\ \MM;
	\textit{-rāṣiḥ} \cod
}} 2 / śreṣṭhabuddhiḥ 3 / jyotiṣaḥ 4 / candrapradīpaḥ 5 / akṣayaḥ 6 / subuddhinetraḥ 7 / \edtext{prīṇitāṅgaḥ}{\lemma{%
	{\rm prīṇitāṅgaḥ\lem}
}\Dfootnote{%
	\emn;
	\textit{prīṇibhāṅgaḥ} \cod;
	\textit{prīnītāngaḥ} \MM
}} 8 / \edtext{prajñārāṣṭraḥ}{\lemma{%
	{\rm prajñārāṣṭraḥ\lem}
}\Dfootnote{%
	\emn\ \MM;
	\textit{prajñārāṣṭrāḥ} \cod
}} 9 / uttamaḥ 10 //}
	%-rāṣiḥ: -rāśiḥ MM.
	%prīṇibhāṅgaḥ: em. prīṇitāṅgaḥ; prīnītāngaḥ MM.
	%prajñārāṣṭrāḥ: em. prajñārāṣṭraḥ, with MM.
\pend

\pstart
\Skt{\edtext{prahāṇatejāḥ}{\lemma{%
	{\rm prahāṇatejāḥ\lem}
}\Dfootnote{%
	\emn;
	\textit{prahāṇatejaḥ} \cod;
	\textit{prahānatejaḥ} \MM
}} 1 / prajñādattaḥ 2 / mañjughoṣaḥ 3 / \edtext{asaṅgakośaḥ}{\lemma{%
	{\rm asaṅgakośaḥ\lem}
}\Dfootnote{%
	\cod;
	\textit{asaṅgakosaḥ} \MM
}} 4 / \edtext{jyeṣṭhadattaḥ}{\lemma{%
	{\rm jyeṣṭhadattaḥ\lem}
}\Dfootnote{%
	\emn\ \MM;
	\textit{jyeṣṭadattaḥ} \cod
}} 5 / śreṣṭhaḥ 6 / jñānavikramaḥ 7 / pārthivaḥ 8 / vimuktigāmī 9 //\edtext{}{\lemma{}\Cfootnote{%
	At the en of this section, MM has \textit{arciṣmān 8, pārthivaḥ 9, vegadharī 10}. A complex scribal error, with two skips forward and a skip back to the wrong place must have occurred to explain the text as it stands in our manuscript.
}} }
	%prahāṇatejaḥ: em. prahāṇatejāḥ; prahānatejaḥ MM.
	%°asa(ṅgako)śaḥ: °asaṅgakosaḥ MM.
	%jyeṣṭadattaḥ: em. jyeṣṭhadattaḥ, with MM.
	%At the en of this section, MM has °arciṣmān 8, pārthivaḥ 9, vegadharī 10. A complex scribal error, with two skips forward and a skip back to the wrong place must have occurred to explain the text as it stands in our manuscript.
\pend

\pstart
\Skt{śrīḥ 3 / surūpaḥ 4 / rā\addition{jaḥ} 5 / arthasiddhiḥ 6 / \edtext{siṃhasenaḥ 7 / \edtext{vāśakaḥ}{\lemma{
	{\rm vāśakaḥ\lem}
}\Dfootnote{%
	\cod;
	\textit{vāsavaḥ} \MM, with note indicating that he reads \textit{vaśakaḥ} or \textit{vāśikaḥ}.
}}}{\lemma{}\Efootnote{%
	siṁhasenaḥ 7 | [46v1] vāśakaḥ
}} 8 / \edtext{yaśāḥ}{\lemma{%
	{\rm yaśāḥ\lem}
}\Dfootnote{%
	\cod;
	\textit{yaśaḥ} \MM
}} 9 / jayaḥ 10 //}
	%At the beginning of this section. MM has tiṣyaḥ 1, suprabhaḥ 2. These elements seem to have been skipped in our manuscript.
	%\unclear{vā}śakaḥ: vāsavaḥ MM, with note indicating that he reads vaśakaḥ or vāśikaḥ.
	%yaśāḥ: yaśaḥ MM.
\pend

\pstart
\Skt{udāragarbhaḥ 1 / puṇyaraśmiḥ 2 / \edtext{suvarṇaprabhaḥ}{\lemma{%
	{\rm suvarṇaprabhaḥ\lem}
}\Dfootnote{%
	\cod;
	\textit{suvarṇaprabhā} \MM
}} 3 / \edtext{śrotriyaḥ}{\lemma{%
	{\rm śrotriyaḥ\lem}
}\Dfootnote{%
	\emn\ \MM;
	\textit{sotriyaḥ} \cod
}} 4 / \edtext{avaivartikapradīparājaḥ}{\lemma{%
	{\rm avaivartikapradīparājaḥ\lem}
}\Dfootnote{%
	\cod;
	\textit{pradīparājaḥ} \MM
}} 5 / ratnakūṭaḥ 6 / asaṅgadevaḥ 7 /  + + + 8 / vimuktigāmī 9 / suvarṇacūḍaḥ 10 //}
	%suvarṇṇaprabhaḥ: suvarṇaprabhā MM.
	%sotriyaḥ: em. śrotriyaḥ, with MM.
	%°avaivarttikapradīparājaḥ: pradīparājaḥ MM.
\pend

\pstart
\Skt{rāhulabhadraḥ 1 / \edtext{durjayaḥ}{\lemma{%
	{\rm durjayaḥ\lem}
}\Dfootnote{%
	\cod;
	\textit{durkṣayaḥ} \MM
}} 2 / muniprasannaḥ 3 / somaraśmiḥ 4 / kāñcanasuvarṇaprabhaḥ 5 / sudattaḥ 6 / guṇendradevaḥ 7 / dharmacchatraḥ 8 / puṇyavāhuḥ 9 / \edtext{aṅgamaḥ}{\lemma{%
	{\rm aṅgamaḥ}
}\Dfootnote{%
	\cod;
	\textit{aṅgadaḥ} \MM
}} 10 //}
	%durjjayaḥ: durkṣayaḥ MM.
	%aṅgama○ḥ: aṅgadaḥ MM.
\pend

\pstart
\Skt{praṇītajñānaḥ 1 / sūkṣmabuddhiḥ 2 \edtext{sarvatejāḥ}{\lemma{%
	{\rm sarvatejāḥ\lem}
}\Dfootnote{%
	\emn;
	\textit{sarvatejaḥ} \cod\ \MM
}} 3 / oṣadhiḥ 4 / vimuktiketuḥ 5 / prajñākośaḥ 6 / \edtext{jñānarājaḥ}{\lemma{%
	{\rm jñānarājaḥ\lem}
}\Dfootnote{%
	\emn\ \MM;
	\textit{jñānarājāḥ} \cod
}} 7 / bhīṣmarūpaḥ 8 / oghakṣayaḥ 9 / asaṅgakīrtiḥ 10 //}
	%sarvatejaḥ: em. sarvatejāḥ; sarvatejaḥ MM.
	%jñānarājāḥ: em. jñānarājaḥ, with MM.
\pend

\pstart
\Skt{satyarāśiḥ 1 / \edtext{susvaraḥ}{\lemma{%
	{\rm susvaraḥ\lem}
}\Dfootnote{%
	\emn\ \MM;
	\textit{suśvaraḥ} \cod
}} 2 / \edtext{girīndrakalpaḥ}{\lemma{%
	{\rm girīndrakalpaḥ\lem}
}\Dfootnote{%
	\emn\ \MM;
	\textit{girindrakalpaḥ} \cod
}} 3 / dharmakūṭaḥ 4 / \edtext{mokṣatejāḥ}{\lemma{%
	{\rm mokṣatejāḥ\lem}
}\Dfootnote{%
	\cod;
	\textit{mokṣatejaḥ} \MM
}} 5 / \edtext{śobhitaḥ}{\lemma{%
	{\rm śobhitaḥ\lem}
}\Dfootnote{%
	\emn\ \MM;
	\textit{sobhitaḥ} \cod
}} 6 / praśāntagātraḥ 7 / \edtext{manojñavākyaḥ}{\lemma{%
	{\rm manojñavākyaḥ\lem}
}\Dfootnote{%
	\emn\ \MM;
	\textit{manojñavakyaḥ} \cod
}} 8 / \edtext{siṃhadaṃṣṭraḥ}{\lemma{%
	{\rm siṁhadaṁṣṭraḥ\lem}
}\Dfootnote{%
	\emn\ \MM;
	\textit{sinhadraṁṣṭraḥ} \cod
}} 9 / \edtext{vāraṇaḥ}{\lemma{%
	{\rm vāraṇaḥ\lem}
}\Dfootnote{%
	\cod;
	\textit{varuṇaḥ} \MM
}} 10 //}
	%suśvaraḥ: em. susvaraḥ, with MM.
	%girindrakalpaḥ: em. girīndrakalpaḥ, with MM.
	%mokṣatejāḥ: mokṣatejaḥ MM.
	%sobhitaḥ: em. śobhitaḥ, with MM.
	%manojñavakyaḥ: em. manojñavākyaḥ, with MM.
	%sinhadraṁṣṭraḥ: em. siṁhadaṁṣṭraḥ, with MM.
	%vāraṇaḥ: varuṇaḥ MM.
\pend

\pstart
\Skt{\edtext{jagatpūjitaḥ}{\lemma{%
	{\rm jagatpūjitaḥ\lem}
}\Dfootnote{%
	\emn\ \MM;
	\textit{jābhatpūjitaḥ} \cod
}} 1 / siṃhapārśvaḥ 2 / dharmavikrāmī 3 / subhagaḥ 4 / akṣobhyavarṇaḥ 5 / tejorājaḥ 6 / bodhanaḥ 7 / \edtext{sulocanaḥ}{\lemma{%
	{\rm sulocanaḥ\lem}
}\Dfootnote{%
	\cod;
	\textit{sunetrī} \MM
}} 8 / \edtext{vicitārthabuddhiḥ}{\lemma{%
	{\rm vicitārthabuddhiḥ\lem}
}\Dfootnote{%
	\cod;
	\textit{sthitārthabuddhiḥ} \MM, with note indicating he reads \textit{vicintārthabuddhiḥ}. Tib. perhaps supports \textit{vicinta-}.
}} 9 / \edtext{abhayaraśmiḥ}{\lemma{%
	{\rm abhayaraśmiḥ\lem}
}\Dfootnote{%
	\cod;
	\textit{abhabhāsaraśmiḥ} \MM
}} 10 //}
	%jābhatpūjitaḥ: em. jagatpūjitaḥ, with MM.
	%su\unclear{loca}naḥ: sunetrī MM.
	%vicittārthavuddhiḥ: sthitārthabuddhiḥ MM, with note indicating he reads vicintārthabuddhiḥ. Tib. perhaps support vicinta-.
	%°abhaya○raśmiḥ: °abhabhāsaraśmiḥ MM.
\pend

\pstart
\Skt{\edtext{gandhatejāḥ}{\lemma{%
	{\rm gandhatejāḥ\lem}
}\Dfootnote{%
	\cod;
	\textit{gandhatejaḥ} \MM
}} 1 / toṣaṇaḥ 2 / \edtext{amoghagāmī}{\lemma{%
	{\rm amoghagāmī\lem}
}\Dfootnote{%
	\emn\ \MM;
	\textit{amoghagāmīḥ} \cod
}} 3 / bhasmakrodhaḥ 4 / vararūpaḥ 5 / \edtext{sukramaḥ}{\lemma{%
	{\rm sukramaḥ\lem}
}\Dfootnote{%
	\emn\ \MM;
	\textit{sakamaḥ} \cod
}} 6 / \edtext{pradānakīrtiḥ}{\lemma{}\Efootnote{%
	pradā[47r1]nakīrtiḥ
}} 7 / śuddhidattaḥ 8 / devasūryaḥ 9 / prajñālokaḥ 10 //}
	%gandhatejāḥ: gandhatejaḥ MM.
	%°amoghagāmīḥ: em. °amoghagāmī, with MM.
	%sakamaḥ: em. sukramaḥ, with MM.
	%devasūyaḥ: em. devasūryaḥ, with MM.
\pend

\pstart
\Skt{\edtext{samāhitaḥ}{\lemma{%
	{\rm samāhitaḥ\lem}
}\Dfootnote{%
	\emn\ \MM;
	\textit{samāhitāṁ} \cod
}} 1 / \edtext{ojatejāḥ}{\lemma{%
	{\rm ojatejāḥ\lem}
}\Dfootnote{%
	\cod;
	\textit{ojatejaḥ} \MM
}} 2 / kṣatriyaḥ 3 / bhāgīrathī 4 / \edtext{suvarṇadāmā}{\lemma{%
	{\rm suvarṇadāmā\lem}
}\Dfootnote{%
	\emn;
	\textit{suvarṇṇadāmāḥ} \cod; 
	\textit{suvarṇadāmam} \MM
}} 5 / \edtext{vimukticūḍaḥ}{\lemma{%
	{\rm vimukticūḍaḥ\lem}
}\Dfootnote{%
	\emn\ \MM;
	\textit{vimukticūḍāḥ} \cod
}} 6 /  samṛddhaḥ 7 / sthitagatiḥ 8 / madaprahīṇaḥ 9 . jñānakośaḥ 10 //}
	%samāhitāṁ: em. samāhitaḥ, with MM.
	%°ojatejāḥ: °ojatejaḥ MM.
	%suvarṇṇadāmāḥ: em. suvarṇadāmā; suvarṇadāmam MM.
	%vimukticūḍāḥ: em. vimukticūḍaḥ, with MM.
\pend

\pstart
\Skt{\edtext{brahmagāmī}{\lemma{%
	{\rm brahmagāmī\lem}
}\Dfootnote{%
	\emn\ \MM;
	\textit{vrahmagāmīḥ} \cod
}} 1 / candanaḥ 2 | \edtext{aśokaḥ}{\lemma{%
	{\rm aśokaḥ\lem}
}\Dfootnote{%
	\emn\ \MM;
	\textit{asokaḥ} \cod
}} 3 / siṃharaśmiḥ 4 / keturāṣṭraḥ 5 / padmagarbhaḥ 6 / \edtext{anantatejāḥ}{\lemma{%
	{\rm anantatejāḥ\lem}
}\Dfootnote{%
	\cod;
	\textit{anantatejaḥ} \MM
}} 7 / devaraśmiḥ 8 / puṇyapuṣpaḥ 9 / vidrumaḥ 10 //}
	%vrahmagāmīḥ: em. brahmagāmī, with MM.
	%°anantatejāḥ: °anantatejaḥ MM.
	%°asokaḥ: em. °aśokaḥ, with MM.
\pend

\pstart
\Skt{samṛddhajñānaḥ 1 / brahmavasuḥ 2 / ratnapāṇiḥ 3 / \edtext{indramaḥ}{\lemma{%
	{\rm indramaḥ\lem}
}\Dfootnote{%
	\cod;
	\textit{indrama} \MM; 
	Emend \textit{indraḥ}?
}} 4 / anupamavādī 5 / \edtext{jyeṣṭhavādī}{\lemma{%
	{\rm jyeṣṭhavādī\lem}
}\Dfootnote{%
	\emn\ \MM;
	\textit{jyeṣṭavādī} \cod
}} 6 / \edtext{pūjyaḥ}{\lemma{%
	{\rm pūjyaḥ\lem}
}\Dfootnote{%
	\emn;
	\textit{pūjya\unclear{|}ḥ} \cod
}} 7 / \edtext{tejāḥ}{\lemma{%
	{\rm tejāḥ\lem}
}\Dfootnote{%
	\cod;
	emend \textit{tejaḥ}, with \MM?
}} 8 / sūryaḥ 9 / \edtext{uttīrṇapaṅkaḥ}{\lemma{%
	{\rm uttīrṇapaṅkaḥ\lem}
}\Dfootnote{%
	\emn\ \MM;
	\edtext{utīrṇapaṅkaḥ} \cod
}} 10 //}
	%°indramaḥ: °indrama MM. Emend °indraḥ?
	%jyeṣṭavādī: em. jyeṣṭhavādī, with MM.
	%tejāḥ: em. tejaḥ, with MM?
	%sūyaḥ: em. sūryaḥ, with MM.
	%°utīrṇapaṅkaḥ: em. °uttīrṇapaṅkaḥ, with MM.
\pend

\pstart
\Skt{jñānaprabhāsaḥ 1 / siddhaḥ 2 / māyuraḥ 3 / dhārmikaḥ 4 / hitaiṣirūpaḥ 5 / jñānī 6 / \edtext{yaśaḥ}{\lemma{%
	{\rm yaśaḥ\lem}
}\Dfootnote{%
	\emn\ \MM;
	\textit{yaśā} \cod
}} 7 / jālaraśmiḥ 8 / \supplied{vi}jitaḥ 9 / vaidūryagarbhaḥ 10 //}
	%yaśā: em. yaśaḥ, with MM.
	%vaidūyagarbhaḥ: em. vaidūryagarbhaḥ, with MM.
\pend

\pstart
\Skt{puṣpaḥ 1 / devaḥ 2 / śaśī 3 / smṛtiprabhaḥ 4 / kuśalaprabhaḥ 5 / sarvaguṇaprabhaḥ 6 / \edtext{ratnaśrīḥ}{\lemma{%
	{\rm ratnaśrīḥ\lem}
}\Dfootnote{%
	should it be \edtext{ratnaśrī}?
}} 7 / guṇacandraḥ 8 / rāhuḥ \supplied{9 / a}mṛtaprabhaḥ 10 //}
	%ratnaśrīḥ: should it be ratnaśrī?
\pend

\pstart
\Skt{\edtext{sumanāḥ}{\lemma{%
	{\rm sumanāḥ\lem}
}\Dfootnote{%
	\cod;
	\textit{sumanaḥ} \MM
}} 1 / jyotiprabhaḥ 2 / \edtext{kasanaśivaḥ}{\lemma{%
	{\rm kasanaśivaḥ\lem}
}\Dfootnote{%
	\cod;
	\textit{gamanaśivaḥ} \MM
}} 3 / jñānasāgaraḥ 4 / dharmeśvaragiriḥ 5 / keśavilokitaḥ 6 / \edtext{guṇakotiva\supplied{laḥ}}{\lemma{}\Efootnote{%
	guṇakotiva[47v1]\supplied{laḥ}
}} 7 / surendraḥ 8 / sughoṣaḥ 9 / supārśvaḥ 10 //}
	%sumanāḥ: sumanaḥ MM.
	%kasanaśivaḥ: gamanaśivaḥ MM.
\pend

\pstart
\Skt{\edtext{sthitārthaḥ}{\lemma{%
	{\rm sthitārthaḥ\lem}
}\Dfootnote{%
	\emn\ \MM;
	\textit{sthitāthaḥ} \cod
}} 1 / \edtext{guṇatejaḥśrīḥ}{\lemma{%
	{\rm guṇatejaḥśrīḥ\lem}
}\Dfootnote{%
	\emn\ \MM;
	\textit{guṇatejāḥśrī} \cod;
	or \textit{guṇatejaḥśrī}?
}} 2 / \edtext{asamantajñānī}{\lemma{%
	{\rm asamantajñānī\lem}
}\Dfootnote{%
	\cod;
	\textit{asamatājñānī} \MM;
	Emend \textit{samantajñānī} or \textit{asamajñānī}?
}} 3 / \edtext{amitaśuraḥ}{\lemma{%
	{\rm amitaśuraḥ\lem}
}\Dfootnote{%
	\cod;
	\textit{amitasvaraḥ} \MM;
	Emend \textit{amitaśūraḥ}?
}} 4 / sukhābhaḥ 5 / \edtext{sumedhaḥ}{\lemma{%
	{\rm sumedhaḥ\lem}
}\Dfootnote{%
	\emn\ \MM;
	\textit{sumedhāḥ} \cod
}} 6 / \edtext{vimokṣā+\supplied{ci}ntī}{\lemma{%
	{\rm vimokṣā+\supplied{ci}ntī\lem}
}\Cfootnote{%
	\cod;
	\textit{vimokṣamohārthacintī} \MM. There doesn't seem to be enough space in the gap to justify \MM's reconstruction based on Tib and Ch. It is possible that a tiny part of \textit{ci} and a bit more of \textit{ntī} remain visible before the number 7. Restore \textit{vimokṣārthacintī}?
}} 7 / \edtext{viśiṣṭhasvaraḥ}{\lemma{%
	{\rm viśiṣṭhasvaraḥ\lem}
}\Dfootnote{%
	\emn\ \MM;
	\textit{visiṣṭhasvaraḥ} \cod
}} 8 / laḍitāgragāmī 9 / \edtext{śamathaḥ}{\lemma{%
	{\rm śamathaḥ\lem}
}\Dfootnote{%
	\cod;
	\textit{śantārthaḥ} \MM\ (sic)
}} 10 //}
	%sthitāthaḥ: em. sthitārthaḥ, with MM.
	%guṇatejāḥśrī: em. guṇatejaḥśrīḥ, with MM, or guṇatejaḥśrī?
	%°asamantajñānī: °asamatājñānī MM. Emend samantajñānī or asamajñānī?
	%°amitaśuraḥ: °amitasvaraḥ MM. Emend °amitaśūraḥ?
	%sumedhāḥ: em. sumedhaḥ, with MM.
	%vimokṣā§ + + + : vimokṣamohārthacintī MM. There doesn't seem to be enough space in the gap to justify MM's reconstruction based on Tib and Ch. It is possible that a tiny part of ci and a bit more of ntī remain visible before the number 7. Restore vimokṣārthacintī?
	%visiṣṭhasvaraḥ: em. viśiṣṭhasvaraḥ, with MM.
	%śamathaḥ: śantārthaḥ MM (sic).
\pend

\pstart
\Skt{adoṣaḥ 1 / śubhacīrṇabuddhiḥ 2 / padmottaraḥ 3 / sūryaraśmiḥ 4 / \edtext{pratibhānavarṇaḥ}{\lemma{%
	{\rm pratibhānavarṇaḥ\lem}
}\Dfootnote{%
	\emn\ \MM;
	\textit{pratibhāṇavarṇṇaḥ} \cod
}} 5 / sutīrthaḥ 6 / \edtext{guṇendraḥ}{\lemma{%
	{\rm guṇendraḥ\lem}
}\Dfootnote{%
	\cod;
	\textit{gaṇendraḥ} \MM. Emend?
}} 7 / vigatabhayaḥ 8 / jñānaruciḥ 9 / \edtext{pratibhānacakṣuḥ}{\lemma{%
	{\rm pratibhānacakṣuḥ\lem}
}\Dfootnote{%
	\emn\ \MM;
	\textit{pratibhāṇacakṣuḥ} \cod
}} 10 //}
	%sūyaraśmiḥ: em. sūryaraśmiḥ, with MM.
	%pratibhāṇavarṇṇaḥ: em. pratibhānavarṇaḥ, with MM.
	%guṇendraḥ: gaṇendraḥ MM. Emend?
	%pratibhāṇacakṣuḥ: em. pratibhānacakṣuḥ, with MM.
\pend

\pstart
\Skt{\edtext{varabuddhiḥ}{\lemma{%
	{\rm varabuddhiḥ\lem}
}\Dfootnote{%
	\cod;
	\textit{valabuddhiḥ} \MM
}} 1 / śaśī 2 / ratnābhacandraḥ 3 / abhayaḥ 4 / mahādarśanaḥ 5 / brahmarutaḥ 6 \textcolor{red}{\erased{ḥ}} / sughoṣah 7 / mahāprajñātīrthaḥ 8 / samantabuddhiḥ 9 / vajrasaṃhatabuddhiḥ 10 //}
	%varabuddhiḥ: valabuddhiḥ MM.
\pend

\pstart
\Skt{buddhimatiḥ 1 / drumendraḥ 2 / ghoṣeśvaraḥ 3 / \edtext{puṇyabāhuḥ}{\lemma{%
	{\rm puṇyabāhuḥ\lem}
}\Dfootnote{%
	\cod;
	\textit{puṇyabalaḥ} \MM;
	Emend?
}} 4 / \edtext{sthāmaśrīḥ}{\lemma{%
	{\rm sthāmaśrīḥ\lem}
}\Cfootnote{%
	should it be \textit{sthāmaśrī}?
}} 5 / āryapriyaḥ 6 / pratāpaḥ 7 / \edtext{jyotīrasaḥ}{\lemma{%
	{\rm jyotīrasaḥ\lem}
}\Dfootnote{%
	\emn;
	\textit{jotīrasaḥ} \cod\ \MM;
	Note other case of \textit{jotīrasaḥ} spelt this way above.
}} 8 / \edtext{duṃdubhimeghasvararājaḥ}{\lemma{%
	{\rm duṁdubhimeghasvararājaḥ\lem}
}\Dfootnote{%
	\cod;
	\textit{dundubhimeghasvaraḥ} \MM
}} 9 / \edtext{ekona}{\lemma{%
	{\rm ekonam\lem}
}\Dfootnote{%
	\emn;
	\textit{ekoṇam} \cod
}}mardhatṛtīyaṃ śatamidaṃ paścimāyāṃ diśi vedikābhyāntare lekhyam //  //}
	%puṇyavāhuḥ: puṇyabalaḥ MM. Emend?
	%sthāmaśrīḥ: should it be sthāmaśrī?
	%°āyapriyaḥ: em. °āryapriyaḥ, with MM.
	%jotīrasaḥ: thus MM; em. jyotīrasaḥ. Note other case of jotīrasaḥ spelt this way above.
	%duṁdubhimeghasvararājaḥ: dundubhimeghasvaraḥ MM.
	%°ekoṇam: em. °ekoam.
\pend

\medskip

\pstart\noindent
[North]
\pend

\medskip

\pstart
\Skt{\edtext{priyacakṣurvaktraḥ}{\lemma{%
	{\rm priyacakṣurvaktraḥ\lem}
}\Dfootnote{%
	\emn\ \MM;
	\textit{priyacakṣuvakraḥ} \cod
}} 1 / sujñānaḥ 2 / samṛddhaḥ 3 / guṇarāśiḥ 4 / \edtext{praṇataḥ}{\lemma{%
	{\rm praṇataḥ\lem}
}\Dfootnote{%
	\cod;
	\textit{praṇādaḥ} \MM
}} 5 / dharmadhvajaḥ 6 / jñānarutaḥ 7 / \edtext{gagaṇaḥ}{\lemma{%
	{\rm gagaṇaḥ\lem}
}\Dfootnote{%
	\cod;
	\textit{gaganaḥ} \MM; 
	The spelling with \textit{ṇ} is probably acceptable.
}} 8 / \edtext{vimalaḥ}{\lemma{%
	{\rm vimalaḥ\lem}
}\Dfootnote{%
	\cod;
	\textit{vihārasvaraḥ} \MM
}} 9 / jñāneśvaraḥ 10 //}
	%priyacakṣuvakraḥ: em. priyacakṣurvaktraḥ, with MM.
	%praṇataḥ: praṇādaḥ MM.
	%gagaṇaḥ: gaganaḥ MM. The spelling with ṇ is probably acceptable.
	%vimalaḥ: vihārasvaraḥ MM.
\pend

\pstart
\Skt{\edtext{prajñāvibhaktarutaḥ}{\lemma{}\Efootnote{%
	prajñāvi[49r1]bhaktarutaḥ;
	Note that scribe assigns number 49 to verso of this folio. He seems inadvertently to have skipped the number 48.
}} 1 / guṇatejoraśmiḥ \addition{2} / \edtext{ṛṣīndraḥ}{\lemma{%
	{\rm ṛṣīndraḥ\lem}
}\Dfootnote{%
	\emn;
	\textit{urṣīndraḥ} \cod;
	\textit{ṛsīndraḥ} \MM
}} 3 / matimān 4 / pratibhāṇakaḥ 5 / \edtext{yajñāḥ \addition{6 /}}{\lemma{%
	{\rm yajñāḥ 6\lem}
}\Dfootnote{%
	\emn;
	\textit{yajñāḥ} \cod;
	\textit{suyajñaḥ 6 |} \MM;
	Emend with \MM?
}} candrānanaḥ 7 / \edtext{sūryaśaśī}{\lemma{%
	{\rm sūryaśaśī\lem}
}\Dfootnote{%
	\cod;
	\textit{suryaśasī} \MM (sic)
}} 8 / \edtext{guṇasāgaraḥ}{\lemma{%
	{\rm guṇasāgaraḥ\lem}
}\Dfootnote{%
	\cod;
	\textit{vimalaḥ} \MM
}} 9 / guṇasañcayaḥ 10 //}
	%°urṣīndraḥ: ṛsīndraḥ MM. Em. °ṛṣīndraḥ.
	%pratibhāṇakaḥ: pratibhānacakraḥ MM. Emend pratibhānakaḥ?
	%yajñāḥ: suyajñaḥ 6 | MM. Emend with MM?
	%sūyaśaśī: em. sūryaśaśī; suryaśasī MM (sic).
	%guṇasāgaraḥ: vimalaḥ MM.
	%Note that scribe assigns number 49 to verso of this folio. He seems inadvertently to have skipped the number 48.
\pend

\pstart
\Skt{ketūttamaḥ 1 / puṇyadhvajaḥ 2 / \edtext{pratibhāṇarāṣṭraḥ}{\lemma{%
	{\rm pratibhāṇarāṣṭraḥ\lem}
}\Cfootnote{%
	emend \textit{pratibhānarāṣṭraḥ}, with MM?
}} 3 / ratnaprabhaḥ 4 / priyacandraḥ 5 / siṃhabalabuddhiḥ 6 / vaśavartirājaḥ 7 / amṛtaprasannaḥ 8 / \edtext{samadhyāyī}{\lemma{%
	{\rm samadhyāyī\lem}
}\Cfootnote{%
	\MM\ emends \textit{samantadhyāyī}. We would prefer to emend \textit{samatādhyāyī}.
}} 9 / \edtext{akalpaḥ}{\lemma{%
	{\rm akalpaḥ\lem}
}\Cfootnote{%
	\MM\ reads \textit{akalya} and emends \textit{akṣobhyaḥ}. Other possible reading: \textit{akarṇyaḥ}.
}} 10 //}
	%pratibhāṇarāṣṭraḥ: em. pratibhānarāṣṭraḥ, with MM?
	%samantadhyāyī: MM emend samantadhyāyī. We would prefer to emend samatādhyāyī.
	%°aka\unclear{lpa}ḥ: MM reads akalya and emends °akṣobhyaḥ. Other possible reading: °akarṇyaḥ.
\pend

\pstart
\Skt{praśāntamalaḥ 1 / \edtext{deśitāmūḍhaḥ}{\lemma{%
	{\rm deśitāmūḍhaḥ\lem}
}\Dfootnote{%
	\cod;
	\textit{deśamūḍhaḥ} \MM
}} 2 / laḍitaḥ 3 / subhadraḥ 4 / \edtext{sthitacetajñānaḥ}{\lemma{%
	{\rm sthitacetajñānaḥ\lem}
}\Dfootnote{%
	\emn\ \MM;
	\textit{sthitacetajñāno} \cod
}} 5 / kathendraḥ 6 / mahādevaḥ 7 / gambhīramatiḥ 8 / amitaḥ \addition{9} / dharmabalaḥ 10 //}
	%deśitāmūḍhaḥ: deśamūḍhaḥ MM.
	%sthitacetajñāno: em. sthitacetajñānaḥ, with MM. 
\pend

\pstart
\Skt{pūjyaḥ 1 / puṣpaprabhaḥ 2 / \edtext{trailokyapūjyaḥ}{\lemma{%
	{\rm trailokyapūjyaḥ\lem}
}\Dfootnote{%
	\cod;
	\textit{trailokapūjyaḥ} \MM
}} 3 / rāhusūryagarbhaḥ 4 / marutpūjitaḥ 5 / mokṣadhvajaḥ 6 / kalyāṇacūḍaḥ 7 / amṛtaprabhaḥ 8 / vajraḥ 9 / dṛḍhasāgaraḥ 10 //}
	%trailokyapūjyaḥ: trailokapūjyaḥ MM.
	%rāhusūyagarbhaḥ: em. rāhusūryagarbhaḥ, with MM.
	%At the end of line 4, the microfilm suggests presence of some more illegible akṣaras, but this impression is probably misleading; we assume a worm hole has caused it.
\pend

\pstart
\Skt{ratnaskandhaḥ 1 / laḍitakramaḥ 2 / bhānumān 3 / śuddhaprabhaḥ 4 / \edtext{ratnacūḍaḥ}{\lemma{%
	{\rm ratnacūḍaḥ\lem}
}\Dfootnote{%
	\cod;
	\textit{ratnacūdaḥ} \MM
}} 5 / \edtext{anupamaśrīḥ}{\lemma{%
	{\rm anupamaśrīḥ\lem}
}\Cfootnote{%
	emend \textit{anupamaśrī?}
}} 6 / siṃhagatiḥ 7 / udgataḥ 8 / puṣpadattaḥ 9 / muktiprabhaḥ 10 //}
	%ratnacūḍaḥ: ratnacūdaḥ MM.
	%°anupamaśrīḥ: em. °anupamaśrī?
\pend

\pstart
\Skt{padmaḥ 1 / jñānapriyaḥ 2 / laḍitavyūhaḥ 3 / amoghavihārī 4 / rūḍhavraṇaḥ 5 / ketudhvajaḥ 6 / sukhacittī 7 / vimoharājaḥ 8 / \edtext{vidhijñaḥ}{\lemma{}\Efootnote{%
	vidhi[49v1]jñaḥ
}} 9 / śuddhasāgaraḥ 10 //}
\pend

\pstart
\Skt{ratnadharaḥ 1 / ajitaḥ 2 / \edtext{jagattoṣaṇaḥ}{\lemma{%
	{\rm jagattoṣaṇaḥ\lem}
}\Dfootnote{%
	\emn\ \MM;
	\textit{jagatoṣaṇaḥ} \cod
}} 3 / mahārutaḥ 4 / \edtext{adanaḥ}{\lemma{%
	{\rm adanaḥ\lem}
}\Dfootnote{%
	\cod;
	\textit{adīnaḥ} \MM
}} 5 / bhavatṛṣṇāmalaprahīṇaḥ 6 / \edtext{cāritratīrthaḥ}{\lemma{%
	{\rm cāritratīrthaḥ\lem}
}\Dfootnote{%
	\emn\ \MM;
	\textit{cārītratīrthaḥ} \cod
}} 7 / bahudevaghuṣṭaḥ 8 / ratnakramaḥ 9 / padmahastī 10 //}
	%jagatoṣaṇaḥ: em. jagattoṣaṇaḥ, with MM.
	%°adanaḥ: °adīnaḥ MM.
	%cārītratīrthaḥ: em. cāritratīrthaḥ, with MM.
\pend

\pstart
\Skt{śrīḥ 1 / jitaśatruḥ 2 / \edtext{samṛddhayaśāḥ}{\lemma{%
	{\rm samṛddhayaśāḥ\lem}
}\Dfootnote{%
	\cod;
	\textit{samṛddhayaśaḥ} MM
}} 3 / surāṣṭraḥ 4 / kusumaprabhaḥ 5 / \edtext{siṃhasvaraḥ}{\lemma{%
	{\rm siṁhasvaraḥ\lem}
}\Dfootnote{%
	\emn\ \MM;
	\textit{siṁhaśvaraḥ} \cod
}} 6 / candrodgataḥ 7 / \edtext{bhasmadāmā}{\lemma{%
	{\rm bhasmadāmā\lem}
}\Dfootnote{%
	\emn;
	\textit{bhasmadāmāḥ} \cod;
	\MM\ emends \textit{bhasmadāma}.
}} 8 / acalaḥ 9 / saṃjñāgatiḥ 10 //}
	%samṛddhayaśāḥ: samṛddhayaśaḥ MM.
	%siṁhaśvaraḥ: em. siṁhasvaraḥ, with MM.
	%bhasmadāmāḥ: em. bhasmadāmā; MM emends bhasmadāma.
\pend

\pstart
\Skt{\edtext{puṇyapradīparājaḥ}{\lemma{%
	{\rm puṇyapradīparājaḥ\lem}
}\Dfootnote{%
	\emn\ \MM;
	\textit{puṇyapradīparājāḥ} \cod
}} 1 / svaracodakaḥ 2 / gautamaḥ 3 / ojabalaḥ 4 / sthitabuddhirūpaḥ 5 / sucandraḥ 6 / bodhyaṅgapuṣpaḥ 7 / praśastaḥ 8 / \edtext{sutejāḥ}{\lemma{%
	{\rm sutejāḥ\lem}
}\Dfootnote{%
	\cod;
	\textit{sutejaḥ} \MM
}} 9 / balatejojñānaḥ10 //}
	%puṇyapradīparājāḥ: em. puṇyapradīparājaḥ, with MM.
	%sutejāḥ: sutejaḥ MM.
\pend

\pstart
\Skt{kuśalaprabhaḥ 1 / dṛḍhavikramaḥ 2 / \edtext{devasutaḥ}{\lemma{%
	{\rm devasutaḥ\lem}
}\Dfootnote{%
	\cod;
	\edtext{devarutaḥ} \MM
}} 3 / praśāntaḥ 4 / sūryānanaḥ 5 / mokṣaprabhaḥ 6 / śīlaprabhaḥ 7 / \edtext{vratasthitaḥ}{\lemma{%
	{\rm vratasthitaḥ\lem}
}\Dfootnote{%
	\emn\ \MM;
	\textit{pratasthitaḥ} \cod
}} 8 / arajaḥ 9 / \edtext{sāgarodgataḥ}{\lemma{%
	{\rm sāgarodgataḥ\lem}
}\Dfootnote{%
	\cod;
	\textit{sārodgataḥ} MM
}} 10 //}
	%devasutaḥ: devarutaḥ MM.
	%sūyānanaḥ: em. sūryānanaḥ, with MM.
	%pratasthitaḥ: em. vratasthitaḥ, with MM.
	%sāgarodgataḥ: sārodgataḥ MM.
\pend

\pstart
\Skt{añjanaḥ 1 / \edtext{ārjitaḥ}{\lemma{%
	{\rm ārjitaḥ\lem}
}\Dfootnote{%
	\cod;
	\textit{(vardhanā)rjitaḥ} \MM
}} 2 / \edtext{gandhābhaḥ}{\lemma{%
	{\rm gandhābhaḥ\lem}
}\Dfootnote{%
	\emn\ \MM;
	\textit{gandhobhaḥ} \cod
}} 3 / vimalaprabhaḥ 4 / smṛtīndraḥ 5 / madhuravaktraḥ 6 / anantadhvajaḥ 7 / varabuddhiḥ 8 / caraṇaprasannaḥ 9 / ratnapriyaḥ 10 //}
	%°ārjitaḥ: (vardhanā)rjitaḥ MM.
	%gandhobhaḥ: em. gandhābhaḥ, with MM. 
\pend

\pstart
\Skt{dharmeśvaraḥ 1 / \edtext{viśvadevaḥ}{\lemma{%
	{\rm viśvadevaḥ\lem}
}\Dfootnote{%
	\emn\ \MM?;
	\textit{visvadevaḥ} \cod
}} 2 / mahāmitraḥ 3 / sumitraḥ 4 / praśāntagatiḥ 5 / \edtext{amṛtādhipaḥ}{\lemma{%
	{\rm amṛtādhipaḥ\lem}
}\Dfootnote{%
	\cod;
	\textit{amṛtādhipatiḥ} \MM
}} 6 / \edtext{meruprabhaḥ 7 / \edtext{āryastutaḥ}{\lemma{%
	{\rm āryastutaḥ\lem}
}\Dfootnote{%
	\emn\ \MM;
	\textit{aryastutaḥ} \cod
}}}{\lemma{}\Efootnote{%
	meruprabhaḥ 7 | [50r1] āryastutaḥ
}} 8 / \edtext{jyotiṣmān}{\lemma{%
	{\rm jyotiṣmān\lem}
}\Dfootnote{%
	\emn\ \MM;
	\textit{jyotimān} \cod
}} 9 / \edtext{dīptatejāḥ}{\lemma{%
	{\rm dīptatejāḥ\lem}
}\Dfootnote{%
	\emn;
	\textit{dīptatejaḥ} \cod\ \MM
}} 10 //}
	%visvadevaḥ: em. viśvadevaḥ, with MM?
	%°amṛtādhipaḥ: °amṛtādhipatiḥ MM.
	%°ayastutaḥ: em. °āryastutaḥ. with MM
	%jyotimān: em. jyotiṣmān, with MM.
	%dīptatejaḥ: em. dīptatejāḥ.
\pend

\pstart
\Skt{\edtext{avabhāsadarśī 1}{\lemma{%
	{\rm avabhāsadarśī 1\lem}
}\Dfootnote{%
	\emn\ \MM;
	\textit{avabhāsvadarśīḥ kṣa} \cod
}} / \edtext{sucīrṇavipākaḥ}{\lemma{%
	{\rm sucīrṇavipākaḥ\lem}
}\Dfootnote{%
	\cod;
	\textit{sucīrnavipākaḥ} \MM
}} 2 / supriyaḥ 3 / \edtext{vigataśokaḥ}{\lemma{%
	{\rm vigataśokaḥ\lem}
}\Dfootnote{%
	\emn\ \MM;
	\textit{vigatasokaḥ} \cod
}} 4 / ratnaprabhāsaḥ 5 / \edtext{cāritrakaḥ}{\lemma{%
	{\rm cāritrakaḥ\lem}
}\Dfootnote{%
	\emn\ \MM;
	\textit{cāritrayaḥ} \cod
}} 6 / puṇyakramaḥ 7 / \edtext{guṇasāgaraḥ}{\lemma{%
	{\rm guṇasāgaraḥ\lem}
}\Dfootnote{%
	\emn;
	\textit{guṇasāraḥ} \cod;
	\textit{gunasāgaraḥ} \MM
}} 8 / caityakaḥ 9 / mārajahaḥ 10 //}
	%°avabhāsvadarśīḥ kṣa: em. °avabhāsadarśī 1, with MM.
	%sucīrṇṇavipākaḥ: sucīrnavipākaḥ MM.
	%vigatasokaḥ: em. vigataśokaḥ, with MM.
	%cāritrayaḥ: em. cāritrakaḥ, with MM.
	%guṇasāraḥ: guna- MM; em. -sāgaraḥ.
\pend

\pstart
\Skt{\edtext{sattamaṅgamaḥ}{\lemma{%
	{\rm sattamaṅgamaḥ\lem}
}\Dfootnote{%
	\cod;
	\textit{mānajahaḥ} \MM; 
	\MM's emendation is supported by Tib.; support from Chin. is less clear.
}} 1 / \edtext{vāsanottīrṇagatiḥ}{\lemma{%
	{\rm vāsanottīrṇagatiḥ\lem}
}\Dfootnote{%
	\cod;
	\edtext{vāsanotīrnagatiḥ} \MM
}} 2 / abhedyavuddhiḥ 3 / udadhiḥ 4 / \edtext{vimārṣṭaḥ}{\lemma{%
	{\rm vimārṣṭaḥ\lem}
}\Dfootnote{%
	\emn\ \MM;
	\textit{vimāṣṭaḥ} \cod;
	See BHSD \textit{vimaṣṭaka}.
}} 5 / gaṇimuktirājaḥ 6 / priyābhaḥ 7 / bodhidhvajaḥ 8 / jñānarutaḥ 9 / \edtext{suśrīmān}{\lemma{%
	{\rm suśrīmān\lem}
}\Dfootnote{%
	\emn\ \MM;
	\textit{suśrīmāṁ} \cod
}}10 //}
	%sattamaṅgamaḥ: mānajahaḥ MM. MM's em. is supported by Tib.; support from Chin. is less clear.
	%vāsanottīrṇṇagatiḥ: -tīrna- MM.
	%vimāṣṭaḥ: em. vimārṣṭaḥ, with MM. See BHSD vimaṣṭaka.
	%suśrīmāṁ: em. suśrīmān, with MM.
\pend

\pstart
\Skt{\edtext{brahmā}{\lemma{%
	{\rm brahmā\lem}
}\Dfootnote{%
	\emn\ \MM;
	\textit{vrahmāḥ} \cod
}} 1 / \edtext{jñānarutaḥ}{\lemma{%
	{\rm jñānarutaḥ\lem}
}\Cfootnote{%
	\cod;
	\textit{jñānarataḥ} \MM. 
	Adopt this emendation or accept two cases of identical name \textit{jñānaruta} in close proximity?
}} 2 / ṛddhiketuḥ 3 / janendrakalpaḥ 4 / \edtext{dharaṇindharaḥ}{\lemma{%
	{\rm dharaṇindharaḥ\lem}
}\Cfootnote{%
	on \textit{dharaṇindhara} = \textit{dharaṇīndhara}, see BHSD.
}} 5 / sūryapriyaḥ 6 / rāhucandraḥ 7 / \edtext{puṣpābhaḥ}{\lemma{%
	{\rm puṣpābhaḥ\lem}
}\Dfootnote{%
	\cod;
	\textit{puṣpaprabhaḥ} \MM
}} 8 / vidyādhipaḥ 9 / ojodhārī 10 //}
	%vrahmāḥ: em. brahmā, with MM.
	%jñānarutaḥ: jñānarataḥ MM. Adopt this emendation or accept two cases of identical name jñānaruta in close proximity?
	%dharaṇindharaḥ: on dharaṇindhara = dharaṇīndhara, see BHSD.
	%sūyapriyaḥ: em. sūryapriyaḥ, with MM.
	%puṣpābhaḥ: puṣpaprabhaḥ MM.
\pend

\pstart
\Skt{puṇyapriyaḥ 1 / ratiprabhaḥ 2 / \edtext{sughoṣaḥ}{\lemma{%
	{\rm sughoṣaḥ\lem}
}\Dfootnote{%
	\emn\ \MM;
	\textit{sughoṣa} \cod
}} 3 / dharmeśvaraḥ 4 / \edtext{brahmarutaḥ}{\lemma{%
	{\rm brahmarutaḥ\lem}
}\Dfootnote{%
	\emn\ \MM;
	\textit{vrahmaruta} \cod
}} 5 / suce\addition{ṣṭaḥ} 6 / askhalitabuddhiḥ 7 / mahāpraṇādaḥ 8 / yaśakīrtiḥ 9 / \edtext{ketumān 10}{\lemma{%
	{\rm ketumān 10\lem}
}\Dfootnote{%
	\emn\ \MM;
	\textit{ketumān} \cod;
}} //}
	%sughoṣa: em. sughoṣaḥ, with MM.
	%vrahmaruta: em. brahmarutaḥ, with MM.
	%ketumān·: em. ketumān 10, with MM.
\pend

\pstart
\Skt{\edtext{vighuṣṭatejāḥ}{\lemma{%
	{\rm vighuṣṭatejāḥ\lem}
}\Dfootnote{%
	\cod;
	\textit{vighuṣṭatejaḥ} \MM
}} 1 / \edtext{jagatīśvaraḥ}{\lemma{%
	{\rm jagatīśvaraḥ\lem}
}\Dfootnote{%
	\cod;
	\textit{jagadīśvaraḥ} \MM. Make this emendation? It is not necessary from point grammatical/semantic viewpoint.
}} 2 / drumaḥ 3 / supraṇaṣṭamohaḥ 4 / amṛtaḥ 5 / \edtext{sucandramāḥ}{\lemma{%
	{\rm sucandramāḥ\lem}
}\Dfootnote{%
	\cod;
	\textit{sucandramaḥ} \MM
}} 6 / \edtext{anantaagnityanaketuḥ}{\lemma{%
	{\rm anantaagnityanaketuḥ\lem}
}\Cfootnote{%
	the reading makes no sense at all; tentatively, we assume the scribe was copying from a damaged exemplar, and accept MM's emendation \textit{anantapratibhānaketuḥ}. Note that previously, \textit{pratibhāna} was always/mostly \textcolor{red}{(CHECK)} spelled \textit{pratibhāṇa}.
}} 7 / līlāprabhaḥ 8 / pūjyaḥ 9 / \edtext{uttīrṇaśokaḥ}{\lemma{%
	{\rm uttīrṇaśokaḥ\lem}
}\Dfootnote{%
	\emn;
	\textit{utīrṇṇasokaḥ} \cod;
	\textit{utīrṇaśokaḥ} \MM
}} 10 //}
	%vighuṣṭatejāḥ: vighuṣṭatejaḥ MM.
	%jagatīśvaraḥ: jagadīśvaraḥ MM. Make this emendation? It is not necessary from point grammatical/semantic viewpoint.
	%sucandramāḥ: sucandramaḥ MM.
	%°ananta°a§gnityanaketuḥ: the reading makes no sense at all; tentatively, we assume the scribe was copying from a damaged exemplar, and accept MM's emendation °anantapratibhānaketuḥ. Note that previously, pratibhāna was always/mostly (CHECK) spelled pratibhāṇa.
	%°u§○tīrṇṇasokaḥ: °utīrṇaśokaḥ MM; em. °uttīrṇaśokaḥ.
\pend

\pstart
\Skt{kṣemapriyaḥ 1 / jagatpratiḥ 2 / priyaṅgamaḥ 3 / caraṇābhijātaḥ 4 / utpalaḥ 5 /\edtext{anantapratibhānaraśmiḥ}{\lemma{%
	{\rm anantapratibhānaraśmiḥ\lem}
}\Dfootnote{
	\emn\ \MM;
	\textit{anantapratibhāṇaraśmi} \cod
}\lemma{}\Efootnote{%
	anantaprati[50v1]bhānaraśmiḥ
}} 6 / puṣpottamaḥ 7 / \edtext{ṛṣipradhānaḥ}{\lemma{%
	{\rm ṛṣipradhānaḥ\lem}
}\Dfootnote{%
	\emn\ \MM;
	\textit{ṛpipradhānaḥ} \cod
}} 8 / guṇavīryaḥ 9 / sāraḥ 10 //}
	%jagatpratiḥ: em. jagadratiḥ, with MM.
	%-bhāṇaraśmi: em. -bhānaraśmiḥ, with MM.
	%°ṛpipradhānaḥ: em. °ṛṣipradhānaḥ, with MM.
	%guṇavīyaḥ: em. guṇavīryaḥ, with MM.
\pend

\pstart
\Skt{marutādhipaḥ 1 / uccarataḥ 2 / prasannaḥ 3 / bhāgīrathiḥ 4 / puṇyamatiḥ 5 / hutārciḥ 6 / \edtext{anantaguṇaraśmitejāḥ}{\lemma{%
	{\rm anantaguṇaraśmitejāḥ\lem}
}\Dfootnote{%
	\emn\ \MM;
	\textit{anantaguṇarasmitejāḥ} \cod
}} 7 / siṃhavikrāmī 8 / acalaḥ 9 / prasannaḥ 10 //}
	%-rasmitejāḥ: -raśmitejaḥ MM; em. -raśmitejāḥ.
\pend

\pstart
\Skt{cīrṇaprabhaḥ 1 / nāgarutaḥ 2 / \edtext{punar nāgarutaḥ}{\lemma{%
	{\rm punar nāgarutaḥ\lem}
}\Dfootnote{%
	\emn;
	\textit{puna nāgarutaḥ} \cod;
	\textit{punarnāgarutaḥ} \MM
}} 3 / cakradhāḥ 4 / \edtext{varṇasuśreṣṭhaḥ}{\lemma{%
	{\rm varṇasuśreṣṭhaḥ\lem}
}\Dfootnote{%
	\emn\ \MM
	\textit{varṇṇasuśreṣṭaḥ} \cod
}} 5 / lokapriyaḥ 6 / \edtext{dharmaśaśī}{\lemma{%
	{\rm dharmaśaśī\lem}
}\Dfootnote{%
	\emn\ \MM
	\textit{dharmmaśaśīḥ} \cod
}} 7 / anantaratnakīrtiḥ 8 / meghadhvajaḥ 9 / \edtext{prajñāmatiḥ}{\lemma{%
	{\rm prajñāmatiḥ\lem}
}\Dfootnote{%
	\cod;
	\textit{prajñāgatiḥ} \MM;
	Accept this emendation, supported by Tib. and Chin.?
}} 10 //}
	%puna nāgarutaḥ: punarnāgarutaḥ MM; em. punar nāgarutaḥ.
	%cakradhāḥ: cakradharaḥ MM. Accept this emendation, or accept the suffix -dhā?
	%varṇṇasu*śreṣṭaḥ: em. varṇasuśreṣṭhaḥ, with MM.
	%dharmmaśaśīḥ: em. dharmaśaśī, with MM.
	%prajñāmatiḥ: prajñāgatiḥ MM. Accept this emendation, supported by Tib. and Chin.?
\pend

\pstart
\Skt{sugandhaḥ 1 / \edtext{gagaṇasvaraḥ}{\lemma{%
	{\rm gagaṇasvaraḥ\lem}
}\Dfootnote{%
	\cod;
	\textit{gaganasvaraḥ} \MM
}} 2 / amaraḥ 3 / devarājaḥ 4 / praṇidhānaḥ 5 / sudhanaḥ 6 / pradīpaḥ 7 / ratnasvaraghoṣaḥ 8 / janendrarājaḥ 9 / rāhuguptaḥ 10 //}
	%gagaṇasvaraḥ: gaganasvaraḥ MM.
\pend

\pstart
\Skt{kṣemaṅkaraḥ 1 / \edtext{siṃhagatiḥ}{\lemma{%
	{\rm siṁhagatiḥ\lem}
}\Dfootnote{%
	\emn\ \MM;
	\textit{sinhagatiḥ} \cod
}} 2 / \edtext{ratnayaśāḥ}{\lemma{%
	{\rm ratnayaśāḥ\lem}
}\Dfootnote{%
	\cod;
	\textit{ratnayaśaḥ} \MM
}} 3 / kṛtārthaḥ 4 / \edtext{kṛtāntadarśī}{\lemma{%
	{\rm kṛtāntadarśī\lem}
}\Dfootnote{%
	\emn\ \MM;
	\textit{kṛtāntadarśīḥ} \cod
}} 5 / bhavapuṣpaḥ 6 / \edtext{ūrṇaḥ}{\lemma{%
	{\rm ūrṇaḥ\lem}
}\Dfootnote{%
	\emn\ \MM;
	\textit{urṇṇaḥ} \cod
}} 7 / \edtext{atulapratibhānarājaḥ}{\lemma{%
	{\rm atulapratibhānarājaḥ\lem}
}\Dfootnote{%
	\emn\ \MM?;
	\textit{atulapratibhāṇarājaḥ} \cod
}} 8 / vibhaktajñāneśvaraḥ 9 / \edtext{siṃhadaṃṣṭraḥ}{\lemma{%
	{\rm siṁhadaṁṣṭraḥ\lem}
}\Dfootnote{%
	\emn\ \MM;
	\textit{sinhadraṣṭraḥ} \cod
}} 10 //}
	%sinhagatiḥ: em. siṁhagatiḥ, with MM.
	%ratnayaśāḥ: ratnayaśaḥ MM.
	%kṛtāntadarśīḥ: em. kṛtāntadarśī, with MM.
	%°urṇṇaḥ: em. °ūrṇaḥ, with MM.
	%-bhāṇa-: em. -bhāna-, with MM?
	%sinhadraṣṭraḥ: em. siṁhadaṁṣṭraḥ, with MM.
\pend

\pstart
\Skt{\edtext{laḍitakramaḥ}{\lemma{%
	{\rm laḍitakramaḥ\lem}
}\Dfootnote{%
	\emn\ \MM;
	\textit{lalitakraḥ} \cod
}} 1 / puṇyapradīpaḥ 2 / \edtext{dharmapradīpacchattraḥ}{\lemma{%
	{\rm dharmapradīpacchattraḥ\lem}
}\Dfootnote{%
	\emn\ \MM;
	\textit{dharmmapradīpacchatraḥ} \cod
}} 3 / maṅgalī 4 / \edtext{aśokarāṣṭraḥ}{\lemma{%
	{\rm aśokarāṣṭraḥ\lem}
}\Dfootnote{%
	\emn\ \MM;
	\textit{aśokarāṣṭraḥ} \cod
}} 5 / maticintī 6 / buddhibalaḥ 7 / dharmapradīpākṣaḥ 8 / \edtext{sudarśī}{\lemma{%
	{\rm sudarśī\lem}
}\Dfootnote{%
	\emn\ \MM;
	\textit{sudarśīḥ} \cod
}} 9 / vegajahaḥ 10 //}
	%lalitakraḥ: em. laḍitakramaḥ, with MM.
	%-cchatraḥ: em. cchattraḥ, with MM.
	%°a*śokarāṣṭrāḥ: em. °aśokarāṣṭraḥ, with MM.
	%sudarśīḥ: em. sudarśī, with MM.
\pend

\pstart
\Skt{atibalaḥ 1 / prajñāpuṣpaḥ 2 / \edtext{dṛḍhaśvaraḥ}{\lemma{%
	{\rm dṛḍhaśvaraḥ\lem}
}\Dfootnote{%
	\emn\ \MM;
	\textit{dṛḍhasvaraḥ} \cod
}} 3 /\edtext{}{\lemma{%
	{\rm Nos.\ 4–10\lem}
}\Cfootnote{%
	\MM\ reconstructs the Sanskrit names for the rest of this group of 10.
}} }
	%dṛḍhaśvaraḥ: em. dṛḍhasvaraḥ, with MM.
	%MM reconstructs the Sanskrit names for the rest of this group of 10.
\pend

\bigskip

\pstart
\mbox{}\hfill (folio 51 is missing)\hfill\mbox{}
\pend

\bigskip

\pstart\noindent
{\large 3.7. *Kalaśasthāpanavidhi}
\pend

\medskip

\pstart
%\newfolio{52r1} %
\Skt{\edtext{\restored{tato ratnamayaṃ mṛnmayaṃ vā kalaśamakālamūlaṃ mahodaramuccagrīvaṃ lamboṣṭhaṃ pravālasuvarṇa\-śaṅkhamuktāpadmarāgaiḥ sarvaratnairbṛhatīkaṇṭakārīsahadevādaṇḍotpalaśvetāparāji}\edtext{tābhi\-\textcolor{red}{ssa}\-rvau\-ṣadhibhiḥ}{\lemma{%
	{\rm tābhis sarvauṣadhibhiḥ\lem}
}\Dfootnote{%
	\emn;
	\unclear{tābhisso}ṣadhībhiḥ \cod;
	\HT's emendation to \textit{-dhibhiḥ} is perhaps not necessary.
}\lemma{}\Efootnote{
	(folio missing) [52r1]tābhissoṣadhībhiḥ
}} śāliyavagodhūmatilamāṣaiḥ sarvadhānyaiḥ \edtext{sugandhodakasitasugandhakusumaiśca}{\lemma{%
	{\rm sugandhodakasita\-sugandhakusumaiś ca\lem}
}\Dfootnote{%
	\emn\ ($\leftarrow$ Tib.: \textit{spos chu dri zhim pa dang | me tog dkar po dri zhim pas});
	\textit{sugandhodakasitasugandhaiḥ kusumaiś ca} \cod
}} paripūrṇaṃ samantato gandhopaliptaṃ \edtext{srakhinaṃ}{\lemma{
	{\rm srakhinaṁ\lem}
}\Dfootnote{%
	\emn\ \HT\ \sil;
	\textit{srakhinaṁ} \cod
%	\textit{sragvinaṁ} \HT.  It seems impossible to us to read \textit{gv}.
}} śrīvajrasattvavajrāṅkitaṃ sadvastrābaddhakaṇṭhakaṃ sa\edtext{\textcolor{red}{tpa}\-llava}{\lemma{%
	{\rm -pallava-\lem}
}\Dfootnote{%
	\emn\ \HT\ \sil:
	\textit{-palava-} \cod
	\HT\ reads \textit{satpallvaphalavaktraṁ} (sic), without note.
}}phala\edtext{vaktraṃ}{\lemma{%
	{\rm -vaktraṁ\lem}
}\Dfootnote{
	\emn\ \HT\ \sil;
	\textit{vakraṁ} \cod
}} vajrasattvena sattvavajrī\edtext{parigṛhītayā}{\lemma{%
	{\rm -parigṛhītayā\lem}
}\Dfootnote{%
	\emn;
	\textit{parigṛhītaṁyā} \cod
}} vajrakusumalatayāṣṭottaraśatajaptaṃ}{\lemma{%
	{\rm tato \dots\ japtaṁ\lem}
}\Bfootnote{%
%
SDPT:
\textit{sarvaratnauṣadhisarvadhānyaparipūrṇaṁ saphalaṃ sapallavaṃ sadvastrāvabaddhakaṇṭhakaṁ kṛtarakṣaṁ bahiḥ samantād divyagandhānuliptaṁ sragviṇaṁ vajrāṃkitam upari mahāvajrādhiṣṭhitaṃ krodhaterintirīparigṛhītayā vajrakusumalatayā} (p.258)
%
}} /  {\om} vajrodaka hū{\cb} iti /}
	%\unclear{tābhisso}ṣadhībhiḥ: emend tābhis sarvauṣadhibhiḥ. HT's emendation to -dhibhiḥ is perhaps not necessary.
	%kusumais: read kusumaiś.
	%\unclear{s}ra\unclear{khin}aṁ: sragvinaṁ HT. It seems impossible to us to read gv.
	%-palava-:  em. -pallava-, with HT? Or emend lasatpallavaphala-?
	%-vakraṁ: em. -vaktraṁ, with HT.
	%HT reads satpallvaphalavaktraṁ (sic), without note.
	%-gṛhītaṁyā: em. -gṛhītayā, with HT (sil. em.).
%\pend
%
%\pstart
\Skt{punaraṣṭottarasahasrābhimantritaṃ kṛtvā / bhagavato vajrasattvasyāgrataḥ sthāpayet / praveśadvārābhimukhaṃ \edtext{ca}{\lemma{%
	{\rm ca\lem}
}\Dfootnote{%
	\emn;
	\textit{caḥ} \cod;
	\HT\ reads \textit{cāḥ}.
}} dvitīyaṃ vajrasattvenāṣṭottaraśatajaptam / \edtext{tenodakenātmāna}{\lemma{%
	{\rm tenodakenātmānam\lem}
}\Dfootnote{
	\emn\ \HT\ \sil;
	\textit{tenodakenātmānaṁm} \cod
}}mabhiṣiñcya / \edtext{praveśakāle}{\lemma{%
	{\rm praveśakāle\lem}
}\Dfootnote{
	\emn;
	\textit{\unclear{p}raveśyakāle} \cod
}} \edtext{śiṣyāṇāṃ}{\lemma{%
	{\rm śiṣyāṇāṃ\lem}
}\Dfootnote{
	\emn;
	\textit{śiṣyāṁ} \cod;
	\HT\ emends \textit{śiṣyaṁ}. %Several passage in SVU show that the guru was surrounded by more than one disciple.
}} vajrasatvavajrīṃ badhnīyā\edtext{dbandhaye}{\lemma{%
	{\rm bandhayed\lem}
}\Dfootnote{
	\emn\ \HT;
	\textit{pandhayed} \cod;
	HT reads \textit{yandhayed}.
}}dvā //} 
\pend

\pstart
\Skt{śrīvairocanā\edtext{dīnāṃ}{\lemma{%
	{\rm -dīnāṁ\lem}
}\Cfootnote{%
	The scribe has spelled -\textit{dināṁ} but apparently wished to make the first vowel long and to so applied an \textit{-ā} sign.
}} tu \edtext{kalaśān pratyekasvacihnān}{\lemma{%
	{\rm kalaśān pratyekasvacihnān\lem}
}\Dfootnote{
	\emn;
	\textit{kalaśaṁ pratyekaṁ svacihnaṁ} \cod
}} bāhyamaṇḍalabāhyataḥ koṇeṣu teṣāṃ svamantrairaṣṭottaraśata\edtext{\textcolor{red}{japtān}}{\lemma{%
	{\rm -japtān\lem}
}\Dfootnote{%
	\emn\ ($\leftarrow$ Tib.: );
	\textit{+ptāṁ paṣṭa} \cod;
	\textit{-japtaṁ piṣṭa} \HT;
%	 Although the emendation \textit{pañca} is not supported by Tib., it seems to follow from the next passage, and the second unclear \textit{akṣara}s could easily be explained as a scribal misspelling of \textit{ñca} in a previous manuscript. CHECK Abhayākaragupta, \textit{Vajrāvalī} for what to do with 5 pots in 4 corners.
}} \textcolor{red}{sthāpayet} / \edtext{teṣā}{\lemma{%
	{\rm teṣām\lem}
}\Dfootnote{
	\emn\ \HT;
	\textit{teṣāṁm} \cod
}}mapi bāhyataḥ pūrṇakumbham //}
	%caḥ: em. ca |. HT reads cāḥ.
	%tenodakenātmānaṁm: em. tenodakenātmānam, with HT (sil. em.).
	% \unclear{p}raveśyakāle: em.  praveśakāle.
	%śiṣyāṁ: em. śiṣyān; HT emend śiṣyaṁ. Several passage in SVU show that the guru was surrounded by more than one disciple.
	%pandhayed: em. bandhayed, with HT. HT reads yandhayed.
	%-dīnāṁ : the scribe has spelled -dināṁ but apparently wished to make the first vowel long and to so applied an -ā sign.
	%kalaśaṁ pratyekaṁ svacihnam: em. kalaśān pratyekasvacihnān.
	%\lost{ja}ptāṁ pa\unclear{ṣva}: -japtaṁ piṣṭa HT; em. -japtān pañca. Although the emendation pañca is not supported by Tib., it seems to follow from the next passage, and the second unclear akṣaras could easily be explained as a scribal misspelling of ñca in a previous manuscript. CHECK Abhayākaragupta, Vajrāvalī for what to do with 5 pots in 4 corners.
	%teṣāṁm: em. teṣām, with HT.
\pend

\pstart
\Skt{abhāve \edtext{śrīvajrasattvasya}{\lemma{%
 	{\rm śrīvajrasattvasya\lem}
 }\Dfootnote{
 	\emn\ \HT\ \sil;
 	\textit{śrīvajrāsatvasya} \cod
}} pañcatathāgatānāṃ ca \edtext{kalaśaṃ}{\lemma{%
	{\rm kalaśaṃ\lem}
}\Dfootnote{
	\emn;
	\textit{kalaśān} \cod;
	\textit{kalaśāt} \HT
}} \edtext{pūrṇakumbhaṃ}{\lemma{
	{\rm pūrṇakumbhaṁ\lem}
}\Dfootnote{
	\emn\ \HT;
	\textit{pūrṇṇakumbhaś} \cod
}} ca dattvā / sattvaratnadharmakarmavajrāṅkaṃ \supplied{sva}kulamantrairaṣṭottaraśatābhijaptaṃ \edtext{kalaśacatuṣṭayaṃ}{\lemma{
	{\rm kalaśacatuṣṭayaṁ\lem}
}\Dfootnote{
	\emn\ \HT;
	\textit{kalaśacatuṣṭhayaṁ} \cod
}} \edtext{pūrṇakumbha\textcolor{red}{catuṣṭayaṃ}}{\lemma{
	{\rm pūrṇṇakumbhacatuṣṭayañ\lem}
}\Cfootnote{
	To be emended to \textit{pūrṇṇakumbhañ}? But Tib. supports extant reading.
}} ca dadyāt / daśanyūnaṃ na kārayediti \edtext{vacanāt}{\lemma{%
	{\rm vacanāt\lem}
}\Dfootnote{
	\emn\ \HT\
	\textit{vacanām} \cod
}} //}  
\pend

\bigskip

\pstart\noindent
{\large 3.8. Abhiṣeka}
\pend

\medskip

\pstart\noindent
{\large 3.8.1. Saṁkṣipta(-abhiṣeka-)kramaḥ}
\pend

\medskip

\pstart\noindent
3.8.1.1. Svādhiṣṭhānam
\pend

\medskip

\pstart
\edtext{}{\lemma{%
	{\rm From the beginning of §3.8 to śeṣa āśvāsaḥ śrīparamādye draṣṭavyaḥ in §3.8.1.\lem}
}\Bfootnote{%
	Closely parallel to the \textit{Tattvālokakarī} (P LOCATION!,  D 115v4–137r3)
}}%
\Skt{tato manasā maṇḍalaṃ devātāṃśca \edtext{pratyakṣān}{\lemma{}\Efootnote{%
	pratya[52v1]kṣān
}} niścitya /}%
	%śrīvajrāsattvasya: em. śrīvajrasattvasya, with HT (sil. em.).
	%kalaśān: kalaśāt HT.
	%pūrṇṇakumbhaś: em. pūrṇakumbhaṁ, with HT.
	%-catuṣṭhayaṁ: em. -catuṣṭayaṁ, with HT.
	%pūrṇṇakumbhacatuṣṭayañ: to be emended to pūrṇṇakumbhañ? But Tib. supports extant reading.
	%vacanām·: em. vacanāt·, with HT.
%\pend
%
%\pstart
\edtext{}{\lemma{%
	{\rm puṣpādibhiḥ \dots\ sādhitaṁ bhavati\lem}
}\Bfootnote{%
%
SDPT:
puṣpādibhir lāsyādibhiś cātmānaṃ saṃpūjya,
gāthāpañcakenānujñāṃ ca tu huṃkāram udgatāvyākaraṇaṃ cādāya, 
punaḥ svādhiṣṭhānādikaṃ kṛtvā, yathābhirucitajāpaṃ ca huṃkāravajro 'haṃ
huṃkāravajro 'ham iti svanāmoccāraṇaṃ kṛtvā, vairocanamahāmudrāṃ
baddhvā, tanmantreṇa aḥkārāntena vairocanasthāne tathāgatavajram ātmānam āveśayet. 
vajro 'ham iti vajrāhaṃkāraṃ vibhāvya, tad vajraṃ vairocanaṃ bhāvayed vajradhātur aham iti. 
evaṃ yāvad vajrāveśamahāmudrāṃ baddhvottaradvāre tanmantreṇāḥkārāntena vajraghaṇṭām ātmānam āveśayet. 
vajraghaṇṭāham ity ahaṃkāram utpādya tāṃ vajrāveśakrodho 'ham iti bhāvayed evaṃ vajreṇa sādhitaṃ bhavati. (p.\ 260)
%
}}%
\Skt{puṣpādibhiḥ saṃpūjya \edtext{catuḥpraṇāmādika}{\lemma{%
	{\rm catuḥpraṇāmādika-\lem}
}\Dfootnote{
	\emn\ \HT (whose reading \textit{catuḥpraṃāṇādika-} is a typing error);
	\textit{catuḥpramāṇādika-} \cod
}}pūrvakaṃ saṃvaramādāya yathāva\edtext{tsattvavajrīṃ}{\lemma{%
	{\rm sattvavajrīṁ\lem}
}\Dfootnote{
	\cod\pc;
	\textit{sartvavajrīṁ} \cod\ac
}} baddhvā praviśet /
tato vajrodakaṃ yathāvatpītvātmānamāveśayet / vāmakrodha\edtext{muṣṭyā}{\lemma{%
	{\rm -muṣṭyā\lem}
}\Dfootnote{
	\emn\ \HT\ \sil;
	\textit{muṣṭhyā} \cod
}} dakṣiṇahasta\edtext{sattvavajrī}{\lemma{%
	{\rm -sattvavajrī-\lem}
}\Dfootnote{%
	\emn\ \HT;
	\textit{satvavajrā} \cod
}}\edtext{madhyamāṅguliṃ}{\lemma{%
	{\rm -madhyamāṅguliṁ\lem}
}\Dfootnote{
	\emn\ \HT;
	\textit{madhyamaṅguli} \cod
}} \edtext{punaḥpunaḥ}{\lemma{%
	{\rm punaḥpunaḥ\lem}
}\Dfootnote{%
	\emn\ \HT;
	\textit{punaḥpuna} \cod
}} \edtext{sphoṭayan}{\lemma{%
	{\rm sphoṭayan\lem}
}\Dfootnote{%
	\cod;
	\textit{sphoṭayet} \emn\ \HT\ \sil
}} / {\ah}kāreṇa yathāvaddṛḍhīkṛtya tadāveśaṃ yāvat /}
	%catuḥpramāṇādika-: em. catuḥpraṇāmādika-, with HT (whose reading catuḥpraṃāṇādika- is a typing error).
	%sa\cancelled{r}tvavajrīṁ: read sattvavajrīṁ; sarvavajrīṁ HT.
	%-muṣṭhyā: em. muṣṭyā, with HT (sil. em.).
	%-satvavajrā-: em. -sattvavajrī-, with HT.
	%-madhyamaṅguli: em. -madhyamāṅguliṁ, with HT.
	%punaḥ puna: HT emends punaḥ punaḥ.
	%sphoṭayan·: HT unnecessarily emends to sphoṭayet.
\pend

\verse
\Skt{\edtext{vajraṃ tattvena \edtext{saṃgṛhya}{\lemma{%
	{\rm saṁgṛhya\lem}
}\Dfootnote{%
	\emn\ \HT\ \sil;
	\textit{sagṛhya} \cod
}} ghaṇṭāṃ dharmeṇa vādya ca /\\
samayena mahāmudrāmadhiṣṭhāya hṛdayaṃ japediti //}{\lemma{%
	{\rm vajraṁ \dots\ japet\lem}
}\Bfootnote{
	\textit{Vajrāvalī} (Abhiṣeka) cites same verse from \textit{Paramādya}; 
	also cited in \textit{Hevajrasekapakriyā}.
	\textcolor{red}{LOCATION!!}
	\textcolor{red}{CROSS REFERENCE!!}
	The same verse is quoted above. See p.\ \pageref{paramadyaverse1}.%
}}}\label{paramadyaverse1-2}
\pend

\pstart\noindent
\Skt{pūrvoktavidhiṃ kṛtvā vakṣyamānagāthāpañcakenānujñāmudgatāvyākaraṇaṃ cādāya śrīvajrasattvātmamantram /}
	%Vajrāvalī (Abhiṣeka) cites same verse from Paramādya; also cited in Hevajrasekapakriyā
	%sa§gṛhya: em. saṁgṛhya, with HT.
%\pend
%
% \pstart
\Skt{tataḥ svādhiṣṭhānādikaṃ kṛtvā suratavajro'hamityādya\edtext{nyataraṃ}{\lemma{%
	{\rm anyataraṁ\lem}
}\Dfootnote{%
	\emn\ \HT;
	\textit{anyantaraṁ} \cod
}} nāmoccā\textcolor{red}{rya} vairocanamahāmudrāṃ baddhvā tatsthāne vajradhātu {\ah} iti / tathāgatavajramātmānamāveśayet / vajro'ham / tato vajradhāturahamiti \edtext{tadvajraṃ}{\lemma{%
	{\rm tadvajraṃ\lem}
}\Dfootnote{%
	\emn\ \HT\ (who reads \textit{tanvajrām});
	\textit{tatvajrām} \cod;
	Or emend tattvajām (sc. ghaṇṭām)? Tib. seems to imply vajradhātu: rdo rje dbyings su bsgom par bya’o/.
}} bhāvayet / evaṃ yāvadvajrāveśamahāmudrāṃ baddhvā tatsthāne vajrāveśa {\ah} iti / \edtext{vajraghaṇṭā}{\lemma{%
	{\rm vajraghaṇṭām\lem}
}\Dfootnote{%
	\emn\ \HT;
	\textit{vajrāghaṇṭām} \cod
}}mātmānamāveśayet / vajraghaṇṭāham / tato vajrāveśo'ham iti tadghaṇṭāṃ bhāvayet / evaṃ vajreṇa \edtext{sādhitaṃ}{\lemma{%
	{\rm sādhitaṁ\lem}
}\Dfootnote{
	\emn\ \HT;
	\textit{mādhitaṁ} \cod
}\lemma{}\Efootnote{%
	sādhita[53r1]ṁ bhavati
}} bhavati /}
 	%anyantaraṁ: em. anyataraṁ, with HT.
 	%nāmoccāya: em. nāmoccārya, with HT.
 	%tatvajrām: em. tadvajram, with HT (who reads tanvajrām). Or emend tattvajām (sc. ghaṇṭām)? Tib. seems to imply vajradhātu: rdo rje dbyings su bsgom par bya’o/.
 	%vajrāghaṇṭām: em. vajraghaṇṭām, with HT.'
 	%mādhitaṁ: em. sādhitaṁ, with HT.
\pend

\verse
\Skt{%
\edtext{sattvavajrāṅkuśīṃ baddhvā \edtext{vajrācā\textcolor{red}{rya}}{\lemma{%
	{\rm vajrācāryas\lem}
}\Dfootnote{%
	\emn\ \HT;
	\textit{vajrāṁcāyās} \cod
}}stataḥ punaḥ /\\
kurva\edtext{nnacchaṭasaṃghātaṃ}{\lemma{%
	{\rm acchaṭasaṁghātaṁ\lem}
}\Dfootnote{%
	\emn;
	\textit{acchaṭasampātaṁ} \cod;
	\textit{acchaṭāsaṁghātaṁ} \emn\ \HT.
	We retain the reading \textit{acchaṭa-} of the ms., because we assume the unexpected short final vowel is intentional (metri causa).
}} sarvabuddhān samājayet //}{\lemma{%
	{\rm sattvavajrāṅkuśīṁ \dots\ samājayet\lem}
}\Bfootnote{%
%
= STTS 208:
\textit{sattvavajrāṅkuśīṁ baddhvā vajrācāryas tataḥ punaḥ |
kurvann acchaṭasaṃghātaṁ sarvabuddhān samājayet ||};
= SDPT: \textit{sattvavajrāṃkuśīṁ baddhvā vajrācāryaḥ tataḥ punaḥ |
kurvan acchaṭāsaṃghātaṃ sarvabuddhān samājayet ||} (p.\ 260)
%
}} }
\pend
\pstart\noindent
\Skt{\edtext{{\om} vajrasamāja jaḥ hū{\cb} va{\cb} hoḥ \edtext{pravartayan}{\lemma{%
	{\rm pravartayan\lem}
}\Dfootnote{%
	\emn\ \HT;
	\textit{pravarttayam} \cod
}}}{\lemma{%
	{\rm oṁ vajrasamāja \dots\ pravartayan\lem}
}\Bfootnote{%
	SDPT: oṃ vajrasamāja jaḥ vaṃ hoḥ iti / ekaviṃśativāraṃ pravartayet / (p.\ 260)
}} /}
	%vajrāṁcāyās: em. vajrācāryas, with HT.
	%acchaṭasampātaṁ: em. acchaṭasaṁghātaṁ. We agree with HT's emendation -saṁghātam, but retain the reading acchaṭa- of the ms., because we assume the unexpected short final vowel is intentional (metri causa).
	%pravarttayam·: em. pravartayan, with HT.
\pend

\verse
\Skt{\edtext{tataḥ \edtext{śīghraṃ}{\lemma{%
	{\rm śīghraṃ\lem}
}\Dfootnote{
	\emn\ \HT;
	\textit{śrīghram} \cod
}} mahāmudrāṃ vajrasattvasya \edtext{sevayan}{\lemma{%
	{\rm sevayan\lem}
}\Dfootnote{%
	\emn\ \HT;
	\textit{sevayam} \cod
}} /\\ 
uccārayetsakṛdvāraṃ nāmāṣṭaśatamuttamam}{\lemma{%
	{\rm tataḥ \dots\ uttamam\lem}
}\Bfootnote{%
%
$\simeq$ STTS 208:
\textit{tataḥ śīghraṁ mahāmudrāṁ bhāvya vajradharasya tu |
uccārayet sakṛdvāraṁ nāmāṣṭaśatam uttamam ||};
$\simeq$ SDPT: 
\textit{tataḥ śīghraṃ mahāmudrāṃ vajrakrodhasamayam uccārayet / 
sakṛdvāraṃ nāmāṣṭakaṃ śataṃ uttamam /} (p.\ 260)
While the relevant part of the \textit{Sarvavajrodayā} is metrical,
this part of the SDPT is unmetrical.
%
}} //}
	%śrīghram: em. śīghram, with HT.
	%sevayam·: em. sevayan, with HT.
\pend

\pstart
\edtext{}{\lemma{%
	{\rm tathaiva \dots\ bhadrakalpikaparyantān\lem}
}\Bfootnote{%
%
SDPT: tato vajrāṁkuśādibhiḥ svadvāreṇākṛṣya praveśya baddhvā vaśīkṛtya,
yathāvac caturhuṃkāreṇārghaṃ datvā, 
śrīvairocanādīn samayamudrābhir bhadrakalpikaparyantān sādhayet. svamantram uccārya vadet, jaḥ huṃ vaṃ hoḥ samayas tvaṃ samayas tvam aham.
tataḥ svamantram uccārayed eva siddhā bhavanti. (p.\ 260)
%
}}%
\Skt{tathaiva vajrāṅkuśādibhiḥ ākṛṣya praveśya baddhvā vaśīkṛtya / vajrayakṣeṇa \edtext{vighnotsāraṇaṃ}{\lemma{%
	{\rm vighnotsāraṇaṁ\lem}
}\Dfootnote{%
	\cod;
	\textit{vighṇotsāraṇaṁ} \HT\ (misprint)
}} \edtext{prākārapañjaraṃ ca}{\lemma{%
	{\rm prākārapañjaraṁ ca\lem}
}\Dfootnote{%
	\emn;
	\textit{prākāraṃ pañjaraṃ} \cod\ \HT.
	Cf.\ §36 \textit{vajradṛṣṭyā diksīmāmaṇḍalabandhaprākārapañjaraṁ pādatalaparigraheṇa bhūmitalam upādāya ...}
	§17 \textit{vajrayakṣeṇa vighnotsāraṇaṁ rakṣāṁś ca kṛtvā ..}
}} kṛtvā samayavajramuṣṭinā maṇḍaladvārāṇi baddhvā \edtext{dvyakṣara}{\lemma{%
	{\rm dvyakṣara-\lem}
}\Dfootnote{%
	\emn\ \HT;
	\textit{\textcolor{red}{trya}kṣara-} \cod;
	KSP Devatāyoga (6-2-1-5): \textit{dvyakṣarakavacena kavacayet. vajramuṣṭidvayena hṛdaye granthyābhinayaṃ kuryāt.} (LOCATION!!);
	SDPT p. 134 \textit{oṃ ṭuṃ iti / anena dvyakṣarakavacena kavacayitvā}
	% \HT\ reads \textit{tryakṣara}- and emends to \textit{dvyakṣara-}, following Tib. An emendation \textit{tya-} to \textit{dvya-} would be palaeographically quite easy.
}}kavacena sarvarakṣāḥ saṃrakṣyārghadānapūrvikābhiḥ svasamaya\edtext{mudrābhirdṛśyaṃ}{\lemma{%
	{\rm -mudrābhir dṛśyaṁ\lem}
}\Dfootnote{%
	\emn\ \HT\ (silently for the -r);
	\textit{-mudrābhidṛśyā} \cod
}} kṛtvā}
	%vadhvā: nomalize here and elsewhere baddhvā.
	%vighnotsāraṇaṁ: vighṇotsāraṇaṁ HT (misprint).
	%tyakṣara-: em. tryakṣara-? HT reads tryakṣara- and emends to dvyakṣara-, following Tib. An emendation tya- to dvya- would be palaeographically quite easy.
	%-mudrābhidṛśyā: em. -mudrābhir dṛśyaṁ, with HT (silently for the -r). 
	%Cf. Devatāyoga section of KSP:  tato vajraa.nku.caadibhir aak.r.sya prave.cya baddhvaa va.ciik.rtyaarghaadibaahyapuujaadibhi.h sa.mpuujya samaya.m ca dar.cayet. \begin{quote} o.m vajradhaatu d.r.cya ja.h huu.m va.m ho.h samayas tva.m samayo 'ham. o.m vajrasattva d.r.cya ja.h huu.m va.m ho.h samayas tva.m samayo 'ham. o.m ratnavajra d.r.cya ja.h huu.m va.m ho.h samayas tva.m samayo 'ham.
%
%
\Skt{\edtext{jaḥ}{\lemma{%
	{\rm jaḥ\lem}
}\Dfootnote{%
	\emn\ \HT\ \sil;
	\textit{jāḥ} \cod
}} hū{\cb} va{\cb} hoḥ \edtext{pravartayan}{\lemma{%
	{\rm pravartayan\lem}
}\Dfootnote{%
	\emn\ \HT\ \sil;
	\textit{pravarttayam} \cod
}} / samayastvaṃ / samayastvamahamiti ca / svahṛdayāni mantrāṃścānte saṃsādhya / dharmakarmamahāmudrābhi\edtext{ścāmudryābhiṣiñce}{\lemma{
	{\rm cāmudryābhiṣiñcen\lem}
}\Dfootnote{%
	\emn;
	\textit{cāmudrābhiṣiñce<n>} \cod;
	\textit{cāmudryābhiṣiñced} \emn\ \HT\ \sil
	\HT\ does not note that \textit{akṣara dmu} is lost at the beginning of line 5.
	\textcolor{red}{Make a new command for "lost akṣara"}
	% \newcommand{\lostaksaras}[1]{\_}
}}\supplied{nmu}drābhiṣekaistathāgatādīnbhadrakalpika\edtext{paryantān}{\lemma{%
	{\rm -paryantān\lem}
}\Dfootnote{%
	\emn\ \HT;
	\textit{pa\textcolor{red}{rya}ntāṁ} \cod
}}~//}
	%pravarttayam·: em. pravartayan, with HT (sil. em.).
	%cāmudrābhiṣiñced: em. cāmudryābhiṣiñced, with HT. 
	%HT does not note that akṣara dmu is lost at the beginning of line 5.
	%-payantāṁ: em. -paryantān, with HT (silently for r).
\pend

\pstart
\Skt{\edtext{tatra sattvavajrādīnāṃ svahṛdayānyeva dharmamudrāḥ / \edtext{\textcolor{red}{vajraratnadharmakarmaṇāṃ}}{\lemma{%
	{\rm vajraratnadharmakarmaṇāṁ\lem}
}\Dfootnote{%
	\emn\ ?
	vajrasatvā | ratna | dharmma | karmaṇām | \cod;
	\HT\ emends \textit{vajrasattvaratnadharmakarmāṇāṁ}.
}}karmamudrāmahā\supplied{mudrā}śca  \edtext{vajrādyantargarbhāḥ}{\lemma{%
	{\rm vajrādyantargarbhāḥ\lem}
}\Dfootnote{%
	\emn;
	\textit{vajrādyantargatāḥ} \emn\ \HT;
	\textit{vajrādyantargarttāḥ} \cod
}} strīrūpadhāriṇyo vajrasattvādirūpāśca tā iti //}{\lemma{%
	{\rm tatra sattvavajrādīnāṁ \dots\ sāmānyeti\lem}
}\Bfootnote{%
	SDPT p.266:
	\textit{tato yathālekhānusārato vairocanādīnāṁ mahāmudrābandhaṁ kuryāt / vajrasattvaratnadharmakarmakrodhānāṁ yā mahāmudrās tāḥ sattvaratnadharmakarmavajriṇām api vajrādyantargatāḥ strīrūpadhāriṇyaś ca tāḥ / yāṁ yāṁ mudrāṁ tu badhnīyād yasya yasya mahātmanaḥ. sarvamudrāsamayaḥ / }
}} }
\pend

\pstart
\Skt{%
\edtext{vāmatathāgata\edtext{muṣṭimuttānaṃ}{\lemma{%
	{\rm -muṣṭim uttānaṁ\lem}
}\Dfootnote{%
	\emn\ \HT;
	\textit{muṣṭimudrāttānaṁ} \cod
}} kṛtvā dakṣiṇahastatarjanyaṅguṣṭhābhyāṃ kanīyasī\edtext{mā\supplied{rabhya}}{\lemma{}\Efootnote{%
	ā[53v1]\supplied{rabhya}
}} vikāsya saṃpuṭāñjaliṃ ku\textcolor{red}{ryā}t / 
\edtext{iyaṃ}{\lemma{%
	{\rm iyaṁ\lem}
}\Dfootnote{%
	\emn\ \HT;
	\textit{īyā} \cod;
	Note particular form of \textit{ī} based on \textit{i} with vowel mark \textit{-ī}.
}} maitreyādīnāṃ samayamudrā / vidyā caiṣāṃ pūrvoktā / \edtext{tāmeva}{\lemma{%
	{\rm tām eva\lem}
}\Dfootnote{%
	\emn\ \HT;
	\textit{nāmeva} \cod
}} vidyāṃ teṣāṃ jihvāsu \edtext{nyaset}{\lemma{%
	{\rm nyaset\lem}
}\Dfootnote{%
	\emn\ \HT;
	\textit{nyasyed} \cod
}} / iyaṃ teṣāṃ dharmamudrā / {\ah}kā\supplied{re}ṇa svahṛdi viśvavajraṃ niṣpādya teṣāṃ lekhyānusārato \edtext{mahāmudrāṃ}{\lemma{%
	{\rm mahāmudrām\lem}
}\Cfootnote{%
	\HT\ needlessly emends to \textit{mahāmudrā}.
}} baddhvā karmamudrā bhavanti / svahṛdi pañcasūcikaṃ \edtext{vajraṃ}{\lemma{%
	{\rm vajraṁ\lem}
}\Dfootnote{%
	\emn\ \HT;
	\textit{vajrām} \cod
}} vicintya lekhyānusārata eva teṣāṃ mahāmudrā bandhanīyāḥ / 
\edtext{saiva}{\lemma{%
	{\rm saiva\lem}
}\Dfootnote{%
	\textit{saivaṃ ca} \cod\ \HT
}} vidyā sāmānyeti //}{\lemma{%
	{\rm vāmatathāgata- \dots\ sāmānyeti\lem}
}\Bfootnote{%
	SDPT p.\ 266: \textit{vāmatathāgatamuṣṭim uttānaṁ kṛtvā dakṣiṇahastatarjanyaṁguṣṭhābhyāṁ kanyasām ārabhya vikāsya saṁpuṭāñjaliṁ kuryāt / iyaṁ maitreyādīnāṁ samayamudrā / vidyā caiṣāṁ pūrvoktā / tām eva vidyāṁ teṣāṁ jihvāsu nyased iyaṁ teṣāṁ dharmamudrāḥ / \textcolor{red}{a}kāreṇa svahṛdi viśvavajraṁ niṣpādya teṣāṁ lekhyānusārato mahāmudrā baddhvā karmamudrā bhavanti / svahṛdi pañcasūcikaṁ vajraṁ vicintya lekhyānusārata eva teṣāṁ mahāmudrā badhnīyāḥ / saiva vidyā sāmānyaiveti /}
}} }
	%vajrasatvā | ratna | dharmma | karmāṇām· |: em. vajraratnadharmakarmaṇāṁ? HT emends vajrasattvaratnadharmakarmāṇāṁ.
	%vajrādyantargarttāḥ: em. vajrādyantargatāḥ, with HT.
	%-muṣṭimudrāttānaṁ: em. -muṣṭim uttānaṁ, with HT.
	%kuyāt: em. kuryāt, with HT (sil. em.).
	%°īyā: em. iyaṁ, with HT. Note particular form of °ī based on °i with vowel mark -ī.
	%nāmeva: em. tām eva, with HT.
	%nyasyed: em. nyasyed, with HT.
	%mahāmudrām: HT needlessly emend to mahāmudrā.
	%vajrām: em. vajram, with HT.
	%SDPT p. 266 tato yathālekhānusārato vairocanādīnāṁ mahāmudrābandhaṁ kuryāt / vajrasattvaratnadharmakarmakrodhānāṁ yā mahāmudrās tāḥ sattvaratnadharmakarmavajriṇām api vajrādyantargatāḥ strīrūpadhāriṇyaś ca tāḥ / yāṁ yāṁ mudrāṁ tu badhnīyād yasya yasya mahātmanaḥ. sarvamudrāsamayaḥ / vāmatathāgatamuṣṭim uttānaṁ kṛtvā dakṣiṇahastatarjanyaṁguṣṭhābhyāṁ kanyasām ārabhya vikāsya saṁpuṭāñjaliṁ kuryāt / iyaṁ maitreyādīnāṁ samayamudrā / vidyā caiṣāṁ pūrvoktā / tām eva vidyāṁ teṣāṁ jihvāsu nyased iyaṁ teṣāṁ dharmamudrāḥ / akāreṇa svahṛdi viśvavajraṁ niṣpādya teṣāṁ lekhyānusārato mahāmudrā baddhvā karmamudrā bhavanti / svahṛdi pañcasūcikaṁ vajraṁ vicintya lekhyānusārata eva teṣāṁ mahāmudrā badhnīyāḥ / saiva vidyā sāmānyaiveti /
\pend

\pstart
\Skt{%
\edtext{}{\lemma{%
	{\rm nānāprakārāṇi \dots\ saptaśo'bhimantrya\lem}
}\Bfootnote{%
	SDPT, p. 270: \textit{pūrṇakumbhāś ca vastrayugalakṣaṃ daśasahasrakaṃ
śataṃ pratyekam ekaṃ vā sarvasāmānyam / nānāprakārāṇi vitānāni catuḥkoṇe vicitrapatākāvasaktāni chatradhvajapatākāśca / oṃkāreṇa śrīvajrahuṁkāreṇa ca parijapya /}
}}%
tataḥ \edtext{pūjāṃ}{\lemma{%
	{\rm pūjāṁ\lem}
}\Dfootnote{%
	\emn\ \HT;
	\textit{pūjyā} \cod;
	\HT\ correctly reports \textit{pūjyā} but silently emends \textit{kuryād}.
}} ku\textcolor{red}{ryā}\textcolor{blue}{t /} arghaṃ dattvā vastrayugalakṣaṃ  daśasahasraṃ sahasraṃ \edtext{śataṃ}{\lemma{%
	{\rm śataṁ\lem}
}\Dfootnote{%
	\emn\ \HT;
	\textit{śata} \cod
}} pratyekaikaṃ vā sarvasāmānyam / 
nānāprakārāṇi vitānāni \edtext{catuḥko\textcolor{blue}{ṇe}}{\lemma{%
	{\rm catuḥko\textcolor{blue}{ṇe}\lem}
}\Cfootnote{%
	\HT\ reports \textit{-kona} as reading (probably a printing error) and emends \textit{catuḥkoṇe} as separate word. Perhaps \HT's emendation is not necessary, although it is also easy to imagine how an original text \textit{-koṇe} would have come to be misread as \textit{-koṇa}.
}\Dfootnote{
	\emn\ \HT;
	\textit{catuḥkoṇa-} \cod
}} vicitrapatākāvasaktāni \textcolor{red}{cha}trapatākāśca / 
{\om}kāreṇa vajrasattvena ca \edtext{saptaśo}{\lemma{%
	{\rm saptaśo\lem}
}\Dfootnote{%
	\emn;
	\textit{saptaso} \cod;
	\HT\ emends \textit{saptam}.
}}\edtext{'bhimantrya}{\lemma{%
	{\rm 'bhimantrya\lem}
}\Dfootnote{%
	\emn\ \HT;
	\textit{bhimantryaṁ} \cod
}} /}
	%pūjyā kuyād: em. pūjāṁ kuryād. HT correctly reports pūjyā but silently emend kuryād.
	%śata: em. śataṁ, with HT.
	%vastrayugalakṣaṁ | daśasahasram· | sahasraṁ śata | pratyekaikam vā sarvasāmānyaṁ:  punctuation to be removed in edition.
	%catuḥkoṇa-: HT reports -kona as reading (probably a printing error) and emends catuḥkoṇe as separate word. Perhaps HT's emendation is not necessary, although it is also easy to imagine how an original text -koṇe would have come to be misread as -koṇa.
	%saptaso: em. saptaśo; HT emends saptam.
	%bhimantryaṁ: em. 'bhimantrya, with HT.
	%SDPT, p. 270: nānāprakārāṇi vitānāni catuḥkoṇe vicitrapatākāvasaktāni chatradhvajapatākāś ca / oṁkāreṇa śrīvajrahuṁkāreṇa ca parijapya / 
% \pend
% 
% \pstart
\Skt{%
\edtext{}{\lemma{%
	{\rm puṣpavṛkṣaśatam \dots\ sarvadevatābhyo niryātayet\lem}
}\Bfootnote{%
	SDPT p. 270: \textit{vajrasphara\textcolor{red}{ṇa}m ity anena sarvadevatān niryātayet / puṣpavṛkṣaśataṁ caturo vā vṛkṣān sarvapuṣpāṇi ca / vajrānalena mudrāyuktena / oṁ vajrapuṣpe huṁ iti ca puṣpamudrayābhimantrya / sarvagandhān suvāsāṁś ca vilepanasugandhakān tathaiva vajrānalena / oṁ vajragandhe huṁ ity anayā ca gandhamudrāyuktayā karpūrāguruturuskāni candanādisaṁmiśrāṇi / tathaiva vajrānalena oṁ vajradhūpe huṁ ity anayā dhūpamudrāsahitayā dhūpaghaṭikālakṣaṁ daśasahasraṁ śataṁ yathālābhaṁ vā pradīpalakṣaṁ daśasahasraṁ śataṁ sarvapradīpān pradīpavartijvalitayuktaṁ kuṇḍaṁ sahasraṁ ca daśaikaṁ vā / tathaiva vajrānalena oṁ vajrāloke *\textcolor{red}{huṁ (CHECK THE EDITION!!)} ity anayā pradīpamudrāyuktayā parijapya / oṁ vajraspharaṇam ity udīrayan niryātayet / balyupahāraṁ ca lakṣaṁ daśasahasraṁ śataṁ daśasaṁkhyāṁ vā svastikam āditaḥ kṛtvā nānāprakārāṇi ca bhakṣāṇi / tathaiva vajrānalena parijapya /}
}}%
puṣpavṛkṣaśataṃ caturo vā \edtext{vṛkṣān}{\lemma{%
	{\rm vṛkṣān\lem}
}\Dfootnote{
	\emn\ \HT;
	\textit{vṛkṣā} \cod
}} sarvapuṣpāṇi ca pūrvava\edtext{dabhimantrya}{\lemma{%
	{\rm abhimantrya\lem}
}\Dfootnote{
	\emn\ \HT;
	\textit{abhimaṁtrya} \cod
}} / \edtext{{\om} vajra sphara khamiti}{\lemma{%
	{\rm oṁ vajra sphara kham iti\lem}
}\Cfootnote{%
	Emend \textit{oṁ vajraspharaṇam iti} following the SDPT?
	\textit{oṁ vajrasphara kham iti} is found in the \textit{Vajrāvalī} 15.1
}} \edtext{niryātayet}{\lemma{%
	{\rm niryātayet\lem}
}\Dfootnote{
	\emn\ \HT\ \sil;
	\textit{niyātayet} \cod
}} /}
\pend

\smallskip

\verse
\Skt{sarvagandhān \edtext{suvāsāṃśca vilepanasugandhikān}{\lemma{%
	{\rm savāsāṁś ca vilepanasugandhikān\lem}
}\Dfootnote{%
	\emn;
	\textit{savāsāṁś ca vilepanasugandhikān} \emn\ \HT;
	\textit{savāsañ ca vilepaṇasugandhikam} \cod;
	\HT's note treats this as the same mistake as the one in the preceding line. The two cases are related but different.
	\textcolor{red}{CHECK TIB.!!}
}} }
\pend

\smallskip

\pstart\noindent
\Skt{gandhavidyayā /} 
%\pend
%
%\smallskip
%
%\verse
\Skt{karpūrā\textcolor{blue}{gu}ru\edtext{turuṣkāṇi}{\lemma{%
	{\rm -turuṣkāṇi\lem}
}\Dfootnote{%
	\emn\ \HT;
	\textit{turaṣkāṇi} \cod
}} candanādi\edtext{saṃmiśrāṇi}{\lemma{%
	{\rm -saṁmiśrāṇi\lem}
}\Dfootnote{%
	\emn\ \HT\ (who, however, misreads \textit{-n-} for \textit{-ṇ-})
	\textit{-samiśrāṇi} \cod
}}} 
%\pend
%
%\smallskip
%
%\pstart\noindent
\Skt{dhūpavidyayābhi\edtext{mantrya}{\lemma{%
	{\rm -mantrya\lem}
}\Dfootnote{%
	\emn\ \HT\ \sil;
	\textit{-maṁtrya} \cod
}} / dhūpaghaṭikālakṣaṃ daśasahasraṃ sahasraṃ śataṃ vā / daśanyūnam  na kā\textcolor{red}{rya}m /} 
\pend

\verse
\Skt{\edtext{ghṛtapradīpa}{\lemma{%
	{\rm ghṛtapradīpa-\lem}
}\Dfootnote{
	\emn\
	\textit{ghṛtapradīpādi-} \cod\ \HT
	\textcolor{red}{CHECK AGAIN!!}
}}lakṣādisaṃkhyaṃ}
\pend

\pstart\noindent
\Skt{\edtext{pradīpakuṇḍasahasraṃ}{\lemma{%
	{\rm pradīpakuṇḍasahasraṁ\lem}
}\Dfootnote{%
	\emn\ \HT\ \sil;
	\textit{pradīpakuṇḍasahasra} \cod
}} \edtext{śataṃ}{\lemma{}\Efootnote{%
	-sahasraṁ [54r1] śataṁ
}} daśakuṇḍāni catvāri vā pūrvavatpradīpamantreṇābhi\edtext{mantrya}{\lemma{%
	{\rm -mantrya\lem}
}\Dfootnote{%
	\emn\ \HT\ \sil;
	\textit{-maṁtrya} \cod
}} /} 
\pend

\verse
\Skt{svastikamāditaḥ kṛtvā} 
\pend

\pstart\noindent
\Skt{\edtext{balyupahāraṃ}{\lemma{%
	{\rm balyupahāraṁ\lem}
}\Dfootnote{%
	\emn\ \HT\ \sil;
	\textit{valyūpahāraṁ} \cod
}} lakṣa\textcolor{red}{rūpakaṃ} daśasahasraṃ śataṃ daśasaṃkhyaṃ vā nānāprakārāṇi ca bhakṣyāni \edtext{pūrvavadeva akāro}{\lemma{%
	{\rm pūrvavad eva akāro\lem}
}\Dfootnote{%
	\emn\ \HT\ \sil;
	\textit{pūrvvavaddevākāro} \cod
}} \edtext{mukha}{\lemma{%
	{\rm mukham\lem}
}\Dfootnote{%
	\emn\ \HT\ \sil;
	\textit{mukhyam} \cod
}}mityādinā \edtext{sarvadevatābhyo}{\lemma{%
	{\rm sarvadevatābhyo\lem}
}\Dfootnote{%
	\cod;
	\textit{sarvadevatābhyāṁ} \HT
}} ni\textcolor{red}{ryā}tayet /}
\pend
	%vṛkṣā: em. vṛkṣān, with HT.
	%abhimaṁtrya: em. abhimantrya, with HT. HT's note treats this as the same mistake as the one in the preceding line. The two cases are related but different.
	%niyātayet: em. niryātayet, with HT (sil. em.).
	%savāsañ ca vilepaṇasugandhikam·: em. savāsāṁś ca vilepanasugandhikān, with HT (who, however, does not emend -ṇ- to -n-). See SDPT parallel below.
	%-turaṣkāṇi: em. -turuṣkāṇi, with HT.
	%-samiśrāṇi: em. -saṁmiśrāṇi, with HT (who, however, misreads -n- for -ṇ-).
	%-maṁtrya: em. -mantrya, with HT (sil. em.).
	%kāyaṁ:  em. kāryaṁ, with HT (sil. em.).
	%pradīpakuṇḍasahasra: em. pradīpakuṇḍasahasraṁ, with HT (sil. em.).
	%-maṁtrya: em. -mantrya, with HT (sil. em.).
	%valyūpahāraṁ: em. balyupahāraṁ, with HT (sil. em.).
	%pūrvvavaddevākāro: em. pūrvavad eva °akāro, with HT (sil. em.).
	%mukhyam: em. mukham, with HT (sil. em.).
	%sarvadevatābhyo: sarvadevatābhyāṁ HT.
	%niyātayet: em. niryātayet, with HT (sil. em.).
	%SDPT p. 270: vajraspharaṇam ity anena sarvadevatān niryātayet / puṣpavṛkṣaśataṁ caturo vā vṛkṣān sarvapuṣpāṇi ca / vajrānalena mudrāyuktena / oṁ vajrapuṣpe huṁ iti ca puṣpamudrayābhimantrya / sarvagandhān suvāsāṁś ca vilepanasugandhakān tathaiva vajrānalena / oṁ vajragandhe huṁ ity anayā ca gandhamudrāyuktayā karpūrāguruturuskāni candanādisaṁmiśrāṇi / tathaiva vajrānalena oṁ vajradhūpe huṁ ity anayā dhūpamudrāsahitayā dhūpaghaṭikālakṣaṁ daśasahasraṁ śataṁ yathālābhaṁ vā pradīpalakṣaṁ daśasahasraṁ śataṁ sarvapradīpān pradīpavartijvalitayuktaṁ kuṇḍaṁ sahasraṁ ca daśaikaṁ vā / tathaiva vajrānalena oṁ vajrāloke ity anayā pradīpamudrāyuktayā parijapya / oṁ vajraspharaṇam ity udīrayan niryātayet / balyupahāraṁ ca lakṣaṁ daśasahasraṁ śataṁ daśasaṁkhyāṁ vā svastikam āditaḥ kṛtvā nānāprakārāṇi ca bhakṣāṇi / tathaiva vajrānalena parijapya /
% \pend
%
\verse
\edtext{}{\lemma{%
	{\rm daśavādyasahasrāṇi \dots\ tathā\lem}
}\Bfootnote{%
%
SDPT (p.272): \textit{daśavādyasahasrāṇi sahasraśataṁ vādyāni daśavādyāni vā huṁkāreṇa vādyamudrābhir vajramuṣṭibhyāṁ karāṁgulibhir vādyābhinayadaśaprakārāḥ. tadyathā vīṇāvaṁśāmurajāmukundākāṁsābherīmṛdaṁgapaṭahaguṁjātimilābhinayaś ceti. vādyanaṭanartakakuṇḍalamukuṭādipājāś ca. oṁkāreṇābhimantrya. tathā \dots}
%
}}%
\Skt{daśavādyasahasrāṇi}
\pend

\pstart\noindent
\Skt{\edtext{daśavādyākāreṇa}{\lemma{%
	{\rm daśavādyākāreṇa\lem}
}\Dfootnote{%
	\emn\ \HT\ \sil;
	\textit{daśavāhy\unclear{aṁ}kāreṇa} \cod
}} vādyamudrābhiḥ vajramuṣṭibhyāṃ \edtext{karāṅgulībhi}{\lemma{%
	{\rm karāṅgulībhir\lem}
}\Dfootnote{%
	\emn;
	\textit{kārāṅgulī\supplied{bhiḥ}} \cod
	\textit{kārāṅgulībhir} \HT;
	However, the first preserved \textit{akṣara} on line 3 does not show a \textit{repha}, so we presume the restoraration must be \textit{-bhiḥ} without \textit{sandhi}, as previously in this sentence. So we emend \textit{karāṅgulībhiḥ}.
}}\edtext{rvādyābhinayo daśaprakāraḥ}{\lemma{%
	{\rm vādyābhinayo daśaprakāraḥ\lem}
}\Dfootnote{%
	\cod;
	\textit{vādyābhinaye | daśaprakāraḥ} \HT
}} tadyathā vīṇāvaṅśamuraja\edtext{\textcolor{red}{kurunda}}{\lemma{%
	{\rm -kurunda-\lem}
}\Dfootnote{%
	\cod;
	\HT\ silently emends \textit{-mukunda-}.
}}\edtext{kānsī}{\lemma{%
	{\rm -kānsī-\lem}
}\Dfootnote{%
	\cod;
	\HT\ silently emends \textit{-kāṁsi-}.
}}bherīmṛdaṅgapaṭahaguñjatimilābhinayaśceti / \edtext{vādya}{\lemma{%
	{\rm vādya-\lem}
}\Dfootnote{%
	\emn\ \HT\ \sil;
	\textit{vāhya-} \cod;
	But note ms. \textit{vāhyanṛtya-} in §76. Maybe we should not emend in either place?
}}naṭa\-nartakamakuṭakaṭaka\-kuṇḍalādi\edtext{pūjā}{\lemma{%
	{\rm -pūjāś\lem}
}\Dfootnote{%
	\emn\ \HT\ \sil;
	\textit{pūjyāś} \cod
}}śca /} 
% \pend
% 
% \verse
\Skt{{\om}kāreṇa hūṃkāreṇa vābhimantrya \edtext{\textcolor{blue}{niryātayet}}{\lemma{%
	{\rm niryātayet\lem}
}\Dfootnote{%
	\emn;
	\textit{nipātayet} \cod\ \textcolor{red}{CHECK!!}
}} /}
%\pend
%
%\pstart\noindent
\Skt{tathā} 
\pend

\smallskip

\verse
\Skt{%
\edtext{\edtext{\textcolor{blue}{paṭṭāvalambanā}}{\lemma{%
	{\rm paṭṭāvalambanā\lem}
}\Dfootnote{%
	\emn;
	\textit{paṭāvalambanā} \cod\ \textcolor{red}{(CHECK!!)}
}} kā\textcolor{red}{ryā} \edtext{srakcāmara}{\lemma{%
	{\rm srakcāmara-\lem}
}\Cfootnote{%
	the correct decipherment of the \textit{akṣara} provisionally transliterated as \textit{-kḍā-} is uncertain. Other options: \textit{-jḍā-, -kḍrā-.} \HT\ emends \textit{srak cāmara-}.
}\Dfootnote{
	\emn\ \HT;
	\textit{srakḍāmara} \cod
}}vibhūṣitā /\\
\edtext{hārārdhahāraracitā sārdhacandropaśobhitā}{\lemma{%
	{\rm hārā\dots śobhitā\lem}
}\Dfootnote{%
	\emn;
	\textit{hārārdhahāraracitasārdhacandropaśobhitā} \cod;
	\textit{hārārdhahāraracitārdhacandropaśobhitā} \HT, without note. 
	Was the removal of \textit{-sārdha-}  a conscious decision on the part of HT? Probably the required meaning can be obtain with \textit{sa-} as 'with'.
}} //\\
\edtext{\textcolor{blue}{turaṃga}}{\lemma{%
	{\rm turaṃga-\lem}
}\Dfootnote{%
	\emn;
	\textit{turagā-} \cod
}}hastigoyūthā dātavyāśca sukalpitāḥ /\\ 
toraṇāni ca ramyāṇi ghaṇṭādisahitāni ca //}{\lemma{%
	{\rm paṭā\dots\ śobhitā\lem}
}\Bfootnote{%
	To be traced.
	SDPT (p.272) \textit{tathā paṭāvalambanā kāryā srak cāmaravibhūṣitā / hārārdhahāra\textcolor{red}{racitāracitā}rdhacandropaśobhitā / turaṁgahastigoyūthā dātavyāś ca sukalpitāḥ /}
}} }
	%daśavāhy\unclear{aṁ}kāreṇa: em. daśavādyākāreṇa, with HT (sil. em.).
	%kārāṅgulī\lost{bhiḥ}: kārāṅgulībhir HT, emended to karāṅgulībhir. However, the first preserved akṣara on line 3 does not show a repha, so we presume the restoraration must be -bhiḥ without sandhi, as previously in this sentence. So we emend karāṅgulībhiḥ.
	%\unclear{vādyā}bhinayo daśaprakāraḥ: HT emends vādyābhinaye | daśaprakāraḥ.
	%-kurunda-: HT silently emends -mukunda-.
	%-kānsī-: HT silently emends -kāṁsi-.
	%vāhya-: em. vādya-, with HT (sil. em.). But note ms. vāhyanṛtya- in §76. Maybe we should not emend in either place?
	%-pūjyāś: em. -pūjāś, with HT.
	%kāyā: em. kāryā, with HT (sil. em.).
	%-srakḍāmara-: the correct decipherment of the akṣara provisionally transliterated as -kḍā-  is incertain. Other options: -jḍā-, -kḍrā-. HT emends srak cāmara-.
	% hārārddhahāraracitasārddhacandropa*śobhitā: hārārdhahāraracitārdhacandropaśobhitā  HT, without note. Was the removal of -sārdha-  a conscious decision on the part of HT? Probably the required meaning can be obtain with sa- as 'with'.
	%SDPT 272: daśavādyasahasrāṇi sahasraśataṁ vādyāni daśavādyāni vā huṁkāreṇa vādyamudrābhir vajramuṣṭibhyāṁ karāṁgulibhir vādyābhinayadaśaprakārāḥ / tadyathā vīṇāvaṁśāmurajāmukundākāṁsābherīmṛdaṁgapaṭahaguṁjātimilābhinayaś ceti / vādyanaṭanartakakuṇḍalamukuṭādipājāś ca / oṁkāreṇābhimantrya / tathā paṭāvalambanā kāryā srak cāmaravibhūṣitā / hārārdhahāraracitāracitārdhacandropaśobhitā / turaṁgahastigoyūthā dātavyāś ca sukalpitāḥ /
		%% Note that two mss. reported by Skorupski read srakdāmara-. There is a common compound sragdāma-.
\pend

\smallskip

\pstart
\Skt{%
\edtext{tato vajralāsyādibhiḥ sampūjya / \edtext{sarvasattvārthaṃ}{\lemma{%
	{\rm sarvasattvārthaṁ\lem}
}\Dfootnote{%
	\emn\ \HT;
	\textit{sarvasatvārtha} \cod
}} kurudhvaṃ sarvasiddhaya i\supplied{ti} / \edtext{sarvatathāgata}{\lemma{%
	{\rm sarvatathāgata-\lem}
}\Dfootnote{%
	\emn\ \HT;
	sa\supplied{r}vatathāgata \cod;
	\HT\ does not note that the \textit{r} is supplied, having presumably been lost by damage to the leaf.
}}vijñaptiṃ ku\textcolor{red}{ryā}t /}{\lemma{%
	{\rm tato vajralāsyādibhiḥ \dots\ kuryāt\lem}
}\Bfootnote{%
%
SDPT (p.\ 272):
\textit{punarvajrakulamaṇḍaloktaṣoḍaśakarmamudrābhir api saṃpūjya sarvakrodhakulavijñaptiṃ kuryāt / sarvasattvārthaṃ kurudhvaṃ sarvasiddhaya iti /};
%
STTS 209: \textit{tatas tu guhyapūjābhiḥ saṃtoṣya sa mahātmanām |
\textbf{vijñāpet sarvasattvārthaṃ kurudhvaṃ sarvasiddhaya} iti ||}
\textcolor{blue}{(Ānandagarbha uses the latter half of the stanza, and change the word order so that the Sanskrit becomes more natural.)};
%
\textit{Guhyasamājamaṇḍalavidhi} v.354:
\textit{cakṣuḥkāyādyadhiṣṭhānasekapūjādikalpite |
vijñāpayet sarvasattvārthaṃ kurudhvaṃ sarvasiddhaye ||}
(hyper metrical.)
%
}} } 
\pend

\medskip

\pstart\noindent
3.8.1.2. bāhyabalim
\pend

\medskip

\pstart
\edtext{}{\lemma{%
	{\rm tato \dots tatreme mudrāmantrā bhavanti\lem}
}\Bfootnote{%
%
SDPT pp.272–274:
\textit{tato bāhyabaliṃ dadyād uttarasādhakaṃ maṇḍale pratiṣṭhāpya, salājaṃ satilaṃ sāmbhaḥ sabhaktaṃ kusumaih saha satilakādibhaktaiś cākārādinā parijapya, pūrvadigbhāgam ārabhya trikṣepāṃ gandhapuṣpadhūpadīpārghaṃ cādāv ante dadyāt. tatra pūrvaṃ tāvan maṇḍalāni kārayet. tata āvāhayet. tataḥ samayaṃ darśayet. arghaṃ ca datvā gandhādibhiḥ saṃpūjya baliṃ dadyāt. tato visarjayed iti.
tatreme mudrāmantrā bhavanti.}%
%
}}%
\Skt{tato bāhyabaliṃ dadyād / uttarasādhakaṃ \edtext{maṇḍale}{\lemma{%
	{\rm maṇḍale\lem}
}\Dfootnote{%
	\emn\ \HT\ \sil;
	\textit{maṇḍala} \cod
}} pratiṣṭhāpya /}
	%sarvasatvārtha: em. sarvasattvārthaṁ, with HT.
	%sa\supplied{r}vatagathāga-: em. sarvatathāgata-, with HT. HT does not note that the r is supplied, having presumably been lost by damage to the leaf.
	%kuyāt·: em. kuryāt, with HT.
	%maṇḍala: em. maṇḍale, with HT (sil. em.).
\pend

\verse
\Skt{salājaṃ satilaṃ sāmbhaḥ sabhaktaṃ kusumai\edtext{ssaha /}{\lemma{%
	{\rm saha |\lem}
}\Dfootnote{
	\emn\ \HT;
	\textit{sahaḥ} \cod
}}\\
\textcolor{blue}{śaṣkulikādi\edtext{bhakṣaiḥ}{\lemma{%
	{\rm -bhakṣaiḥ\lem}
}\Cfootnote{%
	Emend \textit{-bhakṣaiḥ}?
}} \supplied{ca}}} 
\pend

\pstart\noindent
\Skt{\edtext{\supplied{a}kārādinā}{\lemma{}\Efootnote{%
	-bhakṣaiḥ[54v1]\supplied{ca a}kārādinā
}}  parijapya}
	%sahaḥ: em. saha|, with HT.
\pend

\verse
\Skt{\textcolor{blue}{\edtext{pūrvadigbhāga}{\lemma{%
	{\rm pūrvadigbhāgam\lem}
}\Dfootnote{%
	\emn\ \HT\ \sil;
	\textit{pūrvvādagbhāgam} \cod
}}mārabhya}} 
\pend

\pstart\noindent
\Skt{\edtext{trikṣepān}{\lemma{%
	{\rm trikṣepān\lem}
}\Dfootnote{%
	\emn;
	\textit{tṛkṣepāt} \cod\ \HT
}\Cfootnote{%
	to be interpreted somehow as meaning \textit{triḥ kṣipet} or \textit{triḥ kṣepayet}? or rather as \textit{trikṣepān} `three throws'? 
	Or \textit{triḥ triḥ kṣepān}?
	Note SDPT parallel.
}} gandhapuṣpa\edtext{dhū\-padīpārghāṃścādāvante}{\lemma{%
	{\rm dhūpadīpārghāṁś cādāv ante\lem}
}\Dfootnote{%
	\emn;
	\textit{dhūpadīrghārghat | ścādāv artte} \cod;
	slightly differently from \HT's \textit{-dhūpadīpārghaṁ cādāv ante}. \HT\ read \textit{arnte} rather than \textit{artte}.
}} ca dadyāt / tatra pūrvaṃ tāvanmaṇḍalakāni kārayet / \edtext{tata}{\lemma{%
	{\rm tata\lem}
}\Dfootnote{%
	\emn\ \HT;
	\textit{tataḥ}
}} āvāhaye\supplied{t / ta}taḥ samayaṃ darśayet / arghaṃ ca dattvā gandhādibhiḥ sampūjya baliṃ dadyāt / tato visarjaye\edtext{diti}{\lemma{%
	{\rm iti\lem}
}\Dfootnote{
	\emn\ \HT;
	\textit{ibhiḥ} \cod
}} / tatreme mudrāmantrā bhavanti /}
	%pūrvvādagbhāgam: em. pūrvadigbhāgam, with HT (sil. em.).
	%tṛkṣepāt·: to be interpreted somehow as meaning triḥ kṣipet or triḥ kṣepayet? or rather as trikṣepān `three throws'? Note SDPT parallel.
	%-dhūpadīrghārghat· | ścādāvartte: em. -dhūpadīpārghāṁś cādāv ante, slightly differently from HT's -dhūpadīpārghaṁ cādāv ante. HT read arnte rather than artte.
	%tataḥ: em. tata, with HT.
	%ibhiḥ: em. iti, with HT.
\pend

\pstart
\Skt{%
\edtext{ālīḍhapadena \edtext{sthitaḥ}{\lemma{%
	{\rm sthitaḥ\lem}
}\Dfootnote{%
	\emn\ \HT;
	\textit{sthitam} \cod
}} prāṅmukho vāmavajraṃ darśayet / dakṣiṇavajraṃ \edtext{kaṭideśe}{\lemma{%
	{\rm kaṭideśe\lem}
}\Dfootnote{
	\emn\ \HT;
	\textit{kaṭidese} \cod
}} \edtext{saṃdhārya}{\lemma{%
	{\rm saṁdhārya\lem}
}\Dfootnote{
	\emn\ \HT;
	\textit{sadhāya} \cod
}} dakṣiṇatarjanyaṅkuśyāvāhayet / tarjanyaṅkuśarahitā śakrasya samayamudrā /pratyālīḍhapadena sthitvā āvāhanamudrāyāstarjanīṃ prasā\textcolor{red}{rya} visarjanamudrā / \edtext{athāsya mantraḥ}{\lemma{%
	{\rm athāsya mantraḥ\lem}
}\Dfootnote{%
	\emn\ \HT;
	\textit{athāsammantram} \cod
}} / namo \edtext{vajrasya diśi}{\lemma{%
	{\rm vajrasya diśi\lem}
}\Dfootnote{%
	\emn\ \HT;
	\textit{vajrasya ca diśi}
}} diśi vajrapāṇe rakṣa rakṣa svāhā /}{\lemma{%
	{\rm ālīḍhapadena \dots\ rakṣa rakṣa svāhā\lem}
}\Bfootnote{%
%
SDPT p.\ 274:
\textit{ālīḍhapadena sthitaḥ prāṅmukho vāmavajraṃ darśayet. dakṣiṇaṃ kaṭideśe saṃdhārya tarjanyaṃkuśyāvāhayet. tarjany aṃkuśarahitā śakrasya samayamudrā.
pratyālīḍhapadena sthitvāvāhanamudrāyas tarjanīṃ prasārya visarjanamudrā.  athāsya mantraḥ. namo vajrasya diśi vajrapāṇe rakṣa svāhā};
%
KSP (chapter 6, Homa, 6-7-14-1):
\textit{tataḥ samapadaṃ sthitvā tarjanyaṇkuśarāhitā samayamudrā. 
prokṣaṇārghapādyācamanaṃ ca kṛtvā puṣpādibhiḥ saṃpūjyāṣṭālāsyādīś ca kṛtvā, oṃ namo vajrasya ca diśi 2 vajrapāṇe rakṣa 2 svahā ity anena baliṃ dadyāt. tataḥ pratyālīḍhapadena sthitvā, āvāhanamudrayā tarjanīṃ prasārya visarjayet. iti śakrasya}.
%
}} }
	%sthitam· |: em. sthitaḥ, with HT.
	%-dese: em. -deśe, with HT.
	%sadhāya: em. saṁdhārya, with HT.
	%prasāya: em. prasārya, with HT (sil. em.).
	%°athāsammantram·:  em. °athāsya mantraḥ, with HT.
	%vajrasya ca di*śi: em. vajrasya diśi, with HT.
\pend

\pstart
\Skt{%
\edtext{dakṣiṇakara\edtext{tarjanīṃ}{\lemma{%
	{\rm tarjanīṁ\lem}
}\Dfootnote{
	\cod;
	\textit{-tarjanī} \HT. But there is clearly an \textit{anusvāra} and it is required also by the context.
}} kuṇḍalākāreṇa kuñcayitvā \edtext{madhyamā}{\lemma{%
	{\rm -madhyamā-\lem}
}\Dfootnote{%
	\emn;
	\textit{-madhyam-} \cod\ \HT
}}sūcyāstṛtīyaparve dhārayedaṅguṣṭhaṃ ca karamadhye / agnerāvāhanamudrā / \edtext{āvāhanamudrāyā}{\lemma{%
	{\rm āvāhanamudrāyā\lem}
}\Dfootnote{%
	\cod;
	\HT\ reads \textit{āvāhanamudrayā} and emends \textit{āvāhanamudrāyā}.
}} \edtext{aṅguṣṭhaṃ}{\lemma{%
	{\rm aṅguṣṭhaṁ\lem}
}\Dfootnote{%
	\emn;
	\textit{aṅguṣṭha-} \cod\ \HT
}} tarjanīpārśvā\edtext{śritama}{\lemma{%
	{\rm -śritam\lem}
}\Dfootnote{%
	\emn\ \HT;
	\textit{sṛtam} \cod
}}\edtext{gneḥ}{\lemma{%
	{\rm agneḥ\lem}
}\Dfootnote{%
	\emn;
	\textit{agne} \cod;
	\HT's emendation \textit{agner} goes against the rules of \textit{sandhi}.
}} samayamudrā / asyā eva \edtext{mudrāyāḥ}{\lemma{%
	{\rm mudrāyāḥ\lem}
}\Dfootnote{
	\emn\ \HT;
	\textit{mudrayāḥ} \cod
}} karamadhye / abhimukhāvaṅguṣṭhatarjanīnakhāvekato\edtext{yojyau}{\lemma{%
	{\rm -yojyau}
}\Dfootnote{%
	\emn\ \HT;
	\textit{yojyo} \cod
}} % ekatoyojyau: this should be a compound
visarjanamudrā / \edtext{mantraḥ}{\lemma{%
	{\rm mantraḥ\lem}
}\Dfootnote{%
	\emn\ \HT;
	\textit{mantra} \cod
}} / agne \edtext{ehyākapila}{\lemma{%
	{\rm ehy ākapila\lem}
}\Dfootnote{%
	\emn;
	\textit{ehy ākapilākapila} \cod\ \HT.
	\textcolor{red}{Or emend \textit{ehy ehy ākapila}?}
}} jvala 2 daha \edtext{śikhi toli}{\lemma{%
	{\rm śikhi toli\lem}
}\Cfootnote{%
	SDPT has \textit{śikhito lola}. Emend \textit{śikhi loli}?
}\lemma{}\Efootnote{%
	daha[55r1]śikhi toli
}} virūpākṣa svāhā /}{\lemma{%
	{\rm dakṣiṇakaratarjanīṃ \dots\ virūpākṣa svāhā\lem}
}\Bfootnote{%
%
SDPT p.\ 274:
\textit{dakṣiṇakaratarjanī kuṇḍalākāreṇa kuṃcayitvā madhyamāsūcyās tṛtīyaparve dhārayed aṃguṣṭhakaṃ ca karamadhye. agner āvāhanamudrā.
āvāhanamudrāyā aṃguṣṭhaṃ tarjanīpārśvaśritam. agneḥ samayamudrā. asyā eva mudrāyāḥ karamadhye 'bhimukhāv aṃguṣṭhatarjanīnakhāv ekato yojyau visarjanamudrā.
mantraḥ. agne ehi ehi kapila jvala jvala daha śikhito lola virūpākṣa svāhā};
%
KSP (chapter 6, Homa, 6-7-14-2):
\textit{vakṣyamāneṣv apy ākarṣaṇasamayavisarjaneṣu yathākramam ālīḍhasamapratyālīḍhapadāni prokṣaṇādipūjāṃś ca pūrvavat. 
mantramudrānāṃ tu viśeṣaḥ. 
dakṣiṇatarjanīkuṇḍalākāre kuñcayitvā madhyamasūcyās tṛtīyaparva dhārayet. 
aṇguṣṭhaṃ ca karamadhye sthapayet. 
ity āvāhanam.
asyā mudrāyā aṇguṣṭhatarjanīpārśvāśritaḥ samayaḥ.
oṃ ehy ehi kapila jvala 2 ha 2 śikhi toli virupākṣa svāhā.
asyā mudrāyās tarjanyaṇguṣthaṃ nakhāś caikato yojya visarjayet.
ity agneḥ.}
}} }
	%-tarjjanīṁ: -tarjanī HT. But there is clearly an anusvāra and it is required also by the context.
	%-sṛtam: em. -śritam, with HT.
	%agne: em. agneḥ. HT's em. agner goes against the rules of sandhi.
	%mudrayāḥ: em. mudrāyāḥ, with HT.
	%yojyo:  em. yojyau, with HT.
	%śikhi toli: SDPT has śikhito lola. Emend śikhi loli?
	%mantra: em. mantraḥ, with HT.
\pend

\pstart
\Skt{%
\edtext{\edtext{yāmyabhimukho}{\lemma{%
	{\rm yāmyabhimukho\lem}
}\Dfootnote{%
	\emn;
	\textit{yāmyābhimukho} \cod;
	\textit{yāmyām abhimukho} SDPT
}} \edtext{yogī abhimukhau}{\lemma{%
	{\rm yogī abhimukhau\lem}
}\Dfootnote{%
	\cod;
	\textit{yogy abhimukhau} \emn\ \HT
}} karau \edtext{kṛtvā abhyantara}{\lemma{%
	{\rm kṛtvā abhyantara-\lem}
}\Dfootnote{%
	\cod;
	\textit{kṛtvābhyantara} \emn\ \HT
}}\edtext{vajrabandhe}{\lemma{%
	{\rm -vajrabandhe\lem}
}\Dfootnote{%
	\cod;
	\textit{-vajrabandhaṃ} \cod\ \HT
}} madhye'ṅguṣṭhayugalaṃ bahi\edtext{ranāmikā}{\lemma{%
	{\rm anāmikā-\lem}
}\Dfootnote{%
	\emn\ \HT\ \sil;
	\textit{aṇāmikā} \cod
}}dvayāsaktasūcīṃ puna\edtext{rabhyantaraṃ}{\lemma{%
	{\rm abhyantaraṁ\lem}
}\Dfootnote{%
	\emn\ \HT\ (misreading abhyāntara);
	\textit{a\unclear{bhya}ntara} \cod
}} dhārayedyamasyāvāhanamudrā / anāmikāṃ \edtext{puna}{\lemma{%
	{\rm punar\lem}
}\Dfootnote{%
	\emn\ \HT;
	\textit{punā} \cod
}}rbāhyataḥ \edtext{sūcīṃ}{\lemma{%
	{\rm sūcīṁ\lem}
}\Dfootnote{%
	\emn\ \HT;
	\textit{sūcī} \cod
}} tathaiva kṛtvā mudrāṃ \edtext{hṛdaye dhāraye}{\lemma{%
	{\rm hṛdaye dhārayet\lem}
}\Dfootnote{%
	\emn;
	\textit{hṛdaye dvārayet} \cod;
	\textit{hṛdayeddhārayet} \HT, with the same emendation.  Or should we just accept \HT's reading?
}}tsamayamudrā / anayaivā\edtext{nāmikā}{\lemma{%
	{\rm -nāmikā-\lem}
}\Dfootnote{%
	\emn\ \HT\ \sil;
	\textit{-nāsikā-} \cod
}}sūcyā visarjanaṃ bhavati / \edtext{mantraḥ}{\lemma{%
	{\rm mantraḥ\lem}
}\Dfootnote{%
	\emn\ \HT;
	\textit{mantra} \cod
%	\HT\ emends \textit{mantraḥ}.
}} / yamāya svāhā /}{\lemma{%
	{\rm yāmyabhimukho \dots\ yamāya svāhā\lem}
}\Bfootnote{%
%
SDPT p.\ 274:
\textit{yāmyām abhimukho yogī abhimukhakarau kṛtvābhyantaravajrabandhe madhye 'ṃguṣṭhayugalaṃ bahir anāmikādvayāsaktasūcī punar abhyantare dhārayet. yamasyāvāhanamudrā. 
anāmikām punar bāhyataḥ sūcīṃ tathaiva kṛtvā hṛdaye dhārayet. samayamudrā. anayaivānāmikāsūcyā visarjanaṃ bhavati.
asya mantraḥ. yamāya svāhā};
%
KSP (chapter 6, Homa, 6-7-14-3):
\textit{anāmikābhyāṃ sūcīkṛtvā śeṣair abhimukhābhyantaravajrabandhair punar anāmike pṛthak kṛtvā kuñcyākarṣayet. āvāhanam.
asyā mudrāyā ākuñcanarahitā(ṃ) hṛdi dhārayet. samayaḥ.
oṃ yamāya svāhā.
samayamudrāparibhramaṇād visarjanam. 
iti yamasya.}%
%
}} }
	%yāmyābhi-: em. yāmyabhi-. SDPT yāmyām abhi-.
	%yogī °abhimukhau: HT emends yogy abhimukhau.
	%kṛtvā °abhyantara-: HT emends kṛtvābhyantara
	%aṇāmikā-: em. anāmikā-, with HT (sil. em.).
	%a\unclear{bhya}ntara: em. abhyantaraṁ, with HT (misreading abhyāntara).
	%punā: em. punar, with HT.
	%sūcī: em. sūcīṁ, with HT.
	%hṛdaye dvārayet: em. hṛdaye dhārayet; hṛdayeddhārayet HT, with the same emendation.  Or should we just accept HT's reading?
	%-nāsikā-: em. -nāmikā-, with HT (sil. em.).
	%mantra: HT emends mantraḥ.
\pend

\pstart
\Skt{%
\edtext{\edtext{nairṛtya}{\lemma{%
	{\rm nairṛtya-\lem}
}\Dfootnote{%
	\emn\ \HT\ \sil;
	\textit{naiṛtya} \cod
}}bhimukhassamapādasthito dakṣiṇakara\edtext{muṣṭiṃ}{\lemma{%
	{\rm muṣṭiṁ\lem}
}\Dfootnote{%
	\emn\ \HT\ \sil;
	\textit{muṣṭhiṅ} \cod
}} kṛtvā madhyamātarjanyau kuñcayet / \edtext{khaḍgākāreṇa}{\lemma{%
	{\rm khaḍgākāreṇa\lem}
}\Dfootnote{%
	\emn;
	\textit{khaḍgakāreṇa} \cod;
	\HT\ misreads \textit{khadgakāreṇa}, and emends \textit{khadgākāreṇa} (with \textit{d}, not \textit{ḍ}).
}} saṃsthāpya vāmakaraṃ kaṭi\edtext{pradeśe}{\lemma{%
	{\rm -pradeśe\lem}
}\Dfootnote{%
	\emn\ \HT\ \sil;
	\textit{pradese} \cod
}} dhāraye\edtext{dvāmatarjanīṃ}{\lemma{%
	{\rm vāmatarjanīṁ\lem}
}\Dfootnote{%
	\emn\ \HT\ \sil;
	\textit{vāmantarjjanīṁ} \cod
}} kuñcayitvā \edtext{nirṛte}{\lemma{%
	{\rm nirṛter\lem}
}\Dfootnote{%
	\emn;
	\textit{nisater} \cod;
	\HT\ reads \textit{niṛter} and emends \textit{nairṛter}.
}}rāvāhanamudrā / \edtext{asyā}{\lemma{%
	{\rm asyā\lem}
}\Cfootnote{%
	emend \textit{anayā}? But see \textit{asyā} again in next §. \HT\ accepts \textit{asyā}.
}} eva \edtext{mudrāyā}{\lemma{%
	{\rm mudrāyā\lem}
}\Dfootnote{%
	\emn;
	\textit{mudrayā} \cod;
	\HT\ misreads \textit{mudrāyā} which he emends to \textit{mudrayā}.
}} vāmakaraṃ \edtext{kaṭideśe}{\lemma{%
	{\rm kaṭideśe\lem}
}\Dfootnote{%
	\emn\ \HT;
	\textit{kaṭidese} \cod
}}'vasthitaṃ \edtext{khaḍgamudrā}{\lemma{%
	{\rm khaḍgamudrā\lem}
}\Dfootnote{%
	\emn;
	\textit{khadgamudrā} \cod;
	\HT\ just reads \textit{khaḍgamudrā}. Cf.\ preceding §. It seem \HT\ had the opposite understanding from ours of which \textit{akṣara}s represent \textit{ḍ} and \textit{d}.
}} / \edtext{nirṛteḥ}{\lemma{%
	{\rm nirṛteḥ\lem}
}\Dfootnote{%
	\emn;
	\textit{nirsateḥ} \cod;
	\HT\ reads \textit{nirṛteḥ} and emend \textit{nairṛteḥ}.
}\Cfootnote{%
	What is the proper reading of the \textit{akṣara} we tentatively read \textit{rsa}? Can it be an alternative form of \textit{ṛ}?
}} samayamudrā / \edtext{āvāhana}{\lemma{%
	{\rm āvāhana-\lem} 
}\Dfootnote{%
	\emn\ \HT;
	\textit{avāhana-} \cod
}}mudrāyāstarjanīṃ prasā\textcolor{red}{rya} visarjanamudrā / \edtext{mantraḥ}{\lemma{%
	{\rm mantraḥ\lem}
}\Dfootnote{
	\emn\ \HT;
	\textit{mantra} \cod
%	\HT\ emends \textit{mantraḥ}.
}} / sarvabhūtabhayaṅkara kuru 2 svāhā /}{\lemma{%
	{\rm nairṛtyabhimukhas \dots\ svāhā\lem}
}\Bfootnote{%
%
SDPT p.\ 274:
\textit{nairṛtyām abhimukhaḥ samapadasthito dakṣiṇakaramuṣṭim kṛtvā, madhyamātarjanyāv evam ākuṃcya dhārayet. khaḍgākāreṇa saṃsthāpya vāmakaraṃ kaṭideśe dhārayet. vāmatarjanīṃ kuṃcayitvā nairṛtyer āvāhanamudrā.
asyā eva mudrāyā vāmakaraṃ kaṭideśe 'vasthitaṃ khaḍgamudrā. nairṛteḥ samayamudrā. āvāhanamudrāyās tarjanīm prasārya visarjanamudrā. mantro 'sya. sarvabhūtabhayaṃkaram kuru kuru svāhā};
%
KSP (chapter 6, Homa, 6-7-14-4):
\textit{vāmamuṣṭiṃ kaṭipradeśe sthāpya tarjanīṃ kuñcayitvā, dakṣiṇamuṣṭimadhyamātarjanībhyāṃ khaḍgākarṣaṇād āvāhanam.
asyā eva vāmatarjanyākuṃcanarahitā samayaḥ.
oṃ sarvabhūtabhayaṇkara kuru 2 svāhā.
asyā eva vāmatarjanīprasāraṇād visarjanam.
iti nairṛtyasya.}%
%
}}}
	%cf. SDPS p. 274 for  asyā eva ...
	%nai°ṛtya-: em. nairṛtya-, with HT (sil. em.).
	%-muṣṭhiṅ: em. -muṣṭiṁ, with HT (sil. em.).
	%khaḍgakāreṇa: em. khaḍgākāreṇa; HT misreads khadgakāreṇa, and end emends khadgākāreṇa (with d, not ḍ).
	%-pradese: em. -pradeśe, with HT (sil. em.).
	%vāmantarjjanīṁ: em. vāmatarjanīṁ, with HT (sil. em.).
	%nisate§○r: em. nirṛter; HT reads niṛter and emends nairṛter.
	%°asyā: em. °anayā? But see °asyā again in next §. HT accepts °asyā.
	%mudrayā: HT misreads mudrāyā which he emends to mudrayā.
	%kaṭidese: em. kaṭideśe, with HT.
	%khadgāmudrā: em. khaḍgamudrā; HT just reads khaḍgamudrā. Cf. preceding §. It seem HT had the opposite understanding from ours of which akṣaras represent ḍ and d.
	%nirsateḥ: em. nirṛteḥ; HT reads nirṛteḥ and emend nairṛteḥ.
		%%What is the proper reading of the akṣara we tentatively read rsa? Can it be an alternative form of °ṛ?
	%°avāhana-: em. °āvāhana-, with HT.
	%prasāya: em. prasārya, with HT (sil. em.).
	%mantra: HT emends mantraḥ.
\pend

\pstart
\Skt{%
\edtext{\edtext{vāruṇyāṃ}{\lemma{%
	{\rm vāruṇyāṁ\lem}
}\Dfootnote{%
	\emn\ \HT;
	\textit{vāraṇyan} \cod
}} diśi samapādāvasthito dakṣiṇakaratarjanyaṅguṣṭhāvekato yojayet / vāmamuṣṭiṃ hṛdaye saṃdhā\textcolor{red}{rya} vāmatarjanya\edtext{ṅkuśenāvāhayet}{\lemma{%
	{\rm aṅkuśenāvāhayet\lem}
}\Dfootnote{%
	\emn;
	\textit{aṁkusenāvāyet} \cod;
	\HT reads \textit{-aṅkśenāvāyet} and emends to \textit{-aṅkśenāvāhayet} (missing \textit{-u-} no doubt an unintentional error).
}} / varuṇasyāvāhanamudrā / asyā eva vāmatarjanīṃ muṣṭiyogato dhārayet pāśamudrā / \edtext{varuṇasya}{\lemma{}\Efootnote{%
	pāśamudrā | [55v1] varuṇasya
}} samayamudrā / \edtext{āvāhanamudrāyā}{\lemma{%
	{\rm āvāhanamudrayās\lem}
}\Dfootnote{%
	\emn\ \HT;
	\textit{āvāhanamudrayās} \cod
	%emend \textit{āvāhanamudrayā}? or emend \textit{āvāhanamudrāyās}, with \HT?
}}starjanīṃ prasā\textcolor{red}{rya} visarjanamudrā / \edtext{\textcolor{blue}{mantraḥ}}{\lemma{%
	{\rm mantraḥ\lem}
}\Dfootnote{%
	\emn\ \HT;
	\textit{mantra} \cod
	%\HT\ emends \textit{mantraḥ}.
}} / tṛ tṛ puṭa tṛ tṛ śikhi toli virūpākṣa svāhā //}{\lemma{%
	{\rm vāruṇyāṃ \dots\ virūpākṣa svāhā\lem}
}\Bfootnote{%
%
SDPT p.\ 276: 
\textit{vāruṇyāṃ diśi samapadasthito dakṣiṇakaratarjanyaṃguṣṭhāv ekato yojayed vāmamuṣṭiṃ hṛdi saṃdhārya vāmatarjanyaṃkuśenāvāhayet. varuṇasyāvāhanamudrā. asyā eva vāmatarjanīmuṣṭiyogato dhārayet. pāśamudrā. āvāhanamudrāyās tarjanīṃ prasārya visarjanamudrā. 
mantro 'sya. tṛtṛpuṭa tṛtṛśikhitoli virūpākṣa svāhā};
%
KSP (chapter 6, Homa, 6-7-14-5):
\textit{dakṣiṇamuṣṭau tarjanyaṇguṣṭhāv ekato yojyau. 
vāmamuṣṭiṃ hṛdi kṛtvā tarjanyaṇkuśenāvāhayet. 
saivāṇkuśarahitā samayaḥ.
oṃ bhṛ 2 puṭa bhṛ 2 śikhi toli virupākṣa svāhā.
vāmatarjanīṃ prasārya visarjanam.
varuṇasya.}%
%
}} }
	%vāraṇyan: em. vāruṇyāṁ, with HT.
	%saṁdhāya: em. saṁdhārya, with HT (sil. em.).
	%-aṁkusenāvāyet·: em. -aṅkuśenāvāhayet ; HT reads -aṅkśenāvāyet and emends to -aṅkśenāvāhayet (missing -u- no doubt an unintentional error).
	%āvāhanamudrayās: em. āvāhanamudrayā? or emend āvāhanamudrāyās, with HT?
	%prasāya: em. prasārya, with HT (sil. em.).
	%mantra: HT emends mantraḥ.
\pend

\pstart
\Skt{%
\edtext{vāyavyāṃ \edtext{diśya}{\lemma{%
	{\rm diśy\lem}
}\Dfootnote{%
	\emn\ \HT;
	\textit{diśv} \cod
}}bhimukhaṃ sthitvā \edtext{vāmamadhyamāsūcī\textcolor{blue}{yukta}tarjanīṃ}{\lemma{%
	{\rm vāmamadhyamāsūcīyuktatarjanīṁ\lem}
}\Dfootnote{%
	\emn;
	\textit{vāmamadhyamāṁ sūciṁ muktā tarjanī} \cod;
	\textit{vāmamadhyamāṃ sūcīṁ muktā tarjanī} \HT
}} kuṇḍalākāreṇa tṛtīyaparve \edtext{\textcolor{red}{saṃdhāryābhimukhaṃ}}{\lemma{%
	{\rm saṃdhāryābhimukhaṁ\lem}
}\Dfootnote{%
	\emn;
	\textit{saṃdadhyād abhimukhaṃ} \cod\ \HT
}} prasārayet / dakṣiṇakaraṃ \edtext{kaṭideśe}{\lemma{%
	{\rm kaṭideśe\lem}
}\Dfootnote{
	\emn\ \HT\ \sil;
	\textit{kaṭidese} \cod
}} saṃsthāpyākuñcitāṅguṣṭhena vāyorāvāhanamudrā / asyā \edtext{\textcolor{blue}{evāṅguṣṭhaṃ}}{\lemma{%
	{\rm evāṅguṣṭhaṁ\lem}
}\Dfootnote{%
	\emn;
	\textit{evāṅguṣṭa} \cod\
	\textcolor{red}{(CHECK \HT)}
	\textcolor{blue}{Alternatively, this should be emended to \textit{asyā evāṅguṣṭhaṃ dhāryaṃ} (eye skip from \textit{dhā} to \textit{vā} of \textit{vāyor}?)}
}} pūrvavadvāyoḥ samayamudrā / \edtext{āvāhanamudrāyā}{\lemma{%
	{\rm āvāhanamudrāyā\lem}
}\Dfootnote{%
	\emn\ \HT;
	\textit{āvāhanamudrayā} \cod
}} \edtext{aṅguṣṭhaṃ}{\lemma{%
	{\rm aṅguṣṭhaṁ\lem}
}\Dfootnote{%
	\emn\ \HT\ \sil;
	\textit{aṁguṣṭaṁ} \cod\
}} prasā\textcolor{red}{rya} visarjanamudrā / \edtext{mantraḥ}{\lemma{%
	{\rm mantraḥ\lem}
}\Dfootnote{%
	\emn\ \HT;
	\textit{mantra} \cod
}} / {\om} \edtext{śvasa khākha khakhaḥ}{\lemma{
	{\rm śvasa khākha khakhaḥ\lem}
}\Dfootnote{
	\emn;
	\textit{\unclear{sva}sa khākha khakhaḥ} \cod;
	\textit{śvasa khākha khākha} \HT, emending to \textit{śvasa khākha khākha}.
	Cf.\ \textit{Dhātupāṭha} 2.64: \textit{śvasa prāṇane |}
}} svāhā~/}{\lemma{%
	{\rm vāyavyāṃ \dots\ khakhaḥ svāhā\lem}
}\Bfootnote{%
%
SDPT p.\ 276:
\textit{vāyavyāṃ diśy abhimukhaṃ sthitvā vāmakaramadhyamā sūcīmuktā tarjanī kuṇḍalākāreṇa tṛtīyaparve saṃdadhyābhimukhaṃ prasārayet. 
dakṣiṇakaraṃ kaṭideśe saṃsthāpya kuṃcitāṃguṣṭhena vāyor āvāhanamudrā.
asyā \textcolor{red}{cvāṃguṣṭhaṃ} pūrvavat saṃdhārya vāyor samayamudrā.
āvāhanamudrāyā aṃguṣṭhaṃ prasārya visarjanamudrā. 
mantraḥ. oṃ śvasa khākhe khukhaḥ svāhā};
%
KSP (chapter 6, Homa, 6-7-14-6):
\textit{vāmamuṣṭhāv ucchritamadhyamātṛtīyaparva tarjanīkuṇḍalākāreṇa dhārayet. 
dakṣiṇamuṣṭiṃ kaṭipradeśe saṃsthāpy ākuñcitāṇguṣṭhenāvāhayet. 
saiva dakṣiṇāṇguṣṭhapradeśāt samayaḥ.
oṃ susakhakhakhūkha svahā.
atraivāṇguṣṭhaprasāraṇād visarjanam.
vāyoḥ}.%
%
}} }
	%diśva-: em. diśy a-; disva- HT, with the same emendation.
	%sūciṁ: em. sūcīṁ, with HT.
	%-dese: em. -deśe, with HT (sil. em.).
	%-guṣṭa-: em. -guṣṭha-, with HT (sil. em.).
	%-mudrayā: HT emends -mudrāyā.
	%°aṁguṣṭaṁ: em. °aṅguṣṭhaṁ.
	%prasāya: em. prasārya, with HT (sil. em.).
	%mantra: HT emends mantraḥ.
	%\unclear{sva}sa khākha khakhaḥ:  śvasa khākha khākha HT, emending to śvasa khākha khākha.
\pend

\pstart
\Skt{%
\edtext{kaube\textcolor{red}{rya}bhimukhasthitaḥ karadvayamabhimukhaṃ kṛtvābhyantaravajra\edtext{bandhe}{\lemma{%
	{\rm -bandhe\lem}
}\Dfootnote{%
	\emn;
	\textit{-vandhaṁ} \cod\ \HT
}} kaniṣṭhādvayasūcīṃ tasyāḥ \edtext{pṛṣṭhato}{\lemma{%
	{\rm pṛṣṭhato\lem}
}\Dfootnote{%
	\emn\ \HT\ \sil;
	\textit{pṛṣṭato} \cod
}}\edtext{'nāmikāyugalaṃ}{\lemma{%
	{\rm 'nāmikāyugalaṁ\lem}
}\Cfootnote{%
	% nāmikāyuga[55v5]laṁ.%
	It is unclear at the beginning of 55v5 whether the photo shows the two first \textit{akṣara}s in damaged form, or whether we are seeing remnants of some other folio. Anyhow, it is clear which two \textit{akṣara}s must be read here.
}} pṛthakpṛthak saṃdhā\textcolor{red}{rya} \edtext{madhyamāsūcī}{\lemma{%
	{\rm madhyamāsūcī\lem}
}\Dfootnote{%
	\emn;
	\textit{madhyamāṃ sūcīṃ} \cod\ \HT
}} vajrākāreṇa nāmayetkuberāvāhanamudrā /
asyā eva \edtext{mudrāyā}{\lemma{%
	{\rm mudrāyā\lem}
}\Dfootnote{%
	\emn\ \HT;
	\textit{mudrayā} \cod;
}} \edtext{madhyamādvaya}{\lemma{%
	{\rm madhyamādvayam\lem}
}\Dfootnote{%
	\emn;
	\textit{madhyādvayam} \cod\ \HT
}}mabhyantaravajrabandhayogato nyasyetkuberasya samayamudrā /
āvāhana\edtext{mudrāyā}{\lemma{%
	{\rm -mudrāyā\lem}
}\Dfootnote{%
	\emn\ \HT;
	\textit{-mudrayā} \cod;
}} \edtext{madhyamādvayaṃ}{\lemma{%
	{\rm madhyamādvayaṁ\lem}
}\Dfootnote{%
	\emn;
	\textit{madhyādvayaṁ} \cod\ \HT
}} prasā\textcolor{red}{rya} visarjanamudrā / mantraḥ / {\om} kuberāya svāhā /}{\lemma{%
	{\rm kauberyabhimukhasthitaḥ \dots\ kuberāya svāhā\lem}
}\Bfootnote{%
%
SDPT p.\ 276:
\textit{kauberyabhimukhaṃ sthitaḥ karadvayam abhimukhaṃ kṛtvābhyantaravajrabandhaṃ kaniṣṭhādvayasūcīṃ tasyāḥ pṛṣṭhato 'nāmikāyugalaṃ pṛthak saṃdhārya madhyamāsūcīṃ vajrākāreṇa nāmayet. kuberāvāhanamudrā. 
asyā eva mudrāyā madhyamādvayam abhyantaravajrabandhayogato nyasya kuberasya samayamudrā.
āvāhanamudrāyā madhyamādvayaṃ prasārya visarjanamudrā. 
mantro 'sya. oṃ kuberāya svāhā};
%
KSP (chapter 6, Homa, 6-7-14-7):
\textit{abhyantaravajrabandhe kaniṣṭhādvayaṃ sūcīkṛtvā tatpṛṣṭhato 'nāmikāyugalaṃ pṛthak pṛthak saṃdhārya madhyamādvayaṃ sūcīkṛtvā \textcolor{red}{vajrākareṇānayet}. āvāhanam.
asyā eva madhyamādvayaṃ praveśya samayaḥ.
oṃ kuberāya svāhā.
āvāhanamudrāyā madhyamādvayaṃ prasārya visarjanam.
kuberasya}.%
}} }
	%kauveya-: em. kauverya-, with HT (sil. em.).
	%pṛṣṭato: em. pṛṣṭhato, with HT (sil. em.).
	%it's unclear at the beginning of 55v5 whether the photo shows the two first akṣaras in damaged form, or whether we are seeing remnants of some other folio. Anyhow, it is clear which two akṣaras must be read here.
	%saṁdhāya: em. saṁdhārya, with HT (sil. em.).
	%mudrayā: HT emends mudrāyā.
	%-mudrayā: HT emends -mudrāyā.
	%prasāya: em. prasārya, with HT (sil. em.).
\pend

\pstart
\Skt{%
\edtext{\edtext{aiśānyāṃ}{\lemma{%
	{\rm aiśānyāṁ\lem}
}\Dfootnote{%
	\emn\ \HT;
	\textit{aiṣānyan} \cod
}} diśi tadabhimukhā\edtext{vasthitaḥ}{\lemma{%
	{\rm -vasthitaḥ\lem}
}\Dfootnote{
	\emn\ \HT\ \sil;
	\textit{-vasthitāḥ} \cod
}} \edtext{karadvayamekato}{\lemma{}\Efootnote{
	karadvaya[56r1]m ekato
}} \edtext{yojyāñjaliṃ}{\lemma{%
	{\rm yojyāñjaliṁ\lem}
}\Dfootnote{%
	\emn\ \HT\ \sil;
	\textit{yojyāṁjali} \cod
}} kṛtvā \edtext{kanīyasyanāmikai}{\lemma{%
	{\rm kanīyasyanāmikai\lem}
}\Dfootnote{
	\emn;
	\textit{kanyasānāmikais} \cod\ \HT
}}stalavajrabandhaṃ \edtext{kṛtvāṅguṣṭha}{\lemma{%
	{\rm kṛtvāṅguṣṭha-\lem}
}\Dfootnote{%
	\emn\ \HT\ \sil;
	\textit{kṛtvāṅguṣṭa} \cod
}}yugalaṃ \edtext{madhyamāśritaṃ}{\lemma{%
	{\rm madhyamāśritaṁ\lem}
}\Dfootnote{%
	\emn;
	\textit{madhyāśritaṃ} \emn\ \HT\ \sil;
	\textit{madhyāśṛtaṁ} \cod
}} \edtext{madhyamāsūcyo}{\lemma{%
	{\rm madhyamāsūcyo\lem}
}\Dfootnote{%
	\emn;
	\textit{madhyamāsūcyo} \emn\ \HT\ \sil;
	\textit{madyamacūcyā} \cod
}} bahirvajrākāreṇa tarjanīdvayaṃ nyasya ta\edtext{devākuñcyopari}{\lemma{%
	{\rm evākuñcyopari\lem}
}\Dfootnote{%
	\emn;
	\textit{evākucyopari} \cod;
	\textit{evākuṁcyopari} \emn\ \HT\ \sil
}} parasparanakhāsaktaṃ ku\textcolor{red}{ryā}dīśānāvāhanamudrā / \edtext{asyā}{\lemma{%
	{\rm asyā\lem}
}\Dfootnote{%
	\cod;
	\textit{asyāṁ} \HT; 
	We think the ostensible \textit{anusvāra} is a hole in the folio.
}} eva \edtext{tarjanyau}{\lemma{%
	{\rm tarjanyau\lem}
}\Dfootnote{%
	\emn\ \HT;
	\textit{tarjanyo} \cod
}} \edtext{pūrvavadvajrākāreṇa}{\lemma{%
	{\rm pūrvavad vajrākāreṇa\lem}
}\Dfootnote{%
	\emn\ as in SDPT; see note \HT;
	\textit{pūrvvavajrākāreṇa} \cod
}} dhārayedīśānasya samayamudrā | āvāhanamudrāyāstarjanyau \edtext{prasārya}{\lemma{%
	{\rm prasārya\lem}
}\Dfootnote{%
	\emn\ \HT\ \sil;
	\textit{praśā\textcolor{red}{rya}} \cod
}} visarjanamudrā / mantraḥ / {\om} \edtext{jru{\cb} jru{\cb}}{\lemma{%
	{\rm jruṁ jruṁ\lem}
}\Dfootnote{%
	\cod;
	\textit{juṁ juṁ} \HT
}} śiva svāhā /}{\lemma{%
	{\rm aiśānyāṁ \dots\ śiva svāhā\lem}
}\Bfootnote{%
%
SDPT pp.\ 276–278:
\textit{aiśānyāṃ diśy abhimukhaṃ sthitvā karāv ekato yojyāṃjaliṃ kṛtvā kanyasānāmikātalavajrabandhaṃ kṛtvāṃguṣṭhayugalaṃ madhyamāśritaṃ madhyamāsūcyo bahir vajrākāreṇa tarjanīdvayaṃ nyasya tad evākuṃcyopari
parasparanakhāsaktaṃ kuryāt. īśānāvāhanamudrā. 
asyā eva tarjanyau pūrvavad vajrākāreṇa dhāraye. īśānasamayamudrā. 
āvāhanamudrāyās tarjanyau prasārya visarjanamudrā.
mantraḥ. oṃ juṃ juṃ śiva svāhā};
%
KSP (chapter 6, Homa, 6-7-14-8):
\textit{kanyasyanāmikābhyāṃ talavajrabandhaṃ kṛtvā aguṣṭhayugalaṃ madhyamāśritamadhyamasūcyāḥ pṛṣṭhatas tarjanīdvayaṃ vajrākāreṇa saṃdhāryākuñcya parasparaṃ nakhāśaktaṃ kuryāt.  āvahanam.
pūrvavad vajrākāreṇa dhāraṇāt samayaḥ.
oṃ hūṃ śiva svāhā.
saiva tarjanīprasāraṇād visarjanam.
itīśānasya.}%
%
}} }
	%°aiṣānyan: em. °aiśānyāṁ, with HT.
	%yojyāṁjali: em. yojyāñjaliṁ, with HT (sil. em.).
	%kṛtvāṅguṣṭa-: em. kṛtvāṅguṣṭha-, with HT (sil. em.).
	%madhyāśṛtaṁ: em. madhyāśritaṁ, with HT (sil. em.).
	%-cūcyā: em. -sūcyo, with HT (sil. em.).
	%evākucyopari: em. evākuñcyopari; evākuṁcyopari HT (sil. em.).
	%kuyād: em. kuryād, with HT (sil. em.).
	%°asyā: °asyāṁ HT. We think the ostensible anusvāra is a hole in the folio.
	%tarjanyo: em. tarjanyau, with HT.
	%pūrvvavajrākāreṇa: em. pūrvavad vajrākāreṇa, as in SDPT; see note HT.
	%praśāya: em. prasārya, with HT (sil. em.).
	%jru;ṁ jru;ṁ: juṁ juṁ HT.
\pend

\pstart
\Skt{%
\edtext{pratyālīḍhasthānastho'ñjalyākāreṇa \edtext{hastau}{\lemma{%
	{\rm hastau\lem}
}\Dfootnote{%
	\emn\ \HT;
	\textit{hasto} \cod
}} \edtext{saṃdhā\textcolor{red}{ryo}rdhvaṃ}{\lemma{%
	{\rm saṁdhāryordhvaṁ\lem}
}\Dfootnote{%
	\emn\ \HT;
	\textit{saṁdhāyorddha} \cod
}} dṛṣṭvā tarjanīdvayāṅkuśyā brahmādīnāmāvāhanam / asyā eva \edtext{tarjanīdvayaṃ}{\lemma{
	{\rm tarjanīdvayaṁ\lem}
}\Dfootnote{%
	\emn\ \HT;
	\textit{tarjanīdvaya} \cod
}} pūrvavatsaṃsthāpya samayamudrā / āvāhanamudrāyāstarjanīdvayaṃ prasā\textcolor{red}{rya} visarjanamudrā / mantrā bhavanti / \edtext{ūrdhva}{\lemma{%
	{\rm ūrdhva-\lem}
}\Dfootnote{%
	\emn\ \HT;
	\textit{ūrddha} \cod
}}brahmaṇe svāhā / sū\textcolor{red}{ryā}ya grahādhipataye svāhā / candrāya nakṣatrādhipataye svāhā /}{\lemma{%
	{\rm pratyālīḍhasthānastho \dots\ nakṣatrādhipataye svāhā\lem}
}\Bfootnote{%
%
SDPT p.\ 278:
\textit{pratyālīḍhasthānastho 'ṃjalyākāreṇa hastau saṃdhārya, ūrdhvaṃ dṛṣṭvā tarjanīdvayāṃkuśyā brahmādīnāṃ āvāhanamudrā. 
asyā eva tarjanīdvayaṃ pūrvavat saṃsthāpya samayamudrā. 
āvāhanamudrāyās tarjanīdvayaṃ prasārya visarjanamudrā. 
mantrās teṣām. oṃ ūrdhvaṃ brahmaṇe svāhā. oṃ sūryāya grahādhipataye svāhā. oṃ candrāya nakṣatrādhipataye svāhā};
%
KSP (chapter 6, Homa, 6-7-14-9):
\textit{ūrdhvaṃ dṛṣṭvā saṃpuṭāñjaliṃ kṛtvā tarjanīdvayam ākuñcyāvāhayet.
asyā eva tarjanīdvayaṃ pūrvavat saṃsthāpya samayaḥ.
oṃ ūrdhvabrahmaṇe svāhā.
oṃ sūryāya grahādhipataye svāhā.
oṃ candrāya nakṣatrādhipataye svāha.
āvāhanamudrāyās tarjanīdvayaṃ prasārya visarjanam.
brahmādīnām.}
}} }
	%hasto: em. hastau, with HT.
	%saṁdhāyorddha: em.  saṁdhāryordhvaṁ, with HT (who ignores absent repha).
	%tarjanīdvaya: em. tarjanīdvaya, with HT.
	%prasāya: em. prasārya, with HT (sil. em.).
	%°ūrddha-: em. °ūrdhva-, with HT.
	%-vrahmaṇe: em. -brahmaṇe; HT misreads -brahmane.
	%sūyāya: em. sūryāya, with HT (sil. em.).
\pend

\pstart
\Skt{%
\edtext{\edtext{samapadaṃ}{\lemma{%
	{\rm samapadaṁ\lem}
}\Dfootnote{%
	\cod;
	\textit{samapādaṁ} \emn\ \HT
}} sthānamāsthāya hastadvayamekato yojya \edtext{viralānyonyāṅgulyagraṃ}{\lemma{%
	{\rm viralānyonyāṅgulyagraṁ}
}\Dfootnote{
	\emn\ \HT;
	\textit{viralānyonyaṅgulyagra} \cod
}} \edtext{saṃyojyāṅguṣṭhau}{\lemma{%
	{\rm saṁyojyāṅguṣṭhau\lem}
}\Dfootnote{%
	\emn\ \HT\ \sil;
	\textit{saṁyojyāṁṅguṣṭau} \cod
}} \edtext{vartulākāreṇādho dṛṣṭiṃ}{\lemma{%
	{\rm vartulākāreṇādho dṛṣṭiṁ\lem}
}\Dfootnote{
	\emn\ \HT;
	\textit{catu\unclear{lā}kāreṇādho dṛṣṭiṁ} \cod;
	\HT\ reads \textit{cartulā-}.
}} kṛtvā pṛthivyādīnāṃ \edtext{tarjanyaṅkuśābhyā}{\lemma{%
	{\rm tarjanyaṅkuśābhyām\lem}
}\Dfootnote{%
	\emn\ \HT\ \sil;
	\textit{tarjanyaṁkuśābhyā} \cod
}}māvāhanam / tarjanyau pūrvavadvyavasthāpya samayamudrā / āvāhanamudrāyāḥ  \edtext{prasāritatarjanībhyāṃ}{\lemma{%
	{\rm prasāritatarjanībhyāṁ}
}\Dfootnote{%
	\emn\ \HT\ \sil;
	\textit{prasāritarjanībhyāṁ} \cod\ac;
	\textit{prasāritavatarjanībhyāṁ} \cod\pc
}} %\newfolio{56v1} 
\edtext{visarjana\supplied{m}}{\lemma{}\Efootnote{%
	tarjanībhyāṁ [56v1]visarjana\supplied{m}
}} / mantrāḥ / \edtext{adhaḥpṛthivyai}{\lemma{%
	{\rm adhaḥpṛthivyai\lem}
}\Dfootnote{%
	\emn;
	\textit{adhapṛthivyai} \cod;
	\HT\ emends \textit{adhaspṛthivyai}.
}} svāhā / asurebhyaḥ svāhā / nāgebhyaḥ \edtext{svāheti}{\lemma{%
	{\rm svāheti\lem}
}\Dfootnote{
	\emn;
	\textit{svāhoti} \cod;
	\HT\ edits \textit{svāhā || iti}, with silent emendation.
}}}{\lemma{%
	{\rm samapādaṁ \dots\ nāgebhyaḥ svāheti\lem}
}\Bfootnote{%
%
SDPT p.\ 278:
\textit{samapadaṃ sthānam āsthāya hastadvayam ekato yojyāvicalānyony āṃgulyagrā samyojyāṃguṣṭhau vartulākāreṇādhodṛṣṭiṃ kṛtvā pṛthivyādīnāṃ tarjanyaṃkuśābhyām āvāhanam. 
tarjanyau pūrvavad vyavasthāpya samayamudrā. 
āvāhanamudrāyāḥ prasāritatarjanībhyāṃ visarjanamudrā.
mantraḥ. om adhaḥ pṛthivyai svāhā. oṃ asurebhyaḥ svāhā. oṃ nāgebhyaḥ svāhā};
%
KSP (chapter 6, Homa, 6-7-14-10):
\textit{hastadvayena jalāñjaliṃ kṛtvāṇguṣṭhau vartulākārav adho dṛṣṭiṃ kṛtvā tarjanyaṇkuśābhyām āvāhayet.
tarjanyaṇkuśarahitā samayaḥ.
oṃ adhaḥpṛthivyai svāhā.
oṃ asurebhyaḥ svāhā.
oṃ nāgebhyaḥ svāhā.
atraiva tarjanīdvayaṃ prasārya visarjanam.
pṛthivyādīnām.}%
%
}} /}
\pend

\pstart
\Skt{%
\edtext{\edtext{tata}{\lemma{%
	{\rm tata\lem}
}\Dfootnote{%
	\emn\ \HT;
	\textit{tataḥ} \cod
}} ācamanaṃ svamantraireva sarveṣāṃ dattvā saśiṣyagaṇasya \edtext{mamāvighnaṃ}{\lemma{%
	{\rm mamāvighnaṁ\lem}
}\Dfootnote{%
	\emn;
	\textit{mamāvighnān} \HT\ \emn\ \sil;
	\textit{mamāvighnāṁ} \cod
}} kuruta karmasiddhiṃ ca me prayacchate\edtext{tyuktvā}{\lemma{%
	{\rm uktvā\lem}
}\Dfootnote{%
	\emn\ \HT\ \sil;
	\textit{ukvā} \cod
}} \edtext{sarvā}{\lemma{%
	{\rm sarvān\lem}
}\Dfootnote{%
	\emn\ \HT\ \sil;
	\textit{sarvām} \cod
}}nvisarjayediti //}{\lemma{%
	{\rm tata \dots\ visarjayed iti\lem}
}\Bfootnote{%
%
SDPT p.\ 278:
\textit{tata ācamanaṃ svamantrair eva sarveṣāṃ datvā, “saśiṣyagaṇasya mamāvighnaṃ kuruta karmasiddhiṃ ca me prayaccha”\ ity uktvā sarvān visarjayed iti.}
%
}} }
	%samapadaṁ: em. samapādaṁ, with HT.
	%-nyonyaṅgulyagra: em. -nyonyāṅgulyagraṁ, with HT.
	%-yojyāṁṅguṣṭau: em. -yojyāṅguṣṭhau, with HT (sil. em.).
	%catu\unclear{lā}-: em. vartulā, with HT, wo reads cartulā-.
	%-aṁkuśābhyā : em. -aṅkuśābhyām, with HT (sil. em.).
	%°adhapṛ-: em. °adhaḥpṛ-; HT emends °adhaspṛ-.
	%svāhoti: em. svāheti. HT edits svāhā || iti, with silent emendation.
	%tataḥ: em. tata, with HT.
	%ukvā: em. uktvā, with HT (sil. em.).
	%sarvām: em. sarvān, with HT (sil. em.).
\pend

\pstart
\Skt{%
\edtext{atha subāhuparipaṭhita\edtext{gāthābhi}{\lemma{%
	{\rm gāthābhir\lem}
}\Dfootnote{%
	\emn\ \HT\ \sil;
	\textit{gāthābhi} \cod
}}rbaliṃ dadyāt /}{\lemma{%
	{\rm atha \dots\ dadyāt\lem}
}\Bfootnote{%
%
SDPT p.\ 278:
atha subāhuparipaṭhitagāthābhir baliṃ dadyāt.
%
}} }
	%-gāthābhi: -gāthābhir, with HT (sil. em.).
\pend

\verse
\edtext{}{\lemma{%
	{\rm devāsurā \dots mahāvaneṣu\lem}
}\Bfootnote{%
	\textit{Subāhuparipṛcchā} (P LOCATION!!, D f.134r CHECK!!)
	\textcolor{red}{CHECK TANAKA'S AND MORITUCHI'S ARTICLES}
	\textcolor{red}{Padmaśrīmitra also quotes these verses. CHECK MS readings again! f.6v2–};
	\textcolor{blue}{Chinese translation: Taisho vol.18 731a15 ff.}
}}%
\Skt{devāsurā sarvabhujaṅgasiddhā\edtext{stārkṣyāḥ}{\lemma{%
	{\rm tārkṣyāḥ\lem}
}\Dfootnote{%
	\emn\ \HT;
	\textit{tākṣāḥ} \cod
}} suparṇāḥ kaṭapūtanāśca /\\
\edtext{gandharvayakṣā}{\lemma{%
	{\rm gandharvayakṣā\lem}
}\Cfootnote{%
	\HT\ read \textit{gandharvaṁ yakṣā} and emends \textit{gandharvā yakṣā}, against the meter.
}} grahajātayaśca  
ye \edtext{keci}{\lemma{%
	{\rm keci\lem}
}\Cfootnote{%
	\HT\ reads \textit{kecit} and emends to \textit{kecid}, against the meter.
}} bhūmau nivasanti divyāḥ //\\}
	%tākṣāḥ: em. tārkṣyāḥ, with HT.
	%gandharvayakṣā: HT read gandharvaṁ yakṣā and emends gandharvā yakṣā, against the meter.
	%keci: HT read kecit and emends to kecid, against the meter.
%
%
\Skt{nyastaikajānuḥ pṛthivītale'haṃ 
\edtext{kṛtāñjali}{\lemma{%
	{\rm kṛtāñjalir\lem}
}\Dfootnote{%
	\emn\ \HT;
	\textit{kṛtāñjali} \cod
}}rvijñāpayāmi \edtext{tāṃ}{\lemma{%
	{\rm tāṁs\lem}
}\Dfootnote{
	\emn\ \HT;
	\textit{tās} \cod
}}stu /\\
saputradāraiḥ saha bhṛtyasaṃghaiḥ 
śrutvā \edtext{ihāyāntu}{\lemma{%
	{\rm ihāyāntu\lem}
}\Dfootnote{%
	\cod;
	\textit{ihāyānti} \emn\ \HT\ \sil;
%	\textit{ihāyāṁtu} \cod
}} anugrahārtham //\\
	%kṛtāñjali: em. kṛtāñjalir, with HT.
	%tās: em. tāṁs, with HT.
	%-yāṁtu: em. yānti, with HT (sil. em.).
%
%
ye merupṛṣṭhe nivasanti \edtext{bhūtā}{\lemma{%
	{\rm bhūtā\lem}
}\Dfootnote{%
	\emn\ \HT\ \sil;
	\textit{bhutā |} \cod
}} 
ye nandane ye ca surālayeṣu /\\
ye codayāste \edtext{ravimaṇḍale}{\lemma{%
	{\rm ravimaṇḍale\lem}
}\Dfootnote{%
	\emn\ (following SDPT);
	\textit{ravimandare} \cod\ \HT
}} ca 
\edtext{nagareṣu}{\lemma{%
	{\rm nagareṣu\lem}
}\Cfootnote{%
	We should pronounce \textit{nag'reṣu} for metri causa.
}} sarveṣu ca ye vasanti //\\
	%bhutā |: bhūtā, with HT (sil. em.).
% udaya and asta (eastern and western mountains)
% nag'reṣu : we should read like this 
saritsu sarvāsu ca \edtext{saṃgameṣu}{\lemma{%
	{\rm saṁgameṣu\lem}
}\Dfootnote{%
	\emn\ \HT\ \sil;
	\textit{sagameṣu} \cod
}} 
ratnālaye cāpi kṛtādhivāsāḥ /\\
\edtext{vāpītaḍāgeṣu ca palvaleṣu}{\lemma{%
	{\rm vāpītaḍāgeṣu ca palvaleṣu\lem}
}\Dfootnote{%
	\emn\ \HT\ who cites SDPT;
	\textit{vāpītaḍāgeṣu palvaleṣu} \cod
}} 
kūpeṣu \edtext{vapreṣu}{\lemma{%
	{\rm vapreṣu\lem}
}\Dfootnote{%
	\emn\ \HT\ following SDPT;
	\textit{vadheṣu} \cod;
}} ca nirjhareṣu //}\\
	%sagameṣu: em. saṁgameṣu, with HT (sil. em.).
	%vāpītaḍāgeṣu palvaleṣu: em. vāpītaḍāgeṣu ca palvaleṣu, with HT who cites SDPT.
	%vadheṣu: HT emends vapreṣu, after SDPT.
%
%
\Skt{\edtext{ye grāmaghoṣeṣu purakānane vā}{\lemma{%
	{\rm ye grāmaghoṣeṣu purakānane vā\lem}
}\Dfootnote{%
	\emn;
	\edtext{ye grāmaghoṣe surakānane vā} \emn\ \HT\ \sil;
	\textit{ye grāmaghoṣe sarakānane vā} \cod
}\Cfootnote{%
%
\textit{Padmaśrīmitra}: \textbf{ye grāmaghoṣeṣu purakānare vā} gulmālaye devagṛheṣu ye ca |
vihāracaityāvasathāśrameṣu matheṣu śālāsu ca kuñjarāṇām ||
SDPT: grāmaghoṣeṣu surakānane vā;
\textit{Buddhacarita}: 
śrutvā tataḥ strījanavallabhānāṃ manojñabhāvaṃ \textbf{purakānanānām} |
bahiḥprayāṇāya cakāra buddhimantargṛhe nāga ivāvarūddhaḥ || 3.2 ||;
\textit{Brahmapurāṇa}: sa bhikṣām adadād vīraḥ sapta dvīpān vibhāvasoḥ |
purāṇi \textbf{grāmaghoṣāṃś} ca  viṣayāṃś caiva sarvaśaḥ  || 13.190 ||
jajvāla tasya sarvāṇi  citrabhānur didhṛkṣayā;
\textit{Rāmāyana}: 2.077.015a rajakās tunnavāyāś ca \textbf{grāmaghoṣa}mahattarāḥ |
2.077.015c śailūṣāś ca saha strībhir yānti kaivartakās tathā |;
\textit{Kathāsaritsāgara}: ratnair alaṃkṛtāṃ tāṃ ca kṛtvā karṇirathārpitām |
bhrāmayāmāsa nagara\textbf{grāmaghoṣeṣv} itas tataḥ || SoKss 12,27.91 (Vet 20.91) ||
%
}}
śūnyālaye devagṛheṣu ye ca /\\
vihāracaityāvasathāśrameṣu 
\edtext{maṭheṣu}{\lemma{%
	{\rm maṭheṣu\lem}
}\Dfootnote{%
	\emn\ \HT\ \sil;
	\textit{matheṣu} \cod
}} śālāsu ca kuñjarāṇām //}\\
	%sarakānane: em. surakānane, with HT (sil.em.).
	% purakānane? Aśvaghoṣa Buddhacarita
	% śū\linebreak{56v6}nyālaye
% for grāmaghoṣa: Mahāmāyūrī.  Sircar's glossary
%
\Skt{ye bhūbhṛtāṃ citragṛhe vasanti 
rathyāsu vīthīṣu ca catvareṣu /\\
ye \edtext{caikavṛkṣe}{\lemma{}\Cfootnote{%
	In the bottom margin of 56v we find the word \textit{kṣetra}. We are unable to connect this to any part of the text.
}}\supplied{ṣu mahāpatheṣu 
mahāśmaśāneṣu mahāvaneṣu //} }%
	%In the bottom margin of 56v we find the word kṣetra. We are unable to connect this to any part of the text.
\pend

\bigskip

\pstart
\mbox{}\hfill (folios 57–59 are missing)\hfill\mbox{}
\pend

\bigskip

\verse
siṃhebhaṛkṣādhyuṣiteṣu ye ca \edtext{vasanti}{\lemma{%
	vasanti\lem
}\Dfootnote{%
	Padmaśrīmitra
	\textit{vadanti} SDPT
}} ghorāsu mahāṭavīṣu |\\
dvīpeṣu divyeṣu kṛtālayāś ca merau śmaśāne nivasanti ye ca ||\\
hṛṣṭāḥ prasannāḥ srajagandhamālyam dhūpaṃ baliṃ \edtext{dīpavidhiṃ}{\lemma{%
	dīpavidhiṃ\lem
}\Dfootnote{%
	\emn;
	\textit{dīpaṃ vidhiṃ} SDPT;
	\textit{dīpavidhiś} Padmaśrīmitra
}} ca bhaktyā~|\\
gṛhṇantu bhuñjantu pibantu cedam idaṃ ca \edtext{karma}{\lemma{%
	karma\lem
}\Dfootnote{%
	Padmaśrīmitra
	\textit{karmaṃ} SDPT
}} saphalaṃ juṣantu~||\\
evaṃ tu kṛtvā grahapūjanaṃ tu digarcanaṃ tv ekamanā \edtext{prakuryāt}{\lemma{%
	prakuryāt\lem
}\Dfootnote{%
	\emn;
	\textit{prakuryā} SDPT
}} |\\
\edtext{aindryāṃ}{\lemma{
	aindryāṃ\lem
}\Dfootnote{%
	PSM;
	indrā SDPT
}} tu vajrī saha devasaṃghaiḥ imaṃ ca gṛhṇantu baliṃ viśiṣṭam~||\\
agnir yamo \edtext{nairṛtibhūpatiś}{\lemma{%
	nairṛtibhūpatiś\lem
}\Dfootnote{%
	PSM;
	\textit{nairṛtir bhūpatiś} SDPT
}} ca apāṃ patir \edtext{vāyudhanādhipaś ca}{\lemma{%
	vāyudhanādhipaś ca\lem
}\Dfootnote{
	PSM;
	\textit{vāyudhanādhipatiḥ} SDPT
}} |\\
\edtext{aiśānabhūtādhipatiś}{\lemma{%
	aiśānabhūtādhipatiś\lem
}\Dfootnote{%
	PSM;
	\textit{īśānabhūtādhipatiś} SDPT
}} ca devāḥ ūrdhvaṃ tu \edtext{candrārka}{\lemma{%
	candrārka\abbr
}\Dfootnote{
	\textit{candro 'rkaḥ} SDPT
}}\edtext{\textcolor{blue}{pitāmahaś}}{\lemma{%
	pitāmahaś\lem
}\Dfootnote{%
	\emn;
	\textit{pitāmahāś} Padmaśrīmitra;
	\textit{pitā mahāṃś} SDPT
}} ca ||\\
devāḥ samastā bhuvi ye ca nāgāḥ dharādharā guhyagaṇaiḥ sametāḥ~|\\
pratiprati tv \edtext{ekaniveditās}{\lemma{%
	ekaniveditās\lem
}\Dfootnote{%
	\emn;
	\textit{ekanivedanaṃ} SDPT;
	\textit{ekaniveditaṃ} PSM
}} tu \edtext{svakasvakāsv eva}{\lemma{%
	svakasvakāsv eva\lem
}\Dfootnote{%
	\emn;
	\textit{svakasvakāś caiva} SDPT;
	\textit{svakasvakasvāsu} PSM
}} diśāsu \edtext{bhūtvā}{\lemma{%
	bhūtvā\lem
}\Dfootnote{%
	PSM;
	\textit{bhūtāḥ} SDPT
}} ||\\
gṛhṇantu tuṣṭāḥ sabalāḥ sasainyāḥ saputramitrasvajanaiḥ sametāḥ |\\
dhūpaṃ baliṃ \edtext{dīpapuṣpavilepanaṃ}{\lemma{%
	dīpapuṣpavilepanaṃ\lem
}\Dfootnote{%
	\emn;
	\textit{dīpaṃ puṣpavilepanaṃ} SDPT:
	\textit{puṣpanivedanaṃ} PSM.
	N.B.\ PSM's reading is also possible.
}} ca bhuñjantu jighrantu pibantu cedam ||\\
idaṃ ca karmaṃ saphalaṃ juṣantu || iti |\\
(Supplied from SDPT pp.\ 280–282)
\pend

\bigskip

\pstart
%(= T144)
\mbox{}\hfill \textbf{[From here, a long parallel to SDPT starts]} \hfill \mbox{}
\pend

\pstart\noindent
[SVU Tib.] \Tib{de nas nye reg byas la/ dpal rdo rje sems dpas zhi pa'i sbyin sreg gi cho gas/ sbyin sreg brgya rtsa brgyad dbul lo// 
dpal rnam par snang mdzad la sogs pa'i snying pos ni/ sreg blugs bdun bdun dbul yo// 
de nas kyang de %{42a5} 
bzhin gshegs pa thams cad la phyag byas la/ }
\pend

\pstart\noindent
[Skt. Reconstruction] tata upaspṛśya śrīvajrasattvena śāntikahomavidhināṣṭottaraśatāhutiṃ dadyāt. śrīvairocanādihṛdayena saptasaptāhutiḥ. tataḥ sarvatathāgatān praṇamya
\pend

\verse
[SVU Tib.] \Tib{bdag ming 'di zhes bgyi ba yis// rdo rje slob dpon dka' thub ches// \\
sems can kun la phan pa'i phyir// slob ma rnams ni gzhug par \textcolor{blue}{bgyi}//} 
\pend

\verse
[Skt. Reconstruction] aham amukanāmā ca vajrācāryo mahātapāḥ~|\\
śiṣyān praveśayiṣyāmi sarvasattvahitārthataḥ ||
\pend

\pstart\noindent
[SVU Tib.] \Tib{bcom ldan 'das dkyil 'khor chen po 'dir zhugs pa la/ snod du %{42a6} 
gyur pa dang/ ma gyur pa brtag mi 'tshal lo// 
de ci'i phyir zhe na/ sems can kha cig ni/ sdig pa chen po bgyis pa yang mchis te/ de dag gis rdo rje dbyings kyi dkyil 'khor chen po 'dir zhugs te mthong na ngan song thams cad dang bral bar 'gyur ro//} 
\pend

\pstart\noindent
[Skt.\ Reconstruction] \textbf{[parallel to STTS starts $\rightarrow$ ]} \textcolor{blue}{bhagavanto} ’tra mahāmaṇḍalapraveśe pātrāpātraparīkṣā na kāryā. 
tat kasmād dhetoḥ: 
santi, bhagavantas tathāgatāḥ, kecit sattvā mahāpāpakāriṇas, ta idaṃ vajradhātumahāmaṇḍalaṃ dṛṣṭvā praviṣṭvā ca, sarvāpāyavigatā bhaviṣyanti. [STTS 210]
\pend

\pstart\noindent
[SVU Tib.] \Tib{bcom %{42a7} 
ldan 'das nor dang bza' ba dang/ btung ba dang/ 'dod pa'i yon tan thams cad la sogs pa dang dam tshig la mi dga' ba dang/ sngar bshad pa la sogs pa mi nus pa'i sems can yang mchis te/ de dag kyang 'dir ji ltar 'tshal ba bzhin du bgyid cing \edtext{zhugs pa rnams kyis}{\lemma{}\Efootnote{%
	D: zhugs [42b1] pa rnams kyis
}} bsam la thams cad rdzogs par 'gyur ro// }
\pend

\pstart\noindent
[Skt.\ Reconstruction] santi ca, bhagavantaḥ, sattvāḥ sarvārthabhojanapānakāmaguṇagṛddhāḥ, samayadviṣṭāḥ puraścaraṇādiṣv aśaktāḥ,
teṣām apy atra yathākāmakaraṇīyatayā praviṣṭānāṃ sarvāśāparipūrir bhaviṣyati. [STTS 211]
\pend

\pstart\noindent
[SVU Tib.] \Tib{bcom ldan 'das sems can kha cig ni gar dang glu dang gad mo dang/ rtsed mo dang/ kha zas spyod pa la dga' ba rnams de bzhin gshegs pa thams cad kyi theg pa chen po mngon par rtogs pa'i chos nyid khong du %{42b2} 
ma chud pas/ lha gzhan gyi rig sngags dang/ dkyil 'khor du 'jug gi/ de bzhin gshegs pa thams cad kyi dkyil 'khor re ba thams cad yang dag par rdzogs par gyur pa/ bla na med pa'i dga' ba dang/ mgu ba dang/ rangs pa rnams 'byung ba der ni/ bslabs pa'i %{42b3} 
tshig gis 'jigs shing skrag nas mi 'jug pa dag kyang mchis te/ de dag ngan song gi dkyil 'khor gyi lam du 'jug par kha bltas shing/ gnas pa rnams kyang dga' ba dang mgu ba'i dam pa thams cad 'grub cing/ bde ba dang yid bde ba myang ba bgyis pa dang/ ngan song gi %{42b4} 
rgyud thams cad 'jug par kha bltas pa'i lam las bzlog par bgyi ba'i slad du/ rdo rje dbyings kyi dkyil 'khor chen po 'dir 'jug tu rung ngo// }
\pend

\pstart\noindent
[Skt.\ Reconstruction] santi ca, bhagavantaḥ, sattvāḥ nṛttagāyahāsyalāsyāhāravihārapriyatayā sarvatathāgatamahāyānābhisamayadharmatānavabodhatvād anyadevakulamaṇḍalāni praviśanti, sarvāśāparipūrisaṃgrahabhūteṣu niruttararatiprītiharṣasaṃbhavakareṣu sarvatathāgatakulamaṇḍaleṣu, śikṣāpadabhayabhītā na praviśanti,
teṣām apāyamaṇḍalapraveśapathāvasthitamukhānām ayam eva vajradhātumahāmaṇḍalapraveśo yujyate,
sarvaratiprītyuttamasiddhisukhasaumanasyā*nubhavanārthaṃ \textcolor{blue}{(emend \textit{-nubhavārthaṃ}?)} sarvāpāyapratipraveśābhimukhapathavinivartanāya ca.
[STTS 212]
\pend

\pstart\noindent
[SVU Tib.] \Tib{bcom ldan 'das chos can gyi sems can/ de bzhin gshegs pa thams cad kyi tshul khrims kyi phung po dang/ ting nge 'dzin dang/ %{42b5} 
shes rab dam pa grub pa'i thabs kyis/ sangs rgyas rnams kyi byang chub tshol ba na/ bsam gtan dang/ rnam par thar pa dang/ sa rnams kyang 'dab cing nyon mongs pa dag kyang mchis te/ de dag kyang rdo rje dbyings kyi dkyil 'khor chen po 'dir zhugs ma thag tu/ de bzhin %{42b6} 
gshegs pa thams cad dkyil 'khor re ba thams cad yang dag par rdzogs par gyur pa/ bla na med pa thams cad du 'grub par dka' ba yang ma lags na/ dngos grub gzhan lta smos kyang ci 'tshal zhes zhu bar bya'o//} 
\pend

\pstart\noindent
[Skt.\ Reconstruction] santi ca punar, \textcolor{blue}{bhagavantaḥ}, dhārmikāḥ sattvāḥ sarvatathāgataśīlasamādhiprajñottamasiddhyupāyair buddhabodhiṃ prārthayanto dhyānavimokṣādibhir bhūmibhir yatantaḥ kliśyante,
teṣām atraiva vajradhātumahāmaṇḍalapraveśamātreṇaiva sarvatathāgatatvam api na durlabhaṃ, kim aṅga punar \textcolor{blue}{anyāḥ siddhīr} iti [STTS 213]  \textbf{[$\leftarrow$ parallel to STTS ends]} vijñāpayet.
\pend

\pstart\noindent
[SVU Tib.] \Tib{de nas slob ma gzhug par bya ste/ de la bslab pa'i %{42b7} 
gzhi lnga yongs su gzung pa'i dge bsnyen nam/ dge sbyong ngam/ dge slong gi sdom pa yongs su blangs pa/ slob dpon du dbang bskur bar 'os pa/ slob dpon gyi zhabs gnyis la phyag byas te/ 'di skad ces brjod par bya'o// }
\pend

\pstart\noindent
[Skt.\ Reconstruction] ataḥ śiṣyān praveśayet. 
tatra pañcaśikṣāpadaparigṛhīteno\textcolor{blue}{pāsakena} śrāmaṇerakabhikṣusaṃvaragṛhītena vācāryābhiṣekārhenācāryapādayoḥ praṇipatyaivaṃ vaktavyam.
\pend

\verse
[SVU Tib.] \Tib{dga' chen khyod bdag \edtext{ston pa bas/}{\lemma{}\Efootnote{%
	ston pa [43a1] bas
}} slob dpon khyod ni dgongs su gsol//\\ 
byang chub sems dpa'i tshul brtan pa// mgon po chen po bdag 'tshal lo//\\ 
dam tshig de nyid bdag la stsol// sdom pa rnams kyang bdag la stsol// } 
\pend

\verse
[Skt.\ Reconstruction] \edtext{\textcolor{blue}{samanvāhara ācārya}}{\lemma{%
	{\rm adhyeṣayāmi ācāryaṃ\lem}
}\Bfootnote{%
%
Possibly this \textit{pāda} in SVU Tib.\ was inserted by the translators.
Cf. KSP (opening part of the pratiṣṭhā):
adhyeṣayāmi he nātha tvaṃ me śāstā mahāvibho \dag sarvasattvān saṃsārasāgarād uddhṛtya\dag \ buddhatve pratiṣṭhāpanāya yuṣmākaṃ pūjanāya ca vihārādipratiṣṭhāṃ mahāmaṇḍalavartanaṃ vā homaṃ vā kariṣyāmi. 
yuṣmābhir yat kartavyaṃ tat karaṇīyam
%
}} tvaṃ me śāstā mahārata~|\\
icchāmy ahaṃ mahānātha mahābodhinayaṃ dṛḍham |\\
dehi me samayaṃ tattvaṃ saṃvaraṃ ca dadasva ma iti ||
\pend

\pstart\noindent
[SVU Tib.] \Tib{de nas rdo rje slob dpon gyis// rdo rje srung ba yongs su %{43a2} 
bzlas la/ stod g-yogs dang smad g-yogs bskon zhing/ rdo rje gnod sbyin dang/ rdo rje lcags kyu la sogs pa'i sgo srung bzhis/ bzlas pa'i gdong g-yogs bcings la/ phyag bzhi byed du gzhug par bya'o//}
\pend

\pstart\noindent
[Skt.\ Reconstruction] tata \textcolor{blue}{ācāryo} \edtext{vajra\textcolor{blue}{rakṣa}parijaptanivasanottarīyaṃ}{\lemma{%
	{\rm vajrayakṣaparijaptanivasanottarīyaṃ\lem}
}\Cfootnote{%
	thus SVU Tib. (except that SVU Tib.\ reads \textit{rdo rje srung ba} (*\textit{vajrarakṣa})); vajrayakṣaparijaptaṁ nīlavastranivasanottarīyaṁ SDPT.
}} \edtext{vajrāṃkuśādi\-}{\lemma{%
	{\rm vajrāṃkuśādi-\lem}
}\Cfootnote{%
	thus SDPT; in SVU Tib. we find words that seem to represent \textit{vajrayakṣavajrāṃkuśādi-}.
}}dvārapālacatuṣṭayaparijaptamukhaveṣṭanaṃ śiṣyaṃ kṛtvā catuḥpraṇāmaṃ kārayet
% vajrayakṣa: rdo rje srung ba *vajrarakṣa
\pend

\pstart\noindent
[SVU Tib.] \textcolor{blue}{\textbf{(Parallel to \textit{Vajraśekhara} starts here $\rightarrow$)}} \Tib{\edtext{yang lag na me tog thogs pa'i slob mas/ slob dpon la %{43a3} 
phyag byas la/ lag na me tog thogs pa de nyid dang/ bshags pa dang/ rjes su yi rang ba dang/ bskul ba dang/ gsol ba yang byas la/ 'di skad ces brjod par bya'o//}{\lemma{%
	{\rm yang \dots\ brjod par bya'o\lem}
}\Bfootnote{%
%	
\textit{Vajraśekhara}: 
\textit{sngon tu dkyil 'khor rab 'jug cing ||
% sngon tu dkyil [D 183r] 'khor
cho ga bzhin du bcug nas ni || ||
de nas me tog bcas thal mo ||
lus btud dga' ba che rnams kyi ||
bshags pa dang ni gsol ba gdab ||
rjes su yi rang kun tu bya ||} (D f.182v3–183r1)
}} } 
\pend

\pstart\noindent
[Skt.\ Reconstruction] punaḥ puṣpakareṇa śiṣyeṇācāryaṃ \textcolor{blue}{praṇamya} puṣpakareṇaiva deśanānumodanādhyeṣāṇāyācanāṃ \textcolor{blue}{*ca (strange. ca should be removed?} kṛtvā vaktavyam.
\pend

\verse
[SVU Tib.] \Tib{\edtext{gtso bo bdag la sdom pa stsol//}{\lemma{%
	{\rm gtso bo \dots\ stsol\lem}
}\Bfootnote{%
%
\textcolor{blue}{Hereafter parallel to the \textit{Vajraśekhara} added in the second layer (March 2020).}
\textit{Vajraśekhara}:
\textit{gtso bo bdag la sdom pa stso} (D f.183r2)
%
}} }\\
%
\Tib{\edtext{thub pa'i nyi ma ma lus pa'i// sangs rgyas rnams ni bdag %{43a4} 
la dgongs// \\
bdag ming 'di zhes bya ba ni// slob dpon \textcolor{blue}{dpang} du gnas pa la//}{\lemma{%
	{\rm thub pa'i \dots\ gnas pa la\lem}
}\Bfootnote{%
%
\textit{Vajraśekhara}:
\textit{thub pa'i nyi ma ma lus pa'i ||
sangs rgyas bdag la dgongs su gsol ||
bdag ming 'di zhes bgyi ba 'di ||
slob dpon \textcolor{red}{dpang} du gnas pa la ||}
(D f.183r2–3)
%
}}} \\
%
\Tib{\edtext{sangs rgyas rol mo las byung ba// mi ldog pa yi 'khor los 'byor// \\
thar pa'i rang bzhin grong khyer cher// gsang ba chen por 'jug par bgyi//}{\lemma{%
	{\rm sangs rgyas \dots\ bgyi\lem}
}\Bfootnote{%
%
\textit{Vajraśekhara}:
\textit{sangs rgyas rol mo las byung ba ||
mi ldog 'khor lo'i 'byor par ldan ||
thar pa chen po'i grong khyer mchog ||
gsang ba chen po 'jug par bgyi ||}
(D f.183r3)%
%
}} } \\
%
\Tib{\edtext{gsang ba chen por bsdus pa ni// %{43a5} 
slob dpon chen po bdag 'jug mdzod// \\
mi ldog pa yi dbang bskur ba// skal pa chen po bdag la stsol//}{\lemma{%
	{\rm gsang ba \dots\ stsol\lem}
}\Bfootnote{%
%
\textit{Vajraśekhara}:
\textit{gsang ba'i rigs mtho thams cad du ||
slob dpon chen po bdag 'jug mdzod ||
mi ldog pa yi dbang skur ba ||
skal ba chen po bdag la stsol ||}
(D f.183r3–4)
%
}} }\\
%
\Tib{\edtext{mtshan rnams kyis ni rgyas btab cing// dpe byad rnams dang yang dag ldan// \\
\textcolor{blue}{sangs rgyas sku ni} yid 'ong ba// slob dpon chen po bdag la stsol//}{\lemma{%
	{\rm mtshan \dots\ stsol\lem}
}\Bfootnote{%
%
\textit{Vajraśekhara}:
\textit{mtshan rnams kun gyis rgyas btab cing ||
dpe byad rnams dang yang dag ldan ||
sangs rgyas sku ni yid 'ong ba ||
slob dpon chen po bdag la stsol ||}
(D f.183r4)
%
}} } \\
%
\Tib{\edtext{dbang %{43a6} 
bskur chen po rmad byung ba// sems can kun gyi don bgyid phyir// \\
bdag ni rtag tu slob dpon gyur// slob dpon chen po bdag la stsol//}{\lemma{%
	{\rm bskur \dots\ stsol\lem}
}\Bfootnote{%
%
\textit{Vajraśekhara}:
\textit{dbang bskur ba ni rmad byung che ||
sems can kun gyi don bgyid phyir ||
bdag ni rtag tu slob dpon 'gyur ||
slob dpon chen po bdag la stsol ||}
(D f.183r4–5)
%
}}}
\pend

\verse
[Skt.\ Reconstruction]  dehi me saṃvaraṃ vibho ||\\
samanvāharantu māṃ buddhā aśeṣā munibhāskarāḥ |\\
aham amukanāmā vai \edtext{ācāryasamakṣaṃ}{\lemma{%
	{\rm ācāryasamakṣaṃ\lem}
}\Cfootnote{%
	SVU Tib.\ reads \textit{slob dpon dbang du}.
	This supports \textit{ācāryasaṃnaddhaṃ}?
	Negi repots \textit{saṃbaddha} as a word corresponding to \textit{dban du rgyur ba} (p.3944.
}} sthitaḥ ||\\
praviśāmi mahāguhyaṃ buddhanāṭakasaṃbhavam ||\\
avaivartika\edtext{cakrāḍhyaṃ}{\lemma{%
	{\rm -cakrāḍhyaṃ\lem}
}\Dfootnote{%
	 \emn; 
	 \textit{-cakrādyaṃ} ed.
}} mahāmokṣapuraṃ varam ||\\
praveśa māṃ mahācārya \edtext{\textcolor{blue}{mahā}guhyakuloccayam}{\lemma{%
	{\rm sarvaguhyakuloccayam\lem}
}\Cfootnote{%
	SVU Tib.\ reads \textit{gsang ba chen por bsdus pa ni}. %This supports \textit{mahāguhyakuloccayam}?
	\textit{sarvaguhyakuloccayam} SDP
}} |\\
dadasva me \textcolor{blue}{mahābhāga} avaivartyabhiṣecanam ||\\
dadasva me mahācārya \edtext{\textcolor{blue}{lakṣaṇaiḥ sumudritam}}{\lemma{%
	{\rm lakṣaṇasyānumodanam\lem}
}\Cfootnote{%
	SVU Tib. reads \textit{mtshan rnams kyis ni rgyas btab cing}. %This supports \textit{lakṣaṇasyāmudraṇam}?
	\textit{lakṣaṇasyānumodanam} SDP
}} |\\
anuvyañjanasaṃyuktaṃ \edtext{buddha\textcolor{red}{kāyaṃ}}{\lemma{%
	{\rm buddhakāyaṃ\lem}
}\Cfootnote{%
	SVU Tib.\ reads \textit{sangs rgyas kun}, which supports \textit{saravbuddhaṃ}.
}} manoramam ||\\
dadasva me mahācārya abhiṣekaṃ mahādbhutam |\\
ācāryo 'haṃ \edtext{bhave}{\lemma{%
	{\rm bhave\lem}
}\Dfootnote{%
	\emn;
	\textit{bhaven} ed.
}} nityaṃ sarvasattvārthakāraṇāt ||
\pend

\pstart\noindent
[SVU Tib.] \Tib{de nas slob dpon gyis rigs thams cad la gsol ba gdab par bya ste/}
\pend

\pstart\noindent
[Skt.\ Reconstruction] tata ācāryeṇa sarvakul\edtext{\textcolor{blue}{ādhyeṣaṇā}}{\lemma{%
	{\rm -ādhyeṣaṇā\lem}
}\Cfootnote{%
	SVU Tib.\ (\textit{gsol ba gdab par bya}) supports \textit{-adhyeṣaṇā};
	\textit{-vijñaptiḥ} SDP
}} kāryā
\pend

\verse
[SVU Tib.] \Tib{\edtext{che ge zhes bya 'di nyid ni// byang chub sems ni %{43a7} 
yongs 'dzin zhing// \\
dam tshig sdom dang gsang ba yi// 'khor lo 'dir yang 'jug par 'tshal//}{\lemma{%
	{\rm che ge \dots\ 'tshal\lem}
}\Bfootnote{%
%
\textit{Vajraśekhara}:
\textit{bdag nyid nye bar zhi ba nyid ||
byang chub sems ni yongs su gzung ||
gsang ba yi ni 'khor lo ru ||
sdom pa dang bcas 'jug par bgyi ||}
(D f.183r5–6)
%
}}} 
%
\pend

\verse
[Skt.\ Reconstruction] ayaṃ \textcolor{blue}{evāmukanāmā} bodhicittaparigrahaḥ |\\
icchate guhyacakre 'smin \edtext{praveṣṭuṃ}{\lemma{%
	{\rm praveṣṭuṃ\lem}
}\Dfootnote{%
	\emn;
	\textit{praveṣṭāṃ} ed.;
	\textit{'jug par 'tshal} SVU Tib.
}} samayasaṃvaram ||
\pend

\pstart\noindent
[SVU Tib.] \Tib{\edtext{de nas yang slob dpon gyis 'di skad ces brjod par bya ste/}{\lemma{%
	{\rm de nas \dots\ brjod par bya ste\lem}
}\Bfootnote{%
%
\textit{Vajraśekhara}:
\textit{da nas slob dpon gyis 'di skad ces brjod par bya ste}
(D f.183r6)
%
}} } 
\pend

\pstart\noindent
[Skt.\ Reconstruction] tata ācāryeṇa vaktavyaṃ 
\pend

\verse
[SVU Tib.] \Tib{\edtext{gsang ba chen po'i rigs dag pa'i// gsang ba yongs su 'dzin par ni// \\
bdag nyid chen po khyod \edtext{'dod dam}{\lemma{}\Efootnote{%
	'dod [43b1] dam
}}// des kyang 'tshal zhes brjod par bya//}{\lemma{%
	{\rm gsang ba chen po'i \dots\ brjod par bya\lem}
}\Bfootnote{%
%
\textit{Vajraśekhara}:
\textit{gsang ba chen po rigs dag pa ||
gsang ba yongs su 'dzin %@384a *| |
par ni ||
bdag nyid chen po khyod 'dod dam ||
des kyang 'tshal zhes brjod par bya ||}
(D f.183r6–7);
\textit{Bhūtaḍāmara} manual:
\textit{icchasi tvaṃ mahān mahāguhyakulaṃ śuddhaṃ rahasyaṃ parigṛhnitum |
tenaivaṃ vācyam icchāmy aham iti |} (f.8r)
}} }
\pend

\verse
[Skt.\ Reconstruction] \textcolor{blue}{icchasi tvaṃ mahātman |\\ 
mahāguhyakulaṃ śuddhaṃ rahasyaṃ parigṛhṇitum ||} %(metre!!)
\pend
\pstart\noindent
\textcolor{blue}{tenaivaṃ vācyam}
\pend
\verse
\textcolor{blue}{icchāmy aham iti.}
\pend

\pstart\noindent
[SVU Tib.] \Tib{\edtext{yang slob dpon gyis 'di skad cas brjod par bya ste/}{\lemma{%
	{\rm yang \dots\ brjod par bya ste\lem}
}\Bfootnote{%
%
\textit{Vajraśekhara}:
\textit{yang slob dpon gyis 'di skad ces brjod par bya ste} (D f.183r7)
%
}}} 
\pend

\pstart\noindent
[Skt.\ Reconstruction]  punar ācāryeṇaivaṃ vaktavyam. (not in SDP)
\pend

\verse
[SVU Tib.] \Tib{\edtext{sangs rgyas chos dang dge 'dun te// dkon mchog gsum la skyabs su song// \\
'di ni sangs rgyas rigs dag pa'i// dam tshig brtan par gyur pa'o//}{\lemma{%
	{\rm sangs rgyas \dots\ gyur pa'o\lem}
}\Bfootnote{%
%
\textit{Vajraśekhara}:
\textit{sangs rgyas chos dang dge 'dun ste ||
dkon mchog gsum la skyabs su song ||
'di ni sangs rgyas rigs dag pa'i ||
dam tshig brtan par gyur pa'o ||}
(D ff.183r7–183v1)
%
}} } \\
%
\Tib{\edtext{rdo rje dril bu'i phyag %{43b2} 
rgya dag/ blo chen khyod kyis gzung bar bgyi//  \\
byang chub sems gang de rdo rje// shes rab dril bu zhes brjod do//}{\lemma{%
	{\rm rdo rje \dots\ brjod do\lem}
}\Bfootnote{%
%
\textit{Vajraśekhara}:
\textit{rdo rje dril bu phyag rgya yang ||
blo chen gyis ni gzung bar bgyi ||
byang chub sems gang de rdo rje ||
shes rab dril bu zhes bshad do ||}
(D f.183v1)
%
}} }  \\
%
\Tib{\edtext{slob dpon dag kyang gzung bar bya// bla ma sangs rgyas kun dang mnyam//  \\
'di ni rdo rje \textcolor{blue}{rigs dga' ba'i}// sdom pa dam tshig yin par gsungs//}{\lemma{%
	{\rm slob dpon \dots\ gsungs\lem}
}\Bfootnote{%
%
\textit{Vajraśekhara}:
\textit{slob dpon dag kyang gzung bar bya ||
bla ma sangs rgyas kun dang mnyam ||
'di ni rdo rje rigs dag pa'i ||
sdom pa dam tshig yin par gsungs ||}
(D f.183v1–2)%
%
}}}  \\
%
\Tib{\edtext{rin chen rigs %{43b3} 
mchog chen po la// nyin dang mtshan mo lan gsum du//  \\
zang zing mi 'jigs chos dang byams// sbyin pa rnams bzhi rtag tu sbyin//}{\lemma{%
	{\rm rin chen \dots\ sbyin\lem}
}\Bfootnote{%
%
\textit{Vajraśekhara}:
\textit{rin chen rigs mchog chen po la ||
nyin dang mtshan mo lan gsum du ||
zang zing mi 'jigs chos dang byams ||
sbyin pa rnam bzhi rtag tu sbyin ||}
(D f.183v2)%
%
}} }  \\
%
\Tib{\edtext{\textcolor{red}{'di ni rin chen rigs dag pa'i// dam tshig sdom pa yin par gsungs//}}{\lemma{%
	{\rm 'di ni rin chen \dots\ gsungs\lem}
}\Cfootnote{%
%
Compared with the parallel in the \textit{Vajraśekhara},
the Tibetan translation of the SVU has extra two \textit{pāda}s
(which is written in red at this moment).%
%
}}  \\
\edtext{phyi dang gsang ba theg pa gsum// dam chos khyod kyis gzung %{43b4} 
bar bya//\\
'di ni pad+ma'i rigs dag pa'i// dam pa'i sdom pa yin par gsungs//}{\lemma{%
	{\rm 'di ni rin chen \dots\ gzung bar bya\lem}
}\Bfootnote{%
%
\textit{Vajraśekhara}:
\textit{phyi \textcolor{red}{nang} gsang ba'i theg pa gsum ||
dam chos khyod kyis gzung bar bya ||
'di ni pa dma'i rigs dag pa'i ||
dam tshig sdom pa yin par gsungs ||}
(D f.183v2–3)%
%
}}}  \\
%
\Tib{\edtext{las kyi rigs mchog chen po la// sdom pa thams cad ldan par ni//\\
yang dag nyid du gzung bar gyis// mchod pa'i las kyang ci nus bya//}{\lemma{%
	{\rm las \dots\ ci nus bya\lem}
}\Bfootnote{%
%
\textit{Vajraśekhara}:
\textit{las kyi rigs mchog chen po la ||
sdom pa thams cad ldan par ni ||
yang dag nyid du gzung bar bgyi ||
mchod pa'i las kyang ci nus bya ||}
(D f.183v3)
%
}} }
%
\pend

\verse
[Skt.\ Reconstruction]  buddhaṃ dharmaṃ ca saṃghaṃ ca \textcolor{blue}{triratnaṃ} śaraṇaṃ vraja |\\
etad buddhakule \edtext{\textcolor{red}{ramye}}{\lemma{%
	{\rm ramye\lem}
}\Cfootnote{%
	SVU Tib.\ supports \textit{śuddhe} (\textit{dag pa'i}).
}} saṃvaraṃ \edtext{\textcolor{blue}{bhavato}}{\lemma{%
	{\rm bhavato\lem}
}\Cfootnote{%
	genitive of \textit{bhavat}.
}} dṛḍham || [Buddhakula]\\
vajraṃ ghaṇṭā ca mudrā ca tvayā grāhyā mahāmate |\\
yad bodhicittaṃ tad vajraṃ prajñā ghaṇṭā iti smṛtā ||\\
ācāryaś ca gṛhītavyaḥ sarvabuddhasamo guruḥ |\\
etad vajrakule śuddhe saṃvaraṃ samayocyate || [Vajrakula]\\
caturdānaṃ pradātavyaṃ tridive ca trirātrike |\\
āmiṣābhayadharmākhyā maitrī ratnakuloccaye || [Ratnakula]\\
saddharmaṃ ca tvayā grāhyaṃ (bāhyaṃ) guhyaṃ triyānikam |\\
etat padmakule śuddhe saṃvaraṃ samayocyate || [Padmakula]\\
saṃvaraṃ sarvasaṃyuktaṃ parigṛhṇīṣva tattvataḥ |\\
pūjākarma yathāśaktyā mahākarmakuloccaye || [Karmakula]
%
\pend

\verse
[SVU Tib.] \Tib{\edtext{gzhan yang bcu bzhi 'di dag ni// phas pham %{43b5} 
par ni rab tu bshad//  \\
spang zhing dor bar mi bya'o// rtsa ba'i ltung bar shes par bya//}{\lemma{%
	{\rm gzhan yang \dots\ shes par bya\lem}
}\Bfootnote{%
%
\textit{Vajraśekhara}:
\textit{de las gzhan pa bcu bzhi ni ||
phas pham par ni rab tu bshad ||
spang zhing dor bar mi bya ste ||
rtsa ba'i ltung ba zhes bshad do ||}
(D f.183v3–4)%
%
}}  \\
\edtext{nyin dang mtshan mo lan gsum du// nyin re bzhin ni bklag par bya//  \\
gang tshe nyams na rnal 'byor pa// \textcolor{blue}{kha na ma tho} sbom por 'gyur//}{\lemma{%
	{\rm nyin \dots\ sbom por 'gyur\lem}
}\Bfootnote{%
%
\textit{Vajraśekhara}:
\textit{nyin dang mtshan mo lan gsum du ||
nyin re zhing ni bzla bar byed ||
gang tshe nyams gyur rnal 'byor pa ||
kha na ma tho sbom por 'gyur ||}
(D f.184v4)%
%
}}  \\
\edtext{khyod kyis srog chags gsad mi bya// ma byin par %{43b6} 
ni blang mi bya//  \\
'dod pa log par mi spyad cing// brdzun du smra bar mi bya'o//}{\lemma{%
	{\rm khod \dots\ mi bya'o\lem}
}\Bfootnote{%
%
\textit{Vajraśekhara}:
\textit{khyod kyi srog chags gsad mi bya ||
ma byin par yang mi blang ngo ||
'dod pa log par mi spyad cing ||
brdzun du smra bar mi bya'o ||}
(D f.183v4–5)%
%
}}  \\
\edtext{phung khrol kun gyi rtsa ba yi// chang ni rnam par spang bar bya//  \\
sems can gdul phyir ma gtogs pa// bya ba ma yin thams cad spang//}{\lemma{%
	{\rm phung khrol \dots\ spang\lem}
}\Bfootnote{%
%
\textit{Vajraśekhara}:
\textit{phung khrol kun gyi rtsa ba yi ||
chang ni rnam par spang bar bya ||
sems can gdul phyir ma gtogs pa || 
bya ba ma yin thams cad spang ||}
(D f.183v5)
%
}}  \\
\edtext{dam pa nye bar bsten bya zhing// rnal 'byor ba rnams bsnyen %{43b7} 
bkur bya//  \\
lus kyi las ni rnam gsum dang// ngag gi rnam pa bzhi rnams dang//  \\
yid kyi rnam pa gsum dag ni// ci nus par ni rjes su skyong//  \\
theg pa dman pa 'dod mi bya// sems can don la rgyab mi phyogs//}{\lemma{%
	{\rm dam pa \dots\ mi phyongs\lem}
}\Bfootnote{%
%
\textit{Vajraśekhara}:
\textit{dam pa nye bar bsten bya zhing ||
rnal 'byor pa rnams bsnyen bkur bya ||
lus kyi las ni rnam gsum dang ||
ngag gi rnam pa bzhi rnams dang || 
yid kyi rnam pa gsum dag ni ||
ci nus par ni rjes su skyongs ||
theg pa dman la 'dod mi bya ||
sems can don la rgyab phyogs min ||}
(D f.183v5–6)%
%
}}  \\
\edtext{'khor ba dag kyang yongs mi spang// rtag tu \edtext{mya ngan 'das}{\lemma{}\Efootnote{%
	mya ngan [44a1] 'das
}} mi chags//  \\
lha dang lha min gsang ba la// khyod kyis brnyas par mi bya zhing//}{\lemma{%
	{\rm 'khor ba \dots\ mi bya zhing\lem}
}\Bfootnote{%
%
\textit{Vajraśekhara}:
\textit{khor ba dag kyang spang mi bya ||
rtag tu mya ngan 'das ma chags ||
lha dang lha min gsang ba pa ||
khyod kyis brnyas par mi bya zhing ||}
(D f.183v6–7)
%
}}  \\
\edtext{phyag rgya bzhon pa mtshon cha dang// mtshan ma 'gom par mi bya'o//  \\
'di dag dam tshig yin par bshad// khyod kyis rtag tu bsrung bar bya//}{\lemma{%
	{\rm phyag rgya \dots\ bsrung bar bya\lem}
}\Dfootnote{%
%
\textit{Vajraśekhara}:
\textit{phyag rgya bzhon pa mtshon cha dang ||
mtshan ma 'gom par mi bya'o ||
'di dag dam tshig yin par bshad ||
khyod kyi rtag tu bsrung bar bya ||}
(D f.183v7)%
%
}} }
\pend

\verse
[Skt.\ Reconstruction] etat pārājikākhyātaś \textcolor{blue}{caturdaśa-m-ataḥ} param |  \\
na tyājyaṃ na ca kṣeptavyaṃ mūlāpattir iti smṛtam ||\\
tridive ca trirātrau ca vartitavyaṃ dine dine |  \\
yadā hānir bhaved yogī sthūlāpattyo bhaviṣyati ||\\
prāṇinaś ca na te ghātyā adattaṃ naiva cāharet | \\
nācaret kāmamithyāyāṃ mṛṣā naiva ca bhāṣayet ||\\
mūlaṃ sarvasyānarthasya madyapānaṃ vivarjayet |  \\
akriyāṃ varjayet sarvāṃ sattvārthaṃ vinayena ca ||\\
sādhūnām upatiṣṭheta yogināṃ paryupāsanam |  \\
trividhaṃ kāyikaṃ karma vacasā ca caturvidham ||\\
manasā triprakāraṃ ca yathāśaktyānupālayet |\\
hīnayānaspṛhā naiva sattvārthaṃ vimukhaṃ na ca ||\\
na saṃsāraparityāgī na \textcolor{blue}{nirvāṇarataḥ} sadā |  \\
apamānaṃ na te kāryaṃ \edtext{devatāsuraguhyake}{\lemma{%
	{\rm devatā na ca guhyake\lem}
}\Dfootnote{%
	\conj;
	\textit{devatā na ca guhyake} SDPT;
	SVU Tib. translates \textit{na} as \textit{amanuṣya}:
	\textit{lha dang lha min gsang ba la} (D f.44r1).
}} ||\\
na ca cihnaṃ samākramyaṃ mudrā vāhanam āyudham | \\
etat samayam ity uktaṃ rakṣitavyaṃ tvayā \edtext{\textcolor{blue}{sadā}}{\lemma{%
	{\rm sadā\lem}
}\Cfootnote{%
%
Following SVU Tib. and the \textit{Vajraśekhara};
\textit{mate} SDPT
%
}} ||
\pend

\verse
[SVU Tib.] \Tib{\edtext{des kyang 'di skad %{44a2} 
brjod par bya ste// slob dpon 'dir ni bdag la gson//\\ 
gtso bos ji ltar bka' stsal pa// de ltar bdag ni bgyid par 'tshal//}{\lemma{%
	{\rm des kyang \dots\ mgon po rnams\lem}
}\Bfootnote{%
%
\textit{Vajraśekhara}:
\textit{de nyid kyis kyang brjod par bya ||
'dir ni slob dpon bdag la gson ||
gtso bo ji ltar bka' stsal pa ||
de bzhin du ni bgyid par 'tshal ||}
(D f.184r1)%
}} }
\pend

\verse
[Skt.\ Reconstruction] 
tasyaiva cāpi vaktavyam
\edtext{ācāryātra}{\lemma{%
	{\rm ācāryātra\lem}
}\Dfootnote{%
	Supported by SVU Tib. \textit{'dir};
	\textit{ācārya tu} SDPT
}} śṛṇuṣva me |\\ 
evam \textcolor{blue}{astu} kariṣyāmi yathājñāpayase vibho ||
\pend

\verse
[SVU Tib.] \Tib{\edtext{ji ltar dus gsum mgon po rnams//
byang chub tu ni nges mdzad pa//}{\lemma{%
	{\rm ji ltar \dots\ mdzad pa\lem}
}\Bfootnote{%
%
Ānandagarbha abridges the verses.
\textit{Vajraśekhara}:
\textit{ji ltar dus gsum mgon po rnams ||
byang chub tu ni nges mdzad pa'i ||
byang chub sems ni bla na med ||
dam pa bdag gis bskyed par bgyi ||
%
sangs rgyas rnal 'byor sdom pa la ||
tshul khrims gyi ni bslab pa dang ||
dge ba'i chos ni sdud pa dang ||
sems can don byed tshul khrims gsum ||
%
bdag gis brtan por gzung bar bgyi ||
sangs rgyas chos dang dge 'dun te ||
bla na med pa'i dkon mchog gsum ||
deng nas brtsam ste gzung bar bgyi ||
%
rdo rje rigs mchog chen po la ||
rdo rje dril bu phyag rgya yang ||
yang dag nyid du bzung bar bgyi ||
slob dpon dag kyang gzung bar bgyi ||
%
rin chen rigs mchog chen po yi ||
dam tshig yid du 'ong ba la ||
nyin re zhing ni dus drug bya ||
sbyin pa rnam bzhi rab tu sbyin ||
%
byang chub chen po las byung ba ||
padma'i rigs chen dag pa la ||
phyi nang gsang ba'i theg pa gsum ||
dam pa'i chos ni gzung bar bgyi ||
%
las kyi rigs mchog chen po la ||
sdom pa thams cad ldan par ni ||
yang dag nyid du gzung bar bgyi ||
mchod pa'i las kyang ci nus bgyi ||
%
byang chub sems ni bla na med pa || 
dam pa bdag gis bskyed bgyis nas ||
sems can kun gyi don gyi phyir ||
bdag gi sdom pa ma lus gzung ||
%
ma grol ba ni dgrol bar bgyi ||
ma brgal ba ni bsgral bar bgyi ||
dbugs ma phyin pa dbugs dbyung zhing ||
sems can mya ngan 'das la dgod ||}
(D f.184r1–4).
One of the earliest surviving Sanskrit source of this set of verses is \textit{Nāmamantrārthāvalokinī}, a commentary on the \textit{Nāmasaṃgīti} by Vilāsavajra.
\textit{Nāmamantrārthāvalokinī}, \textit{Adhikāra} 4:
\textit{utpādayāmi paramaṃ bodhicittam anuttaram |
yathā traiyadhvikā nāthāḥ saṃbodhau kṛtaniścayāḥ ||
trividhāṃ śīlaśikṣāṃ ca kuśaladharmasaṃgraham |
sattvārthakriyāśīlaṃ ca pratigṛhṇāmy ahaṃ dṛḍham ||
buddhaṃ dharmaṃ ca saṃghaṃ ca triratnāgram anuttaram |
adyāgreṇa grahīṣyāmi saṃvaraṃ buddhayogajam ||
vajraṃ ghaṇṭāṃ ca mudrāṃ ca pratigṛhṇāmi tattvataḥ |
ācāryaṃ ca grahīṣyāmi mahāvajrakuloccaye ||
caturdānaṃ pradāsyāmi ṣaṭkṛtvā tu dine dine |
mahāratnakule yogye samaye ca manorame ||
saddharmaṃ pratigṛhṇāmi bāhyaṃ guhyaṃ triyānikam |
mahāpadmakule śuddhe mahābodhisamudbhave ||
saṃvaraṃ sarvasaṃyuktaṃ pratigṛhṇāmi tattvataḥ |
pūjākarma yathāśaktyā mahākarmakuloccaye ||
utpādayitvā  paramaṃ bodhicittam anuttaram |
gṛhītvā saṃvaraṃ kṛtsnaṃ sarvasattvārthakāraṇāt ||
atīrṇāṃs tārayiṣyāmi amuktān mocayāmy ahaṃ |
anāśvastān āśvāsayiṣyāmi sattvān sthāpayiṣyāmi nirvṛtāv iti  ||}
(\textsc{Tribe} 2016: 248–250)
%
}} } 
\pend
\pstart\noindent
\Tib{zhes bya ba la sogs pa nas//}
\pend
\verse 
\Tib{sems can mya ngan 'das la %{44a3} 
dgod//} 
\pend
\pstart\noindent
\Tib{ces bya ba'i bar du brjod par bya'o//} [\textcolor{blue}{\textbf{$\leftarrow$Parallel to the \textit{Vajraśekhara} ends}}]
\pend

\verse
[Skt.\ Reconstruction] 
utpādayāmi paramam 
\pend
\pstart\noindent
ityādi yāvat 
\pend
\verse
\edtext{sattvān}{\lemma{%
	{\rm sattvān\lem}
}\Dfootnote{%
	supported by SVU Tib. (\textit{sems can})
	\textit{sarvān} SDPT
}} sthāpayiṣyāmi nirvṛtāv ity \textcolor{blue}{uccārayet}.
\pend

\pstart\noindent
\Tib{gang zhig sdom pa 'dzin par mi byed pa de la/ deng khyod ces bya ba la sogs pa yang brjod par mi bya ste/ slob dpon du rjes su gnang ba dang/ dbang bskur ba yang mi bya'o// 
'jug pa tsam cig byed du gzhug go//}
\pend

\pstart\noindent
[Skt. Reconstruction] yas tu saṃvaraṃ na gṛhṇāti tasya praveśamātram eva dātavyam.
\edtext{adya tvam ityādi}{\lemma{%
	{\rm adya tvam ityādi\lem}
}\Cfootnote{
%
STTS: 220: 
adya tvaṃ sarvatathāgatakulapraviṣṭaḥ.
tad ahaṃ te vajrajñānam utpādayiṣyāmi, yena jñānena tvaṃ sarvatathāgatasiddhim api prāpsyasi. 
kim utānyāḥ siddhīḥ //.
na ca tvayādṛṣṭamahāmaṇḍalasya vaktavyaṃ, mā te samayo vyathed iti.
%
}} na brūyāt, \edtext{\textcolor{red}{ācāryānujñām}}{\lemma{%
	{\rm ācāryānujñām\lem}
}\Cfootnote{%
	\textit{ācāryo 'nujñām};
	SVU Tib.\ seems to support \textit{ācāryānujñām} (\textit{slob dpon du rjes su gnang ba}).
}} ācāryābhiṣekaṃ ca na kuryāt.
\pend

\pstart\noindent
[SVU Tib.] \Tib{de nas oṃ %{44a4} 
sa rba yo ga tsi t+ta au t+pā da yā mi zhes bya ba 'dis/}
\pend

\verse
\Tib{byang chub sems ni bla med pa// dam pa bdag gis bskyed nas su// \\
de yi snying gar snying po ni// rdo rje rab tu gzhag par bya//\\
su ra te sa ma ya staṃ ho ba dzra sid dhya ya thā su khaṃ/}
%
\pend

\pstart\noindent
[Skt.\ Reconstruction] tataḥ, oṃ sarvayogacittam utpādayāmīty anena
\pend

\verse
\edtext{utpādayitvā paramaṃ bodhicittam anuttaram |\\
vajram asya pratiṣṭhāpyaṃ hṛdaye hṛdayena tu ||\\
surate samayas tvaṃ hoḥ vajra siddhya yathāsukham.}{\lemma{%
	{\rm utpādayitvā \dots\ yathāsukham\lem}
}\Cfootnote{%
	The same set of the verse and the mantra appears in § 3.3.
	\textcolor{red}{CROSS REFERENCE!!}
}}
\pend

\pstart\noindent
[SVU Tib.] \Tib{de nas rje sems dpar byin gyis brlabs la/ dri dang me tog %{44a5} 
la sogs pas mchod de/ me tog gi phreng ba gdags shing/ gdong dri zhim par byas la/ yon gyi mchog blangs la/ des phyi rol na gnas pa'i bum pa'i chus dbang bskur la/ sa ma ya s t+wāṃ zhes bya ba 'dis/ rdo rje sems ma'i phyag rgya 'ching du gzhug go//} 
\pend


\pstart\noindent
[Skt.\ Reconstruction] tatas taṃ vajrasattvam adhiṣṭhāya gandhapuṣpādibhir abhyarcya sragvinaṃ surabhitānanaṃ ca kṛtvottamāṃ dakṣiṇām ādāya bahiḥsthitakalaśodakenābhi\textcolor{red}{ṣicya} samayas tvam ity anena sattvavajrīṃ bandhayet.
\pend

\pstart\noindent
[SVU Tib.] \Tib{\edtext{de nas gung mo gnyis kyis me tog gi phreng ba %{44a6} 
'dzin du bcug la/ sa ma ya hūṃ zhes bya ba 'dis gzhug par bya'o// 
rdo rje lcags kyu la sogs pas kyang/ rang rang gi sgo nas
ba dzra aṃ ku sha dzaḥ/ badzra pā sha hūṃ/ ba dzra s+pho ṭa baṃ/ ba dzra ā be sha aḥ/ 
zhes bya bas dgug pa dang/ gzhug pa dang/ bcing ba dang/ dbang du bya'o//}{\lemma{%
	{\rm de nas \dots\ dbang du bya'o\lem}
}\Bfootnote{%
%
\textit{Kriyāsaṃgrahapañjikā}, chapter 6 (\textit{Abhiṣeka}):
\textit{tato madhyamāṇgulidvayena mālām ādāya praveśayed anena \textemdash\ 
samaya hūṃ.
svadvāre\textcolor{red}{ṇa} vajrāṇkuśādibhiś cākṛṣya praveśya baddhvā vaśīkuryāt.
oṃ vajrāṇkuśa jaḥ,
oṃ vajrapāśa hūṃ,
oṃ vajrasphoṭa vaṃ,
oṃ vajrāveśa \textcolor{red}{aḥ} iti}
(\textcolor{red}{Sakurai's edition!!}).
%
}} } 
\pend

\pstart\noindent
[Skt.\ Reconstruction] tatas madhyamābhyāṃ puṣpamālāṃ grāhayitvā praveśayed anena hṛdayena. samaya hūṃ. svasvadvāre ca vajrāṇkuśādibhiś cākṛṣya praveśya baddhvā vaśīkuryāt. vajrāṅkuśa jaḥ, vajrapāśa hūṃ, vajrasphoṭa vaṃ, vajrāveśa \edtext{\textcolor{red}{hoḥ}}{\lemma{%
	{\rm hoḥ\lem}
}\Cfootnote{%
%
SVU Tib.\ reads \textit{aḥ}. This should be \textit{hoḥ}, since the seed-syllables \textit{jaḥ hūṃ vaṃ hoḥ} are a strong unit, meaning \textit{ākarṣaṇa, praveśana, bandhana}, and \textit{vaśīkaraṇa}. Probably the cause of this corruption lies in the fact that \textit{aḥ} is the syllable for \textit{āveśa} (possesion).
Cf.\ STTS 224: \textit{vajrāveśa aḥ}.
\textcolor{red}{(CHECK KSP ALSO!!)}
%
}} iti.
\pend

\pstart\noindent
\edtext{}{\lemma{%
	{\rm deng \dots\ nyams par gyur ta re\lem}
}\Bfootnote{%
%
SDPT (p.290): 
punaḥ pūrvadvāreṇa praveśyaivaṃ vadet /
*adya tvaṃ (em.; \textit{abhyarcya} ed.) sarvatathāgatakule praviṣṭas tad ahaṃ tu vajrajñānam
utpādayiṣyāmi / yena jñānena sarvatathāgatasiddhir api prāpsyase / kim
anyā siddhiḥ / na ca tvayādṛṣṭamaṇḍalasya purato vaktavyam / mā te
samayo vyathed iti /
%
STTS 220:
tataḥ praveśyaivaṃ vadet /:
adya tvaṃ sarvatathāgatakulapraviṣṭaḥ. 
tad ahaṃ te vajrajñānam utpādayiṣyāmi, yena jñānena tvaṃ sarvatathāgatasiddhim api prāpsyasi. 
kim utānyāḥ siddhīḥ //.
na ca tvayādṛṣṭamahāmaṇḍalasya vaktavyaṃ, mā te samayo vyathed iti //%
%
}}[SVU Tib.] \Tib{de nas yang %{44a7} 
shar phyogs kyi sgo nas/ de bzhin du bcug la/ 'di skad ces brjod par bya ste/ }
\pend
\verse
\Tib{deng khyod de bzhin gshegs pa thams cad kyi rigs su zhugs kyis/ ngas khyod la ye shes gang gis de bzhin gshegs pa thams cad 'grub pa yang thob na/ dngos grub gzhan lta smos kyang \edtext{ci dgos te/}{\lemma{}\Efootnote{%
	ci [44b1] dgos te
}} de lta bu'i rdo rje ye shes bskyed par bya yis/ khyod kyis dkyil 'khor chen po ma mthong ba rnams la smra bar ma byed cig/ khyod dam tshig nyams par gyur ta re/}
\pend

\pstart\noindent
[Skt.\ Reconstruction] \textbf{[STTS 220]}
tataḥ punaḥ pūrvadvāreṇa praveśyaivaṃ vadet.
\pend
\verse
adya tvaṃ sarvatathāgatakulapraviṣṭaḥ.
tad ahaṃ te vajrajñānam utpādayiṣyāmi, yena jñānena tvaṃ sarvatathāgatasiddhim api prāpsyasi.
kim utānyāḥ siddhīḥ.
na ca tvayādṛṣṭamahāmaṇḍalasya vaktavyaṃ, mā te samayo \textcolor{blue}{vyatheti}.
\pend

\pstart\noindent
\edtext{}{\lemma{%
	{\rm de nas \dots\ klad pa 'gas par 'gyur ro\lem}
}\Bfootnote{%
%
SDPT (p.290): 
tataḥ svayaṃ vajrācāryaḥ krodhaterintirīm evaṃ ūrdhvamukhīṃ baddhvā
vajraṃ śiṣyasya mūrdhni sthāpyaivaṃ vadet / ayaṃ te samayavajro mūrdhniṃ
te sphārayed yadi tvaṃ kasyacid brūyāḥ / tatas tayaiva
samayamudrayodakaṃ śayathā hṛdayena satkṛtya parijapya tasmai
vajraśiṣyāya pāyayed iti / tatredaṃ śayathāhṛdayam /
%
STTS 221–222:
tataḥ svayaṃ vajrācāryaḥ sattvavajrimudrām avamūrdhamukhīṃ baddhvā vajraśiṣyasya mūrdhni sthāpyaivaṃ vadet /:
ayaṃ te samayavajro mūrdhānaṃ sphālayed, yadi tvaṃ kasya cid brūyāt /.
tatas tathaiva samayamudrayodakaṃ śapathāhṛdayena sakṛt parijāpya tasmai śiṣyāya pāyayed iti /.
tatredaṃ śapathāhṛdayaṃ bhavati /
%
}}[SVU Tib.] \Tib{de nas slob dpon rang nyid kyis sems ma rdo rje ma'i phyag rgya kha thur du bstan pa dang/ gyen du bstan pa bcings %{44b2} 
la rdo rje slob ma'i spyi bor bzhag la/ 'di skad ces brjod par bya'o//} 
\pend
\verse
\Tib{'di ni khyod kyi rdo rje'i dam tshig yin gyis/ gal te 'ga' zhig la smras na klad pa 'gas par 'gyur ro//} 
\pend

\pstart\noindent
[Skt.\ Reconstruction] \textbf{[STTS 221–222]}
tataḥ svayaṃ vajrācāryaḥ sattvavajrīmudrām avamūrdhamukhīṃ ūrdhvamukhīṃ ca baddhvā vajraśiṣyasya mūrdhni sthāpyaivaṃ vadet.
\pend
\verse
ayaṃ te samayavajro mūrdhānaṃ sphālayed, yadi tvaṃ kasya cid \textcolor{red}{brūyāḥ}.
\pend

\pstart\noindent
\edtext{}{\lemma{%
	{\rm de nas \dots\ ṭha zhes bya'o\lem}
}\Bfootnote{%
%
SDPT (p.290):
vajrasattvaḥ svayaṃ te 'dya hṛdaye samavasthitaḥ //
nirbhidya tat kṣaṇaṃ yāyād yadi brūyā imaṃ nayam //
vajrodaka / iti /
%
STTS 222:
vajrasattvaḥ svayaṃ te 'dya hṛdaye samavasthitaḥ /
nirbhidya tatkṣaṇaṃ yāyād yadi \textcolor{red}{brūyād} imaṃ nayam //
vajrodaka ṭhaḥ //
%
}}[SVU Tib.] \Tib{de nas dam tshig gi phyag rgya de nyid dang/ ma na'i snying po 'di lan cig bzlas pa'i chu rdo rje slob ma 'thung %{44b3} 
du bcug la/ 'di skad ces brjod par bya'o//  
de la dam tshig gi snying po ni 'di yin te/}
\pend
\verse
\Tib{rdo rje sems dpa' deng khyod kyi// snying la yang dag zhugs par mdzad// \\
gal te tshul 'di smras na ni// de ma thag tu dral te gshegs// \\
%
oṃ ba dzra au da ka ṭha} 
\pend
\pstart\noindent
\Tib{zhes bya'o//} 
\pend

\pstart\noindent
[Skt.\ Reconstruction] tatas tathaiva samayamudrayodakaṃ śapathāhṛdayena sakṛt parijāpya tasmai śiṣyāya pāyayed iti.
tatredaṃ \edtext{śapathāhṛdayaṃ}{\lemma{%
	{\rm śapathāhṛdayaṃ\lem}
}\Cfootnote{%
%
SVU Tib.: \textit{dam tshig gi snying po} (\textit{*samayahṛdayam}). Both STTS ans SDPT supports \textit{śapathāhṛdayaṃ}. The reading of the latter, \textit{śayathāhṛdayaṃ} should be emended.
%
}} bhavati.
\pend

\verse
vajrasattvaḥ svayaṃ te 'dya hṛdaye samavasthitaḥ |\\
nirbhidya tatkṣaṇaṃ yāyād yadi \textcolor{red}{brūyād} imaṃ nayam ||\\
oṃ vajrodaka ṭhaḥ.
\pend

\pstart\noindent
\edtext{}{\lemma{%
	{\rm de nas \dots\ ltung bar gyur ta re\lem}
}\Bfootnote{%
SDPT (p.290):
tataḥ śiṣyāya brūyād adya prabhtti te 'haṃ vajrapāṇir
yad ahaṃ brūyām idaṃ kuru tat kartavyaṃ na ca tvayāham avamantavyo mā te
viṣamāparihāreṇa kālakriyāṃ kṛtvā narake patanaṃ syād iti /
%
STTS 223:
tataḥ śiṣyāya brūyāt /:
adyaprabhṛty ahaṃ te vajrapāṇir yat te 'haṃ brūyām: idaṃ kuru, tat kartavyaṃ /
na ca tvayāham avamantavyo mā te viṣamāparihāreṇa kālakriyāṃ kṛtvā, narakapatanaṃ syād
}}[SVU Tib.] \Tib{de nas slob ma la %{44b4} 
'di skad ces brjod par bya ste/} 
\pend
\verse
\Tib{deng nas brtsams te khyod kyi phyag na rdo rje nga yin gyis/ ngas 'di byos shig ces bsgo ba gang yin pa de bya dgos so// 
khyod kyis nga la brnyas par ma byed cig/ mi bde ba ma spangs par khyod dus byas nas dmyal bar ltung bar gyur ta re/} 
\pend

\pstart\noindent
[Skt.\ Reconstruction] \textbf{[STTS 223]} tataḥ śiṣyāya brūyāt.
\pend

\verse
adyaprabhṛty ahaṃ te vajrapāṇir, yat te 'haṃ brūyām: idaṃ kuru, tat kartavyaṃ.
na ca tvayāham avamantavyo mā te viṣamāparihāreṇa kālakriyāṃ kṛtvā, narakapatanaṃ syād iti.
\pend

\bigskip

\pstart\noindent
3.8.1.3. Āveśaḥ
\pend

\medskip

\pstart\noindent
[SVU Tib.] \Tib{\edtext{de nas yi ge %{44b5} 
a zhes bya bas bdag nyid rdo rje sems dpa' sku mdog dkar por bsam par bya'o// 
de'i rjes la rang gi snying gar yi ge a rdo rje'i 'od zer gyi phreng ba dang ldan pa bsam par bya'o// 
yi ge hūṃ gis kyang slob ma rdo rje sems dpa' rdul dang bral zhing shin tu dang bar bsams la/ %{44b6} 
snying ga dang dpral ba dang lce dang spyi bor/ hūṃ hrīṃ hrīḥ kaṃ zhes bya ba las rdo rje dang rin po che dang padma dang sna tshogs rdo rjes mtshan par byas la/ sgo dbye ba'i phyag rgyas rang gi snying ga dang/ slob ma'i snying gar dbye bar bya'o// 
de nas rang gi snying gar gnas pa'i rdo rje a zhes bya bas phyung %{44b7} 
la/ slob ma'i snying gar gnas pa'i rdo rje'i dbus su blos bcug ste/ lus thams cad gang bar gyur bar bsams la/ 'di skad ces brjod par bya'o// 
de bzhin gshegs pa thams cad kyis byin gyis rlobs la rdo rje sems dpa' bdag la dbab par \edtext{gyur cig/ ces}{\lemma{}\Efootnote{%
	gyur cig/ [45a1] ces
}} smros shig/}{\lemma{%
	{\rm de nas yi ge a \dots\ ces smros shig\lem}
}\Bfootnote{%
%
SDPT: tad anu akāraṃ vajraraśmimālāyuktaṃ svahṛdi cintayet. tataḥ śiṣyahṛdūrṇākaṇṭhamūrdhniṣu candramaṇḍala*sthaṃ(em.; \textit{-stha-} ed.) pañcasūcikaṃ jvālāvajraṃ ratnaṃ padmaṃ viśvavajraṃ ca cintayet. ebhir yathākrameṇa *hūṃ (em.; \textit{huṃ} ed.) *traḥ (em.; \textit{traṃ} ed.) hrīḥ aḥ iti.
kapāṭodghāṭanamudrayā svakīyaṃ śiṣyahṛdayaṃ codghāṭya svahṛdayād akāraṃ niścārya śiṣyahṛdgatavajramadhye buddhyā praveśya sarvakāyam āpūryamāṇaṃ cintayet. evaṃ vadet \textemdash\ brūhi “sarvatathāgatāś \textcolor{red}{cādhiṣṭhantāṃ} vajrasattvo me āviśatu.”;\ 
%
KSP:
śiṣyena vācyam.
“sarvatathāgatāś cādhiṣṭhantām. vajrasattvo me(sic) āviśatu.
tata ātmānaṃ śitavarṇaṃ śrīvajrāttvaṃ vicintayed āḥkāreṇa.”\
tad anv āḥkāraṃ raśmimālāyuktaṃ svaṛdaye vicintya, hūṃkāreṇa ca śiṣyaṃ vajrasattvarūpaṃ vicintya, hūṃ traḥ hrīḥ kaṃ iti hṛdūrujihvāmūrdhasu vajraratnapadmaviśvavajrāṇkitaṃ kapāṭodghāṭanamudrayā svakīyaṃ hṛdayam udghāṭyaṃ śiṣyahṛdayaṃ ca.
tataḥ svahṛdgatavajrād aḥkāraṃ niścārya śiṣyahṛdisthavajramadhye buddhyā praveśya sarvakāyam āpūryamāṇaṃ cintayet.
evaṃ vadet.
”brūhi \textemdash\ sarvatathāgatāś cādhiṣṭhantāṃ vajrasattvo me(sic) āviśatu.
%
STTS 223: brūhi sarvatathāgatā adhitiṣṭhantāṃ vajrasattvo ma āviśatu //%
%
}} }
\pend

\pstart\noindent
%[Roman script: reconstruction based upon Tib.\ and SDPT]
%
[Skt.\ Reconstruction] tataḥ akāreṇātmānaṃ sitaṃ vajrasattvaṃ cintayet. tad anu akāraṃ vajraraśmimālāyuktaṃ svahṛdi cintayet. punar hūṃkāreṇa śiṣyaṃ \textcolor{blue}{vimalam} \textcolor{blue}{atiprasannaṃ} vajrasattvaṃ \textcolor{blue}{vi}cintya, taddhṛdūrṇākaṇṭha\textcolor{blue}{mūrdhasu} hūṃhrīṃhrīḥkaṃkāreṇa vajraratnapadmaviśvavajreṇa cihnayitvā, kapāṭodghāṭanamudrayā svakīyaṃ śiṣyahṛdayaṃ codghāṭayet.
%
tataḥ svahṛdgatavajram akāreṇa niścārya śiṣyahṛdgatavajramadhye buddhyā praveśya sarvakāyam āpūryamāṇam cintayet. evaṃ vadet. brūhi “sarvatathāgatāś cādhitiṣṭhantāṃ vajrasattvo me āviśatu.”\ 
\pend

\pstart\noindent
[SVU Tib.] \Tib{\edtext{de nas slob dpon gyis myur du rdo rje sems ma'i phyag rgya bcings la/ de'i snying gar gzhag ste/ 'di skad ces brjod du gzhug go//}{\lemma{%
	{\rm de nas slob dpon gyis \dots\ brjod du gzhug go\lem}
}\Bfootnote{%
%
SDPT:
tatas \textcolor{blue}{*tvaramānena} (\emn; \textit{vartamānena} ed.) vajrācāryeṇa krodhaterintirīṃ baddhvedam uccārayitavyam.
%
STTS 224:
tatas tvaramāṇena vajrācāryeṇa sattvavajrimudrāṃ baddhvedam uccārayitavyam //%
%
}} }
\pend

\pstart\noindent
[Skt.\ Reconstruction]
tatas tvaramāṇena vajrācāryeṇa sattvavajrīṃ baddhvā tasya hṛdaye sthāpya idam uccārayitavyam. 
\pend

\verse
[SVU Tib.] \Tib{\edtext{'di ni rdo rje dam tshig ste// rdo rje sems dpa' yin par grags// \\
rdo rje ye shes bla med ni// de ring %{45a2} 
nyid du 'bab par shog/ \\
ba dzra ā we sha aḥ zhes bya ba}{\lemma{%
	{\rm 'di ni \dots\ aḥ\lem}
}\Bfootnote{%
%
SDPT:
ayaṃ tat samayavajraṃ vajrasattva iti smṛtam |
āveśayatu te 'dyaiva vajrajñānam anuttaram ||
vajrāveśa aḥ iti.  
%
\noindent
STTS 224:
ayaṃ tat samayo vajraṃ vajrasattva-m-iti smṛtam |
āveśayatu te 'dyaiva vajrajñānam anuttaram ||
vajrāveśa aḥ.%
%
}} }
\pend

\verse
[Skt.\ Reconstruction]
ayaṃ tat samayo vajraṃ vajrasattva-m-iti smṛtam |\\
āveśayatu te 'dyaiva vajrajñānam anuttaram ||\\
vajrāveśa aḥ iti.
\pend

\pstart\noindent
[SVU Tib.] \Tib{\edtext{bcu nas lan brgya'i bar du brjod na nges par 'bab par 'gyur ro// 
de nas gal te 'bab par ma gyur na de'i tshe khro po'i khu tshur bcings la/ rdo rje sems ma'i phyag rgya dral bar bya zhing/ \textcolor{red}{ba dzra sa t+wa ā ā ā āḥ} zhes bya ba %{45a3} 
'di yang brjod la/ bcom ldan 'das rdo rje sems dpa' 'od zer dmar po \textcolor{blue}{'bal bas} kun du khyab pas kun du gang bar bsams la/ lan brgya'i bar du bzla bar bya'o//}{\lemma{%
	{\rm bcu nas\dots\ bzla bar bya'o\lem}
}\Bfootnote{%
%
SDPT: 10,  20, 30, 40, 50, 60, 70, 80, 90, 100vārānuccārya niyatam āviśati.
tataḥ krodhamuṣṭiṃ baddhvā sattvavajrīmudrāṃ sphoṭayed idam udīrayet. oṃ sumbhani sumbhani *hūṃ (em.; \textit{huṃ} ed.). oṃ gṛhṇa gṛhṇa *hūṃ (em.; \textit{huṃ} ed.). oṃ gṛhṇāpaya gṛhṇāpaya *hūṃ (em.; \textit{huṃ} ed.). oṃ ānaya hoḥ bhagavan vajrarāja *hūṃ (em.; \textit{huṃ} ed.) phaṭ. aḥ aḥ aḥ aḥ. 10, 20, 30, 40, 50, 60, 70, 80, 90 śatavārān uccārayet.
bhagavatā ca vajravātamaṇḍalyā ca *vajrahūmkāreṇa (em. \textit{vajrahuṃkāreṇa} ed.) raktavarṇajvālāprabheṇāpūryamāṇaṃ cintayet.%
%
Bhūtaḍāmara manual:
\textit{-ntena raktajvālākulaprabheṇāpūryamāṇa[ṃ] cintayitvā hūṃ aḥ iti bahuśo japet |} (f.9r).
%
}} } 
\pend

\pstart\noindent
[Skt.\ Reconstruction]
\textcolor{blue}{daśa vārān yāvac chata vārān} uccārya niyatam āviśati. \textcolor{blue}{tato} yady āveśo na bhavati tadā krodhamuṣṭiṃ baddhvā sattvavajrīṃ sphoṭayet. \textcolor{red}{vajrasattva aḥ aḥ aḥ aḥ} iti coccārya, \textcolor{blue}{bhagavatā} vajrasattvena raktajvālākulāpūryamāṇaṃ cintayet. yāvac \textcolor{blue}{chata} vārān uccārayet.
\pend

\medskip

\pstart\noindent
\Skt{\edtext{\supplied{evamapi yadyāveśo na bhavati tato ghaṇṭāsahitāṃ vajrā}%\newfolio{60r1}%
\edtext{veśasamayamudrāṃ}{\lemma{}\Efootnote{%
	(foliomissing) [60r1] veśasamayamudrāṁ
}} baddhvā}{\lemma{%
	{\rm evam api \dots\ baddhvā\lem}
}\Bfootnote{%
%
SVU Tib.: de lta bus kyang gal te 'bab par ma gyur na/ de'i tshe rdo rje 'bebs pa'i dam tshig gi phyag rgya dril bu dang bcas pa %{45a4} 
bcings la;
%
SDPT (p.292) punar api yady āveśo na bhavati/ tato ghaṇṭāsahitāṃ vajrāveśasamayamudrāṃ baddhvā%
%
}} \edtext{}{\lemma{%
	{\rm vāmapādena \dots\ tathaivāvartayet\lem}
}\Bfootnote{%
%
SDPT (p.292):
vāmapādena tasya dakṣiṇapādam ākraṃyopary
ākāśadeśe vairocanaṃ śrīvajrahuṃkārasyopari tasyaivāveśanāya
kruddhahuṃkāraraśmisamūhenākramyamāṇam adhastāc ca vajravātamandalyā
huṃkāreṇotthāpyamānam evaṃ pūrvādidiksthitair akṣobhyādibhiḥ / huṃ trāṃ
hrīḥ iti svabījaraśmivyūhaiḥ saṃpātyamānaṃ cintayaṃs tam āveśayet / huṃ
vajrāveśa aḥ \textcolor{red}{śatadhoccārayet} /
%
}}vāmapādena tasya dakṣiṇapāda\edtext{mākramyoparyā}{\lemma{%
	{\rm ākramyopary\lem}
}\Dfootnote{%
	\emn\ \MSK;
	\textit{ākramyamyopay} \cod
}}\edtext{kāśadeśe}{\lemma{
	{\rm ākāśadeśe\lem}
}\Dfootnote{
	\emn\ \MSK\ \sil;
	\textit{kāśadese} \cod
}} vairocanaṃ \edtext{śrīvajrasattvasyopari}{\lemma{%
	{\rm śrīvajrasattvasyopari\lem}
}\Dfootnote{%
	\emn\ \MSK;
	\textit{śrīvajrasatvosyopari} \cod
}} \edtext{tasyaivāveśanāya}{\lemma{%
	{\rm tasyaivāveśanāya\lem}
}\Cfootnote{%
	possibly the ms. shows \textit{tasyaivāveśanāyaṁ}, but the \textit{anusvāra} is unclear and anyhow unwanted.
}} \edtext{kruddhahūṃkāra}{\lemma{%
	{\rm kruddhahūṁkāra-\lem}
}\Dfootnote{%
	\emn\ \MSK;
	\textit{kruddhaṁhū;ṁkāra} \cod
}}raśmisa\edtext{mūhenā}{\lemma{%
	{\rm -mūhenā-\lem}
}\Dfootnote{%
	\emn\ \MSK;
	\textit{-mūkenā-} \cod;
	\MSK\ wrongly reports a \textit{-mūmena-} as reading.
}}kramya\-māṇamadhastācca vajravātamaṇḍalyā  hūṃkāreṇotthāpyamānam 
evaṃ pūrvādi\edtext{diksthitai}{\lemma{%
	{\rm -diksthitair\lem}
}\Dfootnote{
	\emn\ \MSK;
	\textit{-disthitair} \cod
}}\edtext{rakṣobhyādibhiḥ}{\lemma{%
	{\rm akṣobhyādibhiḥ\lem}
}\Dfootnote{%
	\emn\ \MSK;
	\textit{akṣobhyādiḥ} \cod
}} hū{\cb} \edtext{traḥ}{\lemma{%
	{\rm traḥ\lem}
}\Cfootnote{%
	\emn\ \textit{trāḥ}, with \MSK?
}} hrīḥ {\ah}}
     %cf. §49
    %ākramyamyopay: em. ākramyopary, with MSK.
    %ākāśadese: em. ākāśadeśe, with MSK (sil. em.).
    %śrīvajrasatvosyopari: em. śrīvajrasattvasyopari, with MSK.
    %tasyaivāveśanāya: possibly the ms. shows tasyaivāveśanāyaṁ, but the anusvāra is unclear and anyhow unwanted.
    %kruddhaṁhū;ṁkāra-: em. kruddhahū;ṁkāra-, with MSK.
    %-mūkenā-: em. -mūhenā-, with MSK.; MSL wrongly reports a -mūmena- as reading.
    %-disthitair: em. -diksthitair, with MSK.
    %akṣobhyādiḥ: em. akṣobhyādibhiḥ, with MSK.
    %traḥ: em. trāḥ, with MSK?
    % raśmisa\linebreak{60r2}mūkenā
%
%
\Skt{ityebhiḥ svabījaraśmivyūhaiḥ saṃpātyamānaṃ \edtext{cintayaṃsta}{\lemma{%
	{\rm cintayaṁs tam\lem}
}\Dfootnote{
	\emn\ \MSK;
	\textit{cintayens tam} \cod;
	\MSK\ reads \textit{cintayet stam}.
}}māveśayet / vajrasattva {\ah} {\ah} {\ah} {\ah} \edtext{iti tathaivāvartayet}{\lemma{%
	{\rm iti tathaivāvartayet\lem}
}\Dfootnote{%
	\emn\ \MSK\ \sil;
	\textit{i tathaivāvarttayet} \cod
}} /} 
\pend

\pstart
\Skt{\edtext{atha \edtext{pāpabahutvā}{\lemma{%
	{\rm pāpabahutvād\lem}
}\Cfootnote{%
%
\textcolor{blue}{Emend pāpabahulatvād}?
%
}}\edtext{dāveśo}{\lemma{%
	{\rm āveśo\lem}
}\Dfootnote{%
	\emn\ \MSK\ \sil;
	\textit{āveso} \cod
}} na bhavati punaḥ pāpasphoṭanamudrayā tasya \edtext{punaḥ punaḥ}{\lemma{%
	{\rm punaḥ punaḥ\lem}
}\Dfootnote{%
	\emn\ \MSK\ \sil;
	\textit{punaḥ puna} \cod
}} pāpāni \edtext{sphoṭavyāni}{\lemma{%
	{\rm sphoṭavyāni\lem}
}\Dfootnote{%
	\emn\ \MSK\ \sil;
	\textit{sphoṭavyā} \cod
}} /}{\lemma{%
	{\rm atha pāpabahutvād \dots\ kṣaṇād eva bhavati\lem}
}\Bfootnote{%
%
SDPT (pp.292–294):
atha pāpabahutvād āveśo na bhavati / tadā pāpasphoṭanamudrayā  tasya pāpāni sphoṭayet / tataḥ /
%
samidbhir madhurair agniṃ prajvālya susamāhitaḥ /
nirdahet sarvapāpāni tilahomena tasya tu //
%
oṃ sarvapāpadahanavajrāya svāhā / iti dakṣiṇahastatale kṛṣṇatilaiḥ
pāpapratikṛtiṃ kṛtvā huṃkāraṃ madhye vicintya / tarjanyaṃguṣṭhābhyāṃ
homayet / tato homakuṇḍān nirgatya jvālākulair vajrais tasya śarīre
pāpaṃ dahyamānaṃ cintayet / tataḥ punar vajrāveśaṃ tathaivaṃ
baddhvāveśayet / niyatam āviśati / evam api yasyāveśo na bhavati
tasyābhiṣekaṃ na kuryād iti / āviṣṭasya ca pañcābhijñādiniṣpattis tat
kṣaṇād eva bhavati /
%
}} }
\pend

\verse
\Skt{\edtext{\edtext{samidbhi}{\lemma{%
	{\rm samidbhir\lem}
}\Dfootnote{%
	\emn\ \MSK\ \sil;
	\textit{samidgi} \cod
}}\edtext{rmadhurai}{\lemma{%
	{\rm madhurair\lem}
}\Dfootnote{%
	\emn\ \MSK;
	\textit{mudhurair} \cod
}}ragniṃ prajvālya susamāhitaḥ /\\ 
nirdahetsarva\edtext{pāpāni}{\lemma{%
	{\rm -pāpāni\lem}
}\Dfootnote{%
	\emn\ \MSK;
	\textit{-pāpāṇi} \cod
}} tilahomena tasya tu //\\
{\om} sarvapāpadahanavajrāya svāheti //}{\lemma{%
	{\rm samidbhir \dots\ svāheti\lem}
}\Bfootnote{%
%
\textcolor{blue}{\textit{Vajraśekhara}:
\textit{bud shing mngar bdag gi me ||
mnyam par gzhag pas rab spar nas||
til mar gyis ni sbyin sreg byas ||
sdig pa thams cad sreg par 'gyur ||
oṃ sa rba pā paṃ da ha na ba dzra ya svāhā ||}
(D f.269r2–3);
\textit{Pāda}s ab = STTS 1140ab:
\textit{samidbhir madhurair agniṃ prajvālya susamāhitaḥ |
vajrakrodhasamāpattyā tilāṃ hutvā aghān dahet ||}.
The heart mantra (\textit{hṛdaya}) is taught to be \textit{oṃ sarvapāpadahanavajrāya svāha} (STTS1144).}
%
}} }
	%cintayens tam: em. cintayaṁs tam, with MSK; MSK reads cintayet stam.
	%°i tathaivāvarttayet·: em. °iti tathaivāvartayet·, with MSK (sil. em.).
	%āveso: em. āveśo, with MSK (sil. em.).
	%punaḥ puna: em. punaḥ punaḥ, with MSK (sil. em.).
	%sphoṭavyā: em. sphoṭavyāni, with MSK.
	%samidgi: em. samidbhir, with MSK (sil. em.).
	%mudhurair: em. madhurair, with MSK.
	%-pāpāṇi: em. -pāpāni, with MSK.
	%saṁ\linebreak{60r3}pātyamānaṁ
	%tasya § \linebreak{60r4} punaḥ
\pend

\pstart\noindent
\Skt{dakṣiṇahastatale kṛṣṇatilaiḥ pāpapratikṛtiṃ kṛtvā \edtext{hūṃkāraṃ madhye}{\lemma{%
	{\rm hū;ṁkāraṁ madhye\lem}
}\Dfootnote{%
	\emn\ with SDPT;
	\textit{hū;ṁkāramadhyaṁ} \cod
}} vicintya tarjanyaṅguṣṭhābhyāṃ homaṃ kuryāt / tato homakuṇḍānnirgatya \edtext{jvālāmālākulairvajrai}{\lemma{%
	{\rm jvālāmālākulair vajrais\lem}
}\Dfootnote{%
	\emn\ \MSK;
	\textit{jvālāmālākulerdvajres} \cod;
	MSK reads \textit{jvālāmālākuled vajrais}. It is possible that the scribe has either written superscript part of \textit{-ai} on top of \textit{repha}, or converted a \textit{repha} into such a superscript part of \textit{-ai}. In the latter case, ours is the AC reading, and MSK's the PC reading.
}}stasya śarīre \edtext{pāpaṃ}{\lemma{%
	{\rm pāpaṁ\lem}
}\Dfootnote{%
	\emn\ \MSK;
	\textit{pāpāṁ} \cod
}} dahyamānaṃ cintayet \textcolor{blue}{/} niyatamāviśati / evamapi \edtext{yasyāveśo}{\lemma{%
	{\rm yasyāveśo\lem}
}\Dfootnote{%
	\emn\ \MSK\ \sil;
	\textit{yasyāveso} \cod
}} na bhavati tasyābhiṣekaṃ na kuryāditi //}
\pend 

\pstart
\Skt{\edtext{āviṣṭasya}{\lemma{%
	{\rm āviṣṭasya\lem}
}\Dfootnote{%
	\emn\ \MSK;
	\textit{āveṣṭhasya} \cod
}} ca pañcābhijñādiniṣpatti%\newfolio{60v1} 
\edtext{\textcolor{blue}{statkṣaṇādeva}}{\lemma{%
	{\rm tatkṣaṇād\lem}
}\Dfootnote{%
	\emn;
	\textit{tataḥ kṣaṇād} \cod\ \MSK
}\lemma{}\Efootnote{%
	niṣpattis [60v1] tat-
}} \edtext{bhavati}{\lemma{%
	{\rm bhavati\lem}
}\Dfootnote{%
	\emn\ \MSK\ \sil;
	\textit{bhavanti} \cod
}} //}
	%hū;ṁkāramadhyaṁ: em. hū;ṁkāraṁ madhye, with SDPT.
	%kuyāt·: em. kuryāt·, with MSK.
	%jvālāmālākulerdvajres: em. jvālāmālākulair vajrais, with MSK; MSK reads jvālāmālākuled vajrais. It is possible that the scribe has either written superscript part of -ai on top of repha, or converted a repha into such a superscript part of -ai. In the latter case, ours is the AC reading, and MSK's the PC reading.
	%pāpāṁ: em. pāpaṁ, with MSK.
	%yasyāveso: em. yasyāveśo, with MSK (sil em.).
	%kuyād: em. kuryād, with MSK (sil. em.). 
	%°āveṣṭhasya: em. °āviṣṭasya, with MSK.
	%bhavanti: em. bhavati, with MSK (sil. em.).
	%\linebreak{60r5} dakṣiṇahastatale
	%ta\linebreak{60r6}sy śarīre
\pend

\pstart
\Skt{\edtext{tataḥ samāviṣṭaṃ jñātvācāryeṇa he vajrasattva he vajraratna he vajradha\supplied{rma} \supplied{he} vajrakarmeti vajrasattvasamayamudrāṃ baddhvoccāraṇīyam \textcolor{blue}{/} punaḥ nṛtya sattva nṛtya vajreti / sace\edtext{dāviṣṭaḥ}{\lemma{%
	{\rm āviṣṭaḥ\lem}
}\Dfootnote{%
	\emn\ \MSK;
	\textit{ādiṣṭhaḥ} \cod
}} \edtext{śrīvajrasattvamudrāṃ}{\lemma{%
	{\rm śrīvajrasattvamudrāṁ\lem}
}\Dfootnote{%
	\emn\ \MSK;
	\textit{śrīvajrasatvamudrā} \cod
}} badhnīyāt / tadācāryeṇa \edtext{vajramuṣṭimudropadarśanīyā}{\lemma{%
	{\rm vajramuṣṭimudropadarśanīyā\lem}
}\Dfootnote{
	\emn\ \MSK;
	\textit{vajramuṣṭhimudrāpadarśanīyāḥ} \cod
}\Cfootnote{%
%	\MSK\ emends to \textit{vajramuṣṭimudropadarśanīyā}. 
	We follow the emendation by \MSK. But there might be a serious corruption in this part if we can trust the Tibetan translation. Tib.\ adds the following after \textit{-padarśanīyā}: \textit{de bzhin du rdo rje bzhad pa'i phyag rgya 'ching du bcug la | de'i tshe rdo rje chos kyi phyag rgya nye bar bstan pa'i bar du'o ||} (\textit{*evaṁ vajrahāsamudrāṁ baddhvā tadā yāvad vajradharmamudropadeśam}). This in turn may be compared with the SDPT, leading to the conclusion that there is a haprographical error in the relevant part. The original reading might then be as follows: \textit{tadācāryeṇa vajramuṣṭiṁ baddhvā evaṁ vajrahāsamudrāṁ baddhvā tadā yāvad vajradharmamudropadarśanīyā}.
}} / \edtext{evaṃ sarve śrīvajrasattvādayaḥ \edtext{sāṃnidhyaṃ}{\lemma{%
	{\rm sāṁnidhyaṁ\lem}
}\Dfootnote{%
	\emn;
	\textit{sannidhyan} \cod;
	\MSK\ emends \textit{saṁnidhiṁ}, but this is a wrong emendation.
}} kalpayanti}{\lemma{%
	{\rm evaṁ \dots\ kalpayanti\lem}
}\Bfootnote{%
	STTS 892–893:
	\textit{vajrakrodhaterittirimudrāṃ svahṛdayaṃ yathāvat sthāpya, vajrāṅkuśādibhiḥ karmāṇi kṛtvā, 
	punaḥ sarvasamayamudrāṃ bandhayet. tataḥ sarve sāṃnidhyaṃ kalpayanti.}
}} / tato'bhipretavastu pṛcchedanena / jihvāyāṃ \edtext{tasyāviṣṭasya}{\lemma{%
	{\rm tasyāviṣṭasya\lem}
}\Dfootnote{%
	\emn\ \MSK; 
	\textit{tasyāviṣṭasyāviṣṭasya} \cod
}} vajraṃ vicintya brūhi vajreti vaktavyam / tataḥ sarvaṃ vadati //}{\lemma{%
	{\rm tataḥ samāviṣṭaṁ \dots vadati\lem}
}\Bfootnote{%
	SDPT 294:14-15:
	\textit{tataḥ samāviṣṭaṃ jñātvā punaḥ oṃ *vajrasattva}(\emn; \textit{vajrasattvasattva-} ed.)\textit{saṃgrahādigītim uccārya krodhamuṣṭyā tathaiva sattvavajrīmudrāṃ sphoṭayet. *saced} (\emn; \textit{sa ced} ed.) \textit{āviṣṭo vajrasattvakrodhamudrāṃ badhnīyāt, tadācāryeṇa vajramuṣṭiṃ krodhamudrāṃ badhniyāt. evaṃ yāvat sacet vajrahāsamudrāṃ badhnīyāt. tadā vajradharmakrodhamudrāṃ bandhayed ity evaṃ sāṃnidhyaṃ kalpayanti. tatas tasya jihvāyāṃ vajraṃ vicintya brūhi vajreti vaktavyam. tataḥ sarvaṃ kathayati}.
}} }
	%SDPT 294:14-15
	%jñātvācāreṇa: em. jñātvācāryeṇa, with MSK (sil. em.).
	%ādiṣṭhaḥ: em. āviṣṭaḥ, with MSK.
	%śrīvajrasatvamudrā: em. śrīvajrasattvamudrām, with MSK.
	%tadācāyeṇa: em. tadācāryeṇa, with MSK (sil. em.).
	%vajramuṣṭhimudrāpadarśanīyāḥ: MSK emends to vajramuṣṭimudropadarśanīyā. But there must be a serious corruption in thos part. MSK reports that Tib. adds the following after -padarśanīyā:	de bzhin du rdo rje bzhad pa'i phyag rgya 'ching du bcug la | de'i tshe rdo rje chos kyi phyag rgya nye bar bstan pa'i bar du'o || (*evaṁ vajrahāsamudrāṁ badhvā tadā yāvad vajradharmamudropadeśam). A close parallel is found in the SDPT. Compared with the parallel in the SDPT, probably there is a haprographical error in the relevant part. The original reading might be as follows: tadācāryeṇa vajramuṣṭhim badhvā evaṁ vajrahāsamudrām badhvā tadā yāvad vajradharmamudropadarśanīyā.
	% sannidhyan: em. sāṁnidhyaṁ; MSK em. saṁnidhiṁ, but this is a wrong emendation.
	%tasyāviṣṭasyāviṣṭasya: MSK emends tasyāviṣṭasya. But we suspect no emendation should be made here.
\pend

\medskip

\pstart\noindent
3.8.1.4. Puṣpapātaḥ
\pend

\medskip

\pstart
\Skt{\edtext{tatastāṃ mālāṃ mahāmaṇḍale kṣepayet / pratīccha vajra hoḥ iti / tato yatra patati \edtext{so'sya}{\lemma{%
	{\rm so 'sya\lem}
}\Dfootnote{%
	\emn\ \MSK\ \sil;
	\textit{somya} \cod
}} sidhyati / tatastāṃ mālāṃ tasyaiva śirasi bandhayet / {\om} pratigṛhṇa tvamimaṃ \edtext{sattvaṃ}{\lemma{%
	{\rm sattvaṁ\lem}
}\Cfootnote{%
	\MSK\ emends \textit{sattva}, but we do not understand why.
}} mahābaleti /}{\lemma{%
	{\rm tatas tām mālāṁ \dots\ mahābaleti\lem}
}\Bfootnote{%
	STTS 228–229:
	tatas tāṃ mālāṁ mahāmaṇḍale kṣepaed anena hṛdayena.
	pratīccha vajra hoḥ.
	tato yatra patati, so 'sya sidhyati.
	tatas tāṁ mālāṁ gṛhya, tasyaiva śirasi bandhayed anena hṛdayena.
	oṃ pratigṛhṇa tvam imaṃ sattvaṃ *mahābala (\emn; \textit{mahābalaḥ} ed.).
	tayā baddhayā tena mahāsattvena pratīcchito bhavati.
	śīghraṃ cāsya sidhyati;\ 
	SDPT p.294, ll.18–21:
	\textit{tatas tāṁ mālāṁ mahāmaṇḍale kṣepayet / praticcha vajra hoḥ / iti / 
	tato yatra patati so 'sya siddhyati / tatas tāṁ mālāṁ tasyaiva śirasi bandhayet /
	oṃ pratigṛhṇa tvam *imāṁ (\emn\ imaṁ?) vajrasattva mahābala / iti /}%
}} }
	%somya: em. so 'sya, with MSK (sil. em.).
	%satvaṁ:  MSK emends sattva, but we do not understand why.
\pend

\medskip

\pstart\noindent
3.8.1.5. Maṇḍaladarśanam
\pend

\medskip

\pstart
\Skt{tato mukhabandhaṃ muñcedanena /} 
\pend

\verse
\Skt{\edtext{\om\ vajrasattvaḥ svayaṃ te'dya \edtext{cakṣūdghāṭanatatparaḥ}{\lemma{%
	{\rm cakṣūdghāṭanatatparaḥ\lem}
}\Dfootnote{%
	\textit{cakṣutghāṭanatatparaḥ} \cod;
	\MSK\ silently emends \textit{cakṣūdghāṭanatatparaḥ}, which is metrically correct and conforms with reading of STTS. But several texts (\textit{Vajrāvalī} etc.) have 'corrected' the mantra to read \textit{cakṣurudghāṭanatatparaḥ}.
}} /\\
\edtext{udghāṭayati}{\lemma{%
	{\rm udghāṭayati\lem}
}\Dfootnote{%
	\textit{utghāṭayati} \cod
}} sarvākṣo vajracakṣuranuttaram //\\ he vajra \edtext{paśyeti}{\lemma{%
	{\rm paśyeti\lem}
}\Dfootnote{%
	\emn\ \MSK\ \sil;
	\textit{pasyeti} \cod
}} /}{\lemma{%
	{\rm tato \dots\ paśyeti\lem}
}\Bfootnote{%
	SDPT p.294, ll.22–25:
	\textit{tato mukhabandhaṃ muñced anena /
	oṃ vajrasattvaḥ svayaṃ te 'dya cakṣūdghāṭanatatparaḥ //
	udghāṭayati sarvākṣo vajracakṣur anuttaram // iti / he vajra paśya / iti /}
}} }
	%cakṣutghāṭanatatparaḥ: MSK silently emends cakṣūdghāṭanatatparaḥ, which is metrically correct and conforms with reading of STTS. But several texts (Vajrāvalī etc.) have 'corrected' the mantra to read cakṣurudghāṭanatatparaḥ. 
	%utghāṭayati: em. udghāṭayati, with MSK (sil. em.).
	%pasyeti: em. paśyeti, with MSK (sil. em.).
\pend

\pstart
\Skt{\edtext{tato mahāmaṇḍalaṃ vajrāṅkuśā\edtext{dārabhyāvairocanapa\textcolor{red}{rya}ntaṃ}{\lemma{%
	{\rm ārabhyāvairocanaparyantaṁ\lem}
}\Cfootnote{%
	\MSK\ emends \textit{ārabhya vairocanaparyantaṁ}.
}} darśayet //}{\lemma{%
	{\rm tato \dots\ darśayet\lem}
}\Bfootnote{
%
SDPT 294: 
\textit{tato mahāmaṇḍalavajrāṃkuśād ārabhya yāvad vairocanaparyantaṃ darśayet}
%
}} } 
\pend

\medskip

\pstart\noindent
\textbf{\textcolor{blue}{New Section!!}}
\pend

\medskip

\pstart
\Skt{\edtext{tatastiṣṭha vajretyādinā praveśamudrāmokṣaṃ śiṣyahṛdaye kārayet / tato bāhyamaṇḍalābhyantare \edtext{pūrvadvārābhi\edtext{mukhaṃ}{\lemma{%
	{\rm -mukhaṁ\lem}
}\Dfootnote{%
	\emn\ \MSK; %emends \textit{-mukhaṁ}, with SDPT. Another possibility might be to emend \textit{-mukhī}.
	\textit{-mukhīṁ} \cod
}}}{\lemma{}\Efootnote{%
	pūrvadvārā[61r1]bhimukhīṁ
}} bāhyato vā candramaṇḍalamālikhya \edtext{śiṣyaṃ}{\lemma{%
	{\rm śiṣyaṁ\lem}
}\Dfootnote{%
	\emn\ \MSK;
	\textit{śiṣya} \cod
}} sattvavajrādimudrācatuṣṭayena svasamayamudrayā ca śrīvajrasattvādirūpamadhiṣṭhāya / svamahāmudrayā candramaṇḍale \edtext{pratiṣṭhāpyābhiṣiñce}{\lemma{%
	{\rm pratiṣṭhāpyābhiṣiñced\lem}
}\Dfootnote{%
	\emn\ \MSK\ \sil;
	\textit{pratiṣṭhāpyābhiṣiñceṅ} \cod
}}dgandhapuṣpādibhirabhya\textcolor{red}{rcyā}rghaṃ dattvā / chatradhvajapatākādibhi\edtext{stūryaśaṅkha\-nirnāditai}{\lemma{%
	{\rm tūryaśaṅkhanirnāditaiś\lem}
}\Cfootnote{%
	\MSK\ reads \textit{-śaṅkhā-}, emending it to \textit{-śaṅkha-}, but the latter is clearly the reading of the ms.; \MSK\ ignores the presence of \textit{repha} in \textit{-nirnāditaiś}. \textit{nirnādita} is recorded in BHSD; but \textit{ninādita}, as in SDPT may be better.
}}śca~/ tato \edtext{maṅgala}{\lemma{%
	{\rm maṅgala-\lem}
}\Cfootnote{%
	the \textit{-ṅ-} lacks its hook.
}}gāthābhirabhinandya /}{\lemma{%
	{\rm tato \dots\ -nirnāditaiś ca\lem}
}\Bfootnote{%
	SDPT: 294–296 \textcolor{red}{LINE NUMBERS!!}:
	\textit{tatas tiṣṭha vajretyādinā *śiṣyaṁ} (\emn; \textit{śiṣya-} ed.) \textit{praveśamudrāṃ mokṣayet / tato bāhyamaṇḍalābhyantare candramaṇḍalaṃ pūrvadvārābhimukhaṃ saṃlikhya bāhyato vā śiṣyaṃ śrīvajrahuṃkāramudrayā sattvavajrādibhiś cādhiṣṭhāya mahāmudrayā tataḥ pratiṣṭhāpyābhiṣiñcet / gandhapuṣpādibhir abhyarcyārghaṃ datvā / chatradhvajapatākādibhis turyaśaṃkhanināditaiś ca / tato maṅgalagāthābhir abhinandya}
}} }
	%ārabhyāvairocanapayantaṁ: MSK emends ārabhya vairocanaparyantaṁ (silently for the missing repha). We emend only -paryantaṁ.
	%-mukhīṁ: MSK emends -mukhaṁ, with SDPT. Another possibility might be to emend -mukhī.
	%śiṣya: em. śiṣyaṁ, with MSK.
	%pratiṣṭhāpyābhiṣiñceṅ: em. pratiṣṭhāpyābhiṣiñced, with MSK (sil. em.) and SDPT.
	%abhyaccyārghan: em. abhyarcyārghaṁ, with MSK (sil. em.).
	%tūyaśaṅkhanirnāditaiś: em. tūryaśaṅkhanirnāditaiś; MSK reads -śaṅkhā-, emending it to -śaṅkha-, but the latter is clearly the reading of the ms.; MSK ignores the absence of repha in tūya- and its presence in -nirnāditaiś. nirnādita is recorded in BHSD; but ninādita, as in SDPT may be better.
\pend

\medskip

\pstart\noindent
3.8.1.6. Ṣaḍvidhābhiṣekaḥ
\pend

\medskip

\pstart
\Skt{\edtext{ādau tāva\edtext{dudakābhiṣekeṇa}{\lemma{%
	{\rm udakābhiṣekeṇa\lem}
}\Cfootnote{%
	Tib.\ adds \textit{mudrābhiṣekeṇa} after this.
}} makuṭapaṭṭavajrādhipatināmābhiṣekai\edtext{ścābhiṣicya}{\lemma{%
	{\rm cābhiṣicya\lem}
}\Cfootnote{%
	\MSK\ needlessly emends \textit{cābhiṣiñcya}.
}}~/ punaḥ \edtext{puṣpādibhi}{\lemma{%
	{\rm puṣpādibhir\lem}
}\Dfootnote{%
	\emn\ \MSK\ \sil;
	\textit{puṣpādibhi} \cod
}}rlāsyādyaṣṭavidha\edtext{pūjayā}{\lemma{%
	{\rm -pūjayā\lem}
}\Dfootnote{%
	\emn\ \MSK;
	\textit{pūjāyā} \cod
}} ca \edtext{saṃpūjayet //}{\lemma{%
	{\rm saṃpūjayet ||\lem}
}\Dfootnote{%
	\emn\ with SDPT;
	\textit{sampūjyaye ||} \cod;
	\MSK\ emends \textit{saṁpūjya}. %After this, a new paragraph starts, with Ānandagarbha prescribing the manners of a \textit{śiṣya}.
}}}{\lemma{%
	{\rm ādau \dots\ vyākuryāditi\lem}
}\Bfootnote{%
	SDPT 296.\textcolor{red}{LINE NUMBERS!!}:
	\textit{ādau tāvad udakābhiṣekena tato mudrābhiṣekena mukuṭapaṭṭavajrādhipatināmābhiṣekaiś cābhiṣiñcet / 
	punaḥ puṣpādibhir lāsyādyaṣṭavidhapūjayā ca pūjayet / 
	śiṣyenācāryaṃ valitavajrāṃjalinā praṇamyottamāṃ dakṣiṇāṃ datvā puṣpādyabhiṣekāś ca grāhyā iti / 
	ācāryābhiṣekaṃ tu śrīvajrahuṃkāramudrayā tathaiva pratiṣṭhāpya yathā nirdiṣṭeṣu sthāneṣu samayamudrābhis tasya kāye śrīvajrahuṃkārādīn nyasya / punar api anenāṣṭottaraśatasahasraparijaptaṃ vijayakalaśaṃ kṛtvā / 
	oṃ vajrādhipati tvām abhiṣiñcāmi dṛḍho me bhava jaḥ huṃ vaṃ hoḥ huṃ phaṭ iti / tata imaṃ codīrayan / oṃ vajrābhiṣiñca / iti codakābhiṣekaṃ vajramuṣṭinodakaṃ vijayakalaśād gihītvā dadyād idaṃ ca brūyāt //
idaṃ te nārakaṃ vāri samayātikramād dahet //
samayābhirakṣāt siddhiḥ siddhaṃ vajrāmṛtodakam //
vajraghaṇṭāṃ ca mudrāṃ ca yady amaṇḍalino vadet //
hased vāśraddhadānena janasaṃgaṇikāsthitaḥ // iti /
tataḥ sarvavidhim anuṣṭhāya nāmāṣṭaśatena saṃstutya gāthāpañcakenānujñāṃ datvodgatavyākaraṇena sarvaśiṣyān sarvaśiṣyān vyākuryād iti /}
}} }
\pend

\pstart
\Skt{śiṣyeṇa \edtext{cottamāṃ}{\lemma{%
	{\rm cottamāṁ\lem}
}\Dfootnote{%
	\emn\ \MSK;
	\textit{cottamān} \cod
}} dakṣiṇāṃ dattvācāryaṃ \edtext{\textcolor{blue}{praṇamya}}{\lemma{%
	{\rm praṇamya\lem}
}\Dfootnote{
	\emn\ (supported by Tib.);
	\textit{praṇamya praṇamya} \cod\ \MSK.
	This could be a dittography.
}} valitavajrāñjalinā \edtext{puṣpādikamabhiṣekāśca grāhyāḥ}{\lemma{%
	{\rm puṣpādikam abhiṣekāś ca grāhyāḥ\lem}
}\Dfootnote{
	\emn;
	\textit{puṣpādikaṁm abhiṣekāś ca grāhyaḥ} \cod;
	Would \textit{puṣpādikam abhiṣekāya grāhyam} be another possibility? \MSK\ emends \textit{puṣpādikābhiṣekāś ca grāhyāḥ}; SDPT reads \textit{puṣpādyabhiṣekāś ca grāhyā iti}. 
}} / ācāryābhiṣekaṃ tu śrīvajrasattvamahāmudrayā pratiṣṭhāpya / yathānirdiṣṭeṣu sthāneṣu samayamudrābhistasya kāye \edtext{\textcolor{blue}{śrīvajrasattvādīn}}{\lemma{%
	{\rm śrīvajrasattvādīn\lem}
}\Dfootnote{%
	\emn;
	\textit{śrīvajrasattvādiṃ} \cod\ \MSK
}} nyasya śrīvajrasattvapratimāṃ ca śirasi pratiṣṭhāpyodakābhiṣekaṃ \edtext{dattvemaṃ}{\lemma{%
	{\rm dattvemaṁ\lem}
}\Dfootnote{%
	\emn\ \MSK\ \sil;
	\textit{datvaimaṁ} \cod
}} mantramaṣṭottaraśata\edtext{vārān}{\lemma{%
	{\rm -vārān\lem}
}\Dfootnote{%
	\emn;
	\textit{-vārām} \cod;
	\MSK\ emends \textit{-vāram}.
}} āvartayet / {\om} mahāsukha vajrasattva jaḥ hū{\cb} va{\cb} hoḥ \edtext{suratastva}{\lemma{%
	{\rm suratas tvam\lem}
}\Dfootnote{%
	\emn\ \MSK;
	\textit{surata tvam} \cod
}}miti // tataḥ sarvavidhimanuṣṭhāya \edtext{śiṣyaṃ}{\lemma{%
	{\rm śiṣyaṁ\lem}
}\Dfootnote{%
	\emn;
	\textit{śiṣya} \cod;
	\MSK\ takes the word as part of a compound with \textit{śiṣyanāmāṣṭaśatena}.
}} nāmā\supplied{ṣṭa}śatena saṃstutya gāthāpañcakenānujñāṃ dattvā / \edtext{\mbox{}\edtext{\textcolor{blue}{udgatā}}{\lemma{%
	{\rm udgatā-\lem}
}\Dfootnote{%
	\emn;
	\textit{uṅgatā-} \cod;
	\textit{udgāthā-} \MSK
}\Cfootnote{%
%
\emn\ \textit{udgata-}, with SDPT? \MSK\ emends \textit{udgāthā-}. Note error \textit{ṅg} for \textit{dg} also in §73.
This is a name of the \textit{mudrā}. The \textit{Guhyasamājamaṇḍalavidhi} teaches this \textit{mudrā}.
%
}}\edtext{vyākaraṇena}{\lemma{%
	{\rm -vyākaraṇena\lem}
}\Dfootnote{%
	\emn\ \MSK;
	\textit{-vyākaraṇaina} \cod
}}}{\lemma{}\Efootnote{%
	uṅgatāvyā[61v1]karaṇaina
}} \edtext{sarvaśiṣyān}{\lemma{%
	{\rm sarvaśiṣyān\lem}
}\Dfootnote{%
	\emn\ \MSK;
	\textit{sarvaśiṣyā} \cod
}} vyākuryāditi // // }
	%maṅgala-: the -ṅ- lacks its hook.
	%udakābhiṣekeṇa: Tib. adds mudrābhiṣekeṇa after this.
	%cābhiṣicya: MSK needlessly emends cābhiṣiñcya.
	%puṣpādibhi: em. puṣpādibhir, with MSK (sil. em.).
	%-pūjāyā: em. -pūjayā, with MSK.
	%sampūjyaye ||: em. sampūjayet ||, with SDPT. MSK emends saṁpūjya. After this, a new paragraph starts, with Ānandagarbha prescribing the manners of a śiṣya.
	%datvācāyam: em. dattvācāryam, with MSK (sil. em.).
	%puṣpādikaṁm abhiṣekāśca grāhyaḥ: em. puṣpādikam abhiṣekāś ca grāhyāḥ. Would puṣpādikam abhiṣekāya grāhyam be another possibility? MSK emends puṣpādikābhiṣekāś ca grāhyāḥ; SDPT reads puṣpādyabhiṣekāś ca grāhyā iti. 
	%°ācāyābhiṣekaṁ: em. °ācāryābhiṣekaṁ, with MSK (sil. em.).
	%datvaimaṁ: em. dattvemaṁ, with MSK (sil. em.).
	%-vārām: em. -vārān; MSK em. -vāram.
	%surata tvam: em. suratas tvam, with MSK.
	%śiṣya: em. śiṣyaṁ; MSK takes the word as part of a compound with śiṣyanāmāṣṭaśatena.
	%°uṅgatā-: em. °udgata-, with SDPT? MSK emends udgāthā-. Note error ṅg for dg also in §73.
	%-karaṇaina: em. -karaṇena, with MSK.
	%sarvaśiṣyā: em. sarvaśiṣyān, with MSK.
	%vyākuyād: em. vyākuryād, with MSK (sil. em.).
\pend

\medskip

\pstart\noindent
3.8.1.7. Guhyābhiṣekaḥ
\pend

\medskip

\pstart
\Skt{\edtext{atha guhyābhiṣeko bhavati / \edtext{ācāryābhiṣekā\supplied{rhaṃ praveśya sarvamaṇḍaladevatātattvamācā}\textcolor{red}{rya}karma ca śikṣayet}{\lemma{%
	{\rm ācāryā\dots śikṣayet\lem}
}\Cfootnote{%
	SDPT: \textit{ācāryābhiṣekārhaṁ praveśya sarvamaṇḍalaṁ tu tat kurusva / iti}.
	Restore \textit{ācāryābhiṣekārham praveśyam ... ācāryakarma ca...}
}}~/ iyatā \edtext{guhyābhiṣekeṇā}{\lemma{%
	{\rm guhyābhiṣekeṇā-\lem}
}\Dfootnote{%
	\emn\ \MSK\ \sil;
	\textit{guhyābhiṣekaiṇā-} \cod
}}bhiṣikto bhavatīti /}{\lemma{%
	{\rm atha \dots\ bhavatīti\lem}
}\Bfootnote{%
	SDPT 296.\textcolor{red}{LINE NUMBERS!!}:
	\textit{atha guhyābhiṣeko bhavati / 
	ācāryābhiṣekārhaṃ praveśya sarvamaṇḍalaṃ tu tat kurusva / iti /}
}} saṃkṣiptakramaḥ //}
	%SDPT: ācāryābhiṣekārhaṁ praveśya sarvamaṇḍalaṁ tu tat kurusva / iti 
	%Restore ācāryābhiṣekārham praveśyam ... ācāryakarma ca...
	%guhyābhiṣekaiṇā-: em. guhyābhiṣekeṇā-, with MSK (sil. em.).
\pend



\bigskip

% \input{10_Madhyamakrama}



\pstart\noindent
{\large 3.8.2. Madhyābhiṣekakrama}
\pend

\bigskip

\pstart
%MSK (76) atha madhyakrama ucyate / mūlamaṇḍalād dhastadvayamātraṁ parityajya praveśadvārasaṁmukhaṁ garbhamaṇḍalārdhapramāṇaṁ caturasraṁ paścimadvāraṁ pañcabhiś cūrṇair maṇḍalam ālikhya tasya / madhye 'ṣṭapattraṁ padmaṁ / tasyopari raktavarṇaṁ raśmimālinaṁ pañcasūcikaṁ vajram ālikhet / tataḥ puṣpādibhis taṁ maṇḍalaṁ saṁpūjya candramaṇḍalapaṭācchannaṁ siṁhāsanaṁ pīṭhikāṁ vā vajrayakṣābhijaptāṁ pūrvavac chiṣyaṁ vajrasattvam adhiṣṭhāya tatra pratiṣṭhāpya vajrasattvamahāmudrayā / siṁhāsanopari vitānam / dakṣiṇabhāge ca paṭasragdāmabhūṣitaṁ sitaratnacūḍacchatraṁ hūṁkārābhimantritam / vāmato nānāvastravicitradhvajapatākāś ca gaganagañjaparijaptāḥ saṁsthāpyārghaṁ dattvā puṣpādibhiḥ saṁpūjya / śaṅkhapaṭahabherīkāhalādibhir vādyair vādyamudrādhiṣṭhitair vādyanṛtyagītair vajralāsyādibhiś cottamāṁ dakṣiṇām ādāya / śrīvajrasattvapratimāṁ tasya mūrdhni pratiṣthāpya maṅgalagāthām (ed. -gāthā) uccārayet // //
%
%RT&AG
\Skt{atha madhyakrama ucyate / mūla\edtext{maṇḍalāddhasta}{\lemma{%
	{\rm -maṇḍalād dhasta-\lem}
}\Dfootnote{%
	\emn\ \MSK\ \sil;
	\textit{-maṇḍalārddhasta-} \cod
}}\edtext{dvayamātraṃ}{\lemma{%
	{\rm -dvayamātraṁ\lem}
}\Dfootnote{%
	\textit{-dvayamantraṁ} \cod
}} parityajya praveśadvārasaṃmukhaṃ garbhamaṇḍalārdhapramāṇaṃ \edtext{caturasraṃ}{\lemma{%
	{\rm caturasraṁ\lem}
}\Dfootnote{%
	\emn\ \MSK\ \sil;
	\textit{caturaśram} \cod
}} paścimadvāraṃ pañcabhiścūrṇairmaṇḍalamālikhya tasya madhye\edtext{'ṣṭa\-pattraṃ}{\lemma{%
	{\rm 'ṣṭapattraṁ\lem}
}\Dfootnote{%
	\emn\ \MSK\ \sil;
	\textit{ṣṭhapattram} \cod
}} padmaṃ tasyopari raktavarṇaṃ raśmimālinaṃ pañcasūcikaṃ vajramālikhet / tataḥ puṣpādibhi\edtext{stan maṇḍalaṃ}{\lemma{%
	{\rm tan maṇḍalaṁ\lem}
}\Cfootnote{%
	\MSK silently (and ungrammatically) emends \textit{taṁ maṇḍalaṁ}.
}} saṃpūjya candramaṇḍalapaṭācchannaṃ siṃhāsanaṃ \edtext{pīṭhikāṃ}{\lemma{%
	{\rm pīṭhikām\lem}
}\Dfootnote{%
	\emn\ \MSK;
	\textit{pīṭhikam} \cod
}} vā \edtext{\textcolor{blue}{vajrayakṣeṇābhijapya} pūrvavacchiṣyaṃ}{\lemma{%
	{\rm vajrayakṣeṇābhijapya pūrvavac chiṣyaṁ\lem}
}\Dfootnote{%
	\emn;
	\textit{vajrayakṣābhijaptāṁ pūrvavac chiṣyaṁ} \MSK;
	\textit{vajrayakṣābhijaptāpūrṇṇavasthiṣyaṁ} \cod
}} \edtext{vajrasattva}{\lemma{%
	{\rm vajrasattvam\lem}
}\Dfootnote{%
	\emn\ \MSK\ \sil;
	\textit{vajrasatvaṁm} \cod
}}madhiṣṭhāya / \edtext{tatra}{\lemma{%
	{\rm tatra\lem}
}\Dfootnote{%
	\emn\ \MSK\ \sil;
	\textit{tattra} \cod
}} pratiṣṭhāpya / vajrasattvamahāmudrayā siṃhāsanopari vitānaṃ dakṣiṇabhāge ca \edtext{\textcolor{blue}{paṭṭa}sra}{\lemma{%
	{\rm paṭṭasrag-\lem}
}\Dfootnote{%
	\emn;
	\textit{paṭasrag-} \MSK;
	\textit{paṭāsrag-} \cod
}}gdāmabhūṣitaṃ \edtext{sitaṃ \textcolor{blue}{ratnacūḍaṃ chatraṃ}}{\lemma{%
	{\rm sitaṁ ratnacūḍaṁ chatraṁ\lem}
}\Dfootnote{%
	\textit{sitaṁ ratnacūḍacchatraṁ} \cod;
	\textit{sitaratnacūḍacchatraṁ} \emn\ \sil\ \MSK
}} hūṃkārābhimantritam / vāmato nānā\textcolor{red}{vastra}\edtext{vicitra}{\lemma{%
	{\rm -vicitra-\lem}
}\Dfootnote{%
	\emn\ \MSK\ \sil;
	\textit{-vicittra-} \cod
}}dhvajapatākāśca gaganagañjaparijaptāḥ saṃsthāpyārghaṃ dattvā puṣpādibhiḥ saṃpūjya śaṅkhapaṭahabherīkāhalādibhirvādyairvādyamudrādhiṣṭhitai\edtext{rbāhyanṛtya\-gītai}{\lemma{%
	{\rm bāhyanṛtyagītair\lem}
}\Cfootnote{%
	\MSK\ silently emends \textit{vādyanṛtyagītair}. See above on \textit{vāhyanaṭa-} or \textit{vādyanaṭa-}.
}}\edtext{rvajralāsyādibhiśco}{\lemma{}\Efootnote{%
	vajra[62r1]lāsyādibhiś
}}ttamāṃ dakṣiṇāmādāya / śrīvajrasattvapratimāṃ tasya mūrdhni pratiṣṭhāpya maṅgalagāthā uccārayet //}
	%-maṇḍalārddhasta-: em. -maṇḍalād dhasta-, with MSK (sil. em.).
	%-dvayamantraṁ: em. -dvayamātraṁ, with MSK.
	%caturaśraṁ: em. caturasraṁ, with MSK (sil. em.).
	%ṣṭhapattraṁ: em. 'ṣṭapattraṁ, with MSK (sil. em.).
	%tan maṇḍalaṁ: MSK silently (and ungrammatically) emends taṁ maṇḍalaṁ.
	%pīṭhikam: em. pīṭhikām, with MSK.
	%-japtāpūrṇṇavasthiṣyaṁ: em. -japtāṁ pūrvavac chiṣyaṁ, with MSK.
	%vajrasatvaṁm: em. vajrasattvam, with MSK (sil. em.).
	%tattra: em. tatra, with MSK (sil. em.).
	%paṭāsrag-: em. paṭasrag-, with MSK.
	%sitaṁ ratna-: MSK silently emends sitaratna-.
	%-vicittra-: em. -vicitra-, with MSK (sil. em.).
	%vāhyanṛtyagītai: MSK silently emends vādyanṛtyagītair. See above on vāhyanaṭa- or vādyanaṭa-.
	%ca °uttamāṁ: MSK emends cottamāṁ

%RT&AG reached here on 12/1/2018
%photo 30,  top, line 1
\pend

%The following stanzas are also attested, in the same order, in Padmaśrīmitra's Maṇḍalopāyikā. We quote here a provisional transcript from the codex unicus by RT:
%lakṣmīdharaḥ kañcanaparvatābhaḥ trilokanāthas trimalaprahīnaḥ |
%(f.9v1)buddho vibuddhāmbujapatranetras trimaṅgalaṁ śāntikaraṁ tavādya ||
%tenopadiṣṭa pravaras tv akambyyaḥ(?) khyātas triloke naradevapūjyaḥ |
%dharmottamaḥ śāntakaraṁ prajānāṁ loke dvitīye śubhamaṅgalaṁ te ||
%saddharmayuktāḥ śubhamaṅgalādyaḥ saṁghā nṛdevāsuradakṣiṇīyaḥ |
%hrīśrīnivāsaḥ pravaro gaṇānāṁ loke gṛtī(2)ya śubhamaṅgalaṁ te ||
%yan maṅgalaṁ tuṣitadevavimānagarbhād āsīd ihāvatarato jagato hitāya |
%sendraiḥ surair anugatasya tathāgatasya tan maṅgalaṁ bhavatu śāntikaraṁ tavādya ||
%yan maṇgalaṁ kisalayojvalapuṣpanaddhe ramye ca lumbinivane bahudevapuṣṭe |
%nāthasya janmani babhūva bhavāntakasya tan maṅgalaṁ bhava(3)tu śāntikaraṁ tavādya ||
%yan maṅgalaṁ vividhaduḥkhavināśānāya tuṣṭyā tapovanam abhivrajato 'rdharātre |
%pakṣaiḥ surair anugatasya tathāgatasya tan maṅgalaṁ bhavatu śāntikaraṁ tavādya ||
%yan maṅgalaṁ puravare kapilāhvaye ca devair mahātmabhir abhiṣṭutam acyutādyaiḥ |
%āsīd anantakuśalasya tathāgatasya tan maṅgalaṁ bhavatu śā(4)ntikaraṁ tavādya ||
%yan maṅgalaṁ jvalitakāñcanavibhramasya vaidūryavarṇatṛṇasaṁstarasaṁsthitasya |
%paryaṅkabaddhanicitrottamaniścalasya tan maṅgalaṁ bhavatu śāntikaraṁ tvādya ||
%yan maṅgalaṁ pravadato varadharmacakre vārāṇasīṁ sthitavataḥ sugatasya śāstu(ḥ) |
%atyadbhutasphuṭam abhūd bhuvi cāmbare ca tan maṅgalaṁ bhavatu śāntikaraṁ tavā(5)dya ||
%yan maṅgalaṁ hitakaraṁ paramaṁ pavitraṁ puṇyakriyākaraṇa(m) āryajanābhijuṣṭam |
%kṛtsnaṁ jagāta bhagavān muniśākyasiṁhaḥ tan maṅgalaṁ bhagavatu śāntikaraṁ tavādya ||

%MSK (77) lakṣmīdharaḥ kāñcanaparvatābhas trilokanāthas trimalaprahīnaḥ / buddho vibuddhāmbujapattranetras tan maṅgalaṁ bhavatu śāntikaraṁ tavādya // 1 //
     
%RT&AG
\verse
\Skt{\edtext{lakṣmīdharaḥ kāñcana\edtext{parvatābhastri}{\lemma{%
	{\rm -parvatābhas tri-\lem}
}\Dfootnote{%
	\emn\ \MSK;
	\textit{-parvvabhas tṛ-} \cod
}}lokanātha\edtext{strimalaprahīṇaḥ}{\lemma{%
	{\rm trimalaprahīṇaḥ\lem}
}\Dfootnote{%
	\emn;
	\textit{tṛmalaprahīṇaḥ} \cod;
	\textit{trimalaprahīnaḥ} \emn\ \MSK\ \sil
}} /\\
buddho vibuddhāmbujapattranetra\edtext{stanmaṅgalaṃ bhavatu śāntikaraṃ tavādya}{\lemma{%
	{\rm tan \dots\ tavādya\lem}
}\Cfootnote{%
	metrical problem: 4th pāda is Vasantatilaka; Tib. suggests different reading, with elements maṅgala śāntikara prathama. 
	Conjecture tan maṅgalaṁ syāt prathaman tu loke?
}}}{\lemma{%
	{\rm lakṣmīdharaḥ \dots\ tavādya\lem}
}\Bfootnote{%
%
This and the following stanzas are also attested, in the same order, in Padmaśrīmitra's \textit{Maṇḍalopāyikā}. We quote here a provisional transcript from the codex unicus by RT.\
Padmaśrīmitra:
\textit{lakṣmīdharaḥ kañcanaparvatābhaḥ trilokanāthas trimalaprahīnaḥ |}
(f.9v1)\textit{buddho vibuddhāmbujapatranetras trimaṅgalaṁ śāntikaraṁ tavādya ||}; %
\textit{Vajrāvalī} 24.5:
\textit{lakṣmīdharaḥ kāñcanaparvatābhas trilokanāthas trimalaprahīṇaḥ |
buddho vibuddhāmbujapatranetras tan maṅgalaṃ bhavatu śāntikaraṃ tavādya ||}
}} //}
	%-parvvabhas tṛ-: em. -parvatābhas tri-, with MSK. 
	% tṛmalaprahīṇaḥ: em. trimalaprahīṇaḥ; MSK trimalaprahīnaḥ (sil. em.).
	%metrical problem: 4th pāda is Vasantatilaka; Tib. suggests different reading, with elements maṅgala śāntikara prathama. 
	%Conjecture tan maṅgalaṁ syāt prathaman tu loke?
	%These stanzas are also found in Padmaśrīmitra's Maṇḍalopāyikā.
%
%MSK     tenopadiṣṭaḥ pravaras tv akaṁpyaḥ khyātas triloke naradevapūjyaḥ / dharmottamaḥ śāntikaraḥ prajānāṁ loke dvitīyaṁ śubhamaṅgalaṁ tat // 2 //
%
%RT&AG
\pend

\verse
\Skt{\edtext{tenopadiṣṭaḥ pravarastva\edtext{kampyaḥ}{\lemma{%
	{\rm akampyaḥ\lem}
}\Dfootnote{%
	\emn\ \MSK;
	\textit{akaṁpya |} \cod
}} khyāta\edtext{striloke}{\lemma{%
	{\rm triloke\lem}
}\Dfootnote{%
	\emn\ \MSK\ \sil;
	\textit{tṛloke} \cod
}} naradevapūjyaḥ /\\
dharmottamaḥ śāntikaraḥ \edtext{prajānāṃ}{\lemma{%
	{\rm prajānāṁ\lem}
}\Dfootnote{%
	\emn\ \MSK;
	\textit{prajānā} \cod
}} loke \edtext{dvitīyaṃ}{\lemma{%
	{\rm dvitīyaṁ\lem}
}\Cfootnote{%
	The \textit{yaṁ} seems to be the result of a correction of some previous error for this \textit{akṣara}.%
}} \edtext{śubhamaṅgalaṃ tat}{\lemma{%
	{\rm śubhamaṅgalaṁ tat\lem}
}\Dfootnote{%
	\emn\ \MSK;
	\textit{subhaṁ maṅgalan tam} \cod
}}}{\lemma{%
	{\rm tenopadiṣṭaḥ \dots\ śubhamaṅgalaṃ tat\lem}
}\Bfootnote{
%
Padmaśrīmitra:
\textit{tenopadiṣṭa pravaras tv akambyyaḥ(?) khyātas triloke naradevapūjyaḥ |
dharmottamaḥ śāntakaraṁ prajānāṁ loke dvitīye śubhamaṅgalaṁ te ||};
\textit{Vajrāvalī} 24.5:
\textit{tenopadiṣṭaḥ pravaras tv akampyaḥ khyātas triloke naradevapūjaḥ |
dharmottamaḥ śāntikaraḥ prajānāṃ tan maṅgalaṃ bhavatu śāntikaraṃ tavādya ||}
%
}} //}
	%akaṁpya |: em. akampyaḥ, with MSK.
	%tṛloke: em. triloke, with MSK (sil. em.).
	%prajānā: em. prajānāṁ, with MSK.
	%dvitīyaṁ: the yaṁ seems to be the result of a correction of some previous error for this akṣara.
	%subhaṁ ma§○ṅgalan tam·: em.	śubhamaṅgalaṁ tat, with MSK.
%
%MSK     saddharmayuktaḥ śrutimaṅgalāḍhyaḥ saṁgho nṛdevāsuradakṣiṇīyaḥ / hrīśrīnivāsaḥ pravaro gaṇānāṁ loke tṛtīyaṁ śubhamaṅgalaṁ tat // 3 //
%
%RT&AG
\pend

\verse
\Skt{\edtext{saddharmayuktaḥ \textcolor{red}{śruti}maṅgalāḍhyaḥ saṃgho nṛdevāsuradakṣiṇīyaḥ /\\
\edtext{hrīśrīnivāsaḥ}{\lemma{%
	{\rm hrīśrīnivāsaḥ\lem}
}\Dfootnote{%
	\emn\ \MSK\ \sil;
	\textit{hrīśrīnivāśaḥ} \cod
}} pravaro gaṇānāṃ loke \edtext{tṛtīyaṃ}{\lemma{%
	{\rm tṛtīyaṁ\lem}
}\Dfootnote{%
	\emn\ \MSK;
	\textit{tṛtīya} \cod
}} śubhamaṅgalaṃ \edtext{tat}{\lemma{%
	{\rm tat\lem}
}\Dfootnote{%
	\emn\ \MSK;
	\textit{tam} \cod
}}}{\lemma{%
	{\rm saddharmayuktaḥ \dots\ śubhamaṅgalaṃ tat\lem}
}\Bfootnote{%
%
Padmaśrīmitra:
\textit{saddharmayuktāḥ \textcolor{red}{śubha}maṅgalādyaḥ saṁghā nṛdevāsuradakṣiṇīyaḥ |
hrīśrīnivāsaḥ pravaro gaṇānāṁ loke gṛtī(2)ya śubhamaṅgalaṁ te ||};
%
\textit{Vajrāvalī} 24.5:
\textit{saddharmayuktaḥ \textcolor{red}{śruta}maṅgalāḍhyaḥ saṃgho nṛdevāsuradakṣiṇīyaḥ |
yaḥ śrīnivāsaḥ pravaro gaṇānāṃ tan maṅgalaṃ bhavatu śāntikaraṃ tavādya ||}
%
}} //}
	%hrīśrīnivāśaḥ: em. hrīśrīnivāsaḥ, with MSK (sil. em.).
	%tṛtīya: em. tṛtīyaṁ, with MSK.
	%tam·: em. tat, with MSK.
%
%MSK yan maṅgalaṁ tuṣitadevavimānagarbhād āsīd ihāvatarato jagato hitāya / sendraiḥ surair anugatasya tathāgatasya tan maṅgalaṁ bhavatu śāntikaraṁ tavādya // 4 //
%
%RT&AG 
\pend

\verse
\Skt{\edtext{yanmaṅgalaṃ \edtext{tuṣita}{\lemma{%
	{\rm tuṣita-\lem}
}\Dfootnote{%
	\emn\ \MSK\ \sil;
	\textit{tuśita-} \cod
}}devavimānagarbhā\edtext{dāsī}{\lemma{%
	{\rm āsīd\lem}
}\Dfootnote{%
	\emn\ \MSK\ \sil;
	\textit{āśīd} \cod
}}dihāvatarato jagato hitāya /\\
\edtext{sendraiḥ}{\lemma{%
	{\rm sendraiḥ\lem}
}\Dfootnote{%
	\emn\ \MSK;
	\textit{saindraiḥ} \cod
}} surairanugatasya tathāgatasya tanmaṅgalaṃ bhavatu śāntikaraṃ \edtext{tavādya}{\lemma{%
	{\rm tavādya\lem}
}\Dfootnote{%
	\emn\ \MSK\ \sil;
	\textit{taṁvyadya} \cod
}}}{\lemma{%
	{\rm yanmaṅgalaṃ \dots\ tavādya\lem}
}\Bfootnote{
	Padmaśrīmitra:
	\textit{yan maṅgalaṁ tuṣitadevavimānagarbhād āsīd ihāvatarato jagato hitāya |
	sendraiḥ surair anugatasya tathāgatasya tan maṅgalaṁ bhavatu śāntikaraṁ tavādya ||}
}}~//}
	%tuśita-: em. tuṣita-, with MSK (sil. em.).
	%āśīd: em. āsīd, with MSK (sil. em.).
	%saindraiḥ: em. sendraiḥ, with MSK.
	%taṁvyadya: em. tavādya, with MSK (sil. em.).
%
\pend

\verse
%MSK    yan maṅgalaṁ kisalayajvalapuṣpanaddhe ramye ca lumbinivane bahudevajuṣṭhe / nāthasya janmani babhūva bhavāntakasya tan maṅgalaṁ bhavatu śāntikaraṁ tavādya // 5 //
%
%RT&AG
\Skt{\edtext{ya\edtext{nmaṅgalaṃ}{\lemma{%
	{\rm maṅgalaṁ\lem}
}\Dfootnote{%
	\emn\ \MSK;
	\textit{maṅgala} \cod
}} \edtext{kisalayojjvala}{\lemma{%
	{\rm kisalayojjvala-\lem}
}\Dfootnote{%
	\emn;
	\textit{kiśalayojvala-} \cod;
	\MSK\ emends kisalayajvala-, but the phrase kisalayojjvala is found in several Sanskrit texts.
}}puṣpanaddhe ramye ca lumbinivane bahudeva\edtext{juṣṭe}{\lemma{%
	{\rm -juṣṭe\lem}
}\Dfootnote{%
	\emn;
	\textit{-juṣṭhe} \cod;
	\MSK\ reads -juṣṭhe as though it were unproblematic.
}} /\\
nāthasya \edtext{janmani babhūva bhavāntakasya}{\lemma{%
	{\rm janmani vabhūva bhavāntaka\lem}
}\Dfootnote{%
	restoration on the basis of Padmaśrīmitra and other sources.
	\unclear{janmani vabhūva bhavāntaka}sya \cod
}} tanmaṅgalaṃ bhavatu śāntikaraṃ tavādya}{\lemma{%
	{\rm yanmaṅgalaṃ \dots\ tavādya\lem}
}\Bfootnote{
	Padmaśrīmitra:
	\textit{yan maṇgalaṁ kisalayojvalapuṣpanaddhe ramye ca lumbinivane bahudevapuṣṭe |
	nāthasya janmani babhūva bhavāntakasya tan maṅgalaṁ bhava(3)tu śāntikaraṁ tavādya ||}
}\lemma{%
	{\rm After this stanza\lem}
}\Cfootnote{%
%
%
\MSK\ inserts here
     \textit{groṅ khyer ser skye'i gnas rgyal po'i pho braṅ du |
    lhums nas bltams pa ston pa bde gśegs zas gstan sras |
     myur bskrun bdud rtsi'i chus stor pa yi bkra śis gaṅ |
     bkra śis des ni skye dgru rnams la .zi bhyed śog || 6  ||}
With regard to the verse no. 6, \MSK\ states that the Sanskrit ms. omits this verse.
It is reconstructed by H. Takahashi: yan maṅgalaṁ kapilavastuni rājadhānyāṁ garbhād viniḥsṛtavataḥ snapitasya devaiḥ | śauddhodaner amṛtavāribhir āśuvṛddhyai tan maṅgalaṁ bhavatu śāntikaraṁ tavādya ||
For stanzas 6-8, MSK wrongly reconstructs  order and text. We move quotations from MSK to the correct position, but leave MSK's stanza numbering intact in these quotations.
%
}} //}
	%maṅgalaṁ: em. maṅgalaṁ, with MSK.
	%kiśalayojvala-: em. kisalayojjvala-; MSK emends kisalayajvala-, but the phrase kisalayojjvala is found in several Sanskrit texts.
	%-juṣṭhe: em. -juṣṭe. MSK reads -juṣṭhe as though it were unproblematic.
	% \unclear{janmani vabhūva bhavāntaka}sya: restoration on the basis of Padmaśrīmitra and other sources.
%
%MSK inserts here
%     groṅ khyer ser skye'i gnas rgyal po'i pho braṅ du /
%    lhums nas bltams pa ston pa bde gśegs zas gstan sras /
%     myur bskrun bdud rtsi'i chus stor pa yi bkra śis gaṅ /
%     bkra śis des ni skye dgru rnams la .zi bhyed śog // 6 //
% With regard to the verse no. 6, MSK states that the Sanskrit ms. omits this verse.
% It is reconstructed by H. Takahashi: yan maṅgalaṁ kapilavastuni rājadhānyāṁ garbhād viniḥsṛtavataḥ snapitasya devaiḥ | śauddhodaner amṛtavāribhir āśuvṛddhyai tan maṅgalaṁ bhavatu śāntikaraṁ tavādya ||
%
%For stanzas 6-8, MSK wrongly reconstructs  order and text. We move quotations from MSK to the correct position, but leave MSK's stanza numbering intact in these quotations.
%
\pend

\verse
%MSK     yan maṅgalaṁ vividhaduḥkhavināśanāya tuṣtyā tapovanam abhivrajato 'rdharātre /  āsīt suraiḥ parivṛtasya namaskṛtasya tan maṅgalaṁ bhavatu śāntikaraṁ tavādya // 8 //]
%
%RT&AG
%\edtext{}{\lemma{}\Efootnote{tavādya [62v1] yan maṅgalaṁ}}% 
\Skt{\edtext{\edtext{yanmaṅgalaṃ}{\lemma{}\Efootnote{
	{\rm tavādya [62v1] yan maṅgalaṁ\lem}
}} \edtext{vividha}{\lemma{%
	{\rm vividha-\lem}
}\Dfootnote{%
	\emn\ \MSK;
	\textit{vividhi-} \cod
}}duḥkha\edtext{vināśanāya}{\lemma{%
	{\rm -vināśanāya\lem}
}\Dfootnote{%
	\emn\ \MSK;
	\textit{-vināśaya} \cod
}} \edtext{tuṣṭyā}{\lemma{%
	{\rm tuṣṭyā\lem}
}\Dfootnote{%
	\emn\ \MSK\ \sil;
	\textit{tuṣṭhā} \cod
}} tapova\supplied{namabhivrajato'rdharātre~/}\\
\supplied{yakṣaiḥ surairanugatasya tathāgatasya tanmaṅgalaṃ bhavatu śāntikaraṃ tavādya~//}}{\lemma{%
	{\rm yanmaṅgalaṃ \dots\ tavādya\lem}
}\Bfootnote{
	Padmaśrīmitra:
	yan maṅgalaṁ vividhaduḥkhavināśānāya tuṣṭyā tapovanam abhivrajato 'rdharātre |
	\textcolor{red}{pakṣaiḥ} surair anugatasya tathāgatasya tan maṅgalaṁ bhavatu śāntikaraṁ tavādya ||
}} }
% The ms. skips from tapoVA to puraVA in the next stanza. Our restitution within \textcolor{blue}{(commad)}omitted{...} follows Padmaśrīmitra.
	%vividhi-: em. vividha-, with MSK.
	%-vināśaya: em. -vināśanāya, with MSK.
	%tuṣṭhā: em. tuṣṭyā, with MSK (sil. em.).
	%The ms. skips from tapoVA to puraVA in the next stanza. Our restitution within \omitted{...} follows Padmaśrīmitra.
%
\pend

\verse
%MSK     [yan maṅgalaṁ puravare kapilāhvaye ca devair mahātmabhir abhiṣṭutavanditasya / āsīd acintyakuśalasya tathāgatasya tan maṅgalaṁ bhavatu śāntikaraṁ tavādya // 7 //
%
%RT&AG
\Skt{\edtext{\mbox{}\edtext{\supplied{yanmaṅgalaṃ puravare}}{\lemma{%
	{\rm yan maṅgalaṁ puravare\lem}
}\Cfootnote{%
	regarding this restitution, see our comment on the preceding stanza.
}} kapilāhvaye ca devai\edtext{rmahātmabhirabhiṣṭutam}{\lemma{%
	{\rm mahātmabhir abhiṣṭutam\lem}
}\Dfootnote{%
	\emn\ \MSK\ PŚM;
	\textit{mahātmabhi ṣṭhutam} \cod
}} \edtext{}{\lemma{%
	{\rm abhiṣṭhutam acyutādyaiḥ\lem}
}\Cfootnote{%
	\MSK\ here reconstructs abhiṣṭutavanditasya, but PŚM has the same texts as our ms.
}}acyutā\supplied{dyaiḥ}~/\\
\supplied{āsīdanantakuśalasya}
%	{\rm acyutādyaiḥ | āsīd anantakuśalasya\lem}
%	our restitution follows PŚM; MSK restores -anantakuśalasya.
tathāgatasya tanmaṅgalaṃ bhavatu śāntikaraṃ tavādya}{\lemma{%
	{\rm yanmaṅgalaṃ \dots\ tavādya\lem}
}\Bfootnote{
	Padmaśrīmitra:
	yan maṅgalaṁ puravare kapilāhvaye ca devair mahātmabhir abhiṣṭutam acyutādyaiḥ |
	āsīd anantakuśalasya tathāgatasya tan maṅgalaṁ bhavatu śā(4)ntikaraṁ tavādya ||
}}~//}
	%\omitted{yan maṅgalaṁ puravare}: regarding this restitution, see our comment on the preceding stanza.
	%mahātmabhi ṣṭhutam: em. mahātmabhir abhiṣṭutam, with MSK and PŚM.
	%abhiṣṭhutam acyutā\unclear{dyai}\lost{ḥ}: MSK here reconstructs abhiṣṭutavanditasya, but PŚM has the same texts as our ms.
	%acyutā\unclear{dyai}\lost{ḥ | āsīd anantakuśala}\unclear{sya}: our restitution follows PŚM; MSK restores -anantakuśalasya.
%
\pend

\verse
%MSK  yan maṅgalaṁ sakalasattvahitāya bodhau saddharmaratnaniratasya muner babhūva / sarvārthasiddhiṣu viśālaparākramasya tan maṅgalaṁ bhavatu śāntikaraṁ tavādya // 9 //
%
%RT&AG 
\Skt{\edtext{yanmaṅgalaṃ sakalasattvahitāya bodhau saddharmaratnaniratasya mune\edtext{rbabhūva}{\lemma{%
	{\rm babhūva\lem}
}\Dfootnote{%
	\emn\ \MSK;
	\textit{babhūya} \cod
}} /\\
sarvārtha\edtext{siddhisuviśāla}{\lemma{%
	{\rm -siddhisuviśāla-\lem}
}\Dfootnote{%
	\MSK\ silently emends -siddhiṣu and inserts a word break after this. But suviśāla is a very common word.
}}parākramasya tanmaṅgalaṃ bhavatu śāntikaraṃ tavādya}{\lemma{%
	{\rm yanmaṅgalaṃ \dots\ tavādya\lem}
}\Bfootnote{%
	n.e.\ Padmaśrīmitra. \textcolor{red}{CHECK!!}
}}~//}
	%This stanza not in PŚM.
	%babhūya: em. babhūva, with MSK.
	%-siddhisuviśāla-: MSK silently emends -siddhiṣu and inserts a word break after this. But suviśāla is a very common word.
%
\pend

\verse
%MSK    yan maṅgalaṁ jvalitakāñcanavigrahasya vaiḍūryavarṇatṛṇasaṁstaramadhyagasya / paryaṅkabaddhanibiḍottamaniścalasya tan maṅgalaṁ bhavatu śāntikaraṁ tavādya // 10 //
%
%RT&AG 
\Skt{\edtext{yanmaṅgalaṃ jvalitakāñcanavigrahasya vaiḍūryavarṇatṛṇa\edtext{saṃstara}{\lemma{%
	{\rm -saṁstara-\lem}
}\Dfootnote{%
	\emn\ with PŚM, \MSK\ \sil
	\textit{-saptara-} \cod
}}madhyagasya /\\ 
paryaṅkabaddha\edtext{nibiḍottama}{\lemma{%
	{\rm -nibiḍottama-\lem}
}\Dfootnote{%
	\emn\ \MSK; %emends -nibiḍottama-, which is graphically very plausible. 
	\textit{-nivitottama-} \cod\ 
	Check PŚM ms.
}}niścalasya tanmaṅgalaṃ bhavatu śāntikaraṃ tavādya}{\lemma{%
	{\rm yanmaṅgalaṃ jvalita-\dots\ tavādya\lem}
}\Bfootnote{
	Padmaśrīmitra:
	yan maṅgalaṁ jvalitakāñcanavibhramasya vaidūryavarṇatṛṇasaṁstarasaṁsthitasya |
	paryaṅkabaddhanicitrottamaniścalasya tan maṅgalaṁ bhavatu śāntikaraṁ tvādya ||
}\lemma{%
	{\rm After this stanza\lem}
}\Cfootnote{%
	MSK inserts here:  
	chu bo'i 'gram na rtswa ṣa'i phreṅ gis yoṅs bskor ba /   śin tu rmad du byun ba'i srid pa sel mdzad pa /  klu yi rgyal pos phyag byas pa yi bkra śis gaṅ /  bkra śis des ni skye dgu rnams la .zi byed śog // 11 //
}}~//}
	%vaiḍūya-: em. vaiḍūrya-, with MSK and PŚM.
	%-sa○ptara-: em. -saṁstara-, with MSK and PŚM.
	%payaṅka-: em. paryaṅka-, with PŚM and MSK (sil. em.).
	%-nivitottama-: MSK emends -nibiḍottama-, which is graphically very plausible. Check PŚM ms.
%
%MSK inserts here  
%  chu bo'i 'gram na rtswa ṣa'i phreṅ gis yoṅs bskor ba /   śin tu rmad du byun ba'i srid pa sel mdzad pa /  klu yi rgyal pos phyag byas pa yi bkra śis gaṅ /  bkra śis des ni skye dgu rnams la .zi byed śog // 11 //
%
%RT&AG reached here on 16/1/2018
%photo 29,  bottom, line 3, end
%
\pend

\verse
%MSK   yan maṅgalaṁ bhagavato drumarājamūle maitrībalena vijite bahumārapakṣe / nānāprakāram abhavad bhuvi cāmbare ca tan maṅgalaṁ bhavatu śāntikaraṁ tavādya // 12 //
%
%RT&AG
\Skt{\edtext{yanmaṅgalaṃ bhagavato drumarājamūle maitrībalena vijite bahumārapakṣe /\\ 
\textcolor{red}{nānāprakāra}\edtext{mabhavadbhuvi}{\lemma{%
	{\rm abhavad bhuvi\lem}
}\Dfootnote{%
	\emn\ \MSK;
	\textit{abhava bhuvi} \cod
}} cāmvare tanmaṅgalaṃ bhavatu śāntikaraṃ tavādya}{\lemma{%
	{\rm yanmaṅgalaṃ bhagavato \dots\ tavādya\lem}
}\Bfootnote{%
	n.e.\ Padmaśrīmitra. \textcolor{red}{CHECK!!}
}\lemma{%
	{\rm After this stanza\lem}
}\Cfootnote{%
	MSK inserts here  
	sdug bsṅal mtha' dag g.zig phyir rdo rje'i gdan b.zugs te /   nam gyi tho raṅs bdud b.zi po dag rnams btul ba / ston pa bde bar gśegs pa'i bkra śis gaṅ yin pa / bkra śis des ni skye dgru rnams la .zi byed śog // 13 //
}}//}
 	%abhava bhuvi: em. abhavad bhuvi, with MSK.
%
\pend

\verse
%MSK inserts here  
%  sdug bsṅal mtha' dag g.zig phyir rdo rje'i gdan b.zugs te /   nam gyi tho raṅs bdud b.zi po dag rnams btul ba / ston pa bde bar gśegs pa'i bkra śis gaṅ yin pa / bkra śis des ni skye dgru rnams la .zi byed śog // 13 //
%
%MSK  yan maṅgalaṁ pravadato varadharmacakre vārāṇasīsthitavataḥ sugatasya śāstuḥ
%atyadbhutaṁ sphuṭam abhūd bhuvi cāmbare ca tan maṅgalaṁ bhavatu śāntikaraṁ tavādya // 14 //
%
%RT&AG
\Skt{\edtext{yanmaṅgalaṃ pravadato \textcolor{red}{varadharmacakre} % vajradharmacakraṃ?
% the combination of pravadatas and dharmacakra- is strange (PDSz)
vārāṇasīsthitavataḥ sugatasya śāstuḥ~/\\
\edtext{atyadbhutaṃ}{\lemma{%
	{\rm atyadbhutaṁ\lem}
}\Dfootnote{%
	\emn\ with PŚM and \MSK:
	\textit{adbhudbhutaṁ} \cod
}} sphuṭa\edtext{mabhū}{\lemma{%
	{\rm abhūd\lem}
}\Dfootnote{%
	\emn\ \MSK:
	\textit{abhud} \cod
}}dbhuvi cāmvare ca tanmaṅgalaṃ bhavatu śāntikaraṃ tavādya}{\lemma{%
	{\rm yanmaṅgalaṃ pravadato \dots\ tavādya\lem}
}\Bfootnote{%
	Padmaśrīmitra:
	yan maṅgalaṁ pravadato varadharmacakre vārāṇasīṁ sthitavataḥ sugatasya śāstu(ḥ) |
	atyadbhutasphuṭam abhūd bhuvi cāmbare ca tan maṅgalaṁ bhavatu śāntikaraṁ tavā(5)dya ||
}} //}
	%°adbhudbhutaṁ: em. °atyadbhutaṁ, with PŚM and MSK.
	%abhud: em. abhūd, with MSK.
%
%PŚM
%yan maṅgalaṁ pravadato varadharmacakre vārāṇasīṁ sthitavataḥ sugatasya śāstu(ḥ) |
%atyadbhutasphuṭam abhūd bhuvi cāmbare ca tan maṅgalaṁ bhavatu śāntikaraṁ tavā(5)dya ||
%
\pend

\verse
%MSK     yan maṅgalaṁ hitakaraṁ paramaṁ pavitraṁ puṇyakriyākaraṇam āryajanābhijuṣṭam /  kṛtsnaṁ jagāda bhagavān muniśākyasiṁhas tan maṅgalaṁ bhavatu śāntikaraṁ tavādya // 15 //
%
%RT&AG
\Skt{\edtext{yanmaṅgalaṃ hitakaraṃ paramaṃ pavitraṃ \edtext{puṇyakriyākaraṇa}{\lemma{}\Efootnote{%
	puṇyakriyāka[63r1]raṇam
}}māryajanābhijuṣṭam /\\
\edtext{kṛtsnaṃ jagāda bhagavānmuniśākyasiṃhaḥ}{\lemma{%
	{\rm kṛtsnaṃ jagāda bhagavān muniśākyasiṁhaḥ\lem}
}\Cfootnote{%
	kṛtsnañ jagāda bhagavāṁ muniśākyasiṁhaḥ:  kṛtsnaṁ jagāda bhagavn muniśākyasiṁhas MSK (silent normalizations).
	Report this in the apparatus?%
}} tanmaṅgalaṃ bhavatu śāntikaraṃ tavādyeti}{\lemma{%
	{\rm yanmaṅgalaṃ hitakaraṃ \dots\ tavādyeti\lem}
}\Bfootnote{%
	Padmaśrīmitra:
	yan maṅgalaṁ hitakaraṁ paramaṁ pavitraṁ puṇyakriyākaraṇa(m) āryajanābhijuṣṭam |
	kṛtsnaṁ jagāta bhagavān muniśākyasiṁhaḥ tan maṅgalaṁ bhagavatu śāntikaraṁ tavādya ||;\
	Kriyāsaṃgrahapañjikā 6-8-8-2-2:
	\textit{yan maṅgalaṁ hitakaraṁ paramaṁ pavitraṁ puṇyakriyākaraṇam āryajanābhijuṣṭam |
	kṛtsnaṁ jagāta bhagavān muniśākyasiṁhaḥ tan maṅgalaṁ bhavatu te 'dya varābhiṣeke ||}
	(\textsc{Tanemura} 2004: 199–200)
}\lemma{%
	{\rm After this stanza\lem}
}\Cfootnote{%
	MSK inserts here  
     mu stegs byed pa kun gyi ṅa rgyal g.zom pa daṅ /
     'gro la bde ba skyed phyir cho 'phrul 'dam pa dag /
     ñe bar ston pa'i rgyal po'i bkra śis graṅ yin pa /
     bkra śis des ni skye dgu rnams la .zi byed śog // 16 //
     'gro la phan phyir mtho ris gnas ni 'dir gśegs te /
     thaṅs pa la sogs lha thogs lag na rṅa yab gdugs /
     sna tshogs thogs pas yoṅs su bskor ba'i bkra śis gaṅ /
     bkra śis des ni skye dgu rnams la .zi byed śog // 17 //
     de b.zin gśegs pa .zi ba'i mchog tu ñer gśegs pa /
     ma da ra yi me tog maṅ pos mṅon mchod pa /
     lha mchog rnams kyis mṅon par bstod pa'i bkra śis gaṅ /
     bkra śis des ni skye dgu rnams la .zi byed śog // 18 //
}} //}
	%puṇya-: MSK reads puñya- and emends to puṇya-.
	%āya-: em. ārya, with MSK (sil. em.).
	%kṛtsnañ jagāda bhagavāṁ muniśākyasiṁhaḥ:  kṛtsnaṁ jagāda bhagavn muniśākyasiṁhas MSK (silent normalizations).

%yan maṅgalaṁ hitakaraṁ paramaṁ pavitraṁ puṇyakriyākaraṇa(m) āryajanābhijuṣṭam |
%kṛtsnaṁ jagāta bhagavān muniśākyasiṁhaḥ tan maṅgalaṁ bhagavatu śāntikaraṁ tavādya ||

%MSK inserts here  
%     mu stegs byed pa kun gyi ṅa rgyal g.zom pa daṅ /
%     'gro la bde ba skyed phyir cho 'phrul 'dam pa dag /
%     ñe bar ston pa'i rgyal po'i bkra śis graṅ yin pa /
%     bkra śis des ni skye dgu rnams la .zi byed śog // 16 //
%     'gro la phan phyir mtho ris gnas ni 'dir gśegs te /
%     thaṅs pa la sogs lha thogs lag na rṅa yab gdugs /
%     sna tshogs thogs pas yoṅs su bskor ba'i bkra śis gaṅ /
%     bkra śis des ni skye dgu rnams la .zi byed śog // 17 //
%     de b.zin gśegs pa .zi ba'i mchog tu ñer gśegs pa /
%     ma da ra yi me tog maṅ pos mṅon mchod pa /
%     lha mchog rnams kyis mṅon par bstod pa'i bkra śis gaṅ /
%     bkra śis des ni skye dgu rnams la .zi byed śog // 18 //
\pend

\pstart
%MSK (78) nāmāṣṭaśatena ca saṁstutyādau pūrṇakumbhair vajrāṅkuśādihṛdayāni udīrayaṁs tatas tatkalaśaiḥ paścād vijayakalaśād vajramuṣṭinodakam ādāyābhiṣekaṁ pānaṁ ca dattvā kalaśena cābhiṣiñced udakābhiṣekataḥ / tatrāyaṁ prayogaḥ /   vajrāṅkuśa oṁ vajrābhiṣiñca / yāvat / oṁ vajrasattva hūṁ / oṁ vajrābhiṣiñca / oṁ mahāsukha // ityādi pūrvavad āvartayet / śeṣaṁ pūrvavad evam //  madhyakramo 'yam //
%
%RT&AG
\Skt{nāmāṣṭaśatena ca saṃstutyādau \edtext{pūrṇakumbhai}{\lemma{%
	{\rm pūrṇakumbhair vajrāṁkuśādi-\lem}
}\Cfootnote{%
	Tib. seems to imply a word like tatas between pūrṇṇakumbhair and vajrāṁkuśādi-.
}}rvajrāṅkuśādi\edtext{hṛdayāni}{\lemma{%
	{\rm -hṛdayāni\lem}
}\Dfootnote{%
	\emn\ \MSK\ \sil;
	\textit{-hṛdayani} \cod
}} udīrayaṃs \edtext{\textcolor{blue}{tatastatkalaśaiḥ}}{\lemma{%
	{\rm samastakalaśaiḥ\lem}
}\Dfootnote{%
	\emn\ \MSK;
	\textit{tatastakalaśaiḥ} \cod
%	\textit{tatas tatkalaśaiḥ} \emn\ \MSK
}} paścādvijaya\edtext{kalaśā}{\lemma{%
	{\rm -kalaśād\lem}
}\Dfootnote{%
	\emn\ \MSK;
	\textit{-kalaśā} \cod
}}dvajramuṣṭinodakamādāyābhiṣekaṃ pānaṃ ca dattvā \edtext{kalaśena}{\lemma{%
	{\rm kalaśena\lem}
}\Dfootnote{%
	\emn\ \MSK;
	\textit{kalasena} \cod
}} cābhiṣiñcedudakābhiṣekataḥ / 
\edtext{tatrāyaṃ}{\lemma{%
	{\rm tatrāyaṃ\lem}
}\Dfootnote{%
	\emn\ \MSK;
	\textit{tatrāyām} \cod
}} prayogaḥ / {\om} vajrāṅkuśa hū{\cb} vajrābhiṣiñca yāvat  {\om} vajrasattva hū{\cb} / {\om} vajrābhiṣiñceti / {\om} mahāsukhetyādi pūrvavadāvartayet / \edtext{śeṣaṃ}{\lemma{%
	{\rm śeṣaṁ\lem}
}\Dfootnote{%
	\emn\ \MSK;
	\textit{seṣam} \cod
}} pūrvava\edtext{deva}{\lemma{%
	{\rm eva\lem}
}\Dfootnote{%
	\emn\ \MSK;
	\textit{evaṁ} \cod
}} / madhyakramo'yam // //}
	%RT: the object of praise here is the pupil.
	%Tib. seems to imply a word like tatas between pūrṇṇakumbhair and vajrāṁkuśādi-.
	%-hṛdayani: em. -hṛdayāni, with MSK (sil. em.).
	%takalaśaiḥ: em. tatkalaśaiḥ, with MSK?
	%-kalaśā: em. -kalaśād, with MSK?
	%kalasena: em. kalaśena, with MSK.
	%tatrāyām: em. tatrāyam, with MSK.
	%seṣam: em. śeṣaṁ, with MSK.
	%evaṁ: em. eva, with MSK.

\pend



\bigskip

% \input{11_Vistarakrama}


\pstart\noindent
{\large 3.8.3. Vistarābhiṣekakrama}
\pend

\bigskip

\pstart
\Skt{atha vistarābhiṣekakramo bhavati / 
\edtext{tathaiva}{\lemma{%
	{\rm tathaiva\lem}
}\Dfootnote{%
	\emn;
	\textit{athaiva} \cod
}} sarvaṃ kṛtvā lāsyādyaṣṭavidhapūjayā ca \edtext{saṃpūjayet}{\lemma{%
	{\rm saṁpūjayet\lem}
}\Dfootnote{%
	\textit{sampūjayet} \cod\pc;
	\textit{sampūjyayet} \cod\ac;
	\MSK\ notes the presence of \textit{y} but does not note that it is cancelled.
}} / 
tato ratnaśalākāṃ sauvarṇa\edtext{śalākāṃ}{\lemma{
	{\rm -śalākāṁ\lem}
}\Dfootnote{
	\emn\ \MSK\ \sil;
	\textit{-salākām} \cod
}} vādāya purataḥ sthitvā sphuṭavāgevaṃ vadet/}
\pend

\verse 
\Skt{\edtext{ajñānapaṭalaṃ \edtext{vatsa}{\lemma{%
	{\rm vatsa\lem}
}\Dfootnote{
	\emn\ \MSK\ $\leftarrow$ VĀ;
	\textit{vaccha} \cod
}} apanītaṃ jinaistava/\\
\edtext{śalāki}{\lemma{%
	{\rm śalāki-\lem}
}\Dfootnote{%
	\emn;
	\textit{śalākī-} \cod
}}vaidyarājaistu yathā lokasya \edtext{taimiram}{\lemma{%
	{\rm taimiram\lem}
}\Dfootnote{%
	\emn\ \MSK;
	\textit{vai purā} \cod (\MSK\ reads \textit{cai purā})
}}//}{\lemma{
	{\rm ajñānapaṭalaṁ \dots\ purā\lem}
}\Bfootnote{
	The \textit{ajñānapaṭalam} stanza is also found in several other texts. See Tanemura 2004: 180.
}}}
\pend
\pstart\noindent
\Skt{athāsya hṛdayam / {\om} vajranetrāpahara \edtext{paṭalaṃ}{\lemma{
	{\rm paṭalaṁ\lem}
}\Dfootnote{
	\emn\ \MSK;
	\textit{paṭaliṁ} \cod
}} hrīriti/}
\pend

\pstart
\Skt{tato darpaṇamādāya \edtext{dharmalakṣaṇaṃ}{\lemma{
	{\rm dharmalakṣaṇaṁ\lem}
}\Dfootnote{
	\emn\ \MSK;
	\textit{dharmmaṇalakṣaṇaṁ} \cod
}} kathayet/}
\pend

\verse
\Skt{%
\edtext{pratibimbasamā dharmā acchāḥ śuddhā hyanāvi\supplied{lāḥ}/ \\
\supplied{agrāhyānabhilāpyāśca hetu}karmasamudbhavāḥ//}{\lemma{
	{\rm pratibimbasamā \dots\ -samudbhavāḥ\lem}
}\Bfootnote{
	\textit{Guhyasamājamaṇḍalavidhi} v.288
	($\rightarrow$ \textit{Vajrāvalī} (33. \textit{Darpaṇābhiṣekavidhi}))
}}}\\
\Skt{%
\edtext{\mbox{}%
\edtext{evaṃ}{\lemma{
	{\rm evaṁ\lem}
}\Dfootnote{
	\emn\ \MSK \sil;
	\textit{eva} \cod
}} \edtext{jñātvā tu vai}{\lemma{}\Efootnote{%
	jñātvā [63v1] tu vai
}} \edtext{dharmā}{\lemma{
	{\rm dharmān\lem}
}\Dfootnote{
	\emn\ \MSK;
	\textit{dharmmā} \cod
}}\edtext{nniḥsvabhāvā}{\lemma{
	{\rm niḥsvabhāvān\lem}
}\Cfootnote{
	\MSK\ reads bhiḥsvabhāvān, but we think ni- was intended.
}}\edtext{nanālayān}{\lemma{
	{\rm anālayān\lem}
}\Dfootnote{
	\emn\ \MSK \sil;
	\textit{anālayāṁ} \cod
}}/\\
kuru sarvārthamatulaṃ \edtext{jāto 'syu}{\lemma{
	{\rm jāto 'sy\lem}
}\Dfootnote{
	\emn\ \MSK \sil;
	\textit{jātau sy} \cod
}}rasi tāyināmiti//}{\lemma{
	{\rm evaṁ \dots\ tāyinām\lem}
}\Bfootnote{%
%
\textit{Vajrāvalī} 33. \textit{Darpaṇābhiṣekavidhi}:
\textit{tato darpaṇam ādāya āḥ-kāreṇa mantritaṃ darśayan śiṣyam} \textemdash\
\textit{pratibimbasamā dharmā acchāḥ śuddhā hy anāvilāḥ |
agrāhyānabhilāpyāś ca hetukarmasamudbhavāḥ} || 1 ||
\textit{darpaṇavad vajrasattvas teṣv acchaḥ śuddho hy anāvilaḥ} |
\textit{hṛdi tiṣṭhati te vatsa sarvabuddhādhipaḥ svayam} || 2 ||
\textit{evaṃ jñātvā tu vai dharmān niḥsvabhāvān anālayān |
kuru sattvārtham atulaṃ jāto 'sy urasi tāyinām} || 3 ||
\textit{iti vadet.} (p.\ 438)
$\bullet$ teṣv acchaḥ] \emn\ following \cod\ C; te svacchaḥ ed.
As mentioned above, the source of the first verse is Dīpaṃkarabhadra's \textit{Guhyasamājamaṇḍalavidhi} v.288.
%
}} }%
\pend

\pstart\noindent
\Skt{\edtext{\restored{tato ghaṇṭāmādāya vādayan}}{\lemma{
	{\rm tato ghaṇṭāmādāya vādayan\lem}
}\Cfootnote{
	tentative restoration by RT. 
	\MSK\ only cited Tib.:  \textit{de nas dril bu blans la dkrol te}.
}}/} 
\pend

\verse
\Skt{%
\edtext{\edtext{\supplied{ākāśala}}{\lemma{
	{\rm ākāśala-\lem}
}\Cfootnote{
	Supplied based on the \textit{Vajrāvalī}.
}}\edtext{kṣaṇaṃ}{\lemma{
	{\rm -kṣaṇaṁ\lem}
}\Dfootnote{
	\emn\ \MSK\ \sil;
	\textit{kṣaṇa} \cod
}} sarva\edtext{mākāśaṃ cāpya}{\lemma{
	{\rm ākāśaṁ cāpy\lem}
}\Dfootnote{
	\emn\ $\leftarrow$ VĀ;
	\textit{ākāśaś cāpy} \cod;
	\MSK\ emends \textit{ākāśaś cāpy}.
}}lakṣaṇam/\\
ākāśa\edtext{samatāyogā}{\lemma{
	{\rm -samatāyogāt\lem}
}\Dfootnote{
	\emn\ \MSK\ \sil\ ($\leftarrow$ VĀ);
	\textit{samātāyogāt} \cod
}}tsarvāgrasamatā \edtext{sphuṭā}{\lemma{
	{\rm sphuṭā || iti\lem}
}\Dfootnote{
	\emn\ \MSK\ (sphuṭeti ||);
	sphuṭo ti MS
}}//}{\lemma{
	{\rm ākāśa-\dots\ sphuṭā\lem}
}\Bfootnote{
	Quoted in the \textit{Vajrāvalī} (42. \textit{Anujñāvidhiḥ}), vol. 2, p. 460:
	\textit{ākāśalakṣaṇaṃ sarvam ākāśaṃ cāpy alakṣaṇam |
	ākāśasamatāyogāt sarvāgrasamatā sphuṭā ||};
	\textit{Sarvabuddhasamāyogaḍākinījālaśaṃvara} 9.231:
	\textit{ākāśalakṣaṇaṃ sarvā[=am] cākāsarvāpālakṣaṇam [=ākāśaś cāpy alakṣaṇaḥ] |{|} 
māyopamaṃ ca sarva[ṃ] vai traidhātukam aśeṣataḥ [||]}
}}}
\pend

\pstart\noindent
\Skt{%
iti ghaṇṭāṃ dadyāt/ 
{\ah} \edtext{ghaṇṭāpra\restored{dānahṛdayamantraṃ syā}t/}{\lemma{
	{\rm ghaṇṭāpradānahṛdayamantraṃ syāt\lem}
}\Cfootnote{
%%
\textit{ghaṇṭāpra+ + + + + + + + + + +t} \cod\
\MSK\ edits \textit{[sbyin pa'i sñiṅ po yin no /] [tataḥ]}, but the restoration \textit{tataḥ} (inspired by Tib.) cannot be correct as a \textit{-t} is clearly visible at the end. We believe one should restore \textit{pradānahṛdayam bhavati |}, perhaps followed by \textit{tasmāt}. But since \textit{tasmāt} is very rare in our text, we remain uncertain.
%\textcolor{red}{Or, to be reconstructed as \textit{pradānahṛdayenābhimantrayet}, although this is not supported by Tib.\ or any other parallels.
%Or \textit{pradānahṛdayam uccārayet}}
%%
}}} 
\pend

\pstart
\Skt{\edtext{sarvatathāgatānanurāgayasveti}{\lemma{
	{\rm sarvatathāgatān anurāgayasveti\lem}
}\Dfootnote{
	\emn\ ($\leftarrow$ VĀ);
	\textit{sarvvatathāgatārāgayasveti} \cod.
	\MSK\ accepts the text as it stands in the ms.
}}/ 
\edtext{caturdikṣu}{\lemma{
	{\rm caturdikṣu\lem}
}\Dfootnote{
	\emn\ \MSK;
	catudikṣu \cod
}} catvāraḥ \edtext{śarāḥ}{\lemma{
	{\rm śarāḥ\lem}
}\Dfootnote{
	\emn\ \MSK\ \sil;
	\textit{sarāḥ} \cod
}} kṣeptavyāḥ/ 
\edtext{eke tūrdhva}{\lemma{
	{\rm eke tūrdhvam\lem}
}\Dfootnote{
	\conj;
	\textit{ekenaurddham} \cod;
	\textit{ekenordhvam} \emn\ \MSK;
	\MSK\ reads \textit{ekenairddham}.
}}\edtext{\textcolor{red}{mākāraṇīya}}{\lemma{
	{\rm ākāraṇīyam\lem}
}\Dfootnote{
	\emn\ ākṣepaṇīyam? \MSK\ accepts the text as it stands in the ms.
	This should be \textit{āvāraṇīyam} or \textit{uccāraṇīyam}?
}}\edtext{madhastācca}{\lemma{%
	{\rm adhastāc\lem}
}\Dfootnote{%
	\emn\ \MSK;
	\textit{adhastā} \cod
}} hoḥ iti/ }
\pend

\pstart
\Skt{\edtext{puna}{\lemma{
	{\rm punar\lem}
}\Dfootnote{
	\emn\ \MSK;
	\textit{puna} \cod
}}rdarpaṇamādāyaivaṃ vadet/}
\pend
\verse
\Skt{%
\edtext{darpaṇavadvajrasattva\edtext{ste'cchaḥ}{\lemma{
	{\rm te 'cchaḥ\lem}
}\Dfootnote{
	\emn\ \MSK\ \sil;
	\textit{te acchaḥ} MS
}} śuddho hyanāvilaḥ/\\
hṛdaye tiṣṭhate \edtext{vatsa}{\lemma{
	{\rm vatsa\lem}
}\Dfootnote{
	\emn\ \MSK;
	\textit{vaṁsa} \cod
}} sarvabuddhādhipaḥ svayamiti//}{\lemma{
	 {\rm darpaṇavad \dots\ svayam\lem}
}\Bfootnote{
	Quoted in the \textit{Vajrāvalī} (33. \textit{Darpaṇavidhi}):
	\textit{darpaṇavad vajrasattvas te svacchaḥ śuddho hy anāvilaḥ |
	hṛdi tiṣṭhati te vatsa sarvabuddhādhipaḥ svayaṃ ||}
}} }
\pend

\pstart\noindent
\Skt{\edtext{vaktavyaṃ}{\lemma{
	{\rm vaktavyaṁ\lem}
}\Dfootnote{
	\emn\ \MSK\ \sil;
	\textit{vaktavyac} MS
}} ca yo 'yaṃ \edtext{sarvatathāgatādhipati}{\lemma{
	{\rm sarvatathāgatādhipatir\lem}
}\Dfootnote{
	\emn\ \MSK\ \sil;
	\textit{sarvvatathāgata adhipatir} \cod
}}rbodhicitto'yamityavagaccha/ 
athāsya darpaṇagrahaṇe mantraḥ/ ā \edtext{vajrasattveti}{\lemma{
	{\rm vajrasattveti\lem}
}\Dfootnote{
	\emn\ \MSK;
	\textit{vajrāsatveti} \cod
}}/}
\pend

\smallskip

\pstart
\Skt{%
\edtext{tato \edtext{dharmacakraṃ}{\lemma{
	{\rm dharmacakraṁ\lem}
}\Dfootnote{
	\emn\ \MSK\ \sil;
	\textit{dhammacakraṁ} \cod
}} pādayormadhye saṃsthāpya \edtext{śaṅkhaṃ}{\lemma{
	{\rm śaṅkhaṁ\lem}
}\Dfootnote{
	\emn\ \MSK\ \sil;
	\textit{ṣaṅkhañ} \cod
}} ca dakṣiṇahaste \edtext{dattvaivaṃ}{\lemma{
	{\rm dattvaivaṁ\lem}
}\Dfootnote{
	\emn\ \MSK;
	\textit{datvevam} \cod
}} vadet/}{\lemma{%
	{\rm tato \dots\ vadet\lem}
}\Bfootnote{%
%
The opening prose and following 3 stanzas of this section are also founds in the \textit{abhiṣeka} section of the \textit{Kriyāsaṃgrahapañjikā}.
KSP chapter 6: \textit{tato dharmacakraṃ pādayor madhye saṃsthāpya śaṇkaṃ ca dakṣiṇahaste dattvāivaṃ vadet} (\textsc{Sakurai} 1996: 509).
%
}} } 
\pend
\verse
\Skt{%
\edtext{adyaprabhṛti sahacittotpādamātreṇa dharmacakraṃ \edtext{pravartaya}{\lemma{
	{\rm pravartaya\lem}
}\Dfootnote{
	\emn;
	\textit{pravarttaye} \cod;
	\MSK\ emends \textit{pravartayet}.
}}/\\
\edtext{āpūrya hi}{\lemma{
	{\rm āpūrya hi\lem}
}\Dfootnote{
	\emn\ \MSK\ \sil\ ($\leftarrow$ VĀ);
	\textit{āpūrya} \cod
}} \edtext{samantācca}{\lemma{
	{\rm samantāc ca\lem}
}\Dfootnote{
	\emn\ ($\leftarrow$ VĀ);
	\textit{samantāc} \cod;
	\textit{samantād vai} \MSK
}} \edtext{dharmaśaṅkha}{\lemma{
	{\rm dharmaśaṅkham\lem}
}\Dfootnote{
	\emn\ \MSK\ ($\leftarrow$ VĀ); 
	\textit{dharmmacakram} \cod
}}manuttaram//}{\lemma{%
	{\rm adyaprabhṛti \dots\ anuttaram\lem}
}\Bfootnote{%
%
KSP chapter 6:
\textit{adyaprabhṛti saha cittotpādamātreṇa dharmacakraṃ pravartaya |
āpūrya samantāc ca dharmaśaṇkham anuttaram ||} (\textsc{Sakurai} 1996: 509–510);
VĀ 42.1: \textit{adyaprabhṛti sahacittotpādamātreṇa dharmcakraṃ pravartaya |
āpūrya hi samantāc ca dharmaśaṅkham anuttaram ||} \textsc{Mori} 2009: 460)
%
}}\\
\edtext{na te \edtext{kāṅkṣā}{\lemma{%
	{\rm kāṅkṣā\lem}
}\Dfootnote{%
	\emn;
	\textit{kāṅkṣa} \cod
}}vimatirvā \edtext{nirviśaṅkena}{\lemma{
	{\rm nirviśaṅkena\lem}
}\Dfootnote{
	\emn\ \MSK\ \sil;
	\textit{nninniśaṅkona} \cod
}} cetasā/\\
prakāśaya sadā loke \edtext{mantracaryānayaṃ}{\lemma{
	{\rm mantracaryānayaṁ\lem}
}\Dfootnote{
	\emn\ \MSK;
	\textit{mantrācayānayam} \cod.
%	CHECK reading of KSP.
%	\textcolor{red}{mantrācayānayam and vidhim are separate words!!}
}} vidhim//}{\lemma{%
	{\rm na te \dots\ vidhim\lem}
}\Bfootnote{%
%
KSP chapter 6:
\textit{na te kāṅkṣā vimatir vā nirviśaṇkena cetasā |
prakāśaya sadā loke mantracaryā*nayaṁ}(\emn; \textit{naya-} ed.) \textit{vidhim ||} (\textsc{Sakurai} 1996: 510)
%
}}\\
\edtext{evaṃ \edtext{kṛtajño}{\lemma{
	{\rm kṛtajño\lem}
}\Dfootnote{
	\emn\ \MSK\ ($\leftarrow$ KSP);
	\textit{kṛjño} \cod;
	\MSK\ reads \textit{kṛjñe}.
}} \edtext{buddhānāmupakārīti}{\lemma{}\Efootnote{
	upa[64r1]kārīti
}} \edtext{gīyase}{\lemma{
	{\rm gīyase\lem}
}\Dfootnote{
	\emn\ \MSK;
	\textit{bhīyase} \cod
}}/\\
te ca vajradharāḥ sarve rakṣanti tava sarvata iti//}{\lemma{%
	{\rm evaṃ \dots\ sarvata\lem}
}\Bfootnote{%
%
KSP chapter 6:
\textit{evaṃ kṛtajño buddhānām upakārīti gīyate |
te ca vajradharāḥ sarve rakṣanti tāv sarvaśaḥ ||} (\textsc{Sakurai} 1996: 510)
%
}} }
\pend 
\pstart\noindent
\Skt{%
\edtext{vajracakravajrabhāṣau}{\lemma{%
	{\rm vajracakravajrabhāṣau\lem}
}\Dfootnote{%
	\emn\ following Tib.;
	\textit{vajracakrabhāṣau} \cod
}} dharmākṣarasahitau \edtext{prayoktavyau}{\lemma{
	{\rm prayoktavyau\lem}
}\Dfootnote{
	\emn\ \MSK;
	\textit{prayoktavyo} \cod
}}/  
punaḥ/}
\pend

\verse
\Skt{%
\edtext{sarvasattvahitārthāya sarvalokeṣu sarvataḥ/\\
yathāvinayato \edtext{viśvaṃ}{\lemma{
	{\rm viśvaṁ\lem}
}\Dfootnote{
	\emn\ \MSK;
	\textit{visvam} \cod
}} dharmacakraṃ \edtext{pravartyatām}{\lemma{
	{\rm pravartyatām\lem}
}\Dfootnote{
	\emn;
	\textit{pravarttatāṁ} \cod;
	\MSK\ edits \textit{pravartatām}, but translate as 2nd person imp. form.
}}//}{\lemma{%
	{\rm sarvasattvahitārthāya \dots\ dharmacakraṃ pravartyatām\lem}
}\Bfootnote{
%
STTS 2998–3011 teaches the set of the five verses, i.e.\ those of \textit{dharmacakra}, \textit{vajracakra}, \textit{krodhacakra}, \textit{padmacakra}, and \textit{maṇicakra}. 
STTS 2999:
\textit{sarvasattvahitārthāya sarvalokeṣu sarvataḥ |
[yathāvinayato viśvaṃ dharmacakraṃ pravartyatām] ||}
(The words in the square brackets are Horiuchi's reconstruction based upon the other four verses.);
%
KSP chapter 6:
\textit{sarvasattvahitārthāya sarvalokeṣu sarvataḥ |
yathāvinayato viśvaṃ dharmacakraṃ pravartaya ||}
(\textsc{Sakurai} 1996: 510)
= VĀ 42.2: (\textsc{Mori} 2009: 462)
= \textit{Viṃśatividhi} (\textcolor{red}{LOCATION TANAKA's Ed})
%
}}\\
sarvasattvahitārthāya sarvalokeṣu sarvataḥ/\\
yathāvinayato viśvaṃ vajracakraṃ \edtext{pravartyatām}{\lemma{
	{\rm pravartyatām\lem}
}\Cfootnote{
	For unclear reasons, this is emended to \textit{pravartatām} by \MSK.
}}// }\\
\Skt{%
evaṃ krodhapadmamaṇicakraṃ \edtext{pravartyatāmiti}{\lemma{
	{\rm pravartyatām iti\lem}
}\Cfootnote{
	For unclear reasons, this is emended to \textit{pravartatām} by \MSK.
}}  }
\pend

\pstart\noindent
\Skt{gāthāpañcakenānujñāṃ dadyāditi/}
\pend

\verse
%\edtext{}{\lemma{%
%	{\rm tataś codgatayā \dots\ varadānābhinaya iti\lem}
%}\Bfootnote{%
%	See also \textit{Guhyasamājamaṇḍalavidhi} vv.368 – 369:
%	\textit{tatas tathāgato bhūtvā vyākuryād udgatayānayā |
%	hṛnmuṣṭicīvarā vāmā dakṣiṇā tu varapradā ||
%	oṁ esāhaṁ vyākaromi tvāṁ vajrasattvas tathāgataḥ |
%	bhavadurgatitoddhṛtya atyantabhavasiddhaye ||} 
%}}%
\Skt{%
\edtext{\mbox{}\edtext{tataścodgatayā sarvān}{\lemma{
	{\rm tataś codgatayā sarvān\lem}
}\Dfootnote{
	\emn;
	\textit{tataḥ svaugaṁtayā sarvvāṁ} \cod;
	\MSK\ reads \textit{tataḥ śvogaṁtayā sarvāṁ} and 
	emends \textit{tataḥ svodgāthayā sarvaṁ}.
}} sarvabuddhātmabhāvakaḥ}{\lemma{
	{\rm tataḥ \dots\ -bhāvakaḥ\lem}
}\Cfootnote{
	\MSK\ does not recognize this passage as metrical. Parallel in KSP. 
	See also \textit{Hevajrasekaprakriyā}, \textit{Ratnāvalī} (a commentary on the \textit{Kṛṣṇayamāri}), \textit{Vāgīśvarakīrti}'s \textit{Saṁkṣiptābhiṣekavidhi}, \textit{Vajrāvalī}.
}}/\\
\edtext{vajranāmābhiṣekaistu \edtext{vyākuryā}{\lemma{
	{\rm vyākuryād\lem}
}\Dfootnote{
	\emn\ \MSK;
	\textit{vyākuyād} \cod
}}dvai \edtext{tathāgatān}{\lemma{
	{\rm tathāgatān\lem}
}\Dfootnote{
	\textit{tathāgatam} \emn\ \MSK;
	\textit{tathāgatām} \cod
}} }{\lemma{%
	{\rm vajranāmābhiṣekais \dots\ tathāgatān\lem}
}\Bfootnote{%
	Samāyoga 9.427 \textcolor{red}{TEXT!!}
}}//\\ 
{\om} eṣo 'haṃ vyākaromi tvāṃ vajrasattvastathāgataḥ/\\
\edtext{bhavadurgatitoddhṛtya atyantabhavasiddhaye}{\lemma{
	{\rm atyantabhavasiddhaye\lem}
}\Dfootnote{
	\emn;
	\textit{bhavadurggatitoddhṛtya atyantabhavasiddhayet} \cod;
	\MSK\ emends \textit{bhavadurgatita uddhṛtyātyantabhavaṁ siddhayet}.
}}//\\
he vajranāma \edtext{tathāgata}{\lemma{
	{\rm tathāgata\lem}
}\Dfootnote{
	\emn\ \MSK;
	\textit{tathāga} \cod
}} siddhya samayastvaṃ \edtext{bhūrbhuvaḥ}{\lemma{
	{\rm bhūr bhuvaḥ\lem}
}\Dfootnote{
	\emn\ \MSK;
	\textit{bhūbhruvaḥ} \cod
}} svaḥ/ iti /}
\pend
\pstart\noindent
\Skt{%
tatreya\edtext{mudgatā}{\lemma{
	{\rm udgatā\lem}
}\Cfootnote{
	\MSK\ emends \textit{udgāthā}.
}}/ 
\edtext{tathāgata\edtext{muṣṭi}{\lemma{
	{\rm -muṣṭi-\lem}
}\Dfootnote{
	\emn\ \MSK\ \sil;
	\textit{muṣṭhi} \cod (CHECK!!)
}}dvayaṃ baddhvā vāmacīvara\edtext{karṇikādhāraṇābhinayo}{\lemma{
	{\rm -karṇikādhāraṇābhinayo\lem}
}\Cfootnote{
	Thus also KSP; \MSK\ emends \textit{-karṇikadhāraṇābhinayo}.
}} dakṣiṇena varadānābhinaya iti/}{\lemma{%
	{\rm tathāgatamuṣṭidvayaṃ \dots\ varadānābhinaya iti\lem}
}\Bfootnote{%
%
%
KSP:
\textit{tatreyaṃ samudgatā tathāgatamuṣṭidvayaṃ baddhvā vāmacīvarakarṇikādhāraṇābhinayo hṛdi dakṣiṇakare varadābhinayaḥ}
(\textcolor{red}{LOCATION!!}).
See also \textit{Guhyasamājamaṇḍalavidhi} vv.368–369:
\textit{tatas tathāgato bhūtvā vyākuryād udgatayānayā |
hṛnmuṣṭicīvarā vāmā dakṣiṇā tu varapradā || 368 ||
oṁ esāhaṁ vyākaromi tvāṁ vajrasattvas tathāgataḥ |
bhavadurgatitoddhṛtya atyantabhavasiddhaye ||} 369 ||;
\textit{Viṃśatividhi} 
\textcolor{red}{Tanaka Ph.D thesis 703. CHECK RECENT PUBLICATION!!}
%
}} }
\Skt{% 
tata\supplied{ścai}vaṃ \edtext{śiṣyebhyo}{\lemma{
	{\rm śiṣyebhyo\lem}
}\Dfootnote{
	\emn\ \MSK\ \sil;
	\textit{siṣyebhyo} \cod
}} vaktavyam/} 
%\pend
%
%\verse
\Skt{%
\edtext{yasyānayodgatayā mahāmudrayā paramarahasyavidhāne vyākaraṇaṃ \edtext{kriyate}{\lemma{%
	{\rm kriyate\lem}
}\Dfootnote{%
	\emn;
	\textit{kṛyate} \cod;
	\MSK\ accepts \textit{kṛyate}.
}}/ 
tasya \edtext{vajrasattvādayaḥ sarvatathā\supplied{gatāḥ savajradharāḥ samahābo}dhisattvaparṣanmaṇḍalāḥ}{\lemma{
	{\rm vajrasattvādayaḥ \dots\ -maṇḍalāḥ\lem}
}\Dfootnote{
	restored following KSP;
	\textit{vajrasatvādayaḥ sarvatath + + + + + + + + + + + + + dhisattvaparṣanmaṇḍalāḥ} \cod;
	\MSK\ restores \textit{vajrasattvādayaḥ sarvatathāgatāḥ sarvavajradharabodhisattvaparṣanmaṇḍalāḥ} based on Tib.
}} \edtext{\mbox{}\edtext{sarva}{\lemma{
	{\rm sarva-\lem}
}\Dfootnote{
	\cod;
	omitted in \MSK
}}tathāgatamaṇḍale}{\lemma{}\Efootnote{
	sarva[64v1]tathāgatamaṇḍale
}} svasamayācāryāḥ \edtext{samame}{\lemma{%
	{\rm samam\lem}
}\Dfootnote{%
	\emn;
	\textit{samay} \cod
}}kakaṇṭhenā\edtext{nuttarāyāṃ}{\lemma{%
	{\rm -nuttarāyāṃ\lem}
}\Dfootnote{%
	\emn;
	\textit{-nuttarāyā} \cod
}}samya\supplied{ksaṃ}bodhau \edtext{vyāka\supplied{raṇaṃ kurvate / yadutāsyaivodgatāyā mahāmudrāyāḥ paramarahasyo}ttamasiddhyadhiṣṭhānenāsya}{\lemma{%
	{\rm vyākaraṇaṃ \dots\ nāsya\lem}
}\Dfootnote{%
	restored following KSP;
	\textit{ + + + + + + + + + + + + + + + + + + + + + + + + + + ttamasiddhyadhiṣṭhānenāsya} \cod
}} ca mantrasya baleneti śraddhātavyam/}{\lemma{%
	{\rm yasyānayodgatayā \dots\ śraddhātavyam\lem}
}\Bfootnote{%
%
KSP
\textit{yasyānayodgatayā mahāmudrayā paramarahasyābhidhāne vyākaranaṃ kriyate tasya vajrasattvādayaḥ sarvatathāgatāḥ savajradharāḥ samahābodhisattvaparṣannamaṇḍalāḥ sarvatathāgatamaṇḍalebhyaḥ svasamayācāryāḥ samam ekakaṇṭhenānuttarāyāṃ samyaksambodhau vyākaraṇaṃ kurvate. yad utāsyaivodgātāyā mahāmudrāyāḥ paramarahasyottamasiddhyadhiṣṭhānenāsya ca mantrasya baleneti śraddhātavyam} (\textcolor{red}{LOCATION!!});
\textit{Saṃkṣiptābhiṣekavidhi}:
anayā dharmāhṛtayā mudrayā vyākaraṇaṃ kriyate sa sarvatathāgataiḥ sarvabodhisattvāryaparṣanmaṇḍalair ekakaṇṭhenānuttarāyā samyaksambodhau vyākriyate | asyā evāhṛtāyā mahāmudrāyāḥ prabhāvato asya mantrasya ca baleneti śraddhātavyam ||
\textcolor{red}{(Sakurai's edition section 5)}
%
}} }
%\pend
%\pstart\noindent
\Skt{%
\edtext{śeṣa āśvāsaḥ}{\lemma{%
	{\rm śeṣa āśvāsaḥ\lem}
}\Dfootnote{%
	\emn;
	\textit{seṣa āśvāsa |} MS
}} śrīparamādye draṣṭa\supplied{vyaḥ}/}
\pend

\pstart
\Skt{%
\edtext{\supplied{tataḥ sarvamaṇḍalaguhyasamayajñānaṃ} śikṣayet}{\lemma{
	{\rm tataḥ \dots\ śikṣayet\lem}
}\Dfootnote{
	restored after STTS \S\S\ 607–608;
	\textit{+ + + + + + + + + + + +  [śikṣa]yet} \cod
}\Bfootnote{%
	STTS  607:
	\textit{tataḥ sarvamaṇḍalaguhyasamayajñānaṃ śikṣayet}.
}}/}
\pend
%
\verse
\Skt{%
\edtext{\edtext{\supplied{virāga}}{\lemma{
	{\rm \supplied{virāga-}\lem}
}\Dfootnote{
	restored after STTS \S\S\ 607–608;
	\textit{+++} \cod
}}sadṛśaṃ pāpamasattvāsti tridhātuke/\\
tasmā\edtext{tkāmavirāgitvaṃ}{\lemma{
	{\rm kāmavirāgitvaṁ\lem}
}\Dfootnote{
	\cod\ ($\leftarrow$ STTS);
	\textit{kāmavirāgatvaṁ} \MSK; 
	The \textit{i} sign is faint but visible on our photos.
}} na \edtext{kāryaṃ}{\lemma{
	{\rm kāryaṁ\lem}
}\Dfootnote{
	\emn\ \MSK\ \sil\ ($\leftarrow$ STTS);
	\textit{kāyam} \cod
}} bhavatā punaḥ//}{\lemma{
	{\rm virāga\dots\ punaḥ\lem}
}\Bfootnote{
	STTS 608:
	\textit{virāgasadṛśaṃ pāpam anyan nāsti tridhātuke |
	tasmāt kāmavirāgitvaṃ na kāryaṃ bhavatā punaḥ ||};
	\textit{Paramādya}, \textit{Mantrakhāṇda}
	\textcolor{red}{(CHECK!!)}
}} }
\pend

\pstart\noindent
\Skt{%
\edtext{mahāsamaya hana phaḍiti}{\lemma{%
	{\rm mahāsamaya hana phaḍ\lem}
}\Bfootnote{%
	STTS 608:
	\textit{mahāsamaya hana phaṭ}
}} /
imaṃ mahāsamayamantraṃ \edtext{coccārayet}{\lemma{
	{\rm coccārayet\lem}
}\Dfootnote{
	\emn\ \MSK\ \sil;
	\textit{cocārayet} MS
}}/}
\pend

\pstart
\Skt{\edtext{tato \edtext{mantraṃ}{\lemma{
	{\rm mantraṁ\lem}
}\Dfootnote{
	\emn\ \MSK\ \sil;
	\textit{mantra} MS
}} dattvā svadevatā\edtext{caturmudrā}{\lemma{
	{\rm -caturmudrā-\lem}
}\Dfootnote{
	\emn\ \MSK\ \sil;
	\textit{-catumudrā-} \cod
}}jñānaṃ śikṣayet/ 
anena vidhinā vaktavyam/ 
na kasya ci\edtext{ttvayānyasyaiṣāṃ}{\lemma{
	{\rm tvayānyasyaiṣāṁ\lem}
}\Dfootnote{
	\emn\  ($\leftarrow$ STTS);
	\textit{tvayānyasyaiṣā} \cod;
	\MSK\ emends \textit{tvayānyasyāsāṁ}.
}} mudrāṇā\edtext{makovidasyaikatarāpi}{\lemma{
	{\rm akovidasyaikatarāpi\lem}
}\Dfootnote{
	\emn\ \MSK;
	\textit{akovidanya | ekatarāpi} \cod;
	\MSK\ read \textit{akovidamya | ekatarāpi} and emend \textit{akovidasyaikatarāpi}.
}} mudrā darśayitavyā/ 
tatkasya hetoḥ/
tathā hi te sattvā \edtext{adṛṣṭamahāmaṇḍalāḥ}{\lemma{
	{\rm adṛṣṭamahāmaṇḍalāḥ\lem}
}\Dfootnote{
	\emn\ \MSK\ \sil\ ($\leftarrow$ STTS);
	\textit{adṛṣṭhamahāmaṇḍalā |} \cod
}} \edtext{santo}{\lemma{
	{\rm santo\lem}
}\Dfootnote{
	\cod\ ($\leftarrow$ STTS);
	\MSK\ misreads \textit{sattvā}.
}} \edtext{mudrābandhaṃ}{\lemma{
	{\rm mudrābandhaṁ\lem}
}\Dfootnote{
	\emn\ \MSK\ \sil\ ($\leftarrow$ STTS);
	\textit{mudrāvaddhaṁ} \cod
}} prayokṣyanti/ 
tadā teṣāṃ na tathā siddhirbhaviṣyati/
tataste vicikitsā \edtext{prāptā}{\lemma{%
	{\rm prāptā\lem}
}\Dfootnote{%
	\emn;
	\textit{prāpta} \cod
}} \edtext{viṣamā}{\lemma{
	{\rm -viṣamā-\lem}
}\Dfootnote{
	\emn\ ($\leftarrow$ STTS);
	\textit{-viśamā-} \cod;
	 \MSK\ accepts the reading as it stands.
}}parihāreṇa śīghrameva kālaṃ kṛtvāvīcau mahānarake pateyu\edtext{stava cāpāyagamanaṃ}{\lemma{
	{\rm tava cāpāyagamanaṁ\lem}
}\Dfootnote{
	\emn\ ($\leftarrow$ STTS);
	\textit{tava yāpāyagamanaṁ} \cod;
	\MSK\ read \textit{tavayāpāpagamanaṁ}, and emends \textit{tvayā pāpagamanaṁ}.
}} syāditi}{\lemma{%
	{\rm tato \dots\ syād iti\lem}
}\Bfootnote{%
%
STTS 250:
\textit{tato hṛdayaṃ dattvā svakuladevatācaturmudrājñānaṃ śikṣayet.
anena vidhinā vaktavyam.
na kasyacit tvayānyasyaiśāṃ mudrāṇām akovidasyaikasya ekatarāpi mudrā darśayitavyā.
tat kasya hetoḥ.
tathā hi te sattvā adṛṣṭamahāmaṇḍalāḥ santo mudrābandhaṃ prayokṣayanti, tadā teṣāṃ na tathā siddhir bhaviṣyati.
tatas te vicikitsā prāptā viṣamāparihāreṇa śīghram eva kālaṃ kṛtvāvīcau mahānarake pateyuḥ.
tava cāpāyagamanaṃ syād iti}.
}}/}
%\pend
%
%\pstart
\Skt{%
tato \edtext{lāsyādyaṣṭa}{\lemma{
	{\rm lāsyādyaṣṭa-\lem}
}\Dfootnote{
	\emn\ \MSK\ (sil. for \textit{-aṣṭa-});
	\textit{yāsyādyaṣṭha-} \cod
}}vidhapūjayā puṣpādibhiśca \edtext{sarvatathāgatān}{\lemma{
	{\rm sarvatathāgatān\lem}
}\Dfootnote{
	\emn\ \MSK\ \sil;
	\textit{sarvatathāgatāṁ} \cod
}} saṃpūjya 
\edtext{\mbox{}sarve \edtext{yathāśaktyā}{\lemma{
	{\rm yathāśaktyā\lem}
}\Dfootnote{
	\emn\ \MSK;
	\textit{yathāśakyā} \cod
}} \edtext{pūjayantviti}{\lemma{}\Efootnote{
	pūjayan[65r1]tv iti
}}/
\edtext{sarvatathāgatā}{\lemma{
	{\rm sarvatathāgatān\lem}
}\Dfootnote{
	\emn\ \MSK\ \sil;
	\textit{savatathāgatāṁ} \cod
}}nvijñāpya yathecchayā dhūpādibhiśca \edtext{pūjāṃ}{\lemma{
	{\rm pūjāṁ\lem}
}\Dfootnote{
	\emn\ \MSK;
	\textit{pūjā} \cod
}} kārayitvā \edtext{yathāpraviṣṭā}{\lemma{
	{\rm yathāpraviṣṭān\lem}
}\Dfootnote{
	\emn\ \MSK\ (sil. for \textit{ṣṭa});
	\textit{yathāpraviṣṭhā} \cod 
}}nyathāvibhavataḥ sarvarasāhārādibhiḥ \edtext{sarvopakaraṇai}{\lemma{
	{\rm sarvopakaraṇair\lem}
}\Dfootnote{
	\emn\ \MSK\ ($\leftarrow$ STTS);
	sarvvopakaraṇe MS
}}rmahā\edtext{maṇḍala}{\lemma{
	{\rm -maṇḍala-\lem}
}\Dfootnote{
	\cod;
	\MSK\ reads \textit{-maṇḍale-} and emends \textit{-maṇḍala-}.
}}\edtext{niryātitaiḥ}{\lemma{
	{\rm niryātitaiḥ\lem}
}\Dfootnote{
	\emn\ \MSK\ \sil\ ($\leftarrow$ STTS);
	\textit{niḥyātitaiḥ} \cod 
}} \edtext{saṃtarpyedaṃ}{\lemma{
	{\rm saṁtarpyedaṁ\lem}
}\Dfootnote{
	\emn\ \MSK\ \sil\ ($\leftarrow$ STTS);
	\textit{santarpyeda} \cod
}} siddhivajravrataṃ \edtext{dadyā}{\lemma{
	{\rm dadyāt\lem}
}\Dfootnote{
	\emn\ \MSK\ ($\leftarrow$ STTS);
	dadyā | MS
}}\edtext{didaṃ tadityādi}{\lemma{
	{\rm idaṁ tad ityādi\lem}
}\Cfootnote{
	\MSK\ reads \textit{idaṁ tad (sarvabuddhatvam) ityādi} after Tib.
}}/ 
tataḥ sarveṣāṃ punarapi na kasya cidvaktavyamiti \edtext{śapathāhṛdayamākhyeyam}{\lemma{
	{\rm śapathāhṛdayam ākhyeyam\lem}
}\Dfootnote{
	\cod;
	\MSK\ needlessly emends \textit{śapathahṛdayam}; 
	STTS several times has the word \textit{śapathāhṛdaya}.
}}//}{\lemma{
	{\rm sarve \dots\ ākhyeyam\lem}
}\Bfootnote{
%
STTS 314–316:
\textit{tataḥ sarve yathāśaktyā pūjayantv iti.
sarvatathāgatān vijñāpya, yathecchayā dhūpādibhiḥ pūjāṃ kārayitvā, yathā praviṣṭān yathāvibhavataḥ sarvarasāhāravihārādibhiḥ sarvopakaraṇair mahāmaṇḍalaniryātitaḥ saṃtarpyedaṃ sarvatathāgatavajravrataṃ dadyāt.
idaṃ tat sarvabuddhatvaṃ vajrasattvakare sthitam |
tvayāpi hi sadā dhāryaṃ vajrapāṇidṛḍhavratam ||
oṃ sarvatathāgatasiddhivajrasamaya tiṣṭha eṣa *tvāṃ} (\emn; \textit{tv ā-} ed.) \textit{dhārayāmi vajrasattva hi hi hi hi hūṃ.
tataḥ sarveṣāṃ punar api na kasya cid vaktavyam iti.
śapathāhṛdayam ākhyeyam.}%
%
}}}
\pend

\pstart
\Skt{%
\edtext{tataḥ \edtext{praviṣṭān}{\lemma{
	{\rm praviṣṭān\lem}
}\Dfootnote{
	\emn\ \MSK\ \sil;
	\textit{praviṣṭhāṁ} \cod
}} saṃpreṣya}{\lemma{%
	{\rm tataḥ praviṣṭān saṃpreṣya\lem}
}\Bfootnote{%
	STTS 317: \textit{tato yathāpraviṣṭān saṃpreṣya}
}} \edtext{puna}{\lemma{
	{\rm punar\lem}
}\Dfootnote{
	\emn\ \MSK\ \sil;
	\textit{puna} \cod
}}\edtext{rnāmāṣṭaśatena}{\lemma{
	{\rm nāmāṣṭaśatena\lem}
}\Dfootnote{
	\emn\ \MSK\ \sil;
	\textit{nāmāṣṭhaśatena} \cod
}} saṃstutya lāsyādibhiḥ saṃpūjya \edtext{praṇamyārghaṃ}{\lemma{
	{\rm praṇamyārghaṁ\lem}
}\Dfootnote{
	\emn\ \MSK\ \sil;
	\textit{praṇamyā[r]gha} \cod
}} dattvābhipretasiddhaye kuśalaṃ \edtext{pariṇamya}{\lemma{
	{\rm pariṇamya\lem}
}\Dfootnote{
	\emn\ \MSK;
	\textit{pariṇamyapya} \cod;
	alternatively, we could emend \textit{pariṇāmya}, 
	but \textit{pariṇamya} is attested in this kind of context in KSP and \textit{Sādhanamālā}.
}} mudrāmokṣaṃ kṛtvā sattvavajrīṃ baddhvā \edtext{trivārān}{\lemma{
	{\rm trivārān\lem}
}\Dfootnote{
	\emn\ \MSK\ \sil;
	\textit{tri[vā]rāṁ} \cod 
}} \edtext{saptavārān}{\lemma{
	{\rm saptavārān\lem}
}\Dfootnote{
	\emn\ \MSK\ \sil;
	\textit{saptavārām} \cod 
}} vā maṇḍalaṃ pradakṣiṇīkṛtya pūrvavadvisarjanādikaṃ kṛtvā/ 
akāro mukhamityādinā \edtext{maṇḍalaṃ}{\lemma{
	{\rm maṇḍalaṁ\lem}
}\Dfootnote{
	\emn\ \MSK\ ($\leftarrow$ VĀ Tib.);
	\textit{maṇḍalamudrāṁ} \cod 
}} vikopayet/ 
\edtext{nirmālyādika}{\lemma{
	{\rm nirmālyādikam\lem}
}\Dfootnote{
	\emn\ \MSK\ \sil;
	\textit{nirmmalyādikam} \cod 
}}mudake \edtext{prakṣepyamiti}{\lemma{
	{\rm prakṣepyam iti\lem}
}\Dfootnote{%
	\emn
	\textit{prakṣe[pya]ti} \cod;
	\textit{prakṣepayati} \emn\ \MSK
%	or \textit{prakṣepya}? or \textit{prakṣepayet}?
}}/
\edtext{tata\edtext{ścaturhūṃkāreṇa}{\lemma{
	{\rm caturhūṁkāreṇa\lem}
}\Dfootnote{
	\emn\ \MSK\ \sil;
	\textit{catuhūṁkāreṇa} \cod 
}} sarvakīlānutpāṭya/
{\om} ruru sphuru jvala tiṣṭha siddhalocane sarvārthasādhani \edtext{svāhe\supplied{tyaṣṭottara}śa\supplied{taparijapta}kṣīreṇa}{\lemma{
	{\rm svāhe- \dots\ -kṣīreṇa\lem}
}\Cfootnote{
	\MSK\ restores \textit{svāhety aṣṭaśatajaptena kṣīreṇa}, 
	but this does not suffice to fill the gap; 
	Ānandagarbha uses the expression \textit{aṣṭottaraśataparijap-} 
	also at several other places.
}} \edtext{sarvakīlakānpratikṛtīśca}{\lemma{
	{\rm sarvakīlakān pratikṛtīś ca\lem}
}\Dfootnote{
	\emn\ ;
	\textit{sarvvakīlakām | pratikṛtīś ca} \cod;
	\textit{sarvakīlakān prakṛtiṁ ca }\MSK\ (sil. \emn\ / misreading \textit{ñc} for \textit{śc})
}} snāpaye\edtext{dgartāṃ}{\lemma{
	{\rm gartāṁś\lem}
}\Dfootnote{
	\emn;
	\textit{garttāś} \cod
}}ścāpūrayet//}{\lemma{%
	{\rm tataś \dots\ cāpūrayet\lem}
}\Bfootnote{%
\textit{Vajrāvalī}, section \textit{Maṇḍalopasaṁhāra}, ed. Mori:
\textit{tad anu oṁ vajrakīla utkīlaya sarvakīlān vajradharājñayā hūṁ hūṁ hūṁ hūṁ phaṭ hoḥ iti caturhūṁkāramantreṇa kīlakān utpādya oṁ ru ru sphuru jvala tiṣṭha siddhalocane sarvārthasādhanani svāhā iti sāṣṭaśatajaptakṣīreṇa snāpayet. kīlakadevatāś ca visarjayet. garttāṁś cāpūrayet.}
}} }
\pend

\pstart
\Skt{%
praveśadvārābhimukhaṃ \edtext{sārvakarmikakuṇḍaṃ}{\lemma{%
	{\rm sārvakarmikakuṇḍaṁ\lem}
}\Dfootnote{%
	\emn;
	\textit{sarvakarmikakuṇḍaṁ} \cod
}} kṛtvātmaśiṣya\edtext{bhūpāle\restored{bhyaḥ sarvasattvebhyaśca śāntikahomaṃ} kuryāditi}{\lemma{
	{\rm bhūpāle \dots\ kuryād iti\lem}
}\Dfootnote{
	\textit{bhūpāle + + + + + + + + + + + kuyād iti} \cod
}\Cfootnote{%
%
	The \textit{e} in \textit{bhūpāle} seems fairly clear, but was not read by MSK;  at the end, \MSK\ omits \textit{[ku]yād iti}. In the light of Tib., and the parallel passage for the sequence \textit{ātmaśiṣyabhūpāla} in 3.1 of the SVU (\textcolor{red}{CROSS REFERENCE!!}), we can propose the following conjectural restorations: \textit{-bhūpālebhyaḥ sarvasattvebhyaś ca śāntikahomaṁ kuryād iti; -bhūpāleṣu sarvasattveṣu ca śāntikahomaṁ kuryād iti; -bhūpālasarvasattvārthaṁ śāntikahomaṁ kuryād iti}.
%
}}/ }
\Skt{%
tataḥ \edtext{pradhānaśiṣyaṃ}{\lemma{
	{\rm pradhānaśiṣyaṁ\lem}
}\Dfootnote{
	\emn\ \MSK;
	\textit{p[r]adhānaśiṣyāṁ} \cod;
	\MSK\ does not notice the trace of an \textit{-r-} in the first \textit{akṣara}.
}\lemma{}\Efootnote{
	pradhāna[65v1]śiṣyaṁ
}} \edtext{vāmapārśve'vasthāpya}{\lemma{
	{\rm vāmapārśve 'vasthāpya\lem}
}\Cfootnote{
	\textit{vāmapārśve vasthāpya} \cod;
	\MSK\ misunderstand and emends \textit{vāmapārśvena sthāpya}.
}} yathāvadghṛtāhutiśataṃ \edtext{vajrasattvamantreṇa juhuyāt/ tato}{\lemma{
	{\rm vajrasattvamantreṇa juhuyāt/ tato\lem}
}\Dfootnote{
	\emn\ \MSK;
	\textit{vajrasatvamantreṇa juhu<ta>ttato} \cod;
	\MSK's correct emendation starts from a grave misreading \textit{-maṇḍalajupātan tato.}
}} \edtext{buddhalocanā\restored{japtānāṃ mudrāsahitānāṃ ghṛtadadhi}miśrāṇāṃ tilānā}{\lemma{
	{\rm buddhalocanā\dots\ -miśrāṇāṁ tilānām\lem}
}\Dfootnote{
	tentative restoration and emendation;
	\textit{vuddhalocanā + + + + + + + + + + + miśrāṇyatilānām} \cod;
	\textit{[buddhalocanāmantreṇa aṣṭaśatāhutiṁ kuryāt] / [de nas phyag rgya de ñid bcinṅs la źo daṅ miṣrālpatilānām]} \MSK;
	\MSK\ emends \textit{-miśrālpatilānām}. We tentatively propose to restore the gap as follows: \textit{-mantreṇa mudrāṁ baddhvā ghṛtadadhi-}.
%
}}\edtext{māhutiśatam}{\lemma{
	{\rm āhutiśatam\lem}
}\Dfootnote{
	\emn\ \MSK;
	\textit{āhutīśatam} \cod
}} /
tato \edtext{vajrayakṣajaptena vāriṇā}{\lemma{%
	{\rm vajrayakṣajaptena vāriṇā\lem}
}\Dfootnote{
	\emn;
	\textit{vajrayakṣajapte vāriṇā} \cod;
	\MSK\ emends \textit{vajrayakṣajaptavāriṇā}.
}} \edtext{mūrdhni}{\lemma{%
	{\rm mūrdhni\lem}
}\Dfootnote{
	\emn\ \MSK\ \sil;
	\textit{mūdhni} \cod
}} \edtext{paryukṣya}{\lemma{%
	{\rm paryukṣya\lem}
}\Dfootnote{
	\emn\ \MSK\ \sil;
	payukṣya \cod
}} vāmapāṇau tenaiva \edtext{rakṣāsūtrakaṃ}{\lemma{
	{\rm rakṣāsūtrakaṁ\lem}
}\Dfootnote{
	\emn;
	\textit{rakṣyāsūtrakam} \cod;
	\MSK\ accepts the reading of \cod.
}} \edtext{badhnīyāt / \restored{tatastasya hṛdayaṃ kare}ṇa}{\lemma{%
	{\rm badhnīyāt | tatas tasya hṛdayaṁ kareṇa\lem}
}\Dfootnote{%
	%restored;
	\textit{vadhnīyāt· + + + + + + + + + ṇa} \cod;
	\MSK\ restores \textit{badhnīyāt |  tatas tasya hṛdaye hastena}.
}} \edtext{spṛśan saptavārān}{\lemma{
	{\rm spṛśan saptavārān\lem}
}\Dfootnote{
	\emn\ \MSK\ (\sil\ for the second \textit{ṁ});
	\textit{spṛśaṁ saptavārāṁ} \cod
}} parijapet / 
\edtext{anyeṣāṃ tu}{\lemma{
	{\rm anyeṣāṁ tu\lem}
}\Dfootnote{
	\textit{anyeṣātra} \MSK, emended to \textit{anyeṣām atra}.
}} \edtext{yathoktasaptasaptāhutiṃ}{\lemma{
	{\rm yathoktasaptasaptāhutiṁ\lem}
}\Dfootnote{
	\emn\ ($\leftarrow$ Tib.: \textit{ji skad du bshad pa’i sbyin sreg lan bdun bdun});
	\textit{yatho<kta>ptasaptāhutīñ} \cod;
	\MSK\ does not notice the scribal cancellation sign, 
	and emends \textit{yathoktasaptasaptāhutīn}.
}} juhuyāt / 
tataḥ \edtext{paryukṣaṇa}{\lemma{
	{\rm paryukṣaṇa\lem}
}\Dfootnote{
	\textit{paryukṣaṇaṁ} \cod
	\textit{paryukṣaṇaṁ} \MSK
}}rakṣāsūtraṃ \edtext{ca}{\lemma{%
	{\rm ca\lem}
}\Dfootnote{
	\emn\ \MSK\ \sil;
	\textit{va} \cod
}} \edtext{hṛdayālambhanaṃ}{\lemma{%
	{\rm hṛdayālambhanaṁ\lem}
}\Dfootnote{%
	\textit{hṛdayālabhataṁ} \cod;
	\MSK\ emends \textit{hṛdayālabhanaṁ}.
}} ca kuryāditi //}
\pend

\bigskip

\pstart\noindent
{\large 4. Pariṇāmanā}
\pend

\bigskip

\verse
\Skt{%
\edtext{vajrasattvādisatsattva}{\lemma{
	{\rm vajrasattvādisatsattva-\lem}
}\Dfootnote{
	\emn;
	\textit{vajrasattvādisasattva-} \cod
%	Maybe \textit{vajrasattvādisasattva-} is to be understood as 'of the deities, beginning with Vajrasattva, who has/have \textit{sattva}'.
%
}}sarvasiddhi\edtext{pradāyikām}{\lemma{
	{\rm -pradāyikām\lem}
}\Dfootnote{
	\cod;
	\MSK\ emends \textit{-pradāyikam}, which is unlikely.
}}~/} \\
\Skt{%
sarva\edtext{vajrodayāṃ}{\lemma{
	{\rm -vajrodayāṁ\lem}
}\Dfootnote{
	\emn;
	\textit{vajropamāṁ} \cod;
	\textit{vajrodayaṁ} \MSK\ (\emn)
}} kṛtvā ya\edtext{nmayopacitaṃ}{\lemma{
	{\rm mayopacitaṁ\lem}
}\Dfootnote{
	\textit{mayotpacita} \cod
}} śubham~// \\
ānandagarbhavidyāgraḥ sarvasattvaikavāndhavaḥ /\\
\edtext{aśeṣa}{\lemma{
	{\rm aśeṣas\lem}
}\Dfootnote{
	\emn\ \MSK\ \sil;
	\textit{aseṣas} \cod
}}stena \edtext{loko'stu}{\lemma{
	{\rm loko 'stu\lem}
}\Dfootnote{
	\emn;
	\textit{lokās tu} \cod\ \MSK
}} mahāvajradharo vibhuḥ //}
\pend

\bigskip

\pstart\noindent
{\large 5. Samāpti}
\pend

\bigskip

\pstart
\Skt{%
śrīmadāryasarvatathāgatatattvasaṃgrahān mahāyānābhisamayān mahātantrarājād uddhṛtā vajradhātumahāmaṇḍalopayikā \edtext{sarvavajrodayā}{\lemma{%
	{\rm sarvavajrodayā\lem}
}\Dfootnote{%
	\emn;
	\textit{sarvavajrodakā} \cod
}} nāma \edtext{samāptā}{\lemma{%
	{\rm samāptā\lem}
}\Dfootnote{%
	\emn;
	\textit{samāptaḥ} \cod
}} //} 
\Skt{%
// \edtext{kṛteyaṃ mahāvajrācāyānandagarbhapādairiti}{\lemma{%
	{\rm kṛteyaṃ \dots\ -pādai riti\lem}
}\Dfootnote{%
	After this a \textit{pariṇāmanā} verse by the scribe:
	vilikhya puṇyaṃ yatprāptaṃ mayā sādhanasaṃskṛti |\\
	aseṣastena loko'yaṃ bhūyācchrīvajrasambhavaḥ ||
}} //}
\pend

%\medskip
%
%\verse
%\Skt{vilikhya puṇyaṃ yatprāptaṃ mayā sādhanasaṃskṛti /\\
%aseṣastena loko'yaṃ bhūyācchrīvajrasambhavaḥ // //}
%\pend




\endnumbering

%\newpage

%\theendnotes
\end{document}