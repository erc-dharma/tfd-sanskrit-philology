\renewcommand{\dnapp}[1]{}\renewcommand{\rmapp}[1]{#1}
\fejno=0\versno=0
\begin{center}{\Huge  V\textsubring{r}ṣasārasaṁgrahaḥ}\end{center}

\alfejezet{\textbf{prathamo 'dhyāyaḥ}}\jump\jump
\vers
\szam\bek\versno=0\fejno=1
\thispagestyle{empty}


\ujvers\nemsloka 
anādimadhyāntam anantapāraṁ
\dontdisplaylinenum

\nemslokab 
susūkṣmam avyaktajagatsusāram \danda\dontdisplaylinenum

\nemslokac 
harīndrabrahmādibhir āptam agraṁ
\dontdisplaylinenum

\nemslokad 
praṇamya vakṣye v\textsubring{r}ṣasārasaṁgraham \veg\dontdisplaylinenum
            \paral{\textit{{\normalfont Testimonia for this chapter:    \msCa\ ff.\thinspace 193v--195v,
                                                \msCb\ ff.\thinspace 201v--203v,
                                                \msCc\ ff.\thinspace 267r--270r,
                                                \msNa\ ff.\thinspace 1v--3v,
                                                \msNb\ exp.\thinspace 44 (then exp.\thinspace
                                                        43 lower and then upper leaves; 1.62cd--2.22 are missing),
                                                \msNc\ ff.\thinspace 209v--211v,
                                                \msNd\ ff.\thinspace 227v--229v; 
                                                \mssCaCbCc\ = \msCa + \msCb + \msCc; 
                                                lemmata in parentheses indicate a longer chuck of text for
                                                which there may be additional variants given in the apparatus}}}


\alalfejezet{janamejayavaiśampāyanasaṁvādaḥ}
\vers
\varr{
        \ \va °ntam ananta°\lem  \msCa\msCbpcorr\msCc\msNa\msNb\msNc\msNd\Ed; °ntam anta° \msCbacorr\oo
                 °pāraṁ\lem  \mssCaCbCc\msNc\Ed; °pāragaṁ \msNa\msNb\msNd
        \ \vb susūkṣma°\lem  \msCa\msCb\msNa\msNb\msNc\Ed; śusukṣma° \msCc\oo
                 °jagatsusāram\lem  \msCa\msCb\msNa\msNc\Ed; °jagaśusāraṁ \msCc, °jagatsurāsuraṁ \msNb,
                                        °jagatasusāram \msNd
        \ \vc °bhir āpta°\lem  \conj; °bhir āsa° \mssCaCbCc\msNa\msNb\msNc\msNd\Ed}

śatasāhasrikaṁ granthaṁ sahasrādhyāyam uttamam\thinspace{\dandab} \dontdisplaylinenum

parva cāsya śataṁ pūrṇaṁ śrutvā bhāratasaṁhitām \veg\dontdisplaylinenum
            \paral{\textit{\vc {\normalfont  cf. MBh 1.2.70ab: } etat parvaśataṁ pūrṇaṁ vyāsenoktaṁ mahātmanā}}

\vers
\varr{
        \ \vb sahasrādhyāyam u°\lem  \msCa\msCb\msNa\msNb\msNc\msNd; sahaśradhyāyam u° \msCc,
                                                        sahasrādhyāyar u° \Ed
        \ \vc parva cāsya\lem  \msCa\msNa\msNb\msNc; parvañ cāsya \msCb, parvvam asya \msCc\msNd\Ed\oo
                 śataṁ pūrṇaṁ\lem  \msCa\msCb\msNa\msNb\msNc\msNd\Ed; ta \msCc
        \ \vd śrutvā\lem  \msCa\msCc\msNa\msNb\msNc\msNd\Ed; śraddhā \msCb\oo
                 bhāratasaṁhitām\lem  \msCa\msCb\msNa\msNb\msNc; bhārasaṁhitā \msCc, bhāratasaṁhitaṁ \msNd, nārādasaṁhitām \Ed}

at\textsubring{r}ptaḥ puna papraccha vaiśampāyanam eva hi\thinspace{\dandab} \dontdisplaylinenum

janamejaya yat pūrvaṁ tac ch\textsubring{r}ṇu tvam atandritaḥ \veg\dontdisplaylinenum
\varr{
        \ \va at\textsubring{r}ptaḥ puna papraccha\lem  \eme;
                a\uncl{t\textsubring{r}ptaḥ pu}{\il}{\il}praccha \msCa, 
                at\textsubring{r}ptaḥ punaḥ papraccha \msCb\msNa\msNb\msNc,
                at\textsubring{r}ptaḥ punar apracche \msCc,
                at\textsubring{r}ptaḥ puna paḥpraccha \msNd,
                at\textsubring{r}ptā punaḥ papraccha \Ed 
        \ \vb vaiśampāyana°\lem  \msCa\msCb\msNa\msNb\msNc\msNd\Ed; vesampāyana° \msCc
        \ \vc janamejaya yat\lem  \eme;
                  janamejayena yat \msCapcorr\msCb\msNa\msNb\msNc\msNd\Ed,
                  janamejaye yat \msCaacorr,
                  janmejayena yam \msCc\oo
                 pūrvaṁ\lem  \mssCaCbCc\msNc\Ed; pūrva \msNa, pūr\uncl{va} \msNb
          \ \vd tac ch\textsubring{r}ṇu tvam a°\lem  \msCa\msCb\msNa\msNc\Ed; tac ch\textsubring{r}ṇa tvam a° \msCc, {\lost}{\lost}{\lost}{\lost}{\lost}  \msNb, 
                                        tac ch\textsubring{r}ṇu svam a° \msNd\oo
                   °tandritaḥ\lem  \msCc; °tandritam \msCa\msCb\msNc\msNd\Ed, °tandri\uncl{tam} \msNa, {\lost}{\lost}{\lost} \msNb}

janamejaya uvāca~{\dandab}\dontdisplaylinenum 
\varr{
        \ \vo janamejaya\lem  \msCa\msCb\msNa\msNb\msNc\msNd\Ed; janmejaya \msCc}

bhagavan sarvadharmajña sarvaśāstraviśārada\thinspace{\danda} \dontdisplaylinenum

asti dharmaṁ paraṁ guhyaṁ saṁsārārṇavatāraṇam \veg\dontdisplaylinenum
\varr{
       \ \va bhagavan sa°\lem  \msCa\msCb\msNa\msNb\msNc\Ed; bhacāvaṁ sa° \msCc, bhagava sa° \msNd\oo
                °dharmajña\lem  \mssCaCbCc\msNb\msNc\Ed; °jña \msNa, °dharmajñaḥ \msNd
       \ \vb °viśārada\lem  \msCa\msNb\msNc\msNd; °visāradaḥ \msCb\msCc\msNa\Ed
        \ \vd asti dharmaṁ\lem  \msCa\msNa\msNb\msNc\Ed; asti dharmaḥ \msCb, asti dharma \msCc, adharma \msNd\oo
                 paraṁ guhyaṁ\lem  \msCa\msNb\msNd\Ed; paro guhya \msCb, paraṁ guhya \msCc\msNa, paraguhyaṁ \msNc}

dvaipāyanamukhodgīrṇaṁ dharmaṁ yat tad dvijottama\thinspace{\dandab} \dontdisplaylinenum

kathayasva hi me t\textsubring{r}ptiṁ kuru yatnāt tapodhana \veg\dontdisplaylinenum
\varr{
        \ \va dvaipāyana°\lem  \msCa\msCb\msNa\msNb\msNc\msNd\Ed; dvepāyana° \msCc\oo
                 °mukhodgīrṇaṁ\lem  \msCa\msCb\msNa\msNb\msNc; °mukhodgīrṇa \msCc, °mudgīrṇṇa \msNd, mukhād gīrṇaṁ \Ed
         \ \vb dharmaṁ yat tad dvi°\lem  \msCb; dharmaṁ vā yad dvi° \msCa\msNa\msNb\msNc\Ed, 
                                                dharmavat ya dvi° \msCc, dharma vā yad dvi° \msNd\oo 
                  °ttama\lem  \msCa\msCb\msNa\msNb\msNc\msNd\Ed; °ttamaḥ \msCc  
        \ \vc t\textsubring{r}ptiṁ\lem  \mssCaCbCc\msNa\msNb\msNc\Ed; t\textsubring{r}pti \msNd
        \ \vd yatnāt tapodhana\lem  \msCb\msNa\msNb\msNc\Ed; yannāt ta{\il}{\il}na \msCa,
                                                        yatnā tapodhanaḥ \msCc, yatnā tapodhana \msNd}

vaiśampāyana uvāca~{\dandab}\dontdisplaylinenum 

ś\textsubring{r}ṇu rājann avahito dharmākhyānam anuttamam\thinspace{\danda} \dontdisplaylinenum

vyāsānugrahasamprāptaṁ guhyadharmaṁ ś\textsubring{r}ṇotu me \veg\dontdisplaylinenum
\varr{
        \ \va rājann a°\lem  \mssCaCbCc\msNa\msNb\msNc\Ed; rājaṁn a° \msNd
        \ \vb °khyānam anuttamam\lem  \msCa\msNa\msNb\msNc\Ed; °khyānam uttamam \msCb,
                                               °khyānam utamam \msCc, °dharmavyākhyānam uttamaṁ \msNd\ \unmetr
        \ \vc °prāptaṁ\lem  \msCa\msCb\msNa\msNb\msNc\msNd\Ed; °prāpta \msCc
        \ \vd °dharmaṁ\lem  \msCa\msCb\msNa\msNb\msNc\msNd\Ed; °rmaṁ \msCc\oo
                 ś\textsubring{r}ṇotu\lem  \msCa\msCb\msNa\msNb\msNc\msNd\Ed; ś\textsubring{r}ṇota \msCc\oo
                 me\lem  \msCa\msCc\msNa\msNb\msNc\msNd\Ed; mai \msCb}

anarthayajñakartāraṁ tapovrataparāyaṇam\thinspace{\dandab} \dontdisplaylinenum

śīlaśaucasamācāraṁ sarvabhūtadayāparam \veg\dontdisplaylinenum
\varr{
        \ \va °kartāraṁ\lem  \mssCaCbCc\msNa\msNc\msNd\Ed; °karttantaṁ \msNb
        \ \vb °yaṇam\lem  \msCa\msCb\msNb\Ed; °yana \msCc, °yaṇaḥ \msNa, °yanaṁ \msNc, °\uncl{yaṇaṁ} \msNd
        \ \vd °param\lem  \msCa\msCb\msNa\msNc\Ed; °nvitam \msCc\msNd, °\uncl{pa}raṁ \msNb}

jijñāsanārthaṁ praśnaikaṁ viṣṇunā prabhaviṣṇunā\thinspace{\dandab} \dontdisplaylinenum

dvijarūpadharo bhūtvā papraccha vinayānvitaḥ \veg\dontdisplaylinenum


\alalfejezet{brahmavidyā}\varr{
        \ \va °rthaṁ praśnaikaṁ\lem  \msCb\msNa\msNb\msNc; °rthaṁ praśnekaṁ \msCa\msNd, 
                                        °rthapraśnekam \msCc\Ed
        \ \vb prabha°\lem  \msCa\msCb\msNa\msNb\msNd\Ed; prabhu° \msCc, prābha° \msNc
        \ \vc °dharo\lem  \msCb\msCc\msNa\msNc\msNd\Ed; °{\il}ro \msCa, °dharā \msNb
        \ \vd °nvitaḥ\lem  \msCa\msCb\msNa\msNb\msNc\Ed; °nvitaṁ \msCc\msNd}

[vigatarāga uvāca~{\dandab}\dontdisplaylinenum ]

brahmavidyā kathaṁ jñeyā rūpavarṇavivarjitā\thinspace{\danda} \dontdisplaylinenum

svaravyañjananirmuktam akṣaraṁ kimu tat param \veg\dontdisplaylinenum
\varr{
        \ \va jñeyā\lem  \msCa\msNa\msNb\msNc; jñeyaṁ \msCb\msCc, jñeya \msNd, bhūyo \Ed
        \ \vb °varṇa°\lem  \mssCaCbCc\msNa\msNb\msNc\msNd; °varṇā° \Ed\oo
                 °varjitā\lem  \msCa\msCb\msNa\msNb\msNd\Ed; °varjitaṁ \msCc, °varjitāḥ \msNc
        \ \vc °vyañjana°\lem  \mssCaCbCc\msNa\msNb\msNc\msNd; °vyajjana° \Ed
        \ \vcd °muktam akṣa°\lem  \msCa\msCc\msNa\msNb\msNc\Ed; °mukta akṣa° \msCb, °muktaṁ akha° \msNd
        \ \vd kimu tat param\lem  \msCa\msNa\msNc\Ed; kim ataḥ param \msCb\msCc, kim atat paraṁ \msNb\msNd}

anarthayajña uvāca~{\dandab}\dontdisplaylinenum 

anuccāryam asandigdham avicchinnam anākulam\thinspace{\danda} \dontdisplaylinenum

nirmalaṁ sarvagaṁ sūkṣmam akṣaraṁ kimu tatparam \veg\dontdisplaylinenum


\alalfejezet{kālapāśaḥ}\varr{
        \ \va °ccārya°\lem  \msCa\msCb\msNa\msNb\Ed; °cārya° \msCc\msNc\msNd
        \ \vab °sandigdham avicchinnam anākulam\lem  \msCa\msCb\msNa\msNc\msNd\Ed;
                                °vicchinnasandigdham anākuna \msCc, °sandigdha\-m anacchinnam anākulam \msNb
        \ \vd kimu tatparam\lem  \msCa\msNa\msNb\msNc\Ed; kim ataḥ param \msCb,
                                                kim atatparaṁ \msCc\msNd}

vigatarāga uvāca~{\dandab}\dontdisplaylinenum 
\varr{
        \ \vo °rāga uvāca\lem  \mssCaCbCc\msNa\msNb\msNc\Ed; °rāgovāca \msNd}

dehī dehe kṣayaṁ yāte bhūjalāgniśivādibhiḥ\thinspace{\danda} \dontdisplaylinenum
            \paral{\textit{\vb {\normalfont cf.\ Kūrmapurāṇa 2.23.74: } 
                atha kaścit pramādena mriyate 'gniviṣādibhiḥ{\thinspace\danda} 
                tasyāśaucaṁ vidhātavyaṁ kāryaṁ caivodakādikam{\thinspace\ketdanda}}}

yamadūtaiḥ kathaṁ nīto nirālambo nirañjanaḥ \veg\dontdisplaylinenum
\varr{
        \ \va dehe kṣa°\lem  \msCa\msCc\msNc; dehāt kṣa° \msCb, dehakṣa° \msNa\msNb\msNd\Ed\oo
                 yāte\lem  \mssCaCbCc\msNa\msNb\msNc\Ed; yānte \msNd
        \ \vb °gniśivādibhiḥ\lem  \msCa\msCb\msNa\msNb\msNc\Ed; °gniśivādibhi \msCc, °gniṁ śi{\il}dibhi \msNd
        \ \vc °dūtaiḥ\lem  \msCa\msCb\msNa\msNb\msNc\Ed; °dūte \msCc\msNd\oo
                 nīto\lem  \msCa\msCb\msNa\msNb\msNc\msNd; nītvā \msCc, nītā \Ed
        \ \vd nirañjanaḥ\lem  \msCa\msCb\msNa\msNb\msNc\msNd\Ed; nirañjana \msCc}

kālapāśaiḥ kathaṁ baddho nirdehaś ca kathaṁ vrajet\thinspace{\dandab} \dontdisplaylinenum

svargaṁ vā sa kathaṁ yāti nirdeho bahudharmak\textsubring{r}t \danda\dontdisplaylinenum

etan me saṁśayaṁ brūhi jñātum icchāmi tattvataḥ \veg\dontdisplaylinenum
\varr{
        \ \va °pāśaiḥ\lem  \msCa\msCb\msNa\msNb\msNc\Ed; °pāśe \msCc, °pāśai \msNd\oo
                 baddho\lem  \msCa\msCc\msNa\msNb\msNc\Ed; ba\uncl{ddho} \msCb, baddha \msNd
        \ \vb nirdehaś ca\lem  \msCa\msCb\msNa\msNb\msNc\Ed; nirdahaḥ sa \msCc, nirdehasya \msNd\oo
                 vrajet\lem  \mssCaCbCc\msNa\msNc\msNd\Ed; bhavet \msNb
        \ \vc svargaṁ\lem  \msCa\msCb\msNa\msNb\msNc\Ed; svarga \msCc\msNd\oo
                 sa\lem  \mssCaCbCc\msNa\msNc\msNd\Ed; saṁ \msNb\oo
                 yāti\lem  \msNa\msNb\msNc\msNd; yānti \mssCaCbCc\Ed
        \ \ve saṁśayaṁ\lem  \mssCaCbCc\msNc\Ed; saṁśaye \msNa, saṁśayo \msNb\msNd
        \ \vf °tum icchāmi\lem  \msCa\msCc\msNa\msNb\msNc\msNd\Ed; °tum i \msCb}

anarthayajña uvāca~{\dandab}\dontdisplaylinenum 
\varr{
        \ \vo anarthayajña uvāca\lem  \mssCaCbCc\msNapcorr\msNb\msNc\msNd\Ed; \om\ \msNaacorr}

atisaṁśayakaṣṭaṁ te p\textsubring{r}ṣṭo 'haṁ dvijasattama\thinspace{\danda} \dontdisplaylinenum

durvijñeyaṁ manuṣyais tu devadānavapannagaiḥ \veg\dontdisplaylinenum
\varr{
        \ \va atisaṁśayakaṣṭaṁ te\lem  \msCb\msNa\msNb\msNc;
                atiśaṁsa\uncl{ya}kaṣṭan te \msCa, atiśaṁsayakaṣṭam me \msCc\Ed, 
                                                atisaṁśayakaṣṭo mo \msNd
        \ \vb dvijasattama\lem  \msCa\msCb\msNa\msNb\msNc\Ed; ca dvijottamaḥ \msCc, dvijasattamaḥ \msNd
        \ \vc °jñeyaṁ\lem  \msCa\msCb\msNa\msNc; °jñeya \msCc\msNb\msNd\Ed\oo
                 manuṣyais tu\lem  \msCa\msNa\msNb\msNc\Ed; manuṣaiś ca \msCb, maṇukṣe\uncl{ptu} \msCc, 
                                                        manuṣyes tu \msNd}

karmahetuḥ śarīrasya utpattir nidhanaṁ ca yat\thinspace{\dandab} \dontdisplaylinenum

suk\textsubring{r}taṁ duṣk\textsubring{r}taṁ caiva pāśadvayam udāh\textsubring{r}tam \veg\dontdisplaylinenum
\varr{
        \ \va karma°\lem  \msCa\msCb\msNa\msNb\msNc\msNd; anarthayajña uvāca{\danda}{\danda} karma° \msCc\Ed\oo
                 °hetuḥ\lem  \msCb; °hetu \msCa\msNa\msNb\msNc\Ed, °heṁtu \msCc, °hetu \msNd\oo
                 śarīrasya\lem  \msCa\msCb\msNa\msNb\msNc\msNd\Ed; śarīrasyaṁ \msCc
        \ \vb utpattir ni°\lem  \corr; utpattini° \msCa\msCb\msNa\msNb\msNc\Ed, utpatini° \msCc\msNd\oo
                 ca yat\lem  \mssCaCbCc\msNa\msNc\Ed; ca yaḥ \msNb, yat \msNd
        \ \vc suk\textsubring{r}taṁ\lem  \msCa\msCb\msNa\msNb\msNc\Ed; suk\textsubring{r}tak\textsubring{r}tan \msCc, suk\textsubring{r}ta \msNd\oo
                 caiva\lem  \mssCaCbCc\msNa\msNb\msNc\Ed; vāpi \msNd
        \ \vd °h\textsubring{r}tam\lem  \msCa\msCb\msNa\msNb\msNc\msNd\Ed; °h\textsubring{r}taḥ \msCc }

tenaiva saha saṁyāti narakaṁ svargam eva vā\thinspace{\dandab} \dontdisplaylinenum

sukhaduḥkhaṁ śarīreṇa bhoktavyaṁ karmasambhavam \veg\dontdisplaylinenum
\varr{
        \ \va tenaiva\lem  \msCa\msCb\msNa\msNb\msNc\Ed; teneva \msCc\msNd\oo
                 saṁyāti\lem  \msCa\msCb\msNa\msNb\msNc\Ed; sā yānti \msCc\msNd
        \ \vb vā\lem  \mssCaCbCc\msNb\msNc\Ed; ca \msNa\msNd
        \ \vc °duḥkhaṁ\lem  \msCa\msCb\msNa\msNc; °duḥkha \msCc\msNb\Ed
        \ \vd °sambhavam\lem  \msCa\msCb\msNa\msNb\msNc; °sambhavaḥ \msCc\Ed}

hetunānena viprendra dehaḥ sambhavate n\textsubring{r}ṇām\thinspace{\dandab} \dontdisplaylinenum

yaṁ kālapāśam ity āhuḥ ś\textsubring{r}ṇu vakṣyāmi suvrata \veg\dontdisplaylinenum
\varr{
        \ \va °ndra\lem  \mssCaCbCc\msNa\msNc\Ed; °ndraḥ \msNb
        \ \vb dehaḥ\lem  \msCa\msCb\msNa\msNc\Ed; dehe \msCc, deha \msNb\oo
                 n\textsubring{r}ṇām\lem  \msCa\msNa\msNb\msNc\Ed; n\textsubring{r}ṇā \msCb\msCc
        \ \vc yaṁ kālapāśam ity āhuḥ\lem  \eme; yaṁ kālapāśam ity āha \msCa\msCb\msNa, 
                kālapāseti satvāha \msCc, yaṁ kālapāśam ity āhu \msNb\msNc,
                                        yaṁ kālapāśam ity āhu \Ed
        \ \vd °vrata\lem  \msCa\msNa\msNb\msNc\Ed; °vrataḥ \msCb\msCc}

na tvayā viditaṁ kiñcij jijñāsyasi kathaṁ dvija\thinspace{\dandab} \dontdisplaylinenum

kālapāśaṁ ca viprendra sakalaṁ vettum arhasi \veg\dontdisplaylinenum
\varr{
       \ \va viditaṁ\lem  \msCa\msCb\msNa\msNb\msNc\Ed; vidita \msCc
        \ \vab kiñcij ji°\lem  \msCb; kiñcid vi° \msCapcorr\msNa\msNb\msNc\Ed, kid vi° \msCaacorr, 
                                                        kiñci ji° \msCc
        \ \vb kathaṁ dvija\lem  \msCa\msCb\msNa\msNb\msNc\Ed;
                                        {\il}{\il}{\il}{\il}{\il}{\il}{\il}{\il}{\il} \uncl{ma tvayā viditaṁ kiñcid vijñāsyasi} 
                                        \cancelled\ kathaṁ dvija \msCc
        \ \vd vettum arhasi\lem  \mssCaCbCc\msNa\msNb; vettum ūhasi \msNc, vaktum arhasi \Ed}

kalākalitakālaṁ ca kālatattvakalāṁ ś\textsubring{r}ṇu\thinspace{\dandab} \dontdisplaylinenum

truṭidvayaṁ nimeṣas tu nimeṣadviguṇā kalā \veg\dontdisplaylinenum
\varr{
        \ \va kalā°\lem  \msCa\msCb\msNapcorr\msNb\msNc\Ed; kālā° \msCc\msNaacorr\oo
                 °kālaṁ ca\lem  \mssCaCbCc\msNa\msNb\msNc; °kālaś ca \Ed
        \ \vb °kalāṁ\lem  \msCa\msCc\msNb\Ed; °kalā \msCb\msNc, °vidhiṁ \msNa
        \ \vc truṭidvayaṁ\lem  \msCa\msCc\msNc\Ed; tuṭidvaya \msCb\msNb, tuṭidvayaṁ \msNa\oo
                 °meṣas tu\lem  \msCb\msCc\msNb\msNc\Ed; °mevas tu \msCa, °meṣadvi° \msNa}

kalādviguṇitā kāṣṭhā kāṣṭhā vai triṁśatiḥ kalā\thinspace{\dandab} \dontdisplaylinenum

triṁśatkalā muhūrtaś ca mānuṣena dvijottama \veg\dontdisplaylinenum
\varr{
        \ \vb kāṣṭhā vai triṁśatiḥ\lem  \msCa\msNa\msNb\msNc\Ed; vai triṁśatā \msCb, kāṣṭhā vai triṁśati \msCc
        \ \vc muhūrtaś ca\lem  \msCa\msCc\msNa\msNb\msNc; muhūrtta \msCb, muhūrtañ ca \Ed
        \ \vd mānuṣena\lem  \msCa\msCb\msNa\msNb\msNc\Ed; mānu\uncl{ṣaś ca} \msCc\oo
                 °ttama\lem  \mssCaCbCc\msNa\msNcpcorr\Ed; °tamaḥ \msNb, °ttamaḥ \msNcacorr}

muhūrtatriṁśakenaiva ahorātraṁ vidur budhāḥ\thinspace{\dandab} \dontdisplaylinenum

ahorātraṁ punas triṁśan māsam āhur manīṣiṇaḥ \veg\dontdisplaylinenum
\varr{
        \ \va muhūrta°\lem  \mssCaCbCc\msNa\msNb\msNc; muhūrtaṁ \Ed}

samā dvādaśa māsāś ca kālatattvavido janāḥ\thinspace{\dandab} \dontdisplaylinenum

śataṁ varṣasahasrāṇi trīṇi mānuṣasaṁkhyayā \veg\dontdisplaylinenum
\varr{
        \ \va samā\lem  \msCa\msCb\msNa\msNb\msNc\Ed; māsa \msCc\oo
                 °māsā°\lem  \msCa\msCb\msNa\msNb\msNc; °māsa° \msCc\Ed
        \ \vb kāla°\lem  \mssCaCbCc\msNa\msNb\Ed; kalā° \msNc
        \ \vc śataṁ\lem  \mssCaCbCc\msNa\msNb\msNc; śata° \Ed
        \ \vb mānuṣa°\lem  \msCa\msNa\msNb\msNc\Ed; māṇuṣya° \msCb\msCc\ \unmetr}

ṣaṣṭiṁ caiva sahasrāṇi kālaḥ kaliyugaḥ sm\textsubring{r}taḥ\thinspace{\dandab} \dontdisplaylinenum

dviguṇaḥ kalisaṁkhyāto dvāparo yuga saṁjñitaḥ \veg\dontdisplaylinenum
\varr{
        \ \vo (ṣaṣṭiṁ{\normalfont ...} saṁjñitaḥ)\lem  \mssCaCbCc\msNa\msNc\Ed; \om\ \eyeskip{from 21d to 24d} \msNb
        \ \va ṣaṣṭiṁ caiva\lem  \mssCaCbCc\msNc; ṣaṣṭiṁ varṣa° \msNa, \om\ \msNb, ṣaṣṭiś caiva \Ed
        \ \vb °yugaḥ\lem  \mssCaCbCc\msNa\msNc; \om\ \msNb, °yuga \Ed
        \ \vc dviguṇaḥ\lem  \mssCaCbCc\msNa\msNc; \om\ \msNb, dviguṇā \Ed
        \ \vd dvāparo\lem  \mssCaCbCc\msNa\msNc; \om\ \msNb, dvāpare \Ed}

tretā tu triguṇā jñeyā catuḥ k\textsubring{r}tayugaḥ sm\textsubring{r}taḥ\thinspace{\dandab} \dontdisplaylinenum

eṣā caturyugāsaṁkhyā k\textsubring{r}tvā vai hy ekasaptatiḥ \veg\dontdisplaylinenum
\varr{
        \ \vo (tretā{\normalfont ...} ekasaptatiḥ)\lem  \mssCaCbCc\msNa\msNc\Ed; \om\ \eyeskip{from 21d to 24d} \msNb
        \ \va tretā\lem  \msCa\msCb\msNa\Ed; tetrā \msCc, \om\ \msNb, tretrā \msNc
        \ \vb °yugaḥ\lem  \mssCaCbCc\msNa\msNc; \om\ \msNb, °yuga \Ed
        \ \vd hy e°\lem  \mssCaCbCc\msNa\msNb\Ed; he° \msNc}

manvantarasya caikasya jñānam uktaṁ samāsataḥ\thinspace{\dandab} \dontdisplaylinenum

kalpo manvantarāṇāṁ tu caturdaśa tu saṁkhyayā \veg\dontdisplaylinenum
\varr{
        \ \vo (manvantarasya{\normalfont ...} saṁkhyayā)\lem  \mssCaCbCc\msNa\msNc\Ed; \om 
                                        \eyeskip{from 21d to 24d} \msNb
        \ \va caikasya\lem  \mssCaCbCc\msNapcorr\msNc\Ed; \om\ \msNaacorr\msNb
        \ \vc kalpo\lem  \msCb; kalpa \msCa\msCc\msNa\msNc\Ed, \om\ \msNb
        \ \vd °daśa\lem  \msCa\msCc\msNa\msNc\Ed; °daśaṁ \msCb, \om\ \msNb}

daśa kalpasahasrāṇi brahmāhaḥ parikalpitam\thinspace{\dandab} \dontdisplaylinenum

rātrir etāvatī proktā munibhis tattvadarśibhiḥ \veg\dontdisplaylinenum
\varr{
        \ \vb °āhaḥ\lem  \msCb\msCc\msNa\msNb\msNc\Ed; °āha \msCa\oo
                 parikalpitam\lem  \msCa\msNc; karikalpitam \msCb, parikalpitaḥ \msCc\msNb\Ed, parikīrtitāḥ \msNa}

rātryāgame pralīyante jagat sarvaṁ carācaram\thinspace{\dandab} \dontdisplaylinenum

ahāgame tathaiveha utpadyante carācaram \veg\dontdisplaylinenum
\varr{
        \ \va pralīyante\lem  \msCa\msCc\msNa\msNb\msNc\Ed; pralīyate \msCb
        \ \vd ahāgame\lem  \mssCaCbCc\msNa\msNc; ahāga{\lost} \msNb, ahnāgame \Ed}

parārdhaparakalpāni atītāni dvijottama\thinspace{\dandab} \dontdisplaylinenum

anāgataṁ tathaivāhur bh\textsubring{r}gurādimaharṣayaḥ \veg\dontdisplaylinenum
\varr{
        \ \va °rdha°\lem  \mssCaCbCc\msNa\msNc\Ed; °rdhaṁ \msNb
        \ \vcd °vāhur bh\textsubring{r}°\lem  \msCa\msCb\msNa\msNc\Ed; °vāhu bh\textsubring{r}° \msCc\msNb
        \ \vd maharṣayaḥ\lem  \mssCaCbCc\msNapcorr\msNb\Ed; mahayaḥ \msNaacorr, marhaṣayaḥ \msNc}

yathārkagrahatārendu bhramato d\textsubring{r}śyate tv iha\thinspace{\dandab} \dontdisplaylinenum

kālacakraṁ bhramatvaiva viśramaṁ na ca vidmahe \veg\dontdisplaylinenum
\varr{
        \ \vb d\textsubring{r}śyate tv iha\lem  \msCa\msNa\msNb\msNc\Ed; d\textsubring{r}śyandiha \msCb, d\textsubring{r}syate tv ihaḥ \msCc
        \ \vc °cakraṁ\lem  \mssCaCbCc\msNa\msNc\Ed; °cakra \msNb\oo
                 °tvaiva\lem  \msCa\msNa\msNc\Ed; °tveva \msCb\msNb, °tveha \msCc
        \ \vd °śramaṁ\lem  \mssCaCbCc\msNapcorr\msNc\Ed; °śramo \msNaacorr, °śrāman \msNb\oo
                 vidmahe\lem  \msCa\msCc\msNa\msNb\msNc\Ed; vigrahe \msCb}

kālaḥ s\textsubring{r}jati bhūtāni kālaḥ saṁharate punaḥ\thinspace{\dandab} \dontdisplaylinenum

kālasya vaśagāḥ sarve na kālavaśak\textsubring{r}t kvacit \veg\dontdisplaylinenum
            \paral{\textit{\vo \kb\ {\normalfont Kūrmapurāṇa 1.11.32: } 
                kālaḥ s\textsubring{r}jati bhūtāni kālaḥ saṁharate prajāḥ{\thinspace\danda}
                sarve kālasya vaśagā na kālaḥ kasyacid vaśe{\thinspace\ketdanda}}}
\varr{
        \ \vb kālaḥ\lem  \mssCaCbCc\msNa\msNb\msNc; kāla \Ed
        \ \vc vaśagāḥ\lem  \mssCaCbCc\msNa\msNb\msNc; vaśagā \Ed}

caturdaśaparārdhāni devarājā dvijottama\thinspace{\dandab} \dontdisplaylinenum

kālena samatītāni kālo hi duratikramaḥ \veg\dontdisplaylinenum
            \paral{\textit{\vd {\normalfont  = MBh 12.220.41d = Garuḍapurāṇa 1.108.7 } }}
\varr{
        \ \vb devarājā\lem  \mssCaCbCc\msNa\msNb\msNc; devarāja \Ed}

eṣa kālo mahāyogī brahmā viṣṇuḥ paraḥ śivaḥ\thinspace{\dandab} \dontdisplaylinenum

anādinidhano dhātā sa mahātmā namaskuru \veg\dontdisplaylinenum


\alalfejezet{parārdhādi}\varr{
        \ \va kālo\lem  \msCa\msCb\msNa; kāla \msCc\msNb\msNc\Ed
        \ \vb brahmā viṣṇuḥ paraḥ\lem  \msCb; brahmaviṣṇuparaḥ \msCa\msNc, brahmā viṣṇu paraḥ \msCc\msNa\msNb,
                                                        brahmaviṣṇupara \Ed\ \unmetr}

vigatarāga uvāca~{\dandab}\dontdisplaylinenum 

śrutaṁ vai kālacakraṁ tu mukhapadmaviniḥs\textsubring{r}tam\thinspace{\danda} \dontdisplaylinenum

parārdhaṁ ca paraṁ caiva śrotuṁ vaḥ pratidīpitam \veg\dontdisplaylinenum
\varr{
        \ \va °cakraṁ tu\lem  \msCa\msCb\msNa\msNb\msNc\Ed; °cakrasya \msCc
        \ \vb viniḥs\textsubring{r}tam\lem  \corr; vinis\textsubring{r}tam \mssCaCbCc\msNa\msNb\msNc\Ed\ \unmetr
        \ \vc parārdhaṁ\lem  \msCb\msCc\msNa\msNb\msNc\Ed; \uncl{pa}rārddhaṁ \msCa
        \ \vd vaḥ\lem  \mssCaCbCc\msNa\msNb\msNc; yaḥ \Ed}

anarthayajña uvāca~{\dandab}\dontdisplaylinenum 
\varr{
        \ \vo anarthayajña uvāca\lem  \mssCaCbCc\msNapcorr\msNb\msNc\Ed; \om\ \msNaacorr}

ekaṁ daśaṁ śataṁ caiva sahasram ayutaṁ tathā\thinspace{\danda} \dontdisplaylinenum

prayutaṁ niyutaṁ koṭim arbudaṁ v\textsubring{r}ndam eva ca \veg\dontdisplaylinenum
\varr{
        \ \vb °yutaṁ\lem  \mssCaCbCc\msNa\msNc\Ed; °tan \msNb
        \ \vcd koṭim a°\lem  \mssCaCbCc\msNa\msNb\Ed; koṭir a° \msNc
        \ \vd °rbudaṁ\lem  \mssCaCbCc\msNa\msNb\Ed; °budaṁ \msNc}

kharvaṁ caiva nikharvaṁ ca śaṅkuḥ padmaṁ tathaiva ca\thinspace{\dandab} \dontdisplaylinenum
            \paral{\textit{\vab {\normalfont  = Brahmāṇḍapurāṇa 3.2.101 } }}

samudro madhyam antaṁ ca parārdhaṁ ca paraṁ tathā \veg\dontdisplaylinenum
\varr{
        \ \va ca\lem  \mssCaCbCc\msNa\msNc\Ed; tu \msNb
        \ \vb śaṅkuḥ\lem  \corr; śaṅku \mssCaCbCc\msNa\msNb\msNc, śaṁkha \Ed
        \ \vcd (samudro{\normalfont ...} tathā)\lem  \mssCaCbCc\msNa\msNb\msNc; \om\ \Ed
        \ \vc madhyam antaṁ ca\lem  \mssCaCbCc\msNaacorr; madhyamāntaṁ ca \msNapcorr, 
                        madhya\uncl{mantañ ca} \msNb, madhyam antaś ca \msNc, \om\ \Ed}

sarve daśaguṇā jñeyāḥ parārdhaṁ yāvad eva hi\thinspace{\dandab} \dontdisplaylinenum

parārdhadviguṇenaiva parasaṁkhyā vidhīyate \veg\dontdisplaylinenum
\varr{
        \ \vo (sarve{\normalfont ...} vidhīyate)\lem  \mssCaCbCc\msNa\msNb\msNc; \om\ \Ed
        \ \vb parārdhaṁ\lem  \msNc; parārdha \msCb\msCc\msNa\msNb, parā\uncl{rdha} \msCa, \om\ \Ed
        \ \vc parārdha°\lem  \mssCaCbCc\msNa\msNb\Ed; parārdhaṁ \msNc}

parāt parataraṁ nāsti iti me niścitā matiḥ\thinspace{\dandab} \dontdisplaylinenum

purāṇavedapaṭhitā mayākhyātā dvijottama \veg\dontdisplaylinenum


\alalfejezet{brahmāṇḍam}\varr{
        \ \vab parāt parataraṁ nāsti iti me niścitā matiḥ\lem  \mssCaCbCc\msNb\msNcpcorr;
                parāt parataraṁ nāsti iti me niścitā mati \msNa\msNcacorr,
                v\textsubring{r}ndañ caiva mahāv\textsubring{r}nda dviparānantam eva ca{\danda}
                parāt parataraṁ nāsti iti me niścitā matiḥ{\danda}{\danda} \Ed
        \ \ve °veda°\lem  \msCa\Ed; °vede \msCb\msCc\msNb\msNc\ \unmetr, °vedā \msNa
        \ \vf °ākhyātā\lem  \msCa\msCb\msNa; °ākhyātaṁ \msCc\msNb\msNc\Ed}

vigatarāga uvāca~{\dandab}\dontdisplaylinenum 

brahmāṇḍaṁ kati vijñeyaṁ pramāṇaṁ prāpitaṁ kvacit\thinspace{\danda} \dontdisplaylinenum

kati cāṅgulimūrdheṣu sūryas tapati vai mahīm \veg\dontdisplaylinenum
\varr{
        \ \va brahmāṇḍaṁ\lem  \msCa\msCb\msNa\msNb\msNc\Ed; brahmāṇḍa \msCc
        \ \vb pramāṇaṁ prāpitaṁ\lem  \conj; pramāṇañ cāpitaṁ \mssCaCbCc\msNa\msNb\Ed,
                                                                pramāñ cāpitat \msNc
        \ \vd mahīm\lem  \msCb\msCc\msNa; mahī\uncl{m} \msCa, mahī \msNb\msNc\Ed}

anarthayajña uvāca~{\dandab}\dontdisplaylinenum 

brahmāṇḍānāṁ prasaṁkhyātuṁ mayā śakyaṁ kathaṁ dvija\thinspace{\danda} \dontdisplaylinenum

devās te 'pi na jānanti mānuṣāṇāṁ ca kā kathā \veg\dontdisplaylinenum
\varr{
        \ \va prasaṁkhyātuṁ\lem  \mssCaCbCc\msNa\msNc; prasaṁsā tu \msNb, ca saṁkhyātuṁ \Ed
        \ \vb śakyaṁ\lem  \msNa\msNb\Ed; śakyā \mssCaCbCc\msNc}

paryāyeṇa tu vakṣyāmi yathāśakyaṁ dvijottama\thinspace{\dandab} \dontdisplaylinenum

brahmaṇā yat purākhyāto mātariśvā yathā tathā \veg\dontdisplaylinenum
            \paral{\textit{\vcd {\normalfont cf. Brahmāṇḍapurāṇa 3.4.58cd: } 
                        brahmā dadau śāstram idaṁ purāṇaṁ mātariśvane}} 
\varr{
        \ \vc purā°\lem  \mssCaCbCc\msNa\msNb\msNc; mamā° \Ed}

śivāṇḍābhyantareṇaiva \crux{sarveṣām iva bhūritāḥ}\thinspace{\dandab} \dontdisplaylinenum

daśanāma diśāṣṭānāṁ brahmāṇḍe kīrtitaṁ ś\textsubring{r}ṇu \veg\dontdisplaylinenum


\alalfejezet{daśa nāmāni digaṣṭakānām}\varr{
        \ \vb °ṣām iva bhūritāḥ\lem  \msCa\msCb\msNc; °ṣām eva bhūritāḥ \msCc,
                                                        °ṣām iva bhūritā \msNa, °ṣām eva bhūriṇām \msNb,
                                                                         °ṣām eva bhūr imāṁ \Ed
        \ \vc diśā°\lem  \mssCaCbCc\msNa\msNc\Ed; śivā° \msNb
        \ \vd kīrtitaṁ ś\textsubring{r}ṇu\lem  \msCa\msCc\msNa\msNb\msNc\Ed; ya ca kīrtitam \msCb}

sahāsaha sahaḥ sahyo visahaḥ saṁhato 'sabhā\thinspace{\dandab} \dontdisplaylinenum

prasaho 'prasahaḥ sānuḥ pūrvato daśa nāyakāḥ \veg\dontdisplaylinenum
\varr{
        \ \va sahāsaha\lem  \mssCaCbCc\msNa\msNb\Ed; sahāsahaḥ \msNc\oo
                 sahyo\lem  \msCa\msCc\msNa\msNb\msNc; sa\uncl{hyo} \msCb, sajño \Ed
        \ \vb visahaḥ\lem  \msCa\msCb\msNa\msNb\msNc\Ed; visaha \msCc\oo
                 sabhā\lem  \msCa\msCc\msNa\msNb\msNc; sabhāḥ \msCb, satā \Ed
        \ \vc prasaho\lem  \mssCaCbCc\msNa\msNb\msNc; prasaheḥ \Ed\oo
                 prasahaḥ\lem  \msCa\msCb\msNa\msNb\msNc; prasa\uncl{vaḥ} \msCc, saprahaḥ \Ed\oo
                 sānuḥ\lem  \mssCaCbCc\msNa\msNb; sānu \msNc\Ed
        \ \vd pūrvato\lem  \mssCaCbCc\msNa\msNb\msNc; parvato \Ed}

prabhāso bhāsano bhānuḥ pradyoto dyutimo dyutiḥ\thinspace{\dandab} \dontdisplaylinenum

dīptatejāś ca tejāś ca tejā tejavaho daśa \veg\dontdisplaylinenum
\varr{
        \ \va bhāsano\lem  \msCa\msCb\msNa\msNb\msNc; bhāsa{\lost} \msCc, bhāsato \Ed\oo
                 bhānuḥ\lem  \msCa\msCc\msNa\msNb\msNc\Ed; bhānu \msCb
        \ \vb dyutimo\lem  \mssCaCbCc\msNa\msNb; dyutino \msNc\Ed
        \ \vc dīptatejā°\lem  \mssCaCbCc\msNa\msNb\msNc; dīptateja° \Ed\oo
                 tejāś ca\lem  \msCa\msCb\msNa\msNb\msNc\Ed; tejaś ca \msCc}

āgneye tv etad ākhyātaṁ yāmye ś\textsubring{r}ṇv atha bho dvija\thinspace{\dandab} \dontdisplaylinenum

yamo 'tha yamunā yāmaḥ saṁyamo yamuno 'yamaḥ \veg\dontdisplaylinenum
\varr{
        \ \va āgneye\lem  \mssCaCbCc\msNa\msNb\Ed; āgneya \msNc
        \ \vb dvija\lem  \mssCaCbCc\msNa\msNc\Ed; dvijaḥ \msNb
        \ \vc yamunā\lem  \mssCaCbCc\msNa\msNb\Ed; yamanā \msNc
        \ \vd yamuno\lem  \msCa\msCb\msNb;
                        yamano \msCc\msNc, yumunā \msNa, yamunā° \Ed}

saṁyano yamanoyāno yaniyugmā yanoyanaḥ\thinspace{\dandab} \dontdisplaylinenum

nagajo naganā nando nagaro naga nandanaḥ \veg\dontdisplaylinenum
\varr{
        \ \va saṁyano\lem  \msNa; saṁyamo \mssCaCbCc\msNb\Ed, saṁyamā \msNc\oo
                 yamano°\lem  \msCa\msCc\msNa\msNc\Ed; yamuno° \msCb\msNb\oo
                 °yāno\lem  \mssCaCbCc\msNa\msNb\Ed; °yāmo \msNc
        \ \vb yano yanaḥ\lem  \msNb; nayo yanaḥ \msCa\msCc\msNa, nayo nayaḥ \msCb,
                                nayo yamaḥ \msNc, nayonaya \Ed
        \ \vc naganā nando\lem  \msCa\msCc\msNa\msNb\msNc;
                                nagajā nando \msCb, nagano nado \Ed
        \ \vd nagaro naganandanaḥ\lem  \msNb; nagaroraganandanaḥ \msCa\msNc,
                                   nagaro\uncl{nagananda}naḥ \msCb,
                                   naga\uncl{ro}{\lost}{\lost}nandanaḥ \msCc,
                                   nagarogaranandanaḥ \msNa, nagaronnaganandanaḥ \Ed}

nagarbho gahano guhyo gūḍhajo daśa tatparaḥ\thinspace{\dandab} \dontdisplaylinenum

vāruṇena pravakṣyāmi ś\textsubring{r}ṇu vipra nibodha me \veg\dontdisplaylinenum
\varr{
        \ \va nagarbho\lem  \mssCaCbCc\msNa\msNc\Ed; n\textsubring{r}gabho \msNb\oo 
                 guhyo\lem  \mssCaCbCc\msNa\msNb\msNc; guhye \Ed
        \ \vc vāruṇena\lem  \mssCaCbCc\msNa\msNb\msNc; vāruṇe ca \Ed
        \ \vd ś\textsubring{r}ṇu\lem  \msNb; ś\textsubring{r}ṅge \msCa\msCb\msNa\msNc, ś\textsubring{r}\uncl{ṅge} \msCc, m\textsubring{r}ddhe \Ed}

babhraḥ setur bhavodbhadraḥ prabhavodbhavabhājanaḥ\thinspace{\dandab} \dontdisplaylinenum

bharaṇo bhuvano bhartā daśaite varuṇālayāḥ \veg\dontdisplaylinenum
\varr{
        \ \va babhraḥ setur bha°\lem  \corr; babhraṁ setur bha° \msCa\msCb, babhraṁ setu bha° \msCc,
                      babhraḥ setu bha° \msNa, babhraṁ sotur bha° \msNb, babhra setur bha° \msNc,
                                        babhrūn satur bha° \Ed
        \ \vb °bhājanaḥ\lem  \mssCaCbCc\msNa\msNb\msNc; °bhājana \Ed
        \ \vc bharaṇo\lem  \msCb\msNc; bharaṇa \msCa\msNa, bharaṇāṁ \msCc\Ed,
                                        bharaṇā \msNb
        \ \vd daśaite\lem  \mssCaCbCc\msNa\msNb\Ed; daśete \msNc\oo
                °ālayāḥ\lem  \mssCaCbCc\msNa\msNb\msNc; °ālayā \Ed}

n\textsubring{r}garbho 'suragarbhaś ca devagarbho mahīdharaḥ\thinspace{\dandab} \dontdisplaylinenum

v\textsubring{r}ṣabho v\textsubring{r}ṣagarbhaś ca v\textsubring{r}ṣāṅko v\textsubring{r}ṣabhadhvajaḥ \veg\dontdisplaylinenum
\varr{
        \ \va °garbhaś ca\lem  \msCa\msCb\msNb\msNc; °garbhāś ca \msCc\msNa\Ed
        \ \vc °garbhaś ca\lem  \mssCaCbCc\msNb\msNc\Ed; °garbhāś ca \msNa
        \ \vd v\textsubring{r}ṣabha°\lem  \msCa\msCb\msNa\msNb\msNc\Ed; v\textsubring{r}ṣa{\il}° \msCc}

jñātavyaś ca tathā samyag v\textsubring{r}ṣajo v\textsubring{r}ṣanandanaḥ\thinspace{\dandab} \dontdisplaylinenum

nāyakā daśa vāyavye kīrtitā ye mayā dvija \veg\dontdisplaylinenum
\varr{
        \ \va jñātavyaś\lem  \mssCaCbCc\msNa\msNb\msNc; jñānavāñ \Ed
        \ \ab samyag v\textsubring{r}ṣajo\lem  \mssCaCbCc\msNb\msNc;
                        samyag \msNa, satyav\textsubring{r}ṣajo \Ed
        \ \vd ye\lem  \mssCaCbCc\msNa\msNb\Ed; ya \msNc\oo
                 dvija\lem  \msCa\msCb\msNa\msNc\Ed; dvijaḥ \msCc\msNb}

sulabhaḥ sumanaḥ saumyaḥ suprajaḥ sutanuḥ śivaḥ\thinspace{\dandab} \dontdisplaylinenum

sata satya layaḥ śambhur daśanāyakam uttare \veg\dontdisplaylinenum
\varr{
        \ \va sulabhaḥ\lem  \mssCaCbCc\msNa\msNb\msNc; surabhaḥ \Ed\oo
                 sumanaḥ\lem  \mssCaCbCc\msNa\msNb\Ed; sumanāḥ \msNc
        \ \vc sata satya\lem  \mssCaCbCc\msNc; satyasatya \msNa, suta satya \msNb, sata satyā° \Ed\oo
                 layaḥ\lem  \mssCaCbCc\msNa\msNb\Ed; layaṁ \msNc
        \ \vcd śambhur da°\lem  \msCa\msCb\msNb\Ed; śambhu da° \msCc\msNa\msNc
        \ \vd °nāyakam u°\lem  \mssCaCbCc\msNa\msNb\msNc; °nāyaka u° \Ed}

indu bindu bhuvo vajra varado vara varṣaṇaḥ\thinspace{\dandab} \dontdisplaylinenum

ilano valino brahmā daśeśāneṣu nāyakāḥ \veg\dontdisplaylinenum
\varr{
        \ \vb °varṣaṇaḥ\lem  \mssCaCbCc\msNa\msNb; °{\il}\uncl{rśaṇam} \msNc, °daryya ca \Ed
        \ \vd daśe°\lem  \msCa\msNa\msNc\Ed; daśai° \msCb\msCc\msNb}

aparo vimalo moho nirmalo mana mohanaḥ\thinspace{\dandab} \dontdisplaylinenum

akṣayaś cāvyayo viṣṇur varado madhyame daśa \veg\dontdisplaylinenum
\varr{
        \ \vb nirmalo ma°\lem  \eme; nimalo ma° \msCa, nirmalonma° \msCb\msNc,
                                                nirmalotma° \msCc\Ed, nimalorma° \msNa\msNb
        \ \vc akṣayaś cā°\lem  \msCa\msCb\msNa\msNb\msNc; akṣayāś cā° \msCc, akṣayañ cā° \Ed
        \ \vcd viṣṇur va°\lem  \msCa\msCb\msNc\Ed; viṣṇu va° \msCc\msNa, rviṣṇur va \msNb
        \ \vd madhyame daśa\lem  \msCa\msCb\msNc; madhyamo daśa \msCc\msNa, 
                                        varavarṣaṇaḥ \msNb, madhyame daśaḥ \Ed}

sarveṣāṁ daśam īśānāṁ parivāraśataṁ śataṁ\thinspace{\dandab} \dontdisplaylinenum

śatānāṁ p\textsubring{r}thag ekaikaṁ sahasraiḥ parivāritam \veg\dontdisplaylinenum
\varr{
        \ \va sarveṣāṁ\lem  \mssCaCbCc\msNa\msNb\Ed; sarveṣā \msNc\oo
                 daśam īśānāṁ\lem  \mssCaCbCc\msNa\msNb\msNc; daśarīśānāṁ \Ed
        \ \vb parivāra°\lem  \msCa\msCc\msNb\msNc\Ed; pari° \msCb, parivāraṁ \msNa
        \ \vd °vāritam\lem  \msCa\msCb\msCcpcorr\msNa\msNb\msNc; °vāritā \msCcacorr, °vāritāḥ \Ed}

sahasreṣu ca ekaikam ayutaiḥ parivāritam\thinspace{\dandab} \dontdisplaylinenum

ayutaṁ prayutair v\textsubring{r}ndaiḥ prayutaṁ niyutair v\textsubring{r}taḥ \veg\dontdisplaylinenum
\varr{
        \ \vab ekaikam a°\lem  \msCa\msCb\msNb\msNc\Ed; ekaikaṁ ma° \msCc\msNa
        \ \vb parivāritam\lem  \mssCaCbCc\msNa\msNb\msNc; parivāritamāḥ \Ed
        \ \vc ayutaṁ\lem  \Ed; ayutaiḥ \mssCaCbCc\msNa\msNc, ayutai \msNb\oo
                 prayutair v\textsubring{r}°\lem  \mssCaCbCc\msNa\msNb\Ed; prayutai v\textsubring{r}° \msNc
        \ \vd prayutaṁ niyutair v\textsubring{r}tam\lem  \eme; 
                        prayutaṁ niyutair v\textsubring{r}taḥ \Ed, prayutair niyutair v\textsubring{r}taḥ \msCa\msCb\msNa\msNc,
                        prayuter niyutair v\textsubring{r}taḥ \msCc, prayutai niyutai v\textsubring{r}taḥ \msNb}

ekaikasya parīvāro niyutaḥ p\textsubring{r}thag eva ca\thinspace{\dandab} \dontdisplaylinenum

koṭibhir daśakoṭyena ekaikaḥ parivāritaḥ \veg\dontdisplaylinenum
\varr{
        \ \va parīvāro\lem  \mssCaCbCc\msNa\msNb\msNc; parivāro \Ed
        \ \vb niyutaḥ\lem  \msCa\msCb\msNa\msNb\msNc\Ed; niyuta \msCc\oo
                 ca\lem  \mssCaCbCc\msNa\msNb\msNcpcorr\Ed; caḥ \msNcacorr
        \ \vc koṭibhir da°\lem  \msCa\msCc\msNa\msNb\msNc\Ed; koṭibhi \msCb\oo
                 °koṭyena\lem  \mssCaCbCc\Ed; °koṭyona \msNa\msNc, °koṭyenaḥ \msNb
        \ \vd ekaikaḥ\lem  \msCa\msCb\msNa\Ed; ekaika \msCc\msNb\msNc\oo
                 parivāritaḥ\lem  \msCb\msCc\msNa\msNb\msNc\Ed; parivāri\uncl{taḥ} \msCa}

daśakoṭiṣu ekaikaṁ v\textsubring{r}ndav\textsubring{r}ndabh\textsubring{r}tair v\textsubring{r}tam\thinspace{\dandab} \dontdisplaylinenum

v\textsubring{r}ndavargeṣu ekaikaṁ kharvabhiḥ parivāritam \veg\dontdisplaylinenum
\varr{
        \ \va °koṭiṣu\lem  \msCb\msCc\msNb\Ed; °koṭīṣu \msCa\msNa\msNc
        \ \vb °v\textsubring{r}ndabh\textsubring{r}tair v\textsubring{r}tam\lem  \mssCaCbCc\msNb; °v\textsubring{r}ndav\textsubring{r}tair v\textsubring{r}taṁ \msNa,
                                °v\textsubring{r}ndabh\textsubring{r}tai v\textsubring{r}taṁ \msNc, °v\textsubring{r}ndaṁ v\textsubring{r}tair v\textsubring{r}taḥ \Ed
        \ \vd kharvabhiḥ\lem  \mssCaCbCc\msNa\msNb\Ed; kharvarbhiḥ \msNc\oo
                 °vāritam\lem  \mssCaCbCc\msNa\msNb\msNc; °vāritaḥ \Ed}

kharvavargeṣu ekaikaṁ daśakharvagaṇair v\textsubring{r}tam\thinspace{\dandab} \dontdisplaylinenum

daśakharveṣu ekaikaṁ śaṅkubhiḥ parivāritam \veg\dontdisplaylinenum
\varr{
        \ \vb °gaṇair v\textsubring{r}tam\lem  \msCa\msCc\msNa\msNb; °gaṇai v\textsubring{r}tam \msCb, °gaṇe v\textsubring{r}ttaṁ \msNc, 
                                                                        °gaṇair v\textsubring{r}taḥ \Ed
        \ \vc °kharveṣu\lem  \mssCaCbCc\msNa\msNb\Ed; °garveṣu \msNc
        \ \vd °vāritam\lem  \mssCaCbCc\msNa\msNb\msNc; °vāritaḥ \Ed}

śaṅkubhiḥ p\textsubring{r}thag ekaikaṁ padmena parivāritam\thinspace{\dandab} \dontdisplaylinenum

padmavargeṣu ekaikaṁ samudraiḥ parivāritam \veg\dontdisplaylinenum
\varr{
        \ \va p\textsubring{r}thag ekaikaṁ\lem  \eme; p\textsubring{r}thag enaiva \msCa\msCc\msNa\msNb\msNc\Ed, p\textsubring{r}thag ainaiva \msCb
        \ \vb °vāritam\lem  \msNapcorr; °vāritaḥ \mssCaCbCc\msNb\msNc\Ed, °taṁ \msNaacorr
        \ \vd samudraiḥ\lem  \msCc\msNa\msNb\msNc\Ed; samudaiḥ \msCa, damu\uncl{daiḥ} \msCb\oo
                 °vāritam\lem  \mssCaCbCc\msNa\msNb\msNc; °vāritaḥ \Ed}

samudreṣu tathaikaikaṁ madhyasaṁkhyais tu tair v\textsubring{r}tam\thinspace{\dandab} \dontdisplaylinenum

madhyasaṁkhyeṣu ekaikam anantaiḥ parivāritam \veg\dontdisplaylinenum
\varr{
        \ \va tathai°\lem  \msCa\msCb\msNa\msNb\msNc\Ed; tathe° \msCc
        \ \vb madhyasaṁkhyais tu tair v\textsubring{r}tam\lem  \mssCaCbCc\msNa;
                        madhyasakhyais tu tai v\textsubring{r}tam \msNb,
                        madhyasakhyais tu ter v\textsubring{r}taṁ \msNc,
                        madhye śaṅkhyāyutair v\textsubring{r}taḥ \Ed
        \ \vc madhyasaṁkhyeṣu\lem  \mssCaCbCc\msNa\msNb\msNc; madhye śaṁkheṣu \Ed
        \ \vcd ekaikam anantaiḥ\lem  \mssCaCbCc\msNa\msNb\Ed; ekaikaṁ manataiḥ \msNc
        \ \vd °vāritam\lem  \mssCaCbCc\msNa\msNb\msNc; °vāritaḥ \Ed}

ananteṣu ca ekaikaṁ parārdhaparivāritam\thinspace{\dandab} \dontdisplaylinenum

parārdheṣu ca ekaikaṁ pareṇa parivāritam \danda\dontdisplaylinenum

eṣa vai kathito vipra śakyaṁ sāṁkhyam udīritam \veg\dontdisplaylinenum


\alalfejezet{pramāṇam}\varr{
        \ \vb parārdhaparivāritam\lem  \msCa\msCb\msNa\msNb\msNc; parārdha{\lost}{\lost}{\lost}ritam \msCc,
                                                            parārdhaiḥ parivāritaḥ \Ed
        \ \vd °vāritam\lem  \mssCaCbCc\msNa\msNc; °vārivāritaṁ \msNb, °vāritaḥ \Ed
        \ \ve kathito\lem  \mssCaCbCc\msNa\msNc; \uncl{kathito} \msNb, kathitā \Ed
        \ \vf śakyaṁ\lem  \msCa\msCb\msNa\msNb\msNc\Ed; śakya \msCc\oo
                 sāṁkhyam u°\lem  \msCa\msCc\msNb; sākhyam u° \msCb, syakhyam u° \msNa,
                                                saṁkhyam u \msNc, saṁkhyām u° \Ed}

pramāṇaṁ ś\textsubring{r}ṇu me vipra saṁkṣepād bruvato mama\thinspace{\dandab} \dontdisplaylinenum

candrodaye pūrṇamāsyāṁ vapur aṇḍasya tād\textsubring{r}śam \veg\dontdisplaylinenum
\varr{
        \ \va pramāṇaṁ\lem  \msCc\msNa\msNc\Ed; praṇāmaṁ \msCa\msCb, pramāṇa \msNb
        \ \vb saṁkṣepād bruvato\lem  \msCa\msCc\msNa\msNb\Ed; saṁkṣepād vadato \msCb,
                                        saṁkhyepād bruvato \msNc}

koṭikoṭisahasraṁ tu yojanānāṁ samantataḥ\thinspace{\dandab} \dontdisplaylinenum

aṇḍānāṁ ca parīmāṇaṁ brahmaṇā parikīrtitam \veg\dontdisplaylinenum
\varr{
        \ \va parī°\lem  \msCa\msCc\msNa\msNb\msNc\Ed; pari° \msCb\ \unmetr
        \ \vb brahmaṇā\lem  \msCa\msCb\msNa\msNb\msNc\Ed; {\lost}{\lost}{\lost} \msCc\oo
                 °kīrtitam\lem  \msCa\msCb\msNb\msNc\Ed; °kīrti\uncl{tāḥ} \msCc, °kīrtitaḥ \msNa}

saptakoṭisahasrāṇi saptakoṭiśatāni ca\thinspace{\dandab} \dontdisplaylinenum

viṁśakoṭiṣu gulmeṣu ūrdhvatas tapate raviḥ \veg\dontdisplaylinenum
            \paral{\textit{\vcd {\normalfont The folio in \msNb\ ends with } ūrdhva°, {\normalfont and the folios that 
                may have contained verses 1.62cd--2.22 are missing.}}}
\varr{
        \ \vd raviḥ\lem  \mssCaCbCc\msNa\msNc; ravi \Ed}

pramāṇaṁ nāma saṁkhyā ca kīrtitāni samāsataḥ\thinspace{\dandab} \dontdisplaylinenum

brahmāṇḍaṁ cāprameyāṇāṁ lakṣaṇaṁ parikīrtitam \veg\dontdisplaylinenum


\alalfejezet{vyāsāḥ}\varr{
        \ \va pramāṇaṁ\lem  \msCa\msCc\msNa\msNc\Ed; praṇāmaṁ \msCb
        \ \vc brahmāṇḍaṁ cā°\lem  \msNa; brahmāṇḍaś ca \msCa\msCb\msNc, \uncl{brahmāṇḍāś cā}° \msCc, 
                                                                brahmāṇḍāś cā \Ed\oo
                 °meyāṇāṁ\lem  \msCa\msNa\Ed; °meyāṇā \msCb\msCc\msNc
        \ \vd °kīrtitam\lem  \msCa\msCb\msNa\msNc\Ed; °kīrtitāḥ \msCc}

purāṇāśīsahasrāṇi śatāni dvijasattama\thinspace{\dandab} \dontdisplaylinenum

brahmaṇā kathitaṁ pūrṇaṁ mātariśvā yathātatham \veg\dontdisplaylinenum
\varr{
        \ \vb °sattama\lem  \msCa\msCb\msNa\msNc\Ed; {\lost}{\lost}maḥ  \msCc
        \ \vc pūrṇaṁ\lem  \msCa\msCc\msNa\Ed; pūrve \msCb, pūrṇṇa \msNc
        \ \vd °tatham\lem  \msCa\msCb\msNa\msNc\Ed; °tathā \msCc}

vāyunā pāda saṁkṣipya prāptaṁ cośanasaṁ purā\thinspace{\dandab} \dontdisplaylinenum

tenāpi pāda saṁkṣipya prāptavāṁś ca b\textsubring{r}haspatiḥ \veg\dontdisplaylinenum
\varr{
        \ \vb prāptaṁ cośanasaṁ\lem  \msCb\msNa\msNc; prāptaṁ causanasaṁ \msCa, prāpta{\il}ausanasaṁ \msCc,
                                                           prāptaś cośanasaṁ \Ed}

b\textsubring{r}haspatis tu provāca sūryaṁ triṁśatsahasrikam\thinspace{\dandab} \dontdisplaylinenum

pañcaviṁśatsahasrāṇi m\textsubring{r}tyuṁ prāha divākaraḥ \veg\dontdisplaylinenum
\varr{
        \ \vb sūryaṁ\lem  \msCc\Ed; sūryas \msCa\msNa\msNc, sūrya \msCb\oo
                 triṁśatsa°\lem  \msCa\msCb\msNa\msNc\Ed; triṁśasa° \msCc
        \ \vc °viṁśatsahasrāṇi\lem  \corr; °viṁśahasrāṇi \msCa,
                                            °viṁśasahasrāṇi \msCb\msCc\msNa\msNc, °viśatsahasrāṇi \Ed}

ekaviṁśatsahasrāṇi m\textsubring{r}tyunendrāya kīrtitam\thinspace{\dandab} \dontdisplaylinenum 

indreṇāha vasiṣṭhāya viṁśatślokasahasrikam \veg\dontdisplaylinenum
\varr{
        \ \va °viṁśat°\lem  \Ed; °viṁśa° \mssCaCbCc\msNa\msNc
        \ \vb kīrtitam\lem  \Ed; kīrtitaḥ \msCa\msCb\msNa\msNcpcorr, kīrtitāḥ \msCc, kīrttita \msNcacorr
        \ \vc vasiṣṭhāya\lem  \msCa\msCc\msNa\Ed; viśiṣṭhāya \msCb, vahiṣṭhāya \msNc
        \ \vd viṁśatślo°\lem  \corr; viṁśaślo° \msCa\msCc\msNa\msNc\Ed, viśaślo° \msCb}

aṣṭādaśasahasrāṇi tena sārasvatāya tu\thinspace{\dandab} \dontdisplaylinenum

sārasvatas tridhāmāya sahasradaśa sapta ca \veg\dontdisplaylinenum
\varr{
        \ \va aṣṭā°\lem  \mssCaCbCc\msNa\Ed; āṣṭā° \msNc
        \ \vc sārasvatas tri°\lem  \eme; sārasvatā tri° \msCa\msCc\msNa\msNc\Ed, sārasvatās tri° \msCb\oo
                 °dhāmāya\lem  \mssCaCbCc\msNapcorr\msNc\Ed; \om\ \msNaacorr}

ṣoḍaśānāṁ sahasrāṇi bharadvājāya vai tataḥ\thinspace{\dandab} \dontdisplaylinenum

daśa pañcasahasrāṇi triv\textsubring{r}ṣāya abhāṣata \veg\dontdisplaylinenum
\varr{
        \ \vb bhara°\lem  \msCa\msCb\msNa\msNc\Ed; bhāra° \msCc
        \ \vd abhāṣata\lem  \msCa\msCb\msNa; a\uncl{bhāṣata} \msCc, abhāṣataḥ \msNc\Ed}

caturdaśasahasrāṇi antarīkṣāya vai tataḥ\thinspace{\dandab} \dontdisplaylinenum

trayyāruṇiṁ sahasrāṇi trayodaśa abhāṣata \veg\dontdisplaylinenum
\varr{
        \ \vc trayyāruṇiṁ\lem  \corr; tryaiyāruṇi \msCa\msCb\msNa, traiyāruṇi \msCc\Ed, tryaiyārūpini \msNc
        \ \vd abhāṣata\lem  \msCa\msCc\msNc; abhāṣataḥ \msCb, svabhāvata \msNa, hy abhāṣata \Ed}

trayyāruṇis tu viprendro dhanaṁjayam abhāṣata\thinspace{\dandab} \dontdisplaylinenum

dvādaśāni sahasrāṇi saṁkṣipya punar abravīt \veg\dontdisplaylinenum
\varr{
        \ \va trayyāruṇi°\lem  \corr; tryaiyāruṇi° \mssCaCbCc\msNc, traiyāruṇi° \msNa\Ed\oo
                 viprendro\lem  \msCa\msCb\msNa\msNc\Ed; viprenda \msCc
        \ \vb dhanaṁjaya°\lem  \mssCaCbCc\msNapcorr\msNc\Ed; dhana° \msNaacorr\oo
                 °bhāṣata\lem  \msCa\msCc\msNa\msNc; °bhāṣataḥ \msCb\Ed}

k\textsubring{r}taṁjayāya samprāpto dhanaṁjayamahāmuniḥ\thinspace{\dandab} \dontdisplaylinenum

k\textsubring{r}taṁjayād dvijaśreṣṭha \textsubring{r}ṇaṁjayamahātmane \veg\dontdisplaylinenum
\varr{
        \ \vc °jayād dvi°\lem  \msCa\msNa\Ed; °jayā dvi° \msCb\msCc\msNc\oo
                 °śreṣṭha\lem  \mssCaCbCc\msNa\msNc; °śreṣṭho \Ed
        \ \vd \textsubring{r}ṇaṁjaya°\lem  \msCa\msCc\msNa\msNc\Ed; \textsubring{r}ṇaṁjāya° \msCb}

\textsubring{r}ṇañjayāt punaḥ prāpto gautamāya maharṣiṇe\thinspace{\dandab} \dontdisplaylinenum

gautamāc ca bharadvājas tasmād dharyadvatāya tu \veg\dontdisplaylinenum
\varr{
        \ \va prāpto\lem  \mssCaCbCc\msNa\msNc; prāptau \Ed
        \ \vc gautamāc ca\lem  \mssCaCbCc\msNa\Ed; gautamāś ca \msNc\oo
                 bharadvāja°\lem  \msCa\msCc\msNa\msNc\Ed; bharadvāra° \msCb
        \ \vd tasmād dharyadvatāya\lem  \mssCaCbCc\msNa\msNc; tasmād damyāddamāya \Ed}

rājaśravās tataḥ prāptaḥ somaśuṣmāya vai tataḥ\thinspace{\dandab} \dontdisplaylinenum

somaśuṣmāt tataḥ prāptas t\textsubring{r}ṇabindus tu bho dvija \veg\dontdisplaylinenum
\varr{
        \ \va rājaśravās\lem  \eme; rājaśrava \mssCaCbCc\msNa\Ed, rājaśrava \msNc
        \ \vc °śuṣmāt ta°\lem  \mssCaCbCc\msNc\Ed; °śuṣmā ta° \msNa
        \ \vcd prāptas t\textsubring{r}°\lem  \msCa\msCb\msNa\msNc\Ed; prā\uncl{pta t\textsubring{r}}° \msCc
        \ \vd bho\lem  \msCa\msCc\msNa\msNc\Ed; \om\ \msCb}

t\textsubring{r}ṇabindus tu v\textsubring{r}kṣāya v\textsubring{r}kṣaḥ śaktim abhāṣata\thinspace{\dandab} \dontdisplaylinenum

śaktiḥ parāśaraṁ prāha jātūkarṇāya vai tataḥ \veg\dontdisplaylinenum
\varr{
        \ \vb °bhāṣata\lem  \msCa\msCb\msNa\msNc; °bhāṣataḥ \msCc\Ed
        \ \vd jātū°\lem  \corr; jatu° \msCa\msCc\msNa\msNc\Ed, tu° \msCb}

dvaipāyanaṁ tu provāca jātūkarṇo maharṣiṇam\thinspace{\dandab} \dontdisplaylinenum

romaharṣāya samprāpto dvaipāyanamahāmuniḥ \veg\dontdisplaylinenum
\varr{
        \ \va dvaipāyanaṁ tu\lem  \eme; dvaipāyanas tu \mssCaCbCc\msNa\msNc, dvaipāyanāya \Ed
        \ \vb jātūkarṇo maharṣiṇam\lem  \eme;
                         jatukarṇo maharṣiṇam \msCa\msCb\msNapcorr\msNc, jatukarṇā maharṣiṇaḥ \msCc, 
                         jakarṇo maharṣiṇaṁ \msNaacorr, jatukarṇamaharṣiṇā \Ed
        \ \vd °muniḥ\lem  \mssCaCbCc\msNa\msNc; °muni \Ed}

romaharṣāya provāca putrāyāmitabuddhaye\thinspace{\dandab} \dontdisplaylinenum

daśadve ca sahasrāṇi purāṇaṁ samprakāśitam \danda\dontdisplaylinenum

mānuṣāṇāṁ hitārthāya kiṁ bhūyaḥ śrotum icchasi \veg\dontdisplaylinenum


\jump
\begin{center}
\ketdanda iti v\textsubring{r}ṣasārasaṁgrahe brahmāṇḍasaṁkhyā nāmādhyāyaḥ prathamaḥ\ketdanda
\end{center}
\dontdisplaylinenum\vers 
\varr{
        \ \va °harṣāya\lem  \mssCaCbCc\msNa\msNc; °harṣaṇāya \Ed
        \ \vb purāṇaṁ samprakāśitam\lem  \msCa\msCb\msNa\msNc\Ed; purāṇa samprakāśitāṁ \msCc
        \ \ve mānuṣāṇāṁ\lem  \msCa\msCc\msNa\msNc\Ed; manuṣāṇāṁ \msCb
        \ \vf bhūyaḥ\lem  \mssCaCbCc\msNa\msNc; bhūya \Ed\oo
                 °cchasi\lem  \msCa\msCb\msNa\msNc\Ed; °cchasīti \msCc
        \ {\normalfont  \Colo: } nāmādhyāyaḥ prathamaḥ\lem  \mssCaCbCc\msNa\msNc; 
                                nāma prathamo 'dhyāya \Ed}
\bekveg\szamveg\vfill\phpspagebreak\szam\bek\versno=0\fejno=2
\thispagestyle{empty}



\alfejezet{\textbf{dvitīyo 'dhyāyaḥ}}\jump\jump
\vers

vigatarāga uvāca~{\dandab}\dontdisplaylinenum 

śrutaṁ mayā janāgreṇa brahmāṇḍasya tu nirṇayam\thinspace{\danda} \dontdisplaylinenum

pramāṇaṁ varṇarūpaṁ ca saṁkhyā tasya samāsataḥ \veg\dontdisplaylinenum
            \paral{\textit{{\normalfont Testimonia for this chapter: \msCa\ ff.\thinspace 195v--197r, 
                                             \msCb\ ff.\thinspace 203v--204v, 
                                             \msCc\ ff.\thinspace 270r--270v (it breaks off at 2.21 and resumes at 3.30b),
                                             \msNa\ ff.\thinspace 3v--4v, 
                                             \msNb\ exp.\thinspace 43 and 42 (sic!) (it broke off at 1.62d and resumes at 2.23),
                                             \msNc\ ff.\thinspace 211v--213r;
                \mssCaCbCc\ = \msCa + \msCb + \msCc }}}
\varr{
        \ \va janāgreṇa\lem  \msCb\msCc\msNa\msNc\Ed; janā{\lost}{\lost} \msCa}

śivāṇḍeti tvayā prokto brahmāṇḍālayakīrtitaḥ\thinspace{\dandab} \dontdisplaylinenum

kīd\textsubring{r}śaṁ lakṣaṇaṁ jñeyaṁ pramāṇaṁ tasya vā kati \veg\dontdisplaylinenum
\varr{
        \ \vb brahmāṇḍā°\lem  \mssCaCbCc\msNa\msNc; brahmāṇḍa \Ed
        \ \vc jñeyaṁ\lem  \msCa\msCb\msNa\msNc\Ed; jñeyā \msCc
        \ \vd kati\lem  \msCa\msCb\msNa\msNc\Ed; katiḥ \msCc}

kasya vālayanaṁ jñeyaṁ pramāṇaṁ vātra vāsinaḥ\thinspace{\dandab} \dontdisplaylinenum

kā vā tatra prajā jñeyā ko vā tatra prajāpatiḥ \veg\dontdisplaylinenum


\alalfejezet{śivāṇḍasaṁkhyā}\varr{
        \ \va layanaṁ jñeyaṁ\lem  \msCa\msCc\msNa\msNc; layanaṁ \msCb, lakṣaṇaṁ jñeyaṁ \Ed
        \ \vb vāsinaḥ\lem  \msCa\msCc\msNa\msNc\Ed; vāsirānaḥ \msCb
        \ \vc kā\lem  \eme; ko \mssCaCbCc\msNa\msNc, kiṁ \Ed\oo
                 prajā jñeyā\lem  \msCb\msCc\msNa\msNc\Ed; pra\uncl{jā}{\lost}yā \msCa}

anarthayajña uvāca~{\dandab}\dontdisplaylinenum 

śivāṇḍalakṣaṇaṁ vipra na tvaṁ praṣṭum ihārhasi\thinspace{\danda} \dontdisplaylinenum

daivatair api kā śaktir jñātuṁ draṣṭuṁ ca tattvataḥ \veg\dontdisplaylinenum
\varr{
        \ \vb na tvaṁ\lem  \mssCaCbCc\msNa\msNc; tatvaṁ \Ed\oo
                 °rhasi\lem  \mssCaCbCc\msNa\Ed; °hasi \msNc
        \ \vc daivatai°\lem  \msCa\msCb\msNa; devatai° \msCc\msNc\Ed\oo
                 śaktir\lem  \msCa; śakti \msCb\msCc\msNa\msNc\Ed}

agamyagamanaṁ guhyaṁ guhyād api samuddhitam\thinspace{\dandab} \dontdisplaylinenum
             \paral{\textit{\vab {\normalfont cf. Liṅgapurāṇa 1.21.71ab: } namo guṇyāya guhyāya agamyagamanāya ca}}

na prabhur netaras tatra na daṇḍyo na ca daṇḍakaḥ \veg\dontdisplaylinenum
\varr{
        \ \va agamyagamanaṁ\lem  \msCa\msCb\msNa\Ed; agamyagagahanaṁ \msCc, agamyagagamanaṁ \msNc
        \ \vb guhyā°\lem  \msNc\Ed; guhā° \mssCaCbCc\msNa\oo
                 samuddhitam\lem  \mssCaCbCc\msNa; samraddhitaṁ \msNc, sam\textsubring{r}ddhidam \Ed
        \ \vc prabhur ne°\lem  \msCa\msCb\msNa\msNc\Ed; prane° \msCc
        \ \vd daṇḍyo\lem  \msCc\msNa\msNc; daṇḍo \msCa\msCb, daṇḍyā \Ed\oo
                 daṇḍakaḥ\lem  \msCa\msCc\msNa\msNc\Ed; ṇḍakaḥ \msCbacorr, paṇḍakaḥ \msCbpcorr}

na satyo nān\textsubring{r}tas tatra suśīlo no duḥśīlavān\thinspace{\dandab} \dontdisplaylinenum

nān\textsubring{r}jur na ca dambhitvaṁ na t\textsubring{r}ṣṇā na ca īrṣyatā \veg\dontdisplaylinenum
\varr{
        \ \va satyo\lem  \mssCaCbCc\msNa\msNc; satyau \Ed\oo
                 tatra\lem  \mssCaCbCc\msNa\msNc; tatrā \Ed
        \ \vb no\lem  \msCb\msCc\msNa\msNc\Ed; {\lost} \msCa
        \ \vc nān\textsubring{r}jur na\lem  \eme; nā\textsubring{r}jur nna \msCa\Ed, nā\textsubring{r}jur na \msCb\msNc, nā\textsubring{r}jun na \msNa,
                                                \uncl{nā\textsubring{r}ju na} \msCc
        \ \vd na t\textsubring{r}ṣṇā na ca\lem  \mssCaCbCc\msNc\Ed;  na ca t\textsubring{r}ṣṇā na \msNa\oo
                 īrṣyatā\lem  \msCa\msCb\msNa\msNc; īrṣyatāḥ \msCc, irṣyatā \Ed}

na krodho na ca lobho 'sti na māno 'sti na sūyakaḥ\thinspace{\dandab} \dontdisplaylinenum

īrṣyā dveṣo na tatrāsti na śaṭho na ca matsaraḥ \veg\dontdisplaylinenum
\varr{
        \ \va krodho\lem  \msCa\msCb\msNa\msNc\Ed; krodhau \msCc
        \ \vb sūyakaḥ\lem  \msCa\msCc\msNa\msNc; sūcakaḥ \msCb, steyakaḥ \Ed\ \unmetr
         \ \vd śaṭho\lem  \msCa\msCb\msNa\msNc; ṣaṭho \msCc, śaṭhe \Ed\oo
                  matsaraḥ\lem  \mssCaCbCc\msNa\msNc; matsarāḥ \Ed}

na vyādhir na jarā tatra na śoko 'sti na viklavaḥ\thinspace{\dandab} \dontdisplaylinenum

nādhamaḥ puruṣas tatra nottamo na ca madhyamaḥ \veg\dontdisplaylinenum
\varr{
        \ \va vyādhir na\lem  \msCa\msCb\msNa\Ed; vyādhi na \msCc\msNc\oo
                 jarā tatra\lem  \msCb\msNc; jarās tatra \msCa\msCc\msNa\Ed
        \ \vb viklavaḥ\lem  \mssCaCbCc\msNa\msNc; viklava \Ed}

notk\textsubring{r}ṣṭo mānavas tasmin striyaś caiva śivālaye\thinspace{\dandab} \dontdisplaylinenum

na nindā na praśaṁsāsti matsarī piśuno na ca \veg\dontdisplaylinenum
\varr{
        \ \va mānava°\lem  \msCb\msCc\msNa\msNc\Ed; mā{\lost}va° \msCa
        \ \vc praśaṁsāsti\lem  \mssCaCbCc\msNa\msNc; praśaṁsāś ca \Ed}

garvadarpaṁ na tatrāsti krūramāyādikaṁ tathā\thinspace{\dandab} \dontdisplaylinenum

yācamāno na tatrāsti dātā caiva na vidyate \veg\dontdisplaylinenum
\varr{
        \ \vc tatrāsti\lem  \mssCaCbCc\msNapcorr\msNc\Ed; tatrā \msNaacorr}

anarthī vraja tatrasthaḥ kalpav\textsubring{r}kṣasamāśritaḥ\thinspace{\dandab} \dontdisplaylinenum

na karma nāpriyas tatra na kaliḥ kalaho na ca \veg\dontdisplaylinenum
\varr{
        \ \va vraja ta°\lem  \mssCaCbCc\msNa\Ed; vrajas ta° \msNc
        \ \vc karma nā°\lem  \eme; karma na \mssCaCbCc\msNa\msNc, karmaṇā \Ed
        \ \vd kaliḥ\lem  \mssCaCbCc\msNa\msNcpcorr; kali \msNcacorr\Ed}

dvāparo na ca na tretā k\textsubring{r}taṁ cāpi na vidyate\thinspace{\dandab} \dontdisplaylinenum

manvantaraṁ na tatrāsti kalpaś caiva na vidyate \veg\dontdisplaylinenum
\varr{
        \ \va ca na tretā\lem  \msCc\msNa\msNc\Ed; ca na tretrā \msCa, ca tretā na \msCb
        \ \vb k\textsubring{r}taṁ cā°\lem  \msCc\msNa; k\textsubring{r}taś cā° \msCa\msCb\msNc\Ed
        \ \vc manvantaraṁ na tatrāsti\lem  \msCa\msCb\msNa\Ed; 
                                manvantatrāsti \msCc,
                                manvantarananta tatrāsti \msNc
        \ \vd kalpaś caiva\lem  \mssCaCbCc\msNc\Ed; kalpaṁ caiva \msNa}

āhūtasamplavaṁ nāsti brahmarātridinaṁ tathā\thinspace{\dandab} \dontdisplaylinenum

na janmamaraṇaṁ tatra āpadaṁ nāpnuyāt kvacit \veg\dontdisplaylinenum
\varr{
        \ \va āhūta°\lem  \mssCaCbCc\msNa\msNc; ābhūta° \Ed
        \ \vb brahmarātridinaṁ\lem  \mssCaCbCc\msNa\msNc; brahmarātridivas \Ed
        \ \vc janmamaraṇaṁ tatra\lem  \msCc\msNa\Ed; janmaraṇaṁ tatra \msCa\msCb,
                                                janmamaraṇantrata \msNc
        \ \vd āpadaṁ\lem  \mssCaCbCc\msNa\msNc; apadaṁ \Ed}

na cāśāpāśabaddho 'sti rāgamohaṁ na vidyate\thinspace{\dandab} \dontdisplaylinenum

na devā nāsurās tatra na yakṣoragarākṣasāḥ \veg\dontdisplaylinenum
\varr{
        \ \va cāśāpāśa°\lem  \msCb\msNcpcorr; ca sāyāśa° \msCa\msCc\msNa\msNcacorr\Ed\oo 
                 °baddho\lem  \msCa\msCb\msNa\msNc; °ddho \msCc, °v\textsubring{r}ddho \Ed
        \ \vb °mohaṁ\lem  \msCb\msCc\msNa\msNc\Ed; °moho \msCa
        \ \vc devā\lem  \msCa\msCc\msNa\msNc\Ed; devo \msCb}

na bhūtā na piśācāś ca gandharvā \textsubring{r}ṣayas tathā\thinspace{\dandab} \dontdisplaylinenum

tārā grahaṁ na tatrāsti nāgakiṁnaragāruḍam \veg\dontdisplaylinenum
\varr{
        \ \vb gandharvā\lem  \mssCaCbCc\msNa\msNc;  gandharvo \Ed}

na japo nāhnikas tatra nāgnihotrī na yajñak\textsubring{r}t\thinspace{\dandab} \dontdisplaylinenum

na vrataṁ na tapaś caiva na tiryaṁ narakaṁ tathā \veg\dontdisplaylinenum
            \paral{\textit{\vd {\normalfont Cf. 19.48cd: }viśiṣṭhe tv indriyagrāme tiryannarakasādhanam}}
\varr{
        \ \va japo\lem  \msCb\msCc\msNa\msNc\Ed; jayo \msCa\oo
                 nāhnikas ta°\lem  \msCa\msCc\msNa\msNc\Ed; nāhnika ta° \msCb
        \ \vd na tiryaṁ narakaṁ\lem  \eme; nātiryannarakas \msCa\msCc\msNa,
                                 nātiryanarakan \msCb, nātriryaṁ narakas \msNc, na tīrthannarakan \Ed}

tasyeśānasya devasya aiśvaryaguṇavistaram\thinspace{\dandab} \dontdisplaylinenum

api varṣaśatenāpi śakyaṁ vaktuṁ na kenacit \veg\dontdisplaylinenum

harecchāprabhavāḥ sarve paryāyeṇa bravīmi te\thinspace{\dandab} \dontdisplaylinenum

devamānuṣavarjyāni v\textsubring{r}kṣagulmalatādayaḥ \veg\dontdisplaylinenum
\varr{
        \ \va harecchāprabhavāḥ\lem  \msNc; harecchaprabhavāḥ \mssCaCbCc\msNa, harecchāprabhavā \Ed
        \ \vc varjyāni\lem  \mssCaCbCc\msNa\msNc; vajjñāni \Ed}

parārdhadviguṇotsedhā vistāraś ca tathāvidhaḥ\thinspace{\dandab} \dontdisplaylinenum

anekākārapuṣpāṇi phalāni ca manoharam \veg\dontdisplaylinenum
\varr{
        \ \va °guṇotsedhā\lem  \conj; °guṇocchedhā \msCa\msCb\msNa\msNc, °guṇecchedhā \msCc, °guṇācchredhā \Ed
        \ \vb vistāraś ca\lem  \msNc; vistāraṁ ca \mssCaCbCc\msNa\Ed\oo
                 °vidhaḥ\lem  \msNc; °vidhā \mssCaCbCc\msNa\Ed
        \ \vc anekākāra°\lem  \msCb\msCc\msNa\msNc\Ed; anekāra° \msCa}

anye kāñcanav\textsubring{r}kṣāṇi maṇiv\textsubring{r}kṣāṇy athāpare\thinspace{\dandab} \dontdisplaylinenum

pravālamaṇiṣaṇḍāś ca padmarāgaruhāni ca \veg\dontdisplaylinenum
\varr{
        \ \va anye\lem  \mssCaCbCc\msNa\msNc; bahu° \Ed
        \ \vc ṣaṇḍāś ca\lem  \mssCaCbCc\msNa\msNc; ghaṇṭāś ca \Ed
        \ \vd °ruhāni\lem  \msCa\msCb\msNa\msNc; °ruhāṇi \msCc, °sahāni \Ed}

svādumūlaphalāskandalatāviṭapapādapāḥ\thinspace{\dandab} \dontdisplaylinenum

kāmarūpāś ca te sarve kāmadāḥ kāmabhāṣiṇaḥ \veg\dontdisplaylinenum
\varr{
        \ \va svādu°\lem  \msCb\msCc\msNa\msNc\Ed; svādhu° \msCa\oo
                 °mūla°\lem  \mssCaCbCc\msNc\Ed; °mūlā \msNa}

tatra vipra prajāḥ sarve anantaguṇasāgarāḥ\thinspace{\dandab} \dontdisplaylinenum

tulyarūpabalāḥ sarve sūryāyutasamaprabhāḥ \veg\dontdisplaylinenum
\varr{
        \ \vc °bālāḥ\lem  \msCa\msCb\msNa\msNc; °varāḥ \Ed}

parārdhadvayavistāraṁ parārdhadvayam āyatam\thinspace{\dandab} \dontdisplaylinenum

parārdhadvayavikṣepā yojanānāṁ dvijottama \veg\dontdisplaylinenum
\varr{
        \ \vc °dvaya°\lem  \msCa\msCb\msNapcorr\msNb\msNc\Ed; °dva° \msNaacorr\oo
                 vikṣepā\lem  \msCa\msCb\msNa\msNb\msNc; vijñeyā \Ed
        \ \vd °ttama\lem  \msCa\msCb\msNb\msNc\Ed; °ttamaḥ \msNa}

aiśvaryatvaṁ na saṁkhyāsti balaśaktiś ca bho dvija\thinspace{\dandab} \dontdisplaylinenum

adhordhvo na ca saṁkhyāsti na tiryañ caiti kaścana \veg\dontdisplaylinenum
\varr{
        \ \vb balaśaktiś ca bho dvija\lem  \msCa\msCb\msNapcorr\msNb\msNc; 
                                        \om\ \msNaacorr, tava śaktiś ca bho dvija \Ed
        \ \vc adhordhvo na ca saṁkhyāsti\lem  \msCa\msCb\msNapcorr\msNb\msNc\Ed; \om\ \msNaacorr
        \ \vd na tiryañ caiti kaścana\lem  \msNapcorr\msNc;
                                                na tiryañ ceti kaścana \msCa\msCb\msNb\Ed,
                                                na tiryaṁ ceti kaścana \msNaacorr}

śivāṇḍasya ca vistāram āyāmaṁ ca na vedmy aham\thinspace{\dandab} \dontdisplaylinenum

bhogam akṣayas tatraiva janmam\textsubring{r}tyur na vidyate \veg\dontdisplaylinenum
\varr{
        \ \vc bhogam akṣayas ta°\lem  \msCa\msCb\msNa\msNb\msNc; bhogamayās tu ta° \Ed
        \ \vd °m\textsubring{r}tyur na\lem  \msCa\msCb\msNa\msNc\Ed; °m\textsubring{r}tyu na \msNb}

śivāṇḍamadhyam āśritya gokṣīrasad\textsubring{r}śaprabhāḥ\thinspace{\dandab} \dontdisplaylinenum

parārdhaparakoṭīnām īśānānāṁ sm\textsubring{r}tālayaḥ \veg\dontdisplaylinenum
\varr{
        \ \vb prabhāḥ\lem  \msCa\msCb\msNa\msNb\msNc; prabhā \Ed
        \ \vd °śānānāṁ\lem  \msCa\msCb\msNa\Ed; °śānānā \msNb, °gānānāṁ \msNc\oo
                 sm\textsubring{r}tālayaḥ\lem  \msCa\msNb\msNc; sm\textsubring{r}tālaya \msCb, sm\textsubring{r}tālayaṁ \msNa, sm\textsubring{r}tālayā \Ed}

bālasūryaprabhāḥ sarve jñeyās tatpuruṣālaye\thinspace{\dandab} \dontdisplaylinenum

parārdhaparakoṭīnāṁ pūrvasyāṁ diśam āśritāḥ \veg\dontdisplaylinenum
\varr{
        \ \va °bhāḥ\lem  \msCa\msCb\msNa\msNb\msNc; °bhā \Ed
        \ \vb jñeyās ta°\lem  \msCa\msCb\msNb\msNc; jñeyā ta° \msNa\Ed\oo
                 °ālaye\lem  \msCa\msCb\msNa\msNb\msNc; °ālayaṁ \Ed
        \ \vd diśa°\lem  \msCa\msCb\msNa\msNc\Ed; diśi° \msNb}

bhinnāñjanaprabhāḥ sarve dakṣiṇāṁ diśam āśritāḥ\thinspace{\dandab} \dontdisplaylinenum

parārdhaparakoṭīnām aghorālayam āśritāḥ \veg\dontdisplaylinenum
\varr{
        \ \va °prabhāḥ\lem  \msCa\msCb\msNa\msNb\msNc; °prabhā \Ed
        \ \vb dakṣiṇāṁ\lem  \msCa\msCb\msNa\msNb\msNc; dakṣiṇa° \Ed\oo
                 diśam\lem  \msCa\msNa\msNb\msNc; diśim \msCb\Ed
        \ \vd °ghorā°\lem  \msCa\msCb\msNa\msNb\msNc; °dhorā° \Ed\oo
                 °śritāḥ\lem  \msCa\msCb\msNa\msNb\msNc; °śritā \Ed}

kundenduhimaśailābhāḥ paścimāṁ diśam āśritāḥ\thinspace{\dandab} \dontdisplaylinenum

parārdhaparakoṭīnāṁ sadyamiṣṭālayaḥ sm\textsubring{r}taḥ \veg\dontdisplaylinenum
\varr{
        \ \vb paścimāṁ\lem  \msCa\msNa\msNb\msNc\Ed; paścimā \msCb\oo
                 diśa°\lem  \msCa\msCb\msNa\msNb\Ed; diśi° \msNc\oo
                 °śritāḥ\lem  \msCa\msCb\msNa\msNb\msNc; °śritā \Ed
        \ \vd sadyamiṣṭā°\lem  \msCa\msCb\msNb\msNc\Ed; sadyamiṣṭvā° \msNa\oo
                 sm\textsubring{r}taḥ\lem  \msCa\msNa\msNb\msNc\Ed; sm\textsubring{r}tāḥ \msCb}

kuṅkumodakasaṁkāśā uttarāṁ diśam āśritāḥ\thinspace{\dandab} \dontdisplaylinenum

parārdhaparakotīnāṁ vāmadevālayaḥ sm\textsubring{r}taḥ \veg\dontdisplaylinenum
\varr{
        \ \vb uttarāṁ\lem  \msCa\msNa\msNb\msNc\Ed; uttarā \msCb\oo
                 diśam\lem  \msCb\msNa\msNb\msNc\Ed; diśim \msCa
        \ \vd °layaḥ\lem  \msCa\msCb\msNa\msNb\Ed; °laya \msNc}

īśānasya kalāḥ pañca vaktrasyāpi catuṣ kalāḥ\thinspace{\dandab} \dontdisplaylinenum

aghorasya kalā aṣṭau vāmadevās trayodaśa \veg\dontdisplaylinenum
\varr{
        \ \va kalāḥ\lem  \msCa\msCb\msNa\msNb\msNc; kalā \Ed
        \ \vb catuṣ kalāḥ\lem  \msCa\msCb\msNa\msNb\msNc; catuṣtake \Ed
        \ \vd vāmadevā°\lem  \msCa\msCb\msNa\msNc\Ed; vāmadeva° \msNb}

sadyaś cāṣṭau kalā jñeyāḥ saṁsārārṇavatārakāḥ\thinspace{\dandab} \dontdisplaylinenum

aṣṭatriṁśat kalā hy etāḥ kīrtitā dvijasattama \veg\dontdisplaylinenum
\varr{
        \ \va jñeyāḥ\lem  \msCa\msCb\msNa\msNb\msNc; jñeyā \Ed
        \ \vb saṁsārā°\lem  \msCa\msCbpcorr\msNa\msNb\msNc\Ed; saṁsā° \msCbacorr
        \ \vc °triṁśat ka°\lem  \corr; °triṁśaka° \msCa\msCb\msNa\msNb\msNc\Ed\oo
                 hy etāḥ\lem  \msCa\msCb\msNa\msNb\msNc; jñeyāḥ \Ed
        \ \vd °sattama\lem  \msCa\msCb\msNa\msNc; °sattamaḥ \msNb\Ed}

saṁkhyā varṇā diśaś caiva ekaikasya p\textsubring{r}thak p\textsubring{r}thak\thinspace{\dandab} \dontdisplaylinenum

pūrvoktena vidhānena bodhavyās tattvacintakaiḥ \veg\dontdisplaylinenum
\varr{
        \ \va saṁkhyā varṇā\lem  \msCb\msNc; saṁkhyā varṇṇo \msCa\msNb, saṁkhyā vaṇṇā \msNa, saṁdhyā varṇā \Ed
        \ \vb ekaikasya\lem  \msCa\msNb\msNc\Ed; aikaikasya \msCb\msNa
        \ \vd bodhavyās ta°\lem  \eme; bodhavyā ta° \msCa\msCb\msNa\msNb\msNc\Ed}

śivāṇḍagamanāk\textsubring{r}ṣṭyā śivayogaṁ sadābhyaset\thinspace{\dandab} \dontdisplaylinenum

śivayogaṁ vinā vipra tatra gantuṁ na śakyate \veg\dontdisplaylinenum
\varr{
        \ \va °k\textsubring{r}ṣṭyā\lem  \msCa\msCb\msNb\Ed; k\textsubring{r}ṣṭā \msNa\msNc
        \ \vb yogaṁ sadābhyaset\lem  \msCa\msCb\msNa\msNc\Ed; yoga samabhyaset \msNb
        \ \vc °yogaṁ\lem  \msCa\msCb\msNa\msNb\msNc; °yoga \Ed}

aśvamedhādiyajñānāṁ koṭyāyutaśatāni ca\thinspace{\dandab} \dontdisplaylinenum

k\textsubring{r}cchrāditapa sarvāṇi k\textsubring{r}tvā kalpaśatāni ca \danda\dontdisplaylinenum

tatra gantuṁ na śakyeta devair api tapodhana \veg\dontdisplaylinenum
\varr{
        \ \vc °tapa\lem  \Ed; °tapaḥ \msCa\msCb\msNa\msNb\msNc\ \unmetr
        \ \ve śakyeta\lem  \msCa\msNa\msNb\msNc; śakyaita \msCb, śakyete \Ed
        \ \vf devai°\lem  \msCa\msCb\msNa\msNb\Ed; deve° \msNc\oo
                 °dhana\lem  \msCa\msNa\msNb\msNc\Ed; °dhanam \msCb}

gaṅgādisarvatīrtheṣu snātvā taptvā ca vai punaḥ\thinspace{\dandab} \dontdisplaylinenum

tatra gantuṁ na śakyeta \textsubring{r}ṣibhir vā mahātmabhiḥ \veg\dontdisplaylinenum
\varr{
        \ \va gantuṁ\lem  \msCa\msCb\msNa\Ed; gantu \msNb\msNc\oo
                 śakyeta\lem  \msCa\msCb\msNa\msNb\msNc; śakyante \Ed}

saptadvīpasamudrāṇi ratnapūrṇāni bho dvija\thinspace{\dandab} \dontdisplaylinenum
            \paral{\textit{\vab {\normalfont Cf. ŚDhU 2.104: } triḥ pradatvā mahīṁ pūrṇāṁ{\normalfont ...}}}

dattvā vā vedaviduṣe śraddhābhaktisamanvitaḥ \danda\dontdisplaylinenum

tatra gantuṁ na śakyeta vinā dhyānena niścayaḥ \veg\dontdisplaylinenum
\varr{
        \ \va °dvīpa°\lem  \msCa\msCb\msNa\msNb\Ed; °dīpa° \msNc\oo
                 °samudrāṇi\lem  \msCa\msCb\msNa\msNc\Ed; °samudrāya \msNb
        \ \vc gantuṁ\lem  \msCa\msCb\msNa\Ed; gantu \msNb, gaṁntu \msNc\oo
                 śakyeta\lem  \msCa\msCb\msNa\msNb\msNc; śakyante \Ed}

svadehān māṁsam uddh\textsubring{r}tya dattvārthibhyaś ca niścayāt\thinspace{\dandab} \dontdisplaylinenum

svadāraputrasarvasvaṁ śiro 'rthibhyaś ca yo dadet \danda\dontdisplaylinenum

na tatra gantuṁ śakyeta anyair vāpi suduṣkaraiḥ \veg\dontdisplaylinenum
\varr{
        \ \va svadehān māṁsa°\lem  \msCa\msCb\msNa\msNb; svadehāt māṁsa° \msNc, svadehātmāṁ sa° \Ed
        \ \va °svaṁ\lem  \msCa\msCb\msNa\msNc\Ed; °sva \msNb
        \ \ve na tatra gantuṁ\lem  \msCa\msNa\msNb\msNc\Ed; na tatra gantuṁ na \msCb
        \ \vf °duṣkaraiḥ\lem  \msCa\msCb\msNa\msNc\Ed; °duṣk\textsubring{r}taḥ \msNb}

yajñatīrthatapodānavedādhyayanapāragaḥ\thinspace{\dandab} \dontdisplaylinenum

brahmāṇḍāntasya bhogāṁs tu bhuṅkte kālavaśānugaḥ \veg\dontdisplaylinenum
\varr{
        \ \vc °dāna°\lem  \msCa\msCb\msNc\Ed; °dānaṁ \msNa, °dānai \msNb
        \ \vd °pāragaḥ\lem  \msCb\msNa\msNc\Ed; °pāragāḥ \msCa\msNb
        \ \va brahmāṇḍāntasya bhogāṁs tu\lem  \msCa\msCb\msNa\msNc; 
                                        brahmāṇḍāntasya bhogās tu \msNb,
                                        brahmāṇḍāt tasya bhogās tu \Ed
        \ \vb bhuṅkte\lem  \msCa\msCb\msNa\msNb; \uncl{bhuṅkte} \msNc, bhuktvā \Ed\oo
                 °gaḥ\lem  \msCa\msCb\msNapcorr\msNb\msNc\Ed; °gāḥ \msNaacorr}

kālena samapreṣyeṇa dharmo yāti parikṣayam\thinspace{\dandab} \dontdisplaylinenum

alātacakravat sarvaṁ kālo yāti paribhraman \danda\dontdisplaylinenum

traikālyakalanāt kālas tena kālaḥ prakīrtitaḥ \veg\dontdisplaylinenum


\jump
\begin{center}
\ketdanda iti v\textsubring{r}ṣasārasaṁgrahe śivāṇḍasaṁkhyā nāmādhyāyo dvitīyaḥ\ketdanda
\end{center}
\dontdisplaylinenum\vers 
\varr{
        \ \vb dharmo\lem  \msCa\msCb\msNa\msNb\Ed; dharme \msNc
        \ \ve °kalanāt kāla°\lem  \msCa\msCb\msNa\msNc\Ed; °kalanā kāla° \msNb
        \ {\normalfont \Colo:} nāmādhyāyo dvitīyaḥ\lem  \msCa\msCb\msNa\msNc;
                                                                nāmādhyāya dvitīyaḥ \msNb,
                                                                nāma dvitīyo 'dhyāyaḥ \Ed}
\bekveg\szamveg\vfill\phpspagebreak\szam\bek\versno=0\fejno=3
\thispagestyle{empty}



\alfejezet{\textbf{t\textsubring{r}tīyo 'dhyāyaḥ}}\jump\jump

\alalfejezet{dharmapravacanam}
\vers

vigatarāga uvāca~{\dandab}\dontdisplaylinenum 

kimarthaṁ dharmam ity āhuḥ katimūrtiś ca kīrtyate\thinspace{\danda} \dontdisplaylinenum

katipādav\textsubring{r}ṣo jñeyo gatis tasya kati sm\textsubring{r}tāḥ \veg\dontdisplaylinenum
            \paral{\textit{{\normalfont Testimonia for this chapter: \msCa\ ff.\thinspace 197r--198v, 
                                             \msCb\ ff.\thinspace 204v--206r, 
                                             \msCc\ ff.\thinspace 273r--273v (it broke off at 2.21 and resumes at 3.30b; f. 272 is missing),
                                             \msNa\ ff.\thinspace 4v--6r, 
                                             \msNb\ exp.\thinspace 42, 47--48 (sic!),
                                             \msNc\ ff.\thinspace 213r--214v;
                                \mssCaCbCc\ = \msCa + \msCb + \msCc }}}
\varr{
        \ \va āhuḥ\lem  \msCa\msCb\msNa\msNb\msNc; āhu \Ed
        \ \vd sm\textsubring{r}tāḥ\lem  \msCa\msNa\msNb\msNc; sm\textsubring{r}tā \msCb, sm\textsubring{r}taḥ \Ed}

kautūhalaṁ mamotpannaṁ saṁśayaṁ chindhi tattvataḥ\thinspace{\dandab} \dontdisplaylinenum

kasya putro muniśreṣṭha prajās tasya kati sm\textsubring{r}tāḥ \veg\dontdisplaylinenum
\varr{
        \ \va kautūhalaṁ\lem  \msCa\msCb\msNa\msNb\msNc; kautuhala \Ed\oo
                 mamotpannaṁ\lem  \msCa\msCb\msNa\msNb\Ed; samotpannaṁ \msNc
        \ \vb saṁśayaṁ\lem  \msCb\msNa\msNb\msNc\Ed; saśayaṁ \msCa}

anarthayajña uvāca~{\dandab}\dontdisplaylinenum 

dh\textsubring{r}tir ity eṣa dhātur vai paryāyaḥ parikīrtitaḥ\thinspace{\danda} \dontdisplaylinenum

ādhāraṇān mahattvāc ca dharma ity abhidhīyate \veg\dontdisplaylinenum 
            \paral{\textit{\vo\ \kb\ {\normalfont Matsyapurāṇa 145.27: }  dharmeti dhāraṇe dhātur mahatve caiva ucyate{\thinspace\danda}
                                                  ādhāraṇe mahattve vā dharmaḥ sa tu nirucyate{\thinspace\danda}}}
\varr{
        \ \vc ādhāraṇān ma°\lem  \msCa\msNb; ādhāraṇāt pa° \msCb, ādhāraṇāt ma° \msNa\msNc, ādhāreṇa ma° \Ed
        \ \vd °bhidhīyate\lem  \msCa\msNa\msNc\Ed; °vidhīyate \msCb\msNb}

śrutism\textsubring{r}tidvayor mūrtiś catuṣpādav\textsubring{r}ṣaḥ sthitaḥ\thinspace{\dandab} \dontdisplaylinenum

caturāśrama yo dharmaḥ kīrtitāni manīṣibhiḥ \veg\dontdisplaylinenum
\varr{
        \ \vab °sm\textsubring{r}tidvayor mūrtiś ca°\lem  \msCa; °sm\textsubring{r}tidvayo mūrttiś ca° \msCb\msNb, °sm\textsubring{r}tidvayo mūrtti ca° \msNa\msNc, 
                                                                        °sm\textsubring{r}tir dvayo mūrtiś ca \Ed
        \ \vb °v\textsubring{r}ṣaḥ\lem  \msCa\msCb\msNa\msNb\Ed; °v\textsubring{r}ṣa \msNc
        \ \vc caturā°\lem  \msCb\msNa\msNb\Ed; cāturā° \msCa\msNc}

gatiś ca pañca vijñeyāḥ ś\textsubring{r}ṇu dharmasya bho dvija\thinspace{\dandab} \dontdisplaylinenum
            \paral{\textit{\vab {\normalfont \msCb\ reads here } gatiś ca pautrāś ca anekāś ca babhūva ha,
                        {\normalfont skipping to 3.7cd, omitting 3.5--7ab.}}}

devamānuṣatiryaṁ ca narakasthāvarādayaḥ \veg\dontdisplaylinenum
\varr{
        \ \va vijñeyāḥ\lem  \eme; vijñeyaḥ \msCa\msNa\msNb\msNc\Ed, \om\ \msCb}

brahmaṇo h\textsubring{r}dayaṁ bhittvā jāto dharmaḥ sanātanaḥ\thinspace{\dandab} \dontdisplaylinenum
            \paral{\textit{\vab {\normalfont cf.\ Devīpurāṇa 4.59cd: } brahmaṇo h\textsubring{r}dayāj jātaḥ putro dharma iti sm\textsubring{r}taḥ \oo 
                    {\normalfont cf. also MBh 1.60.40ab: } brahmaṇo h\textsubring{r}dayaṁ bhittvā niḥs\textsubring{r}to bhagavān bh\textsubring{r}guḥ}}

tasya patnī mahābhāgā trayodaśa sumadhyamāḥ \veg\dontdisplaylinenum
\varr{
        \ \va brahmaṇo\lem  \msCa\msNa\msNb\msNc; \om\ \msCb, brāhmaṇo \Ed\oo
                 bhittvā\lem  \msCa\msCb\msNa\msNc\Ed; vittvā \msNb
        \ \vb dharmaḥ\lem  \msCa\msCb\msNa\msNc\Ed; dharma \msNb
        \ \vd °madhyamāḥ\lem  \msCa\msNa\msNb\msNc\Ed; \om\ \msCb}

dakṣakanyā viśālākṣī śraddhādyāḥ sumanoharāḥ\thinspace{\dandab} \dontdisplaylinenum

tasya putrāś ca pautrāś ca anekāś ca babhūva ha \danda\dontdisplaylinenum

eṣa dharmanisargo 'yaṁ kiṁ bhūyaḥ śrotum icchasi \veg\dontdisplaylinenum
\varr{
        \ \va °ākṣī\lem  \msCa\msNa\msNb\msNc; \om\ \msCb, °ākṣi \Ed
        \ \vb °ādyāḥ\lem  \eme; °ādyā \msNb\msNc\Ed, °āḍhyāḥ \msNa, \om\ \msCb, °āḍhyā \msCa\oo
                 °harāḥ\lem  \msNb\Ed; °harā \msCa\msNc,  \om\ \msCb, °{\il}\uncl{mā}ḥ \msNa
        \ \vcd tasya putrāś ca pautrāś ca anekāś ca babhūva ha\lem  \msCa\msNb;
                gatiś ca pautrāś ca anekāś ca babhūva ha \eyeskip{from 3.5a} \msCb,
                tasya putrāś ca yotrāś ca anekāś ca babhūva ha \msNa\msNc,
                tasya putrā anekāś ca tathā pautrā babhūvahaḥ \Ed}

vigatarāga uvāca~{\dandab}\dontdisplaylinenum 
\varr{
        \ \vo vigatarāga uvāca\lem  \msCb\msNapcorr\msNc\Ed; vigatarāga u \msCa\msNb, \om\ \msNaacorr}

dharmapatnī viśeṣeṇa putras tebhyaḥ p\textsubring{r}thak p\textsubring{r}thak\thinspace{\danda} \dontdisplaylinenum

śrotum icchāmi tattvena kathayasva tapodhana \veg\dontdisplaylinenum

anarthayajña uvāca~{\dandab}\dontdisplaylinenum 

śraddhā lakṣmīr dh\textsubring{r}tis tuṣṭiḥ puṣṭir medhā kriyā lajjā\thinspace{\danda} \dontdisplaylinenum

buddhiḥ śāntir vapuḥ kīrtiḥ siddhiḥ prasūtisambhavāḥ \veg\dontdisplaylinenum
\varr{
        \ \va lakṣmīr dh\textsubring{r}tis tuṣṭiḥ\lem  \msCa; 
                                lakṣmīr dh\textsubring{r}tis tuṣ \msCb,  
                                lakṣmī ddh\textsubring{r}tir ddh\textsubring{r}tis tuṣṭiḥ \msNaacorr, 
                                lakṣmīr ddh\textsubring{r}tis tuṣṭiḥ \msNapcorr, 
                                lakṣmīṁ dh\textsubring{r}ti tuṣṭiḥ \msNb,
                                lakṣmī dh\textsubring{r}tis tuṣṭiḥ \msNc,
                                lakṣmī dh\textsubring{r}tis tuṣṭī \Ed
        \ \vb puṣṭir me°\lem  \msCa\msCb\msNa\msNb\msNc; puṣṭi me° \Ed\oo
                 lajjā\lem  \msCa\msCb\msNb\msNc\Ed; lajā \msNa
        \ \vc buddhiḥ\lem  \msCb\msNa\msNb\msNc\Ed; buddhi \msCa
        \ \vd siddhiḥ prasūtisambhavāḥ\lem  \conj; siddhiś cābhūtisambhavā \msCa\msNa\msNb\msNc, 
                                        siddhiś cātisambhavā \msCb, siddhiś ca bhūtisambhavā \Ed}

śraddhā kāmaḥ suto jāto darpo lakṣmīsutaḥ sm\textsubring{r}taḥ\thinspace{\dandab} \dontdisplaylinenum

dh\textsubring{r}tyās tu niyamaḥ putraḥ saṁtoṣas tuṣṭijaḥ sm\textsubring{r}taḥ \veg\dontdisplaylinenum
            \paral{\textit{\vo {\normalfont For 3.10--13, see a rather similar 
        passage e.g.\ in Kūrmapurāṇa 1.8.20 ff.: }
        śraddhāyā ātmajaḥ kāmo darpo lakṣmīsutaḥ sm\textsubring{r}taḥ{\thinspace\danda}
        dh\textsubring{r}tyās tu niyamaḥ putras tuṣṭyāḥ saṁtoṣa ucyate{\thinspace\ketdanda} 
        puṣṭyā lābhaḥ sutaś cāpi medhāputraḥ śrutas tathā{\thinspace\danda} 
        kriyāyāś cābhavat putro daṇḍaḥ samaya eva ca{\thinspace\ketdanda}  
        buddhyā bodhaḥ sutas tadvad apramādo vyajāyata{\thinspace\danda} 
        lajjāyā vinayaḥ putro vapuṣo vyavasāyakaḥ{\thinspace\ketdanda}  
        kṣemaḥ śāntisutaś cāpi sukhaṁ siddhir ajāyata{\thinspace\danda}
        yaśaḥ kīrtisutas tadvad ity ete dharmasūnavaḥ{\thinspace\ketdanda}   
        kāmasya harṣaḥ putro 'bhūd devānando vyajāyata{\thinspace\danda}  
        ity eṣa vai sukhodarkaḥ sargo dharmasya kīrtitaḥ{\thinspace\ketdanda}}}
\varr{
        \ \va kāmaḥ\lem  \msNa; kāma° \msCa\msCb\msNb\msNc, dharma° \Ed}

puṣṭyā lābhaḥ suto jāto medhāputraḥ śrutas tathā\thinspace{\dandab} \dontdisplaylinenum

kriyāyās tv abhayaḥ putro daṇḍaḥ samaya eva ca \veg\dontdisplaylinenum
\varr{
        \ \va lābhaḥ\lem  \msCa\msCb\msNb\msNc; lābha° \msNa\Ed
        \ \vb °putraḥ\lem  \eme; °putra \msCa\msCb\msNa\msNb\msNc\Ed\oo
                 śruta°\lem  \msCa\msNa\msNb\msNc\Ed; śrata° \msCb
        \ \vc tv abhayaḥ putro\lem  \msCa\msCb\msNa\msNb\msNc; tūbhayaḥ putrau \Ed
        \ \vd daṇḍaḥ\lem  \corr; daṇḍe \msCa\msNaacorr, daṇḍo \msCb, daṇḍa° \msNapcorr\msNb\msNc\Ed\oo
                 ca\lem  \msCa\msCb\msNa\msNb\msNc; tu \Ed}

lajjāyā vinayaḥ putro buddhyā bodhaḥ sutaḥ sm\textsubring{r}taḥ\thinspace{\dandab} \dontdisplaylinenum

lajjāyāḥ sudhiyaḥ putra apramādaś ca tāv ubhau \veg\dontdisplaylinenum
\varr{
        \ \va lajjāyā vinayaḥ\lem  \msCa\msCb\msNa\msNb\msNc; lajjāyāḥ vinaya° \Ed
        \ \vb sutaḥ sm\textsubring{r}taḥ\lem  \msNa\msNb\msNc\Ed; sutaḥ {\il}{\il} \msCa, sutaḥs tathā \msCb
        \ \vc sudhiyaḥ\lem  \Ed; sudhiya \msCa\msCb\msNa\msNb\msNc\oo
                 putra\lem  \msCa\msCb\msNa\msNb\msNc; putraḥ \Ed
        \ \vd apramāda°\lem  \msCa\msCb\msNb\msNc\Ed; apramādā° \msNa}

kṣemaḥ śāntisuto vindyād vyavasāyo vapoḥ sutaḥ\thinspace{\dandab} \dontdisplaylinenum

yaśaḥ kīrtisuto jñeyaḥ sukhaṁ siddher vyajāyata \danda\dontdisplaylinenum

svāyambhuve 'ntare tv āsan kīrtitā dharmasūnavaḥ \veg\dontdisplaylinenum
\varr{
        \ \vb vapoḥ\lem  \msCa\msCb\msNb\msNc\Ed; vapo \msNa
        \ \vd siddhe°\lem  \msCb\msNa\msNb; siddhi \msCa\msNc\Ed\oo
                 vyajāyata\lem  \msCa\msCb\msNa; vyajāyate \msNb\Ed, vyajāyataḥ \msNc
        \ \ve svāyambhuve\lem  \msCa\msNa\msNc; svāyambhuvo \msCb, svayambhuve \msNb\Ed\oo
                 'ntare tv āsan\lem  \conj; 'ntare tvāsi \msCa\msCb\msNa, 
                                'ntare tv āsīt \msNb, 'ntare tv āsaṁ \msNc, 'ntar evāsi \Ed}

vigatarāga uvāca~{\dandab}\dontdisplaylinenum 

mūrtidvayaṁ kathaṁ dharmaṁ kathayasva tapodhana\thinspace{\danda} \dontdisplaylinenum

kautūhalam atīvaṁ me kartaya jñānasaṁśayam \veg\dontdisplaylinenum
\varr{
        \ \va dharmaṁ\lem  \msCa\msCb\msNa\msNb; ddharma \msNc, dharmaḥ \Ed
        \ \vc kautūhala°\lem  \msCa\msNa\msNb\msNc\Ed; kotūhala° \msCb\oo
                 °tīvaṁ me\lem  \msCa\msNa\msNb\msNc\Ed; °tīva me \msCb
        \ \vd kartaya\lem  \eme; kīrtaya \msCa\msCb\msNa\msNb\msNc\Ed\oo
                 °saṁśayam\lem  \msCa\msNa\msNc\Ed; °saṁśayaḥ \msCb\msNb}

anarthayajña uvāca~{\dandab}\dontdisplaylinenum 

śrutism\textsubring{r}tidvayor mūrtir dharmasya parikīrtitā\thinspace{\danda} \dontdisplaylinenum

dārāgnihotrasambandham ijyā śrautasya lakṣaṇam \danda\dontdisplaylinenum
            \paral{\textit{\vcd {\normalfont cf.\ Manu 3.171ab: }dārāgnihotrasaṁyogaṁ kurute yo 'graje sthite; 
               {\normalfont and also Matsyapurāṇa 142.41: } 
                        dārāgnihotrasambandham \textsubring{r}gyajuḥsāmasaṁhitāḥ{\thinspace\danda}
                        ityādibahulaṁ śrautaṁ dharmaṁ saptarṣayo 'bruvan{\thinspace\ketdanda}}}

smārto varṇāśramācāro yamaiś ca niyamair yutaḥ \veg\dontdisplaylinenum
            \paral{\textit{\vcdef\ \kb\ {\normalfont Matsyapurāṇa 145.31: }dārāgnihotrasambandham ijyā śrautasya lakṣaṇam{\thinspace\danda}
                                smārto varṇāśramācāro yamaiś ca niyamair yutaḥ{\thinspace\ketdanda}
        {\normalfont  \kb\ \MBh\ Indeces 1.36.10: }dānāgnihotram ijyā ca śrautasyaitad dhi lakṣaṇam{\thinspace\danda}
                                       smārto varṇāśramācāro yamaiś ca niyamair yutaḥ{\thinspace\ketdanda} }}


\alalfejezet{yamaniyamabhedaḥ}\varr{
        \ \va śruti°\lem  \msCa\msNa\msNb\msNc; śrutiḥ \msCb\Ed
        \ \vab °dvayor mūrtir dha°\lem  \msCa; °dvayo mūrti dha° \msCb\msNa\msNb, °dvayī mūrti dha° \msNc, 
                                                                                        °dvayor mūrti dha° \Ed
        \ \vb °kīrtitā\lem  \msCa\msCb\msNa\Ed; °kīrttitaḥ \msNb, kīrttitāḥ \msNc
        \ \vcd °bandham i°\lem  \eme; °baddha i° \msCa\msCb\msNa\msNc, °bandha i° \msNb\Ed
        \ \vd śrautasya\lem  \eme; śrotasya \msCa\msCb\msNc, śrautrasya \msNa, srotrasya \msNb, śrutasya \Ed
        \ \ve smārto\lem  \eme; smārta \msCa\msCb\msNa\msNb\msNc\Ed}

yamaś ca niyamaś caiva dvayor bhedam ataḥ ś\textsubring{r}ṇu\thinspace{\dandab} \dontdisplaylinenum

ahiṁsā satyam asteyam ān\textsubring{r}śaṁsyaṁ damo gh\textsubring{r}ṇā \veg\dontdisplaylinenum
\varr{
        \ \va niyama°\lem  \msCa\msCb\msNb\msNc\Ed; niyamai° \msNa
        \ \vd °m ān\textsubring{r}śaṁsyaṁ\lem  \eme; °m an\textsubring{r}śaṁsyo \msCa\msCb\msNa\msNb\Ed, °m ān\textsubring{r}śaṁsyā \msNc}

dhanyāpramādo mādhuryam ārjavaṁ ca yamā daśa\thinspace{\dandab} \dontdisplaylinenum

ekaikasya punaḥ pañcabhedam āhur manīṣiṇaḥ \veg\dontdisplaylinenum


\alalfejezet{yameṣv ahiṁsā (1)}\varr{
        \ \va dhanyā°\lem  \Ed; dhanyaḥ \msCa\msCb\msNb\msNc, dhyanyaṁ \msNa\oo
                 mādhurya°\lem  \Ed; mādhūrya° \msCa\msCb\msNa\msNb\msNc
        \ \vb ārjavaṁ ca\lem  \msCa\msCb\msNa\msNb\msNc; ārjavaś ca \Ed
        \ \vd °m āhur ma°\lem  \msCa\msCb\msNa\msNb\Ed; °m āhu ma° \msNc}

ahiṁsādi pravakṣyāmi ś\textsubring{r}ṇuṣvāvahito dvija\thinspace{\dandab} \dontdisplaylinenum

trāsanaṁ tāḍanaṁ bandho māraṇaṁ v\textsubring{r}ttināśanam \danda\dontdisplaylinenum

hiṁsāṁ pañcavidhām āhur munayas tattvadarśinaḥ \veg\dontdisplaylinenum
\varr{
        \ \vb ś\textsubring{r}ṇuṣvā°\lem  \msCa\msCb\msNc\Ed; ś\textsubring{r}ṇuṣva° \msNa\msNb
        \ \vc bandho\lem  \msCa\msCb\msNa\msNc; baddho \msNb, bandha \Ed
        \ \ve hiṁsāṁ\lem  \msCa\msNa\msNc; hiṁsā \msCb\msNb\Ed\oo
                 °vidhām āhu°\lem  \msCb\msNa\msNc; °vidham āhu° \msCa,
                                        °vidhāny āhu° \msNb, °vidha prāhu° \Ed}

kāṣṭhaloṣṭakaśādyais tu tāḍayantīha nirdayāḥ\thinspace{\dandab} \dontdisplaylinenum

tatprahāravibhinnāṅgo m\textsubring{r}tavadhyam avāpnuyāt \veg\dontdisplaylinenum
\varr{
        \ \va kāṣṭhaloṣṭa°\lem  \msCa\msCb\msNa\msNc\Ed; kā\uncl{ṣṭha}{\lost}{\lost} \msNb
        \ \vb nirdayāḥ\lem  \msCa\msCb\msNa\msNb\msNc; nirdayā \Ed
        \ \vc °bhinnāṅgo\lem  \msCa\msCb\msNa\msNb\msNc; °bhinnāṅgā \Ed
        \ \vd °vadhyam avā°\lem  \msCb\msNa\msNb\msNc\Ed; °vadhyavavā° \msCa}

baddhvā pādau bhujoraś ca śirorukkaṇṭhapāśitāḥ\thinspace{\dandab} \dontdisplaylinenum

anāhatā mriyanty evaṁ vadho bandhanajaḥ sm\textsubring{r}taḥ \veg\dontdisplaylinenum
\varr{
        \ \va bhujoraś ca\lem  \msCa\msCb\msNb\msNc; bhujauraś ca \msNa\Ed
        \ \vb śirorukkaṇṭha°\lem  \eme; śirorukaṇṭha° \msCa\msCb\msNa\msNb\msNc, śiroruḥ kaṇṭha° \Ed
        \ \vc anāhatā mriyanty evaṁ\lem  \msCa\msCb\msNa\msNc\Ed; anāhata mriyaṁty eṣa \msNb
        \ \vd vadho bandhanajaḥ sm\textsubring{r}taḥ\lem  \conj; °najāḥ sm\textsubring{r}tāḥ \msCa\msCb\msNa\msNb,
                                                                 °najāḥ sm\textsubring{r}tā \msNc, °naja sm\textsubring{r}taḥ \Ed}

śatrucaurabhayair ghoraiḥ siṁhavyāghragajoragaiḥ\thinspace{\dandab} \dontdisplaylinenum

trāsanād vadham āpnoti anyair vāpi suduḥsahaiḥ \veg\dontdisplaylinenum
\varr{
        \ \va °caurabhayair ghoraiḥ\lem  \msCa\msCb\msNa\msNc\Ed; °corabhayai ghorai \msNb
        \ \vd anyair vāpi\lem  \msCa\msCb\msNa\msNb\Ed; anye cāpi \msNc}

yasya yasya hared vittaṁ tasya tasya vadhaḥ sm\textsubring{r}taḥ\thinspace{\dandab} \dontdisplaylinenum

v\textsubring{r}ttijīvābhibhūtānāṁ tadvārān nihataḥ sm\textsubring{r}taḥ \veg\dontdisplaylinenum
\varr{
        \ \va hared vi°\lem  \msCa\msCb\msNa\msNc\Ed; hare vi° \msNb
        \ \vb vadhaḥ\lem  \msCa\msCb\msNa\msNb\msNc; vadha \Ed
        \ \va °bhibhūtānāṁ\lem  \msCa\msCb\msNa\msNc\Ed; °vibhūtānāṁ \msNb
        \ \vb tadvārān ni°\lem  \msCa\msCb\msNa\msNb\msNc; taddvārān ni° \Ed}

viṣavahniśaraśastrair māyāyogabalena vā\thinspace{\dandab} \dontdisplaylinenum

hiṁsakāny āhu viprendra munayas tattvadarśinaḥ \veg\dontdisplaylinenum
\varr{
        \ \vab °śastrair māyā°\lem  \msCa\msCb\msNa\msNb; °śastrai mā° \msNc, °śastrair mmayā° \Ed
        \ \vc hiṁsakāny āhu vi°\lem  \msCb\msNb\msNc; 
                                hiṁsakāny āhur vi° \msCa\msNa\ \unmetr, hiṁsakety āhu vi° \Ed}

ahiṁsā paramaṁ dharmaṁ yas tyajet sa durātmavān\thinspace{\dandab} \dontdisplaylinenum

kleśāyāsavinirmuktaṁ sarvadharmaphalapradam \veg\dontdisplaylinenum
\varr{
        \ \vc paramaṁ dharmaṁ\lem  \msCa\msCb\msNa\Ed; paramaṁ dharma \msNb, paramo dharmaṁ \msNc
        \ \vd tyajet sa durātmavān\lem  \msCb\msNc\Ed; tyajec cha durātma{\il} \msCa, tyajet sudurātmavān \msNa,
                                tyajet sa durātmanam \msNb}

nātaḥ parataro mūrkho nātaḥ parataraṁ tamaḥ\thinspace{\dandab} \dontdisplaylinenum

nātaḥ parataraṁ duḥkhaṁ nātaḥ parataro 'yaśaḥ \veg\dontdisplaylinenum
\varr{
        \ \vb °taraṁ\lem  \msCa\msCbpcorr\msNa\msNb\msNc; °tan \msCbacorr\Ed}

nātaḥ parataraṁ pāpaṁ nātaḥ parataraṁ viṣam\thinspace{\dandab} \dontdisplaylinenum

nātaḥ paratarāvidyā nātaḥ paraṁ tapodhana \veg\dontdisplaylinenum
\varr{
        \ \vd paraṁ tapodhana\lem  \msCa\msCb\msNa\msNb\msNc; para tapodyamāḥ \Ed}

yo hinasti na bhūtāni udbhijjādi caturvidham\thinspace{\dandab} \dontdisplaylinenum

sa bhavet puruṣaḥ śreṣṭhaḥ sarvabhūtadayānvitaḥ \veg\dontdisplaylinenum
\varr{
        \ \va yo hinasti na\lem  \msCa\msCb\msNa\msNc; yo na hinsanti \msNb, yo hi nāsti na \Ed
        \ \vb udbhijjādi\lem  \eme; udbhijādi \msCa\msCb\msNb\msNc\Ed, udbhijāni \msNa\oo
                 °vidham\lem  \msCa\msCb\msNa\msNb\Ed; °vidhiṁ \msNc
        \ \vc puruṣaḥ\lem  \msCa\msCb\msNa\msNb\msNc; puruṣa° \Ed}

sarvabhūtadayāṁ nityaṁ yaḥ karoti sa paṇḍitaḥ\thinspace{\dandab} \dontdisplaylinenum

sa yajvā sa tapasvī ca sa dātā sa d\textsubring{r}ḍhavrataḥ \veg\dontdisplaylinenum
\varr{
        \ \va °dayāṁ nityaṁ\lem  \msCa\msNa\Ed; °dayā nityaṁ \msCb\msNb, °dayā nitya \msNc
        \ \vc yajvā\lem  \msCa\msCb\msNa\msNc\Ed; yajyā \msNb}

ahiṁsā paramaṁ tīrtham ahiṁsā paramaṁ tapaḥ\thinspace{\dandab} \dontdisplaylinenum

ahiṁsā paramaṁ dānam ahiṁsā paramaṁ sukham \veg\dontdisplaylinenum
            \paral{\textit{\vo {\normalfont  This and the following verses are similar to MBh 13.117.37--38\oo
        \msCc\ resumes here in 189.jpg, f. 273r (sic!) with } ramaṁ sukham {\normalfont  (3.30b) }}} 
\varr{
        \ \va paramaṁ tī°\lem  \msCa\msNa\msNb\msNc\Ed; paran tī° \msCb}

ahiṁsā paramo yajñaḥ ahiṁsā paramaṁ vratam\thinspace{\dandab} \dontdisplaylinenum

ahiṁsā paramaṁ jñānam ahiṁsā paramā kriyā \veg\dontdisplaylinenum
\varr{
        \ \va yajñaḥ\lem  \msCb\msCc\msNb\Ed; yajñar \msCa, yajña \msNa\msNc
        \ \vc paramaṁ\lem  \mssCaCbCc\msNa\msNb\msNc; paramo \Ed
        \ \vd paramā\lem  \mssCaCbCc\msNa\msNc\Ed; paramāṁ \msNb}

ahiṁsā paramaṁ śaucam ahiṁsā paramo damaḥ\thinspace{\dandab} \dontdisplaylinenum

ahiṁsā paramo lābhaḥ ahiṁsā paramaṁ yaśaḥ \veg\dontdisplaylinenum
            \paral{\textit{{\normalfont After pādas cd, \Ed\ inserts this: }
        ahiṁsā paramā kīrti ahiṁsā paramo damaḥ, 
        {\normalfont which is not to be found in \mssCaCbCc\msNa\msNb\msNc}}} 
\varr{
        \ \vab (ahiṁsā{\normalfont ...} damaḥ)\lem  \mssCaCbCc\msNa\msNb\msNc; \om\ \Ed
        \ \vc lābhaḥ\lem  \msNc; lābha \msCa\msCb\msNa\msNb\Ed, lābho \msCc
        \ \vd paramaṁ\lem  \mssCaCbCc\msNb\msNc\Ed; paramā \msNa}

ahiṁsā paramo dharmaḥ ahiṁsā paramā gatiḥ\thinspace{\dandab} \dontdisplaylinenum

ahiṁsā paramaṁ brahma ahiṁsā paramaḥ śivaḥ \veg\dontdisplaylinenum
\varr{
        \ \va dharmaḥ\lem  \msNa\msNc; dharma \msCa\msCb\Ed, dharmo \msCc, dha{\lost} \msNb
        \ \vb ahiṁsā paramā gatiḥ\lem  \mssCaCbCc\msNa\msNc; {\lost}{\lost}{\lost}{\lost}{\lost}{\lost}{\lost}{\lost} \msNb, ahiṁsā paramo gatiḥ \Ed
        \ \ve ahiṁsā paramaṁ brahma\lem  \mssCaCbCc\msNa\Ed;
                                \uncl{ahiṁsā paramaṁ brahma} \msNb, ahiṁsā paraṁmaṁ brahma \msNc}

māṁsāśanān nivarteta manasāpi na kāṅkṣayet\thinspace{\dandab} \dontdisplaylinenum

sa mahat phalam āpnoti yas tu māṁsaṁ vivarjayet \veg\dontdisplaylinenum
\varr{
        \ \va māṁsāśanān ni°\lem  \msCa\msCb\Ed; mānsāśana ni° \msCc,
                                                        māṁsāśanan ni° \msNa, mansāsanan ni° \msNb,
                                                        \uncl{māṁsaśānān ni°} \msNc
        \ \vd māṁsaṁ\lem  \mssCaCbCc\msNa; māṁsa \msNb\Ed, māsaṁ \msNc}

svamāṁsaṁ paramāṁsena yo vardhayitum icchati\thinspace{\dandab} \dontdisplaylinenum
            \paral{\textit{\vab {\normalfont  = MBh 13.116.14ab and 13.116.34ab \kb\ Uttarottara 2.XXxx: }
                                svamāṁsaṁ paramāṁsena yo dehe v\textsubring{r}ddhim icchati}}

anabhyarcya pit\textsubring{\=r}n devān na tato 'nyo 'sti pāpak\textsubring{r}t \veg\dontdisplaylinenum
            \paral{\textit{\vo {\normalfont \kb\ Manu 5.52}}}
\varr{
        \ \va °māṁsena\lem  \mssCaCbCc\msNa\msNb\Ed; °māsena \msNc
        \ \vb vardhayitu°\lem  \mssCaCbCc\msNa\msNc\Ed; varddhayati \msNb
        \ \vc pit\textsubring{\=r}n\lem  \msCa\msCb\msNa\msNc; pit\textsubring{r}n \msCc\Ed, \uncl{pit\textsubring{\=r}n} \msNb
        \ \vd tato 'nyo\lem  \mssCaCbCc\msNa\msNb\msNc; tad anyo \Ed}

madhuparke ca yajñe ca pit\textsubring{r}daivatakarmaṇi\thinspace{\dandab} \dontdisplaylinenum

atraiva paśavo hiṁsyā nānyatra manur abravīt \veg\dontdisplaylinenum
            \paral{\textit{\vo {\normalfont \kb\ Manu 5.41}}}
\varr{
        \ \vb °daivata°\lem  \msCa\msCb\msNa\msNc\Ed; °devata° \msCc\msNb
        \ \vc atraiva paśavo hiṁsyā\lem  \msCa\msCc\msNc\Ed; 
                                        atraiva paśavo hiṁsā \msCb, atraiva paśavo hiṁsyān \msNa,
                                                        {\lost}{\lost}{\lost}{\lost}{\lost}{\lost}{\lost}{\lost} \msNb
        \ \vd nānyatra manur abravīt\lem  \mssCaCbCc\msNa\msNc\Ed; {\lost}{\lost}\uncl{tra manur abravīt} \msNb}

krītvā svayaṁ vāpy utpādya paropah\textsubring{r}tam eva vā\thinspace{\dandab} \dontdisplaylinenum

devān pit\textsubring{\=r}ṁś cārcayitvā khādan māṁsaṁ na doṣabhāk \veg\dontdisplaylinenum
            \paral{\textit{\vo {\normalfont = Manu 5.32 (in Olivelle's critical edition; other editions read } paropak\textsubring{r}ta°{\normalfont ) }}}
\varr{
        \ \va krītvā\lem  \mssCaCbCc\msNa\msNb\msNc; k\textsubring{r}tvā \Ed\oo
                 °py utpādya\lem  \mssCaCbCc\msNa\msNb\msNc; °py utpādyā° \Ed
        \ \vb °h\textsubring{r}ta°\lem  \mssCaCbCc\msNa\msNb\msNc; °hita° \Ed\oo
                 vā\lem  \mssCaCbCc\msNa\msNb\msNc; ca \Ed
        \ \vc pit\textsubring{\=r}ṁś cārcayitvā\lem  \mssCaCbCc\msNa\msNc; pit\textsubring{\=r}ś cārcayitvā \msNb, pit\textsubring{r}ś cārpayitvā \Ed
        \ \vd māṁsaṁ\lem  \mssCaCbCc\msNa\msNb\Ed; māsaṁ \msNc}

vedayajñatapastīrthadānaśīlakriyāvrataiḥ\thinspace{\dandab} \dontdisplaylinenum

māṁsāhāraniv\textsubring{r}ttānāṁ ṣoḍaśāṁśaṁ na pūryate \veg\dontdisplaylinenum
\varr{
        \ \vb °śīla°\lem  \msCa\msCb\msNa\msNb\msNc\Ed; °śala° \msCc\oo
                 °vrataiḥ\lem  \msCa\msCc\msNa\msNb\msNc\Ed; °vra\uncl{taḥ} \msCb
        \ \vc °v\textsubring{r}ttānāṁ\lem  \mssCaCbCc\msNa\msNc; °v\textsubring{r}ttānā \msNb, °v\textsubring{r}ttīnāṁ \Ed
        \ \vd na\lem  \msCa\msCc\msNa\msNb\msNc\Ed; ta \msCb }

m\textsubring{r}gāḥ parṇat\textsubring{r}ṇāhārād ajameṣagavādibhiḥ\thinspace{\dandab} \dontdisplaylinenum

sukhino balavantaś ca vicaranti mahītale \veg\dontdisplaylinenum
\varr{
        \ \va  parṇa°\lem  \mssCaCbCc\msNb\msNc; paṇṇa° \msNa, parṇā° \Ed
        \ \vab °hārād a°\lem  \msCa\msCc\msNbpcorr\msNc\Ed; °hād a° \msNbacorr, °hārā a° \msCb\msNa}

vānarāḥ phala-m-āhārā rākṣasā rudhirapriyāḥ\thinspace{\dandab} \dontdisplaylinenum

nihatā rākṣasāḥ sarve vānaraiḥ phalabhojibhiḥ \veg\dontdisplaylinenum
\varr{
        \ \vab °hārā rā°\lem  \msCb\msNa\msNb; °hārād rā° \msCa\msCc\msNc\Ed
        \ \vd °bhojibhiḥ\lem  \mssCaCbCc\msNa\msNb\msNc; °bhogibhiḥ \Ed}

tasmān māṁsaṁ na hīheta balakāmena bho dvija\thinspace{\dandab} \dontdisplaylinenum

balena ca guṇākarṣāt parato bhayabhīruṇā \veg\dontdisplaylinenum
\varr{
        \ \va māṁsaṁ\lem  \mssCaCbCc\msNa\msNb\Ed; māsaṁ \msNc
        \ \vb hīheta\lem  \mssCaCbCc\msNc\Ed; hīyeta \msNa\msNb
        \ \vc guṇākarṣā°\lem  \conjTorzsok; guṇākāśā° \mssCaCbCc\msNa\msNb\msNc, guṇā kuryā° \Ed}

ahiṁsakasamo nāsti dānayajñasamīhayā\thinspace{\dandab} \dontdisplaylinenum

iha loke yaśaḥ kīrtiḥ paratra ca parā gatiḥ \veg\dontdisplaylinenum
\varr{
        \ \vb °yajñasamīhayā\lem  \msCa\msCb\msNa\msNb; °dharmasamīhayā \msCc,
                                                °yajñasamīhayāḥ \msNc, °dharmasamīhaya \Ed
        \ \vc yaśaḥ\lem  \msCa\msCb\msNa\msNb\msNc\Ed; ya\uncl{śaṁ} \msCc
        \ \vd parā gatiḥ\lem  \msCc\msNa\msNc; \uncl{parā gatiḥ} \msCa, 
                                                parāṅgatim \msCb\msNb, parāṁ gatiḥ \Ed}

\ujvers\nemsloka 
trailokyaṁ maṇiratnapūrṇam akhilaṁ dattvottame brāhmaṇe
\dontdisplaylinenum

\nemslokab 
koṭīyajñasahasrapadmam ayutaṁ dattvā mahīṁ dakṣiṇām \danda\dontdisplaylinenum

\nemslokac 
tīrthānāṁ ca sahasrakoṭiniyutaṁ snātvā sak\textsubring{r}n mānavaḥ
\dontdisplaylinenum

\nemslokad 
etatpuṇyaphalam ahiṁsakajanaḥ prāpnoti niḥsaṁśayaḥ \veg\dontdisplaylinenum

\vers


\jump
\begin{center}
\ketdanda iti v\textsubring{r}ṣasārasaṁgrahe ahiṁsāpraśaṁsā nāmādhyāyas t\textsubring{r}tīyaḥ\ketdanda
\end{center}
\dontdisplaylinenum\vers 
\varr{
        \ \va trailokyaṁ\lem  \mssCaCbCc\msNa\msNc\Ed; trailokya \msNb\oo
                 akhilaṁ dattvottame brāhmaṇe\lem  \msCb\msCc\msNb\msNc\Ed;
                                        a\uncl{khilaṁ}{\il}{\il}{\il}{\il}{\il}{\il}{\il} \msCa, akhilaṁ dattottame brāhmaṇe \msNa
        \ \vb koṭīyajñasahasrapadmam\lem  \msCb\msCc\msNa\msNb\msNc\Ed; {\il}{\il}{\il}{\il}{\il}{\il}{\il}{\il}{\il} \msCa\oo
                 mahīṁ\lem  \msCa\msCb\msNa\msNb\msNc\Ed; mahī \msCc
        \ \vc °koṭi°\lem  \mssCaCbCc\msNa\msNb\msNc; °koṭī° \Ed\ \unmetr\oo
                 snātvā\lem  \msCa\msCc\msNa\msNb\msNc\Ed; snā ' \msCb
        \ \vd °phalam ahiṁsa°\lem  \mssCaCbCc\msNa\msNb\Ed; °phalaṁ tv ahiṁsa° \msNc\oo
                 niḥsaṁśayaḥ\lem  \msCc\msNa\msNb\msNc; {\il}{\il}{\il}{\il} \msCa, niḥsaṁśaya{\il} \msCb, niḥsaṁśayaṁ \Ed
        \ {\normalfont \Colo:} nāmādhyāyas t\textsubring{r}tīyaḥ\lem  \mssCaCbCc\msNa\msNb; nāmādhyāyas t\textsubring{r}tīya \msNc,
                        nāmas t\textsubring{r}tīyo 'dhyāyaḥ \Ed}
\bekveg\szamveg\vfill\phpspagebreak\szam\bek\versno=0\fejno=4
\thispagestyle{empty}



\alfejezet{\textbf{caturtho 'dhyāyaḥ}}\jump\jump

\alalfejezet{yameṣu satyam (2)}
\vers

anarthayajña uvāca~{\dandab}\dontdisplaylinenum 

sadbhāvaḥ satyam ity āhur d\textsubring{r}ṣṭapratyayam eva vā\thinspace{\danda} \dontdisplaylinenum
            \paral{\textit{\va {\normalfont \kb\ MBh 12.288.45d: } sadbhāvaḥ satyam ucyate {\normalfont  \oo cf.\ also Brahmāṇḍapurāṇa 3.3.86ab: }
                        asadbhāvo 'n\textsubring{r}taṁ jñeyaṁ sadbhāvaḥ satyam ucyate }}

yathābhūtārthakathanaṁ tat satyakathanaṁ sm\textsubring{r}tam \veg\dontdisplaylinenum
            \paral{\textit{\vc {\normalfont cf. Śivadharmaśāstra 11.105: } 
                svānubhūtaṁ svad\textsubring{r}ṣṭaṁ ca yaḥ p\textsubring{r}ṣṭārthaṁ na gūhati{\thinspace\danda}
                yathābhūtārthakathanam ity etat satyalakṣaṇam{\thinspace\ketdanda}}}
\varr{
        \ \va sadbhāvaḥ\lem  \mssCaCbCc\msNa\msNc; sadbhāva° \msNb\Ed
        \ \vab satyam ity āhur d\textsubring{r}°\lem  \msCb\msNa\msNc\Ed; satya\uncl{m i}ty āhu d\textsubring{r}° \msCa,
                                satyam ity āhu d\textsubring{r}° \msCc, satyām ity āhur d\textsubring{r}° \msNb
        \ \vb °pratyaya°\lem  \msCa\msCb\msNa\msNb; °pratya° \msCc, °pratyeya° \msNc, pratyakṣa° \Ed
        \ \vc yathābhūtārthakathanaṁ\lem  \msCa\msCb\msNa\msNb\msNc\Ed; 
                yathābhūtārtha \msCcacorr, yathābhūtārtha{\il}kta kathanaṁ \msCcpcorr
        \ \vd tat satyakathanaṁ\lem  \msCa\msCc\msNa\msNb\msNc\Ed;
                tat satyakathakaṁ \msCb, kathanaṁ sm\textsubring{r}taṁ \msCcacorr, satyakakathanaṁ sm\textsubring{r}taṁ \msCcpcorr}

ākrośatāḍanādīni yaḥ saheta suduḥsaham\thinspace{\dandab} \dontdisplaylinenum

kṣamate yo jitātmā tu sa ca satyam udāh\textsubring{r}tam \veg\dontdisplaylinenum
            \paral{\textit{\vo {\normalfont cf.\ Śivadharmaśāstra 11.82: }
                ākruṣṭas tāḍito vāpi yo nākrośen na tāḍayet {\thinspace\danda}
                vāgādy avik\textsubring{r}taḥ svasthaṁ kṣāntir eṣā sunirmalā {\thinspace\ketdanda}}}
\varr{
        \ \va °tāḍanā°\lem  \msCa\msCc\msNa\msNb\msNc\Ed; °nāḍanā° \msCb
        \ \vb suduḥsaham\lem  \msCa\msCb\msNa\msNb\msNc\Ed; sudusahaṁ \msCc
        \ \vd satyam udāh\textsubring{r}tam\lem  \msCb\msCc\msNa\msNb\msNc\Ed; 
        \uncl{satya}m u\uncl{dā}h\textsubring{r}tam \msCa}

vadhārtham udyataḥ śastraṁ yadi p\textsubring{r}ccheta karhicit\thinspace{\dandab} \dontdisplaylinenum

na tatra satyaṁ vaktavyam an\textsubring{r}taṁ satyam ucyate \veg\dontdisplaylinenum
\varr{
        \ \va °dyataḥ\lem  \mssCaCbCc\msNb\msNc\Ed; °dyata \msNa\oo
                 śastraṁ\lem  \msCa\msCb\msNa\msNb\msNc; śastra \msCc, satya \msCb\Ed
        \ \vb karhicit\lem  \mssCaCbCc\Ed; karhacit \msNa\msNb\msNc
        \ \vc satyaṁ\lem  \msCa\msCc\msNa\msNb\msNc; satya \msCb\Ed}

vadhārhaḥ puruṣaḥ kaścid vrajet pathi bhayāturaḥ\thinspace{\dandab} \dontdisplaylinenum

p\textsubring{r}cchato 'pi na vaktavyaṁ satyaṁ tad vāpi ucyate \veg\dontdisplaylinenum
\varr{
        \ \vb °turaḥ\lem  \msCa\msCc\msNa\msNb\msNc\Ed; °tura \msCb
        \ \vc p\textsubring{r}cchato\lem  \mssCaCbCc\msNa\msNb\msNc; p\textsubring{r}cchate \Ed
        \ \vd tad vāpi\lem  \mssCaCbCc\msNa\msNc\Ed; tad api \msNb}

\ujvers\nemsloka 
na narmayuktam an\textsubring{r}taṁ hinasti
\dontdisplaylinenum

\nemslokab 
na strīṣu rājan na vivāhakāle \danda\dontdisplaylinenum

\nemslokac 
prāṇātyaye sarvadhanāpahāre
\dontdisplaylinenum

\nemslokad 
pañcān\textsubring{r}taṁ satyam udāharanti \veg\dontdisplaylinenum
            \paral{\textit{\vo {\normalfont cf.\ \MBh\ 1.77.16: } na narmayuktaṁ vacanaṁ hinasti na strīṣu rājan na vivāhakāle{\thinspace\danda}
                                          prāṇātyaye sarvadhanāpahāre pañcān\textsubring{r}tāny āhur apātakāni{\thinspace\ketdanda};
        {\normalfont \MBh\ 12.159.28: } na narmayuktaṁ vacanaṁ hinasti na strīṣu rājan na vivāhakāle{\thinspace\danda}
                            na gurvarthe nātmano jīvitārthe pañcān\textsubring{r}tāny āhur apātakāni{\thinspace\ketdanda};
        {\normalfont \MP\ 31.16: } na narmayuktaṁ vacanaṁ hinasti na strīṣu rājan na vivāhakāle{\thinspace\danda}
                       prāṇātyaye sarvadhanāpahāre pañcān\textsubring{r}tāny āhur apātakāni{\thinspace\ketdanda};
        {\normalfont Kauṇḍinya's commentary ad \PS\ 1.9: }
                gobrāhmaṇārthe 'vacanaṁ himasti na strīṣu rājan na vivāhakāle{\thinspace\danda}
                prāṇātyaye sarvadhanāpahāre pañcān\textsubring{r}tāni āhur apātakāni{\thinspace\ketdanda};
        {\normalfont Abhidharmakośabhāṣya 24114--24117 (introduced by } mohajo m\textsubring{r}ṣāvādo yathāha{\normalfont ): }
                na narmayuktam an\textsubring{r}taṁ hi nāsti na strīṣu rājan na vivāhakāle{\thinspace\danda}
                prāṇātyaye sarvadhanāpahāre pañcān\textsubring{r}tāny āhur apātakāni{\thinspace\ketdanda} }}

\vers
\varr{
        \ \va hinasti\lem  \msCa\msCb\msNb\msNc; hi nāsti \msCc\msNa\Ed
        \ \vb rājan na\lem  \msCa\msCb\msNb\msNc\Ed; rāja na \msCc, rājyaṁ na \msNa
        \ \vc °tyaye\lem  \mssCaCbCc\msNa\msNc\Ed; °tyaje \msNb\oo
                 °pahāre\lem  \msCa\msCb\msNa\msNc\Ed; °prahāre \msCc\msNb}

devamānuṣatiryeṣu satyaṁ dharmaḥ paro yataḥ\thinspace{\dandab} \dontdisplaylinenum

satyaṁ śreṣṭhaṁ variṣṭhaṁ ca satyaṁ dharmaḥ sanātanaḥ \veg\dontdisplaylinenum
\varr{
        \ \vb °mānuṣa°\lem  \mssCaCbCc\msNa\msNb\Ed; °mānuṣya° \msNc\oo
                 satyaṁ dharmaḥ paro yataḥ\lem  \msCb\msCc; satyaṁ dharmaḥ payataḥ \msCa,
                        satyaṁ dharma paro yataḥ \msNa\msNc, satyadharma paro yataḥ \msNb, satyadharmaparāyaṇaḥ \Ed
        \ \vc śreṣṭhaṁ\lem  \mssCaCbCc\msNa\msNc; śreṣṭha \msNb\Ed\oo
                 variṣṭhaṁ ca\lem  \msCa\msCbpcorr\msCc\msNa\msNb\msNc\Ed; variṣṭhamvariṣṭhamvañ ca \msCbacorr
        \ \vd satyaṁ\lem  \msCa\msCc\msNa\msNc\Ed; satya° \msCb\msNb\oo
                 dharmaḥ\lem  \msCa\msCb\msNa\msNb\msNc; dharma \msCc\Ed}

satyaṁ sāgaram avyaktaṁ satyam akṣayabhogadam\thinspace{\dandab} \dontdisplaylinenum

satyaṁ potaḥ paratrārthaṁ satyaṁ panthāna vistaram \veg\dontdisplaylinenum
\varr{
        \ \va satyaṁ\lem  \msCa\msCb\msNa\msNb\msNc\Ed; satya \msCc
        \ \vb satyam akṣayabhogadam\lem  \msCa\msNa\msNb\msNc; satyaṁm akṣayabhogadam \msCb\msCc,
                                satyam akṣayate naraṁ \Ed
        \ \vc potaḥ\lem  \mssCaCbCc\msNb\msNc; pota \msNa, proktaḥ \Ed
        \ \vd panthāna vistaram\lem  \mssCaCbCc\msNa\msNb\msNc; yaj jñānavistaram \Ed}

satyam iṣṭagatiḥ proktaṁ satyaṁ yajñam anuttamam\thinspace{\dandab} \dontdisplaylinenum

satyaṁ tīrthaṁ paraṁ tīrthaṁ satyaṁ dānam anantakam \veg\dontdisplaylinenum
\varr{
        \ \va °ṣṭagatiḥ\lem  \mssCaCbCc\msNa\msNc\Ed; °\uncl{ṣṭā}gatiḥ \msNb
        \ \vc tīrthaṁ\lem  \mssCaCbCc\msNa; tīrtha \msNb\msNc, tīrthāt \Ed}

satyaṁ śīlaṁ tapo jñānaṁ satyaṁ śaucaṁ damaḥ śamaḥ\thinspace{\dandab} \dontdisplaylinenum

satyaṁ sopānam ūrdhvasya satyaṁ kīrtir yaśaḥ sukham \veg\dontdisplaylinenum
            \paral{\textit{\vc {\normalfont cf.\ Varāhapurāṇa 193.36cd: } satyaṁ svargasya sopānaṁ pārāvārasya naur iva}}
\varr{
        \ \va satyaṁ\lem  \msCa\msCc\msNa\msNb\msNc\Ed; satya \msCb
        \ \vb śamaḥ\lem  \mssCaCbCc\msNa\msNc\Ed; śamam \msNb
        \ \vc satyaṁ\lem  \msCa\msCc\msNa\msNb\Ed; saṁtyaṁ \msCb, satya \msNc
        \ \vd sukham\lem  \mssCaCbCc\msNa\msNb\msNc; sukhaḥ \Ed}

aśvamedhasahasraṁ ca satyaṁ ca tulayā dh\textsubring{r}tam\thinspace{\dandab} \dontdisplaylinenum

aśvamedhasahasrād dhi satyam eva viśiṣyate \veg\dontdisplaylinenum
            \paral{\textit{\vo {\normalfont  = MBh 1.69.22 = MBh Indeces 13.20.330 = Mārkaṇḍeyapurāṇa 8.42 = Viṣṇudharmottarapurāṇa 3.265.7
        \kb\ MBh 12.156.26 (pāda d reads } satyam evātiricyate{\normalfont ) \kb\ Viṣṇudharma 55.6 
                           (pāda d reads } satyam etad viśiṣyate{\normalfont )} \oo
        {\normalfont cf.\ Śivadharmaśāstra 11.107: }
                        aśvamedhāyutaṁ pūrṇaṁ satyañ ca tulitaṁ purā {\thinspace\danda}
                        aśvamedhāyutāt satyam adhikaṁ bahubhir guṇaiḥ {\thinspace\ketdanda}
        }}
\varr{
        \ \va °sahasraṁ ca\lem  \msCa\msCb\msNa\msNb\msNc\Ed; °sahasrasya \msCc
        \ \vb tulayā\lem  \msCa\msCb\msNa\msNb\msNc\Ed; tulyayā \msCc
        \ \vc °sahasrād dhi\lem  \msCa\msCb\msNa\msNb\msNc\Ed; °sahasrā hi \msCc
        \ \vd eva\lem  \msCa\msCb\msNa\msNb\msNc; evaṁ \msCc\Ed}

satyena tapate sūryaḥ satyena p\textsubring{r}thivī sthitā\thinspace{\dandab} \dontdisplaylinenum

satyena vāyavo vānti satye toyaṁ ca śītalam \veg\dontdisplaylinenum
            \paral{\textit{\vc {\normalfont \kb\ Varāhapurāṇa 193.37: } 
                sūryas tapati satyena vātaḥ satyena vāti ca{\thinspace\danda}  
                agnir dahati satyena satyena p\textsubring{r}thivī sthitā{\thinspace\ketdanda}}}
\varr{
        \ \vab sūryaḥ satyena p\textsubring{r}thivī sthitā\lem  \msNa\msNc; sū\uncl{ryaḥ sa}tyena p\textsubring{r}thi sthitāḥ \msCa,
                sūryaḥ satyaina p\textsubring{r}thivī sthitā \msCb, sūrya  satyena p\textsubring{r}thivī sthitāḥ \msCc,
                sūrya \uncl{satye} {\lost}{\lost}{\lost} vī sthitā \msNb, sūryaḥ satyena p\textsubring{r}thivī sthitāḥ \Ed
        \ \vc vāyavo\lem  \mssCaCbCc\msNa\msNc\Ed; vātyavo \msNb
        \ \vd satye\lem  \mssCaCbCc\msNa\msNb\msNc; satyāt \Ed}

tiṣṭhanti sāgarāḥ satye samayena priyavrataḥ\thinspace{\dandab} \dontdisplaylinenum

satye tiṣṭhati govindo balibandhanakāraṇāt \veg\dontdisplaylinenum
\varr{
        \ \va sāgarāḥ\lem  \msCa\msCb\msNa\msNb\msNc\Ed; sāgarā \msCc
        \ \vb samayena\lem  \mssCaCbCc\msNa\msNb\msNc; satyena ca \Ed}

agnir dahati satyena satyena śaśinācaraḥ\thinspace{\dandab} \dontdisplaylinenum
            \paral{\textit{\vc {\normalfont \kb\ Varāhapurāṇa 193.37cd: } 
                agnir dahati satyena satyena p\textsubring{r}thivī sthitā}}

satyena vindhyās tiṣṭhanti vardhamāno na vardhate \veg\dontdisplaylinenum
\varr{
        \ \vab satyena satyena\lem  \mssCaCbCc\msNapcorr\msNb\Ed; satyena \msNaacorr\msNc
        \ \vb śaśinācaraḥ\lem  \conj; saśi\uncl{bhācaraḥ} \msCa, śaśir ācaraḥ \msNa\msNb\msNc,
                      śa\uncl{si}{\il}caraḥ \msCb, sa śirā varaḥ \msCc, śaśibhāṣkaraḥ \Ed
        \ \vc vindhyās tiṣṭhanti\lem  \msCa\msNa\msNc;
                        vindhyas tiṣṭhanti \msCb\msNb, vindhyā tiṣṭhanti \msCc, tiṣṭhate vindhyo \Ed}

lokālokaḥ sthitaḥ satye meruḥ satye pratiṣṭhitaḥ\thinspace{\dandab} \dontdisplaylinenum

vedās tiṣṭhanti satyeṣu dharmaḥ satye pratiṣṭhati \veg\dontdisplaylinenum
\varr{
        \ \va °lokaḥ\lem  \Ed; °loka \mssCaCbCc\msNa\msNb\msNc\oo
                 sthitaḥ\lem  \mssCaCbCc\msNa\msNb\Ed; sthiḥ \msNc\oo
                 satye\lem  \mssCaCbCc\msNa\msNb\msNc; satyaṁ \Ed
        \ \vb meruḥ\lem  \msCa\msCb\msNa\msNb\msNc; meru \msCc\Ed
        \ \vc vedās ti°\lem  \msCa\msCc\msNa\msNb\msNc; devās ti° \msCb, vedā ti° \Ed
        \ \vd satye\lem  \msCa\msCb\msNa\msNb\msNc\Ed; dharme \msCc\oo
                 pratiṣṭhati\lem  \mssCaCbCc\msNa\msNb\Ed; pratiṣṭhiti \msNcacorr, pratiṣṭhitaḥ \msNcpcorr}

satyaṁ gauḥ kṣarate kṣīraṁ satyaṁ kṣīre gh\textsubring{r}taṁ sthitam\thinspace{\dandab} \dontdisplaylinenum

satye jīvaḥ sthito dehe satyaṁ jīvaḥ sanātanaḥ \veg\dontdisplaylinenum
\varr{
        \ \va gauḥ\lem  \msCa\msCb\msNa\msNc\Ed; gau \msCc\msNb
        \ \vab kṣīraṁ satyaṁ\lem  \msCa\msCc\msNa\msNb\msNc\Ed; kṣītyaṁ \msCbacorr, ksī{\il} nityaṁ \msCbpcorr
        \ \vb kṣīre gh\textsubring{r}taṁ sthitam\lem  \msCa\msCb\msNa\msNc; kṣīraṁ gh\textsubring{r}taṁ sthitam \msCc, kṣīre gh\textsubring{r}ta sthitam \msNb,
                                                                kṣīraṁ sthitaṁ gh\textsubring{r}tam \Ed
        \ \vc satye jīvaḥ\lem  \mssCaCbCc\msNa\msNb; satye jīva \msNc, satyaṁ jīva \Ed
        \ \vd jīvaḥ\lem  \msCa\msCb\msNa\msNb\msNc\Ed; jīva \msCc}

satyam ekena samprāpto dharmasādhananiścayaḥ\thinspace{\dandab} \dontdisplaylinenum

rāmarāghavavīryeṇa satyam ekaṁ surakṣitam \veg\dontdisplaylinenum
\varr{
        \ \va satyam ekena\lem  \msCa\msCc\msNa\msNc\Ed; satyem ekena \msNb, satyam ekaina \msCb
        \ \vb dharma°\lem  \Ed; dharmaḥ \mssCaCbCc\msNa\msNb\msNc\oo
                 °niścayaḥ\lem  \msCb\msCc\msNa\msNb\msNc\Ed; °niścaḥ \msCa
        \ \vd satyam ekaṁ\lem  \mssCaCbCc\msNa\msNc\Ed; satyem ekaṁ \msNb\oo
                 surakṣitam\lem  \msCa\msCc\msNb\msNc\Ed; surakṣitaḥ \msNa, surikṣitam \msCb}

etat satyavidhānasya kīrtitaṁ tava suvrata\thinspace{\dandab} \dontdisplaylinenum

sarvalokahitārthāya kim anyac chrotum icchasi \veg\dontdisplaylinenum
\varr{
        \ \va etat satya°\lem  \msCa\msCc\msNa\msNb\msNc\Ed; evaṁ satya° \msCb
        \ \vb suvrata\lem  \msCa\msNa\msNc; suvrate \msCb\msNb, suvrata\uncl{ḥ} \msCc, suvrataṁ \Ed}

vigatarāga uvāca~{\dandab}\dontdisplaylinenum 

na hi t\textsubring{r}ptiṁ vijānāmi śrutvā dharmaṁ tavāpy aham\thinspace{\danda} \dontdisplaylinenum

upariṣṭād ato bhūyaḥ kathayasva tapodhana \veg\dontdisplaylinenum


\alalfejezet{yameṣv asteyam (3)}\varr{
        \ \va t\textsubring{r}ptiṁ\lem  \msCa\msCb\msNa\msNb\msNc\Ed; t\textsubring{r}pti \msCc\oo
                 vijānāmi\lem  \mssCaCbCc\msNa\msNc\Ed; vināmi \msNb
        \ \vb śrutvā dharmaṁ tavāpy aham\lem  \msCb\msCc\msNa\msNb\msNc; śru dharman tavāpy aham \msCa,
                                                         dharmaṁ śrutvā tathāpy aham \Ed
        \ \vd °dhana\lem  \msCc\msNa\msNb\Ed; °dhūna \msCa, °dhanaḥ \msCb\msNc}

anarthayajña uvāca~{\dandab}\dontdisplaylinenum 

steyaṁ ś\textsubring{r}ṇv atha viprendra pañcadhā parikīrtitam\thinspace{\danda} \dontdisplaylinenum

adattādānam ādau tu utkocaṁ ca tataḥ param \danda\dontdisplaylinenum

prasthavyājas tulāvyājaḥ prasahyasteya pañcamam \veg\dontdisplaylinenum
\varr{
        \ \vb °kīrtitam\lem  \msCa\msCc\msNa\msNb\msNc\Ed; °kīrttitām \msCb
        \ \vd utkocaṁ ca tataḥ\lem  \msCa\msCc\msNa\msNb\msNc; tkoca tataḥ \msCb, utkocaṁ cān\textsubring{r}taḥ \Ed
        \ \vc tulāvyājaḥ\lem  \msCb\msNc\Ed; tulāvyāja \msCa\msCc\msNa\msNb
        \ \vd °sahya°\lem  \mssCaCbCc\msNa\msNc\Ed; °sahye \msNb\oo
                 °steya\lem  \msCb\msCc\msNa\msNb\Ed; °stena \msCa\msNc\oo
                 pañcamam\lem  \msCa\msCb\msNa\msNb\msNc; pañcamaḥ \msCc\Ed}

dh\textsubring{r}ṣṭaduṣṭaprabhāvena paradravyāpakarṣaṇam\thinspace{\dandab} \dontdisplaylinenum

vāryamāṇo 'pi durbuddhir adattādānam ucyate \veg\dontdisplaylinenum
\varr{
        \ \va dh\textsubring{r}ṣṭaduṣṭa°\lem  \msCa\msNa\msNc\Ed; dh\textsubring{r}ṣṭadumna° \msCb, dh\textsubring{r}taduṣṭa° \msCc, d\textsubring{r}ṣtaduṣṭa° \msNb
        \ \vb °karṣaṇam\lem  \mssCaCbCc\msNb\msNc\Ed; °karṣaṇa \msNa
        \ \vb vāryamāṇo 'pi\lem  \msCa\msCc\msNa\msNb\msNc\Ed; vāryamāno vi° \msCb}

utkocaṁ ś\textsubring{r}ṇu viprendra dharmasaṁkarakārakam\thinspace{\dandab} \dontdisplaylinenum

mūlyaṁ kāryavināśārtham utkocaḥ parig\textsubring{r}hyate \danda\dontdisplaylinenum

tena cāsau vijānīyād dravyalobhabalāt k\textsubring{r}tam \veg\dontdisplaylinenum
\varr{
        \ \va utkocaṁ\lem  \msCb\msCc\msNa\msNb\msNc\Ed; utkoca \msCa\oo
                 viprendra\lem  \mssCaCbCc\msNa\msNc\Ed; vidrendra \msNb
        \ \vb °saṁkara°\lem  \msCc\msNa; °śaṅkara° \msCa\msCb\msNb, °sakara° \msNc, °saṁhāra° \Ed\oo
                 °kārakam\lem  \mssCaCbCc\msNb\msNc\Ed; °kārakaḥ \msNa
        \ \vc mūlyaṁ\lem  \conj; mūla \mssCaCbCc\msNa\msNb\msNc\Ed\oo
                 °vināśārtha°\lem  \mssCaCbCc\msNapcorr\msNb\msNc\Ed; °vinārtha° \msNaacorr
        \ \vd °tkocaḥ\lem  \mssCaCbCc\msNa\msNc; °tkocaṁ \msNb, °tkoca \Ed
        \ \vef vijānīyād dra°\lem  \msCa\msCb\msNa\msNb\msNc\Ed; vijānīyā dra° \msCc}

prasthavyāja-upāyena kuṭumbaṁ trātum icchati\thinspace{\dandab} \dontdisplaylinenum

taṁ ca stenaṁ vijānīyāt paradravyāpahārakam \veg\dontdisplaylinenum
\varr{
        \ \vc taṁ ca stenaṁ\lem  \msCa; tañ ca stena \msCb,
                                taṁ ca steyaṁ \msNa, tañ ca teya \msNb, so 'pi tena \msCc\Ed, tañ ca tena \msNc
        \ \vd °hārakam\lem  \msCa\msCb\msNapcorr\msNc\Ed; °hārakaḥ \msCc, °hārakā \msNaacorr °hārakāḥ \msNb}

tulāvyāja-upāyena parasvārthaṁ hared yadi\thinspace{\dandab} \dontdisplaylinenum

cauralakṣaṇakāś cānye kūṭakāpaṭikā narāḥ \veg\dontdisplaylinenum
            \paral{\textit{\vcd {\normalfont cf. Umāmaheśvarasaṁvāda 8.3cd: }kūṭakāpaṭikāś caiva satyārjavavivarjitāḥ}}
\varr{
        \ \va parasvārthaṁ\lem  \msCa\msCc\msNa\msNc; parasvārtha \msCb\msNb, parasyārthaṁ \Ed\oo
                 hared yadi\lem  \msCa\msCc\msNa\msNb\msNc\Ed; hared yati \msCb
        \ \vd kūṭakāpaṭikā\lem  \msNb; \uncl{ku}ṭakā yaṭikā \msCa, kūṭakāyaṭikā \msCb\msCc\msNaacorr\msNc,
                                         kūṭakāryaṭikā \msNapcorr\Ed}

durbalārjavabāleṣu cchadmanā vā balena vā\thinspace{\dandab} \dontdisplaylinenum

apah\textsubring{r}tya dhanaṁ mūḍhaḥ sa coraś cora ucyate \veg\dontdisplaylinenum
\varr{
        \ \va °rjava°\lem  \mssCaCbCc\msNa\msNc\Ed; °java° \msNb
        \ \vb cchadmanā\lem  \Ed; cchanmanā \mssCaCbCc\msNa\msNb, cchatmānā \msNc
        \ \vcd mūḍhaḥ sa\lem  \mssCaCbCc\msNa\msNc\Ed; mūḍhās sa \msNb
        \ \vd coraś cora\lem  \msCa\msCc\msNb\Ed; caura cora \msCb, cauraś caura \msNa, cauraś cora \msNc}

nāsti steyasamaṁ pāpaṁ nāsty adharmaś ca tatsamaḥ\thinspace{\dandab} \dontdisplaylinenum

nāsti stenasamākīrtir nāsti stenasamo 'nayaḥ \veg\dontdisplaylinenum
\varr{
        \ \vab (nāsti{\normalfont ...} tatsamaḥ)\lem  \mssCaCbCc\msNa\msNb\msNc; \om\ \Ed
        \ \va steya°\lem  \msNa\msNc; tena \msCa, stena° \msCb\msCc\msNb, \om\ \Ed
        \ \vb °samaḥ\lem  \msCa\msCb\msNa\msNb\msNc; °samaṁ \msCc, \om\ \Ed
        \ \vcd (nāsti{\normalfont ...} 'nayaḥ)\lem  \mssCaCbCc\msNa\msNb\msNc; \om\ \Ed
        \ \vc stena°\lem  \msCa\msCb\msNa\msNb; tena \msCc, steya° \msNc, \om\ \Ed\oo
                 °samā°\lem  \msCb\msCc\msNb; °samo \msCa\msNa\msNc, \om\ \Ed
        \ \vd stena°\lem  \mssCaCbCc\msNb\Ed; steya° \msNa\msNc}

nāsti steyasamāvidyā nāsti stenasamaḥ khalaḥ\thinspace{\dandab} \dontdisplaylinenum

nāsti stenasama ajño nāsti stenasamo 'lasaḥ \veg\dontdisplaylinenum
\varr{
        \ \va steya°\lem  \msNa\msNc\Ed; stena° \mssCaCbCc\msNb\oo
                 °samā\lem  \msCc\msNb; °samo \msCa\msCb\msNa\msNc\Ed
        \ \vb stena°\lem  \mssCaCbCc\msNb; steya° \msNa\msNc, tena \Ed
        \ \vc stena°\lem  \msCa\msCb\msNb\msNc; steya° \msCc\msNa\Ed\oo
                 °sama\lem  \mssCaCbCc\msNa\msNc\Ed\ \unmetr; °samaṁ \msNb\oo
                 ajño\lem  \msCb; ajña{\il} \msCa, ajña \msCc\msNa\msNb\msNc, ajñaḥ \Ed
        \ \vd stena°\lem  \msCa\msCb\msNb; steya° \msCc\msNa\msNc, tena \Ed}

nāsti stenasamo dveṣyo nāsti stenasamo 'priyaḥ\thinspace{\dandab} \dontdisplaylinenum

nāsti steyasamaṁ duḥkhaṁ nāsti stenasamo 'yaśaḥ \veg\dontdisplaylinenum
\varr{
        \ \va stena°\lem  \msCa\msCb\msNb; steya° \msCc\msNa\msNc, tena \Ed
        \ \vb stena°\lem  \msNb; steya° \mssCaCbCc\msNa\msNc\Ed
        \ \vc steya°\lem  \msCc; stena° \msCa\msCb\msNa\msNb, stenya° \msNc, tena \Ed
        \ \vd stena°\lem  \msCa\msCb\msNa\msNb; steya° \msCc\msNc, tena \Ed}

\ujvers\nemsloka 
pracchanno hriyate 'rtham anyapuruṣaḥ pratyakṣam anyo haret
\dontdisplaylinenum

\nemslokab 
nikṣepād dhanahāriṇo 'nya-m-adhamo vyājena cānyo haret \danda\dontdisplaylinenum

\nemslokac 
anye lekhyavikalpanāh\textsubring{r}tadhanā \crux{anyo h\textsubring{r}tād vai h\textsubring{r}tā}
\dontdisplaylinenum

\nemslokad 
\crux{anyaḥ krītadhano 'paro dhayah\textsubring{r}ta} ete jaghanyāḥ sm\textsubring{r}tāḥ \veg\dontdisplaylinenum
\varr{
        \ \va pracchanno\lem  \msCa\msCc\msNa\msNb\msNc\Ed; prasthanno \msCb\oo
                 'rtham anyapuruṣaḥ\lem  \msCb\msNc; ca vittam athavā \msNapcorr\Ed, 
                                vittam \msCa\msNaacorr\msNb, citta \msCc\oo
                 pratyakṣam anyo\lem  \msCa\msCc\msNa\msNb\msNc; pratyakṣam ano \msCb, pratyakṣyam anye \Ed
        \ \vb nikṣepād dhana°\lem  \msCa\msCb\msNa; nikṣepā dhana° \msCc\msNb\msNc, nikṣepātraya° \Ed\oo
                 °hāriṇo\lem  \msCa\msCc\msNa\msNc\Ed; °hāriṇyo \msCb, °hāriṇā \msNb\oo
                 'nyam adhamo\lem  \msCa\msCb\msNa\msNb\msNc; 'nyam adhano \msCc, 'nyavidhayo \Ed\oo
                 cānyo\lem  \mssCaCbCc\msNa\msNb\msNc; cānyā \Ed\oo
                 haret\lem  \mssCaCbCc\msNb\msNc\Ed; hare \msNa
        \ \vc anye lekhya°\lem  \corr; anyā lekha° \msCb\msCc,
                                 anyo le\uncl{khya}° \msCa, anyo lekhya° \msNa\msNb\msNc, anyollekhya \Ed\oo
                 °dhanā anyo\lem  \msCa\msCc\msNa\msNb\msNc\Ed; °dhanyo \msCb\oo
                 h\textsubring{r}tād vai\lem  \mssCaCbCc\msNc\Ed; h\textsubring{r}tad vai \msNa, h\textsubring{r}tād ve \msNb
        \ \vd anyaḥ krītadhano\lem  \mssCaCbCc\msNa\msNb; anya krītadhano \msNc, anāśrītadhanaṁ \Ed\oo
                 'paro dhayah\textsubring{r}ta\lem  \msCa\msCc\msNb; paro dhayahyata \msCb, paro dhana\uncl{h\textsubring{r}ta} \msNa, 
                                             parodhaprah\textsubring{r}ta \msNc, madā hy apah\textsubring{r}taṁ \Ed\oo
                 jaghanyāḥ\lem  \mssCaCbCc\msNa\msNb\msNc; jaghanyaḥ \Ed}

\ujvers\nemsloka 
stenastulya na mūḍham asti puruṣo dharmārthahīno 'dhamaḥ
\dontdisplaylinenum

\nemslokab 
yāvaj jīvati śaṅkayā narapateḥ saṁtrasyamāno raṭan \danda\dontdisplaylinenum
            \paral{\textit{\vo {\normalfont The lower folio side in exposure 49 in \msNb\ is rather blurred and seems to be partly erased,
                        therefore all the readings in this MS for verses 4.29--46 are rather uncertain,
                        even if not indicated explicitly.}}}

\nemslokac 
prāptaḥśāsana tīvrasahyaviṣamaṁ prāpnoti karmeritaḥ
\dontdisplaylinenum

\nemslokad 
kālena mriyate sa yāti nirayam ākrandamāno bh\textsubring{r}śam \veg\dontdisplaylinenum
\varr{
        \ \va stenastulya\lem  \Ed; stenatulya \msCa\msCb\msNc\ \unmetr, steyastulya \msCc,
                         steyatulya \msNa\ \unmetr, tena tulya \msNb\ \unmetr
        \ \vb yāvaj jīvati\lem  \mssCaCbCc\msNa\msNb\msNc; yāvat taj jīvati \Ed\oo
                 °pateḥ\lem  \msCb\msNb\msNc; °patiḥ \msCa\msCc\msNa\Ed\oo
                 saṁtrasyamāno raṭan\lem  \mssCaCbCc\msNa\msNb\msNc; saṁtrāsyamāno śaṭhaḥ \Ed
        \ \vc prāptaḥ°\lem  \mssCaCbCc\msNb\msNc\Ed; prāpta° \msNa\oo
                 °sahya°\lem  \mssCaCbCc\msNa\msNc; {\lost}{\lost} \msNb, °sadya° \Ed\oo
                 °viṣamaṁ\lem  \eme; °viṣamaḥ \mssCaCbCc\msNa\msNc\Ed, {\lost}{\lost}{\lost} \msNb\oo
                 karmeritaḥ\lem  \msCb\msCc\msNa\msNc\Ed; karme\uncl{rita} \msCa, {\lost}{\lost}\uncl{ritaḥ} \msNb
        \ \vd nirayam ākrandamāno\lem  \mssCaCbCc\msNa; \uncl{nira}yam ākrandamā\uncl{no} \msNb, 
                                                               nirayaṁ sa krandamāno \msNc, niyamam ākrandramāno \Ed}

\ujvers\nemsloka 
nītvā durgatikoṭikalpa nirayāt tiryatvam āyānti te
\dontdisplaylinenum

\nemslokab 
tiryatve ca tathaivam ekaśatikaṁ prabhramya varṣārbudam \danda\dontdisplaylinenum

\nemslokac 
mānuṣyaṁ tad avāpnuvanti vipule dāridryarogākulam
\dontdisplaylinenum

\nemslokad 
tasmād durgatihetu karma sakalaṁ tyaktvā śivaṁ cāśrayet \veg\dontdisplaylinenum


\alalfejezet{yameṣv ān\textsubring{r}śaṁsyam (4)}
\vers
\varr{
        \ \va nirayāt tiryatva°\lem  \msCb\msNa; nirayān tiryatva° \msCa, nirayā tiryatva° \msCc, 
                       ni\uncl{rayāt tiryatva}° \msNb, nirayān tiryakṣa° \msNc, nirayān tiryaktva° \Ed
        \ \vb tiryatve\lem  \mssCaCbCc\msNa\msNc; \uncl{tiryatve} \msNb, tiryaktvaṁ \Ed\oo
                 tathaivam ekaśatikaṁ\lem  \msCb; tathaikam ekaśatikaṁ \msCa\msNa\msNc,
                        tathaikam ekaśatika \msCc, \uncl{tathai}kam ekaśatikaṁ \msNb, tathaikam ekasakikaṁ \Ed\oo
                 °bhramya°\lem  \mssCaCbCc\msNc\Ed; °bhrāmya \msNa, °{\lost}{ā}mya \msNb\oo
                 varṣārbudam\lem  \msNcpcorr; varṣāmbudam \msCa\msCb\msNa\msNb\msNcacorr, varṣāmbudaḥ \msCc\Ed
        \ \vc  mānuṣyaṁ\lem  \msCa\msCc\msNa\msNc\Ed; mānuṣya \msCb\ \unmetr, \uncl{mānuṣya} \msNb\ \toplost\oo
                 vipule\lem  \mssCaCbCc\msNa\msNc; vipu\uncl{la} \msNb\ \toplost, vipulaṁ \Ed\oo
                 dāridrya°\lem  \mssCaCbCc\msNa\msNc; {\il}ri{\il} \msNb, dāridhra° \Ed
        \ \vd tasmād du°\lem  \msCa\msCb\msNa\msNc\Ed; tasmā du° \msCc, \uncl{tasmā du°} \msNb\oo
                 cāśrayet\lem  \mssCaCbCc\msNb\msNc\Ed; cāśrat \msNa}

aṣṭamūrtiśivadveṣṭā pitur mātuś ca yo dviṣet\thinspace{\dandab} \dontdisplaylinenum

gavāṁ vā atither dveṣṭā n\textsubring{r}śaṁsāḥ pañca eva te \veg\dontdisplaylinenum
\varr{
        \ \va °śiva°\lem  \mssCaCbCc\msNa\msNb\Ed; °śivaṁ \msNc
        \ \vc gavāṁ vā\lem  \msCa\msCc\msNa\msNc\Ed; avām vā \msCb, {\il}{\il}\uncl{m vā} \msNb\oo
                 atither dve°\lem  \msCa\msCb\msNb\msNc\Ed; atithidve° \msCc, atithe dve° \msNa
        \ \vd n\textsubring{r}śaṁsāḥ\lem  \msCa\msCc\msNa\msNb; n\textsubring{r}śaṁsā \msCb\msNc\Ed}

aṣṭamūrtiḥ śivaḥ sākṣāt pañcavyomasamanvitaḥ\thinspace{\dandab} \dontdisplaylinenum

sūryaḥ somaś ca dīkṣaś ca dūṣakaḥ tann\textsubring{r}śaṁsakaḥ \veg\dontdisplaylinenum
\varr{
        \ \va °mūrtiḥ\lem  \mssCaCbCc\msNa\msNb\msNc; °mūrti° \Ed
        \ \vb °nvitaḥ\lem  \msCa\msCb\msNa\msNc\Ed; °nvitāḥ \msCc\msNb
        \ \vc sūryaḥ\lem  \mssCaCbCc\msNa; \uncl{sūrya°} \msNb\msNc, sūrya° \Ed\oo
                 dīkṣa°\lem  \mssCaCbCc\msNa\msNc; \uncl{dī}{\il} \msNb, dīkṣu° \Ed
        \ \vd tann\textsubring{r}°\lem  \emeTorzsok; sa n\textsubring{r}° \mssCaCbCc\msNa\msNb\msNc\Ed}

pitākāśasamo jñeyo janmotpattikaraḥ pitā\thinspace{\dandab} \dontdisplaylinenum

pit\textsubring{r}daivatam ādityam ān\textsubring{r}śaṁsa tamanvitaḥ \veg\dontdisplaylinenum
\varr{
        \ \vb °karaḥ pitā\lem  \msCa\msCb\msNa\msNc\Ed;  °\uncl{karaḥ pitā} \msNb, °karapitāḥ \msCc
        \ \vc °daivata°\lem  \msCa\msCc\msNa\msNc\Ed; °devata° \msCb, {\il}vata° \msNb
        \ \vcd °dityam ān\textsubring{r}śaṁsa tamanvitaḥ\lem  \eme; °diścam ān\textsubring{r}śaṁsa tamanvitaḥ \msCa\msCb,
                                 °dityam an\textsubring{r}śaṁsa tamanvitaḥ \msCc\msNb,
                                 °diśca an\textsubring{r}śaṁsa tamānvitaḥ \msNa, 
                                 °diścam an\textsubring{r}śaṁsa tamānvitaḥ \msNc, 
                                 °dityam mān\textsubring{r}śaṁsa tato 'nvitaḥ \Ed}

p\textsubring{r}thvyā gurutarī mātā ko na vandeta mātaram\thinspace{\dandab} \dontdisplaylinenum

yajñadānatapovedās tena sarvaṁ k\textsubring{r}taṁ bhavet \veg\dontdisplaylinenum
\varr{
        \ \va p\textsubring{r}thvyā\lem  \msCa\msCb\msNc; \uncl{p\textsubring{r}thvyā} \msCc\msNa, p\textsubring{r}thvī \msNb, 
                                                                p\textsubring{r}thvyāṁ \Ed
        \ \vb vandeta\lem  \msCa\msNa\msNb\msNc\Ed; vandena vandeta \msCb, vandyeta \msCc
        \ \vd sarvaṁ\lem  \eme; sarva \mssCaCbCc\msNa\msNb\msNc\Ed}

gāvaḥ pavitraṁ maṅgalyaṁ devatānāṁ ca devatāḥ\thinspace{\dandab} \dontdisplaylinenum
            \paral{\textit{\va {\normalfont \kb\ Viṣṇusm\textsubring{r}ti 23.57c: } gāvaḥ pavitramaṅgalyaṁ (goṣu lokāḥ pratiṣṭhitāḥ)
                {\normalfont  cf.\ also MBh Indices 13.15.33: } gāvaḥ pavitraṁ paramaṁ goṣu lokāḥ pratiṣṭhitāḥ 
                {\normalfont and Agnipurāṇa 291.1cd: } gāvaḥ pavitrā māṅgalyā goṣu lokāḥ pratiṣṭhitāḥ}}

sarvadevamayā gāvas tasmād eva na hiṁsayet \veg\dontdisplaylinenum
            \paral{\textit{\vc {\normalfont = Viṣṇudharmottarapurāṇa 3.291.25c}}}
\varr{
        \ \va pavitraṁ\lem  \mssCaCbCc\msNa\msNc\Ed; \uncl{pavitra} \msNb\oo
                 maṅgalyaṁ\lem  \msCa\msCb\msNa; \uncl{maṅgalyaṁ} \msNb, māṅgalyaṁ \msCc\msNc\Ed\oo
                 devatāḥ\lem  \mssCaCbCc\msNc; daivatāḥ \msNa, \uncl{devatāḥ} \msNb, devatā \Ed
        \ \vd °smād eva\lem  \msCa\msCc\msNa\msNb\msNc; °smād uva \msCb, °smād gāvaṁ \Ed}

jātamātrasya lokasya gāvas trātā na saṁśayaḥ\thinspace{\dandab} \dontdisplaylinenum

gh\textsubring{r}taṁ kṣīraṁ dadhi mūtraṁ śak\textsubring{r}t karṣaṇam eva ca \veg\dontdisplaylinenum
                    \paral{\textit{\vo 
                        {\normalfont See a passage in Śivadharmottara 12.92 ff.\ (especially 12.92, 100, 102--104) that
                        seems closely related to VSS 4.36--39: }
t\textsubring{r}ṇāni khādanti vasanty araṇye pibanti toyāny aparigrahāṇi{\thinspace\danda}
duhyanti bāhyanti punanti pāpaṁ gavāṁ rasair jjīvati jīvalokaḥ{\thinspace\ketdanda} 92{\thinspace\ketdanda}
kutas teṣāṁ hi pāpāni yeṣāṁ g\textsubring{r}ham alaṅk\textsubring{r}taṁ{\thinspace\danda}
satataṁ bālavatsābhir gobhiḥ śrībhir iva svayaṁ{\thinspace\ketdanda} 93{\thinspace\ketdanda}
ye kurvanti gavāṁ bhaktiṁ t\textsubring{r}ṇatoyapradānataḥ{\thinspace\danda}
praṇasaṁrohanādyaiś ca gavāṁ lokaṁ vrajanti te{\thinspace\ketdanda} 94{\thinspace\ketdanda}
t\textsubring{r}ṇatoyapradeśeṣu nītvā yatnena mātaraḥ{\thinspace\danda}
ye rakṣanti sadā gopāḥ śivaloke vrajanti te{\thinspace\ketdanda} 95{\thinspace\ketdanda}
ye śivāya guror vāpi gāṁ prayacchanti bhaktitaḥ{\thinspace\danda}
te modanti śiveloke bhogaiḥ kalpāyutan narāḥ{\thinspace\ketdanda} 96{\thinspace\ketdanda}
nivedya gorasaṁ bhakatyā śivāya śivayogine{\thinspace\danda}
sarvān kāmān avāpnoti śivalokam ito gataḥ{\thinspace\ketdanda} 97{\thinspace\ketdanda}
ihalokasukhārthāya paralokahitāya ca{\thinspace\danda}
sarvasvenāpi gāntasmād ādadīta vipaścitaiḥ{\thinspace\ketdanda} 98{\thinspace\ketdanda}
aho sumahadāś caryam aho suṣṭhu subhāṣitaṁ{\thinspace\danda}
yat pavitraś ca m\textsubring{r}ṣṭaś ca goraso buddhivarddhanaḥ{\thinspace\ketdanda} 99{\thinspace\ketdanda}
gomatā nirjjitāḥ lokā v\textsubring{r}tāśāgomatārjitāḥ{\thinspace\danda}
gomatā nirjjitaḥ svargaḥ samastaṁ gomatā jitaṁ{\thinspace\ketdanda} 100{\thinspace\ketdanda}
gāvo bandhur manuṣyāṇāṁ gāvaś ca dhanam uttamaṁ{\thinspace\danda}
gāvo yeṣāṁ g\textsubring{r}he na syur abandhur iva tadg\textsubring{r}haṁ{\thinspace\ketdanda} 101{\thinspace\ketdanda}
sp\textsubring{r}ṣṭvā tu gāvaḥ śamayanti pāpaṁ datvā tu gāvas tridivaṁ nayanti{\thinspace\danda}
saṁrakṣitāś copanayanti vittaṁ gobhir nna tulyaṁ dhanama sti kiñcit{\thinspace\ketdanda} 102{\thinspace\ketdanda}
sampannam aśnāti dadāti nityaṁ pāpāpahaṁ mitravivarddhanaṁ ca{\thinspace\danda}
sa eva cārthaḥ paribhujyate ca gobhir nna tulyaṁ dhanam asti kiñcit{\thinspace\ketdanda} 103{\thinspace\ketdanda}
t\textsubring{r}ṇāni śuṣkāṇi vane caritvā pītvā ca toyāny am\textsubring{r}taṁ sravanti{\thinspace\danda}
yad gomayādyaiś ca punanti lokān gobhir nna tulyan dhanam asti kiñcit{\thinspace\ketdanda} 104{\thinspace\ketdanda}
sa eva loke paramaḥ pavitro gavāṁ rasaḥ puṣṭi vivarddhanaś ca{\thinspace\danda}
aśnāti nityaṁ vividhānnayuktāmiṣṭañ ciraṁ yena śiva prasādāt{\thinspace\ketdanda} 105{\thinspace\ketdanda}
                        {\normalfont ... etc.} 
        }}
\varr{
        \ \va jātamātrasya lokasya\lem  \msCa\msCc\msNa\msNc\Ed; jātamātra\uncl{sya lokasya} \msNb, 
                                                        satasātasya \msCbacorr, satasātasya nokasya \msCbpcorr
        \ \vd śak\textsubring{r}t ka°\lem  \msCa\msCc\msNa\msNc\Ed; \uncl{śak\textsubring{r}t ka°} \msNb, kṣat ka° \msCb}

\ujvers\nemsloka 
pañcām\textsubring{r}taṁ pañcapavitrapūtaṁ
\dontdisplaylinenum

\nemslokab 
ye pañcagavyaṁ puruṣāḥ pibanti \danda\dontdisplaylinenum

\nemslokac 
te vājimedhasya phalaṁ labhanti
\dontdisplaylinenum

\nemslokad 
tad akṣayaṁ svargam avāpnuvanti \veg\dontdisplaylinenum
\varr{
        \ \va °pavitrapūtam\lem  \msCc\msNa\Ed; °pavitrapūtana \msCa\ \unmetr,
                        °pavitraṁ \msCb\ \unmetr, °pavitrapūta \msNb,
                        °pavitrapūtaṁnaṁ \msNc\ \unmetr
        \ \vb °gavyaṁ\lem  \msCa\msCb\msNa\msNc\Ed; °gavyā \msCc, \uncl{°gavyāṁ} \msNb\oo
                 puruṣāḥ\lem  \msCa\msCb\msNa\msNb\msNc; puruṣā \msCc, puruṣaḥ \Ed\oo
                 pibanti\lem  \msCa\msCb\msNa\msNb\msNc\Ed; vivanti \msCc
        \ \vc labhanti\lem  \msCa\msCb\msNa\msNb\msNc\Ed; bhavanti \msCc
        \ \vd svarga°\lem  \msCa\msCc\msNa\msNb\msNc\Ed; sva° \msCb}

\ujvers\nemsloka 
gobhir na tulyaṁ dhanam asti kiṁcid
\dontdisplaylinenum
            \paral{\textit{\va {\normalfont cf. MBh 13.51.26cd: } gobhis tulyaṁ na paśyāmi dhanaṁ kiṁcid ihācyuta}}

\nemslokab 
duhyanti vāhyanti bahiścaranti \danda\dontdisplaylinenum

\nemslokac 
t\textsubring{r}ṇāni bhuktvā am\textsubring{r}taṁ sravanti
\dontdisplaylinenum

\nemslokad 
vipreṣu dattāḥ kulam uddharanti \veg\dontdisplaylinenum
\varr{
        \ \va gobhir na tu°\lem  \msNc; na gobhis tu° \mssCaCbCc\msNa\msNb\ \unmetr, na gāvatu° \Ed
        \ \vd dattāḥ\lem  \msCa\msCb\msNa\msNb\msNc; \uncl{dattā} \msCc, dattā \Ed}

\ujvers\nemsloka 
gavāhnikaṁ yaś ca karoti nityaṁ
\dontdisplaylinenum

\nemslokab 
śuśrūṣaṇaṁ yaḥ kurute gavāṁ tu \danda\dontdisplaylinenum

\nemslokac 
aśeṣayajñatapadānapuṇyaṁ
\dontdisplaylinenum

\nemslokad 
labhaty asau tam an\textsubring{r}śaṁsakartā \veg\dontdisplaylinenum

\vers
\varr{
        \ \va gavāhnikaṁ\lem  \msCb\msCc\msNa\msNb\msNc\Ed; gavāṁhnikaṁ \msCa\oo
                 yaś ca karoti\lem  \mssCaCbCc\msNa\msNb\msNc; yaḥ prakaroti \Ed
        \ \vb gavāṁ tu\lem  \msCb\msNc; gavān tu \msCa\msCc\msNa\msNb, gavānām \Ed
        \ \vc °tapa°\lem  \mssCaCbCc\msNa\msNc; \uncl{°tapa°} \msNb, °japa° \Ed
        \ \vd labhaty asau tam an\textsubring{r}śaṁsakartā\lem  \msCb\msNa\msNb\msNc;
                         labhaty asau bham an\textsubring{r}śaṁsakarttā \msCa,
                                 labhaty asau tam an\textsubring{r}taṁ sa karttā \msCc,
                                 bhavaty asau dharmam aśeṣakartā \Ed}

atithiṁ yo 'nugaccheta atithiṁ yo 'numanyate\thinspace{\dandab} \dontdisplaylinenum

atithiṁ yo 'nupūjyeta atithiṁ yaḥ praśaṁsate \veg\dontdisplaylinenum
\varr{
        \ \vd praśaṁsate\lem  \msCa\msCb\msNa\msNb\msNc\Ed; praśaṁsyate \msCc}

atithiṁ yo na pīḍyeta atithiṁ yo na duṣyati\thinspace{\dandab} \dontdisplaylinenum

atithipriyakartā yaḥ atitheḥ paricārakaḥ \danda\dontdisplaylinenum

atitheḥ k\textsubring{r}tasaṁtoṣas tasya puṇyam anantakam \veg\dontdisplaylinenum
\varr{
        \ \va na pīḍyeta\lem  \msCa\msCb\msNa\Ed; na gaccheta \eyeskip{to 4.40c} \msCc,
                                                 \uncl{na pī}{\il}{\il} \msNb, nipīḍyeta \msNc
        \ \vb atithiṁ\lem  \msCa\msCb\msNa\msNc\Ed; atiṁ \msCc, {\il}{\il}{\il} \msNb\oo
                 na duṣyati\lem  \msCa\msCc\msNa\msNc\Ed; nuduṣyati \msCb, {\il}duṣyati \msNb
        \ \vc atithi°\lem  \msCa\msNa; atithiṁ \msCb\msCc\msNc\Ed, ati\uncl{thiṁ} \msNb\oo
                 °priya°\lem  \msCa\msCb\msNa\msNb\msNc\Ed; priyaḥ \msCc\oo
                 yaḥ\lem  \msCb\msCc\msNb\msNc\Ed; yar \msCa, ya \msNa
        \ \ve atitheḥ\lem  \msCb\msCc\msNc; atithi° \msCa\msNa\msNb, atithiṁ \Ed
        \ \vef °saṁtoṣas tasya\lem  \msCa\msCc\msNa\msNb\msNc\Ed; °saṁtā yasya \msCb
        \ \vf puṇya°\lem  \mssCaCbCc\msNa\msNb\Ed; pūna° \msNc}

āsanenārghapādyena pādaśaucajalena ca\thinspace{\dandab} \dontdisplaylinenum

annavastrapradānair vā sarvaṁ vāpi nivedayet \veg\dontdisplaylinenum
\varr{
        \ \va °ārgha°\lem  \mssCaCbCc\msNa\msNb\msNc; °ārdhya° \Ed
        \ \vc annava°\lem  \msCa\msCb\msNa\msNc\Ed; annam va° \msCc, \uncl{anna}va° \msNb
        \ \vd nivedayet\lem  \mssCaCbCc\msNa\msNb\msNc; pradāpayet \Ed}

putradārātmanā vāpi yo 'tithim anupūjayet\thinspace{\dandab} \dontdisplaylinenum

śraddhayā cāvikalpena aklībamānasena ca \veg\dontdisplaylinenum
\varr{
        \ \va °dārātmanā\lem  \eme; °dārātmano \msCb\msCc\msNa\msNb\msNc,
                                °\uncl{dārā}tmano \msCa, °dārātmako \Ed
        \ \vb °pūjayet\lem  \msCa\msNa\Ed; °pūjyate \msCb\msCc\msNb, °pūjate \msNc
        \ \vc śraddhayā\lem  \msCa\msCb\msNa\msNb\msNc\Ed; śraddhāyā \msCc\oo
                 cāvikalpena\lem  \msCb\msCc\msNa\msNb\msNc\Ed; cāpi kalpena \msCa}

na p\textsubring{r}cched gotracaraṇaṁ svādhyāyaṁ deśajanmanī\thinspace{\dandab} \dontdisplaylinenum
            \paral{\textit{ {\normalfont \vab cf.\ \MBh\ 13.62.18ab:
                } na p\textsubring{r}cched gotracaraṇaṁ svādhyāyaṁ deśam eva vā}}

cintayen manasā bhaktyā dharmaḥ svayam ihāgataḥ \veg\dontdisplaylinenum
\varr{
        \ \va °caraṇaṁ\lem  \mssCaCbCc\msNa\msNb\msNc; °pravaraṁ \Ed
        \ \vb deśajanmanī\lem  \msCb\msCc\msNa\msNb\msNc\Ed; deśajanmanā \msCa
        \ \vc cintayen ma°\lem  \msCa\msCc\msNa\msNb\Ed; cittayet ma° \msCb, cintayet ma° \msNc
        \ \vd °gataḥ\lem  \msCa\msCb\msNa\msNc\Ed; °gatāḥ \msCc, ga\uncl{tam} \msNb}

aśvamedhasahasrāṇi rājasūyaśatāni ca\thinspace{\dandab} \dontdisplaylinenum

puṇḍarīkasahasraṁ ca sarvatīrthatapaḥphalam \veg\dontdisplaylinenum
\varr{
        \ \vb °sūya°\lem  \msCa\msNa\msNc\Ed; °sūrya° \msCb\msCc, °sū\uncl{rya}° \msNb
        \ \vd °tapaḥ°\lem  \mssCaCbCc\msNa\msNb\Ed; °tapa° \msNc\ \unmetr}

atithir yasya tuṣyeta n\textsubring{r}śaṁsamatam uts\textsubring{r}jet\thinspace{\dandab} \dontdisplaylinenum

sa tasya sakalaṁ puṇyaṁ prāpnuyān nātra saṁśayaḥ \veg\dontdisplaylinenum
\varr{
        \ \vb n\textsubring{r}śaṁsamatam uts\textsubring{r}jet\lem  \msCa\msNa\msNc; n\textsubring{r}śaṁsamata uts\textsubring{r}jet \msCb, 
                             n\textsubring{r}śaṁsakamamam uts\textsubring{r}jet \msCc, n\textsubring{r}sasamatam uts\textsubring{r}jet \msNb, na saṁśaya samaśnute \Ed}

\crux{na gatim atithijñasya gatim āpnoti karhicit}\thinspace{\dandab} \dontdisplaylinenum

tasmād atithim āyāntam abhigacchet k\textsubring{r}tāñjaliḥ \veg\dontdisplaylinenum
            \paral{\textit{\vcd {\normalfont = Vāyupurāṇa 2.17.8 = Brahmāṇḍapurāṇa 2.15.8; 
                        \kb Śivadharmottara 4.44ab: }
                        tasmād atithim āyāntam anugacchet k\textsubring{r}tāñjaliḥ}} 
\varr{
        \ \va na gatim a°\lem  \msCa\msCb\msNb\msNc; na gati nā° \msNa, na tithim a° \msCc\Ed
        \ \vb karhicit\lem  \msCa\Ed; karhacit \msCb\msCc\msNa\msNb\msNc
        \ \vc °yānta°\lem  \msCa\msCb\msNa\msNb\msNc\Ed; °yānti° \msCc}

saktuprasthena caikena yajña āsīn mahādbhutaḥ\thinspace{\dandab} \dontdisplaylinenum

atithiprāptadānena svaśarīraṁ divaṁ gatam \veg\dontdisplaylinenum
\varr{
        \ \va saktu°\lem  \eme; śanku° \msCa\msCb, śaṁktu° \msCc, śaktu° \msNa\msNc, śakthu° \msNb, śakti° \Ed\oo
                 caikena\lem  \mssCaCbCc\msNa\msNb\Ed; cekena \msNc
        \ \vb āsīn mahādbhutaḥ\lem  \corr; āsīn mahadbhutaḥ \msCa\msCb\msNa\msNb, āsī mahadbhutaḥ \msCc, 
                                               āsīt mahādbhutaḥ \msNc, āsīn mahadbhutam \Ed
        \ \vc °dānena\lem  \msCa\msCb\msNa\msNb\msNc\Ed; °prādānena \msCc
        \ \vd sva°\lem  \mssCaCbCc\msNa\msNb; \uncl{sa°} \msNc, sa° \Ed\oo
                 °gatam\lem  \msCa\msCb\msNa\msNb\msNc\Ed; °gataḥ \msCc}

nakulena purādhītaṁ vistareṇa dvijottama\thinspace{\dandab} \dontdisplaylinenum

viditaṁ ca tvayā pūrvaṁ prasthavārttā ca kīrtitā \veg\dontdisplaylinenum


\alalfejezet{yameṣu damaḥ (5)}\varr{
        \ \vb °ttama\lem  \msCa\msCb\msNa\msNb\msNc; °ttamam \msCc, °ttamaḥ \Ed
        \ \vd kīrtitā\lem  \msCa\msCb\msNa\msNb\msNc; kīrtitam \msCc, kīrtitāḥ \Ed}

dama eva manuṣyāṇāṁ dharmasārasamuccayaḥ\thinspace{\dandab} \dontdisplaylinenum
            \paral{\textit{\vb {\normalfont cf.\ Mahābhārata Indeces 14.4.2477: } śrotum icchāmi kārtsnyena dharmasārasamuccayam}}

damo dharmo damaḥ svargo damaḥ kīrtir damaḥ sukham \veg\dontdisplaylinenum
\varr{
        \ \vb dharmasāra°\lem  \eme; dharmaḥ sāra° \mssCaCbCc\msNa\msNb\msNc, dharmabhāra° \Ed
        \ \vc svargo\lem  \msCa\msCb\msNa\msNb\msNc\Ed; svarga \msCc
        \ \vd kīrtir da°\lem  \msCa\msCb\msNb\Ed; kīrti da° \msCc\msNa\msNc}

damo yajño damas tīrthaṁ damaḥ puṇyaṁ damas tapaḥ\thinspace{\dandab} \dontdisplaylinenum

damahīna-m-adharmaś ca damaḥ kāmakulapradaḥ \veg\dontdisplaylinenum
\varr{
        \ \va damas tī°\lem  \msCa\msCc\msNa\msNb\msNc\Ed; dama tī° \msCb
        \ \vd damaḥ\lem  \msCa\msCb\msNa\msNb\msNc; dama \msCc, damaṁ \Ed\oo
                 kāma°\lem  \mssCaCbCc\msNa\msNb\Ed; kāmaṁ \msNc}

nirdamaḥ kari mīnaś ca pataṅgabhramaram\textsubring{r}gāḥ\thinspace{\dandab} \dontdisplaylinenum

tvag jihvā ca tathā ghrāṇā cakṣuḥ śravaṇam indriyāḥ \veg\dontdisplaylinenum
\varr{
        \ \va °damaḥ\lem  \msCa\msCb\msNa\msNb\msNc\Ed; °dama \msCc
        \ \vb °bhramara°\lem  \mssCaCbCc\msNa\msNb\Ed\ \unmetr; °bhrama\uncl{rā}° \msNc
        \ \vc ghrāṇā\lem  \msCa\msNa\msNb\msNc\Ed; ghrāṇaṁ \msCb, ghrāṇa \msCc
        \ \vd °ndriyāḥ\lem  \mssCaCbCc\msNa\msNb\msNc; °ndriyaḥ \Ed}

durjayendriyam ekaikaṁ sarve prāṇaharāḥ sm\textsubring{r}tāḥ\thinspace{\dandab} \dontdisplaylinenum

damaṁ yo jayate samyag nirdamo nidhanaṁ vrajet \veg\dontdisplaylinenum
\varr{
        \ \vb sarve\lem  \msCa\msCc\msNa\msNb\msNc\Ed; sarva° \msCb\oo
                 °harāḥ\lem  \mssCaCbCc\msNa\msNb\msNc; °harā \Ed
        \ \vd vrajet\lem  \msCb\msCc\msNa\msNb\msNc\Ed; vraje{\lost} \msCa}

m\textsubring{r}ge śrotravaśān m\textsubring{r}tyuḥ pataṅgāś cakṣuṣor m\textsubring{r}tāḥ\thinspace{\dandab} \dontdisplaylinenum

ghrāṇayā bhramaro naṣṭo naṣṭo mīnaś ca jihvayā \veg\dontdisplaylinenum
            \paral{\textit{\vo {\normalfont Cf. Buddhacarita 11.35: } 
               gītair hriyante hi m\textsubring{r}gā vadhāya rūpārtham agnau śalabhāḥ patanti{\thinspace\danda} 
               matsyo giraty āyasam āmiṣārthī tasmād anarthaṁ viṣayāḥ phalanti{\thinspace\ketdanda}
        }}
\varr{
        \ \va m\textsubring{r}ge\lem  \mssCaCbCc\msNa\msNc; m\textsubring{r}go \msNb\Ed\oo
                 śrotra°\lem  \msCa\msCb\msNa\msNb\Ed; śotra° \msCc, śrota° \msNc\oo
                 °vaśā°\lem  \msCa\msCc\msNa\msNb\msNc\Ed; °vacaśā° \msCb
        \ \vb pataṅgāś ca°\lem  \mssCaCbCc\msNa\msNb\msNc; pataṅgā ca° \Ed\oo
                 °ṣor m\textsubring{r}tāḥ\lem  \msCa\msCb\msNa\msNb\Ed; °so m\textsubring{r}tāḥ \msCc, °ṣo m\textsubring{r}tāḥ \msNc
        \ \vc ghrāṇayā\lem  \msCa\msCc\msNa\msNb\msNc\Ed; ghrātayā \msCb
        \ \vcd naṣṭo naṣṭo\lem  \msCa\msCc\msNa\msNb\msNc\Ed; naṣṭo \msCb}

sparśena ca karī naṣṭo bandhanāvāsaduḥsahaḥ\thinspace{\dandab} \dontdisplaylinenum

kiṁ punaḥ pañcabhuktānāṁ m\textsubring{r}tyus tebhyaḥ kim adbhutam \veg\dontdisplaylinenum
\varr{
        \ \vb °saduḥsahaḥ\lem  \msCa\msCc\msNa\msNc\Ed; °saduḥsaha \msCb, °sudussahaḥ \msNb
        \ \vc punaḥ\lem  \msCapcorr\msCb\msCc\msNa\msNb\msNc\Ed; puna \msCaacorr
        \ \vd tebhyaḥ\lem  \mssCaCbCc\msNa\msNb\msNc; tebhya \Ed}

purūravo 'tilobhena atikāmena daṇḍakaḥ\thinspace{\dandab} \dontdisplaylinenum

sāgarāś cātidarpeṇa atimānena rāvaṇaḥ \veg\dontdisplaylinenum
\varr{
        \ \va purūravo\lem  \msCa\msCb\msNa\msNb\msNc; purorave \msCc, pururavā° \Ed\oo
                 tilobhena atikāmena\lem  \mssCaCbCc\msNa\msNb\msNc; tikāmena atilobhena \Ed
        \ \vb daṇḍakaḥ\lem  \mssCaCbCc\msNa\msNb\msNc; puṇḍakaḥ \Ed
        \ \va sāgarā°\lem  \eme; sagara° \msCa\msCb\msNa\msNb\msNc\Ed, sāgara° \msCc}

atikrodhena saudāsa atipānena yādavāḥ\thinspace{\dandab} \dontdisplaylinenum

atit\textsubring{r}ṣṇāc ca māndhātā nahuṣo dvijavajñayā \veg\dontdisplaylinenum
\varr{
        \ \vb atipānena\lem  \mssCaCbCc\msNa\msNb\msNc; atipāpena \Ed
        \ \vc atit\textsubring{r}ṣṇāc ca māndhātā\lem  \conj;
                atit\textsubring{r}ṣṇā ca māndāto \msCa,
                atit\textsubring{r}ṣṇā ca māndhāto \msCb\msCc\msNa\msNc,
                atit\textsubring{r}ṣṇā ca mandhāto \msNb,
                atit\textsubring{r}ṣṇā ca mānāc ca ca \Ed
        \ \vd nahuṣo\lem  \mssCaCbCc\msNa\msNc\Ed; naghuṣo \msNb}

atidānād balir naṣṭa atiśauryeṇa arjunaḥ\thinspace{\dandab} \dontdisplaylinenum

atidyūtān nalo rājā n\textsubring{r}go goharaṇena tu \veg\dontdisplaylinenum
            \paral{\textit{\vo {\normalfont After this verse, \Ed\ adds: } 
                        tasmād dama sadā sa rakṣet ati sarvatra varjayet
                       {\normalfont (understand: } tasmād damaṁ sadā rakṣet ati sarvatra varjayet{\normalfont )}}}
\varr{
        \ \va °r naṣṭa\lem  \msCa\msNa\msNb\msNc\Ed; °r naṣṭo \msCb, naṣṭo \msCc
        \ \vc atidyūtān nalo\lem  \msCa\msCc\msNb\msNc; atidyūtān naro \msCb\msNa, atikhyātān nalo \Ed
        \ \vd n\textsubring{r}go go°\lem  \Ed; n\textsubring{r}gaṅ go° \msCa\msCc\msNb\msNc, n\textsubring{r}gaṁ go° \msCb\msNa}

\ujvers\nemsloka 
damena hīnaḥ puruṣo dvijendra
\dontdisplaylinenum

\nemslokab 
svargaṁ ca mokṣaṁ ca sukhaṁ ca nāsti \danda\dontdisplaylinenum

\nemslokac 
vijñānadharmakulakīrtināśa
\dontdisplaylinenum

\nemslokad 
bhavanti vipra damayā vihīnāḥ \veg\dontdisplaylinenum


\alalfejezet{yameṣu gh\textsubring{r}ṇā (6)}
\vers
\varr{
        \ \va hīnaḥ puruṣo dvijendra\lem  \mssCaCbCc\msNa\msNc; 
                                hīna puruṣo dvijendra \msNb, hīnaṁ puruṣaṁ dvijendraḥ \Ed
        \ \vc °nāśa\lem  \msCb; °nāśo \Ed °nāma \msCa\msCc\msNa, °naśca \msNb, °nāgā \msNc
        \ \vd vipra\lem  \mssCaCbCc\msNaacorr\msNb\Ed; viprā \msNapcorr\msNc\oo
                 damayā\lem  \msCa\msCbpcorr\msCc\msNa\msNb\msNc\Ed; dayā \msCbacorr}

nirgh\textsubring{r}ṇo na paratrāsti nirgh\textsubring{r}ṇo na ihāsti vai\thinspace{\dandab} \dontdisplaylinenum

nirgh\textsubring{r}ṇe na ca dharmo 'sti nirgh\textsubring{r}ṇe na tapo 'sti vai \veg\dontdisplaylinenum
\varr{
        \ \va nirgh\textsubring{r}ṇo\lem  \msCa\msCb\msNb; nigh\textsubring{r}ṇo \msCc\msNc, nirgh\textsubring{r}ṇa \msNaacorr, 
                                nirgh\textsubring{r}\uncl{ṇe} \msNapcorr, nirgh\textsubring{r}ṇe \Ed
        \ \vb nirgh\textsubring{r}ṇo\lem  \msCa\msCb\msNaacorr\msNb; nigh\textsubring{r}ṇo \msCc\msNc, nirgh\textsubring{r}ṇe \msNapcorr\Ed
        \ \vc nirgh\textsubring{r}ṇe\lem  \msCa\msCb\msNb\Ed; nigh\textsubring{r}ṇe \msCc\msNa\msNc
        \ \vd nirgh\textsubring{r}ṇe\lem  \msCa\msCb\msNa\msNb\Ed; nigh\textsubring{r}ṇe \msCc\msNc}

parastrīṣu parārtheṣu parajīvāpakarṣaṇe\thinspace{\dandab} \dontdisplaylinenum

paranindāparānneṣu gh\textsubring{r}ṇāṁ pañcasu kārayet \veg\dontdisplaylinenum
\varr{
        \ \vb °jīvāpakarṣaṇe\lem  \msCa\msCc\msNa\msNb\msNc; °jīvāparkaṇe \msCb, °jīvopakarṣaṇe \Ed
        \ \vc paranindā°\lem  \msCb\msCc\msNa\msNb\msNc\Ed; paranind{\il}° \msCa\oo
                 °parānneṣu\lem  \mssCaCbCc\msNa\msNc\Ed; °parāṁneṣu \msNb
        \ \vd gh\textsubring{r}ṇāṁ\lem  \msCa\msCb\msNa\msNc; gh\textsubring{r}ṇā \msCc\msNb\Ed}

parastrī ś\textsubring{r}ṇu viprendra gh\textsubring{r}ṇīkāryā sadā budhaiḥ\thinspace{\dandab} \dontdisplaylinenum

rājñī viprī parivrājā svayoniparayoniṣu \veg\dontdisplaylinenum
\varr{
        \ \va gh\textsubring{r}ṇī°\lem  \msCa\msCc\msNa\msNb\msNc\Ed; gh\textsubring{r}ṇā \msCb
        \ \vc °vrājā\lem  \mssCaCbCc\msNc; °vrājī \msNa\msNb, °vrājyā \Ed
        \ \vd °para°\lem  \mssCaCbCc\msNa\msNc\Ed; °paśu° \msNb}

parārthe ś\textsubring{r}ṇu bhūyo 'nya anyāyārtha-m-upārjanam\thinspace{\dandab} \dontdisplaylinenum
            \paral{\textit{\vb {\normalfont cf.\ Bhagvadgītā 16.12: }
                āśāpāśaśatair baddhāḥ kāmakrodhaparāyaṇāḥ{\thinspace\danda}
                īhante kāmabhogārtham anyāyenārthasaṁcayān{\thinspace\ketdanda}}}

āḍhaprasthatulāvyājaiḥ parārthaṁ yo 'pakarṣati \veg\dontdisplaylinenum
\varr{
        \ \vb anyāyā°\lem  \mssCaCbCc\msNa\msNc\Ed; anyayā° \msNb\oo
                 °rjanam\lem  \mssCaCbCc\msNa\msNc\Ed; °rjjavam \msNb
        \ \vc °tulā°\lem  \mssCaCbCc\msNa\msNc\Ed; °tula° \msNb
        \ \vd °rthaṁ\lem  \msCa\msCb\msNa\Ed; °rtha \msCc, \uncl{°rtha} \msNb, °rthe \msNc}

jīvāpakarṣaṇe vipra gh\textsubring{r}ṇīkurvīta paṇḍitaḥ\thinspace{\dandab} \dontdisplaylinenum

vanajāvanajā jīvā vilagāś caraṇācarāḥ \veg\dontdisplaylinenum
\varr{
        \ \va vipra\lem  \msCb\msNa\msNb\msNc\Ed; vi\uncl{pra} \msCa, vipre \msCc
        \ \vb gh\textsubring{r}ṇī°\lem  \mssCaCbCc\msNa\msNb\msNc; gh\textsubring{r}ṇāṁ \Ed
        \ \vc vanajāvanajā\lem  \msCa\msCc\msNa\msNb\Ed; 
                        vanajāva{\il}jā \msCbacorr, vanajā va\uncl{ni}jā \msCbpcorr, vanaja vinajā \msNc
        \ \vd vilagāś caraṇācarāḥ\lem  \corr; 
                        vilagācaraṇācarāḥ \msCa\msCb\msNc, vilagocaragocaraḥ \msCc\Ed, vilagocaragocarāḥ \msNa,
                                                        \uncl{vilagācara}ṇācarāḥ \msNb}

paranindā ca kā vipra ś\textsubring{r}ṇu vakṣye samāsataḥ\thinspace{\dandab} \dontdisplaylinenum

devānāṁ brāhmaṇānāṁ ca gurumātātithidviṣaḥ \veg\dontdisplaylinenum
            \paral{\textit{\vcd {\normalfont These two pādas are illegible in \msNb}}}
\varr{
        \ \vb vakṣye\lem  \mssCaCbCc\msNa\msNb\msNc; vakṣyā \Ed}

parānneṣu gh\textsubring{r}ṇā kāryā abhojyeṣu ca bhojanam\thinspace{\dandab} \dontdisplaylinenum

sūtake m\textsubring{r}take śauṇḍe varṇabhraṣṭakule naṭe \veg\dontdisplaylinenum
            \paral{\textit{\vo {\normalfont This verse is mostly illegible in \msNb}}}
\varr{
        \ \vb abhojyeṣu\lem  \msCa\msCc\msNa\msNb\msNc\Ed; abhojye \msCb
        \ \vc śauṇḍe\lem  \msNa; sauṇḍye \msCa\msCc\msNc, śoṇḍye \msCb, \uncl{sauṇḍe} \msNb, sauṇḍo \Ed}

\ujvers\nemsloka 
ete pañcagh\textsubring{r}ṇāsu saktapuruṣāḥ svargārthamokṣārthinaḥ
\dontdisplaylinenum

\nemslokab 
loke 'nindanam āpnuvanti satataṁ kīrtir yaśo'laṁk\textsubring{r}tam \danda\dontdisplaylinenum

\nemslokac 
prajñābodhaśrutiṁ sm\textsubring{r}tiṁ ca labhate mānaṁ ca nityaṁ labhet
\dontdisplaylinenum

\nemslokad 
dākṣiṇyaṁ sa bhavet sa āyuṣa paraṁ prāpnoti niḥsaṁśayaḥ \veg\dontdisplaylinenum


\alalfejezet{yameṣu pañcavidho dhanyaḥ (7)}
\vers
\varr{
        \ \va °puruṣāḥ\lem  \msNc; °puruṣaḥ \mssCaCbCc\msNa\msNb\Ed\oo
                 °rthinaḥ\lem  \msNcpcorr; °rthināṁ \mssCaCbCc\msNa\msNb\Ed, °rthinā \msNcacorr
        \ \vb 'nindanam āpnuvanti\lem  \msCa\msCb\msNa\msNb\msNc; 
                        'nindanavāpnuvanti \msCc, nandanavāyuvānti \Ed
        \ \vc °śrutiṁ\lem  \msNc; °śruti° \mssCaCbCc\msNa\msNb\Ed\oo
                 nityaṁ\lem  \msCa\msCc\msNa\msNb\msNc\Ed; nitya \msCb
        \ \vd sa āyuṣa\lem  \eme; samāyuṣa \mssCaCbCc\msNc, samāyuṣaḥ \msNa\ \unmetr,
                                \uncl{samāyuṣa} \msNb, sa mānuṣa \Ed\oo
                 niḥsaṁśayaḥ\lem  \mssCaCbCc\msNb\msNc\Ed; nisaṁśayaḥ \msNa}

caturmaunaś catuḥśatruś caturāyatanaṁ tathā\thinspace{\dandab} \dontdisplaylinenum

caturdhyānaṁ catuṣpādaṁ pañcadhanyavidhocyate \veg\dontdisplaylinenum
\varr{
        \ \va caturmauna°\lem  \msCa\msCb\msNa\msNc\Ed; caturmoṇa° \msCc, \uncl{caturmauna°} \msNb
        \ \vab °tuḥ śatruś ca°\lem  \msCa\msCb\msNa\msNb\msNc; °tuśatru ca° \msCc, °tuḥ śatru ca° \Ed
        \ \vb °turāyatanaṁ\lem  \msCb\msCc\msNa\msNc\Ed; °\uncl{tu}rāyatanaṁ \msCa, \uncl{caturāyatanam} \msNb
        \ \vc °pādaṁ\lem  \mssCaCbCc\msNc\Ed; °pādaḥ \msNa, {\il}{\il} \msNb
        \ \vd pañcadhanya°\lem  \mssCaCbCc\msNa\msNb\msNc; dhanyapañca° \Ed}

caturmaunasya vakṣyāmi ś\textsubring{r}ṇuṣvāvahito bhava\thinspace{\dandab} \dontdisplaylinenum

pāruṣyapiśunāmithyāsambhinnāni ca varjayet \veg\dontdisplaylinenum
            \paral{\textit{\vcd {\normalfont cf. Divyāvadāna 186.21: }ārya, kim ebhiḥ karma k\textsubring{r}tam yenaivaṁvidhāni duḥkhāni pratyanubhavantīti? 
                    sa kathayati{\thinspace\danda} ete prāṇātipātikā adattādāyikāḥ kāmamithyācārikā m\textsubring{r}ṣāvādikāḥ paiśunikāḥ pāruṣikāḥ 
                    saṁbhinnapralāpikā abhidhyālavo vyāpannacittā mithyād\textsubring{r}ṣṭikāḥ{\thinspace\danda}}}
\varr{
        \ \va °maunasya\lem  \msCa\msCc\msNa\msNb\msNc\Ed; °monasya \msCb
        \ \vc pāruṣya°\lem  \mssCaCbCc\msNb\msNc\Ed; pāruṣyaṁ \msNa\oo
                 °piśunā°\lem  \mssCaCbCc\msNa\msNb\msNc; °piṇḍānā° \Ed}

kāmaḥ krodhaś ca lobhaś ca mohaś caiva caturvidhaḥ\thinspace{\dandab} \dontdisplaylinenum 

catuḥśatrur nihantavyaḥ so 'rihā vītakalmaṣaḥ \veg\dontdisplaylinenum
\varr{
        \ \vc catuḥśatrur ni°\lem  \msCa\msCb\Ed; catuśatru ni° \msCc\msNa\msNb\msNc
        \ \vd so 'rihā\lem  \msCa\msCc\msNa\msNb\msNc; srorihā \msCb, sarvathā \Ed}

caturāyatanaṁ vipra kathayiṣyāmi tac ch\textsubring{r}ṇu\thinspace{\dandab} \dontdisplaylinenum

karuṇā muditopekṣā maitrī cāyatanaṁ sm\textsubring{r}tam \veg\dontdisplaylinenum
\varr{
        \ \vc mudito°\lem  \mssCaCbCc\msNa\msNb\msNc; muditau° \Ed
        \ \vd cāyatanaṁ\lem  \msCc\msNa\msNb\msNc\Ed; cāyatana \msCa, cāyata\uncl{na} \msCb}

caturdhyānādhunā vakṣye saṁsārārṇavatāraṇam\thinspace{\dandab} \dontdisplaylinenum

ātmavidyābhavaḥ sūkṣmaṁ dhyānam uktaṁ caturvidham \veg\dontdisplaylinenum
\varr{
        \ \vc  °bhavaḥ\lem  \msCb\msCcpcorr\msNa\msNb\msNc; °bhava \msCa\msCcacorr, °bhavaṁ \Ed
        \ \vcd sūkṣmaṁ dhyā°\lem  \msCa\msNa\msNc\Ed; 
                                        sūkṣmā\uncl{nyā}° \msCb,
                                        sū\uncl{kṣma}dhyā° \msCc, sūkṣmadhyāna° \msNb
        \ \vd  °nam uktaṁ caturvidham\lem  \msCc\msNb; °nam uktaś caturvidham \msCa,
                                                                      °nam uktaś caturvidhaḥ \msCb\msNa, 
                                                                      °nam uktaṁ caturvidhiṁ \msNc, °nayajñaś ca \Ed}

ātmatattvaḥ sm\textsubring{r}to dharmo vidyā pañcasu pañcadhā\thinspace{\dandab} \dontdisplaylinenum

ṣaṭtriṁśākṣaram ityāhuḥ sūkṣmatattvam alakṣaṇam \veg\dontdisplaylinenum
\varr{
        \ \va sm\textsubring{r}to\lem  \msCa\msCb\msNa\msNb\msNc; sm\textsubring{r}tā \msCc\Ed\oo
                 dharmo\lem  \mssCaCbCc\msNa\msNb\msNc; dhanyā \Ed
        \ \vcd āhuḥ sū°\lem  \msCb\msCc\msNa\msNb\msNc\Ed; ā{\il}{\il} \msCa}

catuṣpādaḥ sm\textsubring{r}to dharmaś caturāśramam āśritaḥ\thinspace{\dandab} \dontdisplaylinenum

g\textsubring{r}hastho brahmacārī ca vānaprastho 'tha bhaikṣukaḥ \veg\dontdisplaylinenum
            \paral{\textit{\vcd {\normalfont  = MBh 12.234.13ab \kb\ MBh 14.4513ab etc. } }}
\varr{
        \ \vab dharmaś ca°\lem  \msCa\msCb\msNa\msNc\Ed; dharma ca° \msCc\msNb
        \ \vb °śritaḥ\lem  \mssCaCbCc\msNa\msNb\Ed; °śritāḥ \msNc
        \ \vd bhaikṣukaḥ\lem  \mssCaCbCc\msNa\msNb\msNc; bhakṣakaḥ \Ed}

dhanyās te yair idaṁ vetti nikhilena dvijottama\thinspace{\dandab} \dontdisplaylinenum

pāvanaṁ sarvapāpānāṁ puṇyānāṁ ca pravardhanam \veg\dontdisplaylinenum
\varr{
        \ \va yair idaṁ\lem  \msCa\msNa\msNb\msNc\Ed; yer idaṁ \msCb\msCc\oo
                 vetti\lem  \msCa\msCb\msNa\msNb\msNc\Ed; veti \msCc
        \ \vd pravardhanam\lem  \mssCaCbCc\msNa\msNb\msNc; pravardhanaḥ \Ed}

āyuḥ kīrtir yaśaḥ saukhyaṁ dhanyād eva pravardhate\thinspace{\dandab} \dontdisplaylinenum

śāntiḥ puṣṭiḥ sm\textsubring{r}tir medhā jāyate dhanyamānave \veg\dontdisplaylinenum


\alalfejezet{yameṣv apramādaḥ (8)}\varr{
        \ \vb dhanyād eva\lem  \mssCaCbCc\msNa\msNb\msNc; dharmād eva \Ed
        \ \vc puṣṭiḥ\lem  \msCb\msCc\msNa\msNb\msNc\Ed; {\il}ṣṭiḥ \msCa\oo
                 sm\textsubring{r}tir medhā\lem  \msCa\msCb\msNb\msNc\Ed; sm\textsubring{r}ti medhā \msCc\msNa
        \ \vd °mānave\lem  \eme; °mānavaḥ \mssCaCbCc\msNa\msNb\msNc\Ed}

pramādasthāna pañcaiva kīrtayiṣyāmi tac ch\textsubring{r}ṇu\thinspace{\dandab} \dontdisplaylinenum

brahmahatyā surāpānaṁ steyo gurvaṅganāgamam \danda\dontdisplaylinenum
            \paral{\textit{\vcdef {\normalfont \kb\ \MBh\ Indeces 12.30: }
                    brahmahatyāṁ surāpānaṁ steyaṁ gurvaṅganāgamam {\thinspace\danda}
                    mahānti pātakāny āhuḥ saṁyogaṁ caiva taiḥ saha {\thinspace\ketdanda}
                    {\normalfont  \kb\ Manu 11.55 (in Olivelle's edition): }
                    brahmahatyā surāpānaṁ steyaṁ gurvaṅganāgamaḥ {\thinspace\danda}
                    mahānti pātakāny āhuḥ saṁsargaś cāpi taiḥ saha {\thinspace\ketdanda}
                {\normalfont See also Yājñavalkyasm\textsubring{r}ti 3.227: }
                        brahmahā madyapaḥ stenas tathaiva gurutalpagaḥ {\thinspace\danda}
                        ete mahāpātakino yaś ca taiḥ saha saṁvaset {\thinspace\ketdanda} }}

mahāpātakam ity āhus tatsaṁyogī ca pañcamaḥ \veg\dontdisplaylinenum
\varr{
        \ \va °sthāna\lem  \msCa\msCc\msNa\msNb; °sthānaṁ \msCb\msNc\Ed\ \unmetr\oo
                 pañcaiva\lem  \mssCaCbCc\msNa\msNb\msNc; pañcaivaṁ \Ed
        \ \vb kīrtayiṣyāmi\lem  \mssCaCbCc\msNa\msNc\Ed; kīrtiyiṣyāmi \msNb}

an\textsubring{r}taṁ ca samutkarṣe rājagāmī ca paiśunaḥ\thinspace{\dandab} \dontdisplaylinenum

guroś cālīkanirbaddhaḥ samāni brahmahatyayā \veg\dontdisplaylinenum
            \paral{\textit{\vo \kb\ {\normalfont MBh 5.40.3: }
                 an\textsubring{r}taṁ ca samutkarṣe rājagāmi ca paiśunam {\thinspace\danda}
                 guroś cālīkanirbandhaḥ samāni brahmahatyayā {\thinspace\ketdanda}
                        {\normalfont = Manu 11.56 \kb\ Viṣṇusm\textsubring{r}ti 37.1--4 \kb\ Agnipurāṇa 168.25} }}
\varr{
        \ \va samutkarṣe\lem  \eme;
                         samutkarṣaṁ \msCa\msNa, samutkarṣa \msCc\msNb\msNc\Ed, samutka\uncl{rṣa} \msCb
        \ \vb rāja°\lem  \mssCaCbCc\msNa\msNb\msNc; rājñī° \Ed
        \ \vc °nirbaddhaḥ\lem  \msCa\msCb\msNc; nibaddhas \msCa\msCc\msNa\msNb, nirvaddhas \Ed
        \ \vd brahmahatyayā\lem  \msCb\msCc\msNa\msNb\msNc\Ed; bra{\il}{\il}{\il}yā \msCa}

brahmo \textsubring{r}gvedanindā ca kūṭasākṣī suh\textsubring{r}dvadhaḥ\thinspace{\dandab} \dontdisplaylinenum

garhitānādyayor jagdhiḥ surāpānasamāni ṣaṭ \veg\dontdisplaylinenum
            \paral{\textit{\vo \kb\ {\normalfont Manu 11.57: }
                brahmojjhatā vedanindā kauṭasākṣyaṁ suh\textsubring{r}dvadhaḥ {\thinspace\danda}
                garhitānādyayor jagdhiḥ surāpānasamāni ṣaṭ {\thinspace\ketdanda}
                {\normalfont See also Yājñavalkyasm\textsubring{r}ti 3.228: }
                        gurūṇām adhyadhikṣepo vedanindā suh\textsubring{r}dvadhaḥ {\thinspace\danda}
                        brahmahatyāsamaṁ jñeyam adhītasya ca nāśanam {\thinspace\ketdanda}}}
\varr{
        \ \va brahmo\lem  \mssCaCbCc\msNa\msNb\msNc; brahma \Ed
        \ \vb suh\textsubring{r}d vadhaḥ\lem  \mssCaCbCc\msNa\msNb\msNc; sak\textsubring{r}d budhaḥ \Ed
        \ \vc °nādyayor jagdhiḥ\lem  \eme; °nnañ ca yo jagdhis \msCa, °nnañ ca yo jagdhi \msCb,
                                 °nnañ ca yodvignaḥ \msCc, °nnaṁ ca yo jagdhiḥ \msNa, °nnaṁ ca yo jagdhiḥ \msNb,
                                 °nnañ ca yo jave \msNc, °nnaś ca yo vipraḥ \Ed}

retotsekaḥ svayonyāsu kumārīṣv antyajāsu ca\thinspace{\dandab} \dontdisplaylinenum

sakhyuḥ putrasya ca strīṣu gurutalpasamaḥ sm\textsubring{r}taḥ \veg\dontdisplaylinenum
            \paral{\textit{\vo \kb\ {\normalfont Manu 11.59: }
                                retaḥsekaḥ svayonīṣu kumārīṣv antyajāsu ca {\thinspace\danda}
                                sakhyuḥ putrasya ca strīṣu gurutalpasamaṁ viduḥ {\thinspace\ketdanda}}}
\varr{
        \ \va svayonyāsu\lem  \msCa\msCc\msNa\msNb\msNc\Ed; sutonyāsu \msCb
        \ \vc sakhyuḥ\lem  \eme; sakhya \mssCaCbCc\msNa\Ed, {\il}{\il} \msNb, sa\uncl{khyu} \msNc\oo
                 putrasya ca strīṣu\lem  \mssCaCbCc\msNa\msNc; {\il}{\il}{\il}{\il}{\il}{\il} \msNb, putrīṣu cāstrīṣu \Ed
        \ \vd °samaḥ\lem  \mssCaCbCc\msNa\msNc; {\il}{\il} \msNb, °sama \Ed}

nikṣepasyāpaharaṇaṁ narāśvarajatasya ca\thinspace{\dandab} \dontdisplaylinenum

bhūmivajramaṇīnāṁ ca rukmasteyasamaḥ sm\textsubring{r}taḥ \veg\dontdisplaylinenum
            \paral{\textit{\vo {\normalfont = Manu 11.58 }}}
\varr{
        \ \va nikṣepa°\lem  \msCa\msCc\msNa\msNc\Ed; \uncl{nikṣepa°} \msNb, nikhepa° \msCb
        \ \vb narāśvarajatasya\lem  \msCa\msCc\msNa\msNc\Ed; \uncl{narāśvarajatasya} \msNb,
                                                        narāṇāṁ svajanasya \msCb
        \ \vd rukmasteya°\lem  \eme; \uncl{rūgya}{\il}ya° \msCa,
                                rugmasteya° \msCb\msCc\msNa\msNc, {\il}{\il}{\il}{\il} \msNb, h\textsubring{r}tasteya° \Ed\oo
                 °samaḥ\lem  \msCa\msCbpcorr\msCc\msNa\msNb\msNc; saḥ \msCbacorr, °sama \Ed}

catvāra ete sambhūya yat pāpaṁ kurute naraḥ\thinspace{\dandab} \dontdisplaylinenum

mahāpātakapañcaitan tena sarvaṁ prakāśitam \danda\dontdisplaylinenum

pañcapramādam etāni varjanīyaṁ dvijottama \veg\dontdisplaylinenum


\alalfejezet{yameṣu mādhuryam (9)}\varr{
        \ \va ete\lem  \mssCaCbCc\msNa\msNc; \uncl{ete} \msNb, eva \Ed\oo
                 sambhūya\lem  \msCa\msCb\msNa\msNc\Ed; saṁbhūyo \msCc, \uncl{saṁbhūyo} \msNb
        \ \vc °pañcaitan\lem  \mssCaCbCc\Ed; °pañcaitam \msNb, °pañcetan \msNc, °pañcaite \msNa
        \ \ve °mādam\lem  \mssCaCbCc\msNa\msNb\msNc; °māda \Ed
        \ \vf varjanīyaṁ\lem  \msCa\msCb\msNa\msNb\msNc\Ed; varjanīyo \msCc}

kāyavāṅmanamādhuryaṁ cakṣur buddhiś ca pañcamaḥ\thinspace{\dandab} \dontdisplaylinenum

saumyad\textsubring{r}ṣṭipradānaṁ ca krūrabuddhiṁ ca varjayet \veg\dontdisplaylinenum
\varr{
        \ \vab manamādhuryaṁ ca°\lem  \eme; °manasā dhūryaś ca° \msCa\msCc\msNa\msNc,
                                                        °mana\uncl{mā}dhūryaś ca° \msCb,
                                                       °mana{\il}dhūrya{\il}° \msNb, °manasā bhūyaś ca° \Ed
        \ \vb °kṣur buddhi°\lem  \msCa\msCb\msNc\Ed; °kṣu buddhi° \msCc\msNa, {\il}{\il}{\il} \msNb
        \ \vc °dānaṁ ca\lem  \mssCaCbCc\msNa\msNc; {\il}{\il} \msNb, °dānaś ca \Ed
        \ \vd °buddhiṁ ca\lem  \msCa\msNa\msNc; buddhiś ca \msCb, °d\textsubring{r}ṣṭiṁ ca \msCc\Ed, {\il}{\il}{\il} \msNb}

prasannamanasā dhyāyet priyavākyam udīrayet\thinspace{\dandab} \dontdisplaylinenum

yathāśaktipradānaṁ ca svāśramābhyāgato guruḥ \veg\dontdisplaylinenum
\varr{
        \ \va prasanna°\lem  \mssCaCbCc\msNa\Ed; \uncl{prasanna}° \msNb, prasaṁna° \msNc
        \ \vc yathā°\lem  \mssCaCbCc\msNa\msNb\msNc; yasya \Ed\oo
                 °dānaṁ\lem  \mssCaCbCc\msNa\msNb\msNc; °dātaś \Ed
        \ \vd svāśramā°\lem  \msCa\msCb\msNa\msNb\msNc\Ed; svāsamā° \msCc\oo
                 °gato\lem  \mssCaCbCc\msNa\msNb\Ed; °sato \msNc}

indhanodakadānaṁ ca jātavedam athāpi vā\thinspace{\dandab} \dontdisplaylinenum

sulabhāni na dattāni indhanāgnyudakāni ca \danda\dontdisplaylinenum

kṣute jīveti vā noktaṁ tasya kiṁ parataḥ phalam \veg\dontdisplaylinenum


\alalfejezet{yameṣv ārjavam (10)}\varr{
        \ \vb indhano°\lem  \mssCaCbCc\msNa\msNb\Ed; itvano° \msNc\oo
                 jāta°\lem  \msCa\msCc\msNa\msNb\msNc\Ed; jā° \msCb
        \ \vc sulabhāni na\lem  \mssCaCbCc\msNa\msNb\msNc; surabhāni ca \Ed
        \ \vd °dakāni\lem  \mssCaCbCc\msNa\msNc\Ed; °\uncl{ta}kāni \msNb
        \ \ve kṣute\lem  \conj; kṣutaṁ \mssCaCbCc\msNa\msNb\msNc, śataṁ \Ed}

pañcārjavāḥ praśaṁsanti munayas tattvadarśinaḥ\thinspace{\dandab} \dontdisplaylinenum

karmav\textsubring{r}ttyābhiv\textsubring{r}ddhiṁ ca pāratoṣikam eva ca \danda\dontdisplaylinenum

strīdhanotkocavittaṁ ca ārjavo nābhinandati \veg\dontdisplaylinenum
\varr{
        \ \va pañcārjavāḥ\lem  \msCa\msCb\msNa\msNc; pañcārjavaḥ \msCc, {\il}{\il}{\il}{\il} \msNb, pañcārjavā \Ed\oo
                 praśaṁsanti\lem  \mssCaCbCc\msNc; praśasanti \msNa\Ed, \uncl{prasasanti} \msNb
        \ \vc karma°\lem  \msCb\msCc\msNa\msNc\Ed; {\il}rmma° \msCa, \uncl{kammā}° \msNb\oo
                 °v\textsubring{r}ttyābhiv\textsubring{r}ddhiṁ ca\lem  \mssCaCbCc\msNa\msNc;
                                °v\textsubring{r}ttibhiv\textsubring{r}ddhiñ ca \msNb, °v\textsubring{r}tyābhiv\textsubring{r}ddhiś ca \Ed
        \ \ve strīdhanotkoca°\lem  \mssCaCbCc\msNa\msNb\msNc; strīdhanaṅgo ca \Ed\oo
                 °vittaṁ ca\lem  \mssCaCbCc\msNa\msNc\Ed; °vittiñ ca \msNb
        \ \vf ārjavo nā°\lem  \msCa\msCb\msNa\msNb\msNc; ārjavañ ca \msCc, ārjjavenā° \Ed}

ārjavo na v\textsubring{r}thā yajña ārjavo na v\textsubring{r}thā tapaḥ\thinspace{\dandab} \dontdisplaylinenum

ārjavo na v\textsubring{r}thā dānam ārjavo na v\textsubring{r}thāgnayaḥ \veg\dontdisplaylinenum
\varr{
        \ \vab ārjavo na v\textsubring{r}thā yajña ārjavo na v\textsubring{r}thā tapaḥ\lem  \mssCaCbCc\msNb\msNc; \om\ \msNaacorr,
                                                 ārjavo na v\textsubring{r}thā yajña ārjavo na v\textsubring{r}thā tapa \msNapcorr,                  
                                                 ārjavo na v\textsubring{r}thā yajñaś cārrjavo na v\textsubring{r}thā tapaḥ \Ed
        \ \vcd (ārjavo{\normalfont ...} v\textsubring{r}thāgnayaḥ)\lem  \mssCaCbCc\msNa\msNb\msNc; \om\ \Ed}

ārjavasyendriyagrāmaḥ suprasanno 'pi tiṣṭhati\thinspace{\dandab} \dontdisplaylinenum

ārjavasya sadā devāḥ kāye tasya caranti te \veg\dontdisplaylinenum
\varr{
        \ \vab (ārjava°{\normalfont ...} tiṣṭhati)\lem  \mssCaCbCc\msNa\msNb\msNc; \om\ \Ed
        \ \va °grāmaḥ\lem  \msCa\msCb\msNc\Ed; °grāmāt \msCc\msNb, °grāmāḥ \msNa
        \ \vd tasya caranti\lem  \msCb\msCc\msNa\msNb\msNc; tasya ramanti \Ed, ta{\il}{\lost}{\lost}nti \msCa}

\ujvers\nemsloka 
iti yamapravibhāgaḥ kīrtito 'yaṁ dvijendra
\dontdisplaylinenum

\nemslokab 
iha parata sukhārthaṁ kārayet taṁ manuṣyaḥ \danda\dontdisplaylinenum

\nemslokac 
duritamalapahārī śaṅkarasyājñayāste
\dontdisplaylinenum

\nemslokad 
bhavati p\textsubring{r}thivibhartā hy ekachatrapravartā \veg\dontdisplaylinenum

\vers


\jump
\begin{center}
\ketdanda iti v\textsubring{r}ṣasārasaṁgrahe yamavibhāgo nāmādhyāyaś caturthaḥ\ketdanda
\end{center}
\dontdisplaylinenum\vers 
\varr{
        \ \va yamapravibhāgaḥ\lem  \msCa\msCb\msNb\msNc; yamavibhāgaḥ \msCc,
                                yamapraribhāgaḥ \msNa, niyamaparibhāgaḥ \Ed\oo
                 dvijendra\lem  \mssCaCbCc\msNa\msNb\msNc; narendra \Ed
        \ \vb °yet taṁ manuṣyaḥ\lem  \corr; °yet tan manuṣyaḥ \msCa\Ed, °yet ta manuṣyaḥ \msCb,
                                                                 °yet tat manuṣyaḥ \msCc
        \ \vc durita°\lem  \mssCaCbCc\msNa\msNb\msNc; irita° \Ed\oo
                 °pahārī\lem  \msCa\msCb\msNa\msNb\msNc\Ed; °palapahārī \msCc\oo
                 °jñayāste\lem  \mssCaCbCc\msNb\msNc\Ed; °jñayāte \msNa
        \ \vd °vartā\lem  \conj; °v\textsubring{r}ttā \mssCaCbCc\msNb\msNc, °v\textsubring{r}ttāḥ \msNa\Ed
        \ {\normalfont \Colo: } nāmādhyāyaś caturthaḥ\lem  \mssCaCbCc\msNa\msNb\msNc;
                                                         nāmaś caturtho 'dhyāyaḥ \Ed}
\bekveg\szamveg\vfill\phpspagebreak\szam\bek\versno=0\fejno=5
\thispagestyle{empty}



\alfejezet{\textbf{pañcamo 'dhyāyaḥ}}\jump\jump

\alalfejezet{niyamāḥ}
\vers

vigatarāga uvāca~{\dandab}\dontdisplaylinenum 
\varr{
        \ \vo vigatarāga uvāca\lem  \msCb\msCc\msNa\msNb\msNc\Ed; vigata\uncl{rāga uvā}ca \msCa}

\nemsloka 
kathaya niyamatattvaṁ sāmprataṁ tvaṁ viśeṣād
\dontdisplaylinenum

\nemslokab 
am\textsubring{r}tavadanatulyaṁ śrotukāmo gato 'smi \danda\dontdisplaylinenum

\nemslokac 
prak\textsubring{r}tidahanadagdhaṁ jñānatoyair niṣiktam
\dontdisplaylinenum

\nemslokad 
\crux{apara vada matajñā} nāsti dharmeṣu t\textsubring{r}ptiḥ \veg\dontdisplaylinenum

\vers
\varr{
        \ \va kathaya ni°\lem  \mssCaCbCc\msNa\msNb\msNc; kathayati \Ed\oo
                 °tattvaṁ\lem  \msCa\msCc\msNa\msNb\msNc\Ed; taṁ \msCb\oo
                 sāmprataṁ tvaṁ viśeṣād\lem  \msCa\msNa\msNc\Ed; 
                      tvāṁ vaśeṣāt \msCb, sāṁprata tvaṁ viseṣāt \msCc\msNb
        \ \vb °tulyaṁ śro°\lem  \msCa\msCc\msNapcorr\msNb\msNc\Ed; °tulyāṁ śro° \msCb, 
                                                \uncl{°tulyaṁ śro} tulyaṁ sro° \msNaacorr\oo
                 °kāmo\lem  \mssCaCbCc\msNa\msNb\msNc; °kāmā \Ed
        \ \vc °dahana°\lem  \mssCaCbCc\msNa\msNb\msNc; °vadana° \Ed\oo
                 °r niṣiktam\lem  \msCa\msCc\msNa\msNb\msNc\Ed; °r vimuktam \msCb
        \ \vd apara°\lem  \mssCaCbCc\msNb\msNc\Ed; aparaṁ \msNa\ \unmetr\oo
                 °vadama°\lem  \msCapcorr\msCb\msCc\msNa\msNb\msNc; °vada° \msCaacorr, °vadana° \Ed\oo
                 °tajñā nāsti\lem  \msCa\msCb\msNa\msNc; °tajñā\uncl{nn}āsti \msCc, {\il}{\il}{\il}{\il} \msNb, °tajjñān nāsti \Ed}

anarthayajña uvāca~{\dandab}\dontdisplaylinenum 

\nemsloka 
śravaṇasukham ato 'nyat kīrtayiṣye dvijendra
\dontdisplaylinenum

\nemslokab 
niyamakalaviśeṣaḥ pañca pañca prakāraḥ \danda\dontdisplaylinenum

\nemslokac 
hariharamunibhīṣṭaṁ dharmasāraṁ dvijendra
\dontdisplaylinenum

\nemslokad 
kalikaluṣavināśaṁ prāyamokṣaprasiddham \veg\dontdisplaylinenum

\vers
\varr{
        \ \va °sukha°\lem  \mssCaCbCc\msNapcorr\msNb\msNc\Ed; °mukha° \msNaacorr\oo 
                 °m ato 'nyat\lem  \mssCaCbCc\msNa\msNc; °m ato 'nya \msNb, °m ano 'nyat \Ed\oo
                 kīrta°\lem  \mssCaCbCc\msNc\Ed; kīrti° \msNa\msNb
        \ \vb °viśeṣaḥ\lem  \msCc\msNa\msNb\msNc\Ed; viśe{\il} \msCa, °viśeṣa \msCb\oo
                 prakāraḥ\lem  \mssCaCbCc\msNa\msNb\Ed; pakāraḥ \msNc
        \ \vd °vināśaṁ\lem  \msCa\msCb\msNa\msNb\msNc; °vināśa° \msCc\Ed}

śaucam ijyā tapo dānaṁ svādhyāyopasthanigrahaḥ\thinspace{\dandab} \dontdisplaylinenum

vratopavāsamaunaṁ ca snānaṁ ca niyamā daśa \veg\dontdisplaylinenum
            \paral{\textit{\vo = {\normalfont  Liṅgapurāṇa 1.8.29\cd--30\ab } }}


\alalfejezet{niyameṣu śaucam (1)}\varr{
        \ \va ijyā\lem  \msCa\msCb\msNa\msNc\Ed; ījyā \msCc\msNb\oo
                 dānaṁ\lem  \mssCaCbCc\msNa\msNc\Ed; dāna° \msNb}

tatra śaucādinirdeśaṁ vakṣyāmīha dvijottama\thinspace{\dandab} \dontdisplaylinenum

śārīraśaucam āhāro mātrā bhāvaś ca pañcamaḥ \veg\dontdisplaylinenum


\alalalfejezet{śarīraśaucam}
\varr{
        \ \va °nirdeśaṁ\lem  \mssCaCbCc\msNc\Ed; °niyamaṁ \msNa, °rddeśaṁ \msNb
        \ \vc śārīra°\lem  \mssCaCbCc\msNa\msNc\Ed; śarīra° \msNb\oo
                 °śaucam āhāro\lem  \msCb\msCc\msNa\msNb\msNc\Ed; °śauca{\il}hāro \msCa
        \ \vd mātrā bhāvaś ca\lem  \msCb\msCc\msNa\msNc\Ed; mātrā bhāvaṁ ca \msCa, \uncl{sātrābhā}vaś ca \msNb}

tāḍayen na ca bandheta na ca prāṇair viyojayet\thinspace{\dandab} \dontdisplaylinenum

parastrīparadravyeṣu śaucaṁ kāyikam ucyate \veg\dontdisplaylinenum
\varr{
        \ \va tāḍayen na\lem  \mssCaCbCc\msNa\msNb\Ed; tāḍaye na \msNc
        \ \vd śaucaṁ\lem  \mssCaCbCc\msNa\msNb\Ed; śauca \msNc\oo
                 kāyikam ucyate\lem  \mssCaCbCc\msNa\msNb\Ed; kāyikam umucyete \msNc}

śrotraśaucaṁ dvijaśreṣṭha gudopasthamukhādayaḥ\thinspace{\dandab} \dontdisplaylinenum

mukhasyācamanaṁ śaucam āhāravacaneṣu ca \veg\dontdisplaylinenum
\varr{
        \ \va śrotra°\lem  \eme; śrota° \mssCaCbCc\msNa\msNb\msNc\Ed
        \ \vb gudopastha°\lem  \mssCaCbCc\msNa\msNb; gudoprastha° \msNc, gudāpastha° \Ed
        \ \vc mukhasyā°\lem  \msCa\msCc\msNa\msNb\msNc\Ed; mukhasthā° \msCb
        \ \vcd śaucam ā°\lem  \msCa\msCc\msNa\msNc\Ed; śaucaṁm ā° \msCb\msNb}

mūtraviṣṭāsamutsarge devatārādhaneṣu ca\thinspace{\dandab} \dontdisplaylinenum

m\textsubring{r}ttoyais tu gudopasthaṁ śaucayīta vicakṣaṇaḥ \veg\dontdisplaylinenum
\varr{
        \ \va °viṣṭā°\lem  \mssCaCbCc\msNa\msNc\Ed; °viṣṭa° \msNb
        \ \vc m\textsubring{r}ttoyais tu\lem  \msCc\msNa\msNb\Ed; \uncl{m\textsubring{r}}{\il}{\il}{\il} \msCa, 
                                                      m\textsubring{r}toyais tu \msCb, m\textsubring{r}ttoyes tu \msNc\oo
                 °pasthaṁ\lem  \msCa\msCb\msNa\msNb\msNc; °pastha \msCc\Ed}

ekopasthe gude pañca tathaikatra kare daśa\thinspace{\dandab} \dontdisplaylinenum
            \paral{\textit{\vab {\normalfont  \kb\ Manu 5.136ab: } ekā liṅge gude tisras tathaikatra kare daśa}}

ubhayoḥ sapta dātavyā m\textsubring{r}daḥ śuddhiṁ samīhatā \veg\dontdisplaylinenum
            \paral{\textit{\vcd {\normalfont  \kb\ Manu 5.136cd: } ubhayoḥ sapta dātavyā m\textsubring{r}daḥ śuddhim abhīpsatā}}
\varr{
        \ \va °pasthe\lem  \msCa\msCb\msNa\msNc\Ed; °pastha° \msCc\msNb\oo
                 gude\lem  \msCa\msCb\msNa\msNc\Ed; °gudo \msCc\msNb
        \ \vb tathaikatra\lem  \msCa\msCc\msNa\msNb\msNc; tathaika\uncl{tra} \msCb, tathaikaś ca \Ed\oo
                 daśa\lem  \msCa\msCb\msNa\msNb\msNc\Ed; daśaḥ \msCc
        \ \vc dātavyā\lem  \msCa\msCb\msNa\msNb\msNc; dātavyo \msCc\Ed
        \ \vd m\textsubring{r}daḥ\lem  \mssCaCbCc\msNc\Ed; m\textsubring{r}taḥ \msNa, m\textsubring{r}dā \msNb\oo
                 śuddhiṁ samīhatā\lem  \msCa\msCb\msNa; śuddhisamīhayā \msCc, śu\uncl{ddhi} samīhatā \msNb,
                                                                śuddhiḥ samīhatā \msNc, śuddhiṁ samāhitā \Ed}

etac chaucaṁ g\textsubring{r}hasthānāṁ dviguṇaṁ brahmacāriṇām\thinspace{\dandab} \dontdisplaylinenum
            \paral{\textit{\vab {\normalfont  = Manu 5.137ab } }}

vānaprasthasya triguṇaṁ yatīnāṁ tu caturguṇam \veg\dontdisplaylinenum
            \paral{\textit{\vcd {\normalfont  \kb\ Manu 5.137cd: } triguṇaṁ syād vanasthānāṁ yatīnāṁ tu caturguṇam }}


\alalalfejezet{āhāraśaucam}
\varr{
        \ \va etac chaucaṁ\lem  \msCa\msCb\msNa\msNc; cetac hauca \msCc\Ed, eta{\il}{\il} \msNb
        \ \vb °guṇaṁ\lem  \msCa\msCb\msNa\msNb\msNc\Ed; °guṇa \msCc
        \ \vc tri°\lem  \msCa\msCb\msNa\msNb\msNc\Ed; dvi° \msCc}

āhāraśaucaṁ vakṣyāmi ś\textsubring{r}ṇuṣvāvahito bhava\thinspace{\dandab} \dontdisplaylinenum

bhāgadvayaṁ tu bhuñjīta bhāgam ekaṁ jalaṁ pibet \danda\dontdisplaylinenum

vāyusaṁcāradānārthaṁ caturtham avaśeṣayet \veg\dontdisplaylinenum
\varr{
        \ \vb ś\textsubring{r}ṇuṣvāvahito\lem  \msCb\msCc\msNa\msNc\Ed; ś\textsubring{r}ṇu\uncl{ṣvāva}{\il}{\il} \msCa, ś\textsubring{r}ṇuṣvavahito \msNb
        \ \vd pibet\lem  \msCa\msCc\msNa\msNb\msNc\Ed; pibe \msCb
        \ \ve °cāradānārthaṁ\lem  \mssCaCbCc\msNa\msNb\msNc; °cāraṇārthāya \Ed}

snigdhasvādurasaiḥ ṣaḍbhir āhāraṣaḍrasair budhaḥ\thinspace{\dandab} \dontdisplaylinenum

dhātuvaiṣamyanāśo 'sti na ca rogāḥ sudāruṇāḥ \veg\dontdisplaylinenum
\varr{
        \ \va °svādu°\lem  \mssCaCbCc\msNa\msNc; °svā{\il} \msNb, °svāda° \Ed
        \ \vb °ṣaḍrasair bu°\lem  \msCb\Ed; °sadravair bu° \msCa\msNa\msNc,
                                        °sadravai bu° \msCc, °ṣaḍrasai bu° \msNb
        \ \vc °vaiṣamyanāśo 'sti\lem  \msCa\msCc\msNa\msNb\msNc;
                      °\uncl{dai}ṣamyanāśāsti \msCb, °vaiṣamya naśyanti \Ed
        \ \vd sudāruṇāḥ\lem  \mssCaCbCc\msNa\msNb\msNc; sudāruṇaḥ \Ed}

abhakṣyaṁ ca na bhakṣeta apeyaṁ na ca pāyayet\thinspace{\dandab} \dontdisplaylinenum

agamyaṁ na ca gamyeta avācyaṁ na ca bhāṣayet \veg\dontdisplaylinenum
\varr{
        \ \va abhakṣyaṁ\lem  \mssCaCbCc\msNa\msNc; {\il}{\il}{\il} \msNb, abhakṣaṁ \Ed
        \ \vb na ca\lem  \mssCaCbCc\msNa\msNb; ca na \msNc\Ed
        \ \vd avācyaṁ\lem  \msCa\msCb\msNa\msNb\msNc\Ed; avācaṁ \msCc}

laśunaṁ ca palāṇḍuṁ ca g\textsubring{r}ñjanaṁ kacakāni ca\thinspace{\dandab} \dontdisplaylinenum
            \paral{\textit{{\normalfont \vab cf.\ Manu 5.5ab: } laśunaṁ g\textsubring{r}ñjanaṁ caiva palāṇḍuṁ kavakāni ca}}

gauraṁ ca śūkaraṁ māṁsaṁ varjayec ca vidhānataḥ \veg\dontdisplaylinenum
\varr{
        \ \va palāṇḍuṁ\lem  \Ed; palaṇḍuṁ \mssCaCbCc\msNb\msNc, palaḍuṁ \msNa
        \ \vb kavakāni\lem  \mssCaCbCc\msNa\msNb\msNc; ca kacāni \Ed
        \ \vc gauraṁ ca\lem  \eme; gorasva \msCa\msNb, goraś ca \msCb\msCc\msNa\msNc, gauraś ca \Ed\oo
                 māṁsaṁ\lem  \mssCaCbCc\msNa\msNb\msNc; māsaṁ \Ed}

chattrākaṁ viḍvarāhaṁ ca gomāṁsaṁ ca na bhakṣayet\thinspace{\dandab} \dontdisplaylinenum
            \paral{\textit{\vab {\normalfont Cf. Manu 5.19ab: } chatrākaṁ viḍvarāhaṁ ca laśunaṁ grāmakukkuṭam}}

caṭakaṁ ca kapotaṁ ca jālapādāṁś ca varjayet \veg\dontdisplaylinenum
\varr{
        \ \va chattrākaṁ\lem  \msNa\msCa\msCb\msNb\msNc\Ed; chattrāka \msCc\oo
                 viḍva°\lem  \mssCaCbCc\msNb\Ed; vidva° \msNa\msNc
        \ \vb gomāṁsaṁ\lem  \msNa\msCa\msCbpcorr\msCc\msNb\msNc\Ed; gomāñ \msCbacorr
        \ \vc caṭakaṁ\lem  \msCa\msCb\msNa\msNc\Ed; caṭakām \msCc\msNb}

haṁsasārasacakrāhvakukkuṭān śukaśyenakān\thinspace{\dandab} \dontdisplaylinenum

kākolūkaṁ balākaṁ ca matsyādīṁś cāpi varjayet \veg\dontdisplaylinenum
\varr{
        \ \vb °kukkuṭān śu°\lem  \mssCaCbCc\msNc\Ed; °kukkuṭā śu° \msNa, °kukkuṭāṁ śu° \msNb\oo
                 °śyenakān\lem  \msCa\msCc\msNc\Ed; °śonakān \msCb, °śyenakā \msNa, °śyenakāṁ \msNb
        \ \vc kākolūkaṁ balākaṁ ca\lem  \msCb\msNc; kākolūka\uncl{sva}{\il}{\il}ñ ca \msCa,
                                                        kākolūkabalākaṁ ca \msCc\msNa\Ed, 
                                                        \uncl{kākolūkaṁ balākaṁ ca} \msNb}

amedhyāṁś cāpavitrāṁś ca sarvān eva vivarjayet\thinspace{\dandab} \dontdisplaylinenum

śākamūlaphalānāṁ ca abhakṣyaṁ parivarjayet \veg\dontdisplaylinenum
\varr{
        \ \va amedhyāṁś cā°\lem  \mssCaCbCc\msNa\msNc; \uncl{amedhyāś cā°} \msNb, amedhyaś cā° \Ed}

mānaveṣu purāṇeṣu śaivabhāratasaṁhite\thinspace{\dandab} \dontdisplaylinenum

kīrtitāni viśeṣeṇa śaucācāram aśeṣataḥ \veg\dontdisplaylinenum

tvayā jijñāsito 'smy adya saṁkṣiptaḥ kathito mayā\thinspace{\dandab} \dontdisplaylinenum

satyavādī śucir nityaṁ dhyānayogarataḥ śuciḥ \veg\dontdisplaylinenum
\varr{
        \ \va jijñāsito\lem  \mssCaCbCc\msNa\msNb; jijñāsano \msNc, jijñāsato \Ed
        \ \vb °kṣiptaḥ\lem  \msCa\msCc\msNa\msNc\Ed; °kṣipya \msCb, °kṣipta \msNb\oo
                  kathito\lem  \mssCaCbCc\msNa\msNb\msNc; kathitaṁ \Ed
        \ \vc śucir\lem  \msCa\msCb\Ed; śuci \msCc\msNc, śucin \msNa\msNb}

ahiṁsakaḥ śucir dānto dayābhūtakṣamā śuciḥ\thinspace{\dandab} \dontdisplaylinenum

sarveṣām eva śaucānām arthaśaucaṁ paraṁ sm\textsubring{r}tam \veg\dontdisplaylinenum
            \paral{\textit{\vcd {\normalfont  = Manu 5.106ab}}}
\varr{
        \ \va ahiṁsakaḥ\lem  \msCa\msCc\msNa\msNb\msNc\Ed; ahiṁsaka \msCb\oo 
                 śucir dānto\lem  \msCa\msCb\msNa\msNb\Ed; śuci dānto \msCc\msNc, śucir dāntau \Ed
        \ \vd °śaucaṁ paraṁ sm\textsubring{r}tam\lem  \msCa\msNa\msNb\msNc; °śaucaṁ para sm\textsubring{r}tam \msCb\msCc,
                                                      °śaucayanaṁ sm\textsubring{r}taḥ \Ed}

yo 'rthe hi śuciḥ sa śucir na m\textsubring{r}dvāriśuciḥ śuciḥ\thinspace{\dandab} \dontdisplaylinenum
            \paral{\textit{\vab {\normalfont \kb\ Manu 5.106cd: } yo 'rthe śucir hi sa śucir na m\textsubring{r}dvāriśuciḥ śuciḥ}}

kāyavāṅmanasāṁ śaucaṁ sa śuciḥ sarvavastuṣu \veg\dontdisplaylinenum
            \paral{\textit{\vcd {\normalfont  \Ed\ adds here, after pādas cd: } śaucāśaucavidhir jñātvā mucyate sarvakilbiṣāt}}
\varr{
        \ \vab yo 'rthe hi śuciḥ sa śucir na\lem  \mssCaCbCc\msNc\ \unmetr; 
                                yo 'rthe hi śuciḥ sa śuci na \msNa\msNb, yo 'rthe hi suśucir vipra na \Ed
        \ \vb °śuciḥ śuciḥ\lem  \mssCaCbCc\msNa\msNc; śuci śuciḥ \msNb, °śuciḥ śuci \Ed
        \ \vd śuciḥ\lem  \msCa\msCb\msNa\msNb\msNc\Ed; śuci \msCc\oo
                 vastuṣu\lem  \mssCaCbCc\msNa\msNb\Ed; vastuṣuḥ \msNc}

\ujvers\nemsloka 
śaucāśaucavidhijña mānava yadi kālakṣaye niścayaḥ
\dontdisplaylinenum

\nemslokab 
saubhāgyatvam avāpnuvanti satataṁ kīrtir yaśo'laṅk\textsubring{r}taḥ \danda\dontdisplaylinenum

\nemslokac 
prāptaṁ tena ihaiva puṇyasakalaṁ saddharmaśāstreritam
\dontdisplaylinenum

\nemslokad 
jīvānte ca paratra-m-īhitagatiṁ prāpnoti niḥsaṁśayam \veg\dontdisplaylinenum

\vers


\jump
\begin{center}
\ketdanda iti v\textsubring{r}ṣasārasaṁgrahe śaucācāravidhir nāmādhyāyaḥ pañcamaḥ\ketdanda
\end{center}
\dontdisplaylinenum\vers 
\varr{
        \ \va śaucāśauca°\lem  \msCa\msCc\msNa\msNb\msNc\Ed; śaucāśuca \msCb\oo
                 kālakṣaye niścayaḥ\lem  \msNaacorr\msNc;
                      kālakṣayair niścayaḥ \msCa\msCb\msNapcorr,
                      kālakṣayen niścayaḥ \msCc\msNb,
                      kālakṣayetiś ca yaḥ \Ed
        \ \vb kīrtir ya°\lem  \msCb\msNa\msNb\msNc\Ed; kīrtiya° \msCa\msCc \unmetr\oo
                °laṅk\textsubring{r}taḥ\lem  \msCa\msCc\msNa\msNb\msNc\Ed; °lak\textsubring{r}taḥ \msCb
        \ \vc °eritam\lem  \mssCaCbCc\msNa\msNb\msNc; °oditaḥ \Ed
        \ \vd paratra-m-ī°\lem  \mssCaCbCc\msNa\msNb\msNc; pavitram ī° \Ed\oo
                 °gatiṁ\lem  \eme; °gatiḥ \mssCaCbCc\msNa\msNb\msNc\Ed\oo
                 niḥsaṁśayam\lem  \msCa\msNb\msNc; niḥsaṁśayaḥ \msCb\msCc\msNa\Ed
\ {\normalfont \Colo:} °vidhir\lem  \msCa\Ed; °vidhi° \msCb\msCc\msNa\msNc, \uncl{viṁdhi} \msNb\oo
                        nāmādhyayaḥ pañcamaḥ\lem  \mssCaCbCc\msNa\msNb\msNc; 
                                                nāma pañcamo 'dhyāyaḥ \Ed}
\bekveg\szamveg\vfill\phpspagebreak\szam\bek\versno=0\fejno=6
\thispagestyle{empty}



\alfejezet{\textbf{ṣaṣṭho 'dhyāyaḥ}}\jump\jump

\alalfejezet{niyameṣu ijyā (2)}
\vers

[anarthayajña uvāca~{\dandab}\dontdisplaylinenum ]

atha pañcavidhām ijyāṁ pravakṣyāmi dvijottama\thinspace{\danda} \dontdisplaylinenum

dharmamokṣaprasiddhyarthaṁ ś\textsubring{r}ṇuṣvāvahito dvija \veg\dontdisplaylinenum
\varr{
        \ \va °m ijyāṁ\lem  \msCb; °m ījyāṁ \msCa\msCc\msNa\msNb\msNc\Ed
        \ \vb °ttama\lem  \mssCaCbCc\msNa\Ed; °ttamaḥ \msNb\msNc
        \ \vc °mokṣaprasiddhyarthaṁ\lem  \mssCaCbCc\msNc; °mokṣaprasiddhyartha \msNa\msNb, 
                                                        °mokṣeśasiddhyaarthaṁ \Ed
        \ \vd dvija\lem  \mssCaCbCc\msNa\msNb\msNc; bhava \Ed}

arthayajñaḥ kriyāyajño japayajñas tathaiva ca\thinspace{\dandab} \dontdisplaylinenum

jñānaṁ dhyānaṁ ca pañcaitat pravakṣyāmi p\textsubring{r}thak p\textsubring{r}thak \veg\dontdisplaylinenum


\alalalfejezet{arthayajñaḥ}
\varr{
        \ \va arthayajñaḥ\lem  \msCa\msCc\msNa; anarthayajñaḥ \msCb, arthayajña \msNb\msNc, arthayajña° \Ed
        \ \vc jñānaṁ\lem  \msCa\msCb\msNa\msNb\Ed; jñāna \msCc\msNc}

agnyupāsanakarmādi agnihotrakratukriyā\thinspace{\dandab} \dontdisplaylinenum

aṣṭakāḥ pārvaṇī śrāddhaṁ dravyayajñaḥ sa ucyate \veg\dontdisplaylinenum


\alalalfejezet{kriyāyajñaḥ}
\varr{
        \ \vb agni°\lem  \msCb\msCc\msNa\msNc\Ed; \uncl{a}{\lost}° \msCa, {\il}{\il} \msNb\oo
                 °kriyā\lem  \msCa\msNa\msNb\msNc\Ed; °kriyāḥ \msCb\msCc
        \ \vc aṣṭakāḥ\lem  \eme; aṣṭakā \mssCaCbCc\msNa\msNb\msNc\Ed\oo
                 pārvaṇī\lem  \msCa\msCc\msNa\msNc\Ed; parvaṇī \msCb, \uncl{parvaṇī} \msNb
        \ \vd °yajñaḥ\lem  \msCa\msCb\msNa\msNc\Ed; °yajña \msCc, {\il}{\il} \msNb}

ārāmodyānavāpīṣu devatāyataneṣu ca\thinspace{\dandab} \dontdisplaylinenum

svahastak\textsubring{r}tasaṁskāraḥ kriyāyajña sa ucyate \veg\dontdisplaylinenum


\alalalfejezet{japayajñaḥ}
\varr{
        \ \vb °yataneṣu\lem  \msCb\msCc\Ed; °layaneṣu \msCa\msNa\msNc, °yata{\il}{\il} \msNb
        \ \vc °hasta°\lem  \mssCaCbCc\msNa\msNc; {\il}{\il} \msNb, °hastaiḥ \Ed}

japayajñaṁ tato vakṣye svargamokṣaphalapradam\thinspace{\dandab} \dontdisplaylinenum

vedādhyayana kartavyaṁ śivasaṁhitam eva ca \veg\dontdisplaylinenum
\varr{
        \ \va °yajñaṁ tato\lem  \msCa\msNa\msNb\msNc\Ed; °yajñaṁ tapo \msCb °yajñas tato \msCc
        \ \vc vedā°\lem  \mssCaCbCc\msNa\msNc\Ed; adā° \msNb}

itihāsapurāṇaṁ ca japayajñaḥ sa ucyate\thinspace{\dandab} \dontdisplaylinenum


\alalalfejezet{jñānayajñaḥ}

idaṁ karma akarmedam ūhāpohaviśāradaḥ \veg\dontdisplaylinenum
\varr{
        \ \va °purāṇaṁ ca\lem  \mssCaCbCc\msNa\msNb\msNc; °purāṇaś ca \Ed
        \ \vb °yajñaḥ\lem  \msCa\msCb\msNa\msNb\msNc\Ed; °yajña \msCc
        \ \vc karma\lem  \mssCaCbCc\msNa\msNb\msNc; kramam \Ed}

śāstracakṣuḥ samālokya jñānayajñaḥ sa ucyate\thinspace{\dandab} \dontdisplaylinenum

dhyānayajñaṁ samāsena kathayiṣyāmi te ś\textsubring{r}ṇu \veg\dontdisplaylinenum


\alalalfejezet{dhyānayajñaḥ}
\varr{
        \ \va °cakṣuḥ\lem  \msCa\msCb\msNa\msNb\msNc\Ed; °cakṣu \msCc
        \ \vb °yajñaḥ\lem  \msCa\msCb\msNa\msNc\Ed; °yajña \msCc, °\uncl{yajñas} \msNb
        \ \vc °yajñaṁ\lem  \msCa\msCb\msNa\msNc\Ed; °yajña \msCc\msNb}

dhyānaṁ pañcavidhaṁ caiva kīrtitaṁ hariṇā purā\thinspace{\dandab} \dontdisplaylinenum

sūryaḥ somo 'gni sphaṭikaḥ sūkṣmaṁ tattvaṁ ca pañcamam \veg\dontdisplaylinenum
\varr{
        \ \va dhyānaṁ\lem  \mssCaCbCc\msNb\Ed; dhyāna \msNa\msNc
        \ \vc somo\lem  \msCa\msCc\msNa\msNc; somā° \msCb\msNb\Ed
        \ \vd sūkṣmaṁ tattvaṁ ca pañcamam\lem  \msCb;
                                sūkṣmaṁ ta\uncl{tva}{\lost}{\lost}{\lost}ñcamam \msCa, 
                                sūkṣmatattvaṁ ca pañcamaḥ \msCc\msNa\msNb,
                                sūkṣmaṁ tattvañ ca pañcamaḥ \msNc,
                                sūkṣmāṁ tattvaś ca pañcamam \Ed}

sūryamaṇḍalam ādau tu tattvaṁ prak\textsubring{r}tir ucyate\thinspace{\dandab} \dontdisplaylinenum

tasya madhye śaśiṁ dhyāyet tattvaṁ puruṣa ucyate \veg\dontdisplaylinenum
\varr{
        \ \vc śaśiṁ\lem  \mssCaCbCc\msNa\Ed; śaśi \msNb, śaśiṁn \msNc
        \ \vcd dhyāyet ta°\lem  \msCa\msCb\msNa\msNb\msNc\Ed; dhyāye ta° \msCc}

candramaṇḍalamadhye tu jvālām agniṁ vicintayet\thinspace{\dandab} \dontdisplaylinenum

prabhutattvaḥ sa vijñeyo janmam\textsubring{r}tyuvināśanaḥ \veg\dontdisplaylinenum
\varr{
        \ \vb jvālām agniṁ\lem  \mssCaCbCc\msNa\msNb\Ed; jālām agni \msNc
        \ \vc °tattvaḥ\lem  \mssCaCbCc\msNc; °tatva \msNa, °tatvaṁ \msNb\Ed
        \ \vd °nāśanaḥ\lem  \msCa\msCb\msNa\msNb\msNc; °nāśanam \msCc\Ed}

agnimaṇḍalamadhye tu dhyāyet sphaṭika nirmalam\thinspace{\dandab} \dontdisplaylinenum

vidyātattvaḥ sa vijñeyaḥ kāraṇam ajam avyayam \veg\dontdisplaylinenum
\varr{
        \ \vb dhyāyet sphaṭika\lem  \msCapcorr\msCb\msNa\msNb\msNc; dhyāyet sphaṭi \msCaacorr,
                                        dhyāye sphaṭika \msCc\Ed\oo
                 °malam\lem  \mssCaCbCc\msNb\Ed; °malaḥ \msNa, \uncl{°malaḥ} \msNc
        \ \vc tattvaḥ sa\lem  \msCb\msNa\msNb\msNc; ta\uncl{tvan}{\lost} \msCa, tatva sa \msCc, tatvaṁ sa \Ed
        \ \vd °jam avyayam\lem  \msCa\msCb\msNa\msNb\msNc\Ed; °m avyayaṁ \msCc}

vidyāmaṇḍalamadhye tu dhyāyet tattvam anuttamam\thinspace{\dandab} \dontdisplaylinenum

akīrtitam anaupamyaṁ śivam akṣayam avyayam \danda\dontdisplaylinenum

pañcamaṁ dhyānayajñasya tattvam uktaṁ samāsataḥ \veg\dontdisplaylinenum
\varr{
        \ \vab dhyāyet ta°\lem  \msCa\msCb\msNa\msNb\msNc\Ed; dhyāye ta° \msCc
        \ \vc °yajñasya\lem  \msCa\msCb\msNa\msNb\msNc; °yajñañ ca \msCc\Ed
        \ \vd samāsataḥ\lem  \mssCaCbCc\msNa\msNb\msNc; sanātanaḥ \Ed}

vigatarāga uvāca~{\dandab}\dontdisplaylinenum 

ekaikasya hi tattvasya phalaṁ kīrtaya kīd\textsubring{r}śam\thinspace{\danda} \dontdisplaylinenum

kāni lokāḥ prapadyante kālaṁ vāsya tapodhana \veg\dontdisplaylinenum
\varr{
        \ \va hi\lem  \Ed; tri° \mssCaCbCc\msNa\msNb\msNc
        \ \vc lokāḥ\lem  \msCa\msNa\msNc; lokā \msCb\msCc\msNb\Ed\oo
                 prapadyante\lem  \msCb\msCc\msNa\msNb\msNc\Ed; pra{\il}{\il}{\il} \msCa
        \ \vd °dhana\lem  \msCa\msCc\msNa\msNb\Ed; °dhanaḥ \msCb\msNc}

anarthayajña uvāca~{\dandab}\dontdisplaylinenum 

brahmalokaṁ tu prathamaṁ tattvaṁ prak\textsubring{r}ticintayā\thinspace{\danda} \dontdisplaylinenum

kalpakoṭisahasrāṇi śivavan modate sukhī \veg\dontdisplaylinenum
\varr{
        \ \vab prathamaṁ tattvaṁ\lem  \Ed; prathamaṁ tatva \mssCaCbCc\msNapcorr\msNb\msNc, 
                                                                    \om\ \msNaacorr\oo
                  prak\textsubring{r}ticintayā\lem  \mssCaCbCc\msNa\msNb\msNc; ca k\textsubring{r}ticintaya \Ed
        \ \vd sukhī\lem  \mssCaCbCc\msNa\msNb\msNc; sukham \Ed}

dvitīyaṁ tattva puruṣaṁ dhyāyamāno m\textsubring{r}to yadi\thinspace{\dandab} \dontdisplaylinenum

viṣṇulokam ito yāti kalpakoṭyayutaṁ sukhī \veg\dontdisplaylinenum
\varr{
        \ \vc yāti\lem  \mssCaCbCc\msNa\msNb\msNc; yānti \Ed}

prabhutattvaṁ t\textsubring{r}tīyaṁ tu dhyāyamāno mariṣyati\thinspace{\dandab} \dontdisplaylinenum

śivaloke vasen nityaṁ kalpakotyayutaṁ śatam \veg\dontdisplaylinenum
\varr{
        \ \va °tattvaṁ\lem  \msCa\msCb\msNa\msNb\msNc\Ed; °tatva \msCc\oo
                 t\textsubring{r}tīyaṁ\lem  \mssCaCbCc\msNa\msNb\msNc; t\textsubring{r}tīyas \Ed
        \ \vb dhyāyamāno mariṣyati\lem  \msCb\msCc\msNa\msNb\msNc; dhyāya{\il}{\il}{\il}riṣyati \msCa,
                                                        dhayāyāmāno mariṣyati \Ed
        \ \vc śivaloke\lem  \msCa\msCc\msNa\msNb\msNc; śivaloka \msCb, rudraloke \Ed\oo
                 vasen ni°\lem  \msCa\msCb\msNa\msNb\msNc\Ed; vase ni° \msCc 
        \ \vd °yutaṁ\lem  \mssCaCbCc\msNa\msNc\Ed; °yuta \msNb}

vidyātattvām\textsubring{r}taṁ dhyāyet sadāśivam anāmayam\thinspace{\dandab} \dontdisplaylinenum

akṣayaṁ lokam āpnoti kalpānāntaparaṁ tathā \veg\dontdisplaylinenum 
\varr{
        \ \va °tattvām\textsubring{r}taṁ\lem  \msCa\msCb\msNa\msNb\msNc; °tatvam\textsubring{r}tan \msCc, °tattvāmataṁ \Ed
        \ \vc akṣayaṁ\lem  \mssCaCbCc\msNa\msNb\msNc; akṣaya° \Ed}

pañcamaṁ śivatattvaṁ tu sūkṣmaṁ cātmani saṁsthitam\thinspace{\dandab} \dontdisplaylinenum

na kālasaṁkhyā tatrāsti śivena saha modate \veg\dontdisplaylinenum

\ujvers\nemsloka 
pañcadhyānābhiyukto bhavati ca na punarjanmasaṁskārabandhaḥ
\dontdisplaylinenum

\nemslokab 
jijñāsyantāṁ dvijendra bhavadahanakaraḥ prārthanākalpav\textsubring{r}kṣaḥ \danda\dontdisplaylinenum

\nemslokac 
janmenaikena muktir bhavati kimu na vā mānavāḥ sādhayantu
\dontdisplaylinenum

\nemslokad 
pratyakṣān nānumānaṁ sakalamalaharaṁ svātmasaṁvedanīyam \veg\dontdisplaylinenum

\vers


\alalfejezet{niyameṣu tapaḥ (3)}\varr{
        \ \va °yukto\lem  \msCb\msCc\msNa\msNb\msNc; °yu{\il} \msCa\ \toplost, °yuktau \Ed\oo
                 ca\lem  \msCa\msCc\msNa\msNb\msNc; \om\ \msCb\Ed\oo
        punarjanma°\lem  \msCb\msCc\msNa\msNb\msNc\Ed; puna\uncl{ja}nma° \msCa\ \toplost, punajanma° \msCc
        \ \vb jijñāsyantāṁ\lem  \msCa\msNb\msNc\Ed; jijñāsyatāṁ \msCb\msNa\ \unmetr, jijñāsyantā \msCc
        \ \vc janmenaikena\lem  \msCb\msNb\msNc\Ed; janmanaikena \msCa\msCc\msNa\ \unmetr\oo
                 muktir bh°\lem  \msCa\msCb\msNa\msNb\msNc\Ed; mukti bh° \msCc\oo
                 na vā\lem  \mssCaCbCc\msNb\msNc\Ed; bhavā \msNa\oo
                 mānavāḥ\lem  \msCa\msNa\msNb\msNc; mānamānavāḥ \msCb, mānavā \msCc, mānava \Ed
        \ \vd pratyakṣā°\lem  \mssCaCbCc\msNb\msNc\Ed; pratyakṣa° \msNa\oo
                 °vedanīyam\lem  \msCb\msNa\msNb; °vedanīyaḥ \msCa\msCc\msNc, °vedanīya \Ed}

mānasaṁ tapa ādau tu dvitīyaṁ vācikaṁ tapaḥ\thinspace{\dandab} \dontdisplaylinenum

kāyikaṁ ca t\textsubring{r}tīyaṁ tu manovākkarma tatparam \danda\dontdisplaylinenum

kāyikaṁ vācikaṁ caiva tapo miśraka pañcamam \veg\dontdisplaylinenum
\varr{
        \ \va °tapa\lem  \mssCaCbCc\msNa\msNb\msNc; °tapam \Ed
        \ \vc kāyikaṁ ca t\textsubring{r}tīyaṁ tu\lem  \mssCaCbCc\msNa\msNc\Ed; mānasaṁ tapa ādau tu \msNb\ {\normalfont (eyeskip)}
        \ \vd manovākkarma\lem  \msCa\msNc\Ed; manokkarma \msCb, mmanovākarma° \msCc, manovākkāya° \msNa\msNb\oo
                 °param\lem  \msCc; °paraḥ \msCa\msCb\msNa\msNb\msNc\Ed
        \ \vc kāyikaṁ\lem  \mssCaCbCc\msNb\msNc\Ed; kāyika \msNa}

manaḥsaumyaṁ prasādaś ca ātmanigraham eva ca\thinspace{\dandab} \dontdisplaylinenum

maunaṁ bhāvaviśuddhiś ca pañcaitat tapa mānasam \veg\dontdisplaylinenum
\varr{
        \ \va °saumyaṁ\lem  \msNc; °saumya° \msCa\msCb\msNa\msNb\Ed, °saum\uncl{ya}° \msCc\ \toplost\oo 
                 prasādaś ca\lem  \msCa\msCc\msNa\msNc; prasādaṁ ca \msCb\Ed, pradānaś ca \msNb
        \ \vc maunaṁ\lem  \mssCaCbCc\msNa\msNb\msNc; mauna{\il} \Ed\oo
                 °śuddhiś ca\lem  \msCa\msCb\msNa\msNb\msNc; °śuddhiṁ ca \msCc\Ed
        \ \vd pañcaitat\lem  \msCa\msNb\msNc; pañcaite \msCb\msNa, pañcetat \msCc, pañcaitan \Ed}

anudvegakarā vāṇī priyaṁ satyaṁ hitaṁ ca yat\thinspace{\dandab} \dontdisplaylinenum

svādhyāyābhyasanaṁ caiva vācikaṁ tapa ucyate \veg\dontdisplaylinenum
            \paral{\textit{\vcd {\normalfont  \kb\ MBh 6.39.15cd: } svādhyāyābhyasanaṁ caiva vāṅmayaṁ tapa ucyate}}
\varr{
        \ \vc °bhyasanaṁ caiva\lem  \msCb\msCc\msNa\msNc\Ed; °bhyasana{\il}{\il} \msCa, °bhyasa\uncl{naṁ} caiva \msNb}

ārjavaṁ ca ahiṁsā ca brahmacaryaṁ surārcanam\thinspace{\dandab} \dontdisplaylinenum

śaucaṁ pañcamam ity etat kāyikaṁ tapa ucayate \veg\dontdisplaylinenum
\varr{
        \ \va ārjavaṁ ca ahiṁsā ca\lem  \mssCaCbCc\msNa\msNb\msNc;
                                        ārjavatvam ahiṁsāś ca \Ed
        \ \vb °caryaṁ\lem  \msCa\msCb\msNa\msNb\msNc; °carya \msCc\Ed
        \ \vc śaucaṁ\lem  \mssCaCbCc\msNa\msNb\msNc; śauca \Ed}

iṣṭaṁ kalyāṇabhāvaṁ ca dhanyaṁ pathyaṁ hitaṁ vadet\thinspace{\dandab} \dontdisplaylinenum
            \paral{\textit{\vc {\normalfont MBh 5.145.6: } 
                mayā nāgapuraṁ gatvā sabhāyāṁ dh\textsubring{r}tarāṣṭrajaḥ{\thinspace\danda}
                tathyaṁ pathyaṁ hitaṁ cokto na ca g\textsubring{r}hṇāti durmatiḥ{\thinspace\ketdanda}}}

manomiśraka pañcaitat tapa uktaṁ maharṣibhiḥ \veg\dontdisplaylinenum
\varr{
        \ \va iṣṭaṁ\lem  \msCa\msCb\msNa\msNc\Ed; iṣṭa \msCc\msNb\oo
                 °bhāvaṁ\lem  \mssCaCbCc\msNa\msNb\msNc; °bhāvaś \Ed
        \ \vb pathyaṁ\lem  \mssCaCbCc\msNa\msNb\msNc; satyaṁ \Ed
        \ \vc mano°\lem  \mssCaCbCc\msNa\msNb\msNc; mana° \Ed\oo
                 pañcaitat\lem  \mssCaCbCc\msNa\msNb; pañcetat \msNc, pañcaitān \Ed
        \ \vd tapa uktaṁ maharṣibhiḥ\lem  \mssCaCbCc\msNa\msNb\msNc; tapam uktaṁ mahirṣibhiḥ \Ed}

svastimaṅgalam āśīrbhir atithigurupūjanam\thinspace{\dandab} \dontdisplaylinenum

kāyamiśraka pañcaitat tapa uktaṁ mahātmabhiḥ \veg\dontdisplaylinenum
\varr{
        \ \va °śīrbhi°\lem  \msCa\Ed; °śībhi° \msCb\msCc\msNa\msNb\msNc
        \ \vb °tithi°\lem  \mssCaCbCc\msNa\msNb\msNc; °tithiṁ \Ed
        \ \vc °miśraka\lem  \msCc\msNa\msNb\msNc\Ed; °{\il}{\il}ka \msCa, °mityaśraka \msCb\oo
                 pañcaitat\lem  \mssCaCbCc\msNa\msNb\msNc; pañcaitan \Ed
        \ \vd tapa uktaṁ\lem  \mssCaCbCc\msNa\msNb\msNc; tapam uktaṁ \Ed}

maṇḍūkayogī hemante grīṣme pañcatapās tathā\thinspace{\dandab} \dontdisplaylinenum
            \paral{\textit{\vab {\normalfont See MBh and Lalitavistara passages in Umā Playful article (p.627): }
               maṇḍūkayogī hemante grīṣmapañcā tapās bhavet {\normalfont ...
                also Umā 6.26ab :}  maṇḍūkayogo hemante grīṣme pañcatapāstathā{\thinspace\danda}}}

abhrāvakāśe varṣāsu tapaḥ sādhanam ucyate \veg\dontdisplaylinenum
\varr{
        \ \vb grīṣme\lem  \mssCaCbCc\msNa\msNb\msNc; g\textsubring{r}ṣme \Ed
        \ \vd tapaḥ\lem  \msCa\msCb\msNa\msNb\msNc\Ed; tapa \msCc\oo
                 sādhanam u°\lem  \msCa\msNa\msNc\Ed; sādhana u° \msCb\msCc\msNb}

svamāṁsoddh\textsubring{r}tya dānaṁ ca hastapādaśiras tathā\thinspace{\dandab} \dontdisplaylinenum
            \paral{\textit{\vab {\normalfont Cf. 2.38 and 17.xx ff. }}}

puṣpam utpādya dānaṁ ca sarve te tapa sādhanāḥ \veg\dontdisplaylinenum
\varr{
        \ \va dānaṁ\lem  \mssCaCbCc\msNa\msNc; \uncl{dāna} \msNb\ \toplost, dānaś \Ed
        \ \vc dānaṁ\lem  \mssCaCbCc\msNa\msNb\msNc; dānaś \Ed
        \ \vd tapa\lem  \Ed; tapaḥ \mssCaCbCc\msNa\msNb\msNc\ \unmetr}

k\textsubring{r}cchrātik\textsubring{r}cchraṁ naktaṁ ca taptak\textsubring{r}cchram ayācitam\thinspace{\dandab} \dontdisplaylinenum

cāndrāyaṇaṁ parākaṁ ca tapaḥ sāṁtapanādayaḥ \veg\dontdisplaylinenum
\varr{
        \ \va k\textsubring{r}cchrātik\textsubring{r}cchraṁ\lem  \msCa\msCb\msNa\Ed; k\textsubring{r}cchrādik\textsubring{r}cchra \msCc, k\textsubring{r}cchrātik\textsubring{r}cchra \msNb, k\textsubring{r}cchātik\textsubring{r}cchaṁ \msNc
        \ \vb °yācitam\lem  \mssCaCbCc\msNa\msNb\msNc; °yācitaḥ \Ed
        \ \vc cāndrāyaṇaṁ parākaṁ\lem  \msCa\msCc\msNb\msNc; cāndrāyanaṁ parākaṁ \msCb, 
                                                        candrāyaṇaṁ parākaṁ \msNa, cāndrāyaṇavarākaś \Ed
        \ \vd tapaḥ sāṁtapanādayaḥ\lem  \msCa\msCb\msNa\msNb\msNc; tapasāntapanādayaḥ \msCc\Ed}

\ujvers\nemsloka 
yenedaṁ tapa tapyate sumanasā saṁsāraduḥkhacchidam
\dontdisplaylinenum

\nemslokab 
āśāpāśa vimucya nirmalamatis tyaktvā jaghanyaṁ phalam \danda\dontdisplaylinenum

\nemslokac 
svargākāṅkṣyan\textsubring{r}patvabhogaviṣayaṁ sarvāntikaṁ tat phalam
\dontdisplaylinenum

\nemslokad 
jantuḥ śāśvatajanmam\textsubring{r}tyubhavane tanniṣṭhasādhyaṁ vahet \veg\dontdisplaylinenum

\vers


\jump
\begin{center}
\ketdanda iti v\textsubring{r}ṣasārasaṁgrahe ṣaṣṭho 'dhyāyaḥ\ketdanda
\end{center}
\dontdisplaylinenum\vers 
\varr{
        \ \va tapa ta°\lem  \Ed; tapas ta° \mssCaCbCc\msNa\msNb\msNc\ \unmetr\oo
                 °manasā\lem  \eme; °manasaḥ \mssCaCbCc\msNa\msNb\msNc\Ed
        \ \vb nirmalamati°\lem  \msCa\msCc\msNa\msNb\msNc\Ed; nirmalarmati° \msCb\oo
                 jaghanyaṁ\lem  \mssCaCbCc\msNa\msNb\msNc; jagat yaṁ \Ed
        \ \vc °kāṅkṣya°\lem  \mssCaCbCc\msNa\msNb\msNc; °kāṁkṣa° \Ed\oo
                 sarvāntikaṁ\lem  \msCa\msCc\msNa\msNb\msNc\Ed; sarvārttikaṁ \msCb
        \ \vd °bhavane\lem  \mssCaCbCc\msNa\msNb\Ed; °bhavene \msNc\oo
                 °sādhyaṁ vahet\lem  \msCc\msNa\msNb\msNc; °\uncl{sādhyam}{\il}{\il} \msCa, 
                                                  °sādhya vahet \msCb, °sādhyaṁ vadet \Ed}
\bekveg\szamveg\vfill\phpspagebreak\szam\bek\versno=0\fejno=7
\thispagestyle{empty}


\vers


\alfejezet{\textbf{saptamo 'dhyāyaḥ}}\jump\jump

\alalfejezet{niyameṣu dānam (4)}
dānāni ca tathety āhuḥ pañcadhā munibhiḥ purā\thinspace{\dandab} \dontdisplaylinenum

annaṁ vastraṁ hiraṇyaṁ ca bhūmi godāna pañcamam \veg\dontdisplaylinenum


\alalalfejezet{annadānam}
\varr{
        \ \va tathety āhuḥ\lem  \msCa\msCc\msNb\msNc\Ed; tathaity āhuḥ \msCb\msNa
        \ \vc vastraṁ\lem  \msCa\msCb\msNa\msNc\Ed; vastra \msCc\msNb}

annāt tejaḥ sm\textsubring{r}tiḥ prāṇaḥ annāt puṣṭir vapuḥ sukham\thinspace{\dandab} \dontdisplaylinenum

annāc chrīḥ kānti vīryaṁ ca annāt sattvaṁ ca jāyate \veg\dontdisplaylinenum
\varr{
        \ \va annāt tejaḥ sm\textsubring{r}tiḥ prāṇaḥ\lem  \mssCaCbCc\msNapcorr\msNb; annāt tejaḥ sm\textsubring{r}tiḥ prāṇa \msNaacorr, 
                                                        annāt tejaḥ sm\textsubring{r}ti prāṇaḥ \msNc,
                                                        annād bhavanti bhūtāni \Ed
        \ \vc annāc chrīḥ\lem  \mssCaCbCc\msNa\msNc; annāc chrī \msNb\Ed\oo
                 kānti vīryaṁ ca\lem  \msCb\msCc\msNa\msNb\Ed; kāntir vīryañ ca \msCa\msNc\ \unmetr, 
                                                                        kāntivīrśyañ ca \Ed
        \ \vd annāt sattvaṁ ca\lem  \msCa\msCb\msNa\msNb\msNc; annā satvañ ca \msCc, annāt sattvaś ca \Ed\oo
                 jāyate\lem  \msCb\msCc\msNa\msNb\msNc\Ed; jāya{\il} \msCa}

annāj jīvanti bhūtāni annaṁ tuṣṭikaraṁ sadā\thinspace{\dandab} \dontdisplaylinenum

ānnāt kāmo mado darpaḥ annāc chauryaṁ ca jāyate \veg\dontdisplaylinenum
\varr{
        \ \va annāj jī°\lem  \msCa\msNa\msNb\Ed; annā jī° \msCb\msCc\msNc
        \ \vb annaṁ\lem  \msCa\msCb\msNa\msNc\Ed; annāṁ \msCc, annā \msNb\oo
                 °karaṁ\lem  \msCa\msCb\msNa\msNb\msNc; °karaḥ \msCc\Ed
        \ \vc darpaḥ\lem  \msCa\msCc\msNa\msNb; darppa \msCb\msNc, darppo \Ed
        \ \vd annāc chauryaṁ ca\lem  \msCa\msCc\msNc; annāt sauryañ ca \msCb\msNa\msNb, annāc chauryaś ca \Ed}

annaṁ kṣudhāt\textsubring{r}ṣāvyādhīn sadya eva vināśayet\thinspace{\dandab} \dontdisplaylinenum

annadānāc ca saubhāgyaṁ khyātiḥ kīrtiś ca jāyate \veg\dontdisplaylinenum
\varr{
        \ \va annaṁ kṣu°\lem  \msCa\msCb\msNapcorr\msNc; annā kṣu° \msCc\msNaacorr, annāt kṣu° \msNb\Ed
        \ \vab °vyādhīn sa°\lem  \msCb\msNc; °vyādhān sa° \msCa\msCc\msNb, °vādhān sa° \msNa, °vyādhā sa° \Ed
        \ \vb vināśayet\lem  \msCa\msCc\msNa\msNb\msNc\Ed; viśayet \msCb}

annadaḥ prāṇadaś caiva prāṇadaś cāpi sarvadaḥ\thinspace{\dandab} \dontdisplaylinenum

tasmād annasamaṁ dānaṁ na bhūtaṁ na bhaviṣyati \veg\dontdisplaylinenum
            \paral{\textit{\vcd {\normalfont  \kb\ MBh 13.62.6ab: }annena sad\textsubring{r}śaṁ dānaṁ na bhūtaṁ na bhaviṣyati}}


\alalalfejezet{vastradānam}
\varr{
        \ \va annadaḥ\lem  \mssCaCbCc\msNa\msNb\msNc; annada \Ed
        \ \vb prāṇadaś cāpi\lem  \mssCaCbCc\msNa\msNc\Ed; prāṇaś cāpi \msNb\oo
                 sarvadaḥ\lem  \msCa\msCb\msNa\msNb\msNc\Ed; sarvadāḥ \msCc
        \ \vd bhūtaṁ\lem  \msCc\msNa\msNb\msNc; {\lost}tan \msCa, bhūte \msCb, bhūto \Ed}

vastrābhāvān manuṣyasya śriyād api parityajet\thinspace{\dandab} \dontdisplaylinenum

vastrahīno na pūjyeta bhāryāputrasakhādibhiḥ \veg\dontdisplaylinenum
\varr{
        \ \va °bhāvān ma°\lem  \mssCaCbCc\msNb\Ed; °bhāvāt ma° \msNa\msNc
        \ \vb śriyād api\lem  \msCa\msCc\msNa\msNb\Ed; priyād api \msCb, śriyā vāpi \msNc}

vidyāvān sukulīno 'pi jñānavān guṇavān api\thinspace{\dandab} \dontdisplaylinenum

vastrahīnaḥ parādhīnaḥ paribhūtaḥ pade pade \veg\dontdisplaylinenum

apamānam avajñāṁ ca vastrahīno hy avāpnuyāt\thinspace{\dandab} \dontdisplaylinenum

jugupsati mahātmāpi sabhāstrījanasaṁsadi \veg\dontdisplaylinenum
\varr{
        \ \va °vajñāṁ\lem  \mssCaCbCc\msNa\msNb\msNc; °vajñaṁ \Ed
        \ \vb °hīno\lem  \msCa\msCc\msNa\msNb\msNc\Ed; °hī \msCb}

tasmād vastrapradānāni praśaṁsanti manīṣiṇaḥ\thinspace{\dandab} \dontdisplaylinenum

na jīrṇaṁ sphuṭitaṁ dadyād vastraṁ kutsitam eva vā \veg\dontdisplaylinenum
\varr{
        \ \vc jīrṇaṁ sphuṭitaṁ\lem  \mssCaCbCc\msNa\msNc; jīrṇasphaṭitaṁ \msNb\Ed
        \ \vd kutsitam eva vā\lem  \msCa\msCb\msNa\msNb\Ed; kutsitam eva ca \msCc, kutsitmeva vā \msNc}

navaṁ purāṇarahitaṁ m\textsubring{r}du sūkṣmaṁ suśobhanam\thinspace{\dandab} \dontdisplaylinenum

susaṁsk\textsubring{r}tya pradātavyaṁ śraddhābhaktisamanvitam \veg\dontdisplaylinenum
\varr{
        \ \vb sūkṣmaṁ\lem  \msCa\msCb\msNa\msNb\msNc; sūkṣma \msCc, śuklaṁ \Ed
        \ \vc °dātavyaṁ\lem  \msCa\msCb\msNa\msNb\msNc\Ed; °dātavya \msCc
        \ \vd °samanvitam\lem  \mssCaCbCc\msNapcorr\msNb\msNc\Ed; °taṁ \msNaacorr}

śraddhāsattvaviśeṣeṇa deśakālavidhena ca\thinspace{\dandab} \dontdisplaylinenum

pātradravyaviśeṣeṇa phalam āhuḥ p\textsubring{r}thak p\textsubring{r}thak \veg\dontdisplaylinenum
            \paral{\textit{\vo {\normalfont cf. Manu 7.86--87 (the latter usually labelled as an additional verse): }
                        pātrasya hi viśeṣeṇa śraddadhānatayāiva ca{\thinspace\danda} 
                        alpaṁ vā bahu vā pretya dānasya phalam aśnute{\thinspace\ketdanda}
                        deśakālavidhānena dravyaṁ śraddhāsamanvitam{\thinspace\danda}
                        pātre pradīyate yat tu tad dharmasya prasādhanam{\thinspace\ketdanda}}}
\varr{
        \ \va °sattva°\lem  \mssCaCbCc\msNa\msNb\msNc; °sa ca° \Ed}

yād\textsubring{r}śaṁ dīyate vastraṁ tād\textsubring{r}śaṁ prāpyate phalam\thinspace{\dandab} \dontdisplaylinenum

jīrṇavastrapradānena jīrṇavastram avāpnuyāt \danda\dontdisplaylinenum

śobhanaṁ dīyate vastraṁ śobhanaṁ vastram āpnuyāt \veg\dontdisplaylinenum

\ujvers\nemsloka 
dadyād vastra suśobhanaṁ dvijavare kāle śubhe sādaram
\dontdisplaylinenum

\nemslokab 
saubhāgyam atulaṁ labheta sa naro rūpaṁ tathā śobhanam \danda\dontdisplaylinenum

\nemslokac 
tasmin yāti suvastrakoṭi śataśaḥ prāpnoti niḥsaṁśayam
\dontdisplaylinenum

\nemslokad 
tasmāt tvaṁ kuru vastradānam asak\textsubring{r}t pāratrikotkarṣaṇam \veg\dontdisplaylinenum


\alalalfejezet{suvarṇadānam}

\vers
\varr{
        \ \va dvijavare kāle śubhe\lem  \mssCaCbCc\msNa\msNb\msNc; dvijayine ekāśubhaṁ \Ed
        \ \vb naro\lem  \msCa\msCc\msNa\msNb\msNc\Ed; daro \msCb
        \ \vc tasmin yāti\lem  \mssCaCbCc\msNb\msNc\Ed; ta\uncl{smā}n yāti \msNa\oo
                 suvastra°\lem  \mssCaCbCc\msNa\msNb\msNc; sa vastra° \Ed\oo
                 °saṁśayam\lem  \msCa\msCb\msNc; °saṁśayaḥ \msCc\msNa\msNb\Ed
        \ \vd dānam asak\textsubring{r}t pā°\lem  \mssCaCbCc\msNa\msNc\Ed; dānasat pā° \msNb}

suvarṇadānaṁ viprendra saṁkṣipya kathayāmy aham\thinspace{\dandab} \dontdisplaylinenum

pavitraṁ maṅgalaṁ puṇyaṁ sarvapātakanāśanam \veg\dontdisplaylinenum
\varr{
        \ \va °dānaṁ\lem  \mssCaCbCc\msNa\msNc; °dāna \msNb\Ed
        \ \vd °pātaka°\lem   \msCb\msCc\msNa\msNb\msNc\Ed; °pāpaka° \msCa}

dhārayet satataṁ vipra suvarṇakaṭakāṅgulim\thinspace{\dandab} \dontdisplaylinenum

mucyate sarvapāpebhyo rāhunā candramā yathā \veg\dontdisplaylinenum
            \paral{\textit{\vcd {\normalfont  = 22.38 CHECK }}}
\varr{
        \ \vb °kaṭakāṅgulim\lem  \msCb\msCc\msNa\msNc\Ed; °ka{\il}{\il}gulim \msCa, °kaṭakāṅgulam \msNb}

dattvā suvarṇaṁ viprebhyo devebhyaś ca dvijarṣabha\thinspace{\dandab} \dontdisplaylinenum

tuṭimātre 'pi yo dadyāt sarvapāpaiḥ pramucyate \veg\dontdisplaylinenum
\varr{
        \ \va suvarṇaṁ\lem  \mssCaCbCc\msNa\msNc\Ed; suvarṇa \msNb
        \ \vb °rṣabha\lem  \msCa\msCb\msNa\msNc\Ed; °rṣabhaḥ \msCc\msNb
        \ \vc tuṭi°\lem  \mssCaCbCc\msNa\msNb\msNc; truṭi° \Ed\oo
                 °mātre\lem  \mssCaCbCc\msNb\msNc; °mātro \msNa\Ed
        \ \vd sarvapāpaiḥ pramucyate\lem  \msCb\msCc\msNa\msNb\msNc;
                                 sarvapāpaiḥ sa mucyate \msCa, sarvapāpai pramucyate \Ed}

raktimāṣakakarṣaṁ vā palārdhaṁ palam eva vā\thinspace{\dandab} \dontdisplaylinenum

evam eva phalaṁ v\textsubring{r}ddhir jñeyā dānaviśeṣataḥ \veg\dontdisplaylinenum


\alalalfejezet{bhūmidānam}
\varr{
        \ \va raktimāṣaka°\lem  \msNcacorr; rantimāṣaka° \msCa, rattimāṣaka° \msCb\msNa\msNcpcorr, 
                                                        rantimmānsaka° \msCc, rattimānsaka° \msNb, rattimāṣaka° \Ed
        \ \vb °rdhaṁ\lem  \msCa\msCb\msNc\Ed; °ddha \msCc\msNa\msNb
        \ \vcd v\textsubring{r}ddhir jñeyā\lem  \msCa\Ed; v\textsubring{r}ddhi jñeyā \msCb\msCc\msNa\msNb, v\textsubring{r}rddhi jñeyā \msNc}

sarvādhāraṁ mahīdānaṁ praśaṁsanti manīṣiṇaḥ\thinspace{\dandab} \dontdisplaylinenum

annavastrahiraṇyādi sarvaṁ vai bhūmisambhavam \veg\dontdisplaylinenum
\varr{
        \ \va °dhāraṁ\lem  \msCb; °dhāra° \msCa\msCc\msNa\msNb\msNc\Ed
        \ \vab °dānaṁ praśaṁsanti\lem  \msCb\msCc\msNa\msNb\msNc\Ed; dā{\il}\uncl{nam pra}{\lost}santi \msCa
        \ \vd sarvaṁ vai\lem  \msCb\msCc\msNa\msNb\msNc\Ed; sarvaṁ \uncl{ve} \msCa\ \toplost}

bhūmidānena viprendra sarvadānaphalaṁ labhet\thinspace{\dandab} \dontdisplaylinenum

bhūmidānasamaṁ vipra yady asti vada tattvataḥ \veg\dontdisplaylinenum
\varr{
        \ \vb °phalaṁ labhet\lem  \mssCaCbCc\msNa\msNbpcorr\Ed; °lalaṁ bhavet \msNbacorr, °laṁ bhavet \msNc}

māt\textsubring{r}kukṣivimuktas tu dharaṇīśaraṇo bhavet\thinspace{\dandab} \dontdisplaylinenum

carācarāṇāṁ sarveṣāṁ bhūmiḥ sādhāraṇā sm\textsubring{r}tā \veg\dontdisplaylinenum
\varr{
        \ \va °muktas tu\lem  \mssCaCbCc\msNa\msNb\msNc; °muktis tu \Ed
        \ \vb °śaraṇo\lem  \mssCaCbCc\msNa\msNb; °śaraṇa \msNc, °śaraṇāṁ \Ed}

ekahastaṁ dvihastaṁ vā pañcāśac chatam eva vā\thinspace{\dandab} \dontdisplaylinenum

sahasrāyutalakṣaṁ vā bhūmidānaṁ praśasyate \veg\dontdisplaylinenum
\varr{
        \ \va ekahastaṁ\lem  \msCb\msNa\msNb\msNc; ekahasta° \msCa\msCc\Ed
        \ \vd bhūmidānaṁ praśasyate\lem  \msCa\msCc\msNa\msNc\Ed; bhūmidāna praśasyate \msCb, 
                        pañcāśac chatam eva vā{\danda} sahāyutalakṣam vā bhūmidaṁ praśasyate \msNb\ {\normalfont (eyeskip)}}

ekahastāṁ ca yo bhūmiṁ dadyād dvijavarāya tu\thinspace{\dandab} \dontdisplaylinenum

varṣakoṭiśataṁ divyaṁ svargaloke mahīyate \veg\dontdisplaylinenum
\varr{
        \ \va °hastāṁ ca\lem  \msCa\msCc\msNa\msNc\Ed; °hastañ ca \msCb\msNb
        \ \vb dadyād dvi°\lem  \mssCaCbCc\msNa\msNb\msNc; dadyā dvi° \Ed}

evaṁ bahuṣu hasteṣu guṇāguṇi phalaṁ sm\textsubring{r}tam\thinspace{\dandab} \dontdisplaylinenum

śraddhādhikaṁ phalaṁ dānaṁ kathitaṁ te dvijottama \veg\dontdisplaylinenum
\varr{
        \ \vb guṇāguṇi°\lem  \mssCaCbCc\msNa\msNb\msNc; guṇāgaṇi° \Ed
        \ \vc °dhikaṁ\lem  \msCb\msCc\msNa\msNb; °dhika° \msCa\msNc\Ed
        \ \vd °ttama\lem  \mssCaCbCc\msNa\msNb\Ed; °ttamaḥ \msNc}

jāmadagnyena rāmeṇa bhūmiṁ dattvā dvijāya vai\thinspace{\dandab} \dontdisplaylinenum

āyur akṣayam āptaṁ tu ihaiva ca dvijottama \veg\dontdisplaylinenum


\alalalfejezet{godānam}
\varr{
        \ \va jāmadagnyena\lem  \msCb\msNa\msNc; jāmadagnye{\il} \msCa, jāmadagnena \msCc\msNb\Ed\oo
                 rāmeṇa\lem  \msCb\msNc\Ed; rāmena \msCc\msNa\msNb, {\il}{\il}ṇa \msCa
        \ \vb dattvā dvi°\lem  \msCa\msCc\msNa\msNb\msNc\Ed; dadyād dvi° \msCb
        \ \vd ca\lem  \mssCaCbCc\msNa\msNb\msNc; hi \Ed}

hemaś\textsubring{r}ṅgāṁ raupyakhurāṁ cailaghaṇṭāṁ dvijottama\thinspace{\dandab} \dontdisplaylinenum

viprāya vedaviduṣe dattvānantaphalaṁ sm\textsubring{r}tam \veg\dontdisplaylinenum
            \paral{\textit{\vo {\normalfont cf. e.g. MBh 7.58.18: }
                tathā gāḥ kapilā dogdhrīḥ sarṣabhāḥ pāṇḍunandanaḥ{\thinspace\danda}
                hemaś\textsubring{r}ṅgī rūpyakhurā dattvā cakre pradakṣiṇam{\thinspace\ketdanda}
                      {\normalfont and Bhaviṣyapurāṇa Uttara 12.25 CHECK: }
                hemaś\textsubring{r}ṁgīṁ raupyakhurāṁ saghaṁṭāṁ kāṁsyadohanām{\thinspace\danda} 
                mahādevāya gāṁ dadyād dīkṣitāya dvijāya vai{\thinspace\ketdanda}}}


\alalalfejezet{dānapraśaṁsā}
\varr{
        \ \vab (hema°{\normalfont ...} dvijottama)\lem  \mssCaCbCc\msNa\msNc\Ed; \om\ \msNb
        \ \va °ś\textsubring{r}ṅgāṁ\lem  \mssCaCbCc\msNc\Ed; °ś\textsubring{r}ṅgaṁ \msNa, \om\ \msNb\oo
                 raupya°\lem  \mssCaCbCc\msNa\msNb\Ed; ropyaṁ \msNc\oo
                 °khurāṁ\lem  \msCc\Ed; °kṣurāṁ \msCa\msCb\msNa\msNc, \om\ \msNb
        \ \vd dattvānanta°\lem  \mssCaCbCc\msNa\msNb\msNc; dattvānta° \Ed}

\ujvers\nemsloka 
dānābhyāsarataḥ pravartanabhavāṁ śakyānurūpaṁ sadā
\dontdisplaylinenum

\nemslokab 
annaṁ vastrahiraṇyaraupyam udakaṁ gāvas tilān medinīm \danda\dontdisplaylinenum

\nemslokac 
dadyāt pādukachattrapīṭhakalaśaṁ pātrādyam anyac ca vā
\dontdisplaylinenum

\nemslokad 
śraddhādānam abhinnarāgavadanaṁ k\textsubring{r}tvā mano nirmalam \veg\dontdisplaylinenum
\varr{
        \ \va °rūpaṁ\lem  \mssCaCbCc\msNa\msNc\Ed; °rūpa \msNb
        \ \vb °raupya°\lem  \msCa\msCc\msNa\msNb\Ed; °ropya° \msCb, °\uncl{raupya}° \msNc\oo
                 gāvas tilān me°\lem  \eme; gāvas tilām me° \msCa\msCc\msNc, gāvas tilā me° \msCb\msNa,
                                                gāvan tilā me° \msNb, gāvas tilaṁ me° \Ed
        \ \vc dadyāt pā°\lem  \mssCaCbCc\msNa\msNc\Ed; dadyā pā° \msNb\oo
                 pātrādyam anyac ca vā\lem  \msCa\msCc\msNa\msNb\msNc;
                                         patrādyam anyac ca vā \msCb, pātreṣu labdheṣu vai \Ed 
        \ \vd śraddhādāna°\lem  \mssCaCbCc\msNa\msNb\msNc; dattvādāna° \Ed}

\ujvers\nemsloka 
dānād eva yaśaḥ śriyaḥ sukhakarāḥ khyātiṁ ca tulyāṁ labhet
\dontdisplaylinenum

\nemslokab 
dānād eva nigarhaṇaṁ ripugaṇe ānandadaṁ saukhyadam \danda\dontdisplaylinenum

\nemslokac 
dānād durjayatā prasādam atulaṁ saubhāgya dānāl labhet
\dontdisplaylinenum

\nemslokad 
dānād eva anantabhoga niyataṁ svargaṁ ca tasmād bhavet \veg\dontdisplaylinenum
\varr{
        \ \va yaśaḥ\lem  \msCb\msNc\Ed; yaśa \msCa\msCc\msNa\msNb\oo
                 sukhakarāḥ\lem  \mssCaCbCc\msNa\msNb\msNcacorr\Ed; sukhakara \msNcpcorr\oo
                 khyātiṁ ca tulyāṁ\lem  \eme; khyātiś ca tulyaṁ \mssCaCbCc\msNa\msNb\msNc\Ed\oo
                 labhet\lem  \mssCaCbCc\msNa\msNb; bhavet \msNc\Ed
        \ \vb nigarhaṇaṁ\lem  \msCapcorr\msCc\msNa\Ed; nirhaṇaṁ \msCaacorr, nivarhaṇaṁ \msCb\msNc,
                                                                 nigarhana \msNb\oo
                 °gaṇe ānandadaṁ saukhyadam\lem  \msCa\msCb\msNa\msNb\msNc; °gaṇai ānandadaṁ saukhyadam \msCc,
                                                        °gaṇaiś cānandasaukhyapradam  \Ed
        \ \vc dānād du°\lem  \Ed; dānādū° \mssCaCbCc\msNa\msNb\msNc\oo
                 °rjayatā\lem  \mssCaCbCc\msNb\msNc\Ed; °rjayatām \msNa\oo
                 prasāda°\lem  \mssCaCbCc\msNb\msNc\Ed; prāsāda° \msNa\oo
                 saubhāgya\lem  \msCa\msCc\msNa\msNb\msNc; saugāgya \msCb, saubhāgyaṁ \Ed\ \unmetr\oo
                 dānāl labhet\lem  \msCb\Ed; dānaṁ labhet \msCa\msCc\msNa\msNb\msNc
        \ \vd dānād eva\lem  \msCa\msCb\msNa\msNb\msNc\Ed; dānād ova \msCc\oo
                 °niyataṁ\lem  \msCa\msCb\msNa\msNb\msNc\Ed; °niyata \msCc}

\ujvers\nemsloka 
dānād eva ca śakralokasakalaṁ dānāj janānandanam
\dontdisplaylinenum

\nemslokab 
dānād eva mahīṁ samasta bubhuje samrāḍ mahīmaṇḍale \danda\dontdisplaylinenum

\nemslokac 
dānād eva surūpayonisubhagaś candrānano vīkṣyate
\dontdisplaylinenum

\nemslokad 
dānād eva anekasambhavasukhaṁ prāpnoti niḥsaṁśayam \veg\dontdisplaylinenum

\vers


\jump
\begin{center}
\ketdanda iti v\textsubring{r}ṣasārasaṁgrahe dānapraśaṁsādhyāyaḥ saptamaḥ\ketdanda
\end{center}
\dontdisplaylinenum\vers 
\varr{
        \ \va śakralokasakalaṁ\lem  \mssCaCbCc\msNb\msNc; śatrulokasakalaṁ \msNa, śakralokam atulaṁ \Ed\oo
                 dānāj ja°\lem  \msCc\msNa\msNb\msNc\Ed; dānā ja° \msCa, dānārja° \msCb
        \ \vb dānād eva\lem  \msCa\msCc\msNa\msNb\msNc\Ed; dāned eva \msCb\oo
                 mahīṁ samasta\lem  \conj; mahīsamāsu \msCb\msCc, mahīṁ samāṁsu \msCa\msNa\msNc,
                                                 mahī samasta \msNb, mahīyasāṁ sa \Ed\oo
                 samrāḍ ma°\lem  \msCa\msCc\msNa\msNb\msNc\Ed; saṁmrāḍ ma° \msCb
        \ \vc surūpa°\lem  \mssCaCbCc\msNa\msNc\Ed; svarūpa° \msNb\oo
                 °yonisu°\lem  \msNb\Ed; °yonis su° \msCa °yoniḥ su° \msCb\msCc\msNa\msNc\oo
                 °bhagaś ca°\lem  \msCa\msCc\msNb\msNc; °bhaga ca° \msCb\msNa\Ed\oo
                 °ndrānano\lem  \msCa\msCb\msNa\Ed; °ndrānane \msCc\msNb, °ndrānanau \msNc\oo
                 vīkṣyate\lem  \msCb\msCc; vīkṣate \msCa\msNa\msNb\msNc, vikṣate \Ed
        \ \vd niḥsaṁśayam\lem  \msCa\msCb\msNc; nisaṁśayaḥ \msCc, niḥsaṁśayaḥ \msNa\Ed, nissayaḥ \msNb
        \ {\normalfont \Colo:} °praśaṁsādhyāyaḥ saptamaḥ\lem  \msCa\msCc\msNa\msNb\msNc; 
                                °praśaṁsādhyāyaḥ samāptaḥ \msCb,
                                °praśaṁsā saptamo 'dhyāyaḥ \Ed}
\bekveg\szamveg\vfill\phpspagebreak\szam\bek\versno=0\fejno=8
\thispagestyle{empty}



\alfejezet{\textbf{aṣṭamo 'dhyāyaḥ}}\jump\jump

\alalfejezet{niyameṣu svādhyāyaḥ (5)}
\vers

pañcasvādhyāyanaṁ kāryam ihāmutra sukhārthinā\thinspace{\dandab} \dontdisplaylinenum

śaivaṁ sāṁkhyaṁ purāṇaṁ ca smārtaṁ bhāratasaṁhitām \veg\dontdisplaylinenum
\varr{
        \ \va °svādhyāyanaṁ\lem  \mssCaCbCc\msNa\msNb\msP\Ed; °svādhyayanaṁ \msNc
        \ \vb °mutra\lem  \mssCaCbCc\msNa\msNb\msNc\msP; °mūtra \Ed\oo
                 °rthinā\lem  \mssCaCbCc\msNa\msNc\msP\Ed; °rthināṁ \msNb
        \ \vc śaivaṁ\lem  \msCa\msCb\msNa\msNb\msNc\msP\Ed; \uncl{śai}laṁ \msCc\oo
                 sāṁkhyaṁ\lem  \msCa\msCb\msNc\msP\Ed; śāṁkhya \msCc, sākhyaṁ \msNa\msNb
        \ \vd smārtaṁ\lem  \msCa\msCb\msNa\msNc\msP\Ed; smārta \msCc\msNb\oo
                 bhāratasaṁhitām\lem  \mssCaCbCc\msNb\msP\Ed; bhāratasaṁhitāḥ \msNa, bhārattasaṁhitāṁ \msNc}

śaivatattvaṁ vicinteta śaivapāśupatadvaye\thinspace{\dandab} \dontdisplaylinenum

atra vistarataḥ proktaṁ tattvasārasamuccayam \veg\dontdisplaylinenum
\varr{
        \ \va śaiva°\lem  \conj; śaive \msCa\msCc\msNa\msNb\msNc, śaivai \msCb\msP, śaivaṁ \Ed\oo
                 °tattvaṁ\lem  \mssCaCbCc\msNa\msNb\msNc\Ed; °tattva \msP
        \ \vb śaiva°\lem  \msP; śaivaḥ \msCa\msCb\msNb\msNc, śaivāḥ \msCc\Ed, śaivā \msNa\oo
                 °dvaye\lem  \msCa\msCc\msNa\msNb\msNc\msP\Ed; °ye \msCb
        \ \vd °sārasamuccayam\lem  \mssCaCbCc\msNc\msP\Ed; °sāraṁ samuccayam \msNa, °sāraṁ samudyayaṁ \msNb}

saṁkhyātattvaṁ tu sāṁkhyeṣu boddhavyaṁ tattvacintakaiḥ\thinspace{\dandab} \dontdisplaylinenum

pañcatattvavibhāgena kīrtitāni maharṣibhiḥ \veg\dontdisplaylinenum
\varr{
        \ \va saṁkhyātattvaṁ tu\lem  \msNa\msNc\msP; saṁ\uncl{khyā}{\il}{\il}{\il} \msCa, saṁkhyātattvaṁ \msCb, 
                                       śāṅkhyātatvaṁ tu \msCc, sakhyatatvan tu \msNb, saṁkhyātattva tu \Ed\oo
                 sāṁkhyeṣu\lem  \mssCaCbCc\msNa\msNc\msP\Ed; sakhyeṣu \msNb
        \ \vc °tattva°\lem  \msCa\msCc\msNa\msNc\msP\Ed; °tatvā° \msCb, \om\ \msNb}

purāṇeṣu mahīkoṣo vistareṇa prakīrtitaḥ\thinspace{\dandab} \dontdisplaylinenum

adhordhvamadhyatiryaṁ ca yatnataḥ sampraveśayet \veg\dontdisplaylinenum
\varr{
        \ \vc adhordhva°\lem  \mssCaCbCc\msNa\msNc\msP\Ed; adhordhvaṁ \msNb\oo
                 °madhya°\lem  \msCa\msCb\msNa\msNb\msNc\msP\Ed; °madha° \msCc
        \ \vd yatnataḥ\lem  \mssCaCbCc\msNa\msNc\msP\Ed; yatnata \msNb\oo
                 sampraveśayet\lem  \mssCaCbCc\msNa\msNb\msNc\msP; samprabodhayet \Ed}

smārtaṁ varṇāśramācāraṁ dharmanyāyapravartanam\thinspace{\dandab} \dontdisplaylinenum

śiṣṭācāro 'vikalpena grāhyas tatra aśaṅkitaḥ \veg\dontdisplaylinenum
\varr{
        \ \va smārtaṁ varṇā°\lem  \msCa; tasmārttam varṇṇā° \msCb,
                                        smārtavarṇā° \msCc\msNa\msNb\msNc\Ed, smārttaṁ varṇṇa° \msP
        \ \vb dharma°\lem  \msCa\msCb\msNa\msNb\msNc\msP\Ed; dharmaṁ \msCc\oo
                 °vartanam\lem  \mssCaCbCc\msNa\msNb\msNc; °va{\il}naṁ \msP, °vartana \Ed
        \ \vc °cāro\lem  \msCa\msCb\msNb\msNc; °cāra° \msCc\Ed, °cārā \msNa, °cā\uncl{ro}° \msP
        \ \vd grāhyas tatra aśaṅkitaḥ\lem  \msCb\msCc\msNa\msNb\msNc\msP\Ed; grāhyas ta{\il}{\il}{\il}ṅkitaḥ \msCa}

itihāsam adhīyānaḥ sarvajñaḥ sa naro bhavet\thinspace{\dandab} \dontdisplaylinenum

dharmārthakāmamokṣeṣu saṁśayas tena chidyate \veg\dontdisplaylinenum


\alalfejezet{niyameṣv upasthanigrahaḥ (6)}\varr{
        \ \vb °jñaḥ\lem  \msCa\msCb\msNa\msNb\msNc\msP\Ed; °jña \msCc}

ś\textsubring{r}ṇuṣvāvahito vipra pañcopasthavinigraham\thinspace{\dandab} \dontdisplaylinenum

striyo vā garhitotsargaḥ svayaṁmuktiś ca kīrtyate \danda\dontdisplaylinenum

svapnopaghātaṁ viprendra divāsvapnaṁ ca pañcamaḥ \veg\dontdisplaylinenum


\alalalfejezet{striyaḥ}
\varr{
        \ \vb °graham\lem  \mssCaCbCc\msNb\msNc\msP\Ed; °graha\uncl{ḥ} \msNa
        \ \vc garhitotsargaḥ\lem  \msCa\msCb\msNb\msNc\msP; garhitas sarggaḥ \msCc, garhito vipra \msNa,
                                                                                garhito svargaḥ \Ed
        \ \vd svayaṁ°\lem  \msCa\msCc\msNa\msNb\msNc\msP\Ed; svaya° \msCb\oo
                 kīrtyate\lem  \msCa\msCb\msNa\msNb\msNc\msP\Ed; kī\uncl{rtsya}te \msCc
        \ \ve °ghātaṁ\lem  \msCa\msCb\msNa\msNb\msNc\msP; °ghāta \msCc\Ed}

agamyā strī divā parve dharmapatny api vā bhavet\thinspace{\dandab} \dontdisplaylinenum
            \paral{\textit{\vab {\normalfont cf.\ Manu 11.175 (Olivelle's edition): }
                         maithunaṁ tu samāsevya puṁsi yoṣiti vā dvijaḥ {\thinspace\danda}
                         goyāne 'psu divā caiva savāsāḥ snānam ācaret {\thinspace\ketdanda} {\normalfont and Manu 3.45 (Olivelle's edition): }
                         \textsubring{r}tukālābhigāmī syāt svadāranirataḥ sadā {\thinspace\danda}
                         parvavarjaṁ vrajec caināṁ tadvrato ratikāmyayā {\thinspace\ketdanda}}}

viruddhastrī na seveta varṇabhraṣṭādhikāsu ca \veg\dontdisplaylinenum


\alalalfejezet{garhitotsargaḥ}
\varr{
        \ \va strī divā parve\lem  \msCb\msCc\msNa\msNb\msNc; {\il} divā parvve \msCa, {\il}{\il}{\il} parve \msP, strī divāpūrve \Ed
        \ \vb °patny api\lem  \msCa\msCb\msNa\msNb\msNc\msP\Ed; °patnī pi \msCc
        \ \vc viruddhastrī na\lem  \mssCaCbCc\msNb\msNc; viruddhastrī niseveta \msNa\msP, dviruddhāstrīn na \Ed
        \ \vd °dhikāsu ca\lem  \msCa\msCb\msNa\msP; °dhikāsu ta \msCc, °dikāṣu ca \msNb, 
                                                        °vikāṣu ca \msNc, °pikāsu ca \Ed}

ajameṣagavādīnāṁ vaḍavāmahiṣīṣu ca\thinspace{\dandab} \dontdisplaylinenum

garhitotsargam ity etad yatnena parivarjayet \veg\dontdisplaylinenum


\alalalfejezet{svayaṁmuktiḥ}
\varr{
        \ \va °meṣa°\lem  \msCa\msCc\msNa\msNb\msNc\msP\Ed; °meya° \msCb}

ayonyakaṣaṇā vāpi apānakaṣaṇāpi vā\thinspace{\dandab} \dontdisplaylinenum

svayaṁmuktir iyaṁ jñeyā tasmāt tāṁ parivarjayet \veg\dontdisplaylinenum


\alalalfejezet{svapnaghātaṁ}
\varr{
        \ \va ayonya°\lem  \conj; anyonya° \mssCaCbCc\msNa\msNb\msNc\msP\Ed\oo
                 °kaṣaṇā\lem  \msCa\msNa; °karṣaṇā \msCb\msCc\msNb\msNc\msP\Ed
        \ \vb °kaṣaṇāpi\lem  \mssCaCbCc\msNa; °karṣaṇāpi \msNb\msNc\msP\Ed
        \ \vc svayaṁmukti°\lem  \msCa\msCc\msNa\msNb\msNc\msP\Ed; svayamukti° \msCb\oo
                 jñeyā\lem  \mssCaCbCc\msNa\msNc\msP\Ed; jñeyāṁ \msNb
        \ \vd tasmāt tāṁ\lem  \msCa\msCb\msNa\msNc\msP; tasmāt taṁ \msCc, tasmārttā \msNb, tasmāt strī \Ed}

svapnaghātaṁ dvijaśreṣṭha aniṣṭaṁ paṇḍitaiḥ sadā\thinspace{\dandab} \dontdisplaylinenum

svapne strīṣu ramante ca retaḥ prakṣarate tataḥ \veg\dontdisplaylinenum


\alalalfejezet{divāsvapnam}
\varr{
        \ \va svapnaghā°\lem  \mssCaCbCc\msNa\msNb\msNc\msPpcorr\Ed; svapnajā° \msPacorr
        \ \vb paṇḍitaiḥ\lem  \msCa\msCb\msNa\msNb\msP\Ed; paṇḍitai \msCc, paṇḍiteḥ \msNc
        \ \vd prakṣarate\lem  \mssCaCbCc\msNa\msNb\msNc\msP; praskhalatas \Ed\oo
                 tataḥ\lem  \msCa\msCb\msNa\msNb\msNc\msP\Ed; tata \msCc}

divāśayaṁ na kartavyaṁ nityaṁ dharmapareṇa tu\thinspace{\dandab} \dontdisplaylinenum 

svargamārgārgalā hy etāḥ striyo nāma prakīrtitāḥ \veg\dontdisplaylinenum
            \paral{\textit{\vcd {\normalfont cf.\ Padmapurāṇa 1.13.395cd: }parityajadhvaṁ dārāṇi svargamārgārgalāni ca}}


\alalfejezet{niyameṣu vratapañcakam (7)}\varr{
        \ \va divāśayaṁ na\lem  \mssCaCbCc\msP\Ed; divāsayānaṁ \msNb, divāśayen na \msNa, divāśāyaṁ \msNc
        \ \vb nityaṁ\lem  \mssCaCbCc\msNa\msNc\msP\Ed; nitya \msNb\oo
                 °pareṇa tu\lem  \msCb\msNa\msNb\msNc\msP\Ed; °parena tu \msCa, °pareṇa ca \msCc
        \ \vc hy etāḥ\lem  \msNc; hy etā \mssCaCbCc\msNa\msNb\msP\Ed
        \ \vd striyo\lem  \mssCaCbCc\msNa\msNb\msNc\msP; strīyo \Ed\oo
                 °kīrtitāḥ\lem  \mssCaCbCc\msNa\msNb\msP\Ed; °kīrtitā \msNc}

mārjārakabakaśvānagomahīvratapañcakam\thinspace{\dandab} \dontdisplaylinenum


\alalalfejezet{mārjārakavratam}

svaviṣṭhamūtraṁ bhūmīṣu chādayed dvijasattama \danda\dontdisplaylinenum

sūryasomānumodanti mārjāravratikeṣu ca \veg\dontdisplaylinenum


\alalalfejezet{bakavratam}
\varr{
        \ \vab mārjārakabakaśvānagomahīvrata°\lem  \mssCaCbCc\msNa\msNc\msP;
                                 mārjārabakabaśvānagomahīvrata° \msNb,
                                 mārjārakaś ca śvānāś ca gomahīvaka \Ed
        \ \vc °viṣṭha°\lem  \mssCaCbCc\msNa\msNb\msNc\msP; °viṣṭā° \Ed\oo
                 °mūtraṁ\lem  \msCa\msCc\msNa\msNc\msP\Ed; °mūtra° \msCb\msNb
        \ \ve °modanti\lem  \mssCaCbCc\msNa\msNb\msNc\msP; °ṣādanti \Ed}

bakavac cendriyagrāmaṁ suniyamya tapodhana\thinspace{\dandab} \dontdisplaylinenum

sādhayec ca manastuṣṭiṁ mokṣasādhanatatparaḥ \veg\dontdisplaylinenum


\alalalfejezet{śvānavratam}
\varr{
        \ \va tapodhana\lem  \mssCaCbCc\msNa\msNb\msP; tapodhanaḥ \msNc, tapodhanam \Ed
        \ \vc sādhayec ca\lem  \msCa\msCc\msNa\msNb\msNc\msP\Ed; sādhaye ca \msCb\oo
                 manastuṣṭiṁ\lem  \msCa\msNa\msNb\msNc\msP\Ed; manastuṣṭi° \msCb\msCc
        \ \vd °sādhana°\lem  \mssCaCbCc\msNa\msNb\msP\Ed; °sāna° \msNc}

mūtraviṣṭhe na bhūmīṣu kurute śvānadaḥ sadā\thinspace{\dandab} \dontdisplaylinenum

tuṣyate bhagavān śarvaḥ śvānavratacaro yadi \veg\dontdisplaylinenum


\alalalfejezet{govratam}
\varr{
        \ \va mūtraviṣṭhe na\lem  \mssCaCbCc\msNa\msNb\msNc\msP; mūtraviṣṭe ca \Ed
        \ \vb śvānadaḥ\lem  \msNa; dhunadaṁ \mssCaCbCc\msNb\msNc\msP, chādanaṁ \Ed
        \ \vc śarvaḥ\lem  \msCa\msNa\msNc\msP\Ed; sarvaḥ \msCb\msNb, savvaḥ \msCc}

mūtravarco na rudhyeta sadā govratiko naraḥ\thinspace{\dandab} \dontdisplaylinenum

bhīma tuṣṭikaraś caiva purāṇeṣu nigadyate \veg\dontdisplaylinenum


\alalalfejezet{mahīvratam}
\varr{
        \ \va °varco\lem  \msCa\msCc\msNb\msNc\msP; °vacco \msCb\msNa, °varcā \Ed
        \ \vb govratiko\lem  \msCb\msCc\msNa\msNb\msNc\msP\Ed; {\il}{\il}tiko \msCa
        \ \vc bhīma tu°\lem  \msCa\msCb\msNa\msNc\msP; bhīmas tu° \msCc\msNb\Ed}

kuddālair dārayanto 'pi kīlakoṭiśataiś citaḥ\thinspace{\dandab} \dontdisplaylinenum

kṣamate p\textsubring{r}thivī devī evam eva mahīvrataḥ \veg\dontdisplaylinenum
\varr{
        \ \va kuddālair dārayanto\lem  \msNa\msP\Ed; kuddālair dārayanno \msCa, kuddārai dārayanto \msCb,
                        kudārai dārayantā \msCc, kuddālai ddārayām āsa \msNb, kuddālai dārayanto \msNc
        \ \vb kīlakoṭiśataiś citaḥ\lem  \msCa\msCb\msNa\msNb\msNc\msP; kīṭakoṭīśatair api \msCc\Ed
        \ \vd °vrataḥ\lem  \mssCaCbCc\msNa\msNb\msP\Ed; °vrata \msNc}

vratapañcakam ity etad yaś careta jitendriyaḥ\thinspace{\dandab} \dontdisplaylinenum

sa cottamam idaṁ lokaṁ prāpnoti na ca saṁśayaḥ \veg\dontdisplaylinenum


\alalfejezet{niyameṣv upavāsaḥ (8)}\varr{
        \ \vb jitendriyaḥ\lem  \mssCaCbCc\msNa\msNc\msP\Ed; dvijendriyaḥ \msNb}

śeṣānnam antarānnaṁ ca naktāyācitam eva ca\thinspace{\dandab} \dontdisplaylinenum

upavāsaṁ ca pañcaitat kathayiṣyāmi tac ch\textsubring{r}ṇu \veg\dontdisplaylinenum


\alalalfejezet{śeṣānnam}
\varr{
        \ \va śeṣānnam antarānnaṁ ca\lem  \msCa\msCb\msNb\msNc\msPpcorr; śeṣānnam annarānnaṁ ca \msNa,
                                śeṣānnam arānnaṁ ca \msPacorr,
                                śeṣāṇām antarāṇāñ ca \msCc\Ed
        \ \vb naktāyācita°\lem  \mssCaCbCc\msNa\msNb\msP\Ed; nakta\uncl{yā}cita° \msNc\oo
                 ca\lem  \mssCaCbCc\msNa\msNb\msNc\msP; vā \Ed
        \ \vcd pañcaitat ka°\lem  \msCa\msCb\msNa\msNb\msNc\msP\Ed; pañcaite ka° \msCc}

vaiśvadevātithiśeṣaṁ pit\textsubring{r}śeṣaṁ ca yad bhavet\thinspace{\dandab} \dontdisplaylinenum

bh\textsubring{r}tyaputrakalatrebhyaḥ śeṣāśī vighasāśanaḥ \veg\dontdisplaylinenum


\alalalfejezet{antarānnam}
\varr{
        \ \va °śeṣaṁ\lem  \msCa\msCc\msNa\msNb\msNc\msP\Ed; °śeṣāṁ \msCb
        \ \vd vighasāśanaḥ\lem  \msCa\msNa\msNb; vighasāsanam \msCb, vighasāṣinaḥ \msCc, 
                                vighaśāsanaḥ \msNc, vighasāśana\uncl{ḥ} \msPpcorr, ghasāśana \msPacorr, viṣasāsanaḥ \Ed}

antarā prātarāśī ca sāyamāśī tathaiva ca\thinspace{\dandab} \dontdisplaylinenum

sadopavāsī bhavati yo na bhuṅkte kadācana \veg\dontdisplaylinenum
            \paral{\textit{\vcd \kb\ {\normalfont MBh 12.214.9:  }antarā prātarāśaṁ ca sāyamāśaṁ tathaiva ca{\thinspace\danda}
                                         sadopavāsī ca bhaved yo na bhuṅkte kathaṁcana{\thinspace\ketdanda} \kb\
                         {\normalfont MBh 13.93.10: }antarā sāyamāśaṁ ca prātarāśaṁ tathaiva ca{\thinspace\danda}
                                         sadopavāsī bhavati yo na bhuṅkte 'ntarā punaḥ{\thinspace\ketdanda}}}


\alalalfejezet{naktānnam}
\varr{
        \ \va antarā prātarāśī\lem  \eme; antarā prāntarāśī \mssCaCbCc\msNa\msNc, 
                                        antarā \uncl{kranta}rāśī \msNb, 
                                                antārā prā\uncl{tta}rāśī \msP, antasamprāntarāśī \Ed
        \ \vb sāyamāśī\lem  \msCb\msCc\msNa\msNb\msNc\msP; sāyamāśīn \msCa, niyamāśī \Ed
        \ \vc °vāsī bhavati\lem  \msCa\msCb\msNa\msNb\msNc\msP\Ed; °vāsī ca bhavati \msCc
        \ \vd kadācana\lem  \msCa\msCb\msNa\msNb\msNc\msP\Ed; kadācanaḥ \msCc}

na divā bhojanaṁ kāryaṁ rātrau naiva ca bhojayet\thinspace{\dandab} \dontdisplaylinenum

naktavele ca bhoktavyaṁ naktadharmaṁ samīhatā \veg\dontdisplaylinenum


\alalalfejezet{ayācitānnam}
\varr{
        \ \va bhojanaṁ\lem  \mssCaCbCc\msNa\msNb\msP\Ed; nojanaṁ \msNc
        \ \vb ca\lem  \msCa\msCc\msNb\msNc\msP\Ed; tu \msCb, \om\ \msNa\oo
                 bhojayet\lem  \mssCaCbCc\msNa\msNc\msP\Ed; kārayet \msNb
        \ \va °vele ca\lem  \msCa\msCc\msNa\msNb\msP; °velā ca \msCb, °velo ca \msNc, °vele va \Ed
        \ \vb °dharmaṁ samīhatā\lem  \msCa\msCb\msNa\msNc\msP; °dharmasamīhatā \msCc\msNb, °dharmmaḥ samīhitaḥ \Ed}

anārambhasya āhāraṁ kuryān nityam ayācitam\thinspace{\dandab} \dontdisplaylinenum

parair dattaṁ tu yo bhuṅkte tam ayācitam ucyate \veg\dontdisplaylinenum


\alalalfejezet{upavāsaḥ}
\varr{
        \ \vb kuryān ni°\lem  \mssCaCbCc\msNa\msNb\msP\Ed; kuryā ni° \msNc
        \ \vc parair dattaṁ tu\lem  \msCa\msCb\msNa\msP; parai dattañ ca \msCc, parai dattan tu \msNb,
                                                                parair dantan tu \msNc\Ed
        \ \vd tam ayāci°\lem  \mssCaCbCc\msNa\msNb\msNc\Ed; nam ayāci° \msPacorr, \uncl{tam a}yāci° \msPpcorr}

bhakṣyaṁ bhojyaṁ ca lehyaṁ ca coṣyaṁ peyaṁ ca pañcamam\thinspace{\dandab} \dontdisplaylinenum

na kāṅkṣen nopayuñjīta upavāsaḥ sa ucyate \veg\dontdisplaylinenum


\alalfejezet{niyameṣu maunavratam (9)}\varr{
        \ \va bhakṣyaṁ\lem  \mssCaCbCc\msNb\msNc\msP\Ed; bhakṣya \msNa
        \ \vc kāṅkṣen no°\lem  \msCa\msCb\msNa\msNb\msNc\msP\Ed; kāṅkṣe no° \msCc\oo
                 °yuñjīta\lem  \msCc\msNa\msNb; °{\il}{\il}ta \msCa, °yañjīta \msCb,  °bhuñjīta \msP\Ed, °bhujīta \msNc
         \ \vd °vāsaḥ sa\lem  \mssCaCbCc\msNa\msP\Ed; °vāsa sa \msNb, °vāsasya \msNc}

mithyāpiśunapāruṣyatīkṣṇavāg apralāpanam\thinspace{\dandab} \dontdisplaylinenum

maunapañcakam ity etad dhārayen niyatavrataḥ \veg\dontdisplaylinenum


\alalalfejezet{mithyāvacanam}
\varr{
        \ \va °pāruṣya°\lem  \msCa\msCb\msNa\msNb\msNc\msP; °saṁbhinnā \msCc, °yābhinnā \Ed
        \ \vb °tīkṣṇavāg a°\lem  \conj; 
                                °sp\textsubring{r}ṣṭavāg a° \msCa\msCb\msNa\msNb\msNc\msP, p\textsubring{r}ṣṭavāk a° \msCc, p\textsubring{r}ṣtevāk a° \Ed
        \ \vc maunapañcaka°\lem  \msCa\msCb\msNb; maunaṁ pañcaka° \msCc\msNa\msNc\Ed, maunam pañca° \msP\oo
                 °ty eta°\lem  \mssCaCbCc\msNa\msNb\msNc\msPpcorr\Ed; °ty e° \msPacorr
        \ \vd °rayen ni°\lem  \mssCaCbCc\msNa\msNb\msNc\msP; °rayan ni° \Ed}

asambhūtam ad\textsubring{r}ṣṭaṁ ca dharmāc cāpi bahiṣk\textsubring{r}tam\thinspace{\dandab} \dontdisplaylinenum

anarthāpriyavākyaṁ yat tan mithyāvacanaṁ sm\textsubring{r}tam \veg\dontdisplaylinenum


\alalalfejezet{piśunaḥ}
\varr{
        \ \va °d\textsubring{r}ṣṭaṁ ca\lem  \msCa\msCb\msNa\msNb\msNc\msP\Ed; d\textsubring{r}ṣṭa\uncl{ñ ca} \msCc
        \ \vb dharmāc cāpi\lem  \msCa\msCb\msNa\msNb\msNc\msP; dharmaś cāpi \msCc, dharmaṁ cāpi \Ed\oo
                 bahiṣk\textsubring{r}tam\lem  \msCa\msCb\msNa\msNc\msP; bahiṣk\textsubring{r}taḥ \msCc\Ed, nahiṣk\textsubring{r}taṁ \msNb
        \ \vc anarthā°\lem  \msCa\msCb\msNa\msNb\msNc\msP; anartha° \msCc\Ed
        \ \vcd °vākyaṁ yat tan mi°\lem  \msCa\msCb\msNa\msP;   
                vaktāra taṁ mi° \msCc, vākya yat tan mi° \msNb, vākyaṁ yan tan mi° \msNc\Ed
        \ \vd sm\textsubring{r}tam\lem  \msCa\msCc\Ed\msNa\msNb\msNc\msP; sm\textsubring{r}taḥ \msCb}

paraśrīṁ nābhinandanti parasyaiśvaryam eva ca\thinspace{\dandab} \dontdisplaylinenum

aniṣṭadarśanākāṅkṣī piśunaḥ samudāh\textsubring{r}taḥ \veg\dontdisplaylinenum


\alalalfejezet{pāruṣyam}
\varr{
        \ \va paraśrīṁ nā°\lem  \msCa\msCb\msNa\msNc\msP; parastrī nā° \msCc\Ed, parastrīn nā° \msNb\oo
                 °bhinandanti\lem  \msCa\msNa\msNb\msNc\msP\Ed; °bhinnandanti \msCc, °bhinanti \msCb
        \ \vb parasyaiśvarya°\lem  \msCa\msCc\msNa\msNb\msNc\msP\Ed; parasaiśvarya° \msCb
        \ \vc °darśanā°\lem  \msCa\msCb\msNa\msNc\msP\Ed; °da\uncl{bbha}nā° \msCc, °darśanāṁ \msNb
        \ \vd piśunaḥ\lem  \msCa\msCb\msNa\msNb\msNc\msP\Ed; piśuna \msCc}

m\textsubring{r}tamātā pitā caiva hānisthānaṁ kathaṁ bhavet\thinspace{\dandab} \dontdisplaylinenum

bhuṅkṣva kāmam am\textsubring{r}ṣṭānāṁ pāruṣyaṁ samudāh\textsubring{r}tam \veg\dontdisplaylinenum


\alalalfejezet{tīkṣṇavāk}
\varr{
        \ \va m\textsubring{r}ta°\lem  \mssCaCbCc\msNa\msNb\msNc\msPacorr\Ed; m\textsubring{r}tā° \msPpcorr
        \ \vb °sthānaṁ\lem  \msCa\msNa\msNb\msNc\msP\Ed; °sthāna \msCb\msCc
        \ \vc bhuṅkṣva\lem \msNc\msP; bhuktva \msCa, bhuktvā \msCb\msCc, bhuṁ\uncl{kṣa} \msNa, bhukṣa \msNb, 
                                                bhuktā \Ed\oo
                 kāmam am\textsubring{r}ṣṭānāṁ\lem  \msCa\msNa\msNc\msP\Ed; kāmasusam\textsubring{r}ṣtānāṁ \msCc, kamam as\textsubring{r}ṣṭānāṁ \msCb,
                                                kāmam um\textsubring{r}ṣtānā \msNb}

h\textsubring{r}di na sphuṭase mūḍha śiro vā na vidāryase\thinspace{\dandab} \dontdisplaylinenum

evam ādīny anekāni tīkṣṇavādī sa ucyate \veg\dontdisplaylinenum


\alalalfejezet{asatpralāpaḥ}
\varr{
        \ \va sphuṭase\lem  \mssCaCbCc\msNa\msNc\msP\Ed; sphuṭaya \msNb}

dyūtabhojanayuddhaṁ ca madyastrīkatham eva ca\thinspace{\dandab} \dontdisplaylinenum

asatpralāpaḥ pañcaitat kīrtitaṁ me dvijottama \veg\dontdisplaylinenum
\varr{
        \ \va °yuddhaṁ\lem  \mssCaCbCc\msNa\msNb\msNc\msP; °yuddhaś \Ed
        \ \vb °katha°\lem  \msNb\msNc; °kaṣa° \mssCaCbCc\msNa\msP, °karṣa° \Ed
        \ \vcd pañcaitat kī°\lem  \mssCaCbCc\msNa\msP\Ed; pañcaite kī° \msNb, pañcetat kī° \msNc
        \ \vd me\lem  \mssCaCbCc\msNa\msNb\msNc\msP; te \Ed}

maunam eva sadā kāryaṁ vākyasaubhāgyam icchatā\thinspace{\dandab} \dontdisplaylinenum

apāruṣyam asambhinnaṁ vākyaṁ satyam udīrayet \veg\dontdisplaylinenum
\varr{
        \ \va kāryaṁ\lem  \mssCaCbCc\msNa\msNc\msP\Ed; kāryā \msNb
        \ \vb vākya°\lem  \msCa\msCb\msNa\msNc\msP\Ed; vākyaṁ \msCc\msNb\oo
                 °saubhāgya°\lem  \msCa\msCc\msNa\msNb\msNc\msP\Ed; °saubhārya° \msCb
        \ \vc °bhinnaṁ\lem  \msCa\msCb\msNa\msNb\msNc\msP; °bhinna \msCc, °digdhaṁ \Ed}

yas tu maunasya no kartā dūṣitaḥ sa kulādhamaḥ\thinspace{\dandab} \dontdisplaylinenum

janme janme ca durgandho mūkaś caivopajāyate \veg\dontdisplaylinenum
\varr{
        \ \vb dūṣitaḥ\lem  \msCa\msCb\msNa\msNb\msNc\msP; dūṣita \msCc, bhūṣitaḥ \Ed
        \ \vc janme janme\lem  \msCb\msCc\msNa\Ed; janma janma \msCa\msNb\msNc\msP\oo
                 durgandho\lem  \msCa\msNb\msNc\msP; duggandho \msCb, durgandhā \msCc, dugandho \msNa, d\textsubring{r}gandho \Ed}

\ujvers\nemsloka 
tasmān maunavrataṁ sadaiva sud\textsubring{r}ḍhaṁ kurvīta yo niścitaṁ
\dontdisplaylinenum

\nemslokab 
vācā tasya alaṅghyatā ca bhavati sarvāṁ sabhāṁ nandati \danda\dontdisplaylinenum

\nemslokac 
vaktrāc cotpalagandham asya satataṁ vāyanti gandhotkaṭāḥ
\dontdisplaylinenum

\nemslokad 
śāstrānekasahasraśo giri naraḥ proccāryate nirmalam \veg\dontdisplaylinenum

\vers


\alalfejezet{niyameṣu snānam (10)}\varr{
        \ \va tasmān mau°\lem  \msCc\msNb\msNc\msP\Ed; {\il}{\il}t mau° \msCa, tasmāt mau° \msCb\msNa\oo
                 sadaiva\lem  \msCa\msCb\msNa\msP\Ed; sadeva \msCc\msNc, sudaitya \msNb\oo
                 kurvīta yo niścitam\lem  \msCa\msCb\msNc\msP\Ed; kurvanti yen niścitam \msCc\msNa, 
                                                        kurvanti yo nniścita \msNb
        \ \vb alaṅghyatā ca\lem  \msCa\msCb\msNa\msNb\msP; alaṁghyatāñ ca \msCc\msNc\Ed\oo
                 sarvāṁ sabhāṁ\lem  \msCa\msNa\msP\Ed; sarvā sabhā \msCb\msNc, sarvaḥ sabhān \msCc, 
                                                                sarvā subhā \msNb
        \ \vc vaktrāc cotpalagandham asya\lem  \msCa\msCb\msNc\msPacorr; vaktraṁ cotpalam asya \msCc,
                       vaktraṁ cotpalagandham asya \msNa, vaktraṁ cotpala\uncl{ga}ndham asya \msNb,
                       vaktrāś cotpalagandham asya \msPpcorr,
                       vaktrāc cotaragandham asya \Ed
        \ \vd °sahasraśo\lem  \msCa\msCc\msNa\msNb\msNc\msP\Ed; °sahasrāśo \msCb\oo
                 °malam\lem  \msCa\msNa\msNb\msNc\msP; °malaḥ \msCb\msCc\Ed}

snānaṁ pañcavidhaṁ caiva pravakṣyāmi yathātatham\thinspace{\dandab} \dontdisplaylinenum

āgneyaṁ vāruṇaṁ brāhmyaṁ vāyavyaṁ divyam eva ca \veg\dontdisplaylinenum


\alalalfejezet{āgneyaṁ snānam}
\varr{
        \ \va pañcavidhaṁ\lem  \msCa\msCc\msNa\msNb\msNc\msP\Ed; pañcavi \msCb
        \ \vb yathātatham\lem  \msCb\msCc\msNa\msNb\msNc\msP\Ed; {\il}{\il}tatham \msCa
        \ \vc āgneyaṁ\lem  \mssCaCbCc\msNa\msNc\msP\Ed; āgneye \msNb\oo
                 vāruṇaṁ\lem  \mssCaCbCc\msNa\msNb\msNc\msP; brāhmaṇaṁ \Ed\oo
                 brāhmyaṁ\lem  \mssCaCbCc\msNa\msNb\msP\Ed; brahmyaṁ \msNc}

āgneyaṁ bhasmanā snānaṁ toyāc chataguṇaṁ phalam\thinspace{\dandab} \dontdisplaylinenum

bhasmapūtaṁ pavitraṁ ca bhasma pāpapraṇāśanam \veg\dontdisplaylinenum
\varr{
        \ \va snānaṁ\lem  \mssCaCbCc\msNapcorr\msNb\msNc\msP\Ed; snānā \msNaacorr
        \ \vb °guṇaṁ\lem  \mssCaCbCc\msNa\msNb\msP\Ed; °guṇa° \msNc}

tasmād bhasma prayuñjīta dehināṁ tu malāpaham\thinspace{\dandab} \dontdisplaylinenum

sarvaśāntikaraṁ bhasma bhasma rakṣakam uttamam \veg\dontdisplaylinenum
\varr{
        \ \va tasmād bhasma prayuñjīta\lem  \mssCaCbCc\msNa\msNc\msP\Ed; {\il}{\il}{\il}{\il}{\il}{\il}{\il}ta \msNb}

bhasmanā tryāyuṣaṁ k\textsubring{r}tvā brahmacaryavrate sthitam\thinspace{\dandab} \dontdisplaylinenum

bhasmanā \textsubring{r}ṣayaḥ sarve pavitrīk\textsubring{r}tam ātmanaḥ \veg\dontdisplaylinenum
\varr{
        \ \va tryāyuṣaṁ k\textsubring{r}tvā\lem  \msCb\msCc\msNa\msNb\msNc\Ed; tryāyu{\il}{\il}{\il} \msCa, tryāyuṣyaṁ k\textsubring{r}tvā \msP
        \ \vb °vrate\lem  \mssCaCbCc\msNa\msNb\msNc\msP; °vrata° \Ed
        \ \vc \textsubring{r}ṣayaḥ sarve\lem  \mssCaCbCc\msNa\msNb\msNc\msP; \textsubring{r}ṣibhir sarvaiḥ \Ed}

bhasmanā vibudhā muktā vīrabhadrabhayārditāḥ\thinspace{\dandab} \dontdisplaylinenum

bhasmānuśaṁsaṁ d\textsubring{r}ṣṭvaiva brahmaṇānumatiḥ k\textsubring{r}tā \veg\dontdisplaylinenum
\varr{
        \ \va muktā\lem  \mssCaCbCc\msNa\msNb\msNc\msP; muktāḥ \Ed
        \ \vb °rditāḥ\lem  \msCa\msCc\msNa\msNb\msNc\msP\Ed; °rttitāḥ \msCb
        \ \vc bhasmānuśaṁsaṁ d\textsubring{r}ṣṭvaiva\lem  \corrTorzsok;
                      bhasmānusaṁsaṁ d\textsubring{r}ṣṭyaiva \msCa, bhasmānuśaṁsāṁ d\textsubring{r}ṣṭvava \msCb, 
                      bhasmānusaṁsad\textsubring{r}ṣṭaiva \msCc\msNb, bhasmānusaṁsand\textsubring{r}ṣṭvaiva \msNa, 
                                      bhasmānuśaṁsaṁd\textsubring{r}ṣṭyaivaṁ \msNc, bhasmānuśaṁsaṁ d\textsubring{r}ṣṭaiva \msP,
                                                                bhasmanā śaṁ prad\textsubring{r}śyaivaṁ \Ed
        \ \vd brahmaṇānumatiḥ\lem  \eme; brahmaṇānumatā \mssCaCbCc\msNa\msNb\msNc\msP, brāhmaṇānumato \Ed\oo
                 k\textsubring{r}tā\lem  \eme; k\textsubring{r}taḥ \msCa\msCb\msNb\msNc\msP\Ed, k\textsubring{r}tiḥ \msCc, k\textsubring{r}tāḥ \msNa}

caturāśramato 'dhikyaṁ vrataṁ pāśupataṁ k\textsubring{r}tam\thinspace{\dandab} \dontdisplaylinenum

tasmāt pāśupataṁ śreṣṭhaṁ bhasmadhāraṇahetutaḥ \veg\dontdisplaylinenum


\alalalfejezet{vāruṇaṁ snānam}
\varr{
        \ \va caturāśramato\lem  \msCb\msCc\msNb\msP\Ed; cāturāśramato \msCa\msNc, caturāśrato \msNaacorr,
                                                                              cāturāśramato \msNapcorr
        \ \vab 'dhikyaṁ vrataṁ pāśupataṁ k\textsubring{r}tam\lem  \mssCaCbCc\msNa\msNc\msP\Ed;
                                                        \uncl{dhikyavratapāśupata}{\il}{\il}{\il} \msNb\ \toplost
        \ \vc tasmāt pāśupataṁ śreṣṭhaṁ\lem  \mssCaCbCc\msNa\msNc\msP\Ed; \om \msNb
        \ \vd °hetutaḥ\lem  \emeTorzsok; °hetavaḥ \msCa\msCb\msNa\msNc\msP\Ed, °hetunā \msCc, °hetunutaḥ \msNb}

vāruṇaṁ salilaṁ snānaṁ kartavyaṁ vividhaṁ naraiḥ\thinspace{\dandab} \dontdisplaylinenum

nadītoyataḍāgeṣu prasraveṣu hradeṣu ca \veg\dontdisplaylinenum


\alalalfejezet{brāhmyaṁ snānam}
\varr{
        \ \va vāruṇaṁ\lem  \msCb\msCc\msNa\msNb\msP\Ed; vā{\il}{\il} \msCa, vāruṇā \msNcacorr, vāruṇa \msNcpcorr\oo
                 salilaṁ\lem  \mssCaCbCc\msNa\msNb\msP; salila° \msNc\Ed
        \ \vb vividhaṁ naraiḥ\lem  \mssCaCbCc\msNa; vidhivan naraiḥ \msNc\msP\Ed, 
                                                             vivin naraiḥ \msNb
        \ \vc °taḍāgeṣu\lem  \mssCaCbCc\msNa\msNc\msP\Ed; °taḍāgevā \msNb
        \ \vd prasraveṣu\lem  \mssCaCbCc\msNa\msP\Ed; prayeveṣu \msNb, prabhaveṣu \msNc}

brahmasnānaṁ ca viprendra āpohiṣṭhaṁ vidur budhāḥ\thinspace{\dandab} \dontdisplaylinenum

trisaṁdhyam eva kartavyaṁ brahmasnānaṁ tad ucyate \veg\dontdisplaylinenum


\alalalfejezet{vāyavyaṁ snānam}
\varr{
        \ \va viprendra\lem  \mssCaCbCc\msNa\msNb\Ed; vipendra \msNc\msP
        \ \vb vidur bu°\lem  \mssCaCbCc\msNa\msNb\msP\Ed; virdur bu° \msNc}

goṣu saṁcāramārgeṣu yatra godhūlisambhavaḥ\thinspace{\dandab} \dontdisplaylinenum

tatra gatvāvasīdeta snānam uktaṁ manīṣibhiḥ \veg\dontdisplaylinenum


\alalalfejezet{divyaṁ snānam}
\varr{
        \ \vd °ktaṁ\lem  \mssCaCbCc\msNa\msNc\msP\Ed; °kta \msNb}

varṣatoyāmbudhārābhiḥ plāvayitvā svakāṁ tanum\thinspace{\dandab} \dontdisplaylinenum

snānaṁ divyaṁ vadaty eva jagadādimaheśvaraḥ \veg\dontdisplaylinenum
\varr{
        \ \vb tanum\lem  \mssCaCbCc\msNa\msNb\msP\Ed; tanaṁ \msNc
        \ \vc divyaṁ\lem  \mssCaCbCc\msNa\msNc\msP\Ed; divya \msNb
        \ \vd jagadādi°\lem  \msCa\msCc\msNa\msNb\msNc\msP\Ed; gajadādi° \msCb}

\ujvers\nemsloka 
iti niyamavibhāgaḥ pañcabhedena vipra
\dontdisplaylinenum

\nemslokab 
nigadita tava p\textsubring{r}ṣṭaḥ sarvalokānukampya \danda\dontdisplaylinenum

\nemslokac 
sakalamalapahārī dharmapañcāśad etan
\dontdisplaylinenum

\nemslokad 
na bhavati punajanma kalpakoṭyāyute 'pi \veg\dontdisplaylinenum

\vers


\jump
\begin{center}
\ketdanda iti v\textsubring{r}ṣasārasaṁgrahe niyamapraśaṁsā nāmādhyāyo 'ṣṭamaḥ\ketdanda
\end{center}
\dontdisplaylinenum\vers 
\varr{
        \ \va °bhāgaḥ\lem  \mssCaCbCc\msNa\msNb\msP\Ed; °bhāgaṁ \msNc
        \ \vb nigadita tava\lem  \Ed; nigaditas tava \mssCaCbCc\msNa\msNb\msNc\msP\ \unmetr\oo
                 °kampya\lem  \msCa; °kampa \msCb\msCc\msNa\msNc\msP, °kampaḥ \msNb, °kampyaḥ \Ed
        \ \vc °pahārī\lem  \msCb\msCc\msNb; °pahāri \msCa\msNc\unmetr, °prahāri \msNa\msP, °pahāre \Ed\oo
                 °pañcāśad etan\lem  \msCa\msCb\msNa\msNbpcorr\msNc\msP; °pañcāśam etan \msCc\Ed,
                                                °pañcād etan \msNbacorr
        \ \vd punajanma\lem  \msCc\msNb; punarjanma \msCa\msNa\msNc\msP\Ed, puna\uncl{rjarma} \msCb
        \ \Colo:  iti v\textsubring{r}ṣasārasaṁgrahe niyamapraśaṁsā nāmādhyāyo 'ṣṭamaḥ\lem  \msP;
                               iti v\textsubring{r}ṣasārasaṁgrahe niyamapraśaṁsā nāmādhyāya aṣṭamaḥ \msCa\msNa, 
                                                                                 \om \msCb,
                               iti v\textsubring{r}ṣasārasaṁgrahe niyamapraśaṁsā nāmādhyāyāṣṭamaḥ \msCc\msNb, 
                               iti v\textsubring{r}ṣasārasaṁgrahe niyamapraśaṁsā nāmādhyāyā'ṣṭamaḥ \msNc, 
                               iti v\textsubring{r}ṣasārasaṁgrahe niyamapraśaṁsā nāma aṣṭamo 'dhyāyaḥ \Ed}
\bekveg\szamveg\vfill\phpspagebreak\szam\bek\versno=0\fejno=9
\thispagestyle{empty}



\alfejezet{\textbf{navamo 'dhyāyaḥ}}\jump\jump

\alalfejezet{traiguṇyam}
\vers

[anarthayajña uvāca~{\dandab}\dontdisplaylinenum ]

trikālaguṇabhedena bhinnaṁ sarvacarācaram\thinspace{\danda} \dontdisplaylinenum

tasmāt triguṇabandhena veṣṭitaṁ nikhilaṁ jagat \veg\dontdisplaylinenum
\varr{
        \ \va trikāla°\lem  \msCa\msCb\msNa\msNb\msNc\Ed; triṣkāla° \msCc\oo
                 °bhedena\lem  \mssCaCbCc\msNa\msNbpcorr\msNc\Ed; °bhena \msNbacorr
        \ \vb bhinnaṁ\lem  \mssCaCbCc\msNa\msNc\Ed; bhinna \msNb
        \ \vc tasmāt tri°\lem  \msCa\msCb\msNa\msNb\Ed; tasmā tri° \msCc\msNc}

vigatarāga uvāca~{\dandab}\dontdisplaylinenum 

traikālyam iti kiṁ jñeyaṁ traidhātukaśarīriṇaḥ\thinspace{\danda} \dontdisplaylinenum

kiṁcid vistaram eveha kathayasva tapodhana \veg\dontdisplaylinenum
\varr{
        \ \va °kālyam\lem  \msCb\msCc\msNa\msNb\Ed; °kālam \msCa\msNc
        \ \vab kiṁ jñeyaṁ trai°\lem  \msCa\msNc; vijñeyaṁ trai° \msCb\msNa\msNb\Ed, ki jñeyam trai° \msCc
        \ \vb °dhātuka°\lem  \mssCaCbCc\msNa\msNb\msNc; °dhāyukta° \Ed
        \ \vc kiṁci°\lem  \msCa\msCbpcorr\msCc\msNa\msNb\msNc\Ed; sāttviko bhagav  viṣṇu rājasaḥ kamalodbhavaḥ{\danda}  
                                                  tāmaso bhagavān īśaḥ sakalaṁ vika kiñci° \msCbacorr\ 
                                                   \eyeskip{9.5}\oo
                 °veha\lem  \mssCaCbCc\msNa\msNb\msNc; °tad dhi \Ed
        \ \vd kathayasva\lem  \msCb\msCc\msNa\msNb\msNc\Ed; ka{\il}{\il}{\il} \msCa}

anarthayajña uvāca~{\dandab}\dontdisplaylinenum 

traikālyaṁ triguṇaṁ jñeyaṁ vyāpī prak\textsubring{r}tisambhavaḥ\thinspace{\danda} \dontdisplaylinenum

anyonyam upajīvanti anyonyam anuvartinaḥ \veg\dontdisplaylinenum
\varr{
        \ \va °kālyaṁ\lem  \msCa\msCb\msNa\msNb\msNc\Ed; °kālya \msCc\oo
                 °guṇaṁ\lem  \msCa\msCb\msNa\msNb\msNc\Ed; °guṇa \msCc}

sattvaṁ rajas tamaś caiva rajaḥ sattvaṁ tamas tathā\thinspace{\dandab} \dontdisplaylinenum

tamaḥ sattvaṁ rajaś caiva anyonyamithunāḥ sm\textsubring{r}tāḥ \veg\dontdisplaylinenum
\varr{
        \ \va sattvaṁ\lem  \mssCaCbCc\msNa\msNc\Ed; satva \msNb\oo
                 rajas ta°\lem  \mssCaCbCc\msNa\msNb\msNc; rajata° \Ed
        \ \vb rajaḥ\lem  \msCa\msCb\msNa\msNc; raja° \msCc\msNb\Ed\oo
                 sattvaṁ tamas tathā\lem  \msCa\msNa\msNc; sattvaṁ taman tathā \msCb, 
                                        satvas tamas tathā \msCc\msNb, sattvatamas tathā \Ed
        \ \vc tamaḥ sattvaṁ\lem  \msCa\msCb\msNa\msNc; tamasatva° \msCc, tamaḥ satva° \msNb\Ed\oo
                 rajaś caiva\lem  \msCa\msCc\msNa\msNb\msNc\Ed; rajaḥś caiva \msCb
        \ \vd sm\textsubring{r}tāḥ\lem  \msCa\msCb\msNa\msNb\msNc\Ed; \om\ \msCc}

sāttviko bhagavān viṣṇū rājasaḥ kamalodbhavaḥ\thinspace{\dandab} \dontdisplaylinenum

tāmaso bhagavān īśaḥ sakalaṁ vikaleśvaraḥ \veg\dontdisplaylinenum
\varr{
        \ \va °ṣṇū\lem  \corr; °ṣṇu \mssCaCbCc\msNa\msNb\msNc\Ed
        \ \vb rājasaḥ kamalodbhavaḥ\lem  \msCb\msCc\msNa\msNb\msNc\Ed; \uncl{rāja}{\il}{\il}{\il}{\il}{\il}{\il} \msCa
        \ \vcd tāmaso bhagavān īśaḥ sakalaṁ\lem  \msCb\msCc\msNa\msNb\msNc\Ed; {\il}{\il}{\il}{\il}{\il}{\il}{\il}{\il}\uncl{sakalam} \msCa}

sattvaṁ kundenduvarṇābhaṁ padmarāganibhaṁ rajaḥ\thinspace{\dandab} \dontdisplaylinenum

tamaś cāñjanaśailābhaṁ kīrtitāni manīṣibhiḥ \veg\dontdisplaylinenum
\varr{
        \ \va sattvaṁ\lem  \msCa\msCb\msNa\msNb\Ed; satva \msCc\msNc\oo
                 °varṇābhaṁ\lem  \msCa\msCb\msNa\msNb\msNc\Ed; °varṇṇābha \msCc, °vaṇṇābhaṁ \msNa
        \ \vc °bhaṁ\lem  \mssCaCbCc\msNa\msNb\msNc; °bhā \Ed}

sattvaṁ jalaṁ rajo 'ṅgāraṁ tamo dhūmasamākulam\thinspace{\dandab} \dontdisplaylinenum

etadguṇamayair baddhāḥ pacyante sarvadehinaḥ \veg\dontdisplaylinenum
\varr{
        \ \va jalaṁ\lem  \msCa\msCb\msNa\msNc\Ed; rajaṁ \msCc, jvāla \msNb\oo
                 rajo 'ṅgāraṁ\lem  \msCa\msCb\msNa\msNb\msNc; ra\uncl{ṅgo}ṅgāran \msCc, rajoṅgaran \Ed
        \ \vd °dehinaḥ\lem  \msCa\msCc\msNa\msNb\msNc\Ed; °dehinā \msCb}

vigatarāga uvāca~{\dandab}\dontdisplaylinenum 

kena kena prakāreṇa guṇapāśena badhyate\thinspace{\danda} \dontdisplaylinenum

cihnam eṣāṁ p\textsubring{r}thaktvena kathayasva tapodhana \veg\dontdisplaylinenum
\varr{
        \ \vb guṇa°\lem  \msCb\msCc\msNa\msNb\msNc\Ed; \om\ \msCa
        \ \vc °ṣāṁ p\textsubring{r}thaktvena\lem  \mssCaCbCc\msNa\msNb\Ed; °ṣā p\textsubring{r}thakena \msNc}

anarthayajña uvāca~{\dandab}\dontdisplaylinenum 

anekākārabhāvena badhyante guṇabandhanaiḥ\thinspace{\danda} \dontdisplaylinenum

mohitā nābhijānanti jānanti śivayoginaḥ \veg\dontdisplaylinenum
\varr{
        \ \vc °bhijānanti\lem  \msCa\msCb\msNa\msNb\msNc\Ed; °bhijānānti \msCc
        \ \vd jānanti\lem  \msCa\msCbpcorr\msCc\msNa\msNb\msNc\Ed; \om\ \msCbacorr}

ūrdhvaṁgo nityasattvastho madhyago rajasāv\textsubring{r}taḥ\thinspace{\dandab} \dontdisplaylinenum

adhogatis tamo'vasthā bhavanti puruṣādhamāḥ \veg\dontdisplaylinenum
\varr{
        \ \va ūrdhvaṁgo nitya\lem  \conj;
                      ūrdhvāṅgo nitya° \mssCaCbCc\msNapcorr\Ed,
                      urdhvāṅgo nitya° \msNc,
                      ūrdhvāṅgā natya° \msNaacorr, 
                      ūrdhvago sitya° \msNbacorr,
                      ūrdhvago satya° \msNbpcorr\oo
                 °sattva°\lem  \msCa\msCb\msNa\msNc; °satya° \msCc\Ed, °nitya° \msNb
        \ \vb madhyago\lem  \mssCaCbCc\msNb\msNa\msNc; madhyamo \Ed\oo
                 °v\textsubring{r}taḥ\lem  \mssCaCbCc\msNa\msNb\msNc; °v\textsubring{r}tam \Ed
        \ \vc °gatis tamo°\lem  \msCa\msNa\msNb\msNc\Ed; °gatitamo° \msCb\msCc}

svarge 'pi hi trayo vaite bhāvanīyās tapodhana\thinspace{\dandab} \dontdisplaylinenum

mānuṣeṣu ca tiryeṣu guṇabhedās trayas trayaḥ \veg\dontdisplaylinenum


\alalalfejezet{sāttvikottamāḥ}
\varr{
        \ \vc mānuṣeṣu\lem  \msCa\msCc\msNa\msNb\Ed; manuṣyeṣu \msCb, mānuṣyeṣu \msNc\oo
                 tiryeṣu\lem  \mssCaCbCc\msNa\msNb\msNc; tīryeṣu \Ed
        \ \vd °s trayaḥ\lem  \msCa\msCbpcorr\msCc\msNa\msNb\msNc\Ed; °s traḥ \msCbacorr}

brahmā viṣṇuś ca rudraś ca dharma indraḥ prajāpatiḥ\thinspace{\dandab} \dontdisplaylinenum

somo 'gnir varuṇaḥ sūryo daśa sattvottamāḥ sm\textsubring{r}tāḥ \veg\dontdisplaylinenum


\alalalfejezet{sāttvikamadhyamāḥ}
\varr{
        \ \vb dharma indraḥ\lem  \msCa\msCc\msNa\msNb\msNc; irma indra \msCb, dharmar indra° \Ed
        \ \vc gnir varuṇaḥ\lem  \msCa\msNa\msNc; gni varuṇa \msCb\msCc\msNb\Ed
        \ \vd daśa\lem  \mssCaCbCc\msNa\msNb\msNc; daśaḥ \Ed\oo
                 sattvottamāḥ\lem  \msCa\msCc\msNa\msNb\Ed; satvattamāḥ \msCb, sattotamāḥ \msNc}

rudrādityā vasusādhyā viśveśamaruto dhruvaḥ\thinspace{\dandab} \dontdisplaylinenum

\textsubring{r}ṣayaḥ pitaraś caiva daśaite sattvamadhyamāḥ \veg\dontdisplaylinenum


\alalalfejezet{sāttvikādhamāḥ}
\varr{
        \ \vab °dityā vasusādhyā\lem  \msCb\msNa\msNb\msNc; °dityā vasusā{\il} \msCa, °ditya vasusādhyā \msCc,
                                                        °ditya vasusādhyāḥ vi° \Ed
        \ \vb viśveśa°\lem  \msCb\msNa\msNb\msNc\Ed; {\il}śveśa \msCa, viśveśi° \msCc
        \ \vd daśaite\lem  \msCa\msCbpcorr\msCc\msNa\msNb\msNc\Ed; daśaitete \msCbacorr}

tārā grahāḥ surā yakṣā gandharvāḥ kiṁnaroragāḥ\thinspace{\dandab} \dontdisplaylinenum

rakṣobhūtapiśācāś ca daśaite sāttvikādhamāḥ \veg\dontdisplaylinenum


\alalalfejezet{rājasottamāḥ}
\varr{
        \ \va grahāḥ surā\lem  \msCa\msCb\msNa\msNb\msNc; grahāsvarāḥ \msCc, grahā'surā \Ed
        \ \vb gandharvāḥ\lem  \msCa\msNa\msNb\msNc\Ed; gandharvā \msCb\msNa, gandharvvāḥ gandharvvā \msCc
        \ \vc °piśācāś ca\lem  \mssCaCbCc\msNa\msNb\Ed; °piśāścāś ca \msNc
        \ \vd daśaite\lem  \msCa\msCc\msNa\msNb\msNc\Ed; daśete \msCb\oo
                 sāttvikā°\lem  \msCa\msCc\msNa\msNb\msNc\Ed; satvakā° \msCb}

\textsubring{r}tvik purohitācāryayajvāno 'tithivijñanī\thinspace{\dandab} \dontdisplaylinenum

rājamantrī vratī vedī daśaite rājasottamāḥ \veg\dontdisplaylinenum


\alalalfejezet{jātayo rājasādhamāḥ}
\varr{
        \ \vb °vijñanī\lem  \mssCaCbCc\msNa\msNb\msNc; °vijñakau \Ed
        \ \vc °mantrī vratī\lem  \mssCaCbCc\msNa\msNb\msNc; °mantri vrato \Ed
        \ \vd rājaso°\lem  \msCa\msCc\msNa\msNb\msNc\Ed; rāmaso \msCb}

sūto 'mbaṣṭavaṇik cograḥ śilpikārukamāgadhāḥ\thinspace{\dandab} \dontdisplaylinenum

veṇavaidehakāmātyā daśaite rajamadhyamāḥ \veg\dontdisplaylinenum
\varr{
        \ \va sūto 'mbaṣṭa°\lem  \Ed; sūto {\il}ṣṭa° \msCa, sūta\uncl{mbaṣṭa}° \msCb, sūtonvaṣṭha° \msCc, 
                                                                        sūtotvaṣṭā° \msNa, sūtotvaṣṭa° \msNb\msNc\oo
                 °vaṇik co°\lem  \corr; °vaṇiś co° \mssCaCbCc\msNa\msNb\msNc, °vaṇiśvo° \Ed
        \ \vb śilpi°\lem  \msNb; śilpa° \mssCaCbCc\msNa\msNc\Ed\oo
                 māgadhāḥ\lem  \msCa\msCb\msNa\msNb\msNc\Ed; māgadhā \msCc
        \ \vc veṇavaidehakāmātyā\lem  \msCa\msCc\msNa\msNb; vaiṇavedehakāmātyā \msCb, 
                                      venavaidehakāmātyā \msNc, veṇavaidecakau mātyā \Ed}

carmak\textsubring{r}tkumbhak\textsubring{r}tkolī lohak\textsubring{r}ttrapunīlikāḥ\thinspace{\dandab} \dontdisplaylinenum

naṭamuṣṭikacaṇḍālā daśaite rajasādhamāḥ \veg\dontdisplaylinenum


\alalalfejezet{tāmasottamāḥ}
\varr{
        \ \va °k\textsubring{r}tkolī\lem  \mssCaCbCc\msNb\msNc; °kakolī \msNa, °k\textsubring{r}tkālī \Ed
        \ \vb °nīlikāḥ\lem  \mssCaCbCc\msNa\msNb\msNc; °tīlikā \Ed
        \ \vc °muṣṭika°\lem  \msCa\msCb\msNa\msNb\msNc\Ed; °mauṣṭika° \msCc\oo
                 °caṇḍālā\lem  \mssCaCbCc\msNa\msNb\msNc; °cāṇḍālaḥ \Ed
        \ \vd daśaite\lem  \msCa\msCc\msNa\msNb\msNc\Ed; daśete \msCb}

gogajagavayā aśvam\textsubring{r}gacāmarakiṁnarāḥ\thinspace{\dandab} \dontdisplaylinenum

siṁhavyāghravarāhāś ca daśaite tāmasottamāḥ \veg\dontdisplaylinenum


\alalalfejezet{tāmasamadhyamāḥ}
\varr{
        \ \va °gavayā\lem  \mssCaCbCc\msNa\msNc; °gavaya \msNb, °gavayo \Ed
        \ \vb °cāmara°\lem  \msCa\msCb\msNa\msNc; °vānara° \msCc\Ed, °\uncl{vā}nara° \msNb
        \ \vc °varāhā°\lem  \mssCaCbCc\msNa\msNc; °varāha° \msNb\Ed
        \ \vd tāmasottamāḥ\lem  \msCa\msCc\msNa\msNb\msNc; tāmaśottamaḥ \msCb, tamasottamāḥ \Ed}

ajameṣamahiṣyāś ca mūṣikānakulādayaḥ\thinspace{\dandab} \dontdisplaylinenum

uṣṭraraṅkuśaśagaṇḍā daśaite tamamadhyamāḥ \veg\dontdisplaylinenum


\alalalfejezet{tāmasādhamāḥ}
\varr{
        \ \va °mahiṣyāś ca\lem  \mssCaCbCc\msNa\msNc\Ed; °mahiṁṣyā ca \msNb
        \ \vc uṣṭra°\lem  \msCa\msCb\msNa\msNb\msNc; uṣṭa° \msCc, daṁṣṭri° \Ed\oo
                 °śaśagaṇḍā\lem  \mssCaCbCc\msNa\msNb\msNc; °śagaṇḍāś ca \Ed
        \ \vd tamamadhyamāḥ\lem  \msCb\msCc\msNa\msNb\msNc\Ed; tamadhyamāḥ \msCa}

\textsubring{r}kṣagodhām\textsubring{r}gaś\textsubring{r}ṅgibakavānaragardabhāḥ\thinspace{\dandab} \dontdisplaylinenum

sūkaraśvānagomāyur daśaite tāmasādhamāḥ \veg\dontdisplaylinenum


\alalalfejezet{tamasāttvikāḥ}
\varr{
        \ \vb °gardabhāḥ\lem  \mssCaCbCc\msNa\msNb\msNc; °gardabhaḥ \Ed
        \ \vc sūkara°\lem  \msCa\msCc\msNa\msNb\msNc\Ed; sukhara° \msCb
        \ \vcd °gomāyur da°\lem  \mssCaCbCc\msNc\Ed; °gomāyu da° \msNa\msNb
        \ \vd °śaite\lem  \msCa\msCc\msNa\msNb\msNc\Ed; °śete \msCb}

krauñcahaṁsaśukaśyenabhāsabāruṇḍasārasāḥ\thinspace{\dandab} \dontdisplaylinenum

cakrāhvaśukamāyūrā daśaite tamasāttvikāḥ \veg\dontdisplaylinenum


\alalalfejezet{tamarājasāḥ}
\varr{
        \ \va krauñca°\lem  \Ed; kroñca° \mssCaCbCc\msNa\msNb\msNc
        \ \vb °sārasāḥ\lem  \mssCaCbCc\msNa\msNb\Ed; °sārasā \msNc
        \ \vc °hvaśukamāyūrā\lem  \msCb\msCc\msNa\msNb\msNc; °\uncl{ṅga}{\il}{\il}{\il}yūrā \msCa, °ṅgaśukamāyūrā \Ed
        \ \vd daśaite\lem  \msCa\msCc\msNa\msNb\msNc\Ed; daśete \msCb\oo
                 tamasāttvikāḥ\lem  \msCc\msNc\Ed; tamassāttvikāḥ \msCa\msNb\ \unmetr, tamaḥsātvikāḥ \msNa\ \unmetr,
                                         namaḥ sātvikāḥ \msCb\ \unmetr}

balākāḥ kukkuṭāḥ kākāś cillalāvakatittirāḥ\thinspace{\dandab} \dontdisplaylinenum

g\textsubring{r}dhrakaṅkabakaśyena daśaite tamarājasāḥ \veg\dontdisplaylinenum
\varr{
        \ \va balākāḥ\lem  \corr; valākā \msCa\msNa\msNc, valāka° \msCb\msCc\msNb\Ed
        \ \vab kukkuṭāḥ kākāś ci°\lem  \corr; kukkuṭakākāś ci° \msCa\msCb\ \unmetr, kurkuṭā kākāś ci° \msCc\msNc, 
                                                                          kurkuṭakākāś ci \msNa\msNb, kukkuṭo kākā ci° \Ed
        \ \vb °tittirāḥ\lem  \mssCaCbCc\msNa\msNb; °tittarāḥ \msNc, °tittiriḥ \Ed
        \ \vc g\textsubring{r}dhra°\lem  \mssCaCbCc\msNa\msNb\Ed; g\textsubring{r}dha° \msNc}

kokilolūkakiñjalkakapotāḥ pañca eva ca\thinspace{\dandab} \dontdisplaylinenum

śārikāś ca kuliṅgāś ca daśaite tamasādhamāḥ \veg\dontdisplaylinenum
\varr{
        \ \va kokilo°\lem  \msCa\msCc\msNa\msNb\msNc\Ed; kaukilo° \msCb\oo
                 °kiñjalka°\lem  \msCb\msNb\msNc\Ed; °kiñjalya° \msCa\msCc\msNa
        \ \vb ca\lem  \mssCaCbCc\msNa\msNb\Ed; caḥ \msNc
        \ \vc śārikāś ca\lem  \corr; śārikā ca \mssCaCbCc\msNa\msNb\msNc, śālikā ca \Ed\oo
                 kuliṅgāś ca\lem  \corr; kuliṅgā ca \msCa\msNb\Ed, kuliṅkā ca \msCb\msCc\msNc, kulikāṁ ca \msNa}

makaragohanakrāś ca \textsubring{r}kṣāś ca tamasāttvikāḥ\thinspace{\dandab} \dontdisplaylinenum

kacchapa\crux{śuśu}kumbhīramaṇḍūkās tamarājasāḥ \danda\dontdisplaylinenum

śaṅkhaśuktikaśambūka\crux{kabandhyā}s tamatāmasāḥ \veg\dontdisplaylinenum
\varr{
        \ \va °gohanakrāś ca\lem  \msCa\msCb\msNa\msNc\Ed; °gohanakrā ca \msCc, °grohanakrāś ca \msNb
        \ \vb \textsubring{r}kṣāś ca\lem  \conj; \textsubring{r}ṣā ca \mssCaCbCc\msNa\msNb\msNc\Ed\oo
                 tamasāttvikāḥ\lem  \Ed; tama\uncl{ssā}{\il}{\il} \msCa, tamaḥsātvikāḥ \msCb\msCc\msNa\msNb\ \unmetr, 
                                                                        tasamātvikāḥ \msNc
        \ \vc °kumbhīra°\lem  \msCa\msCb\msNa\msNb\msNc; °kambhīrā \msCc\Ed
        \ \vd °maṇḍūkā°\lem  \mssCaCbCc\msNa\msNc; °maṇḍūka° \msNb, °maṇḍukā° \Ed
        \ \ve °śambūka°\lem  \corr; °śambūkā \mssCaCbCc\msNa\msNb\Ed, °\uncl{sa}mbūkāḥ \msNc
        \ \vf °kabandhyā°\lem  \mssCaCbCc\msNa\msNbpcorr\msNc\Ed; °kabana° \msNbacorr\oo
                 °matāmasāḥ\lem  \msCb\Ed; °mastāmasāḥ \msCa\msCc\msNc\ \unmetr, °maḥtāmasāḥ \msNa\msNb\ \unmetr}

candanāgarupadmaṁ ca plakṣodumbarapippalāḥ\thinspace{\dandab} \dontdisplaylinenum

vaṭadāruśamībilvā daśaite tamasāttvikāḥ \veg\dontdisplaylinenum
\varr{
        \ \va °garu°\lem  \mssCaCbCc\msNa\msNb\msNc; °guru° \Ed
        \ \vc °bilvā\lem  \msCa\msCb\msNa\Ed; °bilva \msCc\msNb\msNc
        \ \vd daśaite\lem  \msCa\msCb\msNa\msNb\msNc\Ed; daśai \msCc\oo
                 tamasāttvikāḥ\lem  \Ed; tamassātvikāḥ \msCa\ \unmetr, tamaḥsātvikāḥ \msCb\msCc\msNa\msNb\msNc\ \unmetr}

jāmbīralakucāmrātadāḍimākolavetasāḥ\thinspace{\dandab} \dontdisplaylinenum

nimbanīpo dhravāvaś ca daśaite tamarājasāḥ \veg\dontdisplaylinenum
\varr{
        \ \va jāmbīra°\lem  \msCa\msCb\msNa\msNb\msNc\Ed; jambīra° \msCc
        \ \vb °dāḍimā°\lem  \msCa\msCb\msNb\msNc\Ed; °drāḍimā° \msCc, °drāḍi\uncl{hā}° \msNa
        \ \vc °nīpo\lem  \mssCaCbCc\msNa\msNb\Ed; °nīpau \msNc\oo
                 dhravāvaś ca\lem  \msCaacorr\msCb\msCc\msNa\msNb\msNc; dhavāvaś ca \msCapcorr, dhuvāvaś ca \Ed
        \ \vd daśaite\lem  \msCb\msCc\msNa\msNb\msNc\Ed; {\il}{\il}{\il} \msCa}

v\textsubring{r}kṣavallīlatāveṇutvaksārat\textsubring{r}ṇabhūruhāḥ\thinspace{\dandab} \dontdisplaylinenum

mīrajāś ca śilāśasyā daśaite tamasāttvikāḥ \veg\dontdisplaylinenum
\varr{
        \ \va v\textsubring{r}kṣavallī°\lem  \mssCaCbCc\msNa\msNc\Ed; \uncl{v\textsubring{r}kṣavallī} \msNb
        \ \vb °tvaksāra°\lem  \msCa\msCb\msNa\msNb; °tvaksāras \msCc\Ed, °tvakasāra° \msNc\ \unmetr
        \ \vc mīrajāś ca\lem  \corr; mīrajā ca \msCa\msCc\msNa\msNb\msNc\Ed, mīnajā ca \msCb
        \ \vd tamasāttvikāḥ\lem  \msNc\Ed; tamassātvikāḥ \msCa, 
                                tamaḥsātvikāḥ \msCb\msCc\msNa\ \unmetr, tamaḥsādhikāḥ \msNb\ \unmetr}

bhramarādipataṅgāś ca krimikīṭajalaukasaḥ\thinspace{\dandab} \dontdisplaylinenum

yūkoddaṁśamaśānāṁ ca viṣṭajās tamasāttvikāḥ \veg\dontdisplaylinenum
\varr{
        \ \va pataṅgāś ca\lem  \mssCaCbCc\msNa\msNb\msNc; pataṅgānāṁ \Ed
        \ \vb krimikīṭajalaukasaḥ\lem  \mssCaCbCc\msNa; krimikīṭajalokasaḥ \msNb, 
                                        krimikīṭajalauka\uncl{sāḥ} \msNc, kimikīṭajalaukasāṁ \Ed
        \ \vc yūkoddaṁśamaśānāṁ ca\lem  \msCa;
                      yūkodaṁśamaśānāñ ca \msCb\msNa,
                      yūkodaṁśamasakānāñ ca \msCc\ \unmetr,
                      yūkodaṁśamasānān tu \msNb,
                      \uncl{yūkoddaṁ}{\il}{\il}{\il}{\il}{\il} \msNc,
                      yuktodaṁśamaśānāś ca \Ed
        \ \vd viṣṭajās tamasāttvikāḥ\lem  \corr; 
                             viṣṭajās tamassātvikāḥ \msCa\ \unmetr,
                             viṣṭajās tamaḥsātvikāḥ \msCb\msCc\msNa\ \unmetr,
                             viṣṭajās tamaḥsādhikāḥ \msNb\ \unmetr,
                             {\il}{\il}\uncl{jā}tamassādhikāḥ \msNc\ \unmetr,
                             viṣṭajā tamasāttvikāḥ \Ed}

dayā satyaṁ damaḥ śaucaṁ jñānaṁ maunaṁ tapaḥ kṣamā\thinspace{\dandab} \dontdisplaylinenum

śīlaṁ ca nābhimānaṁ ca sāttvikāś cottamā janāḥ \veg\dontdisplaylinenum
\varr{
        \ \vb jñānaṁ\lem  \msCa\msCc\msNb\Ed; jñāna \msCb\msNc, jñā\uncl{naṁ} \msNa\oo
                 maunaṁ\lem  \mssCaCbCc\msNb\msNc\Ed; mauna \msNa\oo
                 kṣamā\lem  \msCa\msCc\msNa\msNc\Ed; kṣamāḥ \msCb\msNb
        \ \vc śīlaṁ ca\lem  \mssCaCbCc\msNa\msNc; nīlañ ca \msNb, śilaṁ ca \Ed\oo
                 nābhimānaṁ\lem  \mssCaCbCc\msNa\msNb\msNc; nābhimānāṁ \Ed}

kāmat\textsubring{r}ṣṇāratidyūtamāno yuddhaṁ madaḥ sp\textsubring{r}hā\thinspace{\dandab} \dontdisplaylinenum

nirgh\textsubring{r}ṇāḥ kalikartāro rājaseṣūttamā janāḥ \veg\dontdisplaylinenum
\varr{
        \ \va °māno\lem  \msCa\msCb\msNa\msNb\msNc\Ed; °mano \msCc
        \ \vb yuddhaṁ\lem  \mssCaCbCc\msNa\msNb\msNc; yuddha° \Ed\oo
                 sp\textsubring{r}hā\lem  \mssCaCbCc\msNa\msNc\Ed; sm\textsubring{r}ta \msNb
        \ \vc nirgh\textsubring{r}ṇāḥ\lem  \mssCaCbCc; nirgh\textsubring{r}ṇā \msNa\Ed, nigh\textsubring{r}ṇāḥ \msNb\msNc
        \ \vd rājaseṣūttamā\lem  \msCa\msCb\msNa\msNb\msNc; rājasesūtamā \msCc, rājase hy uttamo \Ed}

hiṁsāsūyāgh\textsubring{r}ṇāmūḍhanidrātandrībhayālasāḥ\thinspace{\dandab} \dontdisplaylinenum

krodho matsaramāyī ca tāmaseṣūttamā janāḥ \veg\dontdisplaylinenum
\varr{
        \ \va °sūyā°\lem  \mssCaCbCc\msNa\msNc\Ed; °sa\uncl{yū}° \msNb\oo
                 °mūḍha°\lem  \msCa\msCc\msNa\msNc\Ed; °mūḍhā° \msCb\msNb
        \ \vb °tandrī°\lem  \mssCaCbCc\msNa\msNc\msNb; °tantrī° \Ed
        \ \vc krodho\lem  \mssCaCbCc\msNa\msNb\msNc; krodha° \Ed
        \ \vd tāmaseṣūttamā\lem  \msCa\msCb\msNa\msNb\msNc; tāmasesūtamā \msCc, tāmase hy uttamo \Ed}

laghuprītiprakāśī ca dhyānayoge sadotsukaḥ\thinspace{\dandab} \dontdisplaylinenum

prajñābuddhivirāgī ca sāttvikaṁ guṇalakṣaṇam \veg\dontdisplaylinenum
\varr{
        \ \vb °yoge\lem  \msCb\msCc\msNa\msNb\msNc\Ed; °\uncl{yoge} \msCa
        \ \vc °virāgī ca\lem  \mssCaCbCc\msNb\msNc\Ed; °virāgī \msNa, °virāṅkrī ca \msNc}

bālako nipuṇo rāgī māno darpaś ca lobhakaḥ\thinspace{\dandab} \dontdisplaylinenum

sp\textsubring{r}hā īrṣā pralāpī ca rājasaṁ guṇalakṣaṇam \veg\dontdisplaylinenum
\varr{
        \ \va bālako\lem  \mssCaCbCc\msNa\msNb\Ed; cālako \msNc\oo
                 nipuṇo\lem  \Ed; nipuno \mssCaCbCc\msNa\msNb, nipuṇe \msNc
        \ \vc īrṣā\lem  \msCa\msCc\msNa\msNb\msNc; īrṣyā \msCb\Ed\oo
                 pralāpī\lem  \msCa\msCb\msNa\msNb\msNc\Ed; ca lāpī \msCc
        \ \vd rājasaṁ\lem  \mssCaCbCc\msNa\msNb\msNc; tāmasaṁ \Ed}

udvega ālaso mohaḥ krūras taskaranirdayaḥ\thinspace{\dandab} \dontdisplaylinenum

krodhaḥ piśuna nidrā ca tāmasaṁ guṇalakṣaṇam \veg\dontdisplaylinenum
\varr{
        \ \va ālaso\lem  \msCa\msCc\msNa\msNb\msNc\Ed; alaso \msCb
        \ \vb krūras ta°\lem  \msCa\msCb\msNa; krūrata° \msCc\msNc\Ed, kūras ta° \msNb\oo
                 °nirdayaḥ\lem  \mssCaCbCc\msNa\msNb\Ed; °nirdayāḥ \msNc
        \ \vc krodhaḥ\lem  \msCa\msCc\msNa\msNb\msNc\Ed; krodha° \msCb\oo
                 piśuna\lem  \Ed; piśuno \mssCaCbCc\msNa\msNb\msNc\oo
                 ca\lem  \mssCaCbCc\msNa\msNc\Ed; \om\ \msNb
        \ \vd guṇa°\lem  \msCa\msCbpcorr\msCc\msNa\msNb\msNc\Ed; gu° \msCbacorr}

vigatarāga uvāca~{\dandab}\dontdisplaylinenum 

kena cihnena vijñeya āhāraḥ sarvadehinām\thinspace{\danda} \dontdisplaylinenum

traiguṇyasya p\textsubring{r}thaktvena kathayasva tapodhana \veg\dontdisplaylinenum
\varr{
        \ \vab kena cihnena vijñeya āhāraḥ sarvadehinām\lem  \msCb\msCc\msNa\msNc\Ed;
                      {\il}{\il}{\il}{\il}{\il}{\il}{\il}{\il}{\il}{\il}{\il}{\il}{\il} dehinām \msCa, kena cihnena vijñeya āhāra sarvadehinām \msNb
        \ \vc p\textsubring{r}thaktvena\lem  \mssCaCbCc\msNa\msNb\Ed; p\textsubring{r}thakkeṇa \msNc
        \ \vd °dhana\lem  \mssCaCbCc\msNa\msNb\Ed; °dhanaḥ \msNc}

anarthayajña uvāca~{\dandab}\dontdisplaylinenum 

āyuḥ kīrtiḥ sukhaṁ prītir balārogyavivardhanam\thinspace{\danda} \dontdisplaylinenum

h\textsubring{r}dyasvādurasaṁ snigdha āhāraḥ sāttvikapriyaḥ \veg\dontdisplaylinenum
\varr{
        \ \va kīrtiḥ\lem  \mssCaCbCc\msNa\msNb\msNc; kirtiḥ \Ed\oo
                 sukhaṁ prītir ba°\lem  \msNc; sukhaṁ prītiba° \msCa\msCb\msNa\msNb, 
                                sukhaprīti ba° \msCc, sukhaṁ pritiva° \Ed
        \ \vb °rogya°\lem  \msCa\msCc\msNa\msNb\msNc\Ed; °rogyaṁ \msCb
        \ \vc h\textsubring{r}dya°\lem  \mssCaCbCc\msNa\msNb\msNc; h\textsubring{r}da° \Ed\oo
                 °rasaṁ\lem  \msCa\msCb\msNa; °rasa \msCc, °\uncl{rasa} \msNb, °rasāṁ \msNc, °rasā \Ed\oo
                 snigdha\lem  \mssCaCbCc\msNc\Ed; snigdhaṁ \msNa, \uncl{sandigdha} \msNb
        \ \vd āhāraḥ\lem  \msCapcorr\msNb\msNc\Ed; āhāra \msCaacorr\msCb\msCc\msNa\oo
                 sāttvikapriyaḥ\lem  \msCa\msCb\msNa\msNc; sātvikapriyā \msCc, sātvikapriya \msNb, sātvikaḥ kiyāḥ \Ed}

atyuṣṇam āmlalavaṇaṁ rūkṣaṁ tīkṣṇaṁ vidāhi ca\thinspace{\dandab} \dontdisplaylinenum

rājasaśreṣṭha āhāro duḥkhaśokāmayapradaḥ \veg\dontdisplaylinenum
\varr{
        \ \va °mla°\lem  \mssCaCbCc\msNa\msNb\msNc; °lla° \Ed\oo
                 °lavaṇaṁ\lem  \msCa\msCc\msNa\msNb\msNc\Ed; °lakṣaṇaṁ \msCb
        \ \vb tīkṣṇaṁ\lem  \msCb\msCc\msNa\msNb\msNc; tī\uncl{kṣṇa} \msCa, stīkṣaṁ \Ed\oo
                 vidāhi ca\lem  \msCb\msNa\msNb\msNc; {\il}\uncl{dāhi ca} \msCa, 
                                        vidāhika \msCcpcorr, vidāhikaḥ \msCcacorr\Ed
        \ \vcd rājasaśreṣṭha āhāro duḥkhaśokāmayapradaḥ\lem  \msCb\msNa\msNc;
               {\il}{\il}{\il}{\il}{\il}{\il}{\il}{\il}{\il}{\il}{\il}{\il}{\il}{\il}{\il}{\il} \msCa, rājasaśreṣṭha āhāro duḥkhaśokāmayaḥ pradaḥ \msCc, 
                                       rājasaḥ śreṣṭha āhāro duḥkhaśokāmayapradaḥ \msNb, 
                                       rājase śreṣṭham āhāro duḥkhaśokābhayapradaḥ \Ed}

abhakṣyāmedhyapūtī ca pūti paryuṣitaṁ ca yat\thinspace{\dandab} \dontdisplaylinenum

āyāmarasavisvāda āhāras tāmasapriyaḥ \veg\dontdisplaylinenum
\varr{
        \ \va abhakṣyāmedhyapūtī ca\lem  \eme; abhakṣyamedhyapūtī ca \mssCaCbCc\msNa,
                                abhakṣamedhyapūtī ca \msNb, abhakṣāmedhyapūtī ca \msNc, abhakṣamadyapūtī vai \Ed
        \ \vc āyāma°\lem  \mssCaCbCc\msNa\msNb\msNc; āyāsa° \Ed
        \ \vd °masa°\lem  \msCa\msCb\msNa\msNb\msNc; °masaḥ \msCc\Ed\oo
                 °priyaḥ\lem  \msCa\msCb\msNa\msNb\msNc\Ed; °priyāḥ \msCc}

vigatarāga uvāca~{\dandab}\dontdisplaylinenum 

guṇātītaṁ kathaṁ jñeyaṁ saṁsāraparapāragam\thinspace{\danda} \dontdisplaylinenum

guṇapāśanibaddhānāṁ mokṣaṁ kathaya tattvataḥ \veg\dontdisplaylinenum
\varr{
        \ \va °tītaṁ\lem  \msCa\msCb\msNa\msNc\Ed; °tīta \msCc\msNb
        \ \vb °gam\lem  \msCa\msCb\msNa\msNb\msNc\Ed; °gaḥ \msCc
        \ \vc °baddhānāṁ\lem  \msCa\msCc\msNa\msNb\msNc; °varddhānāṁ \msCb, °baddhāmo \Ed}

anarthayajña uvāca~{\dandab}\dontdisplaylinenum 

ātmavat sarvabhūtāni samyak paśyeta bho dvija\thinspace{\danda} \dontdisplaylinenum

guṇātītaḥ sa vijñeyaḥ saṁsāraparapāragaḥ \veg\dontdisplaylinenum
\varr{
        \ \va °bhūtāni\lem  \mssCaCbCc\msNb\msNc\Ed; °bhūtāṁ \msNa
        \ \vb samyak pa°\lem  \mssCaCbCc\msNb\msNc\Ed; samyat pa° \msNa
        \ \vc °tītaḥ\lem  \msCa\msCb\msNa\msNb; °tīta \msCc\msNc, °tītaṁ \Ed}

īrṣādveṣasamo yas tu sukhaduḥkhasamāś ca ye\thinspace{\dandab} \dontdisplaylinenum

stutinindāsamā ye ca guṇātītaḥ sa ucyate \veg\dontdisplaylinenum
\varr{
        \ \va īrṣā°\lem  \mssCaCbCc\msNa\msNb; īrṣyā° \msNc\Ed
        \ \vb °samāś ca ye\lem  \mssCaCbCc\msNa\msNc\Ed; °samāśraye \msNb
        \ \vd °tītaḥ\lem  \mssCaCbCc\msNa\msNc\Ed; °tīta \msNb}

tulyapriyāpriyo yaś ca arimitrasamas tathā\thinspace{\dandab} \dontdisplaylinenum

mānāpamānayos tulyo guṇātītaḥ sa ucyate \veg\dontdisplaylinenum 
            \paral{\textit{\vo {\normalfont cf.\ MBh 6.36.24cd--25 (BhG 14.24cd--25): }
                        tulyapriyāpriyo dhīras tulyanindātmasaṁstutiḥ {\thinspace\ketdanda}
                        mānāvamānayos tulyas tulyo mitrāripakṣayoḥ {\thinspace\danda}
                        sarvārambhaparityāgī guṇātītaḥ sa ucyate {\thinspace\ketdanda}
        }}
\varr{
        \ \va tulya°\lem  \Ed; tulyaḥ \mssCaCbCc\msNa\msNb\msNc
        \ \vb °sama°\lem  \msCa\msCb\msNa\msNb\msNc\Ed; °samā° \msCc}

eṣa te kathito vipra guṇasadbhāvanirṇayaḥ\thinspace{\dandab} \dontdisplaylinenum

guṇayuktas tu saṁsārī guṇātītaḥ parāṅgatiḥ \veg\dontdisplaylinenum


\jump
\begin{center}
\ketdanda iti v\textsubring{r}ṣasārasaṁgrahe traiguṇyaviśeṣaṇīyo nāmādhyāyo navamaḥ\ketdanda
\end{center}
\dontdisplaylinenum\vers 
\varr{
        \ \va te\lem  \mssCaCbCc\msNa\msNc\Ed; to \msNb
        \ \vb °sadbhāva°\lem  \mssCaCbCc\msNa\msNb\msNc; °madbhāva° \Ed
        \ \vd guṇātītaḥ\lem  \msCa\msCc\msNa; guṇātīta \msCb\msNb\msNc\Ed\oo
                 parāṅgatiḥ\lem  \Ed; parāṅgatim \mssCaCbCc\msNa\msNb\msNc
        \ {\normalfont \Colo: } °viśeṣaṇīyo\lem  \corr; °viśeṣanīyo \mssCaCbCc\msNa\msNb\msNc\Ed\oo
                        nāmādhyāyo navamaḥ\lem  \mssCaCbCc\msNa\msNb\msNc; nāma navamo 'dhyāyaḥ \Ed}
\bekveg\szamveg\vfill\phpspagebreak\szam\bek\versno=0\fejno=10
\thispagestyle{empty}



\alfejezet{\textbf{daśamo 'dhyāyaḥ}}\jump\jump

\alalfejezet{kāyatīrthopavarṇanam}
\vers

vigatarāga uvāca~{\dandab}\dontdisplaylinenum 

katamaṁ sarvatīrthānāṁ śreṣṭham āhur manīṣinaḥ\thinspace{\danda} \dontdisplaylinenum

kathayasva muniśreṣṭha yady asti bhuvi kāmadam \veg\dontdisplaylinenum
\varr{
        \ \va katamaṁ sarva°\lem  \mssCaCbCc\msNa\Ed; katamasarva° \msNb, kathaman sarva° \msNc
        \ \vab °tīrthānāṁ śreṣṭha°\lem  \msCb\msCc\msNa\msNb\msNc\Ed; °tīrthā{\il}{\il}ṣṭha° \msCa
        \ \vb manīṣinaḥ\lem  \mssCaCbCc\msNa\msNb\msNc; manīṣibhiḥ \Ed
        \ \vd bhuvi\lem  \mssCaCbCc\msNa\msNb\msNc; bhūri \Ed\oo
                 °dam\lem  \mssCaCbCc\msNb\msNc\Ed; °daḥ \msNa}

anarthayajña uvāca~{\dandab}\dontdisplaylinenum 

atiguhyam idaṁ praśnaṁ p\textsubring{r}ṣṭaḥ snehād dvijottama\thinspace{\danda} \dontdisplaylinenum

bravīmi vaḥ purāv\textsubring{r}ttaṁ nandinā kathito 'smy aham \veg\dontdisplaylinenum
\varr{
        \ \vb snehād dvi°\lem  \msCa\msCb\msNa\msNb\msNc\Ed; snehā dvi° \msCc
        \ \vd 'smy aham\lem  \msCa\msCb\msNa\msNb\msNc\Ed; sm\textsubring{r}ham \msCc}

nandikeśvara uvāca~{\dandab}\dontdisplaylinenum 
\varr{
        \ \vo nandi°\lem  \msCa\msCc\msNa\msNb\msNc\Ed; nandī° \msCb}

kailāsaśikhare ramye siddhacāraṇasevite\thinspace{\danda} \dontdisplaylinenum
            \paral{\textit{\vab {\normalfont  cf.\ MBh 12.327.18cd: } merau girivare ramye siddhacāraṇasevite }}

tatrāsīnaṁ śivaṁ sākṣād devī vacanam abravīt \veg\dontdisplaylinenum
\varr{
        \ \va kailāsa°\lem  \mssCaCbCc\msNa\msNb\msNc; kailāśe \Ed}

devy uvāca~{\dandab}\dontdisplaylinenum 

bhagavan devadeveśa sarvabhūtajagatpate\thinspace{\danda} \dontdisplaylinenum

praṣṭum icchāmy ahaṁ tv ekaṁ dharmaguhyaṁ sanātanam \veg\dontdisplaylinenum
\varr{
        \ \va °deveśa\lem  \msCa\msCc\msNa\msNb\msNc\Ed; °deśa \msCb
        \ \vb °pate\lem  \mssCaCbCc\msNapcorr\msNb\msNc\Ed; °patim \msNaacorr}

atitīrthaṁ paraṁ guhyaṁ saṁsārād yena mucyate\thinspace{\dandab} \dontdisplaylinenum

manuṣyāṇāṁ hitārthāya brūhi tattvaṁ maheśvara \veg\dontdisplaylinenum
\varr{
        \ \va °tīrthaṁ\lem  \mssCaCbCc\msNa\msNc; °tīrtha \msNb\Ed
        \ \vab guhyaṁ saṁsārād yena mucyate\lem  \mssCaCbCc\msNa\msNc\Ed; \uncl{ga}{\lost} \uncl{saṁ}sārād yena
                                                                                mucyate \msNb
        \ \vd °śvara\lem  \msCa\msCb\msNa\msNb\msNc\Ed; °śvaraḥ \msCc}

maheśvara uvāca~{\dandab}\dontdisplaylinenum 

ko māṁ p\textsubring{r}cchati taṁ praśnaṁ muktvā tvām eva sundari\thinspace{\danda} \dontdisplaylinenum

ś\textsubring{r}ṇu vakṣyāmi tat praśnaṁ devair api sudurlabham \veg\dontdisplaylinenum
\varr{
        \ \va taṁ praśnaṁ\lem  \msNa\msNb; tat praśna \msCa\msCb, tat praśnaṁ \msCc\Ed, 
                                                                taṁ praśna \msNc
        \ \vb muktvā\lem  \mssCaCbCc\msNa\msNb\msNc; muktā \Ed
        \ \vc tat praśnaṁ\lem  \mssCaCbCc\msNa\msNb\Ed; taṁ praśnan \msNc}

kurukṣetraṁ prayāgaṁ ca vārāṇasīm ataḥ param\thinspace{\dandab} \dontdisplaylinenum

gaṅgāgniṁ somatīrthaṁ ca sūryapuṣkaramānasam \veg\dontdisplaylinenum
\varr{
        \ \vc gaṅgāgniṁ\lem  \msCa\msCb; gaṅgāgni \msCc\msNa\msNb\msNc, gaṅgā'gni° \Ed}

naimiṣaṁ bindusāraṁ ca setubandhaṁ surahradam\thinspace{\dandab} \dontdisplaylinenum

ghaṇṭikeśvaravāgīśaṁ jñātvā niścayapāpahā \veg\dontdisplaylinenum
\varr{
        \ \va naimiṣaṁ\lem  \mssCaCbCc\msNa\msNb\Ed; nemisa \msNc
        \ \vb °bandhaṁ\lem  \mssCaCbCc\msNa\msNb\msNc; °bandha° \Ed
        \ \vc °vāgīśaṁ\lem  \mssCaCbCc\msNa\msNc\Ed; {\lost}\uncl{gīśa} \msNb
        \ \vd niścayapāpahā\lem  \msCb\msCc\msNa\msNb\msNc\Ed; niśca\uncl{ya}{\il}{\il}{\il} \msCa}

umovāca~{\dandab}\dontdisplaylinenum 

evamādi mahādeva pūrvavat kathitāsmy aham\thinspace{\danda} \dontdisplaylinenum

svargabhogapradaṁ tīrtham eteṣāṁ suranāyaka \veg\dontdisplaylinenum
\varr{
        \ \vb kathitā°\lem  \msCa\msCc\msNa\msNc; kathito \msCb\msNb\Ed
        \ \vcd tīrtham e°\lem  \msCa\msCb\msNa\msNb\msNc\Ed; tīrthaṁm e° \msCc
        \ \vd suranāyaka\lem  \msCapcorr\msNa\msNc; suranāka \msCaacorr, suranāyakam \msCb\msCc\msNb\Ed}

kathaṁ mucyeta saṁsārāj jñānamātreṇa īśvara\thinspace{\dandab} \dontdisplaylinenum

kautūhalaṁ mahaj jātaṁ chindhi saṁśayakārakam \veg\dontdisplaylinenum
\varr{
        \ \va kathaṁ\lem  \msCa\msCc\msNa\msNb\msNc\Ed; katha \msCb
        \ \vb jñāna°\lem  \msCa\msCc\msNa\msNb\msNc\Ed; jñāta° \msCb\oo
                 īśvara\lem  \mssCaCbCc\msNb\msNc\Ed; ceśvara \msNa
        \ \vc kautūhalaṁ mahaj jātaṁ\lem  \mssCaCbCc\Ed; kautūhalam ma\uncl{ho}j jātaṁ \msNa, 
                                kauhalam mahaj jātaṁ \msNbacorr, kau\uncl{tū}halam mahaj jātaṁ \msNbpcorr,
                                                                     kotūhalaṁ mahaj jātaṁ \msNc
        \ \vd °kārakam\lem  \Ed; °kāraka \mssCaCbCc\msNb\msNc, °kārakaḥ \msNa}

rudra uvāca~{\dandab}\dontdisplaylinenum 

kiṁ na jānāmi tat tīrthaṁ sulabhaṁ durlabhaṁ ca yat\thinspace{\danda} \dontdisplaylinenum

sulabhaṁ gurusevīnāṁ durlabhaṁ tad vivarjayet \veg\dontdisplaylinenum


\alalalfejezet{kurukṣetram}
\varr{
        \ \va jānāmi\lem  \mssCaCbCc\msNb; jānā\uncl{mi} \msNaacorr, jānā\uncl{si} \msNapcorr, 
                                                                                jānāsi \msNc\Ed
        \ \vb durlabhaṁ ca\lem  \msCa\msNa\msNb\Ed; dulabhañ ca \msCb\msNc, dullabhañ ca \msCc
        \ \vc sulabhaṁ gurusevīnāṁ\lem  \msCb\msCc\msNa\msNb\msNc\Ed; {\il}{\il}{\il}{\il}{\il}{\il}vīnāṁ \msCa
        \ \vd °varjayet\lem  \mssCaCbCc\msNb\msNc; °varjaye \msNa, °varjanāt \Ed}

kuruḥ puruṣa vijñeyaḥ śarīraṁ kṣetra ucyate\thinspace{\dandab} \dontdisplaylinenum

śarīrasthaṁ kurukṣetraṁ sarvatīrthaphalapradam \veg\dontdisplaylinenum
\varr{
        \ \va kuruḥ\lem  \mssCaCbCc\msNa\msNc\Ed; guruḥ \msNb\oo
                 puruṣa\lem  \Ed; puruṣaḥ \mssCaCbCc\msNa\msNb\ \unmetr, puruṣo \msNc\ \unmetr
        \ \vb śarīraṁ\lem  \msCb\msCc\msNa\msNb\msNc\Ed; śarī\uncl{ra} \msCa\oo
                 kṣetra ucyate\lem  \mssCaCbCc\msNb\msNc\Ed; kṣetram ucyate \msNa
        \ \vc °sthaṁ\lem  \mssCaCbCc\msNa\msNb\Ed; °stha \msNc\oo
                 °kṣetraṁ\lem  \mssCaCbCc\msNa\msNb\Ed; °kṣetra \msNc}

sarvayajñaphalāvāptiḥ sarvadānaphalāni ca\thinspace{\dandab} \dontdisplaylinenum

sarvavratatapaś cīrṇaṁ tatphalaṁ sakalaṁ bhavet \veg\dontdisplaylinenum
\varr{
        \ \vd tatphalaṁ\lem  \mssCaCbCc\msNa\msNb\Ed; tatphala \msNc}

evam eva phalaṁ teṣāṁ tīrthapañcadaśeṣu ca\thinspace{\dandab} \dontdisplaylinenum

anaghānaṁ mahāpuṇyaṁ mahātīrthaṁ mahāsukham \veg\dontdisplaylinenum
\varr{
        \ \vb tīrthapañcadaśeṣu\lem  \msCa\msCc\msNa\msNb\msNc\Ed; tīrthampaṁcadaśaiṣu \msCb
        \ \vc anaghānaṁ mahāpuṇyaṁ\lem  \msCb\msNc; {\il}{\il}{\il}{\il}{\il}{\il}puṇya \msCa, anapyām mahāpuṇyaṁ \msCc,
                                anadhyānaṁ mahāpuṇyaṁ \msNa, adhvānan tu mahāpuṇyaṁ \msNb, snānadhyānaṁ mahāpuṇyaṁ \Ed}

devy uvāca~{\dandab}\dontdisplaylinenum 

atīva romaharṣo me jāto 'sti tridaśeśvara\thinspace{\danda} \dontdisplaylinenum

sulabhaṁ sukaraṁ sūkṣmaṁ śrutvā tuṣṭiś ca me gatā \veg\dontdisplaylinenum
\varr{
        \ \va atīva\lem  \msCa\msCc\msNa\msNb\msNc\Ed; avīva \msCb
        \ \vb 'sti\lem  \mssCaCbCc\msNa\msNc\Ed; smi \msNb\oo
                 tridaśeśvara\lem  \msCa\msCb\msNa\msNc\Ed; tridaśeśvaraḥ \msCc, tri{\lost}śeśvara \msNb
        \ \vd tuṣṭiś ca\lem  \msCa\msCb\msNa\msNb\msNc\Ed; tuṣṭiñ ca \msCc\oo
                 gatā\lem  \msCa\msCc\msNa\msNb\msNc\Ed; gatāḥ \msCb}

caturdaśa paro bhūyaḥ kathayasva manoharam\thinspace{\dandab} \dontdisplaylinenum

prayāgādi p\textsubring{r}thaktvena tattvatas tu sureśvara \veg\dontdisplaylinenum


\alalalfejezet{prayāgo vārāṇasī ca}
\varr{
        \ \vd tattvatas tu\lem  \mssCaCbCc\msNapcorr\msNb\msNc\Ed; tatvata \msNaacorr}

rudra uvāca~{\dandab}\dontdisplaylinenum 

suṣumnā bhagavatī gaṅgā iḍā ca yamunā nadī\thinspace{\danda} \dontdisplaylinenum

etā srotavahā nadyaḥ prayāgaḥ sa vidhīyate \veg\dontdisplaylinenum
\varr{
        \ \va suṣumnā\lem  \mssCaCbCc\msNa\msNb\msNc; suṣumṇā \Ed\oo
                 bhagavatī gaṅgā\lem  \msCb\msCc\msNa\msNb\msNc; bhagavatī ga{\il} \msCa, bhavatī gaṅgā \Ed
        \ \vc etā srotavahā\lem  \corr; etā śrotavahā \msCa\msNc\Ed, 
                                        ete śrotāvahā \msCb\msCc, etā śrotravahā \msNa\msNb}

dakṣiṇā vāruṇī nāsā vāmanāsā asi sm\textsubring{r}tā\thinspace{\dandab} \dontdisplaylinenum

vāruṇā-asimadhyena tena vārāṇasī sm\textsubring{r}tā \veg\dontdisplaylinenum


\alalalfejezet{gaṅgā}
\varr{
        \ \va dakṣiṇā\lem  \msCb\msNa\msNb\msNc\Ed; dakṣi\uncl{ṇaṁ} \msCa, dakṣiṇaṁ \msCc\oo
                 vāruṇī\lem  \msNapcorr\msNc\Ed; varuṇī \msCa\msCc\msNaacorr\msNb, varuṇā \msCb
        \ \vb °nāsā\lem  \msCa\msCc\msNa\msNc\Ed; °nā \msCb\msNb
        \ \vc vāruṇā-asimadhyena\lem  \Ed; varuṇā asimadhyena \msCa\msCb\msNa\msNc, vāruṇan nāsamadhyeta \msCc,
                                                varuṇa asimadhyena \msNb}

ākāśagaṅgā vikhyātā tasyāḥ sravati cām\textsubring{r}tam\thinspace{\dandab} \dontdisplaylinenum

ahorātram avicchinnaṁ gaṅgā sā tena ucyate \veg\dontdisplaylinenum


\alalalfejezet{somatīrtham}
\varr{
        \ \vb tasyāḥ\lem  \msCa\msCb\msNa\msNc\Ed; tasmā \msCc, tasyā \msNb
        \ \vd tena\lem  \msCa\msCb\msNa\msNb\msNc\Ed; te \msCc}

somatīrtham iḍā nāḍī kiṅkiṇīravacihnitā\thinspace{\dandab} \dontdisplaylinenum

taṁ tu śrutvā na saṁdehaḥ sarvapāpakṣayo bhavet \veg\dontdisplaylinenum


\alalalfejezet{sūryatīrtham}
\varr{
        \ \va °tīrtham iḍā\lem  \msCa\msCc\msNa\msNb\msNc\Ed; °tīrtha iḍā \msCb
        \ \vb kiṅkiṇī°\lem  \msCa\msCb\msNa\msNb\msNc\Ed; ciñcinī° \msCc\oo
                 °rava°\lem  \msCa\msCbpcorr\msCc\msNa\msNb\msNc; °ravi° \msCbacorr, °rāva° \Ed\oo
                 °cihnitā\lem  \msCa\msCb\msNa\msNc\Ed; °cihnikā \msCc, °cihnatā \msNb
        \ \vc taṁ tu\lem  \corr; \uncl{tan tu} \msCa, tan tu \msCb\msCc\msNa\msNc\Ed, 
                                                                                ta\uncl{t tu} \msNb\oo
                 na saṁdehaḥ\lem  \msCa\msCb\msNa\msNb\msNc\Ed; varāroheḥ \msCc}

sūryatīrthaṁ suṣumnā ca nīravāravasaṁyutā\thinspace{\dandab} \dontdisplaylinenum

śrutimātrād vimucyeta pāparāśir mahān api \veg\dontdisplaylinenum


\alalalfejezet{agnitīrtham}
\varr{
        \ \va °tīrthaṁ\lem  \mssCaCbCc\msNa\msNc\Ed; °tīrtha \msNb\oo
                 suṣumnā\lem  \mssCaCbCc\msNa\msNb\msNc; suṣumṇā \Ed
        \ \vb nīravā°\lem  \Ed; vīravā° \msCa\msCc, cīravā° \msCb\msNa\msNb\msNc\oo
                 °yutā\lem  \msCa\msNa\msNc\Ed; °yutam \msCb\msCc, °yutāṁ \msNb
        \ \vc °mātrā°\lem  \msCa\msCb\msNa\msNb\msNc\Ed; °mātā° \msCc}

agnitīrthārjunā nāḍī brahmaghoṣamanoramā\thinspace{\dandab} \dontdisplaylinenum

tat tad akṣaram ākarṇya am\textsubring{r}tatvāya kalpate \veg\dontdisplaylinenum


\alalalfejezet{puṣkaram}
\varr{
        \ \va °rjunā\lem  \msCa\msCb\msNa\msNb\msNc; °junā \msCc, °rjunaṁ \Ed
        \ \vb °ramā\lem  \mssCaCbCc\msNa\msNb; °ramāḥ \msNc\Ed
        \ \vc °karṇya\lem  \msCa\msCc\msNa\msNb\msNc\Ed; °rṇya \msCb
        \ \vd kalpate\lem  \msCb\msNc\Ed; ka{\il}{\lost} \msCa, kalpyate \msCc\msNa\msNb}

puṣkaraṁ h\textsubring{r}di madhyastham aṣṭapattraṁ sakarṇikam\thinspace{\dandab} \dontdisplaylinenum

cintayet sūkṣma tanmadhye janmam\textsubring{r}tyuvināśanam \veg\dontdisplaylinenum


\alalalfejezet{mānasam}
\varr{
        \ \vb °pattraṁ\lem  \msCb\msNa\msNc\Ed; {\il}{\il} \msCa, °patra \msCc\msNb\oo
                 °karṇikam\lem  \msCb\msNa\msCc\msNb\msNc; {\il}{\il}{\il} \msCa, °karṇikām \Ed
        \ \vc sūkṣma\lem  \msCb\msCc\msNa\msNb\msNc; \uncl{sūkṣma} \msCa, sūkṣmaṁ \Ed}

mānasasaramadhyasthaṁ sahaṁsakamalopari\thinspace{\dandab} \dontdisplaylinenum

salīlo līlayācārī parataḥ parapāragaḥ \veg\dontdisplaylinenum


\alalalfejezet{naimiṣam}
\varr{
        \ \va mānasa°\lem  \msCb\msNa; \uncl{mānasa} \msCa, mānasaṁ \msCc\msNb\msNc\Ed
        \ \vb sahaṁsa°\lem  \msCa\msCc\msNa\msNb\msNc\Ed; sahasaṁ \msCb
        \ \vc salīlo\lem  \mssCaCbCc\msNa\msNb\msNc; salīlā \Ed
        \ \vd parataḥ\lem  \mssCaCbCc\msNa\msNc\Ed; parata \msNb}

naimiṣaṁ ś\textsubring{r}ṇu deveśi nimiṣā pratyayo bhavet\thinspace{\dandab} \dontdisplaylinenum

samyag chāyāṁ nirīkṣeta ātmāno vā parasya vā \veg\dontdisplaylinenum
\varr{
        \ \vb nimiṣā pratyayo bhavet\lem  \msCa\msCc\msNa\msNc\Ed; nimi pratyayo bhavet \msCb,
                                                        ni{\lost}\uncl{ṣo} pratyayo \uncl{bhavet} \msNb
        \ \vd ātmano\lem  \msCb\msCc\msNa\msNb\msNc; {\il}nmano \msCa, svātmāno \Ed\oo
                 parasya vā\lem  \mssCaCbCc\msNa\msNb\msNc; parasya ca \Ed}

āyatapy aṅgulīmātraṁ nimiṣākṣi sa paśyati\thinspace{\dandab} \dontdisplaylinenum

d\textsubring{r}ṣṭvā pratyayam evaṁ hi naimiṣajñaḥ sa ucyate \veg\dontdisplaylinenum


\alalalfejezet{bindusaraḥ}
\varr{
        \ \va āyatapy aṅgulī°\lem  \mssCaCbCc\msNa\msNb; āyātapy aṅgulī° \msNc\Ed\oo
                 °mātraṁ\lem  \mssCaCbCc\msNa\msNb; °mātra \msNc, °madhye \Ed
        \ \vd naimiṣajñaḥ\lem  \msCa\msNa\msNb\msNc\Ed; naimisaṁjñaḥ \msCb, naimiṣajña \msCc}

tīrthaṁ bindusaraṁ nāma ś\textsubring{r}ṇu vakṣyāmi sundari\thinspace{\dandab} \dontdisplaylinenum

dehamadhye h\textsubring{r}di jñeyaṁ h\textsubring{r}dimadhye tu paṅkajam \veg\dontdisplaylinenum
\varr{
        \ \va tīrthaṁ bindu°\lem  \mssCaCbCc\msNa\msNb\msNc; tīrtham indu° \Ed
        \ \vc h\textsubring{r}di jñeyaṁ\lem  \msCa\msCc\msNa\msNb\msNc\Ed; \om\ \msCb}

karṇikā padmamadhye tu binduḥ karṇikamadhyataḥ\thinspace{\dandab} \dontdisplaylinenum

bindumadhye sthito nādaḥ sa nādaḥ kena bhidyate \veg\dontdisplaylinenum
\varr{
        \ \va °madhye\lem  \msCb\msCc\msNb\msNc\Ed; °dhye \msCa, °padhye \msNa
        \ \vc bindumadhye\lem  \msCb\msCc\msNa\msNb\msNc\Ed; \uncl{bindu}{\il}{\il} \msCa
        \ \vd bhidyate\lem  \msCb\msNa\msNb\msNc\Ed; \uncl{vi}dyate \msCa, vidyate \msCc}

ukāraṁ ca makāraṁ ca bhittvā nādo vinirgataḥ\thinspace{\dandab} \dontdisplaylinenum

taṁ viditvā viśālākṣi so 'm\textsubring{r}tatvaṁ labheta ca \veg\dontdisplaylinenum


\alalalfejezet{setubandham}
\varr{
        \ \va ukāraṁ ca makāraṁ\lem  \mssCaCbCc\msNa\msNb\msNc; ukāraś ca makāraś \Ed
        \ \vd so 'm\textsubring{r}tatvaṁ\lem  \msCa\msCb\msNa\msNb\msNc; somyatatvaṁ \msCc, somatatvaṁ \Ed\oo
                 ca\lem  \mssCaCbCc\msNa\msNb\msNc; vā \Ed}

\ujvers\nemsloka 
vakṣye te setubandhaṁ duritamalaharaṁ nādatoyapravāham
\dontdisplaylinenum

\nemslokab 
jihvākaṇṭhorakūlā svaragaṇapulināvartaghoṣā taraṅgā \danda\dontdisplaylinenum

\nemslokac 
kumbhīrāghoṣamīnā daśagaṇamakarā bhīmanakrā visargā
\dontdisplaylinenum

\nemslokad 
sānusvāre gabhīre madasukharasanaṁ setubandhaṁ vrajasva \veg\dontdisplaylinenum


\alalalfejezet{surahradaḥ}
\varr{
        \ \va te\lem  \msCapcorr\msCb\msNa\msNb\msNc\Ed; \om\ \msCaacorr, haṁ \msCc\oo
                 °bandhaṁ\lem  \msCa\msCc\msNa\msNb\msNc\Ed; °bandhūṁ \msCb\oo
                 °toya°\lem  \mssCaCbCc\msNa\msNc\Ed; °toyaṁ \msNb
        \ \vb °kaṇṭhora°\lem  \conj; °kaṇṭhoru° \mssCaCbCc\msNa\msNb\msNc\Ed\oo
                 svara°\lem  \msCa\msCb\msNa\msNb\msNc; sura° \msCc\Ed
        \ \vc °mīnā\lem  \mssCaCbCc\msNa\msNb\msNc; °mānā \Ed\oo
                 daśa°\lem  \msCb\msCc\msNa\msNb\msNc\Ed; {\il}{\il} \msCa\oo
                 visargā\lem  \mssCaCbCc; visargāḥ \msNa\msNb\msNc\Ed
        \ \vd °svāre\lem  \msCa\msCb\msNc\Ed; °sāre \msCc, °svāro \msNa, °svā\uncl{reṇa} \msNb\ \unmetr\oo
                 gabhīre\lem  \msCa\msCb\msNc; gambhīre \msCc\msNb\Ed, \uncl{gaṁ}bhīre \msNa\oo
                 °rasanaṁ\lem  \mssCaCbCc\msNa\msNb\msNc; °ramaṇaṁ \Ed\oo
                 °bandhaṁ\lem  \msCa\msCc\msNa\msNb\msNc\Ed; °bandha \msCb\oo
                 vrajasva\lem  \mssCaCbCc\msNa\msNb\msNc; ramasva \Ed}

\ujvers\nemsloka 
saptadvīpāntamadhye ś\textsubring{r}ṇu śaśivadane sarvaduḥkhāntalābham
\dontdisplaylinenum

\nemslokab 
īśānenābhijuṣṭaṁ h\textsubring{r}di hrada vimalaṁ nādaśītāmbupūrṇam \danda\dontdisplaylinenum

\nemslokac 
tatraikaṁ jātapadmaṁ prak\textsubring{r}tidalayutaṁ keśaraśaktibhinnaṁ
\dontdisplaylinenum

\nemslokad 
pañcavyomapraśastaṁ gatiparamapadaṁ prāptukāmena sevyam \veg\dontdisplaylinenum


\alalalfejezet{ghaṇṭikeśvaram}
\varr{
        \ \va °dvīpā°\lem  \mssCaCbCc\msNa\msNb\Ed; °dīpā° \msNc
        \ \vb īśānenābhijuṣṭaṁ\lem  \msCc\msNa\msNc\Ed; īśānenābhiduṣṭaṁ \msCa\msNb,
                                                 īśānebhiduṣṭaṁ \msCbacorr, īśānebhi{\lost}duṣṭaṁ \msCbpcorr\oo
                 vimalaṁ nādaśītā°\lem  \mssCaCbCc\msNa\msNc; vimalān nādaśītā° \msNb, vimalaṁ nāmaśitā° \Ed
        \ \vc keśara°\lem  \msCa\msCc\msNa\msNc; keśaraṁ \msCb\Ed, keśvara° \msNb
        \ \vd °vyoma°\lem  \mssCaCbCc\msNb\msNc\Ed; °vyo\uncl{maṁ} \msNa\oo
                 °śastaṁ ga°\lem  \msCa\msCb\msNa\msNb\msNc\Ed; °śasvaṅ ga° \msCc\oo
                 °parama°\lem  \mssCaCbCc\msNb\msNc\Ed; °paramaṁ \msNa\ \unmetr\oo
                 sevyam\lem  \mssCaCbCc\msNa\msNb\msNc; sarvam \Ed}

\ujvers\nemsloka 
nāḍyaikāsaṅgatāni nipatitam am\textsubring{r}taṁ ghaṇṭikāpārakeṇa
\dontdisplaylinenum

\nemslokab 
t\textsubring{r}pyante tena nityaṁ h\textsubring{r}di kamalapuṭaṁ sthānubhūtāntarātmā \danda\dontdisplaylinenum

\nemslokac 
yaṁ paśyantīśabhaktā kalikaluṣaharaṁ vyāpinaṁ niṣprapañcam
\dontdisplaylinenum

\nemslokad 
deveśaṁ ghaṇṭikeśamarabhavam abhavantīrtham ākāśabindum \veg\dontdisplaylinenum


\alalalfejezet{vāgīśvaratīrtham}
\varr{
        \ \va nipatitam am\textsubring{r}taṁ\lem  \mssCaCbCc\msNc\Ed; nipatitam am\textsubring{r}ta° \msNa\ \unmetr, 
                                        ni{\lost}{\lost}tam am\textsubring{r}taṁ \msNb\oo
                 °pārakeṇa\lem  \msCa\msCb\msNa\msNc; °yāṅkareṇa \msCc\Ed, °\uncl{pārakena} \msNb
        \ \vb °puṭaṁ\lem  \msCa\msCc\msNa\msNb\msNc\Ed; °puṭa \msCb\oo
                 sthānu°\lem  \mssCaCbCc\msNa\msNc; \uncl{sthāna°} \msNb, sthāna° \Ed
        \ \vc yaṁ paśyantīśabhaktā\lem  \msCa\msNb; yaṁ paśyantīśabharttāḥ \msCb, yaṁ pasyantīsabhaktyā \msCc,
                                                yaṁ paśyantīśabhaktāḥ \msNa,
                                                yat paśyantīśabhaktyā \msNc, yaṁ paśyannīśamakṣā \Ed\oo
                 °prapañcam\lem  \msCa\msNa\msNb\msNc; °prapañca \msCb\msCc\Ed
        \ \vd deveśaṁ\lem  \msCb\msNb\Ed; devyeśaṁ \msCa\msCc\msNa, devyeśa \msNc\oo
                 ghaṇṭikeśamara°\lem  \msCa\msCb\msNb\msNc; ghaṇṭikeśāmara° \msCc,
                                              ghaṇṭikeśaṁ mara° \msNa, ghāṇṭakeśāmara° \Ed\oo
                 °bhavantīrtham\lem  \msCb\msCc\msNa\msNb\msNc\Ed; bhava{\il}{\il}rtham \msCa\oo
                 °bindum\lem  \msCa\msCb\msNa\msNb\msNc\Ed; °bindu \msCc}

\ujvers\nemsloka 
mīmāṁsāratnakūlā kramapadapulinā śaivaśāstrārthatoyā
\dontdisplaylinenum

\nemslokab 
mīnaughā pañcarātraṁ śrutikuṭilagatismārtavegā taraṅgā \danda\dontdisplaylinenum

\nemslokac 
yogāvartātiśobhā upaniṣadivahā bhāratāvartaphenā
\dontdisplaylinenum

\nemslokad 
pañcāśadvyomarūpī rasabhavananadī tīrthavāgīśvarīyam \veg\dontdisplaylinenum
\varr{
        \ \va śaiva°\lem  \mssCaCbCc\msNa\msNb\msNc; śarva° \Ed
        \ \vb mīnaughā°\lem  \msNa\msNb\Ed;  mīnoghā° \mssCaCbCc\msNc\oo
                 pañcarātraṁ\lem  \mssCaCbCc\msNa\msNb\msNc; pañcaśatraṁ \Ed\oo
                 °smārtavegā taraṅgā\lem  \mssCaCbCc\msNa\msNc; °smā{\lost}\uncl{vegā taraṅgā} \msNb, 
                                                                °smārtavegās taraṅgā \Ed
        \ \vc °vahā bhāratā°\lem  \mssCaCbCc\msNa\msNc\Ed; mahābhāratā° \msNb
        \ \vd °śadvyoma°\lem  \mssCaCbCc\msNb\msNc; °śavyoma° \msNa, °sadvyoma° \Ed}

\ujvers\nemsloka 
yas taṁ vetti sa vetti vedanikhilaṁ saṁsāraduḥkhacchidam
\dontdisplaylinenum

\nemslokab 
janmavyādhiviyogatāpamaraṇaṁ kleśārṇavaṁ duḥsaham \danda\dontdisplaylinenum

\nemslokac 
garbhāvāsam atīva sahyaviṣayaṁ dustīryaduḥkhālayam
\dontdisplaylinenum

\nemslokad 
prāptaṁ tena na saṁśayaḥ śivapadaṁ duṣprāpya devair api \veg\dontdisplaylinenum

\vers


\jump
\begin{center}
\ketdanda iti v\textsubring{r}sasārasaṁgrahe kāyatīrthopavarṇano nāmādhyāyo daśamaḥ\ketdanda
\end{center}
\dontdisplaylinenum\vers 
\varr{
        \ \va yas taṁ\lem  \msCc\msNa\msNb\msNc\Ed; yasta° \msCa\msCb\oo
                 sa vetti\lem  \mssCaCbCc\msNa\msNb\Ed; \uncl{na} vetti \msNc
        \ \vb °maraṇaṁ\lem  \mssCaCbCc\msNa\msNb\Ed; °maraṇa \msNc\oo
                 °rṇavaṁ\lem  \mssCaCbCc\msNb\msNc; °ṇṇavaṁ \msNa, °rṇava \Ed
        \ \vc garbhāvāsam\lem  \mssCaCbCc\msNa\msNb\msNc; garbhovāsam \Ed\oo
                 °viṣayaṁ\lem  \msCa\msCb\msNb; °viṣamaṁ \msCc\msNa\msNc\Ed\oo
                 °layam\lem  \mssCaCbCc\msNb\Ed\msNc; °laya\uncl{ḥ} \msNa\oo
                 dustīrya°\lem  \mssCaCbCc\msNa\msNb\Ed; dustīryaḥ \msNc
        \ \vd prāptaṁ tena na saṁśayaḥ śivapadaṁ duṣprāpya devair api\lem  \msCa\msCbpcorr\msNa\msNc; 
                      prāptaṁ tena na saṁśayaṁ śivapadaṁ duṣprāpya devair api \msCc\Ed,
                      prāptaṁ tena na saṁśayaḥ   śivadaṁ duṣprāpya devair api \msCbacorr,
                      prā{\lost}{\lost}{\lost}{\lost}{\lost}{\lost}      \uncl{yaḥ śiva} {\il}{\il}{\il}{\il} \uncl{ya devair api} \msNb
            \ kāyatīrthopavarṇano\lem  \msCb\msCc\msNa\msNb\msNc\Ed; kāyatī{\il}{\il}{\il}rṇṇano \msCa\oo
                 nāmādhyāyo daśamaḥ\lem  \mssCaCbCc\msNa\msNb\msNc; nāma daśamo 'dhyāyaḥ \Ed}
\bekveg\szamveg\vfill\phpspagebreak\szam\bek\versno=0\fejno=11
\thispagestyle{empty}



\alfejezet{\textbf{ekādaśamo 'dhyāyaḥ}}\jump\jump 
\vers


\alalfejezet{caturāśramadharmavidhānaḥ}
devy uvāca~{\dandab}\dontdisplaylinenum 

sarvayajñaḥ paraśreṣṭha asti anyaḥ surottama\thinspace{\danda} \dontdisplaylinenum 
            \paral{\textit{{\normalfont Testimonia for this chapter:    \msCa\ ff.\thinspace 208v--210r,
                                                \msCb\ ff.\thinspace 214r--215v,
                                                \msCc\ ff.\thinspace 285v--287v,
                                                \msNa\ ff.\thinspace 15v--17v,
                                                \msNb\ ff.\thinspace 221v--223v,
                                                \msNc\ ff.\thinspace 223v--225v;
                                                        \mssCaCbCc\ = \msCa + \msCb + \msCc }}}

alpakleśam anāyāsa arthaprāyaṁ vineśvara \veg\dontdisplaylinenum
\varr{
        \ \vb anyaḥ\lem  \msCb\msNa\msNc; anya \msCa\msCc\msNb, cānyā \Ed\oo
                 °ttama\lem  \mssCaCbCc\msNa\msNb\Ed; °ttamaḥ \msNc
        \ \vc °nāyāsa\lem  \mssCaCbCc\msNc\Ed; °nāyā\uncl{saṁ} \msNa, °\uncl{nāyā}saṁ \msNb
        \ \vd °rthaprāyaṁ\lem  \msNapcorr\msNc; °rthaprāya \mssCaCbCc,
                      °rthaprārthaprāyaṁ \msNaacorr, \uncl{°rthaprāya} \msNb, °thāmnāya \Ed\oo
                 vineśvara\lem  \mssCaCbCc\msNa\msNc; \uncl{vineśvara} \msNb, sureśvara \Ed}

sarvayajñaphalāvāpti daivataiś cāpi pūjitam\thinspace{\dandab} \dontdisplaylinenum

kathayasva suraśreṣṭha mānuṣāṇāṁ hitāya vai \veg\dontdisplaylinenum
\varr{
        \ \va daivatai°\lem  \msCa\msCb\msNa\Ed; devatai° \msCc\msNc, \uncl{devatai} \msNb      
        \ \vcd °śreṣṭha mānuṣāṇāṁ hitāya vai\lem  \mssCaCbCc\msNa\msNc\Ed; °śre{\lost}{\lost}{\lost}{\lost}{\lost}{\lost}{\lost}{\lost}{\lost}{\lost} \msNb}

maheśvara uvāca~{\dandab}\dontdisplaylinenum 
\varr{
        \ \vo mahe°\lem  \mssCaCbCc\msNa\msNb\Ed; mehe° \msNc}

na tulyaṁ tava paśyāmi dayā bhūteṣu bhāmini\thinspace{\danda} \dontdisplaylinenum

kim anyat kathayiṣyāmi dayā yatra na vidyate \veg\dontdisplaylinenum
\varr{
        \ \va tulyaṁ tava\lem  \msNa\msCb\msCc\msNb\msNc\Ed; {\lost}{\lost}{\lost}{\lost} \msCa
        \ \vb bhāmini\lem  \msCa\msCb\msNa\msNb\msNc\Ed; bhāmi \msCc
        \ \vc kim anya°\lem  \mssCaCbCc\msNa\msNc\Ed; kimyanya° \msNb}

sadāśivamukhāt pūrvaṁ śrutaṁ me varasundari\thinspace{\dandab} \dontdisplaylinenum

ś\textsubring{r}ṇu devi pravakṣyāmi dharmasāram anuttamam \veg\dontdisplaylinenum


\alalfejezet{g\textsubring{r}hasthaḥ(?)}\varr{
        \ \vc devi pravakṣyāmi\lem  \msCb\msCc\msNa\msNb; te devi vakṣyāmi \msCa\msNc\Ed
        \ \vd °sāram anuttamam\lem  \msCa\msCb\msNa\msNb\msNc\Ed; °sārasamuccayam \msCc}

vinārthena tu yo yajñaḥ sa yajñaḥ sārvakāmikaḥ\thinspace{\dandab} \dontdisplaylinenum
             \paral{\textit{\vab {\normalfont  See a sequence or list of the four āśramas in 4.75: }
                g\textsubring{r}hastho brahmacārī ca vānaprastho 'tha bhaikṣukaḥ;
                {\normalfont  see also 5.9: } 
                etac chaucaṁ g\textsubring{r}hasthānāṁ dviguṇaṁ brahmacāriṇām{\thinspace\danda}
                vānaprasthasya triguṇaṁ yatīnāṁ tu caturguṇam{\thinspace\ketdanda}}}

akṣayaś cāvyayaś caiva sarvapātakanāśanaḥ \veg\dontdisplaylinenum
\varr{
        \ \vb yajñaḥ\lem  \mssCaCbCc\msNa\msNb\msNc; yajña \Ed\oo
                 sārvakāmikaḥ\lem  \msCb\Ed; sarvakālikaḥ \msCa\msNc,
                                   sarvakāmika \msCc, sārvakālikaḥ \msNa, sārvakāmikāḥ \msNb
        \ \vc akṣayaś cāvyayaś\lem  \msCb\msNb\msNc\Ed; akṣayaṁ cāvyayaṁ \msCa\msCc\msNa
        \ \vd °nāśanaḥ\lem  \msCa\msNa\msNb\msNc; °nāśanam \msCb\Ed, °nāśana \msCc}

bahuvighnakaro hy artho bahvāyāsakaras tathā\thinspace{\dandab} \dontdisplaylinenum

brahmahatyā ivendrasya pravibhāgaphalā sm\textsubring{r}tā \veg\dontdisplaylinenum
                     \paral{\textit{\vcd {\normalfont See e.g.\ \BhP\ 6.9.6: } brahmahatyām añjalinā jagrāha yad apīśvaraḥ{\thinspace\danda}
                                    saṁvatsarānte tad aghaṁ bhūtānāṁ sa viśuddhaye{\thinspace\danda}
                                    bhūmyambudrumayoṣidbhyaś caturdhā vyabhajad dhariḥ{\thinspace\ketdanda}}}
\varr{
        \ \va °karo\lem  \msCa\msCb\msNa\msNb\msNc; °karā \msCc\Ed\oo
                 hy artho\lem  \mssCaCbCc\msNa\msNb\msNc; hy ertho \Ed
        \ \vb karas tathā\lem  \mssCaCbCc\msNa\msNb\msNc; karatasthā \Ed
        \ \vd pravibhāga°\lem  \msCb; pravibhoga° \msCa\msCc(?)\msNa\msNc\Ed, pratibhoga° \msNb\oo
                 °phalā sm\textsubring{r}tā\lem  \msCc; °phalaḥ sm\textsubring{r}taḥ \msCapcorr\msCb\msNa\msNb\msNc, 
                                                °phala sm\textsubring{r}taḥ \msCaacorr, °pradaḥ sm\textsubring{r}taḥ \Ed}

pañcaśodhyena śodhyeta arthayajño varānane\thinspace{\dandab} \dontdisplaylinenum

śodhite tu phalaṁ śuddham aśuddhe niṣphalaṁ bhavet \veg\dontdisplaylinenum
\varr{
        \ °yajño\lem  \msCa\msCb\msNa\msNb\msNc\Ed; °yajña \msCc
        \ \vcd śuddham aśuddhe\lem  \mssCaCbCc\msNb\msNc; śuddhaṁm aśuddhe \msNa, śuddham aśuddhaṁ \Ed}

devy uvāca~{\dandab}\dontdisplaylinenum 
\varr{
        \ \vo devy uvāca\lem  \mssCaCbCc\msNa\msNbpcorr\msNc\Ed; \om\ \msNbacorr}

pañcaśodhye suraśreṣṭha saṁśayo 'tra bhaven mama\thinspace{\danda} \dontdisplaylinenum

kathayasva vibhāgena śrotum icchāmi tattvataḥ \veg\dontdisplaylinenum
\varr{
        \ \va °śodhye\lem  \mssCaCbCc\msNa; °śodhya \msNb\msNc, °śodhyaḥ \Ed\oo
                 °śreṣṭha\lem  \msCa\msCb\msNa\msNb\msNc\Ed; °sre\uncl{mna} \msCc
        \ \vb 'tra bhave°\lem  \mssCaCbCc\msNa\msNb\msNc; 'trā bhava° \Ed}

rudra uvāca~{\dandab}\dontdisplaylinenum 

manaḥśuddhis tu prathamaṁ dravyaśuddhir ataḥ param\thinspace{\danda} \dontdisplaylinenum

mantraśuddhis t\textsubring{r}tīyā tu karmaśuddhir ataḥ param \danda\dontdisplaylinenum

pañcamī sattvaśuddhis tu kratuśuddhiś ca pañcadhā \veg\dontdisplaylinenum
\varr{
        \ \vb °śuddhir ataḥ\lem  \mssCaCbCc\msNa\msNc\Ed; °śuddhigataḥ \msNb
        \ \va mantraśuddhis t\textsubring{r}tīyā\lem  \mssCaCbCc\msNa\msNb\Ed; mantraddhi t\textsubring{r}tīyā \msNc
        \ \vb karmaśuddhi°\lem  \mssCaCbCc\msNa\msNb\Ed; karmasiddhi \msNc
        \ \vc pañcamī\lem  \mssCaCbCc\msNa\msNb\msNc; pañcamaṁ \Ed\oo
                 °śuddhis tu\lem  \mssCaCbCc\msNb\msNc; °śuddhiś ca \msNa\Ed
        \ \vd °śuddhiś ca pañcadhā\lem  \msCa\msCb\msNb\msNc\Ed; °śuddhis tu pañcadhā \msCc,
                                                                °śuddhir ataḥ param \msNa}

manaḥśuddhir nāma aviparītabhāvanayā\thinspace{\dandab} \dontdisplaylinenum 

dravyaśuddhir nāma ananyāyopārjitadravyena \veg\dontdisplaylinenum 
\varr{
        \ \vab °śuddhir nā°\lem  \msCa\msCb\msNa\msNb\msNc\Ed; °śuddhi nā° \msCc\oo
                  °bhāvanayā\lem  \mssCaCbCc\msNc\Ed; °bhāvanavā \msNa, °bhāvanatayā \msNb
        \ \vcd °śuddhir nā°\lem  \msCa\msCb\msNa\msNb\Ed; °śuddhi nā° \msCc\msNc\oo
                  ananyāyo°\lem  \msCb\msNa\msNb\msNc; ana{\lost}yo° \msCa, anyāyo° \msCc, svalponyāyo° \Ed\oo
                  °dravyena\lem  \mssCaCbCc\msNa\msNc\Ed; °vyena \msNb}

mantraśuddhir nāma svaravyañjanayuktatayā\thinspace{\dandab} \dontdisplaylinenum 

kriyāśuddhir nāma yathākramāviparītatayā \danda\dontdisplaylinenum 

sattvaśuddhir nāma rajastama-apradhānatayā \veg\dontdisplaylinenum 

\vers
\varr{
        \ \vab mantraśuddhir °nā\lem  \msCa\msCb\msNb\Ed; mantraśuddhi nā° \msCc\msNc, mantras tuddinā° \msNa\oo
                  °yuktatayā\lem  \msCa\msCc\msNa\msNb\msNc\Ed; °yuktayā \msCb
        \ \vcd °śuddhir nā°\lem  \msCa\msCb\msNa\msNc\Ed; °śuddhi nā° \msCc\msNb\oo
                  °kramā°\lem  \msCa\msCb\msNa\msNb\msNc\Ed; °krama° \msCc\oo
                  °rītatayā\lem  \msCa\msCc\msNa\msNb\Ed; °rītayā \msCb, °{\lost}{\lost}tayā \msNc
        \ \vef °śuddhir nā°\lem  \msCb\msNa\msNb\msNc\Ed; °śuddhi nā° \msCa\msCc\oo
                  °dhānatayā\lem  \mssCaCbCc\msNa\msNb\Ed; °dhānata \msNc}

vidhim evaṁ yadā śudhyed yadi yajñaṁ karoti hi\thinspace{\dandab} \dontdisplaylinenum

tasya yajñaphalāvāptir janmam\textsubring{r}tyuś ca no bhavet \veg\dontdisplaylinenum
\varr{
        \ \va °dhim evaṁ yadā\lem  \msCb\Ed; °dhim eva yadā \msCa\msCc\msNa, °dhim eva ya \msNb,
                                                              °dhim evaṁ yathā \msNc
        \ \vab śudhyed yadi\lem  \conj; sūyed yadi \msCa\msNa\msNc, pūrya yadi \msCb,  
                                 sūryed yadi \msCc, sūyed yati \msNb, śuddhya ya° \Ed
        \ \vb yajñaṁ\lem  \msCa\msCb\msNa\Ed; yajña \msCc\msNc, saṁjña \msNb\oo
                 hi\lem  \mssCaCbCc\msNa\msNc\Ed; \om\ \msNb
        \ \vcd °vāptir ja°\lem  \msCa\msCb\Ed; °vāpti ja \msCc\msNb\msNc, °vāpi ja° \msNa}

vinārthena tu yo yajñaṁ karoti varasundari\thinspace{\dandab} \dontdisplaylinenum

na tasya tatphalāvāptiḥ sarvayajñeṣv aśeṣataḥ \veg\dontdisplaylinenum
\varr{
        \ \vb °sundari\lem  \mssCaCbCc\msNa\msNb\msNc; °sundarī \Ed
        \ \vd °yajñeṣv aśeṣataḥ\lem  \mssCaCbCc\msNa\msNb\msNc; °yajñeṣu śeṣataḥ \Ed}

yajñavāṭa kurukṣetraṁ sattvāvāsak\textsubring{r}tālayaḥ\thinspace{\dandab} \dontdisplaylinenum

pratyāhāra mahāvediḥ kuśaprastarasaṁyamaḥ \veg\dontdisplaylinenum
\varr{
        \ \va °vāṭa kuru°\lem  \msCa\msCc\msNa\msNb\msNc; °vāṭaṅ kuru° \msCb, °vāṭak\textsubring{r}ta° \Ed\oo
                 °kṣetraṁ\lem  \mssCaCbCc\msNa\msNb\Ed; °kṣetra \msNc
        \ \vb sattvā°\lem  \msCa\msCbpcorr\msCc\msNa\msNb\msNc\Ed; satvāsatvā° \msCbacorr\oo
                 °layaḥ\lem  \msCa\msCb\msNa\msNb\msNc\Ed; °layam \msCc
        \ \vc °vediḥ\lem  \eme; °vedi \mssCaCbCc\msNa\msNb\msNc, °devi \Ed}

vidhi niyamavistāro dhyānavahnipradīpitaḥ\thinspace{\dandab} \dontdisplaylinenum

yogendhanasamijjvālatapodhūmasamākulaḥ \veg\dontdisplaylinenum
\varr{
        \ \va vidhi\lem  \mssCaCbCc\msNa\msNb\msNc; vidhir \Ed\oo
                 °vistāro\lem  \msCa\msCc\msNa\msNb\msNc\Ed; °vistārau \msCb
        \ \vb dhyānavahnipradīpitaḥ\lem  \msCa\msNa; dhyānaṁ vahnipradīpitaḥ \msCb,
                       dhyānam agnipradīpitaḥ \msCc, dhyāna agnipradīpanaḥ \msNb,
                       dhyānavahniḥ pradīpitaḥ \msNc, dhyānav\textsubring{r}ddhir pradīpinaḥ \Ed
        \ \vcd °ndhanasamijjvālatapodhūma°\lem  \msNb\msNc; °ndhanasamijjvālatapodhūpa° \msCa,
               °\uncl{ndha}satvamijjvālatapodhūma° \msCb, °ndhanasamijvālatapodhūma° \msCc,
               °ndhanaśami\uncl{ta}jvālatayodhūya° \msNa, °ndhanasamijjvālā tapodhūma° \Ed}

pātranyāsa śivajñānaṁ sthālīpāka śivātmakaḥ\thinspace{\dandab} \dontdisplaylinenum

ājyāhutim avicchinnaṁ lambakaśruvapātitaḥ \veg\dontdisplaylinenum
\varr{
        \ \va pātra°\lem  \mssCaCbCc\msNa\msNb\Ed; pātrā° \msNc
        \ \vc °cchinnaṁ\lem  \mssCaCbCc\msNa\msNb\Ed; °cchinna \msNc
        \ \vd lambaka°\lem  \msCa\msCb\msNa\msNb\msNc; \uncl{la}mbaka° \msCc, tryambaka° \Ed\oo
                 °pātitaḥ\lem  \mssCaCbCc\msNa\msNb\msNc; °pātitam \Ed}

dhāraṇādhvaryuvat k\textsubring{r}tvā prāṇāyāmaś ca \textsubring{r}tvijaḥ\thinspace{\dandab} \dontdisplaylinenum

tarkayuktaḥ savistāraḥ samādhir vayatāpanaḥ \veg\dontdisplaylinenum
\varr{
        \ \va °dhvaryuva°\lem  \msNb; °dhvaryava° \mssCaCbCc, °\uncl{dhva}ryava° \msNa,
                                                   dhva{\il}{\il} \msNc, dharmava° \Ed
        \ \vc °yuktaḥ\lem  \msCa\msCb\msNb\msNc\Ed; °yukta \msCc, °yuktiḥ \msNa\oo
                 °vistāraḥ\lem  \msCa\msCb\msNa\msNb\msNc\Ed; °vistāro \msCc}

brahmavidyāmayo yūpaḥ paśubandho manonmanaḥ\thinspace{\dandab} \dontdisplaylinenum

śraddhā patnī viśālākṣi saṁkalpaḥ pada śāśvatam \veg\dontdisplaylinenum
\varr{
        \ \vb °nmanaḥ\lem  \msCa\msNa\msNb\Ed; °tmanaḥ \msCb\msCc\msNc
        \ \vc patnī\lem  \msCb\msCc\msNa\msNb\msNc\Ed; \uncl{patnī} \msCa\oo
                 viśālākṣi\lem  \mssCaCbCc\msNa\msNb; viśālākṣī \msNc\Ed
        \ \vd °kalpaḥ\lem  \eme; °kalpa \mssCaCbCc\msNa\msNb\msNc\Ed\oo
                 pada śāśvatam\lem  \msCb\msCc\msNa\msNb\msNc\Ed; pa\uncl{da}{\lost}śvatam \msCa}

pañcendriyajayotpannaḥ puroḍāśo 'm\textsubring{r}tāśanaḥ\thinspace{\dandab} \dontdisplaylinenum

brahmanādo mahāmantraḥ prāyaścittānilo jayaḥ \veg\dontdisplaylinenum
\varr{
        \ \vb °ḍāśo\lem  \mssCaCbCc\msNb\msNc; °bhā \msNaacorr, °bhāse \msNapcorr, °bhāge \Ed\oo
                 m\textsubring{r}tā°\lem  \msCa\msCb\msNa\msNb\msNc\Ed; m\textsubring{r}gā° \msCc
        \ \vd °ttānilo\lem  \msCa\msCb\msNa\msNc\Ed; °ttanilo \msCc\msNb\oo
                 jayaḥ\lem  \mssCaCbCc\msNa\msNb\msNc; jalāḥ \Ed}

somapāna parijñānam upākarma caturyamaḥ\thinspace{\dandab} \dontdisplaylinenum

itihāsa jalasnānaṁ purāṇak\textsubring{r}ta-m-ambaraḥ \veg\dontdisplaylinenum
\varr{
        \ \va pari°\lem  \msCa\msCb\msNa\msNb\msNc\Ed; para° \msCc
        \ \vc °snānaṁ\lem  \msCa\msCc\msNa\msNb\msNc\Ed; °snāna \msCb
        \ \vd purāṇa°\lem  \mssCaCbCc\msNa\msNb\msNc; purāṇaṁ \Ed\oo
                 °k\textsubring{r}tam ambaraḥ\lem  \msCa\msCc\msNa\msNb\msNc\Ed; °k\textsubring{r}tambaram \msCb\ \unmetr}

iḍāsuṣumnāsaṁvedye snānam ācamanaṁ sak\textsubring{r}t\thinspace{\dandab} \dontdisplaylinenum

saṁtoṣātithim ād\textsubring{r}tya dayābhūtadvijārcitaḥ \veg\dontdisplaylinenum
\varr{
        \ \va °suṣumnā°\lem  \msCa\msCb\msNa\msNb\msNc\Ed; °suṣumna° \msCc\oo
                 °vedye\lem   \msCa\Ed; °vedya \msCb\msNb, °vedyeḥ \msCc, °vaidya \msNa, °bhedo \msNc
        \ \vb sak\textsubring{r}t\lem  \msCa\msCb\msNa\msNb\msNc\Ed; viduḥ \msCc
        \ \vc °toṣātithim ād\textsubring{r}tya\lem  \mssCaCbCc\msNa\msNc\Ed; °toṣatithim āv\textsubring{r}tya \msNb
        \ \vd °dvijā°\lem  \msCa\msCc\msNa\msNb\msNc\Ed; °dayā° \msCb}

brahmakūrca guṇātīta havirgandha nirañjanaḥ\thinspace{\dandab} \dontdisplaylinenum

brahmasūtraṁ trayas tattvaṁ bodhanā muṇḍitaṁ śiraḥ \veg\dontdisplaylinenum
\varr{
        \ \vb °havir ga°\lem  \msCa\msCc\msNb\msNc\Ed; °havi\uncl{r ga}° \msCb, °haviga \msNa
        \ \vc °sūtraṁ trayas\lem  \msCb\msNb\msNc\Ed; °sūtran trayastayas \msCa,
                                        °sūtraṁ traya \msCc, °sūtratrayaṁ \msNa
        \ \vd muṇḍitaṁ\lem  \msCa\msCc\msNa\msNb\Ed; muṇḍita° \msCb\msNc\unmetr}

niv\textsubring{r}ttyādi caturvedaś catuḥprakaraṇāsanaḥ\thinspace{\dandab} \dontdisplaylinenum

dakṣiṇām abhayaṁ bhūte dattvā yajñaṁ yajet sadā \veg\dontdisplaylinenum
            \paral{\textit{\vc {\normalfont cf.\ 22.14ab:} dakṣiṇābhaya bhūtebhyaḥ paśubandhaḥ svayaṁk\textsubring{r}taḥ}}
\varr{
        \ \va niv\textsubring{r}ttyā°\lem  \eme; niv\textsubring{r}tyā° \mssCaCbCc\msNa\msNb\msNc, nirv\textsubring{r}tyā° \Ed
        \ \vb °prakaraṇāsanaḥ\lem  \msCa\msCb\msNa\msNb\msNc; prakaranāśanaḥ \msCc, prakaraśāsanaḥ \Ed
        \ \vc °bhayaṁ bhūte\lem  \msCa\msCc\msNa\msNb\msNc\Ed; °bhakṣayam bhūtai \msCb
        \ \vd yajñaṁ yajet\lem  \mssCaCbCc\msNa\msNb\msNc; yajña dadat \Ed}

vinārthaṁ yajñasamprāptiḥ kathitā te varānane\thinspace{\dandab} \dontdisplaylinenum

āsahasrasya yajñānāṁ phalaṁ prāpnoti nityaśaḥ \veg\dontdisplaylinenum
\varr{
        \ \va vinārthaṁ\lem  \msCa\msCb\msNa\msNb\msNc\Ed; vinārtha \msCc
        \ \vb kathitā te\lem  \msCa\msCb\msNa\msNb\msNc; kathi\uncl{to} smi \msCc, kathitas te \Ed\oo
                 varānane\lem  \msCa\msCb\msNa\msNb\msNc\Ed; va\uncl{rā}nane \msCc
        \ \vd prāpnoti\lem  \msCb\msCc\msNa\msNb\msNc\Ed; prā{\lost}ti \msCa\oo
                 nityaśaḥ\lem  \mssCaCbCc\msNa\msNc\Ed; mānavaḥ \msNb}

āśramaḥ prathamas tubhyaṁ kathito 'sti varānane\thinspace{\dandab} \dontdisplaylinenum

sadāśivena saddharmaṁ daivatair api pūjitam \veg\dontdisplaylinenum


\alalfejezet{brahmacaryam}\varr{
        \ \va āśramaḥ\lem  \msCa\msNa\msNb\msNc\Ed; āśrama \msCb\msCc\oo
                 °s tubhyaṁ\lem  \msCa\msCb\msNa\msNb\msNc; °syeṣa \msCc, °syaivaṁ \Ed
        \ \vb 'sti\lem  \msCa\msCb\msNa\msNc; smi \msCc\msNb\Ed
        \ \vc °dharmaṁ\lem  \msCa\msCc\msNa\msNb\msNc; °dha\uncl{rmaṁ} \msCb, °dharme \Ed
        \ \vd daiva°\lem  \mssCaCbCc\msNa\msNc; deva° \msNb\Ed\oo
                 pūjitam\lem  \msCa\msCc\msNa\msNb\msNc\Ed; pūpūjitam \msCb}

brahmacaryaṁ nibodhedaṁ ś\textsubring{r}ṇuṣvāvahitā śubhe\thinspace{\dandab} \dontdisplaylinenum

dvitīyam āśramaṁ devi sarvapāpavināśanam \veg\dontdisplaylinenum
            \paral{\textit{\vcd {\normalfont  cf.\ MBh 12.184.10A: } gārhasthyaṁ khalu dvitīyam āśramaṁ vadanti}}
\varr{
        \ \va °caryaṁ\lem  \mssCaCbCc\msNb\msNc\Ed; °carya \msNa
        \ \vb °vahitā śubhe\lem  \msCa\msCb\msNa\msNc\Ed; °vahito bhava \msCc, °vahito śubhe \msNb
        \ \vd °vināśanam\lem  \mssCaCbCc\msNa\msNc\Ed; °pranāśanam \msNb}

vrataṁ brahmaparaṁ dhyānaṁ sāvitrī prak\textsubring{r}tau layaḥ\thinspace{\dandab} \dontdisplaylinenum
            \paral{\textit{\vab {\normalfont cf.\ 16.8cd: ! }}}

brahmasūtrākṣaraṁ sūkṣmaṁ triguṇālaya mekhalam \veg\dontdisplaylinenum
\varr{
        \ \va °paraṁ dhyānaṁ\lem  \mssCaCbCc\msNa\msNb\msNc; °parijñānaṁ \Ed
        \ \vb °k\textsubring{r}tau layaḥ\lem  \eme; °k\textsubring{r}tir layam \msCa\msNa\msNc\Ed,
                         °k\textsubring{r}tālayam \msCb, °k\textsubring{r}tīlayam \msCc, °k\textsubring{r}tilaḥ \msNb
        \ \vd °laya\lem  \msCb\msCc\msNa\msNb\msNc\Ed; °la{\lost} \msCa\oo
                 mekhalam\lem  \mssCaCbCc\msNa\msNb\msNc; yat phalam \Ed}

dama daṇḍa dayā pātraṁ bhikṣā saṁsāramocanam\thinspace{\dandab} \dontdisplaylinenum

tryāyuṣaṁ dvyakṣarātītaṁ jñānabhaṣma-alaṅk\textsubring{r}tam \veg\dontdisplaylinenum
\varr{
        \ \va daṇḍa dayā\lem  \mssCaCbCc\msNb\msNc; daṇḍādayā \msNa, daṇḍādayo \Ed\oo
                 pātraṁ\lem  \mssCaCbCc\msNa\msNc\Ed; pātra \msNb
        \ \vc °yuṣaṁ\lem  \mssCaCbCc\msNb\msNc\Ed; °yuṣa \msNa
        \ \vd bhasma\lem  \mssCaCbCc\msNa\msNb\msNc; bhasmam \Ed}

snānavrataṁ sadāsatyaṁ śīlaśaucasamanvitam\thinspace{\dandab} \dontdisplaylinenum

agnihotra trayas tattvaṁ japa brahmabilasvaraḥ \veg\dontdisplaylinenum
\varr{
        \ \va °vrataṁ\lem  \msCa\msCc\msNa\msNb; °vrata \msCb\msNc\Ed
        \ \vc °hotra trayas tattvaṁ\lem  \msNa\msNc\Ed; °hotran trayas tatvaṁ \msCa,
                        °hotra\uncl{ta}yas tatvaṁ \msCb, °hotratrayaṁ tatvā \msCc, °hotraṁ trayaṁs tatvaṁ \msNb
        \ \vd °bilasvaraḥ\lem  \corr; °bilaśvaraḥ \mssCaCbCc\msNa\msNb, °bileśvara \msNc\Ed}

dvitīya āśramo devi yathāha bhagavān śivaḥ\thinspace{\dandab} \dontdisplaylinenum

mayāpi kathitaṁ tubhyaṁ janmam\textsubring{r}tyuvināśanam \veg\dontdisplaylinenum


\alalfejezet{vānaprasthaḥ}\varr{
        \ \va dvitīya āśramo\lem  \msCa\msCb\msNa\msNb\msNc; dvitīyam āśramo \msCc, dvitīyam āśramaṁ \Ed
        \ \vb yathāha\lem  \msCa\msCb\msNa\msNc; yathāhaṁ \msCc\msNb, yad āha \Ed
        \ \vc mayāpi kathitaṁ tu°\lem  \eme; mamāpi kathitaṁ tu° \mssCaCbCc\msNa\msNb,
                                mamāpi kathitas tu° \msNc, mayāpi kathito tu° \Ed
        \ \vd °m\textsubring{r}tyu°\lem  \msCb\msCc\msNa\msNb\msNc\Ed; °m\textsubring{r}{\lost}° \msCa\oo
                 °nāśanaṁ\lem  \mssCaCbCc\msNa\msNb\Ed; °nāśanaḥ \msNc}

vānaprasthavidhiṁ vakṣye ś\textsubring{r}ṇuṣvāyatalocane\thinspace{\dandab} \dontdisplaylinenum

yathāśrutaṁ yathātathyam \textsubring{r}ṣidaivatapūjitam \veg\dontdisplaylinenum
\varr{
        \ \va °vidhiṁ\lem  \msCa\msCc\msNa\msNb\msNc\Ed; °vidhi \msCb
        \ \vd °daivata°\lem  \msCa\msCb\msNa\msNb\msNc\Ed; °devata° \msCc}

vairāgyavanam āśritya niyamāśramam āharet\thinspace{\dandab} \dontdisplaylinenum

śīlaśailad\textsubring{r}ḍhadvāre prākāre vijitendriyaḥ \veg\dontdisplaylinenum
\varr{
        \ \va vairāgya°\lem  \mssCaCbCc\msNa\msNb\msNc; vairāgyā \Ed
        \ \vb niyamā°\lem  \mssCaCbCc\msNapcorr\msNb\msNc\Ed; mā° \msNaacorr\oo
                 °śramam ā°\lem  \msCb\msCc\msNa\msNb\msNc\Ed; °śramano haret \msCa
        \ \vc °d\textsubring{r}ḍha°\lem  \mssCaCbCc\msNa\msNb\msNc; °d\textsubring{r}ṣa° \Ed
        \ \vd °kāre\lem  \msCa\msCb\msNa\msNb\msNc\Ed; °kāra° \msCc}

adhibhūtaḥ sm\textsubring{r}to mātā adhyātmaś ca pitā tathā\thinspace{\dandab} \dontdisplaylinenum
            \paral{\textit{\vab {\normalfont cf.\ 22.10ab:} adhyātmanagarasphītaḥ adhibhūtajanākulaḥ}}

adhidaivika-m-ācāryo vyavasāyāś ca bhrātaraḥ \veg\dontdisplaylinenum
\varr{
        \ \va sm\textsubring{r}to\lem  \msCa\msCc\msNa\msNb\msNc; {\lost}{\lost} \msCb, sm\textsubring{r}tau \Ed
        \ \vc adhidaivika°\lem  \emeGoodall; 
                \uncl{a}{\lost}\uncl{bhau}{\lost}ka° \msCa, adhibhautika° \msCb\msCc\msNa\msNc\Ed, adhibhauktika° \msNb
        \ \vd vyavasāyāś ca\lem  \mssCaCbCc\msNa\msNb\msNc; vyavasāyaś ca \Ed}

śrutiḥ sm\textsubring{r}tiḥ sm\textsubring{r}tā bhāryā prajñā putraḥ kṣamānujaḥ\thinspace{\dandab} \dontdisplaylinenum

maitrī bandhur jaṭā cāpaṁ karuṇā supavitrakam \veg\dontdisplaylinenum
\varr{
        \ \va sm\textsubring{r}tā\lem  \msCa\msCc\msNa\msNb\msNc\Ed; sm\textsubring{r}to \msCb
        \ \vc bandhur ja°\lem  \msCa\msCb\msNa\msNc\Ed; bandhu ja° \msCc\msNb}

muditā mauna catvāraḥ sarvakāryam upekṣakā\thinspace{\dandab} \dontdisplaylinenum

yamavalkalasaṁvītas tapaḥk\textsubring{r}ṣṇājinādharaḥ \veg\dontdisplaylinenum
\varr{
        \ \va mauna catvāraḥ\lem  \msCa\msNa\msNb\msNc\Ed; maunaś catvāraḥ \msCb, mauna catvāra \msCc
        \ \vb °kāryam u°\lem  \mssCaCbCc\msNb\msNc\Ed; °kāryām u° \msNa\oo 
                 °pekṣakā\lem  \mssCaCbCc\msNa\msNb\msNc; °pekṣayā \Ed
        \ \vc °saṁvīta°\lem  \mssCaCbCc\msNa\msNb\msNc; °sānvīta° \Ed
        \ \vd °k\textsubring{r}ṣṇā°\lem  \msCa\msCb\msNa\msNb\msNc\Ed; °k\textsubring{r}ṣṇāṁ \msCc\oo
                 °jinādharaḥ\lem  \msNc; °jinadharaḥ \mssCaCbCc\msNa\msNb\ \unmetr, °jinaṁ puraḥ \Ed}
 
uttarāsaṅgam āsīno yogapaṭṭad\textsubring{r}ḍhavrataḥ\thinspace{\dandab} \dontdisplaylinenum

vedaghoṣeṇa ghoṣeṇa prāṇāyāmo 'gnihāvanam \veg\dontdisplaylinenum
\varr{
        \ \vb °d\textsubring{r}ḍha°\lem  \mssCaCbCc\msNa\msNc\Ed; °d\textsubring{r}ṣṭa° \msNb\oo 
                 °vrataḥ\lem  \msCb\msCc\msNa\msNb\msNc\Ed; {\lost}{\lost} \msCa
        \ \vc veda°\lem  \msCb\msCc\msNa\msNb\msNc\Ed; {\lost}da° \msCa\oo
                 °ṇa ghoṣeṇa\lem  \msCa\msCb\msNa\msNb\msNc\Ed; °ṇa ghoṣīṇa \msCc
        \ \vd °hāvanam\lem  \msCa\msNa\msNb\msNc\Ed; °hāvana \msCc, °\uncl{hāvanam} \msCb}

jitaprāṇam\textsubring{r}gākūlo dh\textsubring{r}ti yajñaḥ kriyā japaḥ\thinspace{\dandab} \dontdisplaylinenum

arthasaṁgraha śāstreṣu sakhā damadayādayaḥ \veg\dontdisplaylinenum
\varr{
        \ \vb °japaḥ\lem  \msCa\msCb\msNa\msNb\msNc\Ed; °jiṇaḥ \msCc
        \ \vd sakhā\lem  \mssCaCbCc\msNa\msNc\Ed; sakho \msNb\oo
                 damada°\lem  \msCapcorr\msCb\msNa\msNb\msNc\Ed; dayada° \msCc, dama° \msCaacorr}

śivayajñaṁ prayuñjīta sādhanāṣṭakapūjanam\thinspace{\dandab} \dontdisplaylinenum
            \paral{\textit{\vb {\normalfont cf.\ Dharmaputrikā 2.1:} 
                aṣṭabhiḥ sādhanair ebhiś cittaṁ kāyañ ca yatnataḥ{\thinspace\danda}
                śodhayitvā tato yogī yogābhyāsaṁ samācaret{\thinspace\ketdanda}}}

pañcabrahmajalaiḥ pūtaḥ satyatīrthaśivahrade \veg\dontdisplaylinenum
\varr{
        \ \va °yajñaṁ\lem  \msCa\msCb\msNa\msNb\Ed; °yajña \msCc\msNc
        \ \vb °pūjanam\lem  \msCa\msCb\msNa\msNb\msNc\Ed; °pūjikaṁ \msCc
        \ \vc °brahmajalaiḥ pūtaḥ\lem  \mssCaCbCc\msNa\msNc\Ed; bra{\lost}{\lost}{\lost}{\lost}{\lost} \msNb
        \ \vd °tīrtha\lem  \mssCaCbCc\msNa\msNb\msNc; °tīrthaṁ \Ed}

snānam ācamanaṁ k\textsubring{r}tvā saṁdhyātrayam upāśrayet\thinspace{\dandab} \dontdisplaylinenum
            \paral{\textit{\vb {\normalfont See 11.59cd:} śivasya h\textsubring{r}dayaṁ saṁdhyā tasmāt saṁdhyām upāsayet}}

akṣamālā purāṇārthaṁ japa śāntaṁ divāniśam \veg\dontdisplaylinenum
\varr{
        \ \va °camanaṁ\lem  \msCa\msCc\msNa\msNb\msNc\Ed; °canaṁ \msCb
        \ \vc akṣamālā\lem  \msCb\msCc\msNa\msNb\msNc\Ed; \uncl{akṣa}{\lost}lā \msCa\oo
                 purāṇārthaṁ\lem  \mssCaCbCc\msNa\Ed; purāṇāñ ca \msNb, purāṇā\uncl{rthā} \msNc
        \ \vd °śāntaṁ\lem  \msCapcorr\msCb\msCc\msNb\msNc\Ed; °śanti \msCaacorr\msNa}

jñānasalilasampūrṇamitihāsakamaṇḍaluḥ\thinspace{\dandab} \dontdisplaylinenum

pañcakarmakriyotkrānti japa pañcavidhaḥ sukham \veg\dontdisplaylinenum
\varr{
        \ \va °salila°\lem  \mssCaCbCc\msNa\msNb\msNc; °salīla° \Ed
        \ \vb °kamaṇḍaluḥ\lem  \mssCaCbCc\msNa\msNb\msNc; °kamaṇḍalu \Ed
        \ \vc °tkrāntija°\lem  \msCa\msCb\msNb; °krāntija° \msCc, °tkrāntir ja° \msNa, 
                                        °tkāntija° \msNc, 'krānti ja° \Ed}

sādhanaṁ śivasaṁkalpo yogasiddhiphalapradaḥ\thinspace{\dandab} \dontdisplaylinenum

saṁtoṣaphalam āhāraḥ kāmakrodhaparājitaḥ \veg\dontdisplaylinenum
\varr{
        \ \vd °daḥ\lem  \mssCaCbCc\msNa\msNb\msNc; °dam \Ed}

āśāpāśajayābhyāso dhyānayogaratipriyaḥ\thinspace{\dandab} \dontdisplaylinenum

atithibhyo 'bhayaṁ dattvā vānaprasthaś cared vratam \danda\dontdisplaylinenum

vānaprastham ayaṁ dharmaṁ yat pūrvam avadhāritam \veg\dontdisplaylinenum
\varr{
        \ \va °bhyāso\lem  \mssCaCbCc\msNa\msNb\msNc; °bhyāsa \Ed
        \ \vb °rati°\lem  \msCc\msNa\msNb\msNc; {\lost}{\lost} \msCa, °riti° \msCb, °ratiḥ \Ed
        \ \va atithibhyo 'bhayaṁ\lem  \mssCaCbCc\msNa\msNb\msNc; ārtibhyaś cābhayaṁ \Ed\oo
                 dattvā\lem  \msCa\msCb\msNa\msNb\msNc\Ed; dārā \msCc
        \ \vb °prasthaś ca°\lem  \msCa\msCb\msNa\msNc\Ed; °prastha ca° \msCc\msNb
        \ \vf yat pūrvam avadhāritam\lem  \msCc\Ed; 
                      gaditaṁ pūrvadhāritaṁ \msCa\msCb,
                      gadita pūrvadhāritaṁ \msNb,
                      gaditaṁ yat pūrvadhāritaṁ \msNaacorr\ \unmetr,
                      gaditaṁ yat pūrvam avadhāritaṁ \msNapcorr\ \unmetr,
                      gaditaṁ yat pūrvamedhāritaṁ \msNc\ \unmetr}

\ujvers\nemsloka 
! saṁsāroddharaṇam anityaharaṇam ajñānanirmūlanam 
\dontdisplaylinenum

\nemslokab 
! prajñāv\textsubring{r}ddhikaram amoghakaraṇaṁ kleśārṇavottāraṇam \danda\dontdisplaylinenum

\nemslokac 
! janmavyādhiharam akarmadahanaṁ sevet sa dharmottamam
\dontdisplaylinenum

\nemslokad 
? śraddhāpūrvakam eva yaḥ saniyamaṁ sākṣāc ca jīvan śivaḥ \veg\dontdisplaylinenum


\alalfejezet{parivrājakaḥ}
\vers
\varr{
        \ \va °haraṇam anityaharaṇam ajñā°\lem  \msCa\msCb\msNaacorr\msNb\msNc; 
                                °haraṇaṁ anityaharaṇan tajñā° \msNapcorr, 
                                °haraṇaṁm anityaharaṇam ajñā° \msCc\Ed
        \ \vb (prajñā°{\normalfont ...} °ttāraṇam)\lem  \mssCaCbCc\msNa\msNc\Ed; \om\ \msNb\oo
                 °karam amogha°\lem  \mssCaCbCc\msNa\ \unmetr; \om\ \msNb, °kam amogha° \msNc,
                                                                         °karaṁ prabodha° \Ed\oo
                 kleśārṇavo°\lem  \mssCaCbCc\msNc; kleśāṇṇavo° \msNa, \om\ \msNb, śokārṇavo° \Ed
        \ \vc sevet sa\lem  \msCa\msCb\msNa\msNc\Ed; seve sa \msCc, sevet ta \msNb
        \ \vd (śraddhā°{\normalfont ...} śivaḥ)\lem  \Ed; \om\ \mssCaCbCc\msNa\msNb\msNc}

parivrājakadharmo 'yaṁ kīrtayiṣyāmi tac ch\textsubring{r}ṇu\thinspace{\dandab} \dontdisplaylinenum

sukhaduḥkhaṁ samaṁ k\textsubring{r}tvā lobhamohavivarjitaḥ \veg\dontdisplaylinenum
            \paral{\textit{\vd {\normalfont  cf.\ 4.71: }  
                     kāmaḥ krodhaś ca lobhaś ca mohaś caiva caturvidhaḥ{\thinspace\danda}
                     catuḥśatrur nihantavyaḥ sarvathā vītakalmaṣaḥ{\thinspace\ketdanda}}}
\varr{
        \ \vb kīrtayiṣyāmi\lem  \msCb\msCc\msNa\msNb\msNc\Ed; kīrtayi{\lost}mi \msCa
        \ \vc °duḥkhaṁ\lem  \msCb; °duḥkha \msCa\msCc\msNa\msNb\msNc\Ed
        \ \vd lobhamoha°\lem  \msCb; lābhālobha° \msCa\msNa\msNb\msNc, lābhalobha° \msCc, lābhālābha° \Ed\oo
                 °varjitaḥ\lem  \mssCaCbCc\msNa\msNc\Ed; °varjitāḥ \msNb}

varjayen madhu māṁsāni paradārāṁś ca varjayet\thinspace{\dandab} \dontdisplaylinenum
            \paral{\textit{\va = {\normalfont Kūrmapurāṇa 2.27.12a etc.}}}

varjayec ciravāsaṁ ca paravāsaṁ ca varjayet \veg\dontdisplaylinenum
\varr{
        \ \va varjayen\lem  \msCa\msNb; varjayet \msCb\msCc\msNa\msNc\Ed
        \ \vc °vāsaṁ\lem  \mssCaCbCc\msNa\msNb\msNc; °vāsaś \Ed
        \ \vd °vāsaṁ\lem  \mssCaCbCc\msNa\msNb\msNc; °vāsaś \Ed}

varjayet s\textsubring{r}ṣṭabhojyāni bhikṣām ekāṁ ca varjayet\thinspace{\dandab} \dontdisplaylinenum

varjayet saṁgrahaṁ nityam abhimānaṁ ca varjayet \veg\dontdisplaylinenum
\varr{
        \ \vab (varjayet{\normalfont ...} ca varjayet)\lem  \msCa\msCc\msNa\msNb\msNc\Ed; \om\ \msCb
        \ \va varjayet s\textsubring{r}ṣṭa°\lem  \msCc(?)\msNa\msNc; varjayet m\textsubring{r}ṣṭa° \msCa, \om\ \msCb,
                varjjan m\textsubring{r}ṣṭa° \msNb, varjayen m\textsubring{r}ṣṭa° \Ed\oo
                 °bhojyāni\lem  \mssCaCbCc\msNa\msNb\Ed; °bhojāli(?) \msNc
         \ \vb °kṣām ekāṁ\lem  \msCa\msNb; \om\ \msCb, °kṣām ekaṁ \msCc\msNa,
                                                °kṣam ekañ \msNc, °kṣām ekaś \Ed}

susūkṣmaṁ manasā dhyātvā śucau pādaṁ vinikṣipet\thinspace{\dandab} \dontdisplaylinenum

na kupyeta anālābhe lābhe vāpi na harṣayet \veg\dontdisplaylinenum
\varr{
        \ \vb pādaṁ\lem  \msCb\msCc\msNa\msNc; pā\uncl{daṁ} \msCa, pāda \msNb\Ed\oo
                 vinikṣi°\lem  \msCb\msCc\msNa\msNb\Ed; {\lost}nikṣi° \msCa, vinikṣa° \msNc
        \ \vc kupyeta\lem  \msCa\msCb\msNa\msNb\msNc\Ed; kupeta \msCc\oo
                 anālābhe\lem  \msNa; manolābhe \msCa\msCb\msNb\msNc, manolābho \msCc, manālābhe \Ed}

arthat\textsubring{r}ṣṇāsv anudvigno roṣe vāpi sudāruṇe\thinspace{\dandab} \dontdisplaylinenum

stutinindā samaṁ k\textsubring{r}tvā priyaṁ vāpriyam eva vā \veg\dontdisplaylinenum
\varr{
        \ \va artha°\lem  \msCb\msCc\msNc; arthā° \msCa\msNa\msNb, atha \Ed\oo
                 °nudvigno\lem  \msCa\msCb\msNa\msNb\msNc\Ed; °nudigno \msCc}

niyamās tu parīdhānaṁ saṁyamāv\textsubring{r}tamekhalaḥ\thinspace{\dandab} \dontdisplaylinenum

nirālambaṁ manaḥ k\textsubring{r}tvā buddhiṁ k\textsubring{r}tvā nirañjanām \veg\dontdisplaylinenum
\varr{
        \ \va °dhānaṁ\lem  \msCa\msCb\msNa\msNb\Ed; °\uncl{dhānaṁ} \msNc, °dhānā \msCc
        \ \vb °v\textsubring{r}ta°\lem  \mssCaCbCc\msNa\msNc; °m\textsubring{r}ta° \msNb, °n\textsubring{r}ta° \Ed\oo
                 °mekhalaḥ\lem  \msCa\msCb\msNa\msNc\Ed; °mekhalāḥ \msCc, °mekhalā \msNb
        \ \vc °baṁ manaḥ k\textsubring{r}tvā\lem  \msNc; °bam asatk\textsubring{r}tvā \msCa\msNa,
                                °bam asaṁk\textsubring{r}tvā \msCb, °bam anaṁk\textsubring{r}tvā \msCc,
                                °ba manas k\textsubring{r}tvā \msNb, °bam anaṅk\textsubring{r}tvā \Ed
        \ \vd buddhiṁ\lem  \msCa\msCc\msNa\msNb\msNc; buddhi \msCb\Ed\oo
                 nirañjanām\lem  \eme; nirañjanam \mssCaCbCc\msNb\msNc\Ed, nirañjanaḥ \msNa}

ātmānaṁ p\textsubring{r}thivīṁ k\textsubring{r}tvā khaṁ ca k\textsubring{r}tvā manonmanam\thinspace{\dandab} \dontdisplaylinenum

tridaṇḍaṁ triguṇaṁ k\textsubring{r}tvā pātraṁ k\textsubring{r}tvākṣaro 'vyayaḥ \veg\dontdisplaylinenum
\varr{
        \ \vab k\textsubring{r}tvā khaṁ ca\lem  \msCb\msCc\msNa\msNb\msNc\Ed; k\textsubring{r}\uncl{tvā}{\lost}ñca \msCa
        \ \vb manonmanam\lem  \mssCaCbCc\msNa\msNb; manonmanaḥ \msNc, manonmanaiḥ \Ed
        \ \vd °kṣaro\lem  \mssCaCbCc\msNa\msNc\Ed; °karo \msNb\oo
                 vyayaḥ\lem  \msCa\msCb\msNa\msNb; vyayaṁ \msCc, vyaya \msNc, dvayam \Ed}

nyased dharmam adharmaṁ ca īrṣyādveṣaṁ parityajet\thinspace{\dandab} \dontdisplaylinenum

nirdvandvo nityasatyastho nirmamo nirahaṁk\textsubring{r}taḥ \veg\dontdisplaylinenum
                     \paral{\textit{\vcd {\normalfont cf. BhG 2.45cd: }nirdvandvo nityasatvastho niryogakṣema ātmavān}}
\varr{
        \ \va °dharmaṁ ca\lem  \mssCaCbCc\msNb\msNc\Ed; °dharmaṁ vā \msNa
        \ \vb īrṣyā°\lem  \msNa\msNc\Ed; īrṣā° \mssCaCbCc\msNb\oo
                 °dveṣaṁ\lem  \msCa\msCb\msNa\msNb\msNc\Ed; °dveṣa \msCc
        \ \vc nirdvandvo\lem  \msCa\msCb\msNa\msNb\msNc\Ed; nivaṁdvo \msCc\oo
                 °satya°\lem  \msCa\msCb\msNa\msNb\msNc\Ed; °saṁtya° \msCc
        \ \vd nirmamo\lem  \msNc\Ed; nirmāṁso \mssCaCbCc\msNa, nirmaṁso \msNb\oo
                 °k\textsubring{r}taḥ\lem  \mssCaCbCc\msNb\msNc; °k\textsubring{r}taṁ \msNa, °k\textsubring{r}tiḥ \Ed}

divasasyāṣṭame bhāge bhikṣāṁ saptag\textsubring{r}haṁ caret\thinspace{\dandab} \dontdisplaylinenum

na cāsīta na tiṣṭheta na ca dehīti vā vadet \veg\dontdisplaylinenum
\varr{
        \ \va divasasyā°\lem  \msCa\msCc\msNa\msNb\msNc\Ed; divasatyā° \msCb
        \ \vb bhikṣāṁ\lem  \mssCaCbCc\msNa\msNc\Ed; bhikṣā \msNb}

yathālābhena varteta aṣṭau piṇḍān dine dine\thinspace{\dandab} \dontdisplaylinenum

vastrabhojanaśayyāsu na prasajyeta vistaram \veg\dontdisplaylinenum
\varr{
        \ \va yathālābhena\lem  \msCb\msCc\msNa\msNb\msNc\Ed; yathālā{\lost}{\lost} \msCa
        \ \vb aṣṭau\lem  \mssCaCbCc\msNa\msNb\msNc; aṣṭa \Ed
        \ \vc °śayyāsu\lem  \mssCaCbCc\msNa\msNc; °śayyāñca \msNb, °śaiyyāsu \Ed
        \ \vd °sajyeta\lem  \msCa\msCc\msNa\msNb; °yujye \msCb, °saheta \msNc, °sahyeta \Ed\oo
                 vistaram\lem  \mssCaCbCc\msNa\msNb\msNc; vistaraḥ \Ed}

nābhinandeta maraṇaṁ nābhinandeta jīvitam\thinspace{\dandab} \dontdisplaylinenum
            \paral{\textit{\vab {\normalfont = MBh 12.237.15ab, Manu 6.45ab, Nāradaparivrājakopaniṣad 3.61cd. }}}

indriyāṇi vaśaṁk\textsubring{r}tvā kāmaṁ hatvā yatavrataḥ \veg\dontdisplaylinenum
\varr{
        \ \vc vaśaṁk\textsubring{r}°\lem  \msCa\msCb\msNa\msNb\msNc\Ed; vasaṁtk\textsubring{r}° \msCc
        \ \vd hatvā yatavrataḥ\lem  \mssCaCbCc\msNa\msNc\Ed; k\textsubring{r}tvā yataḥ vrataḥ \msNb}

atītaṁ ca bhaviṣyaṁ ca na bhikṣuś cintayet sadā\thinspace{\dandab} \dontdisplaylinenum

! krodhamānamadadarpān parivrāḍ varjayet sadā \veg\dontdisplaylinenum
\varr{
        \ \vb bhikṣuś ci°\lem  \mssCaCbCc\msNb\msNc; bhikṣuṁś ci° \msNa, bhikṣu ci° \Ed\oo 
                 sadā\lem  \msCa\msCc\msNa\msNb\msNc\Ed; \om\ \msCb
        \ \vcd °darpān pa°\lem  \msCa\msCc\msNa\msNb\msNc\Ed; °darpāt pa° \msCb}

virāgaṁ tu dhanuḥ k\textsubring{r}tvā prāṇāyāmaguṇair yutam\thinspace{\dandab} \dontdisplaylinenum

dhāraṇāśaratīkṣṇena m\textsubring{r}gaṁ hatvā manendriyam \veg\dontdisplaylinenum
\varr{
        \ \va dhanuḥ\lem  \mssCaCbCc\msNa\msNb\msNc; dhanuṣ \Ed
        \ \vb prāṇāyāmagu°\lem  \msCb\msCc\msNa\msNb\msNc\Ed; prāṇāyāmaṅgu° \msCa\oo
                 yutam\lem  \mssCaCbCc\msNb\msNc; yutaḥ \msNa, v\textsubring{r}taṁ \Ed
        \ \va °tīkṣṇena\lem  \msNb\Ed; °tīkṣṇeṇa \mssCaCbCc\msNc, °tīkṣeṇa \msNa}

maitrīkhaḍgasutīkṣṇena saṁsārāriṁ nik\textsubring{r}ntayet\thinspace{\dandab} \dontdisplaylinenum

karuṇāvartacakreṇa krodhamattagajaṁ jayet \veg\dontdisplaylinenum
\varr{
        \ \va sutīkṣṇena\lem  \msNb\msCa\msNc\Ed; sutīkṣṇeṇa \msCb\msCc\msNapcorr, ṇa \msNaacorr
        \ \vb °sārāriṁ\lem  \msCa\msCb\msNa\msNb\Ed; °sārāri \msCc\msNc}

muditāvarmabaddhāṅgas tūṇaṁ pūrṇam upekṣayā\thinspace{\dandab} \dontdisplaylinenum
            \paral{\textit{\vo {\normalfont Cf.\ 4.72: } caturāyatanaṁ vipra kathayiṣyāmi tac ch\textsubring{r}ṇu{\thinspace\danda}
                                 karuṇāmuditopekṣāmaitrī cāyatanaṁ sm\textsubring{r}tam{\thinspace\ketdanda}}}

anakṣaraṁ paraṁ brahma cintayet satataṁ dvija \veg\dontdisplaylinenum
\varr{
        \ \vb tūṇaṁ pūrṇam u°\lem  \emeGoodall; tūṇṇāpūrṇṇam u° \msCa,
                      tūṇāpūrṇṇam u° \msCb, tū\uncl{na}pūrṇṇam u° \msCc,
                      tūṇṇāpūṇṇām u° \msNa, tūrṇṇāpūrṇṇam u° \msNb\msNc, tūṇīpūrṇam u° \Ed 
        \ \vc anakṣaraṁ\lem  \msCb; anākṣaraṁ \msCa\msNa, anākṣara° \msCc\msNc\Ed, anakṣara° \msNb\oo
                 paraṁ\lem  \msCa\msCc\msNa\msNb\Ed; para \msCb\msNc}

brahmaṇo h\textsubring{r}dayaṁ viṣṇur viṣṇoś ca h\textsubring{r}dayaṁ śivaḥ\thinspace{\dandab} \dontdisplaylinenum

śivasya h\textsubring{r}dayaṁ saṁdhyā tasmāt saṁdhyām upāsayet \veg\dontdisplaylinenum
            \paral{\textit{\vo \kb\ {\normalfont  Saubhāgyabhāskara of Bhāskararāya ad Lalitāsahasranāmastotra 302: }
                brahmaṇo h\textsubring{r}dayaṁ viṣṇur viṣṇor api śivaḥ sm\textsubring{r}taḥ{\thinspace\danda}
                śivasya h\textsubring{r}dayaṁ sandhyā tenopāsyā dvijātibhiḥ{\thinspace\ketdanda}
                iti kaśyapādivacanaiḥ kaurmapādmaskāndādinikhilapurāṇeṣu ca tatra 
                tatra devīkālikābrahmāṇḍamārkaṇḍeyādipurāṇeṣu bahuśaḥ 
                śaktirahasyadevībhāgavatat\textsubring{r}tīyaskandhādiṣu{\thinspace\danda} }}
\varr{
        \ \va h\textsubring{r}dayaṁ\lem  \msCb\msCc\msNa\msNb\Ed; {\lost}dayaṁ \msCa, h\textsubring{r}daye \msNc
        \ \vab viṣṇur vi°\lem  \msCa\msNa\Ed; viṣṇum vi° \msCb, viṣṇu vi° \msCc\msNb\msNc
        \ \vb śivaḥ\lem  \Ed; śivaṁ \mssCaCbCc\msNa\msNb\msNc
        \ \vd °sayet\lem  \msCa\msCc\msNb; °śayet \msCb\msNa, °śrayet \msNc\Ed}

\ujvers\nemsloka 
saṁsārārṇavatāraṇaṁ śubhagatiḥ sa brahma saṁdhyākṣaraṁ
\dontdisplaylinenum

\nemslokab 
dhyāyen nityam atandrito hy anupamaṁ vyaktātmavedyaṁ śivam \danda\dontdisplaylinenum

\nemslokac 
rūpair varṇaguṇādibhiś ca vihitaṁ durlakṣyalakṣyottamaṁ
\dontdisplaylinenum

\nemslokad 
yatnoddh\textsubring{r}tya samāśrayet suraguruṁ sarvārtihartā haram \veg\dontdisplaylinenum

\vers


\jump
\begin{center}
\ketdanda iti v\textsubring{r}ṣasārasaṁgrahe caturāśramadharmavidhāno nāmādhyāya ekādaśamaḥ\ketdanda
\end{center}
\dontdisplaylinenum\vers 
\varr{
       \ \va °gatiḥ\lem  \msCc\Ed; °gati \msCa\msCb\msNa\msNb\ \unmetr, °gatiṁ \msNc\oo 
                °kṣaraṁ\lem  \msCa\msCc\msNa\msNb\msNc\Ed; °kṣara \msCb
        \ \vb °tandrito\lem  \msCa\msNa\msNc\Ed; °nandrito \msCb, °tandriya \msCc, °tandriyaṁ \msNb\oo
                 °vedyaṁ\lem  \mssCaCbCc\msNa\msNc\Ed; °vedya \msNb\ \unmetr
        \ \vc rūpair va°\lem  \msCa\msNa\msNc\Ed; rūpai va° \msCb\msCc\msNb\oo
                 vihitaṁ\lem  \mssCaCbCc\msNaacorr(?)\msNb\msNc; rahitaṁ \msNapcorr(?)\Ed\oo
                 durlakṣyalakṣyottamam\lem  \msCa\msNb; dulakṣyalakṣyottamam \msNa,
                                              durlakṣyalakṣottamam \msCb\msCc\msNc\Ed
        \ \vd yatnoddh\textsubring{r}tya\lem  \mssCaCbCc\msNa\msNb\msNc; yatnād dh\textsubring{r}tya \Ed\oo
                 samāśraye°\lem  \mssCaCbCc\msNa\msNc\Ed; maṇāśraye° \msNb\oo
                 sarvārtihartā haram\lem  \mssCaCbCc\msNb; sarvārttiha\uncl{rttā} haraṁ \msNa,
                                         sarvāttiharttā haraṁ \msNc,
                                         sarvārtihan śaṅkaram \Ed
\ {\normalfont \Colo: } nāmādhyāya ekādaśamaḥ\lem  \mssCaCbCc\msNa\msNb; nāmādhyāya ekādaśa \msNc, 
                                                                          nāma ekādaśo 'dhyāyaḥ \Ed}
\bekveg\szamveg\vfill\phpspagebreak\szam\bek\versno=0\fejno=12
\thispagestyle{empty}



\alfejezet{\textbf{dvādaśamo 'dhyāyaḥ}}\jump\jump 

\alalfejezet{ātithyadharmaḥ} 
\vers

devy uvāca~{\dandab}\dontdisplaylinenum 
\varr{
        \ \vo devy uvāca\lem  \msCa\msCb\msNa\Ed; \om\ \msBod}

ahiṁsā paramo dharmaḥ satataṁ parikīrtyate\thinspace{\danda} \dontdisplaylinenum

ātithyakānāṁ dharmaṁ ca kathayasva yad uttamam \veg\dontdisplaylinenum
\varr{
        \ \vc ātithya°\lem  \msCa\msNa\Ed; atithya° \msCb}

maheśvara uvāca~{\dandab}\dontdisplaylinenum 
\varr{
        \ \vo maheśvara\lem  \msCa\msCb\Ed; bhagavān \msNa}

ahiṁsātithyakānāṁ ca ś\textsubring{r}ṇu dharmaṁ yad uttamam\thinspace{\danda} \dontdisplaylinenum

trailokyam akhilaṁ devi ratnapūrṇaṁ sulocane \veg\dontdisplaylinenum
\varr{
        \ \vb ś\textsubring{r}ṇu\lem  \msCb\msNa\Ed; {\lost}ṇu \msCa\oo
                 dharmaṁ\lem  \msCa\msCb\msNa; dharma \Ed\oo
                 °ttamam\lem  \msCa\msCb\msNa; °ttamāṁ \Ed
        \ \vd °pūrṇaṁ\lem  \msCa\msCb\msNa; °pūrṇāṁ \Ed\oo
                 °locane\lem  \msCa\msNa\Ed; °locanaṁ \msCb}

caturvedavide dānaṁ na tattulyam ahiṁsakaḥ\thinspace{\dandab} \dontdisplaylinenum

ś\textsubring{r}ṇu dharmam atithyānāṁ kīrtayiṣyāmi sundari \veg\dontdisplaylinenum


\alalfejezet{vipulopākhyānam}\varr{
        \ \va dānaṁ\lem  \msCa\msNa\Ed; nānaṁ \msCb}

āsīd v\textsubring{r}ttaṁ purākhyānaṁ nagare kusumāhvaye\thinspace{\dandab} \dontdisplaylinenum

kapilasya suto vidvān vipulo nāma viśrutaḥ \veg\dontdisplaylinenum
\varr{
        \ \va āsīd v\textsubring{r}ttaṁ\lem  \msCa\msNa\Ed; āśīdattaṁ \msCb\oo 
                 °khyānaṁ\lem  \msCa\msCb\msNa; °khyātaṁ \Ed}

dharmanityo jitakrodhaḥ satyavādī jitendriyaḥ\thinspace{\dandab} \dontdisplaylinenum
            \paral{\textit{\vb {\normalfont  = MBh 12.218.13b }}}

brahmaṇyaś ca k\textsubring{r}tajñaś ca madbhaktaḥ k\textsubring{r}taniścayaḥ \veg\dontdisplaylinenum
\varr{
        \ \vc brahmaṇya°\lem  \msCb\msNa\Ed; brāhmaṇya° \msCa\oo
                 °jñaś ca\lem  \msCa\msNa\Ed; °jña \msCb
        \ \vd °bhaktaḥ\lem  \msCa\msCb\msNa; °bhakta° \Ed}

dhanāḍhyo 'tithipūjyaś ca dātā dānto dayālukaḥ\thinspace{\dandab} \dontdisplaylinenum

nyāyārjitadhano nityam anyāyaparivarjitaḥ \veg\dontdisplaylinenum
\varr{
        \ \va °pūjyaś ca\lem  \msCa\msNapcorr\Ed; °pūjya \msCb\msNaacorr
        \ \vb dānto\lem  \msCbacorr\Ed; dānta \msCa\msNa, dāntom{\normalfont (?)} \msCbpcorr
        \ \vc nyāyā°\lem  \msNa\Ed; nyāyo° \msCa\msCb}

bhāryā ca rūpiṇī tasya candrabimbaśubhānanā\thinspace{\dandab} \dontdisplaylinenum

pīnottuṅgastanī kāntā sakalānandakāriṇī \danda\dontdisplaylinenum

pativratā patiratā patiśuśrūṣaṇe ratā \veg\dontdisplaylinenum
\varr{
        \ \vb °bimba°\lem  \msCa\msCb\Ed; °biṁ\uncl{bā} \msNa
        \ \vd sakalā°\lem  \msCb\msNa\Ed; {\lost}{\lost}{\lost} \msCa
        \ \ve pativratā\lem  \msCa\msNa\Ed; prativratā \msCb}

atha kenāpi kālena sūryarāgam abhūt tataḥ\thinspace{\dandab} \dontdisplaylinenum

grastabhāgatrayas tv āsīt k\textsubring{r}ṣṇamādhavamāsike \veg\dontdisplaylinenum

snātukāmāvatīryante sarve pauran\textsubring{r}pādayaḥ\thinspace{\dandab} \dontdisplaylinenum

devāś ca pitaraś caiva tarpyante vidhivat tathā \veg\dontdisplaylinenum
\varr{
        \ \va °vatīryante\lem  \msCa\msCb\msNa; ca tīrthante \Ed
        \ \vd tarpyante\lem  \msCa\msNa\Ed; tapyante \msCb}

kecij juhvati tatrāgniṁ kecid viprāṁś ca tarpayet\thinspace{\dandab} \dontdisplaylinenum

kecid dānopatiṣṭhanti kecit stunvanti devatām \veg\dontdisplaylinenum
\varr{
        \ \va °cij juhvati\lem  \msCa\msNa\Ed; °cij juti \msCb
        \ \vb viprāṁś ca\lem  \msCa\msNa\Ed; viprāś ca \msCb
        \ \vc dāno°\lem  \msCa\msCb\msNa; dhyāno° \Ed
        \ \vd kecit stuvanti\lem  \msCa\msCb; keci stuvanti \msNa, kecit stunvanti \Ed\oo
                 devatām\lem  \msCa\msNa\Ed; devatā \msCb}

dhyānayogaratāḥ kecit kecit pañcatape ratāḥ\thinspace{\dandab} \dontdisplaylinenum

evaṁ pravartamāneṣu rājanādiṣu sarvaśaḥ \veg\dontdisplaylinenum
\varr{
        \ \vd rājanā°\lem  \msCa\msCb\msNa; rājānā° \Ed}

vipulo 'pi hi tatraiva gaṅgāgaṇḍakisaṁgame\thinspace{\dandab} \dontdisplaylinenum

bhāryayā saha tatraiva snātvā kṣomavibhūṣaṇaḥ \veg\dontdisplaylinenum
\varr{
        \ \va 'pi hi\lem  \msCa\msNapcorr; pi \msCb, hi na \msNaacorr, pi ca \Ed
        \ \vc bhāryayā\lem  \msCapcorr\msCb\msNa; bhāryāyā \msCaacorr\Ed
        \ \vd °bhūṣaṇaḥ\lem  \msCa\msCb\Ed; °bhūṣitaḥ \msNa}

devatāguruviprāṇām anyeṣāṁ tarpaṇe rataḥ\thinspace{\dandab} \dontdisplaylinenum

tatrāvasarasamprāpto brāhmaṇo 'tithir āgataḥ \veg\dontdisplaylinenum
\varr{
        \ \vab devatāguruviprāṇām anyeṣāṁ tarpaṇe rataḥ\lem  \msCb\msNapcorr;
                                 devatāguruvi{\lost}ṇām anyeṣāṁ tarpaṇe rataḥ \msCa,
                                 \om\ \msNaacorr,
                                 devatāguruviprāṇām anyeṣāṁ tarpaṇā rataḥ \Ed}

bhāryā tasyātirūpeṇa mohitā brahmaṇas tadā\thinspace{\dandab} \dontdisplaylinenum

brāhmaṇo 'pi tathaiveha rūpeṇāpratimo bhavet \veg\dontdisplaylinenum
\varr{
        \ \vb mohitā\lem  \msCa\msNa\Ed; mohito \msCb\oo
                 brahmaṇas tadā\lem  \msCa\msCb; brāhmaṇas tadā \msNa, brāhmaṇasya ca \Ed
        \ \vc brāhmaṇo\lem  \msCa\msNa\Ed; brahmaṇo \msCb\oo
                 tathaiveha\lem  \msCb\msNa\Ed; ta\uncl{the}veha \msCa
        \ \vd rūpeṇā°\lem  \msCa\msNa; rūpenā° \msCb, rūpiṇā° \Ed}

anyonyad\textsubring{r}ṣṭisaṁsaktau jātau tau tu parasparam\thinspace{\dandab} \dontdisplaylinenum

vipulenāñjaliṁ k\textsubring{r}tvā brāhmaṇa saṁśitavrata \veg\dontdisplaylinenum
            \paral{\textit{\vd {\normalfont  = MBh 12.213.18d and 12.347.1d }}}
\varr{
        \ \va °saṁsaktau\lem  \msCc\msNa\Ed; °saṁśaktau \msCa, °śaktau \msCb
        \ \vd brāhmaṇa\lem  \msCb; brāhmaṇaḥ \msCa\msNa\Ed\oo 
                 °śita°\lem  \eme; °śrita° \msCa\msCb\msNa\Ed\oo
                 °vrata\lem  \conj; °vra{\il} \msCa, °vrataḥ \msCb\msNa\Ed}

ājñāpaya dvijaśreṣṭha adya me 'nugrahaṁ kuru\thinspace{\dandab} \dontdisplaylinenum

bhāryābh\textsubring{r}tyapaśugrāma ratnāni vividhāni ca \veg\dontdisplaylinenum
\varr{
        \ \vb °grahaṁ\lem  \msCa\Ed; °graha \msCb}

vipulenaivam uktas tu g\textsubring{r}hīto brāhmaṇo 'bravīt\thinspace{\dandab} \dontdisplaylinenum

yadi satyaṁ pradātāsi suprasannaṁ manas tava \veg\dontdisplaylinenum
\varr{
        \ \vf °sannaṁ\lem  \msCa\msCb; °sanna° \Ed}

vipula uvāca~{\dandab}\dontdisplaylinenum 

suprasannaṁ mano me 'dya suprasannaṁ tapaḥphalam\thinspace{\danda} \dontdisplaylinenum

śīghram ājñāpaya vipra yac cābhilaṣitaṁ tava \danda\dontdisplaylinenum

adeyaṁ nāsti viprasya svaśiraḥprabh\textsubring{r}ti dvija \veg\dontdisplaylinenum
\varr{
        \ \vf °bh\textsubring{r}ti\lem  \msCa\msCb; °bh\textsubring{r}tir \Ed}

brāhmaṇa uvāca~{\dandab}\dontdisplaylinenum 
\varr{
        \ \vo brāhmaṇa\lem  \msCapcorr\msCb\Ed; brāhmaṇā \msCaacorr}

yady evaṁ vadase bhadra bhāryāṁ me dehi rūpiṇīm\thinspace{\danda} \dontdisplaylinenum

svasti bhavatu bhadraṁ vaḥ kalyāṇaṁ bhava śāśvatam \veg\dontdisplaylinenum
\varr{
        \ \vc svasti\lem  \msCa\msCb; svastir \Ed
        \ \vd bhava\lem  \msCa\msCb; tava \Ed}

vipula uvāca~{\dandab}\dontdisplaylinenum 
\varr{
        \ \vo vipula\lem  \msCa\msCb; vipra \Ed}

pratīccha bhāryāṁ suśroṇīṁ rūpayauvanaśālinīm\thinspace{\danda} \dontdisplaylinenum

akutsitāṁ viśālākṣīṁ pūrṇacandranibhānanām \veg\dontdisplaylinenum

bhāryovāca~{\dandab}\dontdisplaylinenum 

parityājyā kathaṁ nātha apāpāṁ tyajase katham\thinspace{\danda} \dontdisplaylinenum

atīva hi priyāṁ bhāryāṁ nirdoṣāṁ sa kathaṁ tyajeḥ \veg\dontdisplaylinenum
\varr{
        \ \va °tyājyā\lem  \msCa\Ed; °tyājya \msCb
        \ \vd tyajeḥ\lem  \msCa; tyajyet \msCb, tyajet \Ed}

sakhā bhāryā manuṣyāṇām iha loke paratra ca\thinspace{\dandab} \dontdisplaylinenum

dānaṁ vā sumahad dattvā yajño vā subahuḥ k\textsubring{r}taḥ \veg\dontdisplaylinenum
\varr{
        \ \vd °bahuḥ\lem  \eme; °bahu  \unmetr \msCa\msCb, °bahūn \Ed}

aputro nāpnuyāt svargaṁ tapobhir vā suduṣkaraiḥ\thinspace{\dandab} \dontdisplaylinenum

śruto me pit\textsubring{r}bhiḥ prokto brāhmaṇaiś ca mamāntike \veg\dontdisplaylinenum
\varr{
        \ \vab svargaṁ tapobhir vā\lem  \msCb\Ed; sva\uncl{rggan} {\lost}{\lost}{\lost}r vvā \msCa
        \ \vd °ntike\lem  \msCa\Ed; °ntikaiḥ \msCb}

aputro nāpnuyāt svargaṁ śrutaṁ me bahuśaḥ purā\thinspace{\dandab} \dontdisplaylinenum

mandapālo dvijaśreṣṭho gataḥ svargaṁ tapobalāt \veg\dontdisplaylinenum
            \paral{\textit{\vc {\normalfont See Mandapāla's story in MBh 1.220.5ff.}}}
\varr{
        \ \va svargaṁ\lem  \msCa\Ed; svarga \msCb
        \ \vc °pālo\lem  \Ed; °pāla \msCa\msCb}

dānāni ca bahūn dattvā yajñāṁś ca vividhāṁs tathā\thinspace{\dandab} \dontdisplaylinenum

vedāṁś ca japa yajñāṁś ca k\textsubring{r}tvā tad dvijasattamaḥ \veg\dontdisplaylinenum
\varr{
        \ \vb yajñāṁś ca vividhāṁs\lem  \msCa; yatvā yajñāṁś ca vividhāṁ \msCb, syajñāś ca vividhās \Ed
        \ \vc vedāṁś ca japayajñāṁś ca\lem  \msCa; vedāś ca japayajñāṁś ca \msCb,
                                                 vedāś ca japayajñāś ca \Ed}

prāptadvāro 'pi yasyāpi devadūtair nivāritaḥ\thinspace{\dandab} \dontdisplaylinenum

aputro nāpnuyāt svargaṁ yadi yajñaśatair api \veg\dontdisplaylinenum
\varr{
        \ \va yasyāpi\lem  \msCa\msCb; yasyāhi \Ed}

ity uktas tu cyutaḥ svargān mandapālo mahān \textsubring{r}ṣiḥ\thinspace{\dandab} \dontdisplaylinenum

putrān utpādayām āsa śāraṅgāś caturo dvijaḥ \veg\dontdisplaylinenum
\varr{
        \ \vd śāraṅgāś ca\lem  \msCa; śāraṅgaṁś ca \msCb, śāraṅgāc ca \Ed}

tena puṇyaprabhāveṇa svargaṁ prāpto hy avāritaḥ\thinspace{\dandab} \dontdisplaylinenum

kulatrāṇāt kalatrāsmi bharaṇād bhārya eva ca \veg\dontdisplaylinenum
\varr{
        \ \vc kula°\lem  \msCb; kala° \msCa\Ed\oo
                 °trāṇāt ka°\lem  \eme; °trāṇāṁ ka° \msCa\msCb\Ed
        \ \vd bhārya eva\lem  \msCa\Ed; bhāryam eva \msCb}

dārasaṁgraha putrārthe kriyate śāstradarśanāt\thinspace{\dandab} \dontdisplaylinenum

yāni santi g\textsubring{r}he dravyaṁ grāmaghoṣag\textsubring{r}hāṇi ca \veg\dontdisplaylinenum
\varr{
        \ \va °graha\lem  \Ed; °grahaḥ \msCa\msCb\oo
                 putrā°\lem  \msCa\msCb; pātrā° \Ed
        \ \vb kriyate\lem  \msCa\Ed; kriyāte \msCb}

dātum arhasi viprāya na māṁ dātum ihārhasi\thinspace{\dandab} \dontdisplaylinenum

bhāryāyā vacanaṁ śrutvā vipulaḥ punar abravīt \veg\dontdisplaylinenum

vipula uvāca~{\dandab}\dontdisplaylinenum 
\varr{
        \ \vo vipula uvāca\lem  \Ed; \om\ \msCa\msCb}

sādhu bhāmini jānāmi sādhu sādhu pativrate\thinspace{\danda} \dontdisplaylinenum

jito 'smy anena vākyena anenāsmi hi toṣitaḥ \veg\dontdisplaylinenum
\varr{
        \ \va jānāmi\lem  \msCb\Ed; jānāsi \msCa}

adya grahaṇakāle ca dvija āgatya yācate\thinspace{\dandab} \dontdisplaylinenum

dadāmīti pratijñāya adattvā narakaṁ vraje \veg\dontdisplaylinenum
\varr{
        \ \vd vraje\lem  \msCa; vrajet \msCb\Ed}

narakaṁ yadi gacchāmi kulena saha sundari\thinspace{\dandab} \dontdisplaylinenum

kalpakoṭisahasre 'pi narakasthād yaśasvini \veg\dontdisplaylinenum
\varr{
        \ \vc °sahasre 'pi\lem  \msCa\msCb; °sahasrāṇi \Ed
        \ \vd °sthād\lem  \msCa; sthā \msCb, °stho \Ed}

muktim eva na paśyāmi janmakoṭiśatair api\thinspace{\dandab} \dontdisplaylinenum

adānāc cāśubhaṁ devi paśyāmi varavarṇini \veg\dontdisplaylinenum
\varr{
        \ \va muktim eva\lem  \msCa\msCb; muktim evan \Ed}

dānena tu śubhaṁ paśye svargaloke yad akṣayam\thinspace{\dandab} \dontdisplaylinenum

noktaṁ mayān\textsubring{r}taṁ pūrvaṁ nityaṁ satyavrate sthitaḥ \veg\dontdisplaylinenum
\varr{
        \ \vb °loke\lem  \msCa\msCb; °lokaṁ \Ed
        \ \vd °vrate\lem  \msCa\msCb; °vrata° \Ed}

satyadharmam atikramya nānyadharmaṁ samācare\thinspace{\dandab} \dontdisplaylinenum

bhāryā dharmasakhety evaṁ tvayi pūrvam udāh\textsubring{r}tam \veg\dontdisplaylinenum
\varr{
        \ \vb °care\lem  \msCa\msCb; °caret \Ed}

yadi dharmasakhāyāsi so 'dya kāla ihāgataḥ\thinspace{\dandab} \dontdisplaylinenum

dvijarūpadharo dharmaḥ svayam eva ihāgataḥ \veg\dontdisplaylinenum
\varr{
        \ \va °sakhāyā°\lem  \msCa\Ed; °sakhā° \msCb
        \ \vc °dharo\lem  \msCa\Ed; °paro \msCb}

jijñāsārtham ahaṁ bhadre na vighnaṁ kartum arhasi\thinspace{\dandab} \dontdisplaylinenum

mātāvyaktaḥ pitā brahmā buddhir bhāryā damaḥ sakhā \veg\dontdisplaylinenum
\varr{
        \ \vd buddhir\lem  \msCa\msCb; buddhi \Ed\oo
                 sakhā\lem  \msCb\Ed; samā \msCa}

putro dharmaḥ kriyācārya ity ete mama bāndhavāḥ\thinspace{\dandab} \dontdisplaylinenum

kālaśreṣṭho grahaḥ sūryo gaṅgā śreṣṭhā nadīṣu ca \veg\dontdisplaylinenum
            \paral{\textit{\vc {\normalfont cf.\ e.g.\ Āgamakalpalatā 3.128:}
                sūryagrahaṇakālasya samānā nāsti bhūtale {\thinspace\danda}
                atra yad yat k\textsubring{r}taṁ karma anantaphaladaṁ bhavet {\thinspace\ketdanda}
                {\normalfont cf.\ also Agastyasaṁhitā X.XXcd (on the proper date for initiation):}
                sūryagrahaṇakālena samāno nāsti kaścana  
                {\normalfont also ibid. X.XX (on image installation): }
                sūryagrahe mahāpuṇye kurukṣetre vidhānataḥ {\thinspace\danda}
                k\textsubring{r}tair yat puṇyam āpnoti tulāpuruṣakādibhiḥ {\thinspace\ketdanda}
                tatpuṇyaṁ prāpnuyāmartyaḥ {\normalfont ...}}}
           \paral{\textit{\vd {\normalfont \kb\ 15.18b: } śreṣṭhā gaṅgā nadīṣu ca}}
\varr{
        \ \vc °śreṣtho\lem  \msCb; °śreṣṭha° \msCa, °śreṣṭhaḥ \Ed}

candrakṣaye dinaṁ śreṣṭhaṁ naraśreṣṭho dvijottamaḥ\thinspace{\dandab} \dontdisplaylinenum

śuśrūṣaṇārthaṁ viprasya mayā dattāsi sundari \danda\dontdisplaylinenum

sarvasvaṁ brāhmaṇe dattvā vanam evāśrayāmy aham \veg\dontdisplaylinenum
\varr{
        \ \va dinaṁ\lem  \msCa\msCb; dina° \Ed
        \ \va °rthaṁ\lem  \msCa\Ed; °rtha \msCb}

śaṅkara uvāca~{\dandab}\dontdisplaylinenum 
\varr{
        \ \vo śaṅkara\lem  \msCa\msCb; maheśvara \Ed}

tūṣṇīmbhūtā tato bhāryā aśrupūrṇākulekṣaṇā\thinspace{\danda} \dontdisplaylinenum

kare g\textsubring{r}hya viśālākṣī brāhmaṇāya niveditā \veg\dontdisplaylinenum
\varr{
        \ \va tūṣṇīmbhūtā\lem  \msCa; tūṣṇībhūtvā \msCb, tūṣṇībhūtāṁ \Ed\oo
                 bhāryā\lem  \msCa\msCb; bhāryāṁ \Ed
        \ \vb °kṣaṇā\lem  \msCa\msCb; °kṣaṇām \Ed
        \ \vc °kṣī\lem  \msCa\msCb; °kṣīṁ \Ed
        \ \vd brāhmaṇāya niveditā\lem  \msCa\Ed; brāhmaya diveditā \msCb}

yāni santi g\textsubring{r}he dravyaṁ hiraṇyaṁ paśavas tathā\thinspace{\dandab} \dontdisplaylinenum

dadāmi te dvijaśreṣṭha grāmaghoṣag\textsubring{r}hādikam \veg\dontdisplaylinenum
\varr{
        \ \vb hiraṇyaṁ\lem  \msCa\msCb; hiraṇya° \Ed
        \ \vc dadāmi\lem  \msCa\Ed; dadāni \msCb\oo
                 te dvija°\lem  \msCb\Ed; {\lost}{\lost}ja° \msCa}

muktā vaiḍūryavāsāṁsi divyāṇy ābharaṇāni ca\thinspace{\dandab} \dontdisplaylinenum

sarvān g\textsubring{r}hāṇa viprendra śraddhayā dattasatk\textsubring{r}tām \veg\dontdisplaylinenum

prīyatāṁ bhagavān dharmaḥ prīyatāṁ ca maheśvaraḥ\thinspace{\dandab} \dontdisplaylinenum

prīyantāṁ pitaraḥ sarve yady asti suk\textsubring{r}taṁ phalam \veg\dontdisplaylinenum
\varr{
        \ \vc prīyantāṁ\lem  \msCa; prīyatāṁ \msCb\Ed
        \ \vd asti\lem  \msCb\Ed; asi \msCa}

rudra uvāca~{\dandab}\dontdisplaylinenum 
\varr{
        \ \vo rudra\lem  \msCa\msCb; maheśvara \Ed}

vipulasya vacaḥ śrutvā brāhmaṇena tapasvinā\thinspace{\danda} \dontdisplaylinenum

āśīḥ suvipulaṁ dattvā vipulāya mahātmane \veg\dontdisplaylinenum
\varr{
        \ \va vacaḥ śrutvā\lem  \msCb\Ed; vaca\uncl{ś śru}{\lost} \msCa}

vaset tatra g\textsubring{r}he ramye bhāryām ādāya tasya ca\thinspace{\dandab} \dontdisplaylinenum

vipulas tu namask\textsubring{r}tvā k\textsubring{r}tvā cāpi pradakṣiṇam \veg\dontdisplaylinenum
\varr{
        \ \va vaset tatra g\textsubring{r}he\lem  \msCb\msNa; vasa tatra g\textsubring{r}he \msCa, vasate ca g\textsubring{r}haṁ \Ed
        \ \vd cāpi\lem  \msCa\msCb; ca vi° \Ed}

brāhmaṇam abhivādyaivaṁ gataḥ śīghraṁ vanāntaram\thinspace{\dandab} \dontdisplaylinenum

vane mūlaphalāhāro vicareta mahītale \veg\dontdisplaylinenum
\varr{
        \ \va °dyaivaṁ\lem  \eme; °dyevaṁ \msCa\Ed, °dyenaṁ \msCb}

ekākī vijane śūnye cintayā ca pariplutaḥ\thinspace{\dandab} \dontdisplaylinenum

kva gacchāmi kva bhokṣyāmi kutra vā kiṁ karomy aham \veg\dontdisplaylinenum
\varr{
        \ \va ekākī\lem  \msCb\Ed; e\uncl{kā}{\lost} \msCa
        \ \vc kva bhokṣyāmi\lem  \msCa; kva bhojyāmi \msCb, kiṁ bhokṣyāmi \Ed\ \unmetr}

na pathaṁ viṣayaṁ vedmi grāmaṁ vā nagarāṇi vā\thinspace{\dandab} \dontdisplaylinenum

kheṭakharvaṭadeśaṁ vā jānāmīha na kaṁcana \veg\dontdisplaylinenum
\varr{
        \ \vc °kharvaṭa°\lem  \Ed; °karppaṭa° \msCa\msCb
        \ \vd kaṁcana\lem  \eme; kaścana \msCa\msCb\Ed}

amuṁ suśailaṁ paśyāmi vipulodarakandaram\thinspace{\dandab} \dontdisplaylinenum

tam āruhya nirīkṣyāmi grāmaṁ nagarapattanam \veg\dontdisplaylinenum

evam uktvā tu vipulaḥ śanaiḥ parvatam āruhat\thinspace{\dandab} \dontdisplaylinenum

v\textsubring{r}kṣacchāyāṁ samālokya niṣasāda śramānvitaḥ \veg\dontdisplaylinenum
\varr{
        \ \va evam u°\lem  \msCa\Ed; ekaṁ u° \msCb
        \ \vb °ruhat\lem  \Ed; °ruhet \msCa\msCb}

etasminn eva kāle tu v\textsubring{r}kṣaśākhāvatārya ca\thinspace{\dandab} \dontdisplaylinenum

apūrvaṁ ca surūpaṁ ca sugandhatvaṁ ca śobhanam \veg\dontdisplaylinenum
\varr{
        \ \va kāle tu\lem  \msCa\msCb; kālena \Ed
        \ \vc surūpaṁ\lem  \msCa\Ed; svarūpaṁ \msCb}

phalaṁ g\textsubring{r}hya vicitraṁ ca h\textsubring{r}dayānandanaṁ śubham\thinspace{\dandab} \dontdisplaylinenum

vipulasyāgrataḥ k\textsubring{r}tvā punar v\textsubring{r}kṣaṁ samāruhat \veg\dontdisplaylinenum

vipulaś citravad d\textsubring{r}ṣṭvā vismayaṁ paramaṁ gataḥ\thinspace{\dandab} \dontdisplaylinenum

aho vā svapnabhūto 'smi aho vā tapasaḥ phalam \veg\dontdisplaylinenum

na paśyāmi na jighrāmi na ca svādaṁ ca vedmy aham\thinspace{\dandab} \dontdisplaylinenum

vārttāpi na ca me śrotrā pratijānāmi kaṁcana \veg\dontdisplaylinenum
\varr{
        \ \va jighrāmi\lem  \msCa\Ed; ca ghrāmi \msCb
        \ \vc śrotrā\lem  \msCa; śrotā \msCb\Ed
        \ \vd kaṁcana\lem  \eme; kaścana \msCa\msCb\Ed}

evam uktvā hy anekāni phalaṁ g\textsubring{r}hya manoramam\thinspace{\dandab} \dontdisplaylinenum

sunirīkṣya punar jighraṁ punar jighraṁ nirīkṣya ca \veg\dontdisplaylinenum

phalaṁ cātra nirūpyanto deśaṁ vāpy avalokayan\thinspace{\dandab} \dontdisplaylinenum

pātheyarahitaś cāsmi devadattaṁ phalaṁ mama \veg\dontdisplaylinenum
\varr{
        \ \va cātra\lem  \msCb\Ed; cā \msCaacorr, cā\uncl{tra} \msCapcorr\oo
                 nirūpyanto\lem  \Ed; nirūpyānti \msCa, nirūpyāṁ cā \msCb
        \ \vb °lokayan\lem  \msCa\Ed; °lokayat \msCb
        \ \vd °dattaṁ\lem  \msCa; °datta° \msCb\Ed}

tat phalaṁ pratig\textsubring{r}hyaiva nagaraṁ praviśāmy aham\thinspace{\dandab} \dontdisplaylinenum

prārthayitvā ca yat kiṁcij jīvanārthaṁ carāmy aham \veg\dontdisplaylinenum
\varr{
        \ \va °g\textsubring{r}hyaiva\lem  \msCb\Ed; °g\textsubring{r}hyeva \msCa}

tataḥ śailam atikramya nagaraṁ praviveśa ha\thinspace{\dandab} \dontdisplaylinenum

pathi kaścij janaḥ p\textsubring{r}ṣṭhaḥ kiṁnāma nagaraṁ tv idam \veg\dontdisplaylinenum
\varr{
        \ \vd nagaraṁ\lem  \msCa\Ed; nagara \msCb}

sa hovāca pathī kena kim apūrvam ihāgataḥ\thinspace{\dandab} \dontdisplaylinenum

dakṣiṇāpathadeśo 'yaṁ naravīrapuraṁ tv adaḥ \veg\dontdisplaylinenum
\varr{
        \ \va sa ho°\lem  \msCa\Ed; aho° \msCb
        \ \vc °patha°\lem  \msCa\Ed; °pathe \msCb
        \ \vd tv adaḥ\lem  \msCb; tvayaḥ \msCa, svayam \Ed}

rājā siṁhajaṭo nāma rājñī tasya ca kekayī\thinspace{\dandab} \dontdisplaylinenum

ativ\textsubring{r}ddho jarāgrastaḥ kekayī ca tathaiva ca \veg\dontdisplaylinenum
\varr{
        \ \va rājā\lem  \msCa\msCb; rāja \Ed\oo
                 °jaṭo\lem  \msCa\msCb; °yato \Ed
        \ \vb kekayī\lem  \msCb\Ed; kaikayī \msCa
        \ \vd kekayī\lem  \msCb\Ed; kaikayī \msCa}

dātā sarvakalājñaś ca yuddhe vīryabalānvitaḥ\thinspace{\dandab} \dontdisplaylinenum

brahmaṇyo vatsalo loke sarvaśāstraviśāradaḥ \veg\dontdisplaylinenum
\varr{
        \ \va dātā\lem  \msCb\Ed; {\lost}tā \msCa\oo
                 °kalā°\lem  \Ed; °kala° \msCa\msCb}

vipula uvāca~{\dandab}\dontdisplaylinenum 

atra śreṣṭhim upāsyāmi nāma vā tasya kiṁ vada\thinspace{\danda} \dontdisplaylinenum

katamo deśas tadvāsaḥ kathayasva na saṁśayaḥ \veg\dontdisplaylinenum
\varr{
        \ \vb nāma\lem  \msCa\msCb; nāmaṁ \Ed\oo
                 vada\lem  \msCa\Ed; vadaḥ \msCb
        \ \vd kathayasva\lem  \msCa\Ed; kathayasya \msCb}

vipulenaivam uktas tu pathikovāca taṁ punaḥ\thinspace{\dandab} \dontdisplaylinenum

mama bhīmabalo nāma śreṣṭhikasya g\textsubring{r}hāgataḥ \veg\dontdisplaylinenum
\varr{
        \ \vc mama bhīmabalo nāma\lem  \msCb\msCc\msNa; mama bhī{\lost}balo nāma \msCa, \om\ \Ed}

śreṣṭhikaḥ puṇḍako nāma khyātaḥ śreṣṭhika ucyate\thinspace{\dandab} \dontdisplaylinenum

kautukaṁ tava yady asti tad āgaccha mayā saha \veg\dontdisplaylinenum

evam astv iti tenokto vipulena mahātmanā\thinspace{\dandab} \dontdisplaylinenum

tenaiva saha niryātaḥ śreṣṭhikasya g\textsubring{r}haṁ prati \veg\dontdisplaylinenum
\varr{
        \ \vc °stv iti\lem  \msCa\msNa\Ed; °stiti \msCb\msCc\oo
                 °kto\lem  \mssCaCbCc\msNa; °ktau \Ed
        \ \vb prati\lem  \msCa\msCb\msNa; pratiḥ \msCc\Ed}

śreṣṭhikaḥ svag\textsubring{r}hāsīno d\textsubring{r}ṣṭaḥ sa vipulena tu\thinspace{\dandab} \dontdisplaylinenum

tasyāntikam upāgamya tat phalaṁ sa niveditaḥ \veg\dontdisplaylinenum
\varr{
        \ \vc śreṣṭhikaḥ\lem  \msCb\msCc\Ed; śreṣṭhitaḥ \msCa, śreṣṭhikaḥ \msNa
        \ \vd d\textsubring{r}ṣṭaḥ sa\lem  \msCb\Ed; \uncl{d\textsubring{r}}{\lost}{\lost} \msCa, d\textsubring{r}ṣṭa sa \msCc}

aho phalam idaṁ śreṣṭham aho phalam ihānitam\thinspace{\dandab} \dontdisplaylinenum

aho rūpam aho gandham aho phalaṁ suśobhanam \veg\dontdisplaylinenum
\varr{
        \ \vc gandham\lem  \msCa\msCbpcorr\Ed; gandham aho gandham \msCbacorr
        \ \vd phalaṁ\lem  \corr; phala \msCa\msCb\Ed}

tat phalaṁ na mahījātaṁ na merau na ca kandare\thinspace{\dandab} \dontdisplaylinenum

devalokika suvyaktaṁ na martya upajāyate \veg\dontdisplaylinenum
\varr{
        \ \va tat pha°\lem  \msCa\msCb; yat pha° \Ed
        \ \vd martya upajāyate\lem  \eme;
                      martya\uncl{mupajā}{\lost}{\lost} \msCa, martya supajāyate \msCb, mahyām upajāyate \Ed}

aho 'smi saphalaṁ bhoktā rājārhaś ca na saṁśayaḥ\thinspace{\dandab} \dontdisplaylinenum

ḍhaukayitvā phalaṁ divyaṁ rājānaṁ toṣayāmy aham \veg\dontdisplaylinenum
\varr{
        \ \va aho\lem  \msCb; {\lost}ho \msCa, adyo \Ed\oo
                 saphalaṁ\lem  \msCb; \uncl{sa}phalam \msCa, tat phalaṁ \Ed}

tatas tvarita gatvaiva phalaṁ g\textsubring{r}hya manoharam\thinspace{\dandab} \dontdisplaylinenum

ādareṇopas\textsubring{r}tyaiva rājānaṁ sa phalaṁ dadau \veg\dontdisplaylinenum
\varr{
        \ \va tvarita\lem  \Ed; tvaritaṁ \msCa\msCb\ \unmetr
        \ \vb g\textsubring{r}hya\lem  \msCa\Ed; g\textsubring{r}ha \msCb\oo
                 °haram\lem  \msCa\msCb; °ramam \Ed
        \ \vd sa phalaṁ\lem  \msCa\msCb; tat phalaṁ \Ed}

rājā ca sa phalaṁ d\textsubring{r}ṣṭvā vismayaṁ paramaṁ gataḥ\thinspace{\dandab} \dontdisplaylinenum

kutaḥ śreṣṭhi tvayā nītaṁ phalaṁ sarvamanoharam \veg\dontdisplaylinenum
\varr{
        \ \va sa phalaṁ\lem  \msCa\msCb; tat phalaṁ \Ed
        \ \vc śreṣṭhi\lem  \msCa\msCb; śreṣṭha \Ed
        \ \vd phalaṁ sarvamanoharam\lem  \Ed; phala{\lost}{\lost}{\lost}{\lost}haram \msCa, phala\uncl{m ya}rvamanoharam \msCb}

svādumūlaphalakandaṁ d\textsubring{r}ṣṭvā pūrvaṁ na tād\textsubring{r}śam\thinspace{\dandab} \dontdisplaylinenum

rūpagandhaguṇopetaṁ h\textsubring{r}dayānandakārakam \veg\dontdisplaylinenum
\varr{
        \ \va °kandaṁ d\textsubring{r}ṣṭvā\lem  \msCa; °skanda d\textsubring{r}ṣṭvā \msCb,°skanda d\textsubring{r}ṣṭā \Ed
        \ \vb tād\textsubring{r}śam\lem  \msCa\msCb; yād\textsubring{r}śam \Ed}

sadya evopabhuñjāmi tvayā dattam idaṁ phalam\thinspace{\dandab} \dontdisplaylinenum

kīd\textsubring{r}śaṁ svāda vijñātum icchāmi kuru māciram \veg\dontdisplaylinenum
\varr{
        \ \va sadya evopayuñjāmi\lem  \msCa\msCb; satya eva prabhuñjāmi \Ed
        \ \vc svādavijñānam\lem  \msCa\msCb; svādu vijñātum \Ed}

tataḥ sa bhakṣayām āsa phalaṁ cām\textsubring{r}tasaṁnibham\thinspace{\dandab} \dontdisplaylinenum

am\textsubring{r}topamasusvādaṁ sarvaṁ ca bubhuje n\textsubring{r}paḥ \veg\dontdisplaylinenum
\varr{
        \ \va tataḥ\lem  \msCa\Ed; tata \msCb
        \ \vcd svādaṁ sarvaṁ ca\lem  \msCb\Ed; svā{\lost}{\lost}{\lost}{\lost} \msCa}

sadya ṣoḍaśavarṣasya yauvanaṁ samapadyata\thinspace{\dandab} \dontdisplaylinenum

na valīpalitaṁ sadyo na jarā na ca durbalaḥ \veg\dontdisplaylinenum
\varr{
        \ \vb °padyata\lem  \msCa\msCb; °padyate \Ed
        \ \vc valī°\lem  \msCa\msCb; vali° \Ed}

keśadantanakhasnigdho d\textsubring{r}ḍhadanto d\textsubring{r}ḍhendriyaḥ\thinspace{\dandab} \dontdisplaylinenum

tejaścakṣurbalaprāṇān sadya sarvān avāptavān \veg\dontdisplaylinenum
\varr{
        \ \vb °danto\lem  \msCa\msCb; °deho \Ed
        \ \vc °cakṣurbalaprāṇān\lem  \msCa\msCb; °cakṣuvalaprāṇaṁ \Ed}

mantrī purohitāmātya sarve bh\textsubring{r}tyajanās tathā\thinspace{\dandab} \dontdisplaylinenum

paurastrī bālav\textsubring{r}ddhāś ca sarve te vismayaṁ gatāḥ \veg\dontdisplaylinenum
\varr{
        \ \vb sarve bh\textsubring{r}tyajanās tathā\lem  \msCa\Ed; janās tathās tathā \msCb
        \ \vc °strī\lem  \msCa\msCb; °stri \Ed
        \ \vd sarve\lem  \msCb\Ed; {\lost}{\lost} \msCa}

rājā siṁhajaṭo nāma tuṣṭim eva parāṁ gataḥ\thinspace{\dandab} \dontdisplaylinenum

praharṣam atulaṁ caiva prāptavān sa nareśvaraḥ \veg\dontdisplaylinenum

uvāca rājā taṁ śreṣṭhiṁ svārthatatparanirdayaḥ\thinspace{\dandab} \dontdisplaylinenum

kuru bhīmabalas tv evaṁ phalam ānaya adya vai \veg\dontdisplaylinenum
\varr{
        \ \va śreṣṭhiṁ\lem  \msCa\msCb; śreṣṭhaṁ \Ed
        \ \vb °dayaḥ\lem  \msCa\msCb; °daya \Ed
        \ \vc kuru\lem  \msCa\msCb; ś\textsubring{r}ṇu \Ed\oo
                 bhīmabalas tv evaṁ\lem  \msCb\msCc; bhīmavastv evaṁ \msCa\Ed}

punar me yauvanaprāptis tvatprasādān narottama\thinspace{\dandab} \dontdisplaylinenum

kekayīṁ durbalāṁ v\textsubring{r}ddhāṁ punaḥ prāpaya yauvanam \veg\dontdisplaylinenum
\varr{
        \ \vb °ttama\lem  \msCa\msCb; °ttamaḥ \Ed
        \ \vc kekayīṁ durbalāṁ\lem  \corr; kaikayīn durbalān \msCa, kekayīṁ \msCb, kekayī durbalā \Ed
        \ \vcd v\textsubring{r}ddhāṁ punaḥ\lem  \msCb; v\textsubring{r}\uncl{ddhā}{\lost}{\lost} \msCa, v\textsubring{r}ddhā punaḥ \Ed}

sa rājñā evam uktas tu śreṣṭhī bhīmabalas tathā\thinspace{\dandab} \dontdisplaylinenum

pratyuvāca ha rājānaṁ prāñjaliḥ praṇataḥ sthitaḥ \veg\dontdisplaylinenum
\varr{
        \ \vb śreṣṭhī\lem  \Ed; śreṣṭhi \msCa\msCb
        \ \vc °vāca ha\lem  \msCa\msCb; °vācāha \Ed}

na phaledaṁ vane rājan na vāṇijyak\textsubring{r}ṣeṇa vā\thinspace{\dandab} \dontdisplaylinenum

kenāpi kulaputreṇa tava darśanakāṁkṣayā \veg\dontdisplaylinenum
\varr{
        \ \va na phaledaṁ\lem  \Ed; na vane na \msCa\msCb\oo
                 rājan na\lem  \msCa\Ed; rājānna \msCb}

datto 'smi tena rājendra mayā datto 'si bhūpate\thinspace{\dandab} \dontdisplaylinenum

na te śaknomy ahaṁ rājan vaktuṁ vaideśinaṁ naram \veg\dontdisplaylinenum
\varr{
        \ \va tena\lem  \msCa\msCb; tava \Ed
        \ \vb datto 'si\lem  \msCa\msCb; prāptoṣi \Ed
        \ \vc te\lem  \msCa\msCb; ca \Ed
        \ \vcd rājan vaktuṁ\lem  \msCb\Ed; rā{\lost}{\lost}ktum \msCa
        \ \vd vaideśinaṁ naram\lem  \msCb; \uncl{vai}deśinan naram \msCa, ca dehi tannaraḥ \Ed}

śrutvā bhīmabalavākyaṁ pratyuvāca tataḥ punaḥ\thinspace{\dandab} \dontdisplaylinenum

amātyakulaputras tvaṁ brūhi madvacanaṁ punaḥ \veg\dontdisplaylinenum
\varr{
        \ \va °bala°\lem  \msCa\msCb\ \unmetr; °balaṁ \Ed}

yadi nāsti kim etat taṁ mayā vā mārgito bhavān\thinspace{\dandab} \dontdisplaylinenum

yatraiko bahavo 'traiva jāyante nātra saṁśayaḥ \veg\dontdisplaylinenum
\varr{
        \ \va kim etat taṁ\lem  \Ed; kim edattaṁ \msCa\msCb
        \ \vb mārgito\lem  \msCa\msCb; prārthito \Ed
        \ \vc yatraiko bahavo 'traiva\lem  \msCb; yatra hy eko bahavo 'tra \msCa\ \unmetr,
                                yatraścaiko bahūn tatra \Ed}

āgamopāyamārgaṁ ca tenaiva sa tu gamyatām\thinspace{\dandab} \dontdisplaylinenum

avaśyaṁ tena gantavyaṁ tena mārgeṇa mārgaya \veg\dontdisplaylinenum
\varr{
        \ \vc avaśyaṁ tena\lem  \msCb\Ed; ava\uncl{sya}{\lost}na \msCa\oo
                 gantavyaṁ\lem  \msCa\Ed; \uncl{buddha}vyaṁ \msCb
        \ \vd mārgaya\lem  \msCa\msCb; mārgayaḥ \Ed}

adattvā phalam anyac ca śiraś chedyāmi durmate\thinspace{\dandab} \dontdisplaylinenum

chedya caṇḍavicaṇḍābhyāṁ rakṣa bhīmabalādhama \veg\dontdisplaylinenum
\varr{
        \ \vc chedya\lem  \Ed; chedye \msCa, chede \msCb
        \ \vd °dhama\lem  \msCb; °dhamaḥ \msCa\Ed}

tato bhīmabalaḥ kruddhaḥ khaḍgaṁ g\textsubring{r}hya śaśiprabham\thinspace{\dandab} \dontdisplaylinenum

alaṅghya vacanaṁ rājñaḥ kulaputra vraja tvaram \veg\dontdisplaylinenum
\varr{
        \ \vb śaśiprabham\lem  \msCa\msCb; śaśī pradam \Ed
        \ \vc alaṅghya\lem  \msCa\msCb; uvāca \Ed
        \ \vd kulaputra vraja tvaram\lem  \Ed; kulaputraṁ vrajatyaram \msCa\msCb}

mā ruṣa kulaputra tvaṁ mayā vadhyo bhaviṣyasi\thinspace{\dandab} \dontdisplaylinenum

yady asti phalam anyad vā dehi rājānam adya vai \veg\dontdisplaylinenum
\varr{
        \ \va °putra tvaṁ\lem  \msCa\msCb; °putras tvaṁ \Ed
        \ \vc yady asti\lem  \Ed; {\lost}dyosti \msCa, sadyo sti \msCb}

yatra prāptaṁ phalaṁ divyaṁ tatra vā deśaya tava\thinspace{\dandab} \dontdisplaylinenum

tatphalena vinā bhadra durlabhaṁ tava jīvitam \veg\dontdisplaylinenum
\varr{
        \ \va prāptaṁ\lem  \msCa; prāpta° \msCb, prāpti \Ed
        \ \vb deśaya\lem  \msCa\msCb; deśayan \Ed}

vipula uvāca~{\dandab}\dontdisplaylinenum 

jīvitāśām ahaṁ prāpto vaideśibhavanaṁ tava\thinspace{\danda} \dontdisplaylinenum

k\textsubring{r}takartā kathaṁ vadhyaḥ prāpnuyām aham adya vai \veg\dontdisplaylinenum
\varr{
        \ \vd prāpnuyām\lem  \msCa\msCb; prāpto 'yam \Ed}

phalaṁ vā na punas tv anyad dātuṁ śakyaṁ na kenacit\thinspace{\dandab} \dontdisplaylinenum

sahyaparvataśailāgre āsīnaḥ śrāntamānasaḥ \veg\dontdisplaylinenum
\varr{
        \ \va vā na\lem  \msCa\Ed; vā \msCb
        \ \vb śakyaṁ na kenacit\lem  \msCb\Ed; śakya{\lost}{\lost}nacit \msCa
        \ \vd āsīnaḥ\lem  \msCa\Ed; āśītaḥ \msCb\oo
                 śrānta°\lem  \msCa\Ed; śrotta° \msCb}

vānaras tat phalaṁ g\textsubring{r}hya mama dattvā punar gataḥ\thinspace{\dandab} \dontdisplaylinenum

mayā dattam idaṁ tubhyaṁ tvayāpi ca narādhipe \veg\dontdisplaylinenum
\varr{
        \ \vb mama\lem  \msCa\msCb; mahyaṁ \Ed}

tatra gacchāva bho śreṣṭhi d\textsubring{r}śyate yadi vānaraḥ\thinspace{\dandab} \dontdisplaylinenum

tvayā mayā ca gatvaiva yācāvaḥ plavagādhipaḥ \veg\dontdisplaylinenum
\varr{
        \ \vd yācāvaḥ\lem  \msCb; yo vāsaḥ \msCa\Ed\oo
                 °dhipaṁ\lem  \msCb; °dhipaḥ \msCa\Ed}

śreṣṭhinā ca tathety āha gacchāmaḥ sahitā vayam\thinspace{\dandab} \dontdisplaylinenum

yatra prāptaṁ phalaṁ tubhyaṁ mokṣayāmo na saṁśayaḥ \veg\dontdisplaylinenum
\varr{
        \ \va tathety āha\lem  \msCa\Ed; tathaity āha \msCb
        \ \vb gacchāmaḥ\lem  \msCb\Ed; ga{\lost}mas \msCa
        \ \vc prāptaṁ\lem  \msCa\msCb; prāpta \Ed}

rudra uvāca~{\dandab}\dontdisplaylinenum 

tam āruhya giriṁ sahyaṁ mārgamāṇaḥ samantataḥ\thinspace{\danda} \dontdisplaylinenum

vipulena tato d\textsubring{r}ṣṭo vānaraḥ plavagādhipaḥ \veg\dontdisplaylinenum
\varr{
        \ \va giriṁ\lem  \msCa\Ed; giri \msCb
        \ \vb °mānaḥ\lem  \msCa\msCb; °mānāḥ \Ed
        \ \vd vānaraḥ\lem  \msCa\Ed; vānara \msCb\oo
                 plavagā°\lem  \msCb\Ed; plagā° \msCa}

ayaṁ sa vānaraśreṣṭho v\textsubring{r}kṣacchāyāsamāśritaḥ\thinspace{\dandab} \dontdisplaylinenum

mama puṇyabalenaiva d\textsubring{r}śyate 'dyāpi vānaraḥ \veg\dontdisplaylinenum
\varr{
        \ \va vānara°\lem  \msCa\msCb; vānaraḥ \Ed
        \ \vb °cchāyā°\lem  \msCb\Ed; °cchāṁyā° \msCa}

vānara kuru mitrārthaṁ sadyo m\textsubring{r}tyur bhaven mama\thinspace{\dandab} \dontdisplaylinenum

pūrvadattaṁ phalam anyad dehi vānara jīvaya \veg\dontdisplaylinenum
\varr{
        \ \va °rthaṁ\lem  \msCa\Ed; °rtha \msCb
        \ \vc °dattaṁ\lem  \msCa\Ed; °datta° \msCb
        \ \vd °hi vānara jīvaya\lem  \msCa; °vi vānara jīvayaḥ \msCb, °hi vā na ca jīvaye \Ed}

vānara uvāca~{\dandab}\dontdisplaylinenum 

gandharveṇa tu me dattaṁ phalaṁ dattaṁ tu te mayā\thinspace{\danda} \dontdisplaylinenum

punar anyat kathaṁ dāsye tatra gaccha yadīcchasi \veg\dontdisplaylinenum
\varr{
        \ \va tu me\lem  \msCa\msCb; mama \Ed}

vipula uvāca~{\dandab}\dontdisplaylinenum 

adattvā tat phalaṁ tubhyaṁ jīvituṁ saṁśayo bhavet\thinspace{\danda} \dontdisplaylinenum

athavā tatra gacchāmo yatra citrarathaḥ svayam \veg\dontdisplaylinenum
\varr{
        \ \vc athavā tatra\lem  \msCb\Ed; a{\lost}{\lost}{\lost}tra \msCa
        \ \vd citrarathaḥ\lem  \msCa\msCbpcorr\Ed; cirathaḥ \msCbacorr}

vānaraḥ punar evāha evaṁ kurvāmahe vayam\thinspace{\dandab} \dontdisplaylinenum

tataś citrarathāvāsam upagamyedam abravīt \veg\dontdisplaylinenum
\varr{
        \ \vb evaṁ\lem  \msCa\Ed; eva \msCb
        \ \vc tataś ci°\lem  \msCa\msCb; tatra ci° \Ed}

gandharvarāja kāryārthī tvāt hy aham punar āgataḥ\thinspace{\dandab} \dontdisplaylinenum

pūrvadattaphalaṁ tv anyad dehi māṁ yadi śakyate \veg\dontdisplaylinenum 
\varr{
        \ \vb tvāt hy aham pu°\lem  \msCb; tvan hy ayam pu° \msCa, tvat hy ahaṁ pu° \Ed}

gandharvarājovāca~{\dandab}\dontdisplaylinenum 
\varr{
        \ \vo gandharvarāja uvāca\lem  \msCb; gandharvarājovāca \msCa\Ed}

sūryalokagataś cāsmi tena dattaṁ phalottamam\thinspace{\danda} \dontdisplaylinenum

mayā dattaṁ phalaṁ tubhyam atyantasuh\textsubring{r}do 'si me \veg\dontdisplaylinenum
\varr{
        \ \va gataś cāsmi\lem  \msCb\Ed; gata\uncl{ś cā}{\lost} \msCa
        \ \vb tena dattaṁ\lem  \msCb\Ed; {\lost}{\lost}{\lost}ttam \msCa
        \ \vc dattaṁ\lem  \corr; datta° \msCa\msCb\Ed
        \ \vd °suh\textsubring{r}do\lem  \msCa\Ed; °suhyado \msCb}

kuto 'nyat phalam ādāsye mama nāsti plavaṅgama\thinspace{\dandab} \dontdisplaylinenum

sūryalokaṁ gamiṣyāmas tatra yācasva bhāskaram \veg\dontdisplaylinenum
\varr{
        \ \va 'nyat phalam ādāsye\lem  \msCa\msCb; 'nyaphala dāsyāmi \Ed
        \ \vb mama nāsti plavaṅgama\lem  \msCa\msCb; matto 'sti plavaṅgamaḥ \Ed
        \ \vc gamiṣyāmas\lem  \msCa\msCb; gamiṣyāmi \Ed}

gandharvenaivam uktas tu tathety āha plavaṅgamaḥ\thinspace{\dandab} \dontdisplaylinenum

sūryalokaṁ tataḥ prāptā gandharvādaya sarvaśaḥ \veg\dontdisplaylinenum
\varr{
        \ \vb tathety āha\lem  \msCa\Ed; tathaity āha \msCb
        \ \vd °daya sa°\lem  \conj; °dayas sa° \msCa, °dayaḥ sa° \msCb\Ed}

gandharva uvāca~{\dandab}\dontdisplaylinenum 
\varr{
        \ \vo gandharva uvāca\lem  \msCb; gandharva \uncl{uvā}{\lost} \msCa, gandharvarājovāca \Ed}

kāryārthena punaḥ prāptas tvatsakāśaṁ khageśvara\thinspace{\danda} \dontdisplaylinenum

pūrvadattaphalaṁ tv anyad dehi jīvam anāśaya \veg\dontdisplaylinenum
\varr{
        \ \vc tv anya°\lem  \msCa; tv a° \msCb, stv anya° \Ed
        \ \vd °nāśaya\lem  \msCa\msCb; °nāśayaḥ \Ed}

sūrya uvāca~{\dandab}\dontdisplaylinenum 

somalokagataś cāsmi tena dattaṁ phalottamam\thinspace{\danda} \dontdisplaylinenum

sa phalaṁ dattam evāsi suh\textsubring{r}datvān mayā tava \veg\dontdisplaylinenum
\varr{
        \ \vc °vāsi\lem  \msCa\msCb; °vābhiḥ \Ed
        \ \vd suh\textsubring{r}datvān mayā\lem  \msCa\msCb; sa ca datvā mayā \Ed}

anyad dātuṁ na śaknomi gaccha somapurādya vai\thinspace{\dandab} \dontdisplaylinenum

taṁ prārthayāvikalpena atriputraṁ graheśvaram \veg\dontdisplaylinenum
\varr{
        \ \va anyad\lem  \Ed; anya \msCa\msCb
        \ \vb °purādya\lem  \msCa\msCb; °parādya \Ed
        \ \vc °vikalpena\lem  \msCb\Ed; °\uncl{vika}{\lost}{\lost} \msCa
        \ \vd °putraṁ\lem  \msCb\Ed; °putra° \msCa}

rudra uvāca~{\dandab}\dontdisplaylinenum 
\varr{
        \ \vo rudra\lem  \msCa\msCb; maheśvara \Ed}

gatāḥ sūryāgrataḥ k\textsubring{r}tvā somalokaṁ tathaiva hi\thinspace{\danda} \dontdisplaylinenum

uvāca sūryaḥ somāya karuṇāpekṣayā śaśim \veg\dontdisplaylinenum
\varr{
        \ \va gatāḥ\lem  \msCb; gata \msCa, gataḥ \Ed
        \ \vd karuṇā°\lem  \msCb; kāraṇā° \msCa\Ed\oo 
                 śaśim\lem  \msCa\msCb; śaśi \Ed}

soma uvāca~{\dandab}\dontdisplaylinenum 

kimartham āgato bhūyaḥ kartavyaṁ tatra bhāskara\thinspace{\danda} \dontdisplaylinenum

phalaṁ dātuṁ punas tv anyan muktvā tv anyat karomy aham \veg\dontdisplaylinenum
\varr{
        \ \vb tatra\lem  \msCa\msCb; tava \Ed\oo
                 °kara\lem  \msCa\msCb; °karaḥ \Ed
        \ \vcd punas tv anyan muktvā tv anyat ka°\lem  \corr;
                                 punas tv anya muktvā tv anyaṅ ka° \msCa,
                                 punas tv anyan muktvās tv anyaṁ ka° \msCb,
                                 punas tv anyat muktā tv anyaṅ ka° \Ed}

sūrya uvāca~{\dandab}\dontdisplaylinenum 

yadi śakyaṁ phalaṁ dehi anyan na prārthayāmy aham\thinspace{\danda} \dontdisplaylinenum

na dattāsi phalam anyan mayā vaddhyo bhaviṣyasi \veg\dontdisplaylinenum
\varr{
        \ \va śakyaṁ phalaṁ dehi\lem  \msCa\Ed; kāphalan dehi \msCbacorr, kāphala{\il}n dehi \msCbpcorr
        \ \vb anyan na\lem  \msCa\msCb; anyān na \Ed
        \ \vcd phalam anyan ma°\lem  \msCa\msCb; phalaṁ manye ma° \Ed
        \ \vd vadhyo\lem  \corr; vaddhyo \msCa\msCb, vaddho \Ed\oo
                 bhaviṣyasi\lem  \msCa\Ed; bhaviṣyati \msCb}

soma uvāca~{\dandab}\dontdisplaylinenum 

āgamaṁ tasya vakṣyāmi ś\textsubring{r}ṇuṣvāvahito bhava\thinspace{\danda} \dontdisplaylinenum

indreṇāsmi phalaṁ dattaṁ saphalaṁ datta me bhavān \veg\dontdisplaylinenum

gatvaivendrasadas tv anyat prārthayāmaḥ sahaiva tu\thinspace{\dandab} \dontdisplaylinenum

evaṁ kurma iti prāha gatvendrasadanaṁ prati \veg\dontdisplaylinenum
\varr{
        \ \va gatvaivendra°\lem  \msCa; gatvevendra° \msCb, gandharvendra° \Ed
        \ \vc kurma\lem  \msCa\msCb; soma \Ed}

somenendram uvācedaṁ phalakāmā ihāgatāḥ\thinspace{\dandab} \dontdisplaylinenum

pūrvadattaphalam anyad dehi śakra mamādya vai \veg\dontdisplaylinenum
\varr{
        \ \vd śakra\lem  \msCa\msCb; śaka \Ed\oo
                 vai\lem  \msCa\Ed; vaiḥ \msCb}

indra uvāca~{\dandab}\dontdisplaylinenum 

yadartham iha samprāptaḥ sa ca nāsti niśākara\thinspace{\danda} \dontdisplaylinenum

viṣṇuhastān mayā prāptam ekam eva phalaṁ śubham \veg\dontdisplaylinenum
\varr{
        \ \vb °kara\lem  \msCa; °karaḥ \msCb\Ed
        \ \vd phalaṁ\lem  \msCa\Ed; phala \msCb}

sarva eva hi gacchāmo viṣṇulokaṁ graheśvara\thinspace{\dandab} \dontdisplaylinenum

sarva evopajagmus te phalārthaṁ madhusūdanam \veg\dontdisplaylinenum
\varr{
        \ \vb °śvara\lem  \msCa\Ed; °śvaraṁ \msCb
        \ \vc °jagmu°\lem  \msCb\Ed; °ñjagmu° \msCa\ \unmetr}

evam uktvā gatāḥ sarve devarājapurask\textsubring{r}tāḥ\thinspace{\dandab} \dontdisplaylinenum

muhūrtenaiva samprāptā viṣṇulokaṁ yaśasvini \veg\dontdisplaylinenum
\varr{
        \ \va °ktvā\lem  \msCa\msCb; °ktā \Ed}

upas\textsubring{r}tya tata indraḥ praṇipatya janārdanam\thinspace{\dandab} \dontdisplaylinenum

sarveṣām uparodhena prārthayāmi yaśodhara \veg\dontdisplaylinenum
\varr{
        \ \vd °dhara\lem  \msCa\msCb; °dharam \Ed}

viṣṇur uvāca~{\dandab}\dontdisplaylinenum 
\varr{
        \ \vo viṣṇur uvāca\lem  \msCapcorr\msCb; viṣṇur uca \msCaacorr, viṣṇu uvāca \Ed}

pūrvadattaphalasyārthe tac ca sarvam ihāgatāḥ\thinspace{\danda} \dontdisplaylinenum

na śaknomi phalaṁ dātuṁ kiṁ vā tv anyat karomy aham \veg\dontdisplaylinenum
\varr{
        \ \va °datta°\lem  \msCa\msCb; °dattaṁ \Ed\oo
                 °rthe\lem  \msCa\msCb; °rthi \Ed
        \ \vc śaknomi\lem  \msCa\Ed; śaknoti \msCb
        \ \vd tv anyat ka°\lem  \eme; tv anyaṅ ka° \msCa\msCb\Ed}

indra uvāca~{\dandab}\dontdisplaylinenum 

brahmāṇḍam api bhettuṁ tvaṁ śaknoṣi garuḍadhvaja\thinspace{\danda} \dontdisplaylinenum

aśakyaṁ tava nāstīti jānāmi puruṣottama \veg\dontdisplaylinenum
\varr{
        \ \va bhettuṁ tvaṁ\lem  \msCa; bhettu tvaṁ \msCb, bhartuṁtvaṁ \Ed
        \ \vb śaknoṣi\lem  \msCa\Ed; śaknoti \msCb
        \ \vc aśakyaṁ\lem  \msCa\Ed; \uncl{aśakya} \msCb
        \ \vd °ttama\lem  \msCa\msCb; °ttamam \Ed}

evam uktvā punar viṣṇuḥ pratyuvāca purandaram\thinspace{\dandab} \dontdisplaylinenum

phalam ekaṁ parityajya sarvaṁ śaknomi kauśika \veg\dontdisplaylinenum
\varr{
        \ \va °ktaḥ\lem  \msCb; °ktvā \msCa\Ed}

upāyo 'tra pravakṣyāmi āgamaṁ ś\textsubring{r}ṇu gopate\thinspace{\dandab} \dontdisplaylinenum

brahmaṇā ca mama dattaṁ tat phalaikaṁ purandara \veg\dontdisplaylinenum
\varr{
        \ \vc mama\lem  \msCa\msCb; mamā° \Ed}

mayā dattaphalaṁ tv ekaṁ kim anyad dātum icchasi\thinspace{\dandab} \dontdisplaylinenum

prārthayāmo 'tra gatvaikaṁ parameṣṭhiṁ prajāpatim \veg\dontdisplaylinenum 
\varr{
        \ \vb °cchasi\lem  \msCb\Ed; °cchati \msCa
        \ \vc prārthayāmo 'tra gatvaikaṁ\lem  \msCa\msCb; prārthayā ca gatvaivaṁ \Ed
        \ \vd °ṣṭhiṁ\lem  \msCb\Ed; °ṣṭhi° \msCa}

tavoparādhād devendra prārthayāmi pitāmaham\thinspace{\dandab} \dontdisplaylinenum

evam uktvā gatāḥ sarve purask\textsubring{r}tya janārdanam \veg\dontdisplaylinenum
\varr{
        \ \va tavo°\lem  \msCa\msCb; tato° \Ed
        \ \vc gatāḥ\lem  \msCa\msCb; gatā \Ed}

indraḥ sūryaḥ śaśī caiva gandharvo vānaras tathā\thinspace{\dandab} \dontdisplaylinenum

vipulaḥ śreṣṭhikaś caiva rājadūtadvayaṁ tathā \veg\dontdisplaylinenum
\varr{
        \ \va sūryaḥ śaśī caiva\lem  \msCa\msCb; somaś ca sūryaś ca \Ed
        \ \vd °dvayaṁ tathā\lem  \Ed; °dvayas tathā \msCa\msCb}

brahmalokaṁ muhūrtena prāptavān surasundari\thinspace{\dandab} \dontdisplaylinenum

d\textsubring{r}ṣṭvā brahmasado ramyaṁ sarvakāmaparicchadam \veg\dontdisplaylinenum
\varr{
        \ \vc °sado\lem  \msCa\msCb; °sadaṁ \Ed}

anekāni vicitrāṇi ratnāni vividhāni ca\thinspace{\dandab} \dontdisplaylinenum

mandārataruśobhāni vaiḍūryamaṇikuṭṭimān \veg\dontdisplaylinenum
\varr{
        \ \vc °taru°\lem  \Ed; °tala° \msCa\msCb
        \ \vd vaiḍūrya°\lem  \msCa\msCb; vaidūrya° \Ed\oo
                 °kuṭṭimān\lem  \corr; °kuṭimām \msCa, °kuṭṭimāṁ \msCb, °kuṭṭimam \Ed}

pravālamaṇistambhāni vajrakāñcanavedikām\thinspace{\dandab} \dontdisplaylinenum

pravālasphāṭiko jāla indranīlagavākṣakaḥ \veg\dontdisplaylinenum
\varr{
        \ \vb °vedikām\lem  \msCa\msCb; °vedikā \Ed
        \ \vc °sphāṭiko jāla\lem  \msCa\msCb; °sphaṭiko jālā \Ed}

paśyate vipulas tatra nānāv\textsubring{r}kṣa manoramāḥ\thinspace{\dandab} \dontdisplaylinenum

puṣpānāmitav\textsubring{r}kṣāgrāḥ phalānāmitakā bhavet \veg\dontdisplaylinenum
\varr{
        \ \va paśyate\lem  \msCa\msCb; d\textsubring{r}śyante \Ed\oo
                 vipula°\lem  \msCa\msCb; vipulā° \Ed
        \ \vc puṣpā°\lem  \msCa\msCb; puṣpa° \Ed\oo
                 °grāḥ\lem  \eme; °grā \msCa\msCb, °yā \Ed
        \ \vd phalānāmitakā\lem  \msCa\msCb; phalanāmitakāṁ \Ed}

sarvaratnamayā v\textsubring{r}kṣāḥ sarvaratnamayaṁ jalam\thinspace{\dandab} \dontdisplaylinenum

v\textsubring{r}kṣagulmalatāvallī kandamūlaphalāni ca \veg\dontdisplaylinenum
\varr{
        \ \va sarva°\lem  \msCb\Ed; sarve \msCa
        \ \vb sarva°\lem  \msCb\msCa; sarve \Ed}

sarve ratnamayā d\textsubring{r}ṣṭvā vipulo vipulekṣaṇaḥ\thinspace{\dandab} \dontdisplaylinenum

anekabhaumaṁ prāsādaṁ muktādāmavibhūṣitam \veg\dontdisplaylinenum
\varr{
        \ \va sarve\lem  \msCb\Ed; sarvai \msCa\oo
                 d\textsubring{r}ṣṭvā\lem  \msCb; d\textsubring{r}ṣṭā \msCa\Ed}

apsarogaṇakoṭībhiḥ sarvābharaṇabhūṣitam\thinspace{\dandab} \dontdisplaylinenum

vimānakoṭikoṭīśaṁ sarvakāmasamanvitam \veg\dontdisplaylinenum
\varr{
        \ \vcd vimānakoṭikoṭīśaṁ sarvakāmasamanvitam\lem  \msCa;
                                vimānakoṭikoṭīnāṁ sarvakāmasamanvitam \msCb, \om\ \Ed}

brahmalokasabhā ramyā sūryakoṭisamaprabhā\thinspace{\dandab} \dontdisplaylinenum

tatra brahmā sukhāsīno nānāratnopaśobhite \veg\dontdisplaylinenum

caturmūrtiś caturvaktraś caturbāhuś caturbhujaḥ\thinspace{\dandab} \dontdisplaylinenum

caturvedadharo devaś caturāśramanāyakaḥ \veg\dontdisplaylinenum

caturvedāv\textsubring{r}tas tatra mūrtimantam upāsate\thinspace{\dandab} \dontdisplaylinenum

gāyatrī vedamātā ca sāvitrī ca surūpiṇī \veg\dontdisplaylinenum

vyāh\textsubring{r}tiḥ praṇavaś caiva mūrtimān samupāsate\thinspace{\dandab} \dontdisplaylinenum

vauṣaṭkāro vaṣaṭkāro namaskāraḥ sa mūrtimān \veg\dontdisplaylinenum
\varr{
        \ \va vyāh\textsubring{r}tiḥ\lem  \msCa\Ed; vyāh\textsubring{r}diḥ \msCb\oo
                 praṇavaś caiva\lem  \msCb\Ed; praṇa\uncl{va}{\lost}va \msCa
        \ \vc vaṣaṭkāro\lem  \msCa\Ed; \om\ \msCb}

śrutiḥ sm\textsubring{r}tiś ca nītiś ca dharmaśāstraṁ samūrtimat\thinspace{\dandab} \dontdisplaylinenum

itihāsaḥ purāṇaṁ ca sāṁkhyayogaḥ patañjalam \veg\dontdisplaylinenum
\varr{
        \ \vb °śāstraṁ samūrtimat\lem  \msCa\msCb; °śāstrasamūrtimān \Ed
        \ \vc purāṇaṁ ca\lem  \msCa; purāṇaś ca \msCb\Ed
        \ \vd °jalam\lem  \msCa\msCb; °jali \Ed}

āyurvedo dhanurvedo vedo gāndharva-m-eva ca\thinspace{\dandab} \dontdisplaylinenum

arthavedo 'nyavedāś ca mūrtimān samupāsite \veg\dontdisplaylinenum
\varr{
        \ \vb gāndharva-m-eva\lem  \msCa; gandharvam eva \msCb, gāndharvar eva \Ed
        \ \vc arthavedo 'nyavedāś ca\lem  \Ed; arthavedānyavedañ ca \msCa\msCb}

tato brahmā samutthāya abhigamya janārdinam\thinspace{\dandab} \dontdisplaylinenum

gāṁ ca arghaṁ ca dattvaivam āsyatām iti cābravīt \veg\dontdisplaylinenum
\varr{
        \ \vc arghaṁ ca\lem  \msCa; a\uncl{gha}ñ ca \msCb, arghyañ ca \Ed}

maṇiratnamaye divye āsane garuḍadhvajaḥ\thinspace{\dandab} \dontdisplaylinenum

devarājo raviḥ somo gandharvaḥ plavageśvaraḥ \veg\dontdisplaylinenum
\varr{
        \ \vc raviḥ somo\lem  \msCa\msCb; śaśī sūryo \Ed
        \ \vd plava°\lem  \msCa\msCbpcorr\Ed; pla° \msCbacorr}

vipulaś ca mahāsattva āsyatāṁ ratna-āsane\thinspace{\dandab} \dontdisplaylinenum

sādhu bho vipula śreṣṭha sādhu bho vipulaṁ tapaḥ \veg\dontdisplaylinenum
\varr{
        \ \va mahāsattva\lem  \msCa\Ed; samāsatva \msCb
        \ \vb āsyatāṁ\lem  \msCa\Ed; āsyatā \msCb\oo
                 °āsane\lem  \msCa\msCb; °śāśane \Ed
        \ \vc bho\lem  \msCa\Ed; ho \msCb
        \ \vd vipulaṁ tapaḥ\lem  \Ed; \uncl{vi}{\lost}{\lost}{\lost}paḥ \msCa, vipulatapaḥ \msCb}

sādhu bho vipulaprājña sādhu bho vipulaśriya\thinspace{\dandab} \dontdisplaylinenum

toṣitāḥ sma vayaṁ sarve brahmaviṣṇumaheśvarāḥ \veg\dontdisplaylinenum
\varr{
        \ \vb °śriya\lem  \msCa; °priyaḥ \msCb, °śriyaḥ \Ed
        \ \vc toṣitāḥ\lem  \msCa\msCb; toṣitā \Ed}

ādityā vasavo rudrāḥ sādhyāśvinau marut tathā\thinspace{\dandab} \dontdisplaylinenum

bhuṅkṣva bhogān yathotsāhaṁ mama loke yathāsukham \veg\dontdisplaylinenum
\varr{
        \ \va rudrāḥ\lem  \msCa\msCb; rudrā \Ed
        \ \vb sādhyāśvinau\lem  \conj; sādhyāśvinyau \msCa\msCb, sādhyā yakṣo \Ed
        \ \vc bhuṅkṣva\lem  \msCa\msCb; bhuṁkṣa \Ed}

iyaṁ vimānakoṭīnāṁ tavārthāyopakalpitā\thinspace{\dandab} \dontdisplaylinenum

sahasrāṇāṁ sahasrāṇi apsarā kāmarūpiṇī \veg\dontdisplaylinenum
\varr{
        \ \vb tavārthāyo°\lem  \msCa\Ed; tavāyo° \msCb\oo    
                 °kalpitā\lem  \msCa\msCb; °kalpitān \Ed
        \ \vc sahasrāṇāṁ\lem  \msCa\Ed; sahasrāṇā \msCb
        \ \vd °rūpiṇī\lem  \msCa\msCb; °rūpiṇi \Ed}

tavārthīyopasarpanti sarvālaṁkārabhūṣitāḥ\thinspace{\dandab} \dontdisplaylinenum

yāvat kalpasahasrāṇi parārdhāni tapodhana \danda\dontdisplaylinenum

yatra yatra prayāsitvaṁ tatra tatropabhujyatām \veg\dontdisplaylinenum
\varr{
        \ \va °rthīyo°\lem  \msCa; tavārthāyo° \msCb, °rtheyo° \Ed
        \ \vd parārdhāni\lem  \msCa\msCbpcorr\Ed; parāṇi \msCbacorr\oo
                 °dhana\lem  \msCa\msCb; °dhanāḥ \Ed}

maheśvara uvāca~{\dandab}\dontdisplaylinenum 

iti śrutvā vacas tasya vipulo vipulekṣaṇaḥ\thinspace{\danda} \dontdisplaylinenum

vepamāno bhayatrasta aśrupūrṇākulekṣaṇaḥ \veg\dontdisplaylinenum
\varr{
        \ \vb vipulo\lem  \msCa\Ed; \om\ \msCb
        \ \vc bhayatrasta\lem  \Ed; bhayas tatra \msCa\msCb}

praṇamya śirasā bhūmau praṇipatya punaḥ punaḥ\thinspace{\dandab} \dontdisplaylinenum

uvāca madhuraṁ vākyaṁ brahma lokapitāmaham \veg\dontdisplaylinenum
\varr{
        \ \vc madhuraṁ\lem  \msCa\Ed; madhura° \msCb
        \ \vd loka°\lem  \msCa\msCb; loke \Ed}

vipula uvāca~{\dandab}\dontdisplaylinenum 

bhagavan sarvalokeśa sarvalokapitāmaha\thinspace{\danda} \dontdisplaylinenum

svapnabhūtam ivāścaryaṁ paśyāmi tridaśeśvara \veg\dontdisplaylinenum

sm\textsubring{r}tibhraṁśaś ca me jāto buddhir jātāndhacetanā\thinspace{\dandab} \dontdisplaylinenum

mūḍho 'haṁ tvāṁ kathaṁ staumi jñānātītaṁ parāt param \veg\dontdisplaylinenum
\varr{
        \ \vb jātāndhacetanā\lem  \msCa\msCb; jāto 'ndhacetanaḥ \Ed
        \ \vcd (mūḍho{\normalfont ...} parāt param)\lem  \Ed; \om\ \msCa\msCb}

\ujvers\nemsloka 
tubhyaṁ trailokyabandho bhava mama śaraṇaṁ trāhi saṁsāraghorāt
\dontdisplaylinenum

\nemslokab 
bhīto 'haṁ garbhavāsāj jaramaraṇabhayāt trāhi māṁ mohabandhāt \danda\dontdisplaylinenum

\nemslokac 
nityaṁ rogādhivāsam aniyatavapuṣaṁ trāhi māṁ kālapāśāt
\dontdisplaylinenum

\nemslokad 
tiryaṁ cānyonyabhakṣaṁ bahuyugaśataśas trāhi mohāndhakārāt \veg\dontdisplaylinenum
\varr{
        \ \va °ghorāt\lem  \msCb; °ghoram \msCa\Ed\oo
                 tubhyaṁ\lem  \msCa\msCb; namas \Ed\oo
                 trailokya°\lem  \msCa\Ed; trelokya° \msCb
        \ \vb °jara°\lem  \msCa\msCb; °janu° \Ed\oo
                 °bhayāt\lem  \Ed; bhayaṁ \msCa\msCb
        \ \vc nityaṁ\lem  \msCa\Ed; nitya° \msCb\ \unmetr\oo
                 rogā°\lem  \msCa\msCb; °rāgā° \Ed\oo
                 °niyata°\lem  \msCa\Ed; °tiyata° \msCb\oo
                 °vapuṣaṁ trāhi māṁ\lem  \msCa\Ed; °\uncl{vapuṣa trāhi mā} \msCb
        \ \vd tiryaṁ\lem  \msCa\msCb; tiryaś \Ed}

\ujvers\nemsloka 
śrutvaivovāca brahmā vipulamati punar mānayitvā yathāvat
\dontdisplaylinenum

\nemslokab 
āhūta samplavan te bhaviṣyasi tava me janmalobho na bhūyaḥ \danda\dontdisplaylinenum

\nemslokac 
garbhāvāsan na ca tvan na ca punamaraṇaṁ kleśam āyāsapūrṇam
\dontdisplaylinenum

\nemslokad 
chittvā mohāndhaśatruṁ vrajasi ca paramaṁ brahmabhūyatvam eṣi \veg\dontdisplaylinenum

\vers
\varr{
        \ \va śrutvaivovāca\lem  \msCa\msCb; śrutvaiva vāca \Ed\oo
                 °mati\lem  \Ed; °matiḥ \msCa\msCb\oo
                 mānayitvā\lem  \msCa\msCb; mānayaṁvā \Ed
        \ \vb āhūta\lem  \msCa\msCb; ābhūta \Ed\oo
                 bhaviṣyasi\lem  \msCa\msCb; avipali \Ed\oo
                 me janmalobho na\lem  \msCa\msCb; yajanmalābhānu \Ed
        \ \vc °vāsan na ca tvan na\lem  \msCa; °vāsanna \msCb, °vāsānubandhaṁ na \Ed\oo
                 punamaraṇaṁ\lem  \Ed; punarmaraṇaṁ \msCa, punarmaṇa \msCb
        \ \vd °śatruṁ\lem  \msCa\Ed; °śatru \msCb}

maheśvara uvāca~{\dandab}\dontdisplaylinenum 

brahmaṇā evam uktas tu viṣṇunā prabhaviṣṇunā\thinspace{\danda} \dontdisplaylinenum

evaṁ bhavatu bhadraṁ vo yathovāca pitāmahaḥ \veg\dontdisplaylinenum
\varr{
        \ \vb viṣṇunā\lem  \msCa\Ed; \om\ \msCb
        \ \vd °mahaḥ\lem  \msCa\Ed; °maha \msCb}

indreṇa raviṇā caiva somena ca punaḥ punaḥ\thinspace{\dandab} \dontdisplaylinenum

sādhyādityair marudrudrair viśvebhir vasavais tathā \veg\dontdisplaylinenum
\varr{
        \ \va raviṇā\lem  \msCa\msCb; śaśinā \Ed
        \ \vb somena\lem  \msCa\msCb; sūryeṇa \Ed\oo
                 punaḥ punaḥ\lem  \msCa\Ed; puna punaḥ \msCb\ \unmetr
        \ \vb °drair viśvebhir\lem  \Ed; °drair viśveśvi \msCa, °drai viśvāśvi \msCb}

aho tapaḥphalaṁ divyaṁ vipulasya mahātmanaḥ\thinspace{\dandab} \dontdisplaylinenum

svaśarīraṁ divaṁ prāptaḥ śraddhayātithipūjayā \veg\dontdisplaylinenum
\varr{
        \ \vc svaśarīraṁ\lem  \msCa; śaśarīro \msCb, saśarīraṁ \Ed\oo
                 prāptaḥ\lem  \msCb; prāptaṁ \msCa\Ed
        \ \vd °pūjayā\lem  \msCa\msCb; °pūjanāt \Ed}

evam ādīny anekāni vipule parikīrtitam\thinspace{\dandab} \dontdisplaylinenum

brahmāṇaṁ punar evāha viṣṇur viśvajagatprabhuḥ \veg\dontdisplaylinenum


\jump
\begin{center}
\ketdanda iti v\textsubring{r}ṣasārasaṁgrahe vipulopākhyāno nāmādhyāyo dvādaśamaḥ\ketdanda
\end{center}
\dontdisplaylinenum\vers 

\vers
\varr{
        \ \vc brahmāṇaṁ\lem  \msCa\Ed; brāhmaṇaḥ \msCb
        \ {\normalfont \Colo: } nāmādhyāyo dvādaśamaḥ\lem  \msCa\msCb; nāma dvādaśo 'dhyāyaḥ \Ed}
\bekveg\szamveg\vfill\phpspagebreak\szam\bek\versno=0\fejno=13
\thispagestyle{empty}



\alfejezet{\textbf{13 garbhotpattiḥ}}\jump\jump 
devy uvāca~{\dandab}\dontdisplaylinenum 

ahiṁsātithyakānāṁ ca śruto dharmaḥ suvistaraḥ\thinspace{\danda} \dontdisplaylinenum

kiṁ na kurvanti manujāḥ sukhopāyaṁ mahat phalam \veg\dontdisplaylinenum

svaśarīrasthito yajñaḥ svaśarīre sthitaṁ tapaḥ\thinspace{\dandab} \dontdisplaylinenum

svaśarīre sthitaṁ tīrthaṁ śruto vistarato mayā \veg\dontdisplaylinenum

kimarthaṁ bhagavan brūhi sukhopāyaṁ mahat phalam\thinspace{\dandab} \dontdisplaylinenum

kiṁ niv\textsubring{r}ttās tu deveśa \textsubring{r}ṣidaivatamānuṣāḥ \veg\dontdisplaylinenum

mahādeva uvāca~{\dandab}\dontdisplaylinenum 
\varr{
        \ \vo mahādeva\lem  \msCa; bhagavān \Ed}

adya p\textsubring{r}ṣṭena kathitaṁ gopitaṁ \textsubring{r}ṣi sundari\thinspace{\danda} \dontdisplaylinenum

mānuṣāṇāṁ hitārthāya tava ca varavarṇini \veg\dontdisplaylinenum

adyaprabh\textsubring{r}ti deveśi khyātir loke bhaviṣyati\thinspace{\dandab} \dontdisplaylinenum

dhanyā evaṁ cariṣyanti adhanyā na ramanti tam \veg\dontdisplaylinenum

triguṇena tu bandhena baddhā pāśad\textsubring{r}ḍhena tu\thinspace{\dandab} \dontdisplaylinenum

tenārthena ramanty atra jānanto 'pi vimohitāḥ \veg\dontdisplaylinenum

devy uvāca~{\dandab}\dontdisplaylinenum 

kiṁ vā triguṇabandheti brūhi saṁśayachedaka\thinspace{\danda} \dontdisplaylinenum

adyāpi mama deveśa mohotpannas tribandhanaiḥ \veg\dontdisplaylinenum

bhagavān uvāca~{\dandab}\dontdisplaylinenum 

prāk\textsubring{r}taṁ vaik\textsubring{r}taṁ caiva dakṣiṇābandham eva ca\thinspace{\danda} \dontdisplaylinenum

etenaiva tu bandhena baddhāḥ varṇāśramāḥ sadā \veg\dontdisplaylinenum

jñānahīnā nivartante paramaṁ prāpya tatparam\thinspace{\dandab} \dontdisplaylinenum

iṣṭastrīṇā nivartante dhanadhānyasamuccaye \danda\dontdisplaylinenum

snehād āk\textsubring{r}ṣya manasāṁ bandhaḥ prāk\textsubring{r}ta ucyate \veg\dontdisplaylinenum

yogayuktena manasā yad yad aiśvaryam āpyate\thinspace{\dandab} \dontdisplaylinenum

tac ca vaik\textsubring{r}tabandhas tu yadi tatrānurajyate \veg\dontdisplaylinenum

ārāmodyānavāpīṣu dānakratuphaleṣu ca\thinspace{\dandab} \dontdisplaylinenum

āśaktamanasā vācā dakṣiṇābandhaḥ kathyate \veg\dontdisplaylinenum

anenaiva tu pāśena baddhāvānaravad yathā\thinspace{\dandab} \dontdisplaylinenum

mokṣitaṁ na ca śaknoti itaś cetaś ca dhāvati \veg\dontdisplaylinenum

devāsuramanuṣyeṣu tiryeṣu narakeṣu ca\thinspace{\dandab} \dontdisplaylinenum

bhramante cakrayantreva ? yāvat tattvaṁ na vindati \veg\dontdisplaylinenum

garbhavāsaparikleśau janmam\textsubring{r}tyu punaḥ punaḥ\thinspace{\dandab} \dontdisplaylinenum

vyādhiḥ śokabhayāyāsa cintayā jarayā hataḥ \veg\dontdisplaylinenum

devy uvāca~{\dandab}\dontdisplaylinenum 

garbhotpattiḥ kathaṁ deva yogī labhati kīd\textsubring{r}śīm\thinspace{\danda} \dontdisplaylinenum

kīd\textsubring{r}śaṁ labhate garbhaḥ śrotuṁ naḥ pratyudīryatām \veg\dontdisplaylinenum

bhagavān uvāca~{\dandab}\dontdisplaylinenum 

ś\textsubring{r}ṇu devi pravakṣyāmi garbhotpattir yathākramam\thinspace{\danda} \dontdisplaylinenum

yathā saṁśayavicchedaṁ bhaviṣyasi varānane \veg\dontdisplaylinenum

akṣarāt prabhavo brahmā karmabaddhasamudbhavam\thinspace{\dandab} \dontdisplaylinenum

karmato yajñaprabhavo yajñato dhūmasambhavaḥ \veg\dontdisplaylinenum

parjanyād annam utpattir annād bhūtāni jajñire\thinspace{\dandab} \dontdisplaylinenum

annād rasasamutpatti rasāc choṇitasambhavaḥ \veg\dontdisplaylinenum

śoṇitāt - māṁsa-m-utpatti māṁsād medasamudbhavaḥ\thinspace{\dandab} \dontdisplaylinenum

medaso 'sthīni jāyante asthibhyo majjasambhavaḥ \veg\dontdisplaylinenum

majjāyās tu bhavec chukraṁ naraḥ śukrasamudbhavaḥ\thinspace{\dandab} \dontdisplaylinenum

śukraśoṇitasaṁyogād garbhotpattis tataḥ sm\textsubring{r}taḥ \veg\dontdisplaylinenum

agnisomātmakaṁ devi śarīradvayadhātutaḥ\thinspace{\dandab} \dontdisplaylinenum

somadhātusm\textsubring{r}taṁ śukram agnidhāturajasm\textsubring{r}tam \danda\dontdisplaylinenum

agnisomāśrayaṁ devi śarīram iti saṁjñitam \veg\dontdisplaylinenum

māsī māsī \textsubring{r}tuḥ strīṇāṁ bhavatīha na saṁśayaḥ\thinspace{\dandab} \dontdisplaylinenum

\textsubring{r}tukāle prasarpyeta na sukhārthaṁ varānane \veg\dontdisplaylinenum

putrakāmaprayuñjīta dharmārthaś ca yaśasvini\thinspace{\dandab} \dontdisplaylinenum

pumān strīpuṁ prayuñjīta araṇī bahutāśanaḥ \veg\dontdisplaylinenum

pumān śukrādhiko jñeyaḥ kanyā raktādhikā bhavet\thinspace{\dandab} \dontdisplaylinenum

samaśukre ca rakte ca sa ca jāyen napuṁsakaḥ \veg\dontdisplaylinenum


\alalfejezet{dviyamā triyamā ca gurviṇī}
devy uvāca~{\dandab}\dontdisplaylinenum 

dviyamā triyamā caiva kathaṁ jāyeta gurviṇī\thinspace{\danda} \dontdisplaylinenum

kathaṁ strīdviyamā jāyet kathaṁ vā puruṣadvayam \veg\dontdisplaylinenum

bhagavān uvāca~{\dandab}\dontdisplaylinenum 

raktādhikā sm\textsubring{r}tā kanyā jāyate varavarṇini\thinspace{\danda} \dontdisplaylinenum

vāyunā ca dvidhā bhinnā kanyakadviyamā sm\textsubring{r}tā \veg\dontdisplaylinenum

śukrādhikās tu puruṣa dvidhā bhinnānilena tu\thinspace{\dandab} \dontdisplaylinenum

dviyamā puruṣo jñeyā triyamās tu tridhā k\textsubring{r}te \veg\dontdisplaylinenum

\textsubring{r}tusnātā yadā nārī yadi garbhādi g\textsubring{r}hyati\thinspace{\dandab} \dontdisplaylinenum

prathame ca dvitīye ca t\textsubring{r}tīye ca sa jīvati \veg\dontdisplaylinenum

sameṣu janayet putraḥ kanyakā viṣame dine\thinspace{\dandab} \dontdisplaylinenum

ṣaṣṭyāṣṭamau ca daśamī dvādaśī ca pumān bhavet \veg\dontdisplaylinenum

pañcamī saptamī caiva navamekādaśī striyaḥ\thinspace{\dandab} \dontdisplaylinenum

samarakte ca śukre ca śyāmaḥ saṁjāyate pumān \veg\dontdisplaylinenum

rudhiraṁ tv ekarātreṇa kalalaṁ pratipadyate\thinspace{\dandab} \dontdisplaylinenum

kalalaṁ pañcarātreṇa arbudatvaṁ prajāyate \veg\dontdisplaylinenum

arbudaḥ saptarātreṇa māṁsapeśī samudbhavaḥ\thinspace{\dandab} \dontdisplaylinenum

dvitīyaṁ saptarātreṇa tat sarvaṁ māṁsaśoṇitam \veg\dontdisplaylinenum

t\textsubring{r}tīyaṁ saptarātreṇa h\textsubring{r}dayaṁ jāyate tataḥ\thinspace{\dandab} \dontdisplaylinenum

tataḥ sarvāṇi gātrāṇi śiraś caivopajāyate \veg\dontdisplaylinenum

h\textsubring{r}daye jāyamāne tu mūrcchāntandrirarocakaḥ\thinspace{\dandab} \dontdisplaylinenum

striyāḥ dhardiḥ praśekaś ca daurbalyaṁ copajāyate \veg\dontdisplaylinenum

tasyā hi h\textsubring{r}dayaṁ nārī yadi bhakṣyati kiṁcana\thinspace{\dandab} \dontdisplaylinenum

bhakṣyaṁ lohyaṁ tathā peyam upabhogās tathāyayat \veg\dontdisplaylinenum

śayanāsanayānāni vastrāṇy ābharaṇāni ca\thinspace{\dandab} \dontdisplaylinenum

yad yad ākāṁkṣate kiṁcit tat tad āsyai pradāpayet \veg\dontdisplaylinenum
            \paral{\textit{ \vo {\normalfont Cf.\ MBh 13.40.12:}
        śayyāsanam alaṁkāram annapānam anāryatām
        durvāgbhāvaṁ ratiṁ caiva dadau stgrībhyaḥ prajāpatiḥ}}

nāyā saṁkārayec cāsyā na caivam avamānayet\thinspace{\dandab} \dontdisplaylinenum

mukham āpāṇḍuraṁ snigdhaṁ kapolastanakeśayoḥ \veg\dontdisplaylinenum

śarīraś ca śriyā jaṣṭuṁ pīnoruśroṇi vakṣasam\thinspace{\dandab} \dontdisplaylinenum

liṅgerebhir vijānīyāṁ garbhe jīvaṁ pratiṣṭhitam \veg\dontdisplaylinenum

caturthe saptarātreṇa śiraś caivopajāyate\thinspace{\dandab} \dontdisplaylinenum

pañcamasaptarātreṇa grīvā tatropajāyate \veg\dontdisplaylinenum

ṣaṣṭhamasaptarātreṇa skandhagātraṁ prajāyate\thinspace{\dandab} \dontdisplaylinenum

saptamasaptarātreṇa p\textsubring{r}ṣṭhavaṁśa prajāyate \veg\dontdisplaylinenum

aṣṭamasaptarātreṇa pāṇī jāyate cobhayau\thinspace{\dandab} \dontdisplaylinenum

saptarātraṁ nava prāpya jāyate h\textsubring{r}di pañjaram \veg\dontdisplaylinenum

daśame saptarātre ca pādau jāyate cobhau\thinspace{\dandab} \dontdisplaylinenum

udaraś copajāyeta saptaikādaśarātrike \veg\dontdisplaylinenum

dvādaśasaptarātreṇa kukṣipārśveḥ prajāyate\thinspace{\dandab} \dontdisplaylinenum

saptatraidaśarātreṇa kuṭisutropajāyate \veg\dontdisplaylinenum

navaty aṣṭamarāteṇa jāyate sūtraviṁśati\thinspace{\dandab} \dontdisplaylinenum

saptapañcadaśāhena sarvamedaḥ prajāyate \veg\dontdisplaylinenum

ṣoḍaśasaptarātreṇa athisarvāṇi jāyate\thinspace{\dandab} \dontdisplaylinenum

saptasaptadaśāhena jāyate snāyubandhanam \veg\dontdisplaylinenum

saptamāṣṭādaśāhena jāyate mukhamaṇḍalam\thinspace{\dandab} \dontdisplaylinenum

saptonaviṁśarātreṇa ghrāṇavaṁśaḥ prajāyate \veg\dontdisplaylinenum

saptaviṁśatirātreṇa naitranāliṁ prajāyate\thinspace{\dandab} \dontdisplaylinenum

saptaikaviṁśarātreṇa karṇayugmaṁ prajāyate \veg\dontdisplaylinenum

dvāviṁśasaptarātreṇa jāyate dvau bhruvau tataḥ\thinspace{\dandab} \dontdisplaylinenum

saptatriviṁśarātreṇa gaṇḍayugmaṁ prajāyate \veg\dontdisplaylinenum

caturviṁśatisaptāhe oṣṭhayugmaṁ prajāyate\thinspace{\dandab} \dontdisplaylinenum

pañcaviṁśatisaptāhe jihvā jāyate sundari \veg\dontdisplaylinenum

ṣaḍviṁśasaptarātreṇa dantapaṅkti prajāyate\thinspace{\dandab} \dontdisplaylinenum

unaviṁśatisaptāhe jāyate ca tvag eva ca \veg\dontdisplaylinenum

triṁśatasaptarātreṇa jāyate nābhimaṇḍalam\thinspace{\dandab} \dontdisplaylinenum

saptaikatriṁśarātreṇa sarvarandhraṁ prajāyate \veg\dontdisplaylinenum

dvātriṁśasaptarātreṇa nakhaviṁśati jāyate\thinspace{\dandab} \dontdisplaylinenum

tritriṁśasaptarātreṇa sarve sandhiḥ prajāyate \veg\dontdisplaylinenum

pañcatriṁśati saptāhe sarvamarma prajāyate\thinspace{\dandab} \dontdisplaylinenum

ṣaḍtriṁśasaptarātreṇa vedanā copajāyate \veg\dontdisplaylinenum

saptatriṁśatisaptāhe īrṣyādveṣaḥ prajāyate\thinspace{\dandab} \dontdisplaylinenum

aṣṭatriṁśatisaptāhe pañcātmakasamanvitam \veg\dontdisplaylinenum

sarvāṅgam aṅgasampūrṇaḥ paripakva(ḥ) sa tiṣṭhati\thinspace{\dandab} \dontdisplaylinenum

mātusvāśitapītaś ca nābhisūtrāganena tu \veg\dontdisplaylinenum

ajātasyopadhāryante garbhasthasyaiva jantavaḥ\thinspace{\dandab} \dontdisplaylinenum

tataḥ praviśate dehe nidrāsvapna yathā tathā \veg\dontdisplaylinenum

nopalabhyati sūkṣmatvād araṇy agnir yathā tathā\thinspace{\dandab} \dontdisplaylinenum

garbhodakena siktāṅgajarāyā pariveṣṭitaḥ \veg\dontdisplaylinenum

jāti smarati tatrastho jantuś cetaḥsamanvitaḥ\thinspace{\dandab} \dontdisplaylinenum

m\textsubring{r}taś cāhaṁ punarjāto bhūyaś caiva punarm\textsubring{r}taḥ \veg\dontdisplaylinenum

sthāvarāṇāṁ sahasreṣu jāto 'smi vividheṣu ca\thinspace{\dandab} \dontdisplaylinenum

caturvarṇavivarṇeṣu mānuṣeṣu sahasraśaḥ \veg\dontdisplaylinenum

sāmprataṁ ca punar garbhaḥ kleśaḥ prāptaḥ suduḥsahaḥ\thinspace{\dandab} \dontdisplaylinenum

idānīṁ jātamātro 'haṁ saṁskāraiś cāpi saṁsk\textsubring{r}taḥ \veg\dontdisplaylinenum

yogam evābhisevāmi sā[ṁ]khyaṁ vā pañcaviṁśakam\thinspace{\dandab} \dontdisplaylinenum

yatra janmajarā nāsti yatra m\textsubring{r}tyuś ca nāsti vai \veg\dontdisplaylinenum

yatra brahma paraṁ vedyaṁ cariṣyāmi yatavrataḥ\thinspace{\dandab} \dontdisplaylinenum

evam ādīny anekāni cintayitvā punaḥ punaḥ \veg\dontdisplaylinenum

yāvat tiṣṭhati garbhastho jāti smarati pūrvikām\thinspace{\dandab} \dontdisplaylinenum

tato jāyati kaṣṭena mahākleśena mānavaḥ \veg\dontdisplaylinenum

yoniyantrasutīvreṇa pīḍyamānasuduḥkhitaḥ\thinspace{\dandab} \dontdisplaylinenum

jātamātrosm\textsubring{r}tibhraṁśo bhavatīha acetaneḥ \veg\dontdisplaylinenum

māyāmudgaratīvreṇa hataḥ kiṁ śubham ācaret\thinspace{\dandab} \dontdisplaylinenum

eṣa garbhasamutpattiḥ kathito 'smi varānane \danda\dontdisplaylinenum

duḥkhasaṁsārapraśamaṁ kiṁ bhūyaḥ śrotum icchasi \veg\dontdisplaylinenum


\jump
\begin{center}
\ketdanda iti v\textsubring{r}ṣasārasaṁgrahe garbhotpattir nāma trayadaśo ! 'dhyāyaḥ\ketdanda
\end{center}
\dontdisplaylinenum\vers 
\bekveg\szamveg\vfill\phpspagebreak\szam\bek\versno=0\fejno=14
\thispagestyle{empty}


\vers


\alfejezet{\textbf{14 praśnavyākaraṇam}}\jump\jump 
devy uvāca~{\dandab}\dontdisplaylinenum 

atidīrghātihrasvaś ca pumān kenopajāyate\thinspace{\danda} \dontdisplaylinenum

atigauro 'tik\textsubring{r}ṣṇaś ca naro bhavati kiṁ prabho \veg\dontdisplaylinenum

bhagavān uvāca~{\dandab}\dontdisplaylinenum 

g\textsubring{r}hītagarbhā yā nārī nityam uttānaśālinī\thinspace{\danda} \dontdisplaylinenum

prasāritavimuktātmā so 'tidīrghaḥ prajāyate \veg\dontdisplaylinenum

g\textsubring{r}hītagarbhā yā nārī śete saṁkucitā sadā\thinspace{\dandab} \dontdisplaylinenum

raso 'nnādīni kaṭukaṁ sevanāḥ hrasva jāyate \veg\dontdisplaylinenum

g\textsubring{r}hītagarbhā yā nārī nityaṁ kṣīropasevitā\thinspace{\dandab} \dontdisplaylinenum

varakodravaśālī ca bhuktā cāpi yavodanam \veg\dontdisplaylinenum

śuklavastrasrajā yuktā sātigauraṁ prajāyate\thinspace{\dandab} \dontdisplaylinenum

g\textsubring{r}hītagarbhā yā nārī bāladhānyāni sevate \veg\dontdisplaylinenum

k\textsubring{r}ṣṇakodravatailādi māṣak\textsubring{r}ṣṇayavodanam\thinspace{\dandab} \dontdisplaylinenum

k\textsubring{r}ṣṇavastrasrajādīni tasyāḥ k\textsubring{r}ṣṇaḥ prajāyate \veg\dontdisplaylinenum

devy uvāca~{\dandab}\dontdisplaylinenum 

jātyandho jāyate kasmānṣaṇḍhobhīrur hatendriyaḥ\thinspace{\danda} \dontdisplaylinenum

kujo vā vāmano vāpi paṅgavaḥ sthūlaśiraḥ katham \veg\dontdisplaylinenum

bhagavān uvāca~{\dandab}\dontdisplaylinenum 

g\textsubring{r}hītagarbhā yā nārī tīkṣṇoṣṇāny upasevate\thinspace{\danda} \dontdisplaylinenum

laśunānipalāṇḍūni karañjamūlakāni ca \veg\dontdisplaylinenum

pippalīś\textsubring{r}ṅgaveraṁ ca sarṣapānmaricāni ca\thinspace{\dandab} \dontdisplaylinenum

āsavaś ca parikliṣṭā ye cānye kaṭutiktakāḥ \veg\dontdisplaylinenum

tīkṣṇaṁ tu sevamānā yā jātyandho jāyate sutaḥ\thinspace{\dandab} \dontdisplaylinenum

mithyāpacārāḥ strīpuṁso vyāpanne śukraśoṇite \danda\dontdisplaylinenum

yadā garbhāśaye raktaṁ striyāḥ pūrvaṁ niṣicyate \veg\dontdisplaylinenum

paścāc chukraṁ raktakāle tadāṣaṇḍaḥ prajāyate\thinspace{\dandab} \dontdisplaylinenum

trastodvigno yadā bhītastrīpuṁsāṁsūpajāyate \veg\dontdisplaylinenum

tatra yo jāyate garbhabhiruḥ krandanako bhavet\thinspace{\dandab} \dontdisplaylinenum

nisargakāle śukrasya vighna utpadyate yadā \veg\dontdisplaylinenum

indriyāvartavighne tu tadā jāyed atindriyaḥ\thinspace{\dandab} \dontdisplaylinenum 

g\textsubring{r}hītagarbhā yā nārī vātalāny upasevate \veg\dontdisplaylinenum

kaṭukāni kaṣāyāni tiktāni ca viśeṣataḥ\thinspace{\dandab} \dontdisplaylinenum

vātaḥ prakupitas tasyā garbham ātuhya tiṣṭhati \veg\dontdisplaylinenum

kubjas tu jāyate tasmād garbhād vātanipīḍanāt\thinspace{\dandab} \dontdisplaylinenum

nityasāsavaśīlāyā tathā cotkaṭukāśanā \veg\dontdisplaylinenum

tasyā saṁhanyate garbho vāmanas tena jāyate\thinspace{\dandab} \dontdisplaylinenum

ativyāyāmaśīlā tu ya nārī viṣamāsanī \veg\dontdisplaylinenum

garbhaḥ saṁkṣubhyate tasyāḥ paṣaṇḍas tenopajāyate\thinspace{\dandab} \dontdisplaylinenum

g\textsubring{r}hītagarbhā yā nārī rūkṣadhānyāni sevate \veg\dontdisplaylinenum

vātaśleṣmaśirastho vai tasyā garbhasya kupyate\thinspace{\dandab} \dontdisplaylinenum

tataḥ sthūlaśirās tena pumān jāyaty asaṁśayaḥ \veg\dontdisplaylinenum

devy uvāca~{\dandab}\dontdisplaylinenum 

karālāṅgā hanuḥ paṅgūr mūko gadgadabhāṣakaḥ\thinspace{\danda} \dontdisplaylinenum

vik\textsubring{r}tākṣas tv anakṣo vā bhavadrasvagudaḥ katham \veg\dontdisplaylinenum

bhagavān uvāca~{\dandab}\dontdisplaylinenum 

karālas tena doṣeṇa jāyate mānavas tathā\thinspace{\danda} \dontdisplaylinenum

atha karālaṁ kurute nārī lamboticūcukā \danda\dontdisplaylinenum

tasmād anena doṣeṇa karālo jāyate pumān \veg\dontdisplaylinenum

g\textsubring{r}hītagarbhā yā nārī raktapittāmayārditā\thinspace{\dandab} \dontdisplaylinenum

gohanuṁ janayet yeṣā raktapittaprakopitaḥ \veg\dontdisplaylinenum

g\textsubring{r}hītagarbhā yā nārī vātaśūlair upadrutā\thinspace{\dandab} \dontdisplaylinenum

śukro dāvartanī cāpi paṅgū janayate sutam \veg\dontdisplaylinenum

kṣudhārtā vedanārtā ca satataś copavāsinī\thinspace{\dandab} \dontdisplaylinenum

mūkaṁ janayate bālaṁ dauh\textsubring{r}daś ca vimānitā \veg\dontdisplaylinenum

g\textsubring{r}hītagarbhā yā nārī vis\textsubring{r}jet - māsa māsikam\thinspace{\dandab} \dontdisplaylinenum

anakṣo jāyate tasyā garbhaśoṇitasaṁkṣayāt \veg\dontdisplaylinenum

atha grastā yadā nārī vāto dāvartapīḍitā\thinspace{\dandab} \dontdisplaylinenum

g\textsubring{r}hītagarbhā rukṣāṇi vātalāny upasevate \veg\dontdisplaylinenum

vātasthānantatas tasyā garbhasyāpīḍitaṁ bhavet\thinspace{\dandab} \dontdisplaylinenum

agudo jāyate tasmāj jātaś cāpi na jīvati \veg\dontdisplaylinenum

devy uvāca~{\dandab}\dontdisplaylinenum 

hīnāṅgo jāyate kasmād adhikāṅgo 'pi vā katham\thinspace{\danda} \dontdisplaylinenum

śvetapiṅgekṣaṇaḥ kasmāt kathaṁ lohitalocanaḥ \veg\dontdisplaylinenum

bhagavān uvāca~{\dandab}\dontdisplaylinenum 

garbhasya jāyamānasya - - -  jāyate nilaḥ\thinspace{\danda} \dontdisplaylinenum

vātābhyāṁ śleṣmaṇāt - - - tadaṅgaṁ parihīyate \veg\dontdisplaylinenum

hīnāṅgo jāyate tasmāt pumān vātaprakopataḥ\thinspace{\dandab} \dontdisplaylinenum

g\textsubring{r}hītagarbhā yā nārī madhurāṇy upasevate \veg\dontdisplaylinenum

ś\textsubring{r}ṅgāṭakakalotyāni śālūkāni viśāni ca\thinspace{\dandab} \dontdisplaylinenum

mocaṁ tālaphalaṁ caiva nārikelaphalaṁ tathā \veg\dontdisplaylinenum

atikṣṇaṁ sevamānā tu adhikāṅgaṁprasūyate\thinspace{\dandab} \dontdisplaylinenum

piṅgākṣaḥ śleṣmapittābhyāṁ śvetākṣaḥ śleṣmaṇā bhavet \veg\dontdisplaylinenum

devy uvāca~{\dandab}\dontdisplaylinenum 

kathaṁ vā jāyate putraḥ kanyakā kena jāyate\thinspace{\danda} \dontdisplaylinenum

apumān kena jāyeta dviyamā triyamā tathā \veg\dontdisplaylinenum

bhagavān uvāca~{\dandab}\dontdisplaylinenum 

śukrādhikaḥ pumān jñeyaḥ kanyā raktādhikā bhavet\thinspace{\danda} \dontdisplaylinenum

raktaśukrasamatvena jāyate sa napuṁsakaḥ \veg\dontdisplaylinenum

piṇḍībhūto yadā garbha mārutau vibhaved dvidhā\thinspace{\dandab} \dontdisplaylinenum

evaṁ te dviyamā jñeyās triyamā ca tridhā k\textsubring{r}te \veg\dontdisplaylinenum

devy uvāca~{\dandab}\dontdisplaylinenum 

śoṇitaṁ māṁsa medaś ca asthi majjā ca pañcamī\thinspace{\danda} \dontdisplaylinenum

śarīrasthāni d\textsubring{r}śyante śukrasthānaṁ na d\textsubring{r}śyate \veg\dontdisplaylinenum

tasyotpattiś ca sthānaṁ ca jñātum icchāmi tattvataḥ\thinspace{\dandab} \dontdisplaylinenum

kathayasva trilokeśa cchettum arhasi saṁśayaḥ \veg\dontdisplaylinenum

bhagavān uvāca~{\dandab}\dontdisplaylinenum 

manaḥ śukrasya prabhavaṁ ghrāṇaṁ śrotraṁ tathākṣiṇī\thinspace{\danda} \dontdisplaylinenum

sthānaṁ tu sarvāṅgasamasparśāt sparśaḥ pravartate \veg\dontdisplaylinenum

yathā niṣiktaṁ kṣīraṁ tu payasād dadhi jāyate\thinspace{\dandab} \dontdisplaylinenum

pramathyamānadadhnas tu sarpiso 'pi tathāgamaḥ \veg\dontdisplaylinenum

evaṁ śarīra nirgaccet - śukraṁ śukravahā śirāḥ\thinspace{\dandab} \dontdisplaylinenum

pūrayitvānupūrveṇa asthayo pratipadyate \veg\dontdisplaylinenum

tatas tu tāḥ śukravahā meḍhranābhīm anus\textsubring{r}tāḥ\thinspace{\dandab} \dontdisplaylinenum

nāśukraṁ tat tu siñcanti tasmād garbhasya sambhavaḥ \veg\dontdisplaylinenum

devy uvāca~{\dandab}\dontdisplaylinenum 

kathaṁ vedayate jāti kathaṁ jātismaro bhavet\thinspace{\danda} \dontdisplaylinenum

etasmin saṁśayaṁ me 'dya chettum arhasi śaṅkara \veg\dontdisplaylinenum

bhagavān uvāca~{\dandab}\dontdisplaylinenum 

bhāvitātmāṁ ca yo jantur devi bhogādhikaṁ ca yat\thinspace{\danda} \dontdisplaylinenum

brahmavid jñānasaṁyuktaḥ sa jātiṁ smarate pumān \veg\dontdisplaylinenum

devy uvāca~{\dandab}\dontdisplaylinenum 

kathaṁ sadyo g\textsubring{r}hītasya liṅgagarbhasya d\textsubring{r}śyate\thinspace{\danda} \dontdisplaylinenum

etat kathaya deveśa rahaḥ kāle maheśvara \veg\dontdisplaylinenum

bhagavān uvāca~{\dandab}\dontdisplaylinenum 

pipāśāromaharṣaṁ ca vepanaṁ gātrasīdanam\thinspace{\danda} \dontdisplaylinenum

nidrāsvedaṁ ca tandrā ca muhūrtam upajāyate \veg\dontdisplaylinenum

nikledatvaṁ kharatvaṁ ca yonyāt samupajāyate\thinspace{\dandab} \dontdisplaylinenum

na cārdravaṁvai d\textsubring{r}śyeta śukrasya rajaso 'pi vā \danda\dontdisplaylinenum 

sadyog\textsubring{r}hītagarbhāyā liṅgāny etāni tattvataḥ \veg\dontdisplaylinenum

devy uvāca~{\dandab}\dontdisplaylinenum 

kena liṅgena vijñeyaṁ putrajanma maheśvara\thinspace{\danda} \dontdisplaylinenum

kanyakā kena liṅgena jñāyate kathayasva me \veg\dontdisplaylinenum

bhagavān uvāca~{\dandab}\dontdisplaylinenum 

pādorujaṅghapārśvaś ca dakṣiṇaṁ yadi hy unnataḥ\thinspace{\danda} \dontdisplaylinenum

dakṣiṇaṁ vipulaṁ tatra tadā putraḥ prajāyate \veg\dontdisplaylinenum

vāmaś caiva yadā paśyet tadā jāyeta kanyakā\thinspace{\dandab} \dontdisplaylinenum

unnataṁ madhyamasthāś ca tadā jāyet - napuṁsakam \veg\dontdisplaylinenum

devy uvāca~{\dandab}\dontdisplaylinenum 

puṁsā kapolaromāni khalitaṁ kena jāyate\thinspace{\danda} \dontdisplaylinenum

kathaṁ strīṇāṁ na jāyeta romāṇi khalitaṁ tathā \veg\dontdisplaylinenum

bhagavān uvāca~{\dandab}\dontdisplaylinenum 

tathā v\textsubring{r}ṣaṇagā jantor yasya retovahā śiraḥ\thinspace{\danda} \dontdisplaylinenum

nibaddhā mastake tālu kapolās tu samāśritāḥ \veg\dontdisplaylinenum

taiḥ kapoleṣu romāṇi jāyante antaretasaḥ\thinspace{\dandab} \dontdisplaylinenum

khalitaṁ śukradoṣeṇa narāṇām upajāyate \veg\dontdisplaylinenum

śirā śukravahā strīṇāṁ na śūnyasyānna jāyate\thinspace{\dandab} \dontdisplaylinenum

yātmāpālo ca kās tv agni d\textsubring{r}ṣṭimaṇḍalasaṁśritaḥ ? \veg\dontdisplaylinenum

śoṇitai soktikoṣṭasthanniśoṣayati tattvataḥ\thinspace{\dandab} \dontdisplaylinenum

nibaddhanty akṣipakṣmāṇi tena romāṇi ca bhruvoḥ \veg\dontdisplaylinenum

aśukratvāc ca nārīṇāṁ khalitaṁ nopajāyate\thinspace{\dandab} \dontdisplaylinenum

chāyāvyapagatasnehā rukṣāgātraśiroruhā \danda\dontdisplaylinenum

grasatosmābhajaṭharā m\textsubring{r}tagarbhaḥ prajāyate \veg\dontdisplaylinenum

devy uvāca~{\dandab}\dontdisplaylinenum 

somadhātu kathaṁ jñeyā agnidhātus tatheśvara\thinspace{\danda} \dontdisplaylinenum

p\textsubring{r}thagbhāgaviśeṣeṇa kathayasva maheśvara \veg\dontdisplaylinenum

bhagavān uvāca~{\dandab}\dontdisplaylinenum 

śleṣmamedas tathā snāyuḥ asthidantanakhāni ca\thinspace{\danda} \dontdisplaylinenum

striyās tanyaś ca śukraś ca yac ca śvetaṁ tathākṣiṣu \veg\dontdisplaylinenum

eteṣāṁ saumyabhāgatvāc chvetatvam upajāyate\thinspace{\dandab} \dontdisplaylinenum

āgneyabhāvād raktatvaṁ k\textsubring{r}ṣṇatvaṁ cāpi gacchati \veg\dontdisplaylinenum

tvagmāṁsarudhiraṁ majjād\textsubring{r}ṣṭiroma tathaiva ca\thinspace{\dandab} \dontdisplaylinenum

āgneyadhātusomaś ca kathito 'smi varānane \danda\dontdisplaylinenum

brūhi brūhi viśālākṣi yady asti tava saṁśayaḥ \veg\dontdisplaylinenum


\jump
\begin{center}
\ketdanda iti v\textsubring{r}ṣasārasaṁgrahe praśnavyākaraṇo nāmaś caturdaśo 'dhyāyaḥ\ketdanda
\end{center}
\dontdisplaylinenum\vers 
\bekveg\szamveg\vfill\phpspagebreak\szam\bek\versno=0\fejno=15
\thispagestyle{empty}



\alfejezet{\textbf{pañcadaśamo 'dhyāyaḥ}}\jump\jump

\alalfejezet{jīvavarṇanam}
devy uvāca~{\dandab}\dontdisplaylinenum 

jīvabhūteti yat proktaṁ lakṣaṇaṁ kīd\textsubring{r}śaṁ bhavet\thinspace{\danda} \dontdisplaylinenum

sthānam asya na jānāmi rūpaṁ varṇaṁ ca īśvara \veg\dontdisplaylinenum
            \paral{\textit{{\normalfont Testimonia for this chapter: \msCa\ ff.\thinspace 219r--220r, 
                                             \msCb\ ff.\thinspace 222v--223v, 
                                             \msCc\ is not available for this chapter,
                                             \msNa\ ff.\thinspace 26r--27r, 
                                             \msNb\ ff.\thinspace 230v--231r, 
                                             \msNc\ ff.\thinspace 234r--235r}}}
\varr{
        \ \vb lakṣaṇaṁ kī°\lem  \msNa\msNb\msNc\Ed; lakṣaṇāṅ kī° \msCa, laṇaṁ kī° \msCb
        \ \vc sthānam asya\lem  \msCb\msNa\msNb\msNc\Ed; {\il}\uncl{na}m asya \msCa
        \ \vd rūpaṁ varṇaṁ\lem  \msCa\msCb\msNa\Ed; rūpavarṇaṁ \msNb\msNc}

etat kautūhalaṁ chindhi saṁśayaṁ parameśvara\thinspace{\dandab} \dontdisplaylinenum

na cānyad eva paśyāmi jīvanirṇaya kīrtaya \veg\dontdisplaylinenum
\varr{
        \ \va etat kautūhalaṁ\lem  \msCa\msCb\msNa\msNb\Ed; etat kautūlaṁ \msNc\oo
                 chindhi\lem  \msCa\msCb\msNa\msNb\Ed; chitvāndhi \msNc
        \ \vb saṁśayaṁ\lem  \msCa\msCb\msNa\msNc\Ed; saṁśaya \msNb}

īśvara uvāca~{\dandab}\dontdisplaylinenum 
\varr{
        \ \vo īśvara\lem  \msCa\msCb\msNa\msNb\msNc; bhagavān \Ed}

jīvasya lakṣaṇaṁ devi kathituṁ kena śakyate\thinspace{\danda} \dontdisplaylinenum

na rūpavarṇaṁ jīvasya vidyate sthānam eva ca \veg\dontdisplaylinenum
\varr{
        \ \va lakṣaṇaṁ\lem  \msCb\msNa\msNb\msNc\Ed; kathitaṁ \msCa
        \ \vc °varṇaṁ\lem  \msCb\msNa\msNc; °varṇa \msCa\msNb\Ed}

vyāpi sarvagataṁ sūkṣmaṁ sarvam āśritya tiṣṭhati\thinspace{\dandab} \dontdisplaylinenum

nirālambam anādhāram anaupamyaṁ nirañjanam \veg\dontdisplaylinenum
\varr{
        \ \va vyāpi\lem  \msCb\msNa\msNb\msNc; vyā\uncl{pi} \msCa, vyāpī \Ed
        \ \va °śritya\lem  \msCb; °ś\textsubring{r}tya \msCa\msNa\msNb, °śrutya \msNc, °v\textsubring{r}tya \Ed
        \ \vd °pamyaṁ\lem  \msCa\msCb\msNa\msNc\Ed; °pamya \msNb}

araṇistho yathā vahniḥ kāṣṭheṣu nopalabhyate\thinspace{\dandab} \dontdisplaylinenum

tadvaj jīvo na paśyeta śarīrastho 'pi sundari \veg\dontdisplaylinenum
\varr{
        \ \vc jīvo na\lem  \msCb\msNa\msNb\msNc; jīvon na \msCa, jīvaṁ na \Ed
        \ \vd 'pi\lem  \msCa\msCb\msNa\msNc\Ed; hi \msNb}

dadhivac ca yathā sarpir d\textsubring{r}śyate na ca d\textsubring{r}śyate\thinspace{\dandab} \dontdisplaylinenum

tadvaj jīvaḥ śarīrastho d\textsubring{r}śyate na ca d\textsubring{r}śyate \veg\dontdisplaylinenum
\varr{
        \ \vc tadvaj jīvaḥ\lem  \msCa\msCb\msNa\msNb; tadva jīvaḥ \msNc, tadvaj jīva \Ed}

devy uvāca~{\dandab}\dontdisplaylinenum 

ad\textsubring{r}ṣṭapratyayo hy asti nāsti pratyayadarśanam\thinspace{\danda} \dontdisplaylinenum

vyāpī kathaṁ mahādeva sarvatrāvasthitaḥ katham \veg\dontdisplaylinenum
\varr{
        \ \vd °sthitaḥ\lem  \msCb\msNc\Ed; °sthitaṁ \msCa\msNa, °sthita \msNb}

maheśvara uvāca~{\dandab}\dontdisplaylinenum 
\varr{
        \ \vo maheśvara\lem  \msCa\msCb\msNb\msNc; mahādeva \msNa, bhagavān \Ed}

asaṁśayo mahādevi vyāpī sarvagataḥ śivaḥ\thinspace{\danda} \dontdisplaylinenum

d\textsubring{r}śyatendriyasaṁyogāj jīvapratyayadarśanam \veg\dontdisplaylinenum
\varr{
        \ \vc d\textsubring{r}śyate°\lem  \msCb\msNa\msNb\msNc; d\textsubring{r}śyete° \msCa, d\textsubring{r}śyante \Ed
        \ \vd °jīva°\lem  \msCa\msCb\msNa\msNb\Ed; °jī° \msNc}

yathākāśasthito vāyuḥ śabdasparśaguṇānvitaḥ\thinspace{\dandab} \dontdisplaylinenum

tadvad dehī vijānīyād guṇaceṣṭena nānyathā \veg\dontdisplaylinenum
\varr{
        \ \vcd vāyuḥ śabda°\lem  \msCb\msNa\msNb\msNc\Ed; vāyu\uncl{śśa}{\il} \msCa
        \ \vd °nvitaḥ\lem  \msCa\msNa\msNb\msNc\Ed; °nvitam \msCb
        \ \vd °ceṣṭena\lem  \msCa\msCb\msNa\msNb; °veṣṭana \msNc, °veṣṭena \Ed}

devy uvāca~{\dandab}\dontdisplaylinenum 

vyāpīti kathitaḥ pūrvaṁ jīvaḥ sarvagato 'pi ca\thinspace{\danda} \dontdisplaylinenum

taṁ v\textsubring{r}thā kathito 'sy adya mriyate kena hetunā \veg\dontdisplaylinenum
\varr{
        \ \va kathitaḥ\lem  \msCa\msNa\msNcpcorr\Ed; kathitaṁ \msCb\msNb, kathatiḥ \msNcacorr
        \ \vc v\textsubring{r}thā\lem  \msCa\msCb\msNa\msNb\Ed; vyathā \msNc\oo
                 'sy adya\lem  \msCa\msCb\msNc; smy adya \msNa\Ed, sy a{\il} \msNb}

īśvara uvāca~{\dandab}\dontdisplaylinenum 
\varr{
        \ \vo īśvara\lem  \msCa\msCb\msNb\msNc; bhagavān \msNa\Ed}

na jīvo mriyate devi sarveṣāṁ surasundari\thinspace{\danda} \dontdisplaylinenum

ghaṭāntastho yathākāśo bahirākāśavad yathā \veg\dontdisplaylinenum

ghaṭabhinne viśālākṣi viśeṣo nopalakṣyate\thinspace{\dandab} \dontdisplaylinenum

dehabhinne yadā devi vināśo nopalabhyate \veg\dontdisplaylinenum
            \paral{\textit{\vo {\normalfont cf.\ Bhāgavatapurāṇa 12.5.5: }
                         ghaṭe bhinne ghaṭākāśa ākāśaḥ syād yathā purā{\thinspace\danda}
                         evaṁ dehe m\textsubring{r}te jīvo brahma sampadyate punaḥ{\thinspace\ketdanda}}}
\varr{
        \ \vb nopalakṣyate\lem  \msCa\msCb\msNb\msNc\Ed; nopalabhyate \msNa
        \ \vc deha°\lem  \msCa\msNa\msNb\msNc\Ed; dehe \msCb\oo
                 yadā devi\lem  \msCa\msCb\msNa\msNb\msNc; tathā dehī \Ed}

susūkṣmaḥ sarvago vyāpī paramātmānam avyayaḥ\thinspace{\dandab} \dontdisplaylinenum

bahir antaś ca bhūtānām acaraś cara eva saḥ \veg\dontdisplaylinenum
\varr{
        \ \va susūkṣmaḥ\lem  \msCa\msCb\msNa\msNb; susūkṣma \msNc, sa sūkṣmaḥ \Ed
        \ \vd °caraś ca°\lem  \msCa\msCb\msNa\msNb\msNc; °caran ca° \Ed\oo
                 saḥ\lem  \msCa\msCb\msNa\msNb\msNc; sa \Ed}

aprameyo 'vināśī ca aprapañcaḥ prapañcakaḥ\thinspace{\dandab} \dontdisplaylinenum

sarvendriyaguṇābhāsaḥ sarvendriyavivarjitaḥ \veg\dontdisplaylinenum
\varr{
        \ \vab (aprameyo{\normalfont ...} prapañcakaḥ)\lem  \msCa\msCb\msNa\msNc\Ed; \om\ \msNb}

evam eṣa mahādevi jīvasya varavarṇini\thinspace{\dandab} \dontdisplaylinenum

kathito 'smi samāsena kim anyac chrotum icchasi \veg\dontdisplaylinenum


\alalfejezet{sāraśreṣṭham}\varr{
        \ \vd icchasi\lem  \msCb\msNa\msNb\msNc\Ed; icchati \msCa}

devy uvāca~{\dandab}\dontdisplaylinenum 

sāraśreṣṭhaṁ mahādeva kathayeśāna īśvara\thinspace{\danda} \dontdisplaylinenum

śrotum icchāmi deveśa mānuṣāṇāṁ hitaṁ vada \veg\dontdisplaylinenum
\varr{
        \ \va °śreṣṭhaṁ\lem  \msCb\msNa\Ed; °śreṣṭha \msCa\msNb\msNc  
        \ \vd vada\lem  \msCa\msCb\msNa\msNb; vadaḥ \msNc\Ed}

īśvara uvāca~{\dandab}\dontdisplaylinenum 
\varr{
        \ \vo īśvara\lem  \msCa\msCb\msNa\msNb\msNc; bhagavān \Ed}

āśramāṇāṁ g\textsubring{r}hī śreṣṭho varṇaśreṣṭhā dvijātayaḥ\thinspace{\danda} \dontdisplaylinenum

aśvamedhaḥ kratuśreṣṭho japaśreṣṭho 'ghamarṣaṇaḥ \veg\dontdisplaylinenum
\varr{
        \ \va āśramāṇāṁ\lem  \msCa\msCb\msNa\msNc\Ed; āśramāṇā \msNb\oo
                 g\textsubring{r}hī\lem  \msCb\msNa\msNb\msNc\Ed; g\textsubring{r}\uncl{hī} \msCa
        \ \vb °śreṣṭhā\lem  \msCa\msCb\msNc; °śreṣṭo \msNa\msNb\Ed
        \ \vd japa°\lem  \msCapcorr\msNa\msNb\msNc\Ed; ja° \msCaacorr, 'japa° \msCb\oo
                 'ghamarṣaṇaḥ\lem  \msCb\msNa\msNb\msNc\Ed; rghamarṣaṇaḥ \msCa}

devatānāṁ hariḥ śreṣṭhaḥ śreṣṭhā gaṅgā nadīṣu ca\thinspace{\dandab} \dontdisplaylinenum

anāśanas tapaḥśreṣṭhas tīrthaśreṣṭhaḥ surahradaḥ \veg\dontdisplaylinenum
\varr{
        \ \vab śreṣṭhaḥ śreṣṭhā gaṅgā\lem  \msCa\msNa\msNb\msNc\Ed; śreṣṭhā gaṅgāṇāñ ca \msCb
        \ \vc anāśana°\lem  \msCa\msCb\msNa\msNb\Ed; anaśana° \msNc
        \ \vd °rthaśreṣṭhaḥ\lem  \msCa\msCb\msNa\msNb\Ed; °rthaśreṣṭha \msNc\oo
                 °hradaḥ\lem  \msCa\msNa\msNb\msNc\Ed; °drahaḥ \msCb}

kṣomaṁ vastreṣu ca śreṣṭhaṁ yaśaḥ śreṣṭhaṁ vibhūṣaṇam\thinspace{\dandab} \dontdisplaylinenum

bhārataṁ śrutiṣu śreṣṭhaṁ vrataśreṣṭho dayāparaḥ \veg\dontdisplaylinenum
\varr{
        \ \va kṣaumaṁ\lem  \msNc\Ed; kṣomaṁ \msCa\msCb\msNa, kṣoma \msNb
        \ \vb śreṣṭhaṁ\lem  \msCa\msCb\msNa\msNc\Ed; śreṣṭha \msNb\oo
                 °bhūṣaṇam\lem  \msCa\msNa\msNb\msNc\Ed; °bhūṣiṇam \msCb
        \ \vd °śreṣṭho\lem  \msCa\msCb\msNa\msNc\Ed; śreṣṭhaṁ \msNb\oo
                 dayāparaḥ\lem  \msCb\msNa\msNb\msNc\Ed; \uncl{dayāpa}raḥ \msCa}

dāneṣu cābhayaṁ śreṣṭhaṁ manaḥ śreṣṭhendriyeṣu ca\thinspace{\dandab} \dontdisplaylinenum

vidyā saṁgrahaṣu śreṣṭhā satyaṁ śreṣṭhaṁ vacaḥsu ca \veg\dontdisplaylinenum
\varr{
        \ \vc saṁgrahaṣu\lem  \msCa\msCb\msNa\msNb\Ed; saṁgraheṣu \msNc\ \unmetr\oo
                 śreṣṭhā\lem  \msCa\msCb\msNa\msNb\msNc; śreṣṭho \Ed }

āyudhānāṁ dhanuḥ śreṣṭhaṁ bāndhaveṣu ca mātaraḥ\thinspace{\dandab} \dontdisplaylinenum

jñānam auṣadhiṣu śreṣṭhaṁ vaidyaśreṣṭhaḥ śivākṣaraḥ \veg\dontdisplaylinenum
\varr{
        \ \va śreṣṭhaṁ\lem  \msCa\msCb\msNa\msNc\Ed; śreṣṭha \msNb
        \ \vb bāndhaveṣu ca mātaraḥ\lem  \msCa\msCb\msNaacorr\msNc\Ed; bāndhaveṣu ca mātaraṁ \msNapcorr,
                grahaśreṣṭho divākaraḥ \eyeskip{to 15.24b} \msNb
        \ \vcd (jñāna°{\normalfont ...} śivākṣaraḥ)\lem  \msCa\msCb\msNa\msNc\Ed; \om\ \msNb
        \ \vc jñānam oṣadhiṣu\lem  \msNc; jñānam auṣadhiṣu \msCa\msCb\msNa\msNb\Ed
        \ \vd vaidya°\lem  \msCa\msCb\msNa; \om\ \msNb, vaidyaḥ \msNc, vaidyo \Ed\oo
                 °śreṣṭhaḥ\lem  \msCb\msNa\msNc\Ed; °śreṣṭha \msCa, \om\ \msNb}

akāraś cākṣaraḥ śreṣṭho dharmaśreṣṭho hy ahiṁsakaḥ\thinspace{\dandab} \dontdisplaylinenum

paśuṣu saurabhī śreṣṭhā nareṣu ca narādhipaḥ \veg\dontdisplaylinenum
\varr{
        \ \vo (akāra°{\normalfont ...} narādhipaḥ)\lem  \msCa\msCb\msNa\msNc\Ed; \om\ \msNb}

māsi mārgaśiraḥ śreṣṭhaṁ k\textsubring{r}taḥ śreṣṭhaś caturyuge\thinspace{\dandab} \dontdisplaylinenum

vasanta \textsubring{r}tuṣu śreṣṭhaḥ śreṣṭhaṁ cāyanam uttaram \veg\dontdisplaylinenum
\varr{
        \ \vo (māsi{\normalfont ...} cāyanam uttaram)\lem  \msCa\msCb\msNa\msNc\Ed; \om\ \msNb
        \ \va māsi\lem  \msCa\msCb\msNa\msNc; \om\ \msNb, māsī \Ed\oo
                 °śiraḥ\lem  \msCa\msCb\msNa\msNb\Ed; °śira \msNc
        \ \vb śreṣṭhaś caturyuge\lem  \msCa\msNa\Ed; śreṣṭhaṁ caturyuge \msCb, \om\ \msNb, śreṣṭhaś caryuge \msNc
        \ \vd śreṣṭhaṁ cā°\lem  \msCa\msCb\msNc\Ed; śreṣṭhaś cā° \msNa, \om\ \msNb\oo
                 °ttaram\lem  \msCa\msNa\msNc\Ed; °tta\uncl{me}m \msCb, \om\ \msNb}

amāvāsyā dinaśreṣṭhā grahaśreṣṭho divākaraḥ\thinspace{\dandab} \dontdisplaylinenum

strīṣu lakṣmīr dh\textsubring{r}tiḥ śreṣṭhā vasuśreṣṭho hutāśanaḥ \veg\dontdisplaylinenum
\varr{
        \ \va amāvāsyā dinaśreṣṭhā\lem  \msCa\msCb\msNc\Ed; \om\ \msNb, amāvāsyā dinaśreṣṭho \msNa
        \ \vb grahaśreṣṭho divākaraḥ\lem  \msCa\msCb\msNa\msNb; grahaḥ śreṣṭho divākaraḥ \msNc, vasuśreṣṭho hutāśanaḥ \Ed
        \ \vcd (strīṣu{\normalfont ...} hutāśanaḥ)\lem  \msCa\msCb\msNa\msNb\msNc; \om\ \Ed
        \ \vc strīṣu\lem  \msCa\msNa\msNb\msNc; strī \msCb, \om\ \Ed\oo
                 lakṣmīr dh\textsubring{r}tiḥ\lem  \msCa; lakṣmīdh\textsubring{r}tiḥ \msCb\msNa\msNb\msNc, \om\ \Ed}

\textsubring{r}ṣiṣu uṣaṇā śreṣṭhaḥ kāntiśreṣṭho niśākaraḥ\thinspace{\dandab} \dontdisplaylinenum

nakṣatreṣv abhijit śreṣṭhaḥ kālaḥ śreṣṭhaḥ kaleṣu ca  \veg\dontdisplaylinenum
\varr{
        \ \va uṣaṇā\lem  \corr; uśanāḥ \msCa\msCb\msNa\msNb\msNc, uśanaḥ \Ed
        \ \vb kānti°\lem  \msCb\msNa\msNb\Ed; kāntiḥ \msNc, kā{\il} \msCa
        \ \vc °bhijit śre°\lem  \Ed; °bhijiḥ śre° \msCa\msCb\msNa\msNbpcorr\msNc, °bhiji \msNbacorr
        \ \vd kālaḥ\lem  \msCa\msCb\msNa\msNb\msNc; kaliḥ \Ed}

vedeṣu ca varaṁ sāma sthāvareṣu himālayaḥ\thinspace{\dandab} \dontdisplaylinenum

aśvattho vaṭa v\textsubring{r}kṣeṣu bhūteṣu vara cetanaḥ \veg\dontdisplaylinenum
\varr{
        \ \vc vaṭa\lem  \msCa\msCb\msNa\msNb; vara \msNc\Ed
        \ \vd vara cetanaḥ\lem  \msCb\Ed; varaś cetanaḥ \msCa\msNa\msNc\ \unmetr, vaś cetanaḥ \msNb}

adhyātma sarvavidyāsu vākya satya vara sm\textsubring{r}taḥ\thinspace{\dandab} \dontdisplaylinenum

prahlādo vara daityeṣu yakṣarakṣo dhaneśvaraḥ \veg\dontdisplaylinenum
\varr{
        \ \va adhyātma\lem  \msCb\msNb\Ed; adhyātmā \msCa\msNc, ādhyātmaṁ \msNa\oo
                 sarvavidyāsu\lem  \msCa\msNa\msNb\msNc; sarvavidyānāṁ \msCb, varavidyāsu \Ed
        \ \vb vākya\lem  \msCb; vāhu \msCa\msNa\msNb\msNc, vācaḥ \Ed\oo
                 vara\lem  \msCa\msCb\Ed; va\uncl{ra}ḥ \msNa, varaḥ \msNb\msNc
        \ \vc prahlādo\lem  \msCa\msCb\msNa\Ed; prahrādo \msNb\msNc
        \ \vd °śvaraḥ\lem  \msCa\msCb\msNa\msNc\Ed; °śvara \msNb}

marīcir vara vāteṣu hariḥ śreṣṭho m\textsubring{r}geṣu ca\thinspace{\dandab} \dontdisplaylinenum

sādhya nārāyaṇaḥ śreṣṭhaḥ pit\textsubring{\=r}ṇāṁ ca pitāmahaḥ \veg\dontdisplaylinenum
\varr{
        \ \va marīcir vara\lem  \msNc; marīci vara \msCb\msNa\msNb\Ed, ma{\il}{\il}{\il}{\il} \msCa
        \ \vb hariḥ\lem  \msCa\msCb\msNb\msNc\Ed; hari \msNa}

etat samāsato devi kathito 'si varānane\thinspace{\dandab} \dontdisplaylinenum

sarvasāraṁ samuddh\textsubring{r}tya kiṁ bhūyaḥ kathayāmy aham \veg\dontdisplaylinenum


\jump
\begin{center}
\ketdanda iti v\textsubring{r}ṣasārasaṁgrahe jīvanirṇayo nāmādhyāyaḥ pañcadaśamaḥ\ketdanda
\end{center}
\dontdisplaylinenum\vers 
\varr{
        \ \vb 'si\lem  \msCa\msCb\msNa\msNb; smi \msNc\Ed
        \ \vd kiṁ\lem  \msCb\msNa\msNb\msNc\Ed; ki \msCa
        \ {\normalfont \Colo:} nāmādhyāyaḥ pañcadaśamaḥ\lem  \msCa\msCb\msNa; nāmādhyāyaḥ pañcamaḥ \msNb,
                                nāmādhyāyaḥ pañcadaśama \msNc, nāma pañcadaśo 'dhyāyaḥ \Ed}
\bekveg\szamveg\vfill\phpspagebreak\szam\bek\versno=0\fejno=16
\thispagestyle{empty}



\alfejezet{\textbf{ṣoḍaśamo 'dhyāyaḥ}}\jump\jump

\alalfejezet{yogasadbhāvanirṇayaḥ}
\vers

devy uvāca~{\dandab}\dontdisplaylinenum 

adhunā śrotum icchāmi yogasadbhāvanirṇayam\thinspace{\danda} \dontdisplaylinenum
            \paral{\textit{\vo {\normalfont   \msCa\ 435.jpg line 2; 
                      \msCb\ 448.jpg line 2;
                      This chapter is missing in \msCc.
                      \msNa\ 220.jpg lower image line 5; 
                      \msNb\ 65.jpg upper image line 6;
                      \msNc\ f. 235r line 3}}}

karaṇaṁ ca yathānyāyaṁ kathayasva sureśvara \veg\dontdisplaylinenum
\varr{
        \ \vb °sadbhāva°\lem  \msCa\msCb\msNa\msNb\Ed; °saṁdbhāva° \msNc\oo
                 °nirṇayam\lem  \msCa\msCb\msNa\msNb\msNc; °nirṇayaḥ \Ed
        \ \vc karaṇaṁ ca\lem  \msCa\msCb\msNa\msNb\msNc; karaṇaś ca \Ed
        \ \vd sureśvara\lem  \msCa\msCb\msNa\msNb; sureśvare \msNc, sureśvaraḥ \Ed}

īśvara uvāca~{\dandab}\dontdisplaylinenum 
\varr{
        \ \vo īśvara\lem  \msCa\msCb\msNb\msNc; sureśa \msNa, bhagavān \Ed}

ś\textsubring{r}ṇu devi pravakṣyāmi yogasadbhāvam uttamam\thinspace{\danda} \dontdisplaylinenum

yaṁ viditvā na paśyanti janāḥ saṁsārabandhanam \veg\dontdisplaylinenum
\varr{
        \ \vb °m uttamam\lem  \msCa\msCb\msNa\msNc\Ed; °nirṇayam \msNb}

brahmahā gurutalpī vā surāpasteya eva vā\thinspace{\dandab} \dontdisplaylinenum

athavā saṁkare jātas tat sarvam apanodati \veg\dontdisplaylinenum
\varr{
        \ \vb vā\lem  \msCb\msNa\msNb\msNc; \uncl{vā} \msCa, ca \Ed
        \ \vc saṁkare\lem  \msNa\msNc; ś\textsubring{r}ṅkare \msCa, śaṅkare \msCb\Ed, \uncl{śaṁ}kare \msNb
        \ \vd tat sarva°\lem  \msCa\msCb\msNa\msNb\msNc; tasarva° \Ed}

muhūrtārdhe muhūrte vā prāṇāyāmaparāyaṇaḥ\thinspace{\dandab} \dontdisplaylinenum
            \paral{\textit{\vo {\normalfont  cf.\ 16.10. }}}

dhyeyaṁ cintayamānasya tatpāpaṁ kṣīyate narāt \veg\dontdisplaylinenum
            \paral{\textit{\vo {\normalfont  \kb\ a citation in Kauṇḍinya's commentary ad \PS\ 5.24: }
                muhūrtārdhaṁ muhūrtaṁ vā prāṇāyāmāntare 'pi vā{\thinspace\danda} 
                dhyeyaṁ cintayamānas tu pāpaṁ kṣapayate naraḥ{\thinspace\ketdanda} }}
\varr{
        \ \va muhūrtārdhe muhūrte vā\lem  \msCa\msNb\msNc; muhūrtārdhe vā \msCb,
                muhūrttārddha muhūrte vā \msNa, muhūrtārdhamuhūrtaṁ ca \Ed
        \ \vc dhyeyaṁ\lem  \msCa\msNb\msNc; dheyañ \msCb, dhyeya \msNa\Ed
        \ \vd narāt\lem  \msCb\msNa\msNc; narān \msCa\msNb\Ed}

na yamo nāntakaḥ kruddho na m\textsubring{r}tyur bhīmavigrahaḥ\thinspace{\dandab} \dontdisplaylinenum
                     \paral{\textit{\vb \kb\ {\normalfont  MBh 12.289.25ab: } na yamo nāntakaḥ kruddho na m\textsubring{r}tyur bhīmavikramaḥ}}

nāviśanti mahātmāno yogino balavattarāḥ \veg\dontdisplaylinenum
\varr{
        \ \vb m\textsubring{r}tyur bhī°\lem  \msCa\msCb\Ed; m\textsubring{r}tyu bhī° \msNa\msNb\msNc\oo 
                 bhīmavigrahaḥ\lem  \msCa\msCb\msNa\msNb\msNc; nāpavigrahaḥ \Ed
        \ \vc nāviśanti\lem  \msCa\msCb\msNa\msNb\msNc; viśanti sma \Ed
        \ \vd balavattarāḥ\lem  \msCa\msCb\msNa\msNb\msNc; varavattarā \Ed}

yathā vai sarvadhātūnāṁ doṣā dahyanti dhāmyatām\thinspace{\dandab} \dontdisplaylinenum

tathā pāpāḥ pradahyante dhruvaṁ prāṇasya nigrahāt \veg\dontdisplaylinenum
            \paral{\textit{\vo  {\normalfont  \kb\ Bhaviṣyapurāṇa 1.145.9: }
                                dhyāyamānasya dahyante cānte doṣā yathāgninā{\thinspace\danda}
                                tathendriyak\textsubring{r}tā doṣā dahyante prāṇanigrahāt{\thinspace\ketdanda}
                    {\normalfont  \kb\ Gheraṇḍasaṁhitā (ed.\ Thomi) 4.11: }
                                yathā parvatadhātūnāṁ doṣā dahyanti dhāmyatām{\thinspace\danda}
                                tathendriyak\textsubring{r}tā doṣā dahyante prāṇanigrahāt{\thinspace\ketdanda}}}
\varr{
        \ \va °dhātūnāṁ\lem  \msCa\msCb\msNa\msNb\Ed; °dhātīnāṁ \msNcacorr, °dhāt\textsubring{r}nāṁ \msNcpcorr
        \ \vb doṣā dahyanti\lem  \msNb\msNc; \uncl{doṣāṁ dahya}nti \msCa, doṣāṁ dahyanti \msCb\msNa\Ed
        \ \vc pāpāḥ\lem  \msCa\msCb\msNa\msNb\msNc; pāpaḥ \Ed
        \ \vd nigrahāt\lem  \msCa\msCb\msNa\msNc\Ed; nigrahān \msNb}

aśvamedhasahasraṁ ca rājasūyaśataṁ tathā\thinspace{\dandab} \dontdisplaylinenum

prāṇāyāmaśataṁ caiva na tattulyaṁ kadācana \veg\dontdisplaylinenum
\varr{
        \ \vd kadācana\lem  \msCa\msCb\msNa\msNbpcorr\msNc\Ed; kadāca \msNbacorr}

yajñena devān āpnoti rājyaṁ vai tapasaḥ phalam\thinspace{\dandab} \dontdisplaylinenum
            \paral{\textit{\vo \kb\ {\normalfont  Agnipurāṇa 378.1: } 
                yajñaiś ca devān āpnoti vairājaṁ(?)\ tapasā padam{\thinspace\danda}
                brahmaṇaḥ karmasannyāsād vairāgyāt prak\textsubring{r}tau layam{\thinspace\ketdanda}
                \kb\ {\normalfont  Maskarin's commentary CHECK ad Gautamadharmasūtra 3.1: }
                        yajñena devān āpnoti vairājaṁ(!)\ tapasā punaḥ{\thinspace\danda}
                        saṁnyāsād brahmaṇaḥ sthānaṁ vairāgyāt prak\textsubring{r}tau layam{\thinspace\ketdanda}}}

saṁnyāsād brahmaṇaḥ sthānaṁ vairāgyāt prak\textsubring{r}tau layam \veg\dontdisplaylinenum
                     \paral{\textit{\vcd {\normalfont cf.\ 11.27ab above: } sāvitrī prak\textsubring{r}tau layaḥ}}
\varr{
        \ \va devān āpnoti\lem  \msCa\msCb\msNa\msNbpcorr\msNc\Ed; devāpnoti \msNbacorr
        \ \vc °d brahmaṇaḥ\lem  \msCa\msCb\msNa\msNc\Ed; °d brāhmaṇaḥ \msNb
        \ \vd vairāgyāt prak\textsubring{r}tau layam\lem  \eme; vairāgyāt prak\textsubring{r}tālayam \msCa\msNa\msNb\msNc\Ed,
                                                        mahātmāno prak\textsubring{r}tālayam \msCb\ ({\normalfont eyeskip to 16.5c?})}

jñānāt prāpnoti kaivalyaṁ paraṁ brahma sanātanam\thinspace{\dandab} \dontdisplaylinenum

ity etā gatayaḥ pañca vidhivat parikīrtitāḥ \veg\dontdisplaylinenum
\varr{
        \ \vb paraṁ\lem  \msCa\msCb\msNa\msNc\Ed; para° \msNb}

muhūrtārdhaṁ muhūrtaṁ vā yogaṁ yuñjīta yogavit\thinspace{\dandab} \dontdisplaylinenum
            \paral{\textit{\vo {\normalfont  cf.\ 16.4. }}}

nistaret sarvapāpāni am\textsubring{r}tatvaṁ ca gacchati \veg\dontdisplaylinenum
\varr{
        \ \va muhūrtārdhaṁ muhūrtaṁ vā\lem  \msCa\msCb\msNb;
                         muhūrtārddha muhū\uncl{rttaṁ} vā \msNa, 
                         muhūtārthaṁ muhūrttam vā \msNc, 
                         muhūrtārdha muhūrtaṁ vā \Ed
        \ \vb yogaṁ\lem  \msCa\msCb\msNa\msNc\Ed; yoga \msNb\oo
                 yogavit\lem  \msCa\msCb\msNa\msNb\Ed; yovit \msNc
        \ \vc nistaret sa°\lem  \msCb\msNa\msNb\Ed; vistaret sa° \msCa, \uncl{ni}stareṇat sa° \msNc
        \ \vd am\textsubring{r}tatvaṁ\lem  \msCa\msCb\msNa\msNc\Ed; am\textsubring{r}tatva \msNb}

yuñjāno 'pi prayatnena yāvat tattvaṁ na vindati\thinspace{\dandab} \dontdisplaylinenum

brahmaloke dhruvaṁ vāso viṣṇuloke ca sundari \veg\dontdisplaylinenum
\varr{
        \ \vb yāvat tattvaṁ na vindati\lem  \msNa\msNc\Ed;
                yāvantanna vindati \unmetr\ \msCa, yāva tatvaṁ na vindati \msCb, yāvattaṁn na vindati \msNb
        \ \vc brahmaloke\lem  \msCa\msCb\msNa\msNc\Ed; brahmaloko \msNb\oo
                 vāso\lem  \msCa\msNa\msNb\msNc\Ed; vāsvā \msCb}

bhuktvā karmasahasrāṇi sarvakāmasamanvitaḥ\thinspace{\dandab} \dontdisplaylinenum

kṣīṇapuṇyas tato martye jāyate vipule kule \veg\dontdisplaylinenum
\varr{
        \ \vc °puṇyas ta°\lem  \msNa\msNb\msNc\Ed; °puṇye ta° \msCa\msCb\oo
                 martye\lem  \msCa\msCb\msNa\msNb\msNc; martyāṁ \Ed}

yogam evābhiseveta pūrvajātismaro naraḥ\thinspace{\dandab} \dontdisplaylinenum

saṁsārārṇavam uttīrya sa śivatvam avāpnuyāt \veg\dontdisplaylinenum


\alalfejezet{yogavidhiḥ}
devy uvāca~{\dandab}\dontdisplaylinenum 

yogasya vidhim icchāmi śrotuṁ me puruṣottama\thinspace{\danda} \dontdisplaylinenum

dhyānadhāraṇasiddhīnāṁ kathayasva sureśvara \veg\dontdisplaylinenum
\varr{
        \ \vb śrotuṁ me\lem  \msCa\msCb\msNc\Ed; śrotuṁ vai \msNa, śrotu me \msNb\oo
                 °ttama\lem  \msCa\msCb\msNa\msNb\msNc; °ttamaḥ \Ed
        \ \vc °siddhīnāṁ\lem  \msCa\msCb\msNa\msNb\msNc; °siddhānāṁ \Ed
        \ \vd sureśvara\lem  \msCa\msCb\msNa\msNb\msNc; sureśvaraḥ \Ed}

maheśvara uvāca~{\dandab}\dontdisplaylinenum 
\varr{
        \ \va maheśvara\lem  \msCa\msCb\msNa\msNb\msNc; bhagavān \Ed}

ś\textsubring{r}ṇu yogavidhiṁ vakṣye bhavapāśanik\textsubring{r}ntanam\thinspace{\danda} \dontdisplaylinenum

śucir ekāgracittas tu janaśabdavivarjite \danda\dontdisplaylinenum

tatrāsīnāsane yogī paramātmāna cintayet \veg\dontdisplaylinenum
\varr{
        \ \vd °cittas tu\lem  \msCa\msNa\msNb\msNc\Ed; °cittasyastu \msCb\oo  
                 jana°\lem  \msCa\msCb\msNa\msNb\msNc; dhyāna° \Ed\oo
                 °vivarjite\lem  \msNa; °vivarjitaḥ \msCa\msCb\msNb\msNc, °vivarjitam \Ed            
        \ \vf °tmāna cintayet\lem  \msCb\msNa\msNcpcorr; °tmānaṁ cintayet \msCa\Ed\ \unmetr, °tmānā vicintayet \msNb,
                                               °tmāna cintayā \msNcacorr}

padmakaṁ svastikaṁ caiva niṣkalam añjalis tathā\thinspace{\dandab} \dontdisplaylinenum

ardhacandraṁ ca daṇḍaṁ ca paryaṅkaṁ bhadram eva ca \veg\dontdisplaylinenum
            \paral{\textit{\vo {\normalfont  cf.\ Sarvajñānottara 27:9cd--10ab: } 
                padmakaṁ svastikaṁ vāpi upasthāñjalikaṁ tathā{\thinspace\ketdanda} 
                pīṭhārdham ardhacandraṁ vā sarvatobhadram eva vā{\thinspace\danda}}}
\varr{
        \ \va padmakaṁ\lem  \msCb\msNa\msNb\msNc\Ed; padmaka \msCa
        \ \vb niṣkalam añjalis tathā\lem  \msCa\msCb\msNb\msNc; niṣkalaṁ mañjalis tathā \msNa,
                                                                niṣkalamakañjalintathā \Ed
        \ \vd paryaṅkaṁ\lem  \msCb\msNa\msNb\msNc\Ed; pa{\il}ṅkaṁ \msCa}

etadāsanabandhena baddhvā yogaṁ samabhyaset\thinspace{\dandab} \dontdisplaylinenum

samaṁ kāyaśirogrīvaṁ dhārayann acalasthitaḥ \veg\dontdisplaylinenum
            \paral{\textit{\vcd \kb\ {\normalfont  MBh 6.28.13ab (BhG 6.13ab): } samaṁ kāyaśirogrīvaṁ dhārayann acalaṁ sthiraḥ}}
\varr{
        \ \vb baddhvā yogaṁ\lem  \msCa\msCb\msNa\msNb\msNc; baddhā yoga \Ed
        \ \vc samaṁ\lem  \msCa\msCb\msNa\Ed; sama° \msNb\msNc}

pratyāhāras tathā dhyānaṁ prāṇāyāmaś ca dhāraṇā\thinspace{\dandab} \dontdisplaylinenum

tarkaś caiva samādhiś ca ṣaḍaṅgo yoga ucyate \veg\dontdisplaylinenum
         \paral{\textit{\vo {\normalfont  = Dharmaputrikā 1.13 (with } prāṇāyāmo 'tha {\normalfont  ) }
             {\normalfont  cf.\ Sarvajñānottarav\textsubring{r}tti ad Yogapāda 27(?):1: } 
             yad uktaṁ śrīmanmataṅge{\thinspace\danda}
                prāṇāyāmas tathā dhyānaṁ pratyāhāro 'tha dhāraṇam{\thinspace\danda}
                tarkaś caiva samādhiś ca ṣaḍaṅgo yoga uccyate iti{\thinspace\ketdanda}}}
\varr{
        \ \va pratyāhāras ta°\lem  \msCa\msCb\msNa\msNb\msNc; pratyahāras ta° \Ed\oo
                 dhyānaṁ\lem  \msCa\msCb\msNb\msNc\Ed; dhyāna \msNa
        \ \vb prāṇāyāmaś ca\lem  \msCa\msCb\msNa\msNb\msNc; prāṇāyāmañ ca \Ed}

viṣayāsaktacittānām indriyāṇāṁ prati prati\thinspace{\dandab} \dontdisplaylinenum

manasākarṣayed yas tu pratyāhāraḥ sa ucyate \veg\dontdisplaylinenum
            \paral{\textit{\vo \kb\ {\normalfont  Dharmaputrikā 1.14: } 
                viṣayeṣv atisaktāni indriyāṇi prati prati{\thinspace\danda}
                cittenākarṣaṇaṁ yatra pratyāhāraḥ sa ucyate{\thinspace\ketdanda}}}
\varr{
        \ \vb prati prati\lem  \msCb\msNa\msNb; pratisrati \msCa, prati pati \msNc, pratiṣṭhati \Ed
        \ \vc manasā°\lem  \msCa\msCb\msNa\msNb\msNc; manamā° \Ed
        \ \vd pratyāhāraḥ sa\lem  \msCa\msCb\msNa\msNb\msNc; pratyahāras tad \Ed}

śabdādiviṣayān devi vartulīk\textsubring{r}tya dhārayet\thinspace{\dandab} \dontdisplaylinenum
            \paral{\textit{\vo {\normalfont  cf.\ Dharmaputrikā 1.16cd: 
                } ekatra vartulīk\textsubring{r}tya dhyeye vastuni yojayet }}

vītarāgaḥ samādhistho dhyeye vastuni yojayet \veg\dontdisplaylinenum
\varr{
        \ \va °viṣayān de°\lem  \msCa\msNa\msNb\msNc\Ed; viṣayā de° \msCb
        \ \vc vītarāgaḥ\lem  \msCa\msNb\msNc; vītarāga° \msCb\msNa\Ed
        \ \vd dhyeye vastuni\lem  \msCb\msNa; dhyeyastuni \msCa, dhyeyavastuni \msNb\msNc\Ed}

ātmā dhyātā mano dhyānaṁ dhyeyaḥ śuddhaḥ paraḥ śivaḥ\thinspace{\dandab} \dontdisplaylinenum
            \paral{\textit{\vo \kb\ {\normalfont Sarvajñānottara Yogapāda 27(?):4: }
               ātmā dhyātā mano dhyānaṁ dhyeyaḥ sūkṣmo maheśvaraḥ{\thinspace\danda}
               yat paraṁ paramaiśvaryam etad dhyānaprayojanam{\thinspace\ketdanda}
               \vab \kb\ {\normalfont  Agnipurāṇa 165.22cd: }
                ātmā dhyātā mano dhyānaṁ dhyeyo viṣṇuḥ phalaṁ hariḥ
                {\normalfont  \kb\ Dharmaputrikā 1.18: }
                dhyeyaḥ śivo dhyāt\textsubring{r} mano dhyānam ekāgracittatā{\thinspace\danda}
                duḥkhahānir guṇaiśvaryaṁ svātantryañ ca prayojanam{\thinspace\ketdanda}}}

yat paraṁ paramaiśvaryam ekaṁ tatra prayojanam \veg\dontdisplaylinenum
\varr{
        \ \va ātmā\lem  \msCa\msCb\msNa\msNb\msNc; ātma \Ed\oo
                 dhyātā\lem  \msCa\msCb\msNa\msNc\Ed; dhyātaṁ \msNb
        \ \vb paraḥ śivaḥ\lem  \msCb\msNa\msNb\msNc; paraśivaḥ \msCa\Ed\ \unmetr
        \ \vc paramai°\lem  \msCa\msCb\msNa\msNb\msNcpcorr\Ed; parame° \msNcacorr
        \ \vcd °m ekaṁ tatra\lem  \msCa\msCb\msNa\msNc\Ed; °m etat tatra \msNb}

pūrakaḥ kumbhakaś caiva recakas tadanantaram\thinspace{\dandab} \dontdisplaylinenum
            \paral{\textit{\vo = {\normalfont  Dharmaputrikā 1.19ab (with } caiva{\normalfont  for } ceti{\normalfont  )} }}

praśāntaś ceti vikhyātaḥ prāṇāyāmaś caturvidhaḥ \veg\dontdisplaylinenum
            \paral{\textit{\vcd {\normalfont  See NiśvāsaNaya 4:113: }
                nābhyāṁ h\textsubring{r}dayasaṁcārān manaś cendriyagocarāt{\thinspace\danda}
                prāṇāyāmaś caturthas tu supraśāntas tu viśrutaḥ{\thinspace\ketdanda} 
                {\normalfont  See also Svaccandatantra 7.298ab: }
                prāṇāyāmaś caturthas tu supraśānta iti śrutaḥ}} 
\varr{
        \ \vc praśānta°\lem  \msCb\msNa\msNb\msNc\Ed; \uncl{pra}śānta° \msCa\oo
                 vikhyātaḥ\lem  \msCa\msCb\msNb\msNc\Ed; vikhyātāḥ \msNa
        \ \vd °vidhaḥ\lem  \msCa\msCb\msNa\msNb\msNc; °vidhāḥ \Ed}

pūrake sthāpayed vahniṁ pādāṅguṣṭhena buddhimān\thinspace{\dandab} \dontdisplaylinenum

kumbhakena virudhyeta dahyamānaṁ vicintayet \veg\dontdisplaylinenum
\varr{
        \ \va pūrake\lem  \eme; pūrakaḥ \msCa\msCb\msNa\msNb\msNc\Ed\oo
                 vahniṁ\lem  \msCa\msCb\msNa\msNc;  vahni \msNb\Ed
        \ \vb °ṣṭhena\lem  \msCa\msCb\msNb\msNc\Ed; °ṣṭheṣu \msNa
        \ \vcd virudhyeta dahyamānaṁ\lem  \msCa\msCb\msNa\msNc; nirudhyeta dahyamānam \msNb,
                                                                               nirudhyeta daihyamāna \Ed}

bhasmībhūtaṁ tathātmānaṁ recakena vicintayet\thinspace{\dandab} \dontdisplaylinenum

śuddhadehas tataś cātmā śuddhasphaṭikanirmalaḥ \veg\dontdisplaylinenum
\varr{
        \ \va °bhūtaṁ\lem  \msCa\msCb\msNa\msNb\Ed; °bhūta \msNc
        \ \vc °dehas tataś cātmā\lem  \msCa\msCb\msNa\msNb\Ed; °dehaś cataś cātmā \msNc}

tālaśabdās tu nirvāṇaṁ daśa dve ca prakīrtitaḥ\thinspace{\dandab} \dontdisplaylinenum

prāṇāyāmān na saṁdeho dviguṇā dhāraṇā sm\textsubring{r}tā \veg\dontdisplaylinenum
\varr{
        \ \va tālaśabdās tu\lem  \eme; tālāśabdas tu \msCa\msCb\msNa\msNb\msNcacorr\Ed, 
                                                        tālaśabdastustu \msNcpcorr\oo
                 nirvāṇaṁ\lem  \msCa\msCb\msNc\Ed; nirvāṇa \msNa, \uncl{nirvvā}ṇaṁ \msNb
        \ \vc prāṇāyāmān na\lem  \msCa\msNa\msNb\msNc\Ed; prāṇāyān na \msCb
        \ \vd sm\textsubring{r}tā\lem  \msCa\msCb\msNa\msNb\msNc; sm\textsubring{r}tāḥ \Ed}

yoge tu triguṇā proktā saṁkrame ca caturguṇā\thinspace{\dandab} \dontdisplaylinenum

tathotkrāntau pañcaguṇā yogasiddhis tu ṣaḍguṇā \veg\dontdisplaylinenum
\varr{
        \ \va °guṇā\lem  \msCa\msCb\msNb\msNc\Ed; °guṇāḥ \msNa
        \ \vab proktā saṁkrame ca caturguṇā\lem  \msCa\msCb; 
                         proktāḥ saṁkrame ca caturguṇā \msNa,
                         proktāṁ saṁkrame ca caturguṇā \msNb,
                         proktā sakrame ca cartuguṇā \msNc,
                         proktāḥ saṁkrameṇa caturguṇāḥ \Ed
        \ \vc tathotkrāntau\lem  \msCa\msCb\msNa\msNb\msNc; tathākratau \Ed
        \ \vd ṣaḍguṇā\lem  \eme; ṣaḍguṇāḥ \msCa\msCb\msNa\msNb\Ed, ṣaḍaṅguṇāḥ \msNc}

ṣaḍaṅgena samāyukto yogayuktas tu nityaśaḥ\thinspace{\dandab} \dontdisplaylinenum

mānaso yaugapadyaś ca dvirūpo yoga ucyate \veg\dontdisplaylinenum
            \paral{\textit{\vcd = {\normalfont  Dharmaputrikā 1.54ab.} }}
\varr{
        \ \va ṣaḍaṅgena\lem  \msCa\msCb\msNa\msNb\Ed; sadaṅgena \msNc
        \ \vb yogayuktas tu\lem  \msCa\msCb\msNa\msNb\msNc; yogamuktas tu \Ed
        \ \vcd yaugapadyaś ca dvirūpo\lem  \msNa\msNc; 
                yaugapadya\uncl{ś ca} {\il}{i}{\il}{\il} \msCa,
                yogapadyaś ca dvirūpo \msCb\msNb,
                yogapadyañ ca dvirūpo \Ed}

ak\textsubring{r}tvā prāṇasaṁrodhaṁ manasaikena kevalam\thinspace{\dandab} \dontdisplaylinenum

dhyāyeta paramaṁ sūkṣmaṁ sa yogo mānasaḥ sm\textsubring{r}taḥ \veg\dontdisplaylinenum
            \paral{\textit{\vo = {\normalfont  Dharmaputrikā 1.54cd--55ab.} }}
\varr{
        \ \va °saṁrodhaṁ\lem  \msCa\msNb\msNc\Ed; °saṁrodha \msCb\msNa
        \ \vb manasaikena\lem  \msCb\msNb\msNc\Ed; manasekena \msCa\msNa
        \ \vc dhyāyeta pa°\lem  \msCa\msCb\msNa\msNc\Ed; dhyāyetat pa \msNb\oo
                 paramaṁ\lem  \msCa\msCb\msNa\msNb\Ed; parama \msNc
        \ \vd sa yogo\lem  \msCa\msNa\msNb\msNc\Ed; saṁyogo \msCb\oo
                 mānasaḥ\lem  \msCapcorr\msCb\msNb\msNc\Ed; mānasa \msCaacorr\msNa\oo
                 sm\textsubring{r}taḥ\lem  \msCa\msCb\msNa\msNb\msNc; sm\textsubring{r}tam \Ed}

saṁyamya manasā prāṇaṁ prāṇāyāmān manas tathā\thinspace{\dandab} \dontdisplaylinenum

evaṁ dhyāyet paraṁ sūkṣmaṁ yaugapadyaḥ sa ucyate \veg\dontdisplaylinenum
            \paral{\textit{\vo \kb\ {\normalfont  Dharmaputrikā 1.55cd--56ab: }
                saṁyamya manasā prāṇaṁ prāṇāyāmair manas tathā{\thinspace\danda}
                evaṁ dhyāyet paraṁ sūkṣmaṁ yaugapadyaḥ sa ucyate{\thinspace\ketdanda}}}


\alalfejezet{siddhilakṣaṇam}\varr{
        \ \va saṁyamya\lem  \msCa\msCb\msNa\msNc\Ed; sayamya \msNb\oo
                 prāṇaṁ\lem  \msCa\msNa\msNb\msNc\Ed; \om\ \msCb
        \ \vb prāṇāyāmān ma°\lem  \eme; prāṇāyāmām ma° \msCa\msNb, prāṇāyāmā ma° \msCb,
                                prāṇāyāmaṁ ma° \msNa, prāṇāyāmāṁ ma° \msNc, prāṇāyāmātma° \Ed
        \ \vb yaugapadyaḥ\lem  \msCa\msCb\msNc\Ed; yogapadyaḥ \msNa, yogapadya \msNb}

siddhilakṣaṇa yogasya ś\textsubring{r}ṇu vakṣyāmi sundari\thinspace{\dandab} \dontdisplaylinenum

śaṅkhabherīm\textsubring{r}daṅgaṁ ca veṇudundubhim eva ca \danda\dontdisplaylinenum

tāḍitaṁ na ca vindeta yadā tanmayatāṁ gataḥ \veg\dontdisplaylinenum
            \paral{\textit{\vo \kb\ {\normalfont  Kulasāra f.\ 38r: }
                 śaṁkhabherīm\textsubring{r}daṁgaiś ca vīṇāveṇuśatair api{\thinspace\danda} 
                 tāḍyamānair na vindeta yadā tanmayatāṁ gataḥ{\thinspace\ketdanda}}}
            \paral{\textit{\vef {\normalfont  cf. NiśvāsaMukha 4:65: }
                tāḍitañ ca na vindeta cakṣuṣā na ca paśyati{\thinspace\danda}
                divyad\textsubring{r}ṣṭiḥ prajāyeta yadā tanmayatāṅ gataḥ{\thinspace\ketdanda}}}
\varr{
        \ \va siddhi°\lem  \msCa\msCb\msNa\msNb\msNc; siddhir \Ed
        \ \vc śaṅkhabherīm\textsubring{r}daṅgaṁ ca\lem  \msNb; śaṅkha{\il}{\il}{\il}{\il}{\il}ś ca \msCa, śaṅkhabherīm\textsubring{r}daṅgaś ca \msCb\msNa\msNc\Ed
        \ \vd °dundubhim eva\lem  \msCa\msCb\msNa\msNb\msNc; °dundubhir eva \Ed}

śītoṣṇaṁ sukhaduḥkhaṁ ca t\textsubring{r}ṣṇābhukṣaṁ tathaiva ca\thinspace{\dandab} \dontdisplaylinenum

vedanāṁ naiva jānāti yogasiddhas tu sundari \veg\dontdisplaylinenum
\varr{
        \ \vb t\textsubring{r}ṣṇābhukṣaṁ\lem  \msCa\msCb\msNa\msNb\msNc; t\textsubring{r}ḍbubhukṣāṁ \Ed
        \ \va vedanāṁ\lem  \msNa; vedanān \msCa\msCb, vedanā \msNb\Ed, vaidanā \msNc
        \ \vb °siddha°\lem  \msCa\msNa\msNc; °siddhi° \msCb\msNb, °yukta° \Ed}

eṣa yogavidhir devi tava p\textsubring{r}ṣṭena sundari\thinspace{\dandab} \dontdisplaylinenum

kathito 'smi samāsena kim anyat kathayāmy aham \veg\dontdisplaylinenum
\varr{
        \ \va °vidhir devi\lem  \msCa\msCb\msNa\msNb\Ed; °vidhin devi \msNc}

devy uvāca~{\dandab}\dontdisplaylinenum 

vinā yogena deveśa saṁsāratāraṇaṁ mama\thinspace{\danda} \dontdisplaylinenum

kathayasva mahādeva nirvikalpakaraṁ manaḥ \veg\dontdisplaylinenum
\varr{
        \ \va deveśa\lem  \msCb\msNa\msNb\msNc\Ed; veśa \msCa
        \ \vb saṁsāratāraṇaṁ mama\lem  \msCa\msCb\msNc\Ed; saṁsārāt tāraṇaṁ mama \msNa, saṁsārārṇṇavatāraṇa \msNb
        \ \vc mahādeva\lem  \msCa\msCb\msNa\msNb\Ed; sureśāna \msNc}

maheśvara uvāca~{\dandab}\dontdisplaylinenum 
\varr{
        \ \vo maheśvara\lem  \msCa\msCb\msNb\msNc; deveśa \msNa, bhagavān \Ed}

sadāśivas tu niśvāsa ūrdhvaśvāsaḥ paraḥ śivaḥ\thinspace{\danda} \dontdisplaylinenum

tayor madhye tu vijñeyaḥ paramātmā śivo 'vyayaḥ \veg\dontdisplaylinenum
\varr{
        \ \vd ūrdhvaśvāsaḥ\lem  \msCa\msCb\msNa\msNc; ūrdhvaśvāsa \msNb, arddhaśvāsaḥ \Ed}

dhyānayogaṁ na tasyāsti karaṇaṁ ca na vidyate\thinspace{\dandab} \dontdisplaylinenum

jñātamātreṇa mucyante kim anyat parip\textsubring{r}cchasi \veg\dontdisplaylinenum


\alalfejezet{pañca śāstrāṇi}\varr{
        \ \vc jñāta°\lem  \msCa\msCb\msNa\msNb; jñāna° \msNc\Ed
        \ \vcd mucyante kim anyat pa°\lem  \msCb\msNa\msNc\Ed; mucya\uncl{nte}{\il}m anyat pa° \msCa,
                                                \uncl{mucya}nte kim at pa° \msNb}

jñānam anyat pravakṣyāmi ś\textsubring{r}ṇu devi nibodha me\thinspace{\dandab} \dontdisplaylinenum

śāstrapañcasu yat proktaṁ ś\textsubring{r}ṇu saṁkṣepa nirṇayam \danda\dontdisplaylinenum

sāṁkhye yoge pañcarātre śaive vede ca nirmitam \veg\dontdisplaylinenum
\varr{
        \ \vd saṁkṣepa\lem  \msCb\msNa\msNb\msNc\Ed; saṁkṣepe \msCa\ \unmetr
        \ \ve sāṁkhye\lem  \msCa\msCb\msNa\msNb\msNc; sāṁkhya° \Ed\oo
                 pañca°\lem  \msCa\msCb\msNb\msNc\Ed; paca° \msNa
        \ \vf śaive\lem  \msCa\msCb\msNa\msNb\msNc; śaiva° \Ed}

\ujvers\nemsloka 
yat sāṁkhyasiddhaṁ kathayāmy ahaṁ te
\dontdisplaylinenum

\nemslokab 
saṁsāraghorārṇavayogasāram \danda\dontdisplaylinenum

\nemslokac 
yogeṣu sāreṣv atha pañcarātre
\dontdisplaylinenum

\nemslokad 
vedeṣu śaiveṣu ca niścayas te \veg\dontdisplaylinenum
\varr{
        \ \va °siddhaṁ\lem  \msCa\msCb\msNa\msNc\Ed; °siddhiṁ \msNb\oo
                 te\lem  \msCa\msCb\msNapcorr\msNb\msNc\Ed; \om\ \msNaacorr
        \ \vb °rṇava°\lem  \msCa\msCb\msNb\msNc\Ed; °ṇṇa° \msNaacorr, °ṇṇava° \msNapcorr\oo 
                 °sāram\lem  \msCa\msNa\msNb\msNc\Ed; sāgaram \msCb
        \ \vc °ṣv atha\lem  \msCa\msNa\msNb\msNc\Ed; °ṣv etha \msCb\oo
                 pañcarātre\lem  \msCb\msNa\msNb\msNc\Ed; pañca\uncl{rātre} \msCa
        \ \vd vedeṣu\lem  \msCb\msNa\msNc\Ed; {\il}deṣu \msCa, deveṣu \msNb\oo
                 niścayas te\lem  \msCa\msNc; niścayan te \msCb\Ed,
                                         niścayās te \msNa, niścaya\uncl{sve} \msNb}

\ujvers\nemsloka 
ghrāṇendriyādyeṣu ca yat samastam
\dontdisplaylinenum

\nemslokab 
manaś ca līnaṁ bhavatīva yasya \danda\dontdisplaylinenum

\nemslokac 
buddhyā niyamya sakalān hi bhāvān
\dontdisplaylinenum

\nemslokad 
sa labdhalakṣyaḥ śivam abhyupaiti \veg\dontdisplaylinenum
\varr{
        \ \vb manaś ca\lem  \msCa\msCb\msNa\msNb\msNc; nabhaś ca \Ed
        \ \vc sakalān hi\lem  \corr; sakalāṁ hi \msCa\msNa\msNb\msNc, sakalā hi \msCb, śakalāṁ hi \Ed
        \ \vd °lakṣyaḥ\lem  \msCa\msCb\msNb; °lakṣya° \msNa\Ed, °lakṣa° \msNc\oo
                 °paiti\lem  \msCa\msNa\msNb\Ed; °peti \msCb\msNc}

\ujvers\nemsloka 
śrotrādisarvendriyaniścalatve
\dontdisplaylinenum

\nemslokab 
ekāgracittaṁ manasā niyamya \danda\dontdisplaylinenum

\nemslokac 
svadehaśūnyaḥ sa bhavec cireṇa
\dontdisplaylinenum

\nemslokad 
saṁyogasiddhiṁ pravadanti tajjñāḥ \veg\dontdisplaylinenum

\nemslokalong

\varr{
        \ \va śrotrā°\lem  \msCa\msCb\msNa\msNb\Ed; śrotā° \msNc\oo
                 °calatve\lem  \emeHaru; °calatvam \msCa\msCb\msNa\msNb\msNc\Ed
        \ \vc °śūnyaḥ\lem  \msCa\msNa\msNb\msNc\Ed; °śūnyaṁ \msCb
        \ \vd saṁyogasiddhiṁ\lem  \msNa; saṁyogasi{\il} \msCa, saṁgasiddhiṁ \msCb, sa yogasiddhiṁ \msNb\Ed,
                                                       saṁyogasiddhaṁ \msNc}

\ujvers\nemsloka 
ādāv eva manaḥ śanair uparamet k\textsubring{r}tvā ca vaśyendriyaṁ
\dontdisplaylinenum

\nemslokab 
yāvat tal layatāṁ vrajeta manasā niḥsaṁjñadehas tathā \danda\dontdisplaylinenum

\nemslokac 
etad dhyānasamādhiyogasakalaṁ prāpnoti niḥsaṁśayaṁ
\dontdisplaylinenum

\nemslokad 
kiṁ tac chāstrasahasrakoṭipaṭhitaṁ sāraṁ na yo 'nviṣyati \veg\dontdisplaylinenum
\varr{
        \ \va uparamet k\textsubring{r}°\lem  \msCa\msCb\msNb\msNc\Ed; uparame k\textsubring{r}° \msNa\oo
                 °ndriyam\lem  \msCa\msNa\msNb\msNc\Ed; °ndriyaḥ \msCb
        \ \vb tallayatāṁ\lem  \msCa\msCb\msNa\msNb\msNc; tattapatāṁ \Ed\oo
                 manasā niḥsaṁjña°\lem  \Ed; manasān nissaṁjña° \msCa, manasāṁ niḥsaṁjña° \msCb, manasān nisaṁjña° \msNa,
                                                manasān nissajña° \msNb, manasān niḥsajña° \msNc
        \ \vc °sakalaṁ\lem  \msCa\msCb\msNa\msNb\Ed; °sakala \msNc\oo
                 niḥsaṁśayam\lem  \msCa\msNb\Ed; niḥsaṁśayaḥ \msCb\msNa, nisaṁśayaṁ \msNc
        \ \vd kiṁ ta°\lem  \msCa\msCb\msNa\msNb\msNc; citsa° \Ed\oo
                 °koṭi°\lem  \msCb\msNa\msNb\msNc\Ed; °ṭoki° \msCa\oo
                 °paṭhitaṁ\lem  \msCa\msCb\msNa\msNb\msNc; °mathitaṁ \Ed\oo 
                 na yo 'nviṣyati\lem  \msCa\msCb\msNc; na yo 'nviṣyate \msNa\msNb, tayer iṣyati \Ed}

\ujvers\nemsloka 
ātmārāmajitaḥ samādhinirato vairāgyam apy āśritaḥ
\dontdisplaylinenum

\nemslokab 
cittaṁ yasya parikṣayo yadi bhavet tiṣṭhet tanutvaṁ yathā \danda\dontdisplaylinenum

\nemslokac 
taj jñeyaṁ gatim uttamaṁ śivapadaṁ saṁsāraduḥkhacchidaṁ
\dontdisplaylinenum
            \paral{\textit{\vc {\normalfont  cf.\ 22.41d: } uttamāṁ gatim āpnuyāt}}

\nemslokad 
vedānteṣu ca niṣṭha eṣa kathitaḥ kiṁ śāstram anyad viśet \veg\dontdisplaylinenum
\varr{
        \ \va ātmārāmajitaḥ\lem  \msCb\msNa\msNb\msNc; ātmārā\uncl{ma}{\il}{\il} \msCa, ātmārāmaḥ jitaḥ \Ed\oo
                 vairāgyam apy āśritaḥ\lem  \msCa\msCb\msNa\msNb; vairāgya yasyāśritaḥ \msNc,
                                                                            vairāgaśayyāśritaḥ \Ed
        \ \vb pari°\lem  \msCa\msCb\msNb\msNc\Ed; parī° \msNa\oo
                 °ṣṭhet ta°\lem  \msCa\msCb\msNa\msNb\Ed; °ṣṭhan ta° \msNc
        \ \vd anyad vi°\lem  \msCa\msCb\msNb\msNc\Ed; anyaṁ vi° \msNa}

\ujvers\nemsloka 
h\textsubring{r}tpadme karṇikāyām upari ravir avadyotayanto 'ntarālam
\dontdisplaylinenum

\nemslokab 
yattejastejamārgair bahalatamaghanair dyotanād dīptadīpam \danda\dontdisplaylinenum

\nemslokac 
bhittvā yat tāludeśe mukham uparigataṁ tāludeśena mūrdhni
\dontdisplaylinenum

\nemslokad 
! mūrdhni dvārāntareṇa śivaparamapadaṁ yānti yogena yuktāḥ \veg\dontdisplaylinenum
\varr{
        \ \va °padme\lem  \conj; °padma° \msCa\msCb\msNa\msNb\msNc\Ed\unmetr\oo
                 ravir avadyotayanto\lem  \eme; raviravaṁdyotayanto \msCa\msCb\msNa\msNb,
                                 ravir iva dyontayanto \msNc, raviratadyotayanto \Ed
        \ \vb yat te°\lem  \msCb; yas te° \msCa\msNa\msNb\Ed, saste° \msNc\oo
                 °mārgair bahala°\lem  \msCa\msCb\msNb; °mārgai bahala° \msNa, °mārge bahula° \msNc, °mārgau bahula° \Ed\oo
                 °tamaghanair dyotanād dīptadīpam\lem  \conj; 
                                         °tamaghanair ghātanād dīptadīpam \msCa, 
                                         °maghanai ghāṭanādīptadīpam \msCb,     
                                         °tamaghanair ghāṭanādīptadīpam \msNa, 
                                         °tamaghanai ghāṭanādīptadīpam \msNb, 
                                         °tamaghanair dyoṭanā dīptadīpaṁ \msNc,
                                         °tamaghanair dyotanād dīptadīpaḥ \Ed
        \ \vc yat tālu°\lem  \Ed; ghaṁṭṭāla° \msCa, ghatola° \msCb ghaṇṭāla° \msNa\msNb, dyaṇṭāla° \msNc\oo
                 °gataṁ\lem  \msNc\Ed; °gata° \msCa\msNa\msNb, °gatas° \msCb
        \ \vd mūrdhni\lem  \msNa; mūrdhna \msCa\msCb\msNb, mūrddha° \msNc, mūrdhnyā \Ed}

\ujvers\nemsloka 
k\textsubring{r}ṣṇaḥ k\textsubring{r}ṣṇatamottamo 'timahato yas tejatejātmakaḥ
\dontdisplaylinenum

\nemslokab 
lokālokadharādharaḥ śriyapatiḥ prāṇapraviṣṭālayaḥ \danda\dontdisplaylinenum

\nemslokac 
kartā kāraṇam avyayo 'vyayam asau vyāpī vibhaktāvidam
\dontdisplaylinenum

\nemslokad 
viṣṇur bhāvamayo vibhaktaviṣayair viśveśvaro viśvavit \veg\dontdisplaylinenum
\varr{
        \ \va k\textsubring{r}ṣṇaḥ\lem  \emeKafle; k\textsubring{r}ṣṇaṁ \msCa\msCb\msNa\msNb\msNc, k\textsubring{r}tsnaṁ \Ed\oo
                 °tamottamo\lem  \conj; °tamotamo \msCa\msCb\msNa\msNb\msNc\Ed\oo
                 'ti°\lem  \msCa\msNa\msNb\msNc\Ed; hi \msCb\oo
                 yas tejate°\lem  \Ed; yas tejaste° \msCa\msCb\msNa\msNb\msNc\ \unmetr
        \ \vb °dharādharaḥ śriyapatiḥ\lem  \Ed; 
                      °dharo dharādharadharaḥ \msCa\msCb\msNb\msNc,
                      °dharo dharadharadharaḥ \msNa\ \unmetr\oo
                 śriyapatiḥ\lem  \msCa\msCb\msNa\Ed; \om\ \msNb\msNc\oo
                 prāṇa°\lem  \msCa\msCb\msNa\msNb\Ed; prāṇaḥ \msNc\oo
                 °praviṣṭālayaḥ\lem  \msCb\msNa\msNb\msNc; °\uncl{pra}viṣṭo layaḥ \msCa, pratiṣṭhālayaḥ \Ed
        \ \vd bhāvamayo\lem  \msCa\msCb\msNa\msNb\msNc; bhāvamayair \Ed\oo
                 viśvavit\lem  \msCa\msCb\msNa\msNb\Ed; viśvavat \msNc}

\ujvers\nemsloka 
! eṣa tattvavaraḥ parāparamayas tejaḥ parasthānadaḥ
\dontdisplaylinenum

\nemslokab 
buddhyā bhāvanabhāvayendriyamano dehāntar ālokayan \danda\dontdisplaylinenum

\nemslokac 
h\textsubring{r}tpadmāyatanasthitaḥ sa puruṣo niśvāsam ucchvāsadaḥ
\dontdisplaylinenum

\nemslokad 
nādas tasya sadā sadā nadati taṁ nādopariṣṭhā haraḥ \veg\dontdisplaylinenum
\varr{
        \ \va parāparamayas te°\lem  \conj; paraḥ paramayas te° \msCa\msNa\msNb\Ed, paraḥ paramas te° \msCb, 
                                                                paraḥ paramayes te° \msNc\oo
                 °parasthānadaḥ\lem  \conj; °paraḥ sthānadaḥ \msCa\msCb\msNa\msNb\msNcpcorr\Ed, °paraḥ sthānada \msNcacorr
        \ \vb °bhāvayendriyamano\lem  \msCa\msNa\msNb; °bhāvayandriyamano \msCb,
                                °bhāva\uncl{va}yendriyamano \msNc, °bhāvayan niyamano \Ed\oo
                 dehāntar ālokayan\lem  \msCa\msNa\msNb\msNc; dehāntarālokayat \msCb, dehāntarostokayan \Ed
        \ \vc sa puruṣo ni°\lem  \msNa\msNb\msNc\Ed; \uncl{sa puruṣo} {\il} \msCa, puruṣau ni° \msCb\oo
                °cchvāsadaḥ\lem  \msCa\msCb\msNa\msNb\msNc; °cchvāsadām \Ed
        \ \vd nādas tasya\lem  \msCa\msCb\msNa\msNb\msNc; nādantasya \Ed\oo
                 nadati taṁ\lem  \msCa\msCb\msNa\msNb\msNc; na patitaṁ \Ed\oo
                 °pariṣṭhā haraḥ\lem  \msCa\msCb\msNa\msNb; °pariṣṭhāraddharaḥ \msNc, °pariṣṭadvaraḥ \Ed}

\ujvers\nemsloka 
yas tejas tejate 'jo bahuniviḍaghano granthimālopagūḍhaḥ
\dontdisplaylinenum

\nemslokab 
mūrtir mūrtānusārī bahukaraṇabh\textsubring{r}taṁ kāraṇād dehabandhaḥ \danda\dontdisplaylinenum

\nemslokac 
bhittvā granthiṁ sapāśaṁ viṣam iva viṣayaṁ tyaktasaṅgaikabhāvāḥ
\dontdisplaylinenum

\nemslokad 
paśyanty ete tam īśaṁ guṇakalarahitaṁ nirvikāraṁ prakāśam \veg\dontdisplaylinenum
\varr{
        \ \va yas tejas tejate 'jo\lem  \conj;
                        yas tejas tejas tejo \msCa\msCb\msNa\msNb\msNc\ \unmetr\ 
                        yas tejas tejaso vā \Ed\oo
                 °niviḍa°\lem  \msCa\msCb\msNa\msNb\msNc; °nividu° \Ed\oo
                 °ghano\lem  \msNc\msCb; °ghanaḥ \msCa\msNa\msNb\Ed\oo
                 granthimālo°\lem  \msCa\msNa\msNb\msNc; gratthimāno° \msCb\Ed
        \ \vb mūrtir mūrtā°\lem  \msCa; mūrtimūrtā° \msCb\msNa\msNb\msNc, mūrtir mūrtya° \Ed\oo
                 bahukaraṇa°\lem  \msCa\msNa\msNb\msNcpcorr\Ed; bahukara° \msNcacorr, bahyakaraṇa° \msCb\unmetr\oo
                 °bh\textsubring{r}taṁ\lem  \msCa\msCb\msNa\msNcpcorr\Ed; °v\textsubring{r}taṁ \msNb, °bh\textsubring{r}ta \msNcacorr\oo
                 kāraṇād dehabandhaḥ\lem  \msCa\msCb\msNa\msNb; kāraṇād dehabandha \msNc, kāraṇaṁ dehabandhaḥ \Ed
        \ \vc granthiṁ\lem  \msCa\msCb\msNa\msNb\Ed; granthi \msNc\oo
                 sapāśaṁ\lem  \msNa\Ed; sapāśāṁ \msCa\msCb\msNb\msNcpcorr, sapāśā \msNcacorr\oo
                 °saṅgaika°\lem  \msCb\msNa\msNb\msNc\Ed; °saṅsaika° \msCa
        \ \vb paśyanty ete tam ī°\lem  \msCb\msNa\msNb; paśyanty e{\il}{\il}m ī° \msCa, 
                                paśyantete tam ī° \msNc, paśyanty etenam ī° \Ed}

\ujvers\nemsloka 
yo 'sau tejāntarātmā kamalapuṭakuṭīsaṁkaṭasthānalīnaḥ
\dontdisplaylinenum

\nemslokab 
indor bhāsānurūpī vimaladalasadācchāditaḥ karṇikāyām \danda\dontdisplaylinenum

\nemslokac 
tatra sthāne sthito 'sau tribhuvananilayaḥ sarvabhūtādhivāsaḥ
\dontdisplaylinenum

\nemslokad 
ākāśād ūrdhvatattvasthitavikasakalāsaṁhato muktabandhaḥ \veg\dontdisplaylinenum

\nemslokanormal

\varr{
        \ \va (yo{\normalfont ...} °līnaḥ)\lem  \msCapcorr\msCb\msNa\msNb\msNc\Ed; \om\ \msCaacorr\oo
                 yo 'sau tejāntarātmā\lem  \msCb\msNa\msNc\Ed; {\il}{\il}{\il}\uncl{jānta}rātmā \msCapcorr, \om\ \msCaacorr, 
                                                yo sau tejāntarāla° \msNb\oo
                °kuṭī°\lem  \msCapcorr\msCb\msNa\msNb\msNc; \om\ \msCaacorr, °kuṭi° \Ed
        \ \vb indor bhāsānu°\lem  \msCa\msNb\msNcpcorr\Ed; indo bhāsānu° \msCb\msNa, 
                                indor bhāsānurūpī vimalaḥ indor bhāsānu° \msNcacorr\oo 
                 °rūpī\lem  \msCa\msCb\msNa\msNb\msNc; °rūpi \Ed\ \unmetr\oo
                 °cchāditaḥ\lem  \msCa\msCb\msNa\msNbpcorr\msNc\Ed; °cchādi \msNbacorr
        \ \vd ākāśād ūrdhva°\lem  \msCa\msCb\msNa\msNb\Ed; ākāśād dūrdhva° \msNc\oo
                 °sthita°\lem  \conj; °sita° \msCa\msCb\msNa\msNb\Ed\ \unmetr, °sisita° \msNc\ \unmetr\oo 
                 °kalāsaṁhato\lem  \Ed; °kasāsaṁhato \msCa\msCb\msNa\msNb\msNc\oo
                 mukta°\lem  \conj; mukti° \msCa\msCb\msNa\msNb\msNc\Ed}

\ujvers\nemsloka 
etāni tattvāny akhilāni devi
\dontdisplaylinenum

\nemslokab 
saṁkṣepataḥ kīrtitaḥ pañcabhedaḥ \danda\dontdisplaylinenum

\nemslokac 
śrotuṁ kim anyad vijigīṣitārtham
\dontdisplaylinenum

\nemslokad 
saṁsāramokṣeṇa ca tatparo 'sti \veg\dontdisplaylinenum

\vers
\varr{
        \ \va °khilāni\lem  \msCa\msNa\msNb\msNc\Ed; °khikāti \msCb\oo 
                 devi\lem  \msCb\msNa\msNb\msNc\Ed; \uncl{de}{\il} \msCa
        \ \vc śrotuṁ kim\lem  \msCa\msCb\msNa\msNb\msNc; śrotakim \Ed\oo
                 vijigīṣitā°\lem  \msCa\msNa\msNb\msNc\Ed; vijigīṣatā° \msCb}

devy uvāca~{\dandab}\dontdisplaylinenum 

\nemsloka 
tuṣṭāsmi deva mama saṁśayam adya naṣṭam
\dontdisplaylinenum

\nemslokab 
adya prasannaparameśvara īśvara tvam \danda\dontdisplaylinenum

\nemslokac 
adya śrutaṁ tvayi ca puṇyaphalaprabhāvam
\dontdisplaylinenum

\nemslokad 
pūrṇāni cādya mama iṣṭamanorathāni \veg\dontdisplaylinenum
\varr{
        \ \va tuṣṭā°\lem  \msCa\msNa\msNb\msNc; tu\uncl{ṣṭā}° \msCb, tuṣṭo \Ed
        \ \vb °parameśvara\lem  \msCa\msCb\msNa\msNc\Ed; °parameraśvara \msNb\oo
                 īśvara\lem  \msCa\msCb\msNa\msNb\msNc; īśvarama \Ed
        \ \vc (adya{\normalfont ...} °prabhāvam)\lem  \msCa\msNa\msNb\msNc\Ed; \om\ \msCb
        \ \vd iṣṭamanorathāni\lem  \msCb\msNa\msNb\msNc\Ed; \uncl{iṣṭa}{\il}{\il}{\il}thāni \msCa}

\ujvers\nemsloka 
ajñānapaṅkaghanamadhyanilīyamānām
\dontdisplaylinenum

\nemslokab 
uttārayeśa sakalārtivināśanāya \danda\dontdisplaylinenum

\nemslokac 
sarveśa tattvaparamārtha namo namas te
\dontdisplaylinenum

\nemslokad 
adyāpi t\textsubring{r}ptir iha nāsti mamāpi śambho \veg\dontdisplaylinenum
\varr{
        \ \va °nilīyamānām\lem  \msCa\msNa\msNc; °nilīyamānam \msCb\msNb\Ed
        \ \vb uttārayeśa\lem  \msCb\msNa\msNb\msNc\Ed; uttarāyeśa \msCaacorr, uttarayeśa \msCapcorr
        \ \vd nāsti mamāpi\lem  \msCa\msCb\msNapcorr\msNb\msNc\Ed; nā pi \msNaacorr}

\ujvers\nemsloka 
pītvām\textsubring{r}taṁ cottamavaktrajātam
\dontdisplaylinenum

\nemslokab 
ākhyāhi dānaṁ phaladharmasāram \danda\dontdisplaylinenum

\nemslokac 
saṁsārapāraṁ paramaṁ nayasva
\dontdisplaylinenum

\nemslokad 
k\textsubring{r}pāṁ mayīśāna kuru prasīda \veg\dontdisplaylinenum

\vers


\jump
\begin{center}
\ketdanda iti v\textsubring{r}ṣasārasaṁgrahe 'dhyātmanirṇayo nāmādhyāyaḥ ṣoḍaśamaḥ\ketdanda
\end{center}
\dontdisplaylinenum\vers 

\vers
\varr{
        \ \va °vaktra°\lem  \msCa\msCb\msNb\msNc\Ed; °vacaktra° \msNaacorr, °caktra° \msNapcorr
        \ \vd paramaṁ\lem  \msCa\msCb\msNa\msNc\Ed; parama \msNb\oo
                 nayasva\lem  \msCb\msNa\msNb\msNc\Ed; naya{\il} \msCa
        \ \vd k\textsubring{r}pāṁ mayīśāna kuru prasīda\lem  \Ed; \om\ \msCa\msCb\msNa\msNb\msNc
        \ {\normalfont \Colo:} 'dhyātma°\lem  \corr; adhyātma° \msCa\msNa\msNb\msNc\Ed, ātma° \msCb\oo 
                        nāmādhyāyaḥ ṣoḍaśamaḥ\lem  \msCa\msCb\msNa\msNb\msNc; 
                             nāma ṣoḍaśo 'dhyāyaḥ \Ed}
\bekveg\szamveg\vfill\phpspagebreak\szam\bek\versno=0\fejno=17
\thispagestyle{empty}



\alfejezet{\textbf{17 dānadharmaviśeṣaḥ}}\jump\jump
\vers

devy uvāca~{\dandab}\dontdisplaylinenum 

p\textsubring{r}thagdānasya icchāmi śrotuṁ māṁ dātum arhasi\thinspace{\danda} \dontdisplaylinenum

annavastrahiraṇyānāṁ gobhūmikanakasya ca \veg\dontdisplaylinenum
\varr{
        \ \vb śrotuṁ māṁ dātum arhasi\lem  \msCa; 
                māhātmyaṁ vaktum arhasi \Ed}

bhagavān uvāca~{\dandab}\dontdisplaylinenum 

\nemsloka 
! susaṁsk\textsubring{r}tam annam atipradadyāt
\dontdisplaylinenum

\nemslokab 
! gh\textsubring{r}taprabhūtam avadaṁśayuktam \danda\dontdisplaylinenum

\nemslokac 
gh\textsubring{r}taprapakvaṁ suk\textsubring{r}taṁ ca pūpaṁ
\dontdisplaylinenum

\nemslokad 
sitena khaṇḍena guḍena yuktam \veg\dontdisplaylinenum
\varr{
        \ \vc suk\textsubring{r}taṁ ca pūpaṁ\lem  \msCa; 
                        suk\textsubring{r}tammapūpaṁ \Ed}

\ujvers\nemsloka 
mārgaṁ khagaś codakajaṅgamaś ca
\dontdisplaylinenum

\nemslokab 
dadyād vaṭaṁ nāgaravaṁśamūlam \danda\dontdisplaylinenum

\nemslokac 
śākaṁ phalaṁ cāmlamadhūratiktam
\dontdisplaylinenum

\nemslokad 
pānaṁ payaḥ śītasugandhatoyam \veg\dontdisplaylinenum
\varr{
        \ \va mārgaṁ\lem  \msCa; mārga° \Ed\ \unmetr\oo
                 khagaś\lem  \Ed; khañ \msCa\oo
                 °jaṅgalaṁ ca\lem  \msCa; °jaṅgamaś ca \Ed
        \ \vb vaṭaṁ\lem  \msCa; vaṭa \Ed\ \unmetr}

\ujvers\nemsloka 
dadhi pradadyād guḍamiśritaṁ ca
\dontdisplaylinenum

\nemslokab 
m\textsubring{r}ṇālaśālūkavanālakā ca \danda\dontdisplaylinenum

\nemslokac 
sadakṣiṇālepapavitrapuṣpam
\dontdisplaylinenum

\nemslokad 
śraddhānvitaḥ satk\textsubring{r}tayā praṇamya \veg\dontdisplaylinenum
\varr{
        \ \vd satk\textsubring{r}tayā\lem  \msCa; saktatayā \Ed}

\ujvers\nemsloka 
prayāti lokaṁ jagadīśvarasya
\dontdisplaylinenum

\nemslokab 
vimānayānaiḥ sahito 'psarobhiḥ \danda\dontdisplaylinenum

\nemslokac 
ekaikasiṣṭasya sahasravarṣam
\dontdisplaylinenum

\nemslokad 
annaprado modati devaloke \veg\dontdisplaylinenum

\ujvers\nemsloka 
cyutaś ca martye sa bhaved dhanāḍhyaḥ
\dontdisplaylinenum

\nemslokab 
kulodgataḥ sarvaguṇopapannaḥ \danda\dontdisplaylinenum

\nemslokac 
yaśaḥ śriyaṁ sarvakalajñatā ca
\dontdisplaylinenum

\nemslokad 
bhavet sa bhogī sakalatraputraḥ \veg\dontdisplaylinenum

\ujvers\nemsloka 
dadyād daridraḥ k\textsubring{r}paṇārtadīno
\dontdisplaylinenum

\nemslokab 
bālāgadatvāturamāgatānām \danda\dontdisplaylinenum

\nemslokac 
t\textsubring{r}ṣṇābubhukṣāgatikāgatānām
\dontdisplaylinenum

\nemslokad 
dattvā sadharmasya phalaṁ kaniṣṭa \veg\dontdisplaylinenum

\ujvers\nemsloka 
vāṇijyadharmādiphalāśritānām
\dontdisplaylinenum

\nemslokab 
dharmo hi tasya na ca nirmalo 'sti \danda\dontdisplaylinenum

\nemslokac 
toyaṁ ca dadyāl laghupūrṇakambham
\dontdisplaylinenum

\nemslokad 
śītaṁ sugandhaṁ parivāritaṁ ca \veg\dontdisplaylinenum

\ujvers\nemsloka 
sa yāti lokaṁ salileśvarasya
\dontdisplaylinenum

\nemslokab 
na tasya janmānit\textsubring{r}ṣābhibhūtaḥ \danda\dontdisplaylinenum

\nemslokac 
upānahaṁ yo dadati dvijāya
\dontdisplaylinenum

\nemslokad 
suśobhanaṁ tailasudī surapitaṁ ca \veg\dontdisplaylinenum

\ujvers\nemsloka 
te yānti lokam amarādhipasya
\dontdisplaylinenum

\nemslokab 
yamālayaṁ kaṣṭapathāna yānti \danda\dontdisplaylinenum

\nemslokac 
prakṣīṇapuṇyā punar atra loke
\dontdisplaylinenum

\nemslokad 
jāto bhaved divyakulopapannaḥ \veg\dontdisplaylinenum

\ujvers\nemsloka 
dhanaiḥ sam\textsubring{r}ddhodhopatitvatāś ca
\dontdisplaylinenum

\nemslokab 
rathāś ca nāgā prabhavanti tasya \danda\dontdisplaylinenum

\nemslokac 
vastrapradānena bhavanti devi
\dontdisplaylinenum

\nemslokad 
rūpottamasarvakalajñatāṁ ca \veg\dontdisplaylinenum

\ujvers\nemsloka 
sam\textsubring{r}ddhisaubhāgyaguṇānvitāś ca
\dontdisplaylinenum

\nemslokab 
svargacyutās te puruṣā bhavanti \danda\dontdisplaylinenum

\nemslokac 
vastrapradānābhiratasya puṁsaḥ
\dontdisplaylinenum

\nemslokad 
anyat pravakṣyāmi tataḥ praśastām \veg\dontdisplaylinenum

\ujvers\nemsloka 
vastraṁ tu lokeṣv atipūjanīyam
\dontdisplaylinenum

\nemslokab 
vastraṁ narāṇāṁ tv atimānanīyam \danda\dontdisplaylinenum

\nemslokac 
vastraṁ tu bhūyo na ca mānalābhaḥ
\dontdisplaylinenum

\nemslokad 
parābhavaś cāti jugupsanaś ca \veg\dontdisplaylinenum

\ujvers\nemsloka 
tasmād dhi vastraṁ satataṁ pradeyam
\dontdisplaylinenum

\nemslokab 
yaśaḥ śriyaḥ svargasamāntalābham \danda\dontdisplaylinenum

\nemslokac 
yāvanti sūtrāṇi bhavanti vastre
\dontdisplaylinenum

\nemslokad 
tāvad yugaṁ gacchanti somalokam \veg\dontdisplaylinenum

\ujvers\nemsloka 
puṇyakṣayāj jāyati m\textsubring{r}tyuloke
\dontdisplaylinenum

\nemslokab 
vastraprabhūte dhanadhānyakīrṇo ? \danda\dontdisplaylinenum

\nemslokac 
surūpasaubhāgyayaśaśivanaś ca
\dontdisplaylinenum

\nemslokad 
vidyādharo lokaprabhutvatāś ca \veg\dontdisplaylinenum

\ujvers\nemsloka 
dvijebhyac chatraṁ suk\textsubring{r}taṁ pradadyāt
\dontdisplaylinenum

\nemslokab 
varṣātapatraṁ d\textsubring{r}ḍhaśobhanaṁ ca \danda\dontdisplaylinenum

\nemslokac 
aṅgāravarṣatraṣu khaḍgamādyam
\dontdisplaylinenum

\nemslokad 
asaṁśayaṁ trāyati yāmyamārge \veg\dontdisplaylinenum

\ujvers\nemsloka 
svargaṁ ca yānti grahanāyakaś ca
\dontdisplaylinenum

\nemslokab 
sa varṣakoṭyāyutam antakāle \danda\dontdisplaylinenum

\nemslokac 
jāyanti te mānuṣamartyaloke
\dontdisplaylinenum

\nemslokad 
g\textsubring{r}hottame bhogapatir bhavanti \veg\dontdisplaylinenum

\ujvers\nemsloka 
k\textsubring{r}tvā maṭhaṁ śobhanavipradātā
\dontdisplaylinenum

\nemslokab 
dravyeṇa śuddhena tu pūjayitvā \danda\dontdisplaylinenum

\nemslokac 
sa yāti devendrasadaṁ yatheṣṭam
\dontdisplaylinenum

\nemslokad 
savarṣakoṭiśatadivyasaṁkhyaiḥ \veg\dontdisplaylinenum

\ujvers\nemsloka 
tadantakāle yadi mānuṣatvam
\dontdisplaylinenum

\nemslokab 
jāyanti te saptamahīprabhoktā \danda\dontdisplaylinenum

\nemslokac 
sa saptarathyatrayasamprayuktā
\dontdisplaylinenum

\nemslokad 
balādhiko yajñasahasrakartā \veg\dontdisplaylinenum

\ujvers\nemsloka 
bhūmipradātā dvijahīnadīnam
\dontdisplaylinenum

\nemslokab 
saṁm\textsubring{r}ddhasasyo jalasaṁnik\textsubring{r}ṣta \danda\dontdisplaylinenum

\nemslokac 
sa yāti lokam amarādhipasya !
\dontdisplaylinenum

\nemslokad 
vimānayānena manohareṇa \veg\dontdisplaylinenum

\ujvers\nemsloka 
manvantaraṁ yāvad abhuktabhogān
\dontdisplaylinenum

\nemslokab 
tadantakāle cyutamartyaloke \danda\dontdisplaylinenum

\nemslokac 
sa javamukhaṇḍādhipatir bhavet
\dontdisplaylinenum

\nemslokad 
vīryānvito rājasahasranāthaḥ \veg\dontdisplaylinenum

\ujvers\nemsloka 
sa cailaghaṇṭāṁ kanakāgraś\textsubring{r}ṅgām
\dontdisplaylinenum

\nemslokab 
dogdhīṁ savatsāṁ payasāṁ dvijānām \danda\dontdisplaylinenum

\nemslokac 
dattvā dvijebhyaḥ samalaṅk\textsubring{r}tānām
\dontdisplaylinenum

\nemslokad 
prayānti lokaṁ surabhīsutānām \veg\dontdisplaylinenum

\ujvers\nemsloka 
yāvanti romāṇi bhavanti gāvaḥ
\dontdisplaylinenum

\nemslokab 
tāvad yugānām anubhūyabhogān \danda\dontdisplaylinenum

\nemslokac 
tasmāc cyutā martyamahībhujās te
\dontdisplaylinenum

\nemslokad 
sahasrarājānugato mahātmā \veg\dontdisplaylinenum
\varr{
        \ \va yāvanti\lem  \Ed; prayānti \msCa}

\ujvers\nemsloka 
suvarṇakāṁsyāyasaraupyadātā
\dontdisplaylinenum

\nemslokab 
tāmrapravālāmaṇimauktikādyān \danda\dontdisplaylinenum

\nemslokac 
dattvā dvijebhyo vasusādhyaloke
\dontdisplaylinenum

\nemslokad 
prāpnoti varṣaṁ daśapañcakoṭyo !  \veg\dontdisplaylinenum

\ujvers\nemsloka 
bhuktvā yatheṣṭaṁ kramadevalokān
\dontdisplaylinenum

\nemslokab 
cyutaṁ ca martye sa bhaven narendraḥ \danda\dontdisplaylinenum

\nemslokac 
sudurjayaḥ śakrasahasrajetā
\dontdisplaylinenum

\nemslokad 
sudīrgham āyuś ca parākramaś ca \veg\dontdisplaylinenum

\ujvers\nemsloka 
yat prekṣaṇaṁ darśayituṁ pradātā
\dontdisplaylinenum

\nemslokab 
surūpasaubhāgya phalaṁ labheta \danda\dontdisplaylinenum

\nemslokac 
t\textsubring{r}ṇāśanāmūlaphalāśanena
\dontdisplaylinenum

\nemslokad 
labheta rājyāni kaṇṭakāni \veg\dontdisplaylinenum

\ujvers\nemsloka 
labhetaparṇāśanasvargavāsam
\dontdisplaylinenum

\nemslokab 
payaḥ prayogena ca devaloke \danda\dontdisplaylinenum

\nemslokac 
śuśrūṣaṇo yo gurave ca nityam
\dontdisplaylinenum

\nemslokad 
vidyādharo jāyati martyaloke \veg\dontdisplaylinenum

\ujvers\nemsloka 
dadyād gavāṁ dhāsat\textsubring{r}ṇasya muṣṭiḥ
\dontdisplaylinenum

\nemslokab 
gavāḍhyatāṁ jāyati martyaloke \danda\dontdisplaylinenum

\nemslokac 
śrāddhaṁ ca dattvā prayato dvijāya
\dontdisplaylinenum

\nemslokad 
sam\textsubring{r}ddhasantāna bhaved yugānte \veg\dontdisplaylinenum

\ujvers\nemsloka 
ahiṁsako jāyati dīrgham āyuḥ
\dontdisplaylinenum

\nemslokab 
kulottamaṁ jāyati dīkṣitena \danda\dontdisplaylinenum

\nemslokac 
kālatrayaṁ snānak\textsubring{r}tena rājyaṁ
\dontdisplaylinenum

\nemslokad 
pītvā ca vāyus tridaśādhipatvam \veg\dontdisplaylinenum

\ujvers\nemsloka 
anaśnatāyāḥ phalam īśaloke
\dontdisplaylinenum

\nemslokab 
t\textsubring{r}ptir bhavet toyapradānaśīlaḥ \danda\dontdisplaylinenum

\nemslokac 
annapradātā puruṣaḥ sam\textsubring{r}ddhaḥ
\dontdisplaylinenum

\nemslokad 
sa sarvakāmā labhatīha loke \veg\dontdisplaylinenum

\ujvers\nemsloka 
śraddhāmatir yaḥ praviśed dhutāsanaṁ !
\dontdisplaylinenum

\nemslokab 
sa yāti lokaṁ prapitāmahasya \danda\dontdisplaylinenum

\nemslokac 
satyaṁ vaded yo 'pi ca dharmaśīlo
\dontdisplaylinenum

\nemslokad 
modaty asau devi sahāpsarobhiḥ \veg\dontdisplaylinenum

\ujvers\nemsloka 
rasās tu ṣaḍyo parivarjayanti
\dontdisplaylinenum

\nemslokab 
atīva saubhāgya labheta sādhvī \danda\dontdisplaylinenum

\nemslokac 
dānena bhogān atulyaṁ labheta
\dontdisplaylinenum

\nemslokad 
cirāyutāṁ yāti hi brahmacaryāt \veg\dontdisplaylinenum

\ujvers\nemsloka 
dhanāḍhyatāṁ yānti hi puṇyakarmān
\dontdisplaylinenum

\nemslokab 
maunena - ājñā labhate alaṅghyām \danda\dontdisplaylinenum

\nemslokac 
prāpnoti kāmaṁ tapasaḥ sutaptaṁ
\dontdisplaylinenum

\nemslokad 
kīrtir yaśaḥ svargam anantabhogam \veg\dontdisplaylinenum

\vers

āyuḥ śriyārogyadhanaprabhutvaṁ\thinspace{\dandab} \dontdisplaylinenum

jñānādilābhaṁ tapasā labheta \veg\dontdisplaylinenum

\ujvers\nemsloka 
trailokyādhipatitvaśakram agamat k\textsubring{r}tvā tapo duṣkaram
\dontdisplaylinenum

\nemslokab 
yakṣeśo 'pi tapaḥ prabhāvaguruṇā guhyādhipatvaṁ mahat \danda\dontdisplaylinenum

\nemslokac 
rakṣeśo 'pi bibhīṣaṇas tv amaratāṁ prāptas tapasyaiva tu
\dontdisplaylinenum

\nemslokad 
rudrārādhanatatparās tapaphalāt nandīgaṇatvaṁ gataḥ \veg\dontdisplaylinenum

\ujvers\nemsloka 
jñānaṁ dvijān tapaso āha viṣṇuḥ
\dontdisplaylinenum

\nemslokab 
kṣatraṁ taporakṣaṇam āha sūrya \danda\dontdisplaylinenum

\nemslokac 
vaiśyaṁ tapaś cāñjanam āha vāyuḥ
\dontdisplaylinenum

\nemslokad 
śūdraṁ hi śilpaṁ tapa āha indraḥ \veg\dontdisplaylinenum

\ujvers\nemsloka 
raṇotsahaṁ kṣatriyayajñam iṣṭaṁ
\dontdisplaylinenum

\nemslokab 
vaiśyaṁ havir yajñam udāharanti \danda\dontdisplaylinenum

\nemslokac 
śūdrasya yajñaḥ paricaryam iṣṭaṁ
\dontdisplaylinenum

\nemslokad 
yajñaṁ dvijānāṁ japamuktamokṣam \veg\dontdisplaylinenum

\vers

devy uvāca~{\dandab}\dontdisplaylinenum 

svamāṁsarudhiraṁ dānaṁ dānaṁ putrakalatrayoḥ\thinspace{\danda} \dontdisplaylinenum

kiṁ praśasyaṁ mahādeva tattvaṁ vaktum ihārhasi \veg\dontdisplaylinenum

maheśvara uvāca~{\dandab}\dontdisplaylinenum 

svamāṁsarudhiraṁ dānaṁ praśaṁsanti manīṣiṇaḥ\thinspace{\danda} \dontdisplaylinenum

śrūyatāṁ pūrvav\textsubring{r}ttāni saṁkṣipya kathayāmy aham \veg\dontdisplaylinenum

uśīnaras tu rājarṣiḥ kayo ?tārthe svakāntantu? \thinspace{\dandab} \dontdisplaylinenum

tyaktvā svargam anuprāptaḥ parārthe paratatparaḥ \veg\dontdisplaylinenum

putramāṁsaṁ svayaṁ chitvā agnidattaṁ purānaghe\thinspace{\dandab} \dontdisplaylinenum

tena dānaprabhāvena alarkas tridivaṁ gataḥ \veg\dontdisplaylinenum

\ujvers\nemsloka 
svadānadānena mudā sa putra
\dontdisplaylinenum

\nemslokab 
aputrabhūtasya ca putra jātaḥ \danda\dontdisplaylinenum

\nemslokac 
svarge svayaṁ cokvaya bhogalābhaṁ
\dontdisplaylinenum

\nemslokad 
prāpto mahaddānay?la prabhāvāt \veg\dontdisplaylinenum

\vers

yādavaś cārjano devi dattvā khaṇḍavabhājanam \veg\dontdisplaylinenum

tapanasya prasādena saptadvīpeśvaro bhavet\thinspace{\dandab} \dontdisplaylinenum

hariṇā ca śiro bhitvā dattaṁ me rudhiraṁ purā \veg\dontdisplaylinenum

pratīcchitaṁ kapālena brahmasambhavajena me\thinspace{\dandab} \dontdisplaylinenum

divyavarṣasahasrāṇi dhārā tasya na chidyate \veg\dontdisplaylinenum

parituṣṭo 'smi tenāhaṁ karmaṇānena sundari\thinspace{\dandab} \dontdisplaylinenum

varaṁ dattaṁ mayā devi purāṇapuruṣo 'vyayaḥ \veg\dontdisplaylinenum

akṣayaṁ valamūrjaṁ ca ajarāmaram eva ca\thinspace{\dandab} \dontdisplaylinenum

mamādhikaṁ bhaved viṣṇur māma yitvam vijeṣyasi \veg\dontdisplaylinenum

evamādīny anekāni mayoktāni janārdane\thinspace{\dandab} \dontdisplaylinenum

niṣkampa niścalamanaḥ sthāṇubhūta iva sthitaḥ \veg\dontdisplaylinenum

da?ciḥ svatanuṁ dattvā vibudhānāṁ varānane\thinspace{\dandab} \dontdisplaylinenum

bhuktvā lokān kramāt sarvān śivaloke pratiṣṭhitaḥ \veg\dontdisplaylinenum

jāmadagnir mahīṁ dattvā kāśyapāya mahātmane\thinspace{\dandab} \dontdisplaylinenum

ihaiva sa yālaṁ bhoktā devarājyam avāpsyati \veg\dontdisplaylinenum 

dattvā go sakalaṁ devi vyāsasyāmitatejasaḥ\thinspace{\dandab} \dontdisplaylinenum

yudhiṣṭhira mahīyāsa dehas tridivadbhataḥ \veg\dontdisplaylinenum

satyanāmaḥ ? (bhīmaḥ?) svakaṁ bhartā dattvā nārādasatk\textsubring{r}tam\thinspace{\dandab} \dontdisplaylinenum

dānasyāsya prabhāvena akṣayaṁ tridivadbhataḥ ? \veg\dontdisplaylinenum

catuḥṣaṣṭhisahastāṇi gavāṁ dattvā dvijanmane\thinspace{\dandab} \dontdisplaylinenum

duryodhanamahīyā?o gataḥ svargam anantakam \veg\dontdisplaylinenum

vāsukis sarparājendro dattvā viprasusaṁsk\textsubring{r}tam\thinspace{\dandab} \dontdisplaylinenum

ratkāruś ca ? sābhānyā sarve nāgavimokṣitāḥ \veg\dontdisplaylinenum

gobhūmikanakādīnāṁ dānaṁ kanyasam ucyate\thinspace{\dandab} \dontdisplaylinenum

bh\textsubring{r}tyaputrakalatrāṇāṁ dānaṁ madhyamam ucyate \veg\dontdisplaylinenum

svadehaṁ pisitādīnāṁ dānam uttamam ucyate\thinspace{\dandab} \dontdisplaylinenum

etat sarvaṁ yadā dānaṁ tad dānam uttamottamam \veg\dontdisplaylinenum
\varr{
        \ \vd °ottamam\lem  \msCapcorr; °otta \msCaacorr}

jāvaj janmasahasrāṇi bhoktā bhavati kanyasaḥ\thinspace{\dandab} \dontdisplaylinenum

śatajanmasahasrāṇi bhoktā bhavati madhyamaḥ \veg\dontdisplaylinenum

uttamaḥ palabhoktā (phala?) vi ? janmakoṭiśatatrayam\thinspace{\dandab} \dontdisplaylinenum 

parārdhadvayajanmānāṁ bhoktā vai cottamottamaḥ \veg\dontdisplaylinenum

bhūtānām anukampayā yadi dhanaṁ dātā sadānvarṣine\thinspace{\dandab} \dontdisplaylinenum

dīnānvak\textsubring{r}yaṇeṣv anāthamalineśvānādini?? ca \veg\dontdisplaylinenum

yady eva kurute sadārtiharaṇaṁ śraddhānvitau bhaktimān\thinspace{\dandab} \dontdisplaylinenum

tasyānantayālaṁ vadanti vibudhāṁs sa yasya sandarśanāt \veg\dontdisplaylinenum

\vers


\jump
\begin{center}
\ketdanda iti v\textsubring{r}ṣasārasaṁgrahe dānadharmaviśeṣaṁ nāma saptādaśamo 'dhyāyaḥ\ketdanda
\end{center}
\dontdisplaylinenum\vers 
\bekveg\szamveg\vfill\phpspagebreak\szam\bek\versno=0\fejno=18
\thispagestyle{empty}



\alfejezet{\textbf{18 pūrvakarmavipākaḥ}}\jump\jump
devy uvāca~{\dandab}\dontdisplaylinenum 

\nemsloka 
bhuktvā tu bhogān suciraṁ yatheṣṭaṁ
\dontdisplaylinenum

\nemslokab 
puṇyakṣayān martyam upāgatānām \danda\dontdisplaylinenum

\nemslokac 
cihnāni teṣāṁ kathayasva me 'dya
\dontdisplaylinenum

\nemslokad 
yathākramaṁ karmaphalaṁ viśeṣāt \veg\dontdisplaylinenum

\vers

maheśvara uvāca~{\dandab}\dontdisplaylinenum 

\nemsloka 
sadānnadātā k\textsubring{r}paṇārtidīnāṁ
\dontdisplaylinenum

\nemslokab 
sa varṣakoṭyāyutam īśaloke \danda\dontdisplaylinenum

\nemslokac 
bhuktvā ca bhogān samam apsarobhiḥ
\dontdisplaylinenum

\nemslokad 
prakṣīṇapuṇyaḥ punar eti martyam \veg\dontdisplaylinenum
\varr{
        \ \vb °yutam īśaloke\lem  \msCapcorr; °yutam īnaśaloke \msCaacorr}

\ujvers\nemsloka 
jāyanti divyeṣu kuleṣu puṁsaḥ
\dontdisplaylinenum

\nemslokab 
sastrīsam\textsubring{r}ddhe bahubh\textsubring{r}tya \danda\dontdisplaylinenum

\nemslokac 
pūrṇe gaurava? śvarannādi dhanā
\dontdisplaylinenum

\nemslokad 
kuleṣu \textsubring{r}ṣo ?jjvalakāntisamāyutaṁ ca \veg\dontdisplaylinenum 

\ujvers\nemsloka 
vastraṁ susatk\textsubring{r}tya dvijasya dānāt
\dontdisplaylinenum

\nemslokab 
svargeṣu modanti sa varṣakoṭyaḥ \danda\dontdisplaylinenum

\nemslokac 
punaś ca te martyam upāgatāś ca
\dontdisplaylinenum

\nemslokad 
cihna?āha?krīyavam āpnuvanti \veg\dontdisplaylinenum

\ujvers\nemsloka 
kūpaprayāpuṣkaraṇī pradātā
\dontdisplaylinenum

\nemslokab 
sa lokam āpnoti jaleśvarasya \danda\dontdisplaylinenum

\nemslokac 
tatas sa tasmāc cyutim āpya lokā
\dontdisplaylinenum

\nemslokad 
akhīsut\textsubring{r}pteṣu kuleṣu jāyet \veg\dontdisplaylinenum

\ujvers\nemsloka 
rannipramāṇād api hemadānāt
\dontdisplaylinenum

\nemslokab 
surendralokaṁ samavāpnuvanti \danda\dontdisplaylinenum

\nemslokac 
tasmāc cyuto martyam upāgatānaṁ
\dontdisplaylinenum

\nemslokad 
cihn?? (saja?) dvi? nadhānyalakṣyāḥ \veg\dontdisplaylinenum

\ujvers\nemsloka 
adūṣya bhūmīvaravipradānāt
\dontdisplaylinenum

\nemslokab 
sa lokam āpnoti sureśvarasya \danda\dontdisplaylinenum

\nemslokac 
bhuktvā tu bhogān cyuta martyaloke
\dontdisplaylinenum

\nemslokad 
cihnaṁ labhed vai viṣayādhipatvam \veg\dontdisplaylinenum

\ujvers\nemsloka 
dvijasya satk\textsubring{r}tya tilapradātā sa
\dontdisplaylinenum

\nemslokab 
lokam āpnoti ca keśavasya \danda\dontdisplaylinenum

\nemslokac 
bhraṣṭas tato martyam upāgatas tu
\dontdisplaylinenum

\nemslokad 
cihnaṁ labhed akṣayam arthalābham \veg\dontdisplaylinenum

\ujvers\nemsloka 
gadā ? sva?ayāṁ vidhivad dvijānām
\dontdisplaylinenum

\nemslokab 
dattvā ca gokolam avāpnuvanti \danda\dontdisplaylinenum

\nemslokac 
kaplāvasāne samupetya martye
\dontdisplaylinenum

\nemslokad 
cihnaṅsavāḍhyaṁ śatagoyutaṁ ca \veg\dontdisplaylinenum

\ujvers\nemsloka 
svargaṁ satānāṁ puruṣasya cihnaṁ
\dontdisplaylinenum

\nemslokab 
vanāḍhyatā śrī mukhabhogalābham \danda\dontdisplaylinenum

\nemslokac 
āyuryaśorūpakalatraputram
\dontdisplaylinenum

\nemslokad 
samyaṅ vibhūti kulakīrtim artham \veg\dontdisplaylinenum

\ujvers\nemsloka 
dānā?(ṣṭa?)bhūñco?ttamakīrtanante
\dontdisplaylinenum

\nemslokab 
cihnaṁ ca lokaṁ ca samāsato me \danda\dontdisplaylinenum

\nemslokac 
ś\textsubring{r}ṇotu devī nirayāgatānāṁ
\dontdisplaylinenum

\nemslokad 
cihnaṁ ca karmaṁ ca vipākatāṁ ca \veg\dontdisplaylinenum

\ujvers\nemsloka 
hatvā ca vipraṁ manasā ca vācā
\dontdisplaylinenum

\nemslokab 
sa yāti pāraṁ nirayasya ghoram \danda\dontdisplaylinenum

\nemslokac 
aśītikalpaṁ niraye krameṇa
\dontdisplaylinenum

\nemslokad 
bhuktvā punas tirya śatāyutānām \veg\dontdisplaylinenum

\ujvers\nemsloka 
jayanti te mānuṣahīnavidyā
\dontdisplaylinenum

\nemslokab 
pratyantavāmāḥ kulavittahīnāḥ \danda\dontdisplaylinenum

\nemslokac 
nityaṁ ca tasyākṣayarogapīḍā
\dontdisplaylinenum

\nemslokad 
idan tu cihnaṁ dvijajīvahartuḥ \veg\dontdisplaylinenum

\ujvers\nemsloka 
pītvā ca madyaṁ dvijaḥ ? kāmato vā
\dontdisplaylinenum

\nemslokab 
āghrāti gadhvaṁ svamanīṣikeṇa \danda\dontdisplaylinenum

\nemslokac 
sa yāti ghoraṁ narakam asahyaṁ
\dontdisplaylinenum

\nemslokad 
yāvac ca kalpaṁ daśa atra bhuktvā \veg\dontdisplaylinenum

\ujvers\nemsloka 
tīryaṁ ca sarvam anubhūya??
\dontdisplaylinenum

\nemslokab 
svaṁ sa kaṣṭakaṣṭena manuṣyajanvā \danda\dontdisplaylinenum

\nemslokac 
caṇḍālaśaunaśvayacanvam eti
\dontdisplaylinenum

\nemslokad 
śyāmaṁ ca tāla bhavatīha cihnam \veg\dontdisplaylinenum

\ujvers\nemsloka 
nindanti ye vedasasnūya jihvā
\dontdisplaylinenum

\nemslokab 
yaḥ kūṭasākṣī sa ca khalv alā?au \danda\dontdisplaylinenum

\nemslokac 
suh\textsubring{r}dvadhām\textsubring{r}tyuśataṁ hi garbhe
\dontdisplaylinenum

\nemslokad 
garhāśanocchiṣṭabhujo bhavanti \veg\dontdisplaylinenum

\ujvers\nemsloka 
stainyas tu yaiḥ kurvati pāpasattvam
\dontdisplaylinenum

\nemslokab 
te pāpadoṣān narakaṁ vrajanti \danda\dontdisplaylinenum

\nemslokac 
manvantarādīny anubhūyaduḥkham
\dontdisplaylinenum

\nemslokad 
punaś ca tiryak śataśo 'nubhūyāt \veg\dontdisplaylinenum

\ujvers\nemsloka 
mānuṣyajanmeṣu ca duḥkhabhāgī
\dontdisplaylinenum

\nemslokab 
steneyamāyāti punaś ca mūḍhaḥ \danda\dontdisplaylinenum

\nemslokac 
suvarṇacaurakunakhatvacihnam
\dontdisplaylinenum

\nemslokad 
viśīrṇagātro rajatāpahārī \veg\dontdisplaylinenum

\ujvers\nemsloka 
tāmrāpahāri sphaṭitāgrapāṇīr
\dontdisplaylinenum

\nemslokab 
lohāpahārī bhujacchedacihnaṁ \danda\dontdisplaylinenum

\nemslokac 
kāṁsāpahārī karabhagnacihnam
\dontdisplaylinenum

\nemslokad 
h\textsubring{r}tvā carīti trapusīsakānām \veg\dontdisplaylinenum

\ujvers\nemsloka 
nāsauṣṭhakarṇaśravaṇasya chedaḥ
\dontdisplaylinenum

\nemslokab 
cihnaṁ n\textsubring{r}ṇāṁ vastraharaṁ kucelaḥ \danda\dontdisplaylinenum

\nemslokac 
dhānyāpahārī bhavaty eṅgahīnaḥ
\dontdisplaylinenum

\nemslokad 
dīpopahārī bhavaty andhacihnam \veg\dontdisplaylinenum

\ujvers\nemsloka 
nirvāpahā kāṇa bhaveta cihnam
\dontdisplaylinenum

\nemslokab 
yaḥ strī haret so 'pi jitaḥ striyā syāt \danda\dontdisplaylinenum

\nemslokac 
sasyāpahārī bhavatennahīnaḥ
\dontdisplaylinenum

\nemslokad 
h\textsubring{r}tvāyudhayantrahatatvacihnaṁ \veg\dontdisplaylinenum

\ujvers\nemsloka 
annāpahārī paradattabhoktā
\dontdisplaylinenum

\nemslokab 
h\textsubring{r}tvā tu gāvaḥ sa bhavet daridraḥ \danda\dontdisplaylinenum

\nemslokac 
hariharettaddhariṇā dahanti
\dontdisplaylinenum

\nemslokad 
h\textsubring{r}tvā tu meṣān ajagardabhaś ca \veg\dontdisplaylinenum

\ujvers\nemsloka 
sa bhārabh\textsubring{r}jjīvam udāharanti
\dontdisplaylinenum

\nemslokab 
ratnāpahārī anapatyatā ca \danda\dontdisplaylinenum

\nemslokac 
chatrāpahārī apavitratā ca
\dontdisplaylinenum

\nemslokad 
h\textsubring{r}tvā ca bījaṁ sa bhaved abījaḥ \veg\dontdisplaylinenum

\ujvers\nemsloka 
godhūmaśāliyavamudgamāṣān
\dontdisplaylinenum

\nemslokab 
h\textsubring{r}tvā masūraṁ vilayaṁ vrajanti \danda\dontdisplaylinenum

\nemslokac 
kāmāturo mātaramāt\textsubring{r}putrī
\dontdisplaylinenum

\nemslokad 
māt\textsubring{r}śvasāṅ gacchati mātulānīm \veg\dontdisplaylinenum

\ujvers\nemsloka 
rājāṅganāṁ putrasutāṁ snuṣāṁ ca
\dontdisplaylinenum

\nemslokab 
pravrājinīṁ brāhmaṇīmantyajāṁ ca \danda\dontdisplaylinenum 

\nemslokac 
ajāśvameṣasurabhīsutāś ca
\dontdisplaylinenum

\nemslokad 
yat kāmayet teṣu vimūḍhacetaḥ \veg\dontdisplaylinenum

\ujvers\nemsloka 
sa yāti k\textsubring{r}cchraṁ narakaṁ sughoraṁ
\dontdisplaylinenum

\nemslokab 
sa varṣakoṭīśataśo bhramitvā \danda\dontdisplaylinenum

\nemslokac 
tīryañ ca bhūyaḥ śataśovyatītya
\dontdisplaylinenum

\nemslokad 
kaṣṭena vai jāyati mānuṣatvam \veg\dontdisplaylinenum

\ujvers\nemsloka 
hīnāṅgatādīnaśarīratāś ca
\dontdisplaylinenum

\nemslokab 
yo māt\textsubring{r}gāmī sa bhaved aliṅgaḥ \danda\dontdisplaylinenum

\nemslokac 
māt\textsubring{r}svasātalpagavānaliṅgā
\dontdisplaylinenum

\nemslokad 
liṅge 'parodhaḥ sutaputrikāmaḥ \veg\dontdisplaylinenum

\ujvers\nemsloka 
snuṣāṁ ca yaḥ sevati raktamehī
\dontdisplaylinenum

\nemslokab 
dauḥ carmatāś ca dvijasundarīṣu \danda\dontdisplaylinenum

\nemslokac 
rājāṅganāyāsu ca liṅgacchedaḥ
\dontdisplaylinenum

\nemslokad 
pravrājinī kāmukamūtrak\textsubring{r}cchram \veg\dontdisplaylinenum

\ujvers\nemsloka 
savyādhiliṅga labhatentyajāsu
\dontdisplaylinenum

\nemslokab 
vilīnaliṅgaḥ paśuyonigāmī \danda\dontdisplaylinenum

\nemslokac 
jāyanti te mūṣikadhānyacaurī
\dontdisplaylinenum

\nemslokad 
kṣīraṁ hared vāyasatāṁ prayāti \veg\dontdisplaylinenum

\ujvers\nemsloka 
haṁsāpahārī sa bhaven nihaṁsaḥ
\dontdisplaylinenum

\nemslokab 
śvānatvam āyāti rasāpahārī \danda\dontdisplaylinenum

\nemslokac 
h\textsubring{r}tvā ca sūcīn tu bhavet sa daṁśaḥ
\dontdisplaylinenum

\nemslokad 
h\textsubring{r}tvā tu sarpir v\textsubring{r}ṣatāṁ prayāti \veg\dontdisplaylinenum

\ujvers\nemsloka 
māṁsaṁ tu h\textsubring{r}tvā sa bhaveta g\textsubring{r}dhraḥ
\dontdisplaylinenum

\nemslokab 
tailāpahārī khagatāṁ prayāti \danda\dontdisplaylinenum

\nemslokac 
guḍaṁ ca h\textsubring{r}tvā guḍikā bhavanti
\dontdisplaylinenum

\nemslokad 
śākāpahārī sa bhaven mayūram \veg\dontdisplaylinenum

\ujvers\nemsloka 
h\textsubring{r}tvā paśuṁ paṅgurajāyatehaḥ
\dontdisplaylinenum

\nemslokab 
citratvam āyāti suvastrahārī \danda\dontdisplaylinenum

\nemslokac 
h\textsubring{r}tvā dukūlaṁ sa ca sārasattvaṁ
\dontdisplaylinenum

\nemslokad 
kṣaumaṁ ca h\textsubring{r}tvā sa ca durbalatvam \veg\dontdisplaylinenum

\ujvers\nemsloka 
ūrnāni vastrāṇy apah\textsubring{r}tya meṣaḥ
\dontdisplaylinenum

\nemslokab 
chuchundarī jāyati gandhahārī \danda\dontdisplaylinenum

\nemslokac 
brahmasvam alpam apah\textsubring{r}tya bhoktā
\dontdisplaylinenum

\nemslokad 
sa g\textsubring{r}dhra ucchiṣṭabhujo bhavanti \veg\dontdisplaylinenum

\ujvers\nemsloka 
pādena yaḥ sparśayate dvijāṅghriṁ
\dontdisplaylinenum

\nemslokab 
tacchītaraktaṁ caraṇau bhaveta \danda\dontdisplaylinenum

\nemslokac 
pādena yaḥ sparśayate ca gāvaḥ
\dontdisplaylinenum

\nemslokad 
sa pādarogān vividhāṁl labheta \veg\dontdisplaylinenum

\ujvers\nemsloka 
yo mātaraḥ tāḍayate pādena
\dontdisplaylinenum

\nemslokab 
pāde tadīye k\textsubring{r}mayaḥ patanti \danda\dontdisplaylinenum

\nemslokac 
pādāt p\textsubring{r}śed yaḥ pitaraṁ durātmā
\dontdisplaylinenum

\nemslokad 
sūnonnapādaḥ sa bhavet paratra \veg\dontdisplaylinenum

\ujvers\nemsloka 
padāt p\textsubring{r}śet toyam anādareṇa
\dontdisplaylinenum

\nemslokab 
saślīpadīpādayuge bhaveta \danda\dontdisplaylinenum

\nemslokac 
pādena ya sparśayate hutāśaṁ
\dontdisplaylinenum

\nemslokad 
sa cāgnipādaḥ satataṁ bhaveta \veg\dontdisplaylinenum

\ujvers\nemsloka 
pādena yaś cāryam upasp\textsubring{r}śeta
\dontdisplaylinenum

\nemslokab 
sa pādacchedaṁ bahuśo labheta \danda\dontdisplaylinenum

\nemslokac 
granthāpahārī sa bhaveta mūkaḥ
\dontdisplaylinenum

\nemslokad 
durgandhavaktraḥ parichidravādī \veg\dontdisplaylinenum

\ujvers\nemsloka 
paiśunyavādī sa ca pūtināsām
\dontdisplaylinenum

\nemslokab 
anamravaktras tv an\textsubring{r}tāpavādī \danda\dontdisplaylinenum

\nemslokac 
pāruṣyavaktā mukhapākarāgī
\dontdisplaylinenum

\nemslokad 
asat pralāpī sa ca dantarogaḥ \veg\dontdisplaylinenum

\ujvers\nemsloka 
stīkṣṇapradāyī sa ca vakranāsa
\dontdisplaylinenum

\nemslokab 
sambhinnavaktā sa ca kaṇṭharogī \danda\dontdisplaylinenum

\nemslokac 
kruddhekṣaṇaḥ paśyati yas tu vipraṁ
\dontdisplaylinenum

\nemslokad 
tīvrākṣirogī sa tu jāyate hi \veg\dontdisplaylinenum

\ujvers\nemsloka 
pradveṣayālokayate 'tithīn ya
\dontdisplaylinenum

\nemslokab 
utpāditākṣis sa bhavet paratra \danda\dontdisplaylinenum

\nemslokac 
vairūpya cakṣus tv atisūkṣmacakṣuḥ
\dontdisplaylinenum

\nemslokad 
sa jāyate kekarapiṅgayakṣuḥ \veg\dontdisplaylinenum

\ujvers\nemsloka 
gartākṣikādīni vipāṇḍurāṇi
\dontdisplaylinenum

\nemslokab 
netrāmayāny eva ca pāpadoṣāt \danda\dontdisplaylinenum

\nemslokac 
ś\textsubring{r}ṇvanti ye pāpakathāṁ praśastāṁ
\dontdisplaylinenum

\nemslokad 
tāṁ karṇasarpiḥ paripīḍiyeta \veg\dontdisplaylinenum

\ujvers\nemsloka 
ś\textsubring{r}ṇvanti nindāṁ hariśarvayor yaḥ
\dontdisplaylinenum

\nemslokab 
sa karṇaśūlena tu jīvatī vā \danda\dontdisplaylinenum

\nemslokac 
mātāpit\textsubring{\=r}ṇāṁ ś\textsubring{r}ṇute 'pavādaḥ
\dontdisplaylinenum

\nemslokad 
sa karṇasāphena vināśam eti \veg\dontdisplaylinenum

\ujvers\nemsloka 
ś\textsubring{r}ṇoti nindāṁ guruviprajā yaḥ
\dontdisplaylinenum

\nemslokab 
sa karṇapūyaṁ sravate saraktam \danda\dontdisplaylinenum

\nemslokac 
virūpyadāridhrakulādhameṣu
\dontdisplaylinenum

\nemslokad 
aniṣṭakarmabh\textsubring{r}tijīvanāś ca \veg\dontdisplaylinenum

\ujvers\nemsloka 
akīrtanaṁ darśanavarjanaṁ ca
\dontdisplaylinenum

\nemslokab 
śvāpākato śvādiṣu jāyate saḥ \danda\dontdisplaylinenum

\nemslokac 
etāni cihnaṁ nirayāgatānāṁ
\dontdisplaylinenum

\nemslokad 
mānuṣyaloke kuk\textsubring{r}tasya d\textsubring{r}ṣṭam \veg\dontdisplaylinenum

\vers

samāsataḥ kīrtita eva devi\thinspace{\dandab} \dontdisplaylinenum

yathaiva muktis tv iha karmabhaṅgaḥ \veg\dontdisplaylinenum

\ujvers\nemsloka 
mātāpitroghato yāsutaduhit\textsubring{r}vahā bhrāt\textsubring{r}gambhīravegā
\dontdisplaylinenum

\nemslokab 
bhāryāvartā vivartā kuṭilagativadhur bāndhavormītaraṅgā \danda\dontdisplaylinenum

\nemslokac 
kāmakrodhobhakūlā karimakarajhaṣā grāhakāmā bhayante
\dontdisplaylinenum

\nemslokad 
m\textsubring{r}tyor ākhyārṇave 'smin na śaraṇavivaśākālad\textsubring{r}ṣṭo prayāti \veg\dontdisplaylinenum

\ujvers\nemsloka 
nityaṁ yena vinā na yāti divasaṁ pañcatvam āpadyate
\dontdisplaylinenum

\nemslokab 
tyaktvā deha vanāntareṣu viṣame śvānaśrigālākule \danda\dontdisplaylinenum

\nemslokac 
bandhuḥ sarvanivartate gatadayā dharmaika tatra sthitaḥ
\dontdisplaylinenum

\nemslokad 
tasmād dharmaparo na cānyaḥ suh\textsubring{r}daḥ sevet paratrārthinaḥ \veg\dontdisplaylinenum

\vers


\jump
\begin{center}
\ketdanda iti v\textsubring{r}ṣasārasaṁgrahe pūrvakarmavipākacihnāṣṭādaśo 'dhyāyaḥ\ketdanda
\end{center}
\dontdisplaylinenum\vers 

\vers
\bekveg\szamveg\vfill\phpspagebreak\szam\bek\versno=0\fejno=19
\thispagestyle{empty}



\alfejezet{\textbf{19 dānayajñaviśeṣaḥ}}\jump\jump
vigatarāga uvāca~{\dandab}\dontdisplaylinenum 

kriyāsūkṣmo mahādharmaḥ karmaṇā kena prāpyate\thinspace{\danda} \dontdisplaylinenum

alpopāyaṁ narārthāya p\textsubring{r}cchāmi kathayasva me \veg\dontdisplaylinenum

anarthayajña uvāca~{\dandab}\dontdisplaylinenum 

alpopāyaṁ mahādharmaṁ kathayāmi dvijottama\thinspace{\danda} \dontdisplaylinenum

sukhena labhate svargaṁ karmaṇā yena tac ch\textsubring{r}ṇu \veg\dontdisplaylinenum

lokānaṁ mātaro gāvo gobhiḥ sarvaṁ jagad dh\textsubring{r}tam\thinspace{\dandab} \dontdisplaylinenum

gomayam am\textsubring{r}taṁ sarvaṁ jātaṁ sarvaśivecchayā \veg\dontdisplaylinenum

sarvadevamayī gāvaḥ sarvadevamayo dvijaḥ\thinspace{\dandab} \dontdisplaylinenum

sarvadevamayo bhūmiḥ sarvadevamayaḥ śivaḥ \veg\dontdisplaylinenum

tasmād gāvaḥ sadā sevyā dharmamokṣārthasiddhidā\thinspace{\dandab} \dontdisplaylinenum

paricaryā yathāśaktyā grāsavāsajalādibhiḥ \veg\dontdisplaylinenum

tāḍayen nātivegena vācayen m\textsubring{r}dunācaret\thinspace{\dandab} \dontdisplaylinenum

pālayan tarpanād yeṣu bhagnodvigneṣu yatnataḥ \veg\dontdisplaylinenum

vyādhivanaparikleśa oṣadhopakramaś caret\thinspace{\dandab} \dontdisplaylinenum

kaṇḍūyanaṁ ca kartavyaṁ yathāsaukhyaṁ bhaved gavām \veg\dontdisplaylinenum

gavāṁ pradakṣiṇaṁ k\textsubring{r}tvā śraddhābhaktisamanvitaḥ\thinspace{\dandab} \dontdisplaylinenum

sāgarāntā mahī sarvā n pradakṣiṇīk\textsubring{r}tā bhavet \veg\dontdisplaylinenum

p\textsubring{r}ṣṭasaṁsparśanād yañ ca śraddhayā yadi mānavaḥ\thinspace{\dandab} \dontdisplaylinenum

ahorātrak\textsubring{r}taṁ pāpaṁ naśyate nātrasaṁśayaḥ \veg\dontdisplaylinenum

lāṅgūlenoddh\textsubring{r}taṁ toyaṁ mūrddhnā g\textsubring{r}hṇāti yo naraḥ\thinspace{\dandab} \dontdisplaylinenum

yāvaj jīva k\textsubring{r}taṁ pāpaṁ naśyate nātra saṁśayaḥ \veg\dontdisplaylinenum

vidhivat snāpayed gāṁś ca mantrayuktena vāriṇā\thinspace{\dandab} \dontdisplaylinenum

tenāmbhasā svayaṁ snātvā sarvapāpakṣayo bhavet \veg\dontdisplaylinenum

vyādhivighnam alakṣmītvaṁ naśyate sadya eva ca\thinspace{\dandab} \dontdisplaylinenum

m\textsubring{r}tāpatyāś ca gāvāś ca snānam eva praśasyate \veg\dontdisplaylinenum

gavāṁ ś\textsubring{r}ṅgodakaṁ g\textsubring{r}hya mūrdhni yo dhārayen naraḥ\thinspace{\dandab} \dontdisplaylinenum

sa sarvatīrthasnānasya phalaṁ prāpnoti mānavaḥ \veg\dontdisplaylinenum

grāsamuṣṭipradānena goṣu bhaktisamanvitaḥ\thinspace{\dandab} \dontdisplaylinenum

agnihotraṁ hutaṁ tena sarvadevāḥ sutarpitāḥ \veg\dontdisplaylinenum

catvāraḥ stanadhārās tu yas tu mūrdhnā pratīcchati\thinspace{\dandab} \dontdisplaylinenum

sa catuḥsāgaraṁ gatvā snānapuṇyaphalaṁ labhet \veg\dontdisplaylinenum

gavārthaṁ yas tyajet prāṇān gograheṣu dvijottama\thinspace{\dandab} \dontdisplaylinenum

kalpakoṭiśataṁ divyaṁ śivaloke mahīyate \veg\dontdisplaylinenum

cyutabhagnādisaṁskāraṁ sarvaṁ yaḥ kurute naraḥ\thinspace{\dandab} \dontdisplaylinenum

bhāryākoṭiśataṁ dānaṁ yat phalaṁ parikīrtitam \veg\dontdisplaylinenum 

tatphalaṁ labhate martyaḥ śivalokaṁ ca gacchati\thinspace{\dandab} \dontdisplaylinenum

śivalokaparibhraṣṭaḥ p\textsubring{r}thivyām ekarāḍ bhavet \veg\dontdisplaylinenum

samāsataḥ samākhyātaṁ yathātattvaṁ dvijottama\thinspace{\dandab} \dontdisplaylinenum

na śakyaṁ vistarād vaktuṁ gomahātmyasamuttamam \veg\dontdisplaylinenum

vigatarāga uvāca~{\dandab}\dontdisplaylinenum 

devāḥ r aṣṭavidhāḥ proktāḥ tiryak pañcavidhaḥ sm\textsubring{r}taḥ\thinspace{\danda} \dontdisplaylinenum

mānuṣyam ekam evāhuś cāturvarṇyaḥ kathaṁ bhavet \veg\dontdisplaylinenum

anarthayajña uvāca~{\dandab}\dontdisplaylinenum 

pūrvakalpas\textsubring{r}jaty eṣa viṣṇunā prabhaviṣṇunā\thinspace{\danda} \dontdisplaylinenum

evaṁ varṇā dvijaś cāsīt sarvakalpāgram agrataḥ \veg\dontdisplaylinenum

sarvavedavido viprāḥ sarvavedavidas tathā\thinspace{\dandab} \dontdisplaylinenum

tathā viprasahasrāṇāṁ yajñotsāhamano bhavet \veg\dontdisplaylinenum

v\textsubring{r}ddhaviprasahasrāṇāṁ matam āśritya brāhmaṇaiḥ\thinspace{\dandab} \dontdisplaylinenum

kartuṁ karma samārabdhakarmaś cāpi vibhajyate \veg\dontdisplaylinenum

\textsubring{r}tvajatve sthitāḥ kecit kecit saṁrakṣaṇe sthitāḥ\thinspace{\dandab} \dontdisplaylinenum

arthopārjanayuktān ye anye śilpe niyojitāḥ \veg\dontdisplaylinenum

evaṁ yajñavidhānena kartum arebhire purā\thinspace{\dandab} \dontdisplaylinenum

yathoddiṣṭena karmeṇa yajñotsāham avartata \veg\dontdisplaylinenum

āgatā \textsubring{r}ṣayaḥ sarve devatāḥ pitaras tathā\thinspace{\dandab} \dontdisplaylinenum

anyonyam abruvan tatra devarṣipit\textsubring{r}devatāḥ \veg\dontdisplaylinenum

yajñārtam as\textsubring{r}jad varṇaṁ vidhinā pātuhetavaḥ\thinspace{\dandab} \dontdisplaylinenum

evam eva pravartantu bhavatir dvijasattamāḥ \veg\dontdisplaylinenum

ijyādhyādhyayanasampannā brahmaṇā yatra kalpitāḥ\thinspace{\dandab} \dontdisplaylinenum

suviprā vipratāṁ yāntu ṣaḍkarmāniratāḥ sadā \veg\dontdisplaylinenum

rakṣaṇārtaṁ tu ye viprāḥ kalpitāḥ śastrapāṇayaḥ\thinspace{\dandab} \dontdisplaylinenum

k\textsubring{r}tatrāṇāya viprāṇāṁ nityaṁ kṣātravratodbhavāḥ \veg\dontdisplaylinenum

arthopārjanam uddiśya kalpitā ye dvijātayaḥ\thinspace{\dandab} \dontdisplaylinenum

te tu vaiśyatvam āyāntu vārto āpaṇatodbhavāḥ \danda\dontdisplaylinenum

vadhabandhanakarmeṣu śilpasthānavadheṣu ca \veg\dontdisplaylinenum

kalpitā ye dvijātīnāṁ sarve śūdrā bhavantu te\thinspace{\dandab} \dontdisplaylinenum

prājāpatyaṁ brāhmaṇānām ījyādhyayanatatparām \veg\dontdisplaylinenum

sthānam aindraṁ kṣatriyāṇāṁ prajāpālanatatparam\thinspace{\dandab} \dontdisplaylinenum

vaiśyānāṁ vāsavasthānaṁ vāṇijyaṁ k\textsubring{r}ṣijīvinām \veg\dontdisplaylinenum

śūdrāṇāṁ marutaḥ sthānaṁ śuśrūṣāniratātmanām\thinspace{\dandab} \dontdisplaylinenum

maharṣipit\textsubring{r}devānāṁ matam ājñāya niścitaḥ \danda\dontdisplaylinenum

eṣa saṁkalpito brahmā padmayoniḥ pitāmahaḥ \veg\dontdisplaylinenum

saṁkalpaprabhavāḥ sarve devadānavamānavāḥ\thinspace{\dandab} \dontdisplaylinenum

paśupakṣim\textsubring{r}gāmukhyā yāvanti jagasambhavāḥ \veg\dontdisplaylinenum

bhūtasaṁkalpakartā ya kalpam āsīd dvijottama\thinspace{\dandab} \dontdisplaylinenum

kīrtitāni samāsena kim anyac chrotum icchasi \veg\dontdisplaylinenum

vigatarāga uvāca~{\dandab}\dontdisplaylinenum 

kiṁ tapaḥ sarvavarṇānāṁ v\textsubring{r}ttir vāpi tapodhana\thinspace{\danda} \dontdisplaylinenum

yajñāś caiva p\textsubring{r}thaktvena śrotum icchāmi tattvataḥ \veg\dontdisplaylinenum

anarthayajña uvāca~{\dandab}\dontdisplaylinenum 

brāhmaṇasya tapo yajñāḥ - tapaḥ kṣātrasya rakṣaṇam\thinspace{\danda} \dontdisplaylinenum

vaiśyaś ca tapa vāṇijya tapaḥ śūdrasya sevanam \veg\dontdisplaylinenum

pratigraha dhano vipraḥ kṣatriyasya dhanur dhanam\thinspace{\dandab} \dontdisplaylinenum

k\textsubring{r}ṣir dhanaṁ tathā vaiśyaḥ śūdraḥ śuśrūṣaṇaṁ dhanam \veg\dontdisplaylinenum

ārambhayajñaḥ kṣatrasya havir yajño viśas tathā\thinspace{\dandab} \dontdisplaylinenum
            \paral{\textit{\vab {\normalfont  \kb\ MBh 12224061ab and 12230012ab } }}

śūdraḥ paricaro yajño japayajño dvijātayaḥ \veg\dontdisplaylinenum

satya tīrtha dvijātīnāṁ raṇa tīrthaṁ tu kṣatriyāḥ\thinspace{\dandab} \dontdisplaylinenum

āryā tīrthaṁ tu vaiśānāṁ ! śūdratīrthaṁ tu vai dvijāḥ \veg\dontdisplaylinenum

nāsti vidyāsamo mitro nāsti dānasamaḥ sakhā\thinspace{\dandab} \dontdisplaylinenum

nāsti jñānasamo bandur nāsti yajño japaḥ samaḥ \veg\dontdisplaylinenum

dharmahīno m\textsubring{r}tas tulyo devatulyo jitendriyaḥ\thinspace{\dandab} \dontdisplaylinenum

yajñatulyo 'bhayaṁ dātā śivatulyao manonmanaḥ \veg\dontdisplaylinenum

vigatarāga uvāca~{\dandab}\dontdisplaylinenum 

dāna yajñas tapas tīrthaṁ saṁnyāsaṁ yoga eva ca\thinspace{\danda} \dontdisplaylinenum

eteṣu katamaḥ śreṣṭhaḥ śrotum icchāmi kīrtaya \veg\dontdisplaylinenum

anarthayajña uvāca~{\dandab}\dontdisplaylinenum 

dānadharmasahasrebhyaḥ yajñayājī viśiṣyate\thinspace{\danda} \dontdisplaylinenum

yajñayājīsahasrebhyas tīrthayātrī viśiṣyate \veg\dontdisplaylinenum

tīrthayātrisahasrebhyas tapaniṣṭo viśiṣyate\thinspace{\dandab} \dontdisplaylinenum

tapaniṣṭhasahasrebhyaḥ śreṣṭhaḥ saṁnyāsikaḥ sm\textsubring{r}taḥ \veg\dontdisplaylinenum

saṁnyāsīnāṁ sahasrebhyaḥ śreṣṭho yac ya jitendriyaḥ\thinspace{\dandab} \dontdisplaylinenum

jitendriyasahasrebhyaḥ yogayukto viśiṣyate \veg\dontdisplaylinenum

yogayuktasahasrebhyaḥ śreṣṭho līnamanaḥ sm\textsubring{r}taḥ\thinspace{\dandab} \dontdisplaylinenum

tasmāt sarvaprayatnena ādau mana viśodhayet \veg\dontdisplaylinenum

nig\textsubring{r}hītendriyagrāmaḥ svargamokṣau tu sādhanam\thinspace{\dandab} \dontdisplaylinenum

viśiṣṭhe tv indriyagrāme tiryannarakasādhanam \veg\dontdisplaylinenum

vigatarāga uvāca~{\dandab}\dontdisplaylinenum 

carācarāṇāṁ bhūtānāṁ katamaḥ śreṣṭha ucyate\thinspace{\danda} \dontdisplaylinenum

kathayasva mamādya tvaṁ chettum arhasi saṁśayam \veg\dontdisplaylinenum

anarthayajña uvāca~{\dandab}\dontdisplaylinenum 

carācarāṇāṁ bhūtānāṁ tatra śreṣṭho - carāḥ sm\textsubring{r}tāḥ\thinspace{\danda} \dontdisplaylinenum

carāṇāṁ caiva sarveṣāṁ buddhimān śreṣṭha ucyate \veg\dontdisplaylinenum

buddhimānṣu ! ca sarveṣu tataḥ śreṣṭha narāḥ sm\textsubring{r}tāḥ\thinspace{\dandab} \dontdisplaylinenum

narāṇāṁ caiva sarveṣāṁ brāhmaṇaḥ śreṣṭha ucyate \veg\dontdisplaylinenum

vidvarsv api ca sarveṣu k\textsubring{r}tabuddhir viśiṣyate\thinspace{\dandab} \dontdisplaylinenum

k\textsubring{r}tabuddhiṣu sarveṣu śreṣṭhaḥ kartā sa ucyate \veg\dontdisplaylinenum

kart\textsubring{r}ṣv api ca sarveṣu brahmavedī viśiṣyate\thinspace{\dandab} \dontdisplaylinenum

brahmavedi paraṁ ! vipraḥ nānyaṁ vedmi paraṁtapaḥ \danda\dontdisplaylinenum

sa vipraḥ sa tapasvī ca sa yogī sa śivaḥ sm\textsubring{r}taḥ \veg\dontdisplaylinenum


\jump
\begin{center}
\ketdanda iti v\textsubring{r}ṣasārasaṁgrahe dānayajñaviśeṣo nāma unaviṁśatitamo 'dhyāyaḥ\ketdanda
\end{center}
\dontdisplaylinenum\vers 
\bekveg\szamveg\vfill\phpspagebreak\szam\bek\versno=0\fejno=20
\thispagestyle{empty}



\alfejezet{\textbf{viṁśatimo 'dhyāyaḥ}}\jump\jump
\vers

vigatarāga uvāca~{\dandab}\dontdisplaylinenum 

pañcaviṁśati yat tattvaṁ jñātum icchāmi tattvataḥ\thinspace{\danda} \dontdisplaylinenum

kathayasva mamādya tvaṁ chidyate yena saṁśayaḥ \veg\dontdisplaylinenum


\alalfejezet{tattvanirṇayam}
anarthayajña uvāca~{\dandab}\dontdisplaylinenum 

sarvapratyakṣadarśitvaṁ kathaṁ māṁ praṣṭum arhasi\thinspace{\danda} \dontdisplaylinenum

p\textsubring{r}ṣṭena kathanīyo 'smi eṣa me k\textsubring{r}taniścayaḥ \danda\dontdisplaylinenum

ś\textsubring{r}ṇu te sampravakṣyāmi tattvasadbhāvam uttamam \veg\dontdisplaylinenum


\alalfejezet{puruṣaḥ-śivaḥ-brahmā (25)}\varr{
        \ \va sarva°\lem  \mssCaCbCc\msNb; sarvaṁ \msNa\Ed\oo
                 °darśitvaṁ\lem  \msCa\msCc\msNa\msNb\Ed; °darśītvaṁ \msCb
        \ \vb māṁ\lem  \mssCaCbCc\msNb\Ed; maṁ \msNa
        \ \vc 'smi\lem  \msCa\msCc\msNa\msNb\Ed; smī \msCb}

nādimadhyaṁ na cāntaṁ ca yan na vedyaṁ surair api\thinspace{\dandab} \dontdisplaylinenum

atisūkṣmo hy atisthūlo nirālambo nirañjanaḥ \veg\dontdisplaylinenum
\varr{
        \ \va °madhyaṁ\lem  \mssCaCbCc\msNa\Ed; °madya \msNb\oo
                 cāntaṁ ca\lem  \mssCaCbCc\msNa\msNb; cāntaś ca \Ed
        \ \vb surair api\lem  \msCa\msCc\msNa\msNb\Ed; surer api \msCb
        \ \vc hy ati°\lem  \msCa\msCb\msNa\msNb\Ed; hy adi° \msCc}

acintyaś cāprameyaś ca akṣarākṣaravarjitaḥ\thinspace{\dandab} \dontdisplaylinenum

sarvaḥ sarvagato vyāpī sarvam āv\textsubring{r}tya tiṣṭhati \veg\dontdisplaylinenum
\varr{
        \ \vcd (sarvaḥ{\normalfont ...} tiṣṭhati)\lem  \mssCaCbCc\msNa\msNb; \om\ \Ed}

sarvendriyaguṇābhāsaḥ sarvendriyavivarjitaḥ\thinspace{\dandab} \dontdisplaylinenum

ajarāmarajaḥ śāntaḥ paramātmā śivo 'vyayaḥ \veg\dontdisplaylinenum
\varr{
        \ \vab sarve°{\normalfont ...} varjitaḥ\lem  \mssCaCbCc\msNa\msNb; \om\ \Ed
        \ \vc °jaḥ\lem  \mssCaCbCc\msNa\msNb; yaḥ \Ed}

alakṣyalakṣaṇaḥ svastho brahmā puruṣasaṁjñitaḥ\thinspace{\dandab} \dontdisplaylinenum

pañcaviṁśaḥ sa vijñeyo janmam\textsubring{r}tyuharaḥ prabhuḥ \veg\dontdisplaylinenum
\varr{
        \ \vb brahmā\lem  \msCa\msCb\Ed; brahma \msCc\msNa\msNb
        \ \vc °viṁśaḥ\lem  \mssCaCbCc\msNb\Ed; °viṁśat \msNaacorr, °viṁśa \msNapcorr\oo 
                 sa vijñeyo\lem  \mssCaCbCc\msNaacorr\msNb\Ed; sarvajñeyo \msNapcorr}

kalākalaṅkanirmukto vyomapañcāśavarjitaḥ\thinspace{\dandab} \dontdisplaylinenum

jalapakṣī yathā toyair na lipyeta jale caran \danda\dontdisplaylinenum

tadvad doṣair na lipyeta pāpakarmaśatair api \veg\dontdisplaylinenum


\alalfejezet{prak\textsubring{r}tiḥ (24)}\varr{
        \ \va °nirmukto\lem  \mssCaCbCc\msNa\Ed; °lirmukto \msNb
        \ \vb °pañcāśa°\lem  \mssCaCbCc\msNa\Ed; °pañcasa° \msNb
        \ \vcd yathā toyair na\lem  \msCa\Ed; yathā toyī na \msCbacorr, 
                        yathā toyer na \msCbpcorr, yathā toyai na \msCc\msNa, yadā toyai nna \msNb
        \ \vd lipyeta\lem  \mssCaCbCc\msNa\msNb; lipyate \Ed\oo
                 jale\lem  \msCa\msCc\msNa\msNb\Ed; jalai \msCb
        \ \vc °ṣair na\lem  \msCa\msCb\msNa\Ed; °ṣai na \msCc, °ṣai nna \msNb}

caturviṁśati yat tattvaṁ prak\textsubring{r}tiṁ viddhi niścayam\thinspace{\dandab} \dontdisplaylinenum

vik\textsubring{r}tiś ca sa vijñeyas tattvataḥ sa manīṣibhiḥ \veg\dontdisplaylinenum
\varr{
        \ \va yat tattvaṁ\lem  \msCa\msCb\msNa\msNb; ya tatvaṁ \msCc, yan tatvaṁ \Ed
        \ \vb prak\textsubring{r}tiṁ viddhi niścayam\lem  \conj; prak\textsubring{r}tir vidhiniścayaḥ \msCa\msCb\msNa\msNb\Ed, 
                                                                prak\textsubring{r}ti vidhiniścayaḥ \msCc
        \ \vc vik\textsubring{r}tiś ca\lem  \msCa\msCb\msNa\msNb\Ed; vik\textsubring{r}tiñ ca \msCc
        \ \vd °jñeyas tattva°\lem  \mssCaCbCc\msNa\Ed; °jñeyos tatva° \msNb}

prak\textsubring{r}tiprabhavāḥ sarve buddhyahaṁkāra-ādayaḥ\thinspace{\dandab} \dontdisplaylinenum

vik\textsubring{r}tiṁ pratilīyante bhūmyādi kramaśas tu vai \veg\dontdisplaylinenum


\alalfejezet{matiḥ-buddhiḥ (23)}\varr{
        \ \va °bhavāḥ\lem  \mssCaCbCc\msNa\Ed; °bhāvaḥ \msNb
        \ \vb buddhyahaṁkāra-ādayaḥ\lem  \msCa\msCb\msNa\msNb; bubuddhyahaṁkāra ādayaḥ \msCc,
                                                         buddhyāhaṁkārakādayaḥ \Ed
        \ \vc vik\textsubring{r}tiṁ\lem  \mssCaCbCc\msNa\Ed; vik\textsubring{r}ti \msNb
        \ \vd kramaśas tu vai\lem  \msCa\msCb\msNb\Ed; yaḥ kramas tu vai \msNa, kramasaṁs tu vaiḥ \msCc}

matitattva trayoviṁśa dharmādiguṇasaṁyutaḥ\thinspace{\dandab} \dontdisplaylinenum

sattvādhikasamutpannaboddhāraṁ viddhi dehinaḥ \veg\dontdisplaylinenum


\alalfejezet{ahaṁkāraḥ (22)}\varr{
        \ \vb °yutaḥ\lem  \msCa\msCc\msNa\msNb\Ed; °yutam \msCb
        \ \vc °samutpanna°\lem  \msCa\msCc\msNa\msNb\Ed; °samutpanno \msCb
        \ \vd °boddhāraṁ\lem  \eme; °bodhāta \mssCaCbCc\msNb, °boddhātaṁ \msNa, °boddhāta \Ed\oo
                 viddhi\lem  \eme; vidhi \mssCaCbCc\msNa\msNb\Ed}

dvāviṁśati ahaṁkāras tattvam uktaṁ manīṣibhiḥ\thinspace{\dandab} \dontdisplaylinenum

bhūtādi mama pañcāha rajādhikasamudbhavam \veg\dontdisplaylinenum


\alalfejezet{ākāśaḥ (suṣiratvaṁ) śabdaś ca (21-20)}\varr{
        \ \vb uktaṁ\lem  \msCa\msCb\msNa; ukta \msCc\msNb\Ed
        \ \vc bhūtādi mama pañcāha\lem  \mssCaCbCc\msNa\msNb; bhūtādir nāma pañcāha \Ed
        \ \vd rajā°\lem  \mssCaCbCc\msNa\msNb; rajo° \Ed\oo
                 °dbhavam\lem  \msCa\msCc\msNa\msNb\Ed; °dbhavaḥ \msCb}

ekaviṁśati yat tattvaṁ suṣiraṁ viddhi bho dvija\thinspace{\dandab} \dontdisplaylinenum

śabdātītaṁ suṣiratvaṁ saśabdaguṇalakṣaṇam \veg\dontdisplaylinenum


\alalalfejezet{śabdaḥ}
\varr{
        \ \va yat tattvaṁ\lem  \msCa\msCb\msNa\msNb\Ed; ya tatvaṁ \msCc
        \ \vb suṣiraṁ viddhi\lem  \eme; suśiraṁ viddhi \msCa\msCb\msNa\Ed, susira v\textsubring{r}ddhi \msCc, 
                                                        susiraṁ v\textsubring{r}ddhi \msNb\oo
                 dvija\lem  \mssCaCbCc\msNa\Ed; dvijaḥ \msNb
        \ \vc suṣiratvaṁ\lem  \eme; suśiratvaṁ \mssCaCbCc\msNa\msNb\Ed
        \ \vd °lakṣaṇam\lem  \msCb\msCc\msNa\msNb\Ed; °\uncl{la}{\lost}ṇam \msCa}

saptasvarās trayo grāmā mūrchanās tv ekaviṁśatiḥ\thinspace{\dandab} \dontdisplaylinenum

tānā-m-ekonapañcāśac chabdabhedas tadādayaḥ \veg\dontdisplaylinenum
\varr{
        \ \va grāmā\lem  \mssCaCbCc\msNa\msNb; grāmāḥ \Ed
        \ \vb mūrchanā°\lem  \mssCaCbCc\msNa\Ed; mūrcchānā° \msNb\oo
                 °viṁśatiḥ\lem  \msCc\Ed; °viṁśati \msCa\msCb\msNa\msNb
        \ \vc °kona°\lem  \msCa\Ed; °kūna° \msCb\msCc\msNa\msNb}

evam ādīny anekāni svarabhedā dvijottama\thinspace{\dandab} \dontdisplaylinenum

gāndharvasvaratattvajñair munibhiḥ samudāh\textsubring{r}tam \veg\dontdisplaylinenum
\varr{
        \ \vb °bhedā\lem  \mssCaCbCc\msNa\msNb; °bhedān \Ed\oo
                 °ttama\lem  \msCa\msCb\msNa\msNb\Ed; °ttamaḥ \msCc
        \ \vc gāndharvasvaratattva°\lem  \msCb\msNa\msNb\Ed; gāndharvvāsuratatva° \msCa,
                                                 gandharvvāsurastatva° \msCc
        \ \vcd °jñair muninibhiḥ\lem  \msCa\msCb\msNa\Ed; °jñair munibhi \msCc, °jñai munibhiḥ \msNb}

veṇumurajatantrīṇāṁ dundubhīnāṁ svanāni ca\thinspace{\dandab} \dontdisplaylinenum

śaṅkhakāhalakāṁsyānāṁ śabdāni vividhāni ca \veg\dontdisplaylinenum


\alalalfejezet{ākāśaḥ}
\varr{
        \ \va °tantrīṇāṁ\lem  \mssCaCbCc\msNa\Ed; °tantīnāṁ \msNb
        \ \vb dundubhīnāṁ\lem  \mssCaCbCc\msNa\Ed; dundubhīnā \msNb\oo
                 svanāni\lem  \msCb\msCc\msNa\msNb\Ed; stanāni \msCa
        \ \vcd °kāhalakāṁsyānāṁ śabdāni\lem  \msNa\msNb\Ed; °kāhalakāṁsyā\uncl{nāṁ} {\lost}{\lost}ni \msCa, 
                                °kāhalakāsyānāṁ śabdāni \msCb, °kāṁsyānāṁ śabdāni \msCc}

ākāśadhātu viprendra ś\textsubring{r}ṇu vakṣyāmi te daśa\thinspace{\dandab} \dontdisplaylinenum

pāyūpasthodara kaṇṭha śaṅkhalau mukha nāsikau \veg\dontdisplaylinenum
\varr{
        \ \va °dhātu\lem  \msCb\msCc\msNa\msNb\Ed; °dhātuṁ \msCa
        \ \vc °dara\lem  \mssCaCbCc\msNa\msNb; °daraḥ \Ed
        \ \vd śaṅkhalau\lem  \mssCaCbCc\msNa\msNb; śrotau ca \Ed}

h\textsubring{r}diṁ ca daśamaṁ jñeyaṁ deha ākāśasambhavaḥ\thinspace{\dandab} \dontdisplaylinenum

punar anyat pravakṣyāmi tac ch\textsubring{r}ṇuṣva dvijottama \veg\dontdisplaylinenum
\varr{
        \ \va h\textsubring{r}diṁ\lem  \mssCaCbCc\msNa\msNb; h\textsubring{r}diś \Ed\oo
                 daśamaṁ\lem  \msCa\msCb\msNa\msNb\Ed; daśama \msCc
        \ \vc anyat pra°\lem  \mssCaCbCc\Ed; anyaṁ pra° \msNa, anya pra° \msNb
        \ \vd dvijottama\lem  \mssCaCbCc\msNa\Ed; jijottama \msNb}

daśa dhātuguṇā jñeyāḥ pañcabhūtaḥ p\textsubring{r}thak p\textsubring{r}thak\thinspace{\dandab} \dontdisplaylinenum

ākāśasya guṇāḥ śabdo vyāpitvaṁ chidratāpi ca \veg\dontdisplaylinenum
            \paral{\textit{\vcd {\normalfont  \kb\ MBh 12.247.7ab: }ākāśasya guṇaḥ śabdo vyāpitvaṁ chidratāpi ca}}
\varr{
        \ \vb °bhūtaḥ\lem  \msCa\msCc\msNa\msNb\Ed; °bhūta \msCb
        \ \vc ākāśasya\lem  \msCb\msCc\msNa\msNb\Ed; ākāśa{\lost} \msCa
        \ \vd vyāpitvaṁ\lem  \msCa\msCc\msNa\msNb\Ed; vyāpitvāṁ \msCb}

anāśrayanirālambam avyaktam avikāritā\thinspace{\dandab} \dontdisplaylinenum

apratīghātitā caiva bhūtatvaṁ prak\textsubring{r}tāni ca \veg\dontdisplaylinenum
            \paral{\textit{\vo {\normalfont  \kb\ MBh 12.247.7cd--8ab: }
                         anāśrayam anālambam avyaktam avikāritā{\thinspace\ketdanda}
                         apratīghātatā caiva bhūtatvaṁ vik\textsubring{r}tāni ca{\thinspace\danda}}}


\alalfejezet{vāyuḥ sparśaś ca (19-18)}\varr{
        \ \vc apratīghātitā\lem  \msCa\msNa\msNb\Ed; apratīghātatā \msCb\msCc}

ākāśadhātor viprendra tato vāyusamudbhavaḥ\thinspace{\dandab} \dontdisplaylinenum

śabdapūrvaguṇaṁ g\textsubring{r}hya vāyoḥ sparśaguṇaḥ sm\textsubring{r}taḥ \veg\dontdisplaylinenum
\varr{
        \ \va °dhātor vi°\lem  \msCb; °dhāto vi° \msCa\msCc\msNa\msNb\Ed
        \ \vc śabda°\lem  \mssCaCbCc\msNa\Ed; śabdaḥ \msNb\oo
                 °pūrva°\lem  \msCa\msCc\msNa\msNb\Ed; °pūrvaṁ \msCb}

śabda pūrvaṁ mayākhyātaṁ ś\textsubring{r}ṇu sparśaṁ dvijottama\thinspace{\dandab} \dontdisplaylinenum

kaṭhinaś cikkaṇaḥ ślakṣṇo m\textsubring{r}dusnigdhakharadravāḥ \veg\dontdisplaylinenum
            \paral{\textit{\vcd {\normalfont  \kb MBh 12.177.34ab: } kaṭhinaś cikkaṇaḥ ślakṣṇaḥ picchalo m\textsubring{r}dudāruṇaḥ}}
\varr{
        \ \va pūrvaṁ\lem  \msCa\msCb\msNa\msNb\Ed; pūrva \msCc
        \ \vb sparśaṁ dvijottama\lem  \msCc\msNa; sparśa{\lost}jottama \msCa, sparśaṁ dvijottamaḥ \msCb,
                                                        sparśa dvijottama \msNb\Ed
        \ \vc cikkaṇaḥ\lem  \corr; cikkanaḥ \msCa\msCb\msNa\msNb, cikkalaḥ \msCc, cikkaraḥ \Ed
        \ \vd °snigdha°\lem  \msCa\msCc\msNa\msNb\Ed; °śnidha° \msCb}

karkaśaḥ paruṣas tīkṣṇaḥ śītoṣṇa daśa ca dvayam\thinspace{\dandab} \dontdisplaylinenum

iṣṭāniṣṭadvayasparśa vapuṣā parig\textsubring{r}hyate \veg\dontdisplaylinenum
            \paral{\textit{\vc {\normalfont  Folio 309v in \msCc\ ends with } iṣṭāniṣṭadvaya {\normalfont  and the next folio is missing.
                  \msCc\ resumes on folio 311r with 20.51c }(mānsañ ca medañ ca)}}


\alalalfejezet{prāṇāḥ}
\varr{
        \ paruṣas tīkṣṇaḥ\lem  \msCb\msNb; paruṣas trīkṣṇaś \msCa, 
                                tīkṣṇaḥ \msNaacorr, paruṣā tīkṣṇaḥ \msNapcorr, paruṣas tīkṣṇa \msCc\Ed
        \ \vb dvayam\lem  \msNa; dvaya \msCa\msCc\msNb\Ed, dvayaḥ \msCb
        \ \vc °dvaya°\lem  \mssCaCbCc\msNa\Ed; °dvayo \msNb
        \ \vd °g\textsubring{r}hyate\lem  \msCa\msNa\msNb\Ed; °g\textsubring{r}hate \msCb}

prāṇo 'pānaḥ samānaś ca udāno vyāna eva ca\thinspace{\dandab} \dontdisplaylinenum

nāgakūrmo 'tha k\textsubring{r}karo devadatto dhanaṁjayaḥ \veg\dontdisplaylinenum
            \paral{\textit{\vo {\normalfont  The next XX verses are parallel to a
                         passage in the B\textsubring{r}hatkālottara (NGMPP Reel No.\ B 29/59 Manuscript No.\ pra - 89): }
        prāṇopānaḥ samānaś ca udāno vyāna eva ca{\thinspace\ketdanda}
        nāgaḥ kurmodhva k\textsubring{r}karo devadattadhanaṁyayau{\thinspace\danda}
        prāṇas tu prathamo vāyur daśānām api sa prabhuḥ{\thinspace\ketdanda}
        prāṇaḥ prāṇamayaḥ prāṇa visargāpūraṇaṁ prati{\thinspace\danda}
        nityam āpūrayaty eṣa prāṇinām urasi sthitaḥ{\thinspace\ketdanda}
        niśvāsocchvāsakāmais tu prāṇo jīvasamāśritaḥ{\thinspace\danda}
        prayāṇaṁ kurute yasmāt tasmāt prāṇa prakīrtitaḥ{\thinspace\ketdanda}
        apānasahāpānas tu āhāraṁ ca n\textsubring{r}ṇām adhaḥ{\thinspace\danda}
        mūtraśukravahovāyur apānas tena kīrtitaḥ{\thinspace\ketdanda}
        pītaṁ bhakṣitam āghrātaṁ raktapitakaphānilaṁ{\thinspace\danda}
        samaṁ nayati mātreṣu samāno nāma mārutaḥ{\thinspace\ketdanda}
        spadaṁyabhyadharaṁ vaktraṁ netragātra prakopanaḥ{\thinspace\danda}
        udvejayati marmāṇi udāto nāma mārutaḥ{\thinspace\ketdanda}
        vyāno vināmayatyaṁgaṁ vyāno vyādhiprakopakaḥ{\thinspace\danda}
        prītecināsī kathito vāddhikyāt vyāna ucyate{\thinspace\ketdanda}
        {\normalfont ...
         cf.\ also Sārdhatriśatikālottara, Agnipurāṇa and Dīpikā by Aghoraśivācārya on the M\textsubring{r}gendra } }}
\varr{
        \ \va 'pānaḥ\lem  \msCa\msNa\Ed; pāna° \msCb\msNb
        \ \vc nāga°\lem  \msCb\msNa\msNb\Ed; nāma° \msCa\oo
                 k\textsubring{r}karo\lem  \msCa\msCb\msNa\msNb; k\textsubring{r}kalo \Ed}

daśa vāyupradhānaite kīrtitā dvijasattama\thinspace{\dandab} \dontdisplaylinenum

dhanaṁjayo bhaved ghoṣo devadatto vij\textsubring{r}mbhakaḥ \veg\dontdisplaylinenum
\varr{
        \ \vb kīrtitā\lem  \msCb\msNa\msNb; \uncl{kīrtti}tā \msCa, kīrtitāḥ \Ed
        \ \vc bhaved ghoṣo\lem  \msCa\msCb\msNb\Ed; bhaved yoṣo \msNa}

k\textsubring{r}karaḥ kṣudhak\textsubring{r}n nityaṁ kūrmonmīlitalocanaḥ\thinspace{\dandab} \dontdisplaylinenum

nāga udghāṭanaṁ puṣyaṁ karoti satataṁ dvija \veg\dontdisplaylinenum
\varr{
        \ \va k\textsubring{r}karaḥ\lem  \msCa\msCb\msNa\msNb; k\textsubring{r}kara \Ed\oo
                 °k\textsubring{r}n nityaṁ\lem  \msCa\msNa\msNb\Ed; k\textsubring{r}n nitya \msCb
        \ \vb kūrmonmīlitalocanaḥ\lem  \msCa\msNa\Ed; karmolmīnalocanaḥ \msCb, kūrmonmīnalocanaḥ \msNb
        \ \vc puṣyaṁ\lem  \msCa\msCb\msNa\Ed; punsāṁ \msNb
        \ \vd dvija\lem  \msCa\msCb\msNa\Ed; dvijaḥ \msNb}

prāṇaḥ śvasati bhūtānāṁ niśvasanti ca nityaśaḥ\thinspace{\dandab} \dontdisplaylinenum

prayāṇaṁ kurute yasmāt tasmāt prāṇa iti sm\textsubring{r}taḥ \veg\dontdisplaylinenum
\varr{
        \ \va prāṇaḥ\lem  \msCa\msCb\msNa\msNb; prāṇāḥ \Ed
        \ \vb nityaśaḥ\lem  \msCa\msCb\msNa\msNb; nitya yaḥ \Ed
        \ \vc prayāṇaṁ\lem  \msCa\msCb\msNa\msNb; prayāṇā \Ed}

apanayaty apānas tu āhāraṁ manujām adhaḥ\thinspace{\dandab} \dontdisplaylinenum

śukramūtravaho vāyur apānas tena kīrtitaḥ \veg\dontdisplaylinenum
\varr{
        \ \va apanaya°\lem  \msCb\msNa\msNb\Ed; a\uncl{pa} {\lost} ya° \msCa
        \ \vb āhāraṁ manujām adhaḥ\lem  \msCa\msCb; āhāraṁ manujādhamaḥ \msNa,
                                        āhāra manujādhamaḥ \msNb, āhāraṁ manujāpavaḥ \Ed
        \ \vd °pānas tena\lem  \msCa\msNa\msNb\Ed; °vānas tena \msCb}

pītabhakṣitam āghrātaṁ raktapittakaphānilam\thinspace{\dandab} \dontdisplaylinenum

samaṁ nayati gātreṣu samāno nāma mārutaḥ \veg\dontdisplaylinenum
\varr{
        \ \va °ghrātaṁ\lem  \msCa\msCb\msNb\Ed; °ghrāti \msNa
        \ \vb raktapitta°\lem  \msCa\msCb\msNa\Ed; raktaḥ pittaḥ \msNb}

spandayaty adharaṁ vaktraṁ netragātraprakopanam\thinspace{\dandab} \dontdisplaylinenum

udvejayati marmāṇi udāno nāma mārutaḥ \veg\dontdisplaylinenum
\varr{
        \ \vo (spandayaty adharaṁ{\normalfont ...} mārutaḥ)\lem  \msCa\msCb\msNa\Ed; \om\ \msNb
        \ \va °dharaṁ\lem  \msCa\msCb\msNa; \om\ \msNb, °dhara° \Ed\ \unmetr
        \ \vb °gātrapra°\lem  \msCa\msNa\Ed; °gātram pra° \msCb, \om\ \msNb
        \ \vc marmāṇi\lem  \msCa\msCb\msNa; \om\ \msNb, karmāṇi \Ed
        \ \vd udāno nāma\lem  \msCb\msNa\Ed; \uncl{u}{\lost}{\lost}{\lost}{\lost} \msCa, \om\ \msNb}

vyāno vināmayaty aṅgaṁ vyaṅgo vyādhiprakopanaḥ\thinspace{\dandab} \dontdisplaylinenum

prītivināśakathitaṁ vārdhikyaṁ vyāna ucyate \veg\dontdisplaylinenum
\varr{
        \ \va vyāno vi°\lem  \msCa\msCb\msNa\Ed; vyāno pi \msNb
        \ \vb °kopanaḥ\lem  \msCa\msCb\msNa\Ed; °kopamaḥ \msNb
        \ \vc prītivi°\lem  \msCa\msCb\msNa; prītir vi° \msNb\Ed}

daśavāyuvibhāge ca kīrtito me dvijottama\thinspace{\dandab} \dontdisplaylinenum

daśavāyuguṇāṁś cānyāṁ ch\textsubring{r}ṇu kīrtayato mama \veg\dontdisplaylinenum
\varr{
        \ \vb me\lem  \msCa\msCb\msNa\msNb; ye \Ed
        \ \vcd °vāyuguṇāṁś cānyāṁ ś\textsubring{r}ṇu\lem  \msCa\msCb\msNa; °vāyuguṇāś cānyaṁ ś\textsubring{r}ṇu \msNb,
                        °dhātuguṇāś cānyac ch\textsubring{r}ṇu \Ed
        \ \vd kīrtayato mama\lem  \msCa\msNa\msNb; kīrttiyato mama \msCb, kīrtaya me dvija \Ed}

vāyor aniyama sparśo vādasthānaṁ svatantratā\thinspace{\dandab} \dontdisplaylinenum

balaṁ śīghraṁ ca mokṣaṁ ca ceṣṭā karmātmanā bhavaḥ \veg\dontdisplaylinenum
            \paral{\textit{\vo {\normalfont  \kb\ MBh 12.247.6: }
                        vāyor aniyamaḥ sparśo vādasthānaṁ svatantratā{\thinspace\danda}
                        balaṁ śaighryaṁ ca mohaś ca ceṣṭā karmak\textsubring{r}tā bhavaḥ{\thinspace\ketdanda}}}


\alalfejezet{tejo rūpaś ca (17-16)}\varr{
        \ \vb vādasthānaṁ\lem  \eme; vāta{\lost}ne \msCa, vātasthāne \msNa\msNb\Ed}

vāyunāpi s\textsubring{r}jas tejas tadrūpaṁ guṇam ucyate\thinspace{\dandab} \dontdisplaylinenum

śabdasparśasama jyotis triguṇaṁ samudāh\textsubring{r}tam \veg\dontdisplaylinenum
\varr{
        \ \vab s\textsubring{r}jas tejas ta°\lem  \msCa\msCb\msNa\Ed; s\textsubring{r}jatvejatta° \msNb
        \ \vb °rūpaṁ guṇa°\lem  \msCa\msCb\msNa\msNb; °rūpaguṇa° \Ed
        \ \vcd °jyotis tri°\lem  \msCa\msNa\msNb\Ed; °jyotitri° \msCb}

śabdaḥ sparśaḥ purā proktaḥ ś\textsubring{r}ṇu rūpaguṇaṁ tataḥ\thinspace{\dandab} \dontdisplaylinenum

hrasvaṁ dīrgham aṇu sthūlaṁ v\textsubring{r}ttamaṇḍalam eva ca \veg\dontdisplaylinenum
\varr{
        \ \va śabdaḥ\lem  \msNa\Ed; śabda° \msCa\msCb\msNb\oo
                 proktaḥ\lem  \msCa\msNa\msNb; proktāḥ \msCb\Ed
        \ \vb rūpaguṇaṁ\lem  \msCa\msCb\msNb\Ed; rūpaṁ guṇaṁ \msNa
        \ \vc hrasvaṁ\lem  \msCa\msNa; hrasva° \msCb\msNb\Ed\oo
                 dīrgham aṇu\lem  \msCa\msCb\msNa; °dīrgham anu \msNb, °dīrghalaghu \Ed\oo
                 sthūlaṁ\lem  \msCa\msCb\msNa\Ed; sthūla \msNb
        \ \vd °maṇḍalam eva\lem  \msCa\msNa\Ed; °maṇḍam eva \msCb}

caturasraṁ dvirasraṁ ca tryasraṁ caiva ṣaḍasrakam\thinspace{\dandab} \dontdisplaylinenum

śuklaḥ k\textsubring{r}ṣṇas tathā rakto nīlaḥ pīto 'ruṇas tathā \veg\dontdisplaylinenum
            \paral{\textit{\vcd {\normalfont  = MBh 12.177.32cd } }}
\varr{
        \ \va caturasraṁ dvirasraṁ\lem  \msCb\msNa;
                                 caturaśran dvi{\lost}śraṁ \msCa,
                                 caturasradvirasraś \Ed
        \ \vb tryasraṁ\lem  \msCa\msCb\msNa; tisraś \Ed
        \ \vc śuklaḥ\lem  \msCa\msCb\msNa; śuklaṁ \Ed
        \ \vd nīlaḥ\lem  \msCa\msCb\msNa; nīla° \Ed}

śyāmaḥ piṅgala babhruś ca nava raṅgāḥ prakīrtitāḥ\thinspace{\dandab} \dontdisplaylinenum

navadhā navaraṅgānām ekāśīti guṇāḥ sm\textsubring{r}tāḥ \veg\dontdisplaylinenum
\varr{
        \ \va śyāmaḥ piṅgala babhruś ca\lem  \Ed; 
                        śyāmaḥ piṅgalo babhruś ca \msCa\msCb,
                       śyāmaś ca piṅgalo babhruś ca \msNaacorr,
                       śyāma piṅgalo bhruś ca \msNapcorr
        \ \vb raṅgāḥ\lem  \msCa\msCb\msNa; raṅgaḥ \Ed
        \ \vd sm\textsubring{r}tāḥ\lem  \msCa\msCb\msNa; sm\textsubring{r}taṁ \Ed}

tejodhātu daśa brūmaḥ ś\textsubring{r}ṇuṣvāvahito bhava\thinspace{\dandab} \dontdisplaylinenum

kāmas tejo kṣaṇaḥ krodho jaṭharāgniś ca pañcamaḥ \veg\dontdisplaylinenum
\varr{
        \ \va tejodhātu daśa\lem  \msCa\msCb\msNa; 
                                tejodhātur daśaṁ \Ed
        \ \vc tejo kṣaṇaḥ\lem  \msCa; tejaḥ kṣaṇaḥ \msCb, teja kṣaṇaḥ \msNa, tejekṣaṇaḥ \Ed
        \ \vd jaṭharāgniś ca\lem  \msNa\Ed; jaṭha{\lost}gniś ca \msCa}

jñānaṁ yogas tapo dhyānaṁ viśvāgnir daśamaḥ sm\textsubring{r}taḥ\thinspace{\dandab} \dontdisplaylinenum

daśa tejoguṇāṁś cānyān pravakṣyāmi dvijottama \veg\dontdisplaylinenum
\varr{
        \ \vb viśvāgnir da°\lem  \msCa\msCb\Ed; viśvāgni da° \msNa
        \ \vc daśa tejoguṇāṁś cā°\lem  \msCa\msNa; daśa tejoguṇāś cā° \msCb, daṁśatejoguṇāś cā° \Ed}

agner durdharṣatāpnoti tāpapākaprakāśanaḥ\thinspace{\dandab} \dontdisplaylinenum

śaucaṁ rāgo laghus taikṣṇyaṁ daśamaṁ cordhvabhāgitā \veg\dontdisplaylinenum
            \paral{\textit{\vcd {\normalfont  \kb\ MBh 12.247.5cd: } 
                agner durdharṣatā tejas tāpaḥ pākaḥ prakāśanam{\thinspace\danda}
                śaucaṁ rāgo laghus taikṣṇyaṁ daśamaṁ cordhvabhāgitā{\thinspace\ketdanda}}}


\alalfejezet{āpo rasaś ca (15-14)}\varr{
        \ \va agner durdharṣatāpnoti\lem  \conj; agner durddhaṣatāpnoti \msCa,
                                agne durddhaṣatāpnoti \msCb\msNa, agner durdharṣavāpnoti \Ed
        \ \vc rāgo\lem  \msCa\msCb\msNa; gaṅgā \Ed\oo
                 laghus taikṣṇyaṁ\lem  \corr; laghus taikṣṇaṁ \msCa\msNa\Ed, laghus tīkṣṇaṁ \msCb
        \ \vd daśamaṁ cordhvabhāgitā\lem  \eme; daśapañcorddhabhāṣitam \msCa, daśamaṁ cordhabhāṣitam \msCb\msNa, 
                                                                daśamaś cordhabhāṣitam \Ed}

jyotiso 'pi s\textsubring{r}jaś cāpaḥ saraso guṇasaṁyutaḥ\thinspace{\dandab} \dontdisplaylinenum

caturguṇāḥ sm\textsubring{r}tā āpaḥ vijñeyā ca manīṣibhiḥ \veg\dontdisplaylinenum
\varr{
        \ \va jyotiso\lem  \msCb; jyotiḥ so \msCa\msNa\Ed\oo
                 s\textsubring{r}jaś cāpaḥ\lem  \msCb; s\textsubring{r}jaś cāpi \msCa\msNa\Ed
        \ \vd vijñeyā ca manīṣibhiḥ\lem  \Ed; \om\ \msCa\msCb\msNa}

śabdaḥ sparśaś ca rūpaṁ ca rasaś ca sa caturguṇaḥ\thinspace{\dandab} \dontdisplaylinenum
            \paral{\textit{\vab \kb\ {\normalfont  MBh 12.299.11ab: } 
                śabdaḥ sparśaś ca rūpaṁ ca raso gandhaś ca pañcamaḥ}}

rūpādiguṇa pūrvokta adhunātha rasaṁ ś\textsubring{r}ṇu \veg\dontdisplaylinenum
\varr{
        \ \va rūpaṁ\lem  \msCa\msCb\msNa; rūpaś \Ed
        \ \vc pūrvokta\lem  \msCa\msCb\msNa; pūrvoktaṁ \Ed}

kaṭutiktakaṣāyāś ca lavaṇāmlas tathaiva ca\thinspace{\dandab} \dontdisplaylinenum

madhuraś ca rasān ṣaḍ vai pravadanti manīṣiṇaḥ \veg\dontdisplaylinenum
\varr{
        \ \va lavaṇāmlas ta°\lem  \msCa\msCb\msNa; lavaṇāntas ta° \Ed
        \ \vc rasān ṣaḍ vai\lem  \corr; rasāṁ ṣaḍ vai \msCa, rasā ṣaḍ vai \msNa\msCb\Ed}

ṣaḍrasāḥ ṣaḍvibhedena ṣaṭtriṁśaguṇa ucyate\thinspace{\dandab} \dontdisplaylinenum

āpadhātu daśa tv anyān ś\textsubring{r}ṇu kīrtayato mama \veg\dontdisplaylinenum
\varr{
        \ \va °rasāḥ\lem  \msCa\msCb\msNa; °rasā \Ed\oo
                 ṣaḍvibhedena\lem  \msCa\msCb\Ed; ṣaḍbhir bhedena \msNa
        \ \va āpa°\lem  \msCb\msNa\Ed; \uncl{ā}pa° \msCa\oo
                 daśa tv anyān\lem  \corr; daśa tv anyāṁ \msCa\msNa, daśatvaṁnyāṁ \msCb, daśa tv anyā \Ed
        \ \vb kīrtayato\lem  \msCa\msNa\Ed; kīrttiyato \msCb}

lālā siṅghāṇikā śleṣmā raktaḥ pittaḥ kaphas tathā\thinspace{\dandab} \dontdisplaylinenum

svedam aśru rasaś caiva medaś ca daśamaḥ sm\textsubring{r}taḥ \veg\dontdisplaylinenum
\varr{
        \ \va lālā\lem  \msCa\msNa\Ed; lalāṁ \msCb\oo
                 siṅghāṇikā\lem  \corr; sighānikā \msCa, si\uncl{ghā}nikā \msCb, siṁghānikā \msNa\Ed\oo 
                 śleṣmā\lem  \msCa\msCb\msNa; śoṣmā \Ed
        \ \vb raktaḥ\lem  \msCa\msNa; rakta° \msCb\Ed
        \ \vc rasaś caiva\lem  \msCa\msCb\Ed; rasaṁś caiva \msNa
        \ \vd medaś ca\lem  \msCa\msCb\msNa; medaṁ ca \Ed\oo
                 daśamaḥ\lem  \msCa\msNa\Ed; madaḥ \msCbacorr, madanaḥ \msCbpcorr}

daśa āpaguṇāś cānye kīrtayiṣyāmi tān ś\textsubring{r}ṇu\thinspace{\dandab} \dontdisplaylinenum

adbhya śaityaṁ rasa kledo dravatvaṁ snehasaumyatā \veg\dontdisplaylinenum
\varr{
        \ \va daśa āpa°\lem  \msCa\msCb\msNa; daśaś cāpa° \Ed\oo
                 cānye\lem  \msCa\msCb; cānyā \msNa\Ed
        \ \vb tān\lem  \Ed; tāṁ \msCa\msCb\msNa
        \ \vc adbhya śaityaṁ\lem  \msCa; aṅgaśaityaṁ \msCb, aṅgyaśaityaṁ \msNa, agnyaśaitya° \Ed}

jihvā viṣyandinī caiva bhaumānyaśravaṇādhamaḥ\thinspace{\dandab} \dontdisplaylinenum
            \paral{\textit{\vo {\normalfont \kb\ MBh 12.247.4 (with } adbhyaḥ{\normalfont  as a variant in the critical edition): }
                apāṁ śaityaṁ rasaḥ kledo dravatvaṁ snehasaumyatā{\thinspace\danda}
                jihvā viṣyandinī caiva bhaumāpyāsravaṇaṁ tathā{\thinspace\ketdanda}}}


\alalfejezet{bhūmir gandhaś ca (13-12)}
āpaś cāpy as\textsubring{r}jad bhūmis tasyā gandhaguṇaḥ sm\textsubring{r}taḥ \veg\dontdisplaylinenum
\varr{
        \ \va viṣyandinī\lem  \msCb\msNa; °vi\uncl{ṣ}{\lost}{\lost}nī \msCa, °niṣpandinī \Ed
        \ \vb bhaumānyaśravaṇādhamaḥ\lem  \msCa\msCb\msNa; bhaumān daśaguṇāñ ś\textsubring{r}ṇu \Ed
        \ \vc āpaś cāpy as\textsubring{r}jad bhū°\lem  \msCb;
                                āpaś cāpījyajā bhū° \msCa\msNa, āpaś ca bījyajā bhū° \Ed }

caturāpaguṇān g\textsubring{r}hya bhūmer gandhaguṇaḥ sm\textsubring{r}taḥ\thinspace{\dandab} \dontdisplaylinenum

śabdaḥ sparśaś ca rūpaṁ ca raso gandhaś ca pañcamaḥ \veg\dontdisplaylinenum
\varr{
        \ \va °guṇān g\textsubring{r}°\lem  \msCa\Ed; °guṇaṁ g\textsubring{r}° \msCb, °guṇā g\textsubring{r}° \msNa
        \ \vc rūpaṁ ca\lem  \msCa\msNa; rūpaś ca \msCb\Ed
        \ \vd pañcamaḥ\lem  \msCa\msCb\msNa; pañcama \Ed}

āpaḥpūrvaguṇāḥ proktā bhūmer gandhaguṇaṁ ś\textsubring{r}ṇu\thinspace{\dandab} \dontdisplaylinenum

iṣṭāniṣṭadvayor gandhaḥ surabhir durabhis tathā \veg\dontdisplaylinenum
\varr{
        \ \va āpaḥ°\lem  \msCa\msCb\msNa; āpa° \Ed\oo
                 proktā\lem  \msCa\msCb\msNa; prokto \Ed
        \ \vb bhūmer ga°\lem  \msCa\msCb; bhūme ga° \msNa, bhūmir ga° \Ed\oo
                 ś\textsubring{r}ṇu\lem  \msCa\msCb\msNa; sm\textsubring{r}ta \Ed
        \ \vc dvayor gandhaḥ\lem  \msCb\msNa\Ed; dvayo{\lost}{\lost} \msCa}

karpūraḥ kasturīkaṁ ca candanāgarum eva ca\thinspace{\dandab} \dontdisplaylinenum

kuṅkumādisugandhāni ghrāṇam iṣṭaṁ prakīrtitam \veg\dontdisplaylinenum
\varr{
        \ \va kasturīkaṁ ca\lem  \msCa\msCb\msNa; kastūrīkaś ca \Ed\ \unmetr
        \ \vb °garu°\lem  \msCa\msCb\msNa; °guru° \Ed
        \ \vd ghrāṇam iṣṭaṁ kīrtitam\lem  \msCa\msCb\msNapcorr; \om\ \msNaacorr, ghrāṇam iṣṭaṁ kīrtitaḥ \Ed}

viṅmūtrasvedagandhāni vaktragandhaṁ ca duḥsaham\thinspace{\dandab} \dontdisplaylinenum

jīrṇasphoṭitagandhāni aniṣṭānīti kīrtitam \veg\dontdisplaylinenum
\varr{
        \ \va viṅmūtrasvedagandhāni\lem  \msCa\msCb\msNapcorr\Ed; \om\ \msNaacorr
        \ \vb °gandhaṁ ca\lem  \msCa\msNa; °gandhaś ca \Ed
        \ \vc °sphoṭita°\lem  \msCa\msNa; °sphuṭita° \msCb, °sphoṭaka° \Ed
        \ \vd kīrtitam\lem  \msCa\msCbpcorr\msNa\Ed; kītam \msCbacorr}

bhūmer dhātu daśa tv anyān kathayiṣyāmi tac ch\textsubring{r}ṇu\thinspace{\dandab} \dontdisplaylinenum

tvacaṁ māṁsaṁ ca medaṁ ca snāyu majjā sirā tathā \veg\dontdisplaylinenum
            \paral{\textit{\vc {\normalfont \msCc\ resumes here with } mānsañ ca medañ ca}}
\varr{
        \ \va bhūmer dhā°\lem  \msCa\msCb\msNa; bhūme dhā° \Ed
        \ \vab tv anyān ka°\lem  \msCa; tv anyāṁ \msCb\msNa, tv anyā ka° \Ed
        \ \vb tac ch\textsubring{r}ṇu\lem  \msCb\msNa\Ed; \uncl{ta}{\lost}ṇu \msCa
        \ \vc tvacaṁ māṁsaṁ ca medaṁ ca\lem  \msCa; tvacaṁ māṁsañ ca \msCb, {\lost}{\lost} mānsañ ca medañ ca \msCc,
                                tvacaṁ māsaṁ ca medaṁ ca \msNa, tvacā māṁsaś ca medaś ca \Ed
        \ \vd snāyu\lem  \msCc\msNa\Ed; śnāyuṁ \msCa\msCb\oo
                 sirā tathā\lem  \eme;  śirās tathā \msCa\msNa, śiras tathā \msCb\msCc\Ed}

nakhadantaruhāś caiva keśaś ca daśamas tathā\thinspace{\dandab} \dontdisplaylinenum

daśa tv anyān pravakṣyāmi ś\textsubring{r}ṇu bhūmiguṇān dvija \veg\dontdisplaylinenum
\varr{
        \ \vb keśa°\lem  \mssCaCbCc\msNa; keśā° \Ed
        \ \vc tv anyān pra°\lem  \Ed; tv anyām pra° \msCa, tv anyāṁ pra° \msCb\msCc\msNa
        \ \vd °guṇān dvi°\lem  \msCa\msNa\Ed; °guṇā dvi° \msCb, °guṇāṁ div° \msCc}

bhūmeḥ sthairyaṁ rajastvaṁ ca kāṭhinyaṁ prasavātmakam\thinspace{\dandab} \dontdisplaylinenum

gandho guruś ca śaktiś ca nīhārasthāpanāk\textsubring{r}tiḥ \veg\dontdisplaylinenum
            \paral{\textit{\vo {\normalfont \kb\ MBh 12.247.3: }
                        bhūmeḥ sthairyaṁ p\textsubring{r}thutvaṁ ca kāṭhinyaṁ prasavātmatā{\thinspace\danda}
                        gandho gurutvaṁ śaktiś ca saṁghātaḥ sthāpanā dh\textsubring{r}tiḥ{\thinspace\ketdanda}}}
\varr{
        \ \va bhūmeḥ\lem  \msCa\msCc\msNa\Ed; bhūmiḥ \msCb\oo
                 sthairyaṁ\lem  \msCa\msCb\msNa\Ed; sthairya \msCc\oo
                 rajastvaṁ ca\lem  \mssCaCbCc\msNa; rajatvaś ca \Ed
        \ \vb kāṭhinyaṁ\lem  \mssCaCbCc\msNa; kaṭhinyaṁ \Ed
        \ \vd °k\textsubring{r}tiḥ\lem  \msCa\msCb\msNa\Ed; °k\textsubring{r}ti \msCc}

guṇadhātuviśeṣaś ca utpattiś ca dvijottama\thinspace{\dandab} \dontdisplaylinenum

yathā śrutaṁ mayā pūrvaṁ kīrtitaṁ nikhilena tu \veg\dontdisplaylinenum


\alalfejezet{buddhīndriyāṇi karmendriyāṇi ca (11-2)}\varr{
        \ \va guṇadhātu°\lem  \msNa; \uncl{guṇandhā}tu° \msCa, guṇātvātu° \msCb, guṇaṁ dhātu° \msCc\Ed
        \ \vcd pūrvaṁ kīrtitaṁ\lem  \msCa\msCc\msNa\Ed; pūrva kīrtita \msCb}

vaikārikam ahaṁkāraṁ sattvodriktāt tu sāttvikaḥ\thinspace{\dandab} \dontdisplaylinenum
            \paral{\textit{\vab {\normalfont \kb\ Liṅgapurāṇa 1.70.38cd (= Śivapurāṇa 7.1.10.14cd): }
                vaikārikād ahaṁkārāt sattvodriktāt tu sāttvikāt}}

śrotraṁ tvak cakṣuṣī jihvā nāsikā caiva pañcamī \veg\dontdisplaylinenum
\varr{
        \ \vab sattvodriktāt tu\lem  \corr;
                                 sattvod\textsubring{r}ktāt tu \msCa, sattvonuktānu \Ed}

buddhīndriyāṇi pañcaiva kīrtitāni dvijottama\thinspace{\dandab} \dontdisplaylinenum

hastapādas tathā pāyur upastho vāk ca pañcamaḥ \veg\dontdisplaylinenum


\alalalfejezet{śrotram (11)}
\varr{
        \ \vc pāyu°\lem  \msCa; snāyu° \Ed
        \ \vd °pastho vāk ca\lem  \Ed; °pa\uncl{stho vā}{\lost} \msCa\oo
                 pañcamaḥ\lem  \msCa; pañcamam \Ed}

śrotreṇa g\textsubring{r}hyate śabdo vividhas tu dvijottama\thinspace{\dandab} \dontdisplaylinenum

veṇuvīṇāsvanānāṁ ca tantrīśabdam anekadhā \veg\dontdisplaylinenum

muraja saunda paṇavabherīpaṭahanisvanam\thinspace{\dandab} \dontdisplaylinenum

śaṅkhakāhalaśabdaṁ ca śabdaṁ ḍiṇḍimagomukham \veg\dontdisplaylinenum
\varr{
        \ \va muraja\lem  \Ed; murava \msCa\oo
                 saunda\lem  \Ed; maunda \msCa}

kāṁsikāhalamiśraṁ ca gītāni vividhāni ca\thinspace{\dandab} \dontdisplaylinenum


\alalalfejezet{tvak (10)}

tvacayā g\textsubring{r}hyate sparśaḥ sukhaduḥkhasamanvitaḥ \veg\dontdisplaylinenum
\varr{
        \ \va °kāhala°\lem  \Ed; °kātāla° \msCa
        \ \vc g\textsubring{r}hyate\lem  \Ed; g\textsubring{r}hya{\lost} \msCa}

m\textsubring{r}dusūkṣmasukhaṁ sparśaḥ vastraśayyāsanādayaḥ\thinspace{\dandab} \dontdisplaylinenum

tīkṣṇaśastrajalaṁ śaitya uṣṇataptakṣatekṣaraḥ \veg\dontdisplaylinenum
\varr{
        \ \va °sukhaṁ\lem  \Ed; °sukha° \msCa\ \unmetr
        \ \vc śaitya\lem  \Ed; śaitye \msCa}

evamādīny anekāni jñeyānīṣṭaṁ dvijottama\thinspace{\dandab} \dontdisplaylinenum


\alalalfejezet{cakṣuḥ (9)}

cakṣuṣā g\textsubring{r}hyate rūpaṁ sahasrāṇi śatāni ca \veg\dontdisplaylinenum

devarūpavikārāṇi nakṣatragrahatārakāḥ\thinspace{\dandab} \dontdisplaylinenum

mānuṣānāṁ vikārāṇi grāmaṁ nagarapattanam \veg\dontdisplaylinenum

v\textsubring{r}kṣagulmalatānāṁ ca paśupakṣiśarīs\textsubring{r}pāṁ\thinspace{\dandab} \dontdisplaylinenum

k\textsubring{r}mikīṭapataṅgānāṁ jalajānām anekadhā \veg\dontdisplaylinenum

śailadāravaromāṇi rūpāṇi vividhāni ca\thinspace{\dandab} \dontdisplaylinenum

dhātudravyavikārāṇi rūpāṇi dvijasattama \veg\dontdisplaylinenum


\alalalfejezet{jihvā (8)}
\varr{
        \ \va °romāṇi\lem  \Ed; °homāni \msCa
        \ \vd dvijasattama\lem  \Ed; dvija\uncl{sa}{\lost}{\lost} \msCa}

jihvayā g\textsubring{r}hyate svādo h\textsubring{r}dyāh\textsubring{r}dyo dvijottama\thinspace{\dandab} \dontdisplaylinenum

phalamūlāni śākāni kandāni piśitāni ca \veg\dontdisplaylinenum
\varr{
        \ \va jihvayā\lem  \Ed; {\lost}{\lost}yā \msCa}

pakvāpakva viśeṣāṇi dadhikṣīragh\textsubring{r}tāni ca\thinspace{\dandab} \dontdisplaylinenum

vrīhyoṣadhirasānāṁ ca miśrāmiśram anekadhā \veg\dontdisplaylinenum
\varr{
        \ \vc °ṣadhi°\lem  \Ed; °ṣadha° \msCa}

ṣaṭkarmapratibhedena rasabhedaśata sm\textsubring{r}tam\thinspace{\dandab} \dontdisplaylinenum


\alalalfejezet{ghrāṇam (7)}

ghrāṇena g\textsubring{r}hyate gandha iṣṭāniṣṭo dvijarṣabhaḥ \veg\dontdisplaylinenum
\varr{
        \ \vb °śataṁ\lem  \msCa; °śata \Ed
        \ \vcd g\textsubring{r}hyate gandha iṣṭā°\lem  \Ed; g\textsubring{r}\uncl{hyate ga}{\lost}{\lost}ṣṭā° \msCa
        \ \vd °niṣṭo\lem  \msCa; °niṣṭā \Ed}

guḍājyaṁ guggulur bhaṣmacandanāgarukaṁ tathā\thinspace{\dandab} \dontdisplaylinenum

kastūrikuṅkumādīnām iṣṭo gandho manoharaḥ \veg\dontdisplaylinenum
\varr{
        \ \va guḍājyaṁ guggulur\lem  \msCa; guḍājyaguggulu \Ed
        \ \vb °garukaṁ\lem  \msCa; °gurukas \Ed
        \ \vd gandho\lem  \msCa; gandha \Ed}

vraṇamūtrapurīṣāṇāṁ māṁsaparyuṣitāni ca\thinspace{\dandab} \dontdisplaylinenum

vātakarmādidurgandha aniṣṭaḥ samudāh\textsubring{r}taḥ \veg\dontdisplaylinenum


\alalalfejezet{hastakarma (6)}
\varr{
        \ \vb māṁsa°\lem  \Ed; māsa° \msCa}

hastena kurute karma vividhāni dvijottama\thinspace{\dandab} \dontdisplaylinenum

māhendraṁ vāruṇaṁ caiva vāyavyāgneyam eva ca \veg\dontdisplaylinenum
\varr{
        \ \va hastena\lem  \msCa; hastābhyāṁ \Ed
        \ \vc māhendraṁ vāruṇaṁ\lem  \msCb; {\lost}{\lost}ndram vāruṇañ \msCa, mohendravāruṇaṁ \Ed}

āgneyapavanādīni kāṁsyo lohas trapus tathā\thinspace{\dandab} \dontdisplaylinenum

agnikarmāṇy anekāni yajñahomakriyās tathā \veg\dontdisplaylinenum
\varr{
        \ \va °pavanā°\lem  \Ed; °pacanā° \msCa}

sūryavyajanavātena mukhavātena vai tathā\thinspace{\dandab} \dontdisplaylinenum

camaracarmavātena vātayantraṁ ca vāyavam \veg\dontdisplaylinenum

vāruṇaṁ toyakarmāṇi kurute vividhāni ca\thinspace{\dandab} \dontdisplaylinenum

rasoparasakarmāṇi tasya poṣaṇakarma ca \veg\dontdisplaylinenum
\varr{
        \ \vb kurute\lem  \Ed; kuru{\lost} \msCa}

snānācamanakarmāṇi vastraśaucādayas tathā\thinspace{\dandab} \dontdisplaylinenum

kāyaśaucaṁ ca kurute t\textsubring{r}ṣānāśanam eva ca \veg\dontdisplaylinenum
\varr{
        \ \vd t\textsubring{r}ṣā°\lem  \Ed; t\textsubring{r}ṣa° \msCa}

navamāni hy anekāni vāruṇaṁ karma ucyate\thinspace{\dandab} \dontdisplaylinenum

māhendraṁ pārthivaṁ karma anekāni dvijottama \veg\dontdisplaylinenum

kulālakarmabhūkarma karma pāṣāṇam eva ca\thinspace{\dandab} \dontdisplaylinenum

dārudantimaś\textsubring{r}ṅgādi karma pārthivam ucyate \veg\dontdisplaylinenum
\varr{
        \ \va kulālakarma°\lem  \Ed; ku\uncl{la}{\lost}{\lost}rmma° \msCa
        \ \vb karma\lem  \msCa; karmaṁ \Ed}

catuṣkarma samāsena hastataḥ parikīrtitam\thinspace{\dandab} \dontdisplaylinenum


\alalalfejezet{pādakarma (5)}

pādābhyāṁ gamanaṁ karma diśaś ca vidiśas tathā \veg\dontdisplaylinenum
\varr{
        \ \vd diśaś ca vidiśas\lem  \msCa; diśañ ca vidiśan \Ed}

nimnonnatasame deśe śilāsaṁkaṭakoṭare\thinspace{\dandab} \dontdisplaylinenum

toyakardamasaṁghāte bahukaṇṭakasaṁkule \veg\dontdisplaylinenum


\alalalfejezet{pāyukarma (4)}
\varr{
        \ \vd bahukaṇṭaka°\lem  \Ed; \uncl{bahu}{\lost}{\lost}ka° \msCa\oo
                 °kule\lem  \msCa; °yute \Ed}

pāyukarma visargaṁ tu kaṭhinadravapicchilam\thinspace{\dandab} \dontdisplaylinenum

saraktaphenilādīni pāyuśakti pramuñcate \veg\dontdisplaylinenum


\alalalfejezet{upasthakarma (3)}
\varr{
        \ \va pāyu°\lem  \msCa; pāpa° \Ed
        \ \vd pāyuśakti\lem  \Ed;  pāyucchakti \msCa\oo
                 °muñcati\lem  \msCa; °muñcate \Ed}

upasthakarma ānandaṁ karoti jananaṁ prajā\thinspace{\dandab} \dontdisplaylinenum

strīpuṁnapuṁsakaṁ caiva upasthaṁ kurute dvija \veg\dontdisplaylinenum


\alalalfejezet{vākkarma (2)}
\varr{
        \ \va ānandaṁ\lem  \msCa; ānanda \Ed}

vācā tu kurute karma navadhā dvijapuṅgava\thinspace{\dandab} \dontdisplaylinenum

stutinindā praśaṁsā ca ākrośaḥ priya eva saḥ \veg\dontdisplaylinenum
\varr{
        \ \vd ākrośaḥ\lem  \Ed; {\lost}krośaḥ \msCa}

praśno 'nujñā tathākhyānam āśīś ca vidhayo nava\thinspace{\dandab} \dontdisplaylinenum

etā navavidhā vāṇī kīrtito me dvijottama \veg\dontdisplaylinenum


\alalfejezet{manaś conmanaś ca (1)}\varr{
        \ \vb cā vidhayo nava\lem  \eme; ca vidhayo naya \msCa, cāvidhiyo nayaḥ \Ed}

adhunā kathayiṣyāmi manaso nava vai guṇān\thinspace{\dandab} \dontdisplaylinenum

calopapattiḥ sthairaṁ ca visargakalpanākṣamā \danda\dontdisplaylinenum

sad asac cāśutā caiva manaso nava vai guṇāḥ \veg\dontdisplaylinenum
            \paral{\textit{\vcdef {\normalfont  = MBh 12.247.9 } }} 
\varr{
        \ \vd visarga°\lem  \Ed; visarge \msCa\oo
                 °kṣamā\lem  \msCa; °samā \Ed
        \ \ve cāśutā\lem  \Ed; cāśutāñ \msCa}

iṣṭāniṣṭavikalpaś ca vyavasāyaḥ samādhitā\thinspace{\dandab} \dontdisplaylinenum
            \paral{\textit{\vab {\normalfont  = MBh 12.247.10ab }}} 

manaso dvividhaṁ rūpaṁ manaś conmana eva ca \veg\dontdisplaylinenum
\varr{
        \ \va iṣṭā°\lem  \Ed; {\lost}ṣṭā° \msCa
        \ \vb samādhitā\lem   \msCa; samādhinā \Ed
        \ \vd conmana\lem  \Ed; cotmana \msCa}

manas tv indriyabhāvatve unmanastvam atīndriye\thinspace{\dandab} \dontdisplaylinenum

nig\textsubring{r}hītā vis\textsubring{r}ṣtaṁ ca bandhamokṣau tu sādhanam \veg\dontdisplaylinenum
\varr{
        \ \vb °tīndriye\lem  \eme; °nīndriye \msCa, °tīndriya \Ed
        \ \vcd (nig\textsubring{r}hītā{\normalfont ...} sādhanam)\lem  \msCa; \om\ \Ed}

nig\textsubring{r}hītendriyagrāmaḥ svargamokṣau tu sādhanam\thinspace{\dandab} \dontdisplaylinenum

vis\textsubring{r}ṣṭe indriyagrāme duḥkhasaṁsārasādhanam \veg\dontdisplaylinenum
\varr{
        \ \vd duḥkha°\lem  \corr; {\lost}kha° \msCa, duḥkhaṁ \Ed}

sakalaṁ niṣkalaṁ caiva mana eva vidur budhāḥ\thinspace{\dandab} \dontdisplaylinenum

sakalaṁ mananānātve ekatve mananiṣkalam \veg\dontdisplaylinenum


\alalfejezet{unmanaḥ}
vigatarāga uvāca~{\dandab}\dontdisplaylinenum 

manaḥ svavedyaṁ lokānām unmanas tu na vidyate\thinspace{\danda} \dontdisplaylinenum

unmanaḥ kathayāsmākaṁ kīd\textsubring{r}śaṁ lakṣaṇaṁ bhavet \veg\dontdisplaylinenum

anarthayajña uvāca~{\dandab}\dontdisplaylinenum 

unmanastvaṁ gate vipra nibodha daśalakṣaṇam\thinspace{\danda} \dontdisplaylinenum

na śabdaṁ ś\textsubring{r}ṇute śrotraṁ śaṅkhabherīsvanād api \veg\dontdisplaylinenum
\varr{
        \ \vb nibodha\lem  \Ed; {\lost}bodha \msCa
        \ \vc śrotraṁ\lem  \msCa; śrotre \Ed}

tvacaḥ sparśaṁ na jānāti śītoṣṇam api duḥsaham\thinspace{\dandab} \dontdisplaylinenum

rūpaṁ paśyati no cakṣuḥ parvatābhyadhiko 'pi vā \veg\dontdisplaylinenum

jihvā rasaṁ na vindeta madhurāmlavato 'pi vā\thinspace{\dandab} \dontdisplaylinenum

gandhaṁ jighrati na ghrāṇā tīkṣṇaṁ vāpy aśucīny api \veg\dontdisplaylinenum
\varr{
        \ \vb °mlavato\lem  \Ed; °mlavano \msCa
        \ \vc ghrāṇā\lem  \msCa; ghrāṇo \Ed}

unmanas tava me khyātaṁ sarvadvaitavināśanam\thinspace{\dandab} \dontdisplaylinenum

bhavapāragasuvyaktaṁ niṣkalaṁ śivam avyayam \veg\dontdisplaylinenum
\varr{
        \ \va unmanas tava me\lem  \Ed; {\lost}{\lost}{\lost}{\lost}{\lost}{\lost} \msCa}

sa śivaḥ sa paro brahmā sa viṣṇuḥ sa paro 'kṣaraḥ\thinspace{\dandab} \dontdisplaylinenum

sa sūkṣmaḥ sa paro haṁsaḥ so 'kṣaraḥ kṣaravarjitaḥ \veg\dontdisplaylinenum

eṣa unmana jānīhi śivaś ca dvijapuṅgava\thinspace{\dandab} \dontdisplaylinenum

kīrtito 'smi samāsena kim anyat parip\textsubring{r}cchasi \veg\dontdisplaylinenum


\jump
\begin{center}
\ketdanda iti v\textsubring{r}ṣasārasaṁgrahe pañcaviṁśatitattvanirṇayo nāma viṁśatimo 'dhyāyaḥ\ketdanda
\end{center}
\dontdisplaylinenum\vers 
\varr{
        \ \vd parip\textsubring{r}cchasi\lem  \Ed; pa\uncl{ri}{\lost}{\lost}{\lost} \msCa
        \ {\normalfont \Colo: }  viṁśatimo\lem  \msCa; viṁśatitamo \Ed}
\bekveg\szamveg\vfill\phpspagebreak\szam\bek\versno=0\fejno=21
\thispagestyle{empty}


\vers


\alfejezet{\textbf{ekaviṁśatimo 'dhyāyaḥ}}\jump\jump

\alalfejezet{viṣṇuḥ svarūpaṁ darśayati}
vigatarāga uvāca~{\dandab}\dontdisplaylinenum 

aho matimatāṁ śreṣṭha aho dharmabh\textsubring{r}tāṁ vara\thinspace{\danda} \dontdisplaylinenum

aho dama śamaḥ satya aho yajña aho tapaḥ \veg\dontdisplaylinenum
\varr{
        \ \va matimatāṁ\lem  \mssCaCbCc\msNa\msNb\msNc; matimanā \Ed
        \ \vb vara\lem  \msCa\msCc\msNa\msNb\msNc; varaḥ \msCb\Ed
        \ \vc dama śamaḥ\lem  \msCa\msCb\msNa\msNb; damaḥ śamaḥ \msCc\msNc\Ed}

anenām\textsubring{r}tavākyena vismayo me paro gataḥ\thinspace{\dandab} \dontdisplaylinenum

prīto 'smi ca tapādhārajñānādbhutarasena ca \veg\dontdisplaylinenum
\varr{
        \ \vb me paro gataḥ\lem  \mssCaCbCc\msNa\msNc\Ed; {\lost}{\lost}{\lost}{\lost}{\lost} \msNb
        \ \vc prīto 'smi ca\lem  \msCb\msCc\msNa\msNc\Ed; \uncl{pr}{\lost}{\lost} ca \msCa, {\lost}{\lost}{\lost}{\lost} \msNb
        \ \vd tapādhārajñānādbhutarasena ca\lem  \mssCaCbCc\msNa\msNc; {\lost}{\lost}{\lost}{\lost}{\lost}{\lost}{\lost}{\lost}{\lost}{\lost}{\lost}{\lost} \msNb, tapādhārajñānādbhūtarasena ca \Ed}

kiṁ dadāmi varaṁ brūhi dātāsmi tava cepsitam\thinspace{\dandab} \dontdisplaylinenum

etac chrutvā tatas tena pratyuvāca śubhāṁ giram \veg\dontdisplaylinenum
\varr{
        \ \va kiṁ dadāmi varaṁ brūhi\lem  \mssCaCbCc\msNa\msNc\Ed; {\lost}{\lost}{\lost}{\lost}{\lost}{\lost}{\lost}hi \msNb
        \ \vb cepsitam\lem  \msCa\msCc\msNa\msNb\msNc\Ed; cesmitam \msCb
        \ \vd śubhāṁ giram\lem  \mssCaCbCc\msNa\msNb\msNc; śubhāṅgirām \Ed}

[anarthayajña uvāca~{\dandab}\dontdisplaylinenum ]

ko bhavān varadaśreṣṭha devadānavarākṣasāḥ\thinspace{\danda} \dontdisplaylinenum

athavā bhagavān viṣṇur mama jijñāsur āgataḥ \veg\dontdisplaylinenum
\varr{
        \ \va bhavān\lem  \msCapcorr\msCb\msCc\msNa\msNb\msNc\Ed; bhagavān \msCaacorr\oo
                 varada śreṣṭha\lem  \mssCaCbCc\msNa\msNb\msNc; varadaḥ śreṣṭhaḥ \Ed
        \ \vb °rākṣasāḥ\lem  \mssCaCbCc\msNa\msNb\msNc; °rākṣasaḥ \Ed
        \ \vd °gataḥ\lem  \mssCaCbCc\msNa\msNc\Ed; °grataḥ \msNb}

vyaktaṁ tvāṁ puruṣaśreṣṭha jānāmi puruṣottama\thinspace{\dandab} \dontdisplaylinenum

rūpaṁ darśaya govinda yady asti tapasaḥ phalam \veg\dontdisplaylinenum
\varr{
         \ \va vyaktaṁ tvāṁ\lem  \msCa\msCb\msNa; vyaktatvaṁ \msCc\Ed, vyaktatva \msNb, vyaktaṁ tvaṁ \msNc\oo
                 °śreṣṭha\lem  \mssCaCbCc\msNa\msNb\msNc; °śreṣṭhaḥ \Ed
         \ \vb puruṣottama\lem  \msCb\msNa\msNb\msNc; \uncl{pu}{\lost}{\lost}ttama \msCa, puruṣotta{\lost} \msCc, puruṣottamaḥ \Ed
        \ \vc rūpaṁ darśaya govinda\lem  \msCa\msCb\msNa\msNb\msNc\Ed; {\lost}{\lost}{\lost}{\lost}{\lost}{\lost}vinda \msCc 
        \ \vd tapasaḥ phalam\lem  \mssCaCbCc\msNa\msNc\Ed; ta{\lost}{\lost}{\lost}{\lost} \msNb}

tatas tu puṇḍarīkākṣo darśayāmāsa svāṁ tanum\thinspace{\dandab} \dontdisplaylinenum

śaṅkhacakragadāpāṇiḥ pītāmbaradharo hariḥ \veg\dontdisplaylinenum
\varr{
        \ \vab tatas tu puṇḍarīkākṣo darśayāmāsa svāṁ tanum\lem  \mssCaCbCc\msNa\msNc\Ed; {\lost}{\lost}{\lost}{\lost}{\lost}{\lost}{\lost}{\lost}{\lost}{\lost}{\lost}{\lost}{\lost}{\lost}{\lost}{\lost} \msNb
        \ \vc śaṅkhacakragadāpāṇiḥ\lem  \mssCaCbCc\msNa\msNc\Ed; {\lost}{\lost}{\lost}{\lost}{\lost}{\lost}{\lost}{\lost} \msNb}

anarthayajñas taṁ d\textsubring{r}ṣṭvā vismayaṁ paramaṁ gataḥ\thinspace{\dandab} \dontdisplaylinenum

praharṣam atulaṁ labdhvā aśrupūrṇākulekṣaṇaḥ \veg\dontdisplaylinenum
\varr{
        \ \vb vismayaṁ\lem  \mssCaCbCc\msNa\msNb\Ed; vismasaṁ \msNc
        \ \vc labdhvā\lem  \msCa\msCb\msNa\msNc\Ed; labba \msCc, \uncl{ladyā} \msNb}

vepamānasvareṇātra uvāca ca janārdanam\thinspace{\dandab} \dontdisplaylinenum

adya me saphalaṁ janma adya me saphalaṁ tapaḥ \veg\dontdisplaylinenum
                     \paral{\textit{\vcd {\normalfont  cf.\ MBh 5.113.5ab: } adya me saphalaṁ janma tāritaṁ cādya me kulam
                                {\normalfont  and MBh  13.14.179a: } adya jāto hy ahaṁ deva adya me saphalaṁ tapaḥ}}
\varr{
        \ \vab vepamānasvareṇātra uvāca ca janārdanam\lem  \msCb\msNa\msNb\msNc; 
                vepamāna{\lost}{\lost}{\lost}{\lost}{\lost}{\lost}ca ca janārdanam \msCa,
                vepamāna{\lost}{\lost}{\lost}tra u{\lost}{\lost}{\lost}{\lost}{\lost}{\lost}{\lost} \msCc, 
                vepamānasvareṇārta uvāca ca janārdanam \Ed
        \ \vc adya me saphalaṁ janma\lem  \msCa\msCb\msNa\msNb\msNc\Ed; \uncl{adya}{\lost}{\lost}{\lost}{\lost}\uncl{janma} \msCc}

\ujvers\nemsloka 
namo namas te 'stu janādisambhave
\dontdisplaylinenum

\nemslokab 
namo namas te 'stu ca viśvarūpiṇe \danda\dontdisplaylinenum

\nemslokac 
namo namas te 'stu janābhisambhave
\dontdisplaylinenum

\nemslokad 
namo namas te 'stu pitāmahodbhave \veg\dontdisplaylinenum
\varr{
        \ \vo (namo{\normalfont ...} °pitāmahodbhave)\lem  \msCa\msCc\msNa\msNb\msNc\Ed; \om\ \msCb
        \ \vb namas te\lem  \mssCaCbCc\msNapcorr\msNb\msNc\Ed; namas tu \msNaacorr\oo
                 'stu ca viśvarūpiṇe\lem  \mssCaCbCc\msNa\msNc\Ed; {\lost}{\lost}{\lost}{\lost}{\lost}{\lost}{\lost} \msNb
        \ \vc (namo{\normalfont ...} °bhisambhave)\lem  \mssCaCbCc\msNc\Ed; \om\ \msNa, {\lost}{\lost}{\lost}{\lost}{\lost}{\lost}{\lost}{\lost}{\lost}{\lost}{\lost}{\lost}{\lost} \msNb
        \ \vd namo namas te 'stu pitāmahodbhave\lem  \mssCaCbCc\msNa\msNc\Ed; {\lost}{\lost}{\lost}{\lost}{\lost}{\lost}{\lost}{\lost}{\lost}hottave \msNb}

\ujvers\nemsloka 
namo namas te 'stu sahasraśīrṣiṇe
\dontdisplaylinenum

\nemslokab 
namo namas te 'stu sahasracakṣuṣe \danda\dontdisplaylinenum

\nemslokac 
namo namas te 'stu sahasraliṅgine
\dontdisplaylinenum

\nemslokad 
namo namas te 'stu sahasravakṣase \veg\dontdisplaylinenum
\varr{
        \ \vo (namo namas te 'stu sahasraśīrṣiṇe{\normalfont ...} °vakṣase)\lem  \msCc\msNa\msNb\msNc\Ed; \om\ \msCa\msCb
        \ \va °śīrṣiṇe\lem  \msNa\msNb\msNc\Ed; \om\ \msCa\msCb, °śīrṣaṇe \msCc}

\ujvers\nemsloka 
namo namas te 'stu sahasramūrtaye
\dontdisplaylinenum

\nemslokab 
namo namas te 'stu sahasrabāhave \danda\dontdisplaylinenum

\nemslokac 
namo namas te 'stu sahasravaktriṇe
\dontdisplaylinenum

\nemslokad 
namo namas te 'stu sahasramāyine \veg\dontdisplaylinenum
\varr{
        \ \vo (namo namas te 'stu sahasramūrtaye{\normalfont ...} °māyine)\lem  \msCb\msCc\msNa\msNb\msNc\Ed; \om\ \msCa
        \ \vc °vaktriṇe\lem  \Ed; \om\ \msCa, °cakriṇe \msCb\msCc\msNc, °vakriṇe \msNa\msNb}

\ujvers\nemsloka 
namo namas te 'stu varāharūpiṇe
\dontdisplaylinenum

\nemslokab 
namo namas te 'stu mahīsamuddh\textsubring{r}te \danda\dontdisplaylinenum

\nemslokac 
namo namas te 'stu ca bhūtas\textsubring{r}ṣṭine
\dontdisplaylinenum

\nemslokad 
namo namas te caturāśramāśraye \veg\dontdisplaylinenum
\varr{
        \ \va (namo{\normalfont ...} varāharūpiṇe)\lem  \msCb\msCc\msNa\msNb\msNc\Ed; \om\ \msCa
        \ \vc °s\textsubring{r}ṣṭine\lem  \msCb\msCc\msNa\msNb\msNc\Ed; °s\textsubring{r}{\lost}{\lost} \msCa
        \ \vd namas te\lem  \msCa\msCb\msNapcorr\msNb\msNc\Ed; namas te stu \msCc\msNaacorr\oo
                 °śraye\lem  \msCa\msCc\msNa\msNb\msNc\Ed; °śrame \msCb}

\ujvers\nemsloka 
namo namas te narasiṁharūpiṇe
\dontdisplaylinenum

\nemslokab 
namo namas te ditijoradāriṇe \danda\dontdisplaylinenum

\nemslokac 
namo namas te 'suracakrasūdane
\dontdisplaylinenum

\nemslokad 
namo namas te 'suradarpanāśane \veg\dontdisplaylinenum
\varr{
        \ \vb namo namas te ditijoradāriṇe\lem  \mssCaCbCc\Ed; \om\ \msNa, namo namas te ditijoradāruṇe \msNb,
                       namo namas te stu ditijoradāriṇe \msNc, namo namas te 'ditijoradāraṇe \Ed
        \ \vc °cakra°\lem  \conj; °śakra° \mssCaCbCc\msNa\msNb\msNc\Ed\oo
                 °sūdane\lem  \mssCaCbCc\msNa\msNb\Ed; °sūdene \msNc}

\ujvers\nemsloka 
namo namas te ditiputradāmane
\dontdisplaylinenum

\nemslokab 
namo namas te baliyajñasūdane \danda\dontdisplaylinenum

\nemslokac 
namo namas te 'stu ṣaḍardhavikrame
\dontdisplaylinenum

\nemslokad 
namo namas te tridaśārtināśane \veg\dontdisplaylinenum
\varr{
        \ \va °dāmane\lem  \msCa\msCb\msNa; °vāmane \msCc\Ed, °nāsane \msNb, °\uncl{vā}mane \msNc
        \ \vc ṣaḍardhavikrame\lem  \msCb\msCc\msNa\msNb\msNc\Ed; \uncl{ṣa}{\lost}{\lost}{\lost}krame \msCa}

\ujvers\nemsloka 
namo namas te 'stu ananta acyute
\dontdisplaylinenum

\nemslokab 
namo namas te jagadartināśane \danda\dontdisplaylinenum

\nemslokac 
namo namas te madhukaiṭanāśane
\dontdisplaylinenum

\nemslokad 
namo namas te 'stu trilokabāndhave \veg\dontdisplaylinenum
\varr{
        \ \vb jagada°\lem  \mssCaCbCc\msNa\msNb\msNc; jagadā° \Ed
        \ \vc °kaiṭa°\lem  \msCa\msCb\msNa\msNc; °kīṭa° \msCc\msNb\Ed}

\ujvers\nemsloka 
namo namas te tridaśābhinandane
\dontdisplaylinenum

\nemslokab 
namo namas te 'stu ca divyacakṣuṣe \danda\dontdisplaylinenum

\nemslokac 
namo namas te 'stu bhavāntapārage
\dontdisplaylinenum

\nemslokad 
namo namas te 'stu trilokapūjite \veg\dontdisplaylinenum

\ujvers\nemsloka 
namo namas te 'stu gadāgrapāṇaye
\dontdisplaylinenum

\nemslokab 
namo namas te varacakrapāṇaye \danda\dontdisplaylinenum

\nemslokac 
namo namas te 'stu ca śaṅkhapāṇaye
\dontdisplaylinenum

\nemslokad 
namo namas te 'stu ca kambupāṇaye \veg\dontdisplaylinenum
\varr{
        \ \v varacakrapāṇaye\lem  \mssCaCbCc\msNa\msNb\Ed; varakrapāṇase \msNc}

\ujvers\nemsloka 
namo namas te 'stu jalaughaśāyine
\dontdisplaylinenum

\nemslokab 
namo namas te haramardarūpiṇe \danda\dontdisplaylinenum

\nemslokac 
namo namas te khagarājaketave
\dontdisplaylinenum

\nemslokad 
namo namas te śaśisūryalocane \veg\dontdisplaylinenum
\varr{
        \ \va jalaugha°\lem  \msCa\msCb\Ed; jalogha° \msCc\msNa\msNb\msNc
        \ \vb namas te haramardarūpiṇe\lem  \msCb\msNa\msNb\msNc\Ed; 
                        nama{\lost}{\lost}{\lost}{\lost}rddarūpiṇe \msCa,
                        {\lost}{\lost}{\lost}{\lost}{\lost}{\lost}{\lost}marddarūpiṇe \msCc
        \ \vc °ketave\lem  \mssCaCbCc\msNa\msNb\msNc; °ketane \Ed}

\ujvers\nemsloka 
namo namas te uragārivāhane
\dontdisplaylinenum

\nemslokab 
namo namas te 'dbhutarūpadarśine \danda\dontdisplaylinenum

\nemslokac 
namo namas te 'yutasūryatejase
\dontdisplaylinenum

\nemslokad 
namo namas te 'm\textsubring{r}tamanthanadhruve \veg\dontdisplaylinenum
\varr{
        \ \vb °darśine\lem  \mssCaCbCc\msNa\msNc\Ed; °darśane \msNb
        \ \vc 'yuta°\lem  \mssCaCbCc\msNa\msNb\msNc; 'stu ca \Ed\oo
                 °tejase\lem  \msCa\msCb\msNa\msNc\Ed; °te \msCc, °locane \msNb}

\ujvers\nemsloka 
namo namas te 'maralokasaṁstute
\dontdisplaylinenum

\nemslokab 
namo namas te jagamaṇḍapāśraye \danda\dontdisplaylinenum

\nemslokac 
namo namas te jagadekavatsale
\dontdisplaylinenum

\nemslokad 
namo namas te śivasarvade namaḥ \veg\dontdisplaylinenum
            \paral{\textit{\vd {\normalfont Cf.\ B\textsubring{r}hatkālottara (NGMPP B 29/59) f.\ 87a: } 
                jñāna 2 śabda 2 sūkṣma 2 śivasarvada oṁ namaḥ śivāya{\thinspace\danda}}}
\varr{
        \ \va 'maralokasaṁstute\lem  \mssCaCbCc\msNb\msNc; maralokavandite \msNa, malalokasaṁstute \Ed
        \ \vb namo namas te jagamaṇḍapāśraye\lem  \msCa\msCb\msNc\Ed;
                                 {\lost}{\lost}{\lost}{\lost}{\lost}{\lost}{\lost}{\lost}{\lost}{\lost}\uncl{śraye} \msCc, \om\ \msNa, namo namas te jagamaṇḍalāśraye \msNb
        \ \vc jagadeka°\lem   \msCa\msCb\msNa\msNb\msNc\Ed; jaga\uncl{de}ka° \msCc\oo      
                °vatsale\lem  \mssCaCbCc\msNa\msNb\msNc; °vatsare \Ed
        \ \vd °sarvade\lem  \mssCaCbCc\msNa\msNc\Ed; °sarvado \msNb}

\ujvers\nemsloka 
kṣamasva govinda mamāparādham
\dontdisplaylinenum

\nemslokab 
atīva p\textsubring{r}ṣṭena durātmanena \danda\dontdisplaylinenum

\nemslokac 
mayeda sarvaṁ kathitaṁ smayena
\dontdisplaylinenum

\nemslokad 
dayāṁ kuru tvaṁ tridaśeśvareṇa \veg\dontdisplaylinenum

\vers
\varr{
        \ \va mamā°\lem  \msCa\msCb\msNa\msNb\msNc\Ed; mama \msCc
        \ \vb °tmanena\lem  \msCa\msCc\msNa\msNb\msNc\Ed; °tmane \msCb
        \ \vc mayeda\lem  \mssCaCbCc\msNa\msNb\msNc; mayedaṁ \Ed\ \unmetr
        \ \vd °śeśvareṇa\lem  \msCa\msCc\msNa\msNb\msNc\Ed; °śaiśvareṇa \msCb}

vaiśampāyana uvāca~{\dandab}\dontdisplaylinenum 

stotreṇānena saṁtuṣṭaḥ keśavaḥ paravīrahā\thinspace{\danda} \dontdisplaylinenum

pratyuvāca mahāseno girayā nirupasp\textsubring{r}hā \veg\dontdisplaylinenum
\varr{
        \ \va stotre°\lem  \mssCaCbCc\msNa\Ed; strotre° \msNb, \uncl{sto}tre° \msNc
        \ \vb keśavaḥ paravīrahā\lem  \msCa\msCb\msNa\msNb\msNc\Ed; 
                        keśava\uncl{ḥ paravīrahā} \msCc\ \toplost
        \ \vc pratyuvāca\lem  \msCa\msCb\msNa\msNb\msNc\Ed; \uncl{pratyuvāca} \msCc\ \toplost\oo
                 mahāseno\lem  \msCb\msCc\msNa\msNb\msNc\Ed; ma{\lost}{\lost}{\lost} \msCa
        \ \vd girayā\lem  \msCa\msCb\msNa\msNb\msNc\Ed; giriyā \msCc\oo      
                 nirupa°\lem  \mssCaCbCc\msNa\msNb; nirūpa° \msNc\Ed}

stotreṇānena me tāta tuṣṭo 'smi bh\textsubring{r}śam ejitaḥ\thinspace{\dandab} \dontdisplaylinenum

durlabhāny api trailokye dadāmi varam īpsitam \veg\dontdisplaylinenum
\varr{
        \ \va stotre°\lem  \mssCaCbCc\msNa\msNc\Ed; strotre° \msNb\oo
                 me tāta\lem  \msCa\msCc\msNb\msNc\Ed; mattāta \msCb, saṁtāta \msNa
        \ \vc trailokye\lem  \msCa\msCb\msNa\msNb\msNc\Ed; trailokya \msCc}

\ujvers\nemsloka 
anena māṁ stauti nirāśritena
\dontdisplaylinenum

\nemslokab 
tvayoktavedārthamanohareṇa \danda\dontdisplaylinenum

\nemslokac 
yāvanti tatrākṣarasaṁkhyam asti
\dontdisplaylinenum

\nemslokad 
tāvanti kalpān divi te vasanti \veg\dontdisplaylinenum
\varr{
        \ \va stauti\lem  \msCa\msCb\msNa\msNc\Ed; stoti \msCc\msNb
        \ \vb °vedārtha°\lem  \mssCaCbCc\msNa\msNc\Ed; °vedārthi° \msNb
        \ \vd kalpān\lem  \msCa\msNa; kalpaṁ \msCb\msNb\msNc\Ed, kalpa \msCc}

\ujvers\nemsloka 
tvaṁ cāpi me brūhi varaṁ yatheṣṭaṁ
\dontdisplaylinenum

\nemslokab 
trailokyarājyād api nirviśaṅkam \danda\dontdisplaylinenum

\nemslokac 
dadāmi kiṁ saptamahīśvaratvam
\dontdisplaylinenum

\nemslokad 
athārtharāśiṁ bahukanyakāṁ vā \veg\dontdisplaylinenum

\vers
\varr{
        \ \va tvaṁ cāpi me brūhi\lem  \msCb\msCc\msNa\msNb\msNc\Ed;
                                        tvañc{\lost}{\lost}{\lost}{\lost}hi \msCa
        \ \vb °rājyā°\lem  \mssCaCbCc\msNb\msNc\Ed; °rā° \msNaacorr, °rājā° \msNapcorr\oo
                 °śaṅkam\lem  \mssCaCbCc\msNa\msNb\msNc; °śaṅka \Ed
        \ \vc kiṁ\lem  \msCa\msCc\msNa\msNb\msNc\Ed; ki \msCb\oo
                 °tvam\lem  \msCa\msCb\msNa\msNb\msNc\Ed; °tvaṁm \msCc
        \ \vd athārtharāśiṁ\lem  \msCa\msCb\msNa\msNc; athārtharāśi \msCc, arthārtharāsi \msNb,
                                        athārthaṁ rāśīṁ \Ed\oo
                 °kanyakāṁ vā\lem  \msNa\msNc\Ed; °kanyakā vā \msCa\msCc\msNb, °kanyakā{\lost}{ā}\ \msCb}

vaiśampāyana uvāca~{\dandab}\dontdisplaylinenum 
\varr{
        \ \vo vaiśampāyana uvāca\lem  \eme; anarthayajña uvāca \msCa\msCb\msNa\msNc,
                                      vigatarāga uvāca \msCc\msNb, \om\ \Ed}

\nemsloka 
śrutvaiva divyaṁ varam acyutasya
\dontdisplaylinenum

\nemslokab 
praṇamya pādadvayapaṅkaje tu \danda\dontdisplaylinenum

\nemslokac 
vijñāya viṣṇuṁ varadaṁ vareṇyaṁ
\dontdisplaylinenum

\nemslokad 
? prah\textsubring{r} cetaḥ pukāncito 'to 'bravīt \veg\dontdisplaylinenum
\varr{
        \ \va śrutvaiva\lem  \mssCaCbCc\msNa\msNc\Ed; śrutaiva \msNb\oo
                 varam acyutasya\lem  \mssCaCbCc\msNa\msNb\Ed; varam ucyutasya \msNc
        \ \vb °je tu\lem  \msCa\msCc\msNb\msNc\Ed; °hetu \msCb, °je nu \msNa
        \ \vcd (vijñāya{\normalfont ...} 'bravīt)\lem  \Ed; \om\ \mssCaCbCc\msNa\msNb\msNc}

\ujvers\nemsloka 
na kāmaye 'nyapravaraṁ tu deva
\dontdisplaylinenum

\nemslokab 
asaṁśayaṁ bandhanasāram ekam \danda\dontdisplaylinenum

\nemslokac 
vimuktabandho bhavataḥ prasādād
\dontdisplaylinenum

\nemslokad 
bhavāmi govinda rataś ca dharme \veg\dontdisplaylinenum

\vers
\varr{
        \ \va na kāmaye\lem  \mssCaCbCc\msNa\msNb\msNc; anarthayajña uvāca na kāmaye \Ed\oo
                 'nyapravaraṁ tu\lem  \msCa\msCc\msNa\msNb\msNc; nyaprabhavan tu \msCb, 'nyaṁ pravaraṁ tu \Ed\oo
                 deva\lem  \mssCaCbCc\msNa\msNc\Ed; dedeva \msNb
        \ \vb asaṁśayaṁ\lem  \mssCaCbCc\msNa\msNc; asaṁśaya \msNb\Ed\oo
                °sāram ekam\lem  \msCb\msCc\msNa\msNb\msNc\Ed; °sārame{\lost} \msCa
        \ \vc vimuktabandho\lem  \msCb\msCc\msNa\msNb\msNc\Ed; {\lost}{\lost}{\lost}{\lost}{\lost} \msCa\oo
                 prasādād\lem  \mssCaCbCc\msNa\msNb\msNc; pramādād \Ed
        \ \vd rataś ca\lem  \mssCaCbCc\msNa\msNb\msNc; ratañ ca \Ed}

bhagavān uvāca~{\dandab}\dontdisplaylinenum 

\nemsloka 
yathaiva cittaṁ tava suprasannaṁ
\dontdisplaylinenum

\nemslokab 
maharṣidevair api naiva d\textsubring{r}ṣṭam \danda\dontdisplaylinenum

\nemslokac 
akalmaṣaṁ duḥkhavivarjitatvam
\dontdisplaylinenum

\nemslokad 
bhavārṇavas tīrṇam asaṁśayena \veg\dontdisplaylinenum
\varr{
        \ \vc akalmaṣaṁ\lem  \msNb\Ed; akalmaṣas tvaṁ \msCa\msNa\ \unmetr, akalmaṣatvaṁ \msCb\msNc\ \unmetr,
                                                                              akalmatvaṁ \msCc\ \unmetr\oo
                 duḥkha°\lem  \msCapcorr\msCb\msCc\msNa\msNb\msNc\Ed; duḥ° \msCaacorr}

\ujvers\nemsloka 
gacchāma bho sāmprata śvetadvīpam
\dontdisplaylinenum

\nemslokab 
agamya devair api durnirīkṣyam \danda\dontdisplaylinenum

\nemslokac 
madbhaktipūtamanasā prayāti
\dontdisplaylinenum

\nemslokad 
ghorārṇave naiva punaś caranti \veg\dontdisplaylinenum

\vers
\varr{
        \ \va gacchāma bho\lem  \mssCaCbCc\msNa\msNb\msNc; gacchāmato \Ed\oo
                 sāmprata\lem  \mssCaCbCc\msNa\msNb\msNc; samprati \Ed\oo
                 °dvīpam\lem  \mssCaCbCc\msNa\msNc\Ed; °dvīpa \msNb
        \ \vb durnirīkṣyam\lem  \msCb\msCc\msNa\msNb\Ed;
                                 durnirī\uncl{kṣ}{\lost} \msCa, durnirīkṣaṁ \msNc
        \ \vc madbhakti°\lem  \msCb\msCc\msNa\msNb\msNc\Ed; {\lost}{\lost}kti \msCa\oo
                °pūta\lem  \mssCaCbCc\msNa; °pūtaṁ \msNb\msNc\Ed}

vaiśampāyana uvāca~{\dandab}\dontdisplaylinenum 
\varr{
        \ \vo vaiśampāyana uvāca\lem  \msCa\Ed; \om\ \msCb\msCc\msNaacorr\msNb\msNc,
                                 vaiśaṁ u \msNapcorr}

evam uktvā haris tatra kare g\textsubring{r}hya tapodhanam\thinspace{\danda} \dontdisplaylinenum

tataḥ so 'ntarhitas tatra tenaiva saha keśavaḥ \veg\dontdisplaylinenum
\varr{
        \ \vb g\textsubring{r}hya tapodhanam\lem  \mssCaCbCc\msNb\msNc\Ed; {\lost}{\lost}{\lost}{\lost}dha\uncl{na} \msNa
        \ \vc tatas so 'ntarhitas ta°\lem  \msCa\msCb;
                                        ta\uncl{taḥ so nta}rhitas ta° \msNa,
                                        tatas te ntarhitās ta° \msCc,
                                        tatas te ttarhitās ta° \msNb,
                                        tateṁs te tarhitas ta° \msNc,
                                        tatas te karhitās ta° \Ed
        \ \vd keśavaḥ\lem  \mssCaCbCc\msNa\msNb\msNc; keśava \Ed}

\ujvers\nemsloka 
evaṁ hi dharmas tv adhikaprabhāvād
\dontdisplaylinenum

\nemslokab 
gataḥ sa lokaṁ puruṣottamasya \danda\dontdisplaylinenum

\nemslokac 
aśeṣabhūtaprabhavāvyayasya
\dontdisplaylinenum

\nemslokad 
sanātanaṁ śāśvatam akṣarasya \veg\dontdisplaylinenum
\varr{
        \ \va adhika°\lem  \msCa\msCb\msNa\msNc; adhikaṁ \msCc\msNb\Ed
        \ \vb gataḥ\lem  \mssCaCbCc\msNa\msNb\Ed; gatā \msNc\oo
                 lokaṁ\lem  \mssCaCbCc\msNa\msNc\Ed; loka \msNb
        \ \vd sanātanaṁ\lem  \Ed; sanātana \msCa\ \toplost\          
                                                \msCb\msCc\msNa\msNb\msNc\oo
                 °kṣarasya\lem  \msCb\msCc\msNa\msNb\msNc\Ed; {\lost}{\lost}{\lost} \msCa}

\ujvers\nemsloka 
tvam eva bhaktiṁ kuru keśavasya
\dontdisplaylinenum

\nemslokab 
janārdanasyāmitavikramasya \danda\dontdisplaylinenum

\nemslokac 
yathā hi tasyaiva dvijarṣabhasya
\dontdisplaylinenum

\nemslokad 
gatiṁ labhasva puruṣottamasya \veg\dontdisplaylinenum
\varr{
        \ \va tvam eva\lem  \msCb\msCc\msNa\msNb\msNc\Ed; {\lost}m eva \msCa
        \ \vc hi tasyaiva\lem  \msCa\msCb\msNc\Ed; jitasyaiva \msCc\msNb, \uncl{hi tasyava} \toplost\ \msNa
        \ \vd labhasva\lem  \mssCaCbCc\msNa\msNb\Ed; labhatvaṁ \msNc}

\ujvers\nemsloka 
kim anya bhūyaḥ kathayāmi rājan
\dontdisplaylinenum

\nemslokab 
yad asti kautūhalam anyaśeṣam \danda\dontdisplaylinenum

\nemslokac 
p\textsubring{r}cchasva māṁ tāta yathepsitaṁ te
\dontdisplaylinenum

\nemslokad 
bhaviṣyabhūtaṁ bhavato yatheṣṭam \veg\dontdisplaylinenum

\vers
\varr{
        \ \va kim anya bhū°\lem  \msCc\Ed\msNb\msNc; kim anyad bhū° \msCa\msCb\msNa\ \unmetr\oo
                 rājan\lem  \mssCaCbCc\msNa\msNb\Ed; rājad \msNc
        \ \vd bhaviṣya°\lem  \mssCaCbCc\msNa\Ed; bhavasva° \msNb, bhavasya \msNc\oo
                 bhavato\lem  \mssCaCbCc\msNa\msNb\Ed; bhavate \msNc\oo
                 yatheṣṭam\lem  \msCa\msCb\msNa\msNb\msNc\Ed; yatheṣṭa \msCc}

janamejaya uvāca~{\dandab}\dontdisplaylinenum 
\varr{
        \ \vo janamejaya uvāca\lem  \mssCaCbCc\msNapcorr\msNb\Ed; \om\ \msNaacorr, jayamejaya uvāca \msNc}

\nemsloka 
kiyanti kalpāni gatāni pūrvam
\dontdisplaylinenum

\nemslokab 
bhaviṣyakalpāni kiyanti vipra \danda\dontdisplaylinenum

\nemslokac 
ekaikakalpaṁ kiyad indram uktam
\dontdisplaylinenum

\nemslokad 
pravartamānād api kīrtayasva \veg\dontdisplaylinenum

\vers
\varr{
        \ \va kiyanti\lem  \mssCaCbCc\msNa\msNb\msNc; kiyanta \Ed
        \ \vb kiyanti\lem  \mssCaCbCc\msNa\msNb\Ed; kiyanta \msNc
        \ \vc °kalpaṁ\lem  \msCa\msCc\msNa\msNb\msNc\Ed; °kalpa \msCb\msNa}

vaiśampāyana uvāca~{\dandab}\dontdisplaylinenum 
\varr{
        \ \vo vaiśampāyana uvāca\lem  \msCa\msCc\msNa\Ed; veśanampāyana uvāca \msCb, {\il}{\il}{\il}{\il}{\il}{\il}{\il}{\il} \msNb, vaiśaṁpārāya uvāca \msNc}

\nemsloka 
parārdhakalpaṁ gata pūrva rājyam
\dontdisplaylinenum

\nemslokab 
caturdaśaivendra narendra kalpam \danda\dontdisplaylinenum

\nemslokac 
tathaiva manvantara kalpam ekam
\dontdisplaylinenum

\nemslokad 
bhaviṣyakalpaṁ ca parārdham eva \veg\dontdisplaylinenum
\varr{
        \ \vc manvantarakalpam ekam\lem  \mssCaCbCc\msNapcorr\msNb; manvarakalpam ekam \msNaacorr,
                                           maṇvantarakalpam ekaṁ \msNc, manvantaram ekakalpam \Ed
        \ \vd kalpaṁ ca parārdham eva\lem  \msCb\msCc\msNa\msNc\Ed; 
                         ka{\lost}{\lost}{\lost}{\lost}{\lost}{\lost}{\lost} \msCa, bhūtañ ca parārddhatañ ca \eyeskip{to 21.33d} \msNb}

\ujvers\nemsloka 
varāhakalpaḥ prathamo babhūva
\dontdisplaylinenum

\nemslokab 
gatāś ca manvantara ṣaḍ narendra \danda\dontdisplaylinenum

\nemslokac 
caturyugaṁ saptati ekayuktaṁ
\dontdisplaylinenum

\nemslokad 
manvantarā saṁkhyam udāharanti \veg\dontdisplaylinenum
\varr{
        \ \va babhūva\lem  \msCa\msCc\msNa\msNb\msNc\Ed; babhū \msCb
        \ \vb manvantara ṣaḍ narendra\lem  \msCa\msCb\msNapcorr\msNb\msNc;
                      manvaraṣaṭnarendra \msCc,
                      manvantaṣaṭnarendra \msNaacorr,
                      manvantaraṣaṭnarendraḥ \Ed
        \ \vc °yugaṁ\lem  \mssCaCbCc\msNa\msNb\msNc; °yuga° \Ed\oo
                 ekayuktaṁ\lem  \msCa\msCc\msNa\msNb\msNc\Ed; ekamuktaṁ \msCb}

\ujvers\nemsloka 
manvantarāṇāṁ ca caturdaśaiva
\dontdisplaylinenum

\nemslokab 
kalpasya saṁkhyā munayo vadanti \danda\dontdisplaylinenum

\nemslokac 
kalpāyutaś cāha pitāmahasya
\dontdisplaylinenum

\nemslokad 
tathā ca rātriṁ pravadanti tajjñāḥ \veg\dontdisplaylinenum
\varr{
        \ \va manvantarāṇāṁ ca caturdaśaiva\lem  \msCa\msCb\msNa\msNc; manvantarāṇān tu caturdaśaiva \msCc\msNb, \om\ \Ed
        \ \vc kalpāyutaś cāha\lem  \mssCaCbCc\msNa\Ed; {\il}{\il}{\il}{\il}{\il}ha \msNb, kalpāyutaś cāha \msNc
        \ \vd rātriṁ\lem  \mssCaCbCc\msNa\msNc\Ed; rātri \msNb}

\ujvers\nemsloka 
ṣaḍlakṣakalpena tu māsam āhus
\dontdisplaylinenum

\nemslokad 
taddvādaśā varṣam udāharanti \veg\dontdisplaylinenum
\varr{
        \ \va lakṣakalpena tu māsam āhus\lem  \msCb\msCc\msNa\msNb\msNc\Ed;
                       lakṣaka{\lost}{\lost}{\lost}{\lost}{\lost}m āhus \msCa
        \ \vb taddvādaśā va°\lem  \corr; tadvādaśā va° \msCa\msCb\msNb, tatadvādaśā va° \msCc,
                         tadvādaśād va° \msNa, taddvādaśād va° \msNc, tvaddvādaśava° \Ed}

\ujvers\nemsloka 
tenābdena parārdhakalpaguṇitaṁ brahmāyur ity ucyate
\dontdisplaylinenum

\nemslokab 
trailokyādhipatiḥ pradhānapuruṣo brahmāpy anityaḥ sm\textsubring{r}taḥ \danda\dontdisplaylinenum

\nemslokac 
śeṣaṁ bhūtacaturvidhasya niyataṁ jīvasya kiṁ śocyate
\dontdisplaylinenum

\nemslokad 
tasmān nāsti jagatsusāravimalaṁ muktvā śivaṁ śāśvatam \veg\dontdisplaylinenum
            \paral{\textit{\vd {\normalfont See the expression } jagatsusāra {\normalfont also in 1.1b}}}

\vers


\jump
\begin{center}
\ketdanda iti v\textsubring{r}ṣasārasaṁgrahe kalpanirṇayo nāmaikaviṁśatimo 'dhyāyaḥ\ketdanda
\end{center}
\dontdisplaylinenum\vers 
\varr{
        \ \va °bdena\lem  \mssCaCbCc\msNa\msNb\msNc; °rdhena \Ed
        \ \vb °puruṣo\lem  \msCa\msCb\msNa\msNb\msNc\Ed; °puruṣā \msCc\oo
                 °py anityaḥ\lem  \mssCaCbCc\msNa\msNb\Ed; °pi nityaḥ \msNc
        \ \vc niyataṁ\lem  \mssCaCbCc\msNa\msNb\Ed; niyitaṁ \msNc\oo
                 kiṁ\lem  \mssCaCbCc\msNa\msNb\Ed; ki \msNc
        \ \vd °vimalaṁ muktvā\lem  \msCc; °viralaṁ muktvā \msCa\msCb\msNa\msNb\msNc, °viralamuktā \Ed
        \ {\normalfont \Colo: } °viṁśatimo\lem  \mssCaCbCc\msNa\msNb\msNc; °viṁśatitamo \Ed\oo
                        'dhyāyaḥ\lem  \mssCaCbCc\msNa\msNb\Ed; dhyāya \msNc}
\bekveg\szamveg\vfill\phpspagebreak\szam\bek\versno=0\fejno=22
\thispagestyle{empty}



\alfejezet{\textbf{dvāviṁśo 'dhyāyaḥ}}\jump\jump
\vers

janamejaya uvāca~{\dandab}\dontdisplaylinenum 

śruto 'thābjamukhād dharmasārasaṁgraham uttamam\thinspace{\danda} \dontdisplaylinenum
            \paral{\textit{{\normalfont Testimonia for this chapter: \msCa\ ff.\thinspace 232r--234v, 
                                             \msCb\ ff.\thinspace 233v--235r, 
                                             \msCc\ ff.\thinspace 314r--317r,
                                             \msNa\ ff.\thinspace 39r--41v,
                                             \msNb\ ff.\thinspace 241v--243v, 
                                             \msNc\ ff.\thinspace 247v--250r;
                                                \mssCaCbCc\ = \msCa + \msCb + \msCc }}}

madhuraślakṣṇavāṇībhiḥ samyagvedārthasaṁyutam \veg\dontdisplaylinenum
\varr{
        \ \va śruto 'thābjamukhād dharma°\lem  \eme; 
        śruto vābjamukhād dharmaḥ \msCa, śruto vābjamukhod dharmaḥ \msCb, śruto vābjamukhā dharmaḥ \msCc, 
        śruto cābjamukhād dharmaḥ \msNa\msL, śruto cābdamukhā dharmaḥ \msNb, śrutvā vābjamukhād dharmaḥ \msNc,
                        śruto vā tvanmukhād dharmaḥ \Ed
        \ \vc °ślakṣṇavāṇī°\lem  \msCb\msCc\msNa\msNb\msNc;
                       ślakṣṇaṇī° \msCa, °ślakṣyavānī° \msL, °ślakṣṇāvāṇī° \Ed}

nyāyayuktaṁ mahāsāraṁ guhyajñānam anuttaram\thinspace{\dandab} \dontdisplaylinenum

t\textsubring{r}pto 'smīhām\textsubring{r}taṁ pītvā janmam\textsubring{r}tyurujāpaham \veg\dontdisplaylinenum
\varr{
        \ \va nyāyayuktaṁ mahāsāraṁ\lem  \msCa\msCc\msNb\msNc\Ed; nyāyam uktaṁ mahat sāraṁ \msCb,
                                        nyāyayuktaṁ mahat sāraṁ \msNa\msL
        \ \vb guhya°\lem  \mssCaCbCc\msNa\msNb\msNc\msL; guhyaṁ \Ed\oo
                 °nuttaram\lem  \msCa\msNa\msNb\msL; °nuttamam \msCb\msCc\msNc, °nantaram \Ed
        \ \vcd pītvā janma°\lem  \msCb\msCc\msNa\msNb\msNc\msL\Ed; \uncl{pī}{\lost}{\lost}nma \msCa
        \ \vd  °rujā°\lem  \msCa\msCc\msNa\msNb\msNc\msL\Ed; °mujā° \msCb}

praśnam ekānya p\textsubring{r}cchāmi nāmahetuṁ tapodhana\thinspace{\dandab} \dontdisplaylinenum

varṇagotrāśramaṁ tasmāc chrotum icchāmi te punaḥ \veg\dontdisplaylinenum
\varr{
        \ \va praśna°\lem  \mssCaCbCc\msNa\msNc\msL\Ed; prasta° \msNb\oo
                 °kānya\lem  \mssCaCbCc\msNb\msNc; °kānyat \msNa\ \unmetr, 
                                   °kāṁnyat \msL\ \unmetr, °konya \Ed
        \ \vb nāma°\lem  \mssCaCbCc\msNa\msNb\msL\Ed; nāya° \msNc\oo
                 °hetuṁ\lem  \msCa\msCb\msNa\msL; °hetu \msCc\msNb\msNc\Ed\oo
                 °dhana\lem  \mssCaCbCc\msNb\msNc\Ed; °dhanam \msNa\msL
        \ \vc varṇa°\lem  \mssCaCbCc\msNa\msNb\msNc\msBod\msL; varṇaṁ \Ed}

vaiśampāyana uvāca~{\dandab}\dontdisplaylinenum 
\varr{
        \ \vo uvāca\lem  \mssCaCbCc\msNa\msNb\msL\Ed; {\lost}{\lost}{\lost} \msNc}

ś\textsubring{r}ṇu rājann avahito yogendrasya mahātmanaḥ\thinspace{\danda} \dontdisplaylinenum

āśramaṁ varṇajātīnāṁ vakṣyāmy eva narādhipa \veg\dontdisplaylinenum
\varr{
        \ \va rājann a°\lem  \msCb\msCc\msNa\msNc\msL\Ed; rājan a° \msCa\msNb
        \ \vab °vahito yogendrasya\lem  \mssCaCbCc\msNapcorr\msNb\msNc\Ed; °vahito yogendra \msNaacorr, 
                                                °hito yogandrasya \msL
        \ \vd vakṣyāmy eva\lem  \msCa\msCc\msNa\msNb\Ed; vakṣyām eva \msCb\msNc\msL\oo
                 °pa\lem  \mssCaCbCc\msNa\msNb\msNc\msL; °paḥ \Ed}

himavaddakṣiṇe pārśve m\textsubring{r}gendraśikhare n\textsubring{r}pa\thinspace{\dandab} \dontdisplaylinenum

mahendrapathagā nāma nadītīre narādhipa \veg\dontdisplaylinenum
\varr{
        \ \vb m\textsubring{r}gendra°\lem  \msCb\msCc\msNa\msNb\msNc\msL\Ed; \uncl{m\textsubring{r}}{\lost}ndra° \msCa\oo
                 n\textsubring{r}pa\lem  \mssCaCbCc\msNa\msNb; n\textsubring{r}paḥ \msNc\msL\Ed
        \ \vc mahendra°\lem  \mssCaCbCc\msNa\msNc\Ed; m\textsubring{r}gendra° \msNb, mahindra° \msL
        \ \vd °pa\lem  \mssCaCbCc\msNa\msNb\msNc\msL; °paḥ \Ed}

tatrāśramapadaṁ tasya puline sumanorame\thinspace{\dandab} \dontdisplaylinenum

vasati sma mahābhāgas tattvapāraganisp\textsubring{r}haḥ \veg\dontdisplaylinenum
\varr{
        \ \vb puline su°\lem  \msCa\msCb\msNa; pulineṣu \msCc\msNb\msNc\Ed, puline pu° \msL
        \ \vc vasati\lem  \mssCaCbCc\msNa\msNb\msNc\Ed; vasanti \msL
        \ \vd °pāraga°\lem  \msCa\msCc\msNa\msNb\msNc\msL\Ed; °pāra° \msCb\oo
                 °sp\textsubring{r}haḥ\lem  \mssCaCbCc\msNa\msNb\msNc\msL; °sp\textsubring{r}hāḥ \Ed}

śīlaśaucasamācāro jitadvandvo jitaśramaḥ\thinspace{\dandab} \dontdisplaylinenum

jitamānabhayakrodho jitasarvaparigrahaḥ \veg\dontdisplaylinenum
\varr{
        \ \vd jita°\lem  \msCa\msCc\msNa\msNb\msNc\msL\Ed; jija° \msCb}

somavaṁśaprasūtās te kṣatriyā dvijatāṁ gatāḥ\thinspace{\dandab} \dontdisplaylinenum

tapasā vinayācārair viṣṇunā dvijakalpitāḥ \veg\dontdisplaylinenum
\varr{
        \ \va soma°\lem  \mssCaCbCc\msNa\msNb\msNc\Ed; soya° \msL\oo
                 prasūtās te\lem  \msCb\msCc\msNb\msNc\Ed; pra{\lost}{\lost}{\lost} \msCa, prasūtas te \msNa\msL
        \ \vb kṣatriyā\lem  \mssCaCbCc\msNb; kṣatriyo \msNa\msNc\msL\Ed\oo
                 gatāḥ\lem  \mssCaCbCc\msNb\Ed; gataḥ \msNa\msNc\msL 
        \ \vc °cārair vi°\lem  \msCa\msCb\msNa\msNb\msNc\msL\Ed; °cārai vi° \msCc
        \ \vd dvijakalpitāḥ\lem  \Ed; dvijaḥ kalpitaḥ \mssCaCbCc\msNc\ \unmetr,
                                                dvijakalpitaḥ \msNa\msNb\msL}

ajitā nāma tat pūrvaṁ kāmakrodhajitena tu\thinspace{\dandab} \dontdisplaylinenum

saṁkalpas tasya rājendra kathayiṣyāmi tac ch\textsubring{r}ṇu \veg\dontdisplaylinenum
\varr{
        \ \va pūrvaṁ\lem  \mssCaCbCc\msNb\msNc\Ed; pūrva \msNa\msL
        \ \vc saṁkalpas ta\lem  \mssCaCbCc\msNa\msNb\msNc\Ed; saṁkalpa ta \msL}

adhyātmanagarasphītaḥ adhibhūtajanākulaḥ\thinspace{\dandab} \dontdisplaylinenum

adhidaivatasāṁnidhyaṁ daśāyatana pañca ca \veg\dontdisplaylinenum
            \paral{\textit{\vo {\normalfont Cf.\ 4.72: } caturāyatanaṁ vipra kathayiṣyāmi tac ch\textsubring{r}ṇu{\thinspace\danda}
                                 karuṇāmuditopekṣāmaitrī cāyātanaṁ sm\textsubring{r}tam{\thinspace\ketdanda}}}
\varr{
        \ \vab °sphītaḥ adhi°\lem  \msCb\msCc\msNa\msNb\msNc\msL\Ed; °sphītaradhi° \msCa
        \ \vc °sāṁnidhyaṁ\lem  \msCa\Ed; sānaidhyaṁ \msCb\msCc\msNa\msNb\msL, sānnaidhyaṁ \msNc
        \ \vd daśā°\lem  \mssCaCbCc\msNa\msNb\msNc\msL; deśā° \Ed}

daśayajñavrataṁ cīrṇaṁ daśakāmaparājitaḥ\thinspace{\dandab} \dontdisplaylinenum

niyamān daśa saṁśritya daśa vāyava \textsubring{r}tvijaḥ \veg\dontdisplaylinenum
            \paral{\textit{\vd {\normalfont cf.\ 11.17ab: } dhāraṇādhvaryuvat k\textsubring{r}tvā prāṇāyāmaś ca \textsubring{r}tvijaḥ}}
\varr{
        \ \va daśayajñavrataṁ cīrṇaṁ\lem  \msNa\msNb\msNc\msL; da\uncl{śayajñaṁ} {\lost}{\lost}ñ cīrṇan \msCa, 
                        daśayajñavratacīrṇan \msCb\msCc, daśayajñaṁ vrataṁ cīrṇa° \Ed
        \ \vb °parājitaḥ\lem  \msCa\msCc\msNa\msNb\msNc\msL\Ed; °paparājitaḥ \msCb
        \ \vc niyamān daśa\lem  \mssCaCbCc\msNa\msNb\msNc\Ed; nimāyā daśa \msLacorr, niyamā daśa \msLpcorr}

daśākṣareṇa mantreṇa daśadharmakriyāpadaḥ\thinspace{\dandab} \dontdisplaylinenum

daśasaṁyamadīptāgnau jihvātejodaśendriyaḥ \veg\dontdisplaylinenum
\varr{
        \ \vb °dharmakriyāpadaḥ\lem  \msCa\msCb\msNa\msNb\msNc\msL\Ed; °dharmaḥ kripadaḥ \msCc
        \ \vc °saṁyama°\lem  \mssCaCbCc\msNa\msNb\msNc\Ed; °saṁśaya° \msL\oo
                 °dīptā°\lem  \mssCaCbCc\msNa\msNc\msL; °dīpto \msNb, °dīpā° \Ed
        \ \vd °daśe°\lem  \mssCaCbCc\msNa\msNb\msL; °jite° \msNc\Ed}

daśayogāsanāsīno daśadhyānaparāyaṇaḥ\thinspace{\dandab} \dontdisplaylinenum

buddhir vedī mano yūpaḥ somapāno 'm\textsubring{r}tākṣaraḥ \veg\dontdisplaylinenum
\varr{
        \ \va °sanāsīno\lem  \mssCaCbCc\msNa\msNb\msNc\msL; samāsīnā \Ed
        \ \vb °yaṇaḥ\lem  \mssCaCbCc\msNb\msNc\Ed; °yaṇāḥ \msNa\msL
        \ \vc buddhir vedī\lem  \mssCaCbCc\msNa\msNb\msL; buddhi vedī \msNc, buddhir vedi \Ed
        \ \vd °pāno 'm\textsubring{r}tākṣaraḥ\lem  \msCb\msNa\msNb\msNc\msL; {\lost}{\lost}{\lost}{\lost}{\lost}{\lost} \msCa, 
                                              °pānam\textsubring{r}tākṣaraḥ \msCc, °dānam\textsubring{r}tākṣaraḥ \Ed}

dakṣiṇābhaya bhūtebhyaḥ paśubandha svayaṁk\textsubring{r}taḥ\thinspace{\dandab} \dontdisplaylinenum

vinārthaṁ yajñam iṣṭvā tu kālaṁ ca kṣapayaty asau \danda\dontdisplaylinenum

anarthayajñaṁ taṁ prāhur munayas tattvadarśinaḥ \veg\dontdisplaylinenum
\varr{
        \ \va °bhaya\lem  \mssCaCbCc\msNa\msNb\msNc\msL; °gnaya \Ed
        \ \va °rthaṁ\lem  \msCa\msCb\Ed; °rtha° \msCc\msNa\msNb\msNc\msL
        \ \vb kālaṁ\lem  \mssCaCbCc\msNa\msNb\msNc\msL; kālāñ \Ed\oo
                 kṣapayaty asau\lem  \mssCaCbCc\msNa\msNc\msL;
                        \uncl{kṣapayaty asau} \msNb, kṣapayaty asauḥ \Ed
        \ \vcd °yajñaṁ taṁ prāhur munayas ta°\lem  \msCa\msCb\msNb\msNc\Ed;
                °yajña taṁ prāhu munayas ta° \msCc, °yajñan taṁ prāhur munaya ta° \msNa,
                °yajñaṁ prāhur munaya ta° \msL}

janamejaya uvāca~{\dandab}\dontdisplaylinenum 

daśayajñam ahaṁ śrotuṁ dehi māṁ dvijasattama\thinspace{\danda} \dontdisplaylinenum

daśakāmadaśadhyānaṁ daśayogadaśākṣaram \veg\dontdisplaylinenum
\varr{
        \ \va °yajñam ahaṁ\lem  \mssCaCbCc\msNa\msNb\msNc\msLpcorr; °yajñam idaṁ \Ed
        \ \vb māṁ\lem  \msCa\msCb\msNa\msNb\msNc\msL\Ed; mā \msCc\oo
                 °ttama\lem  \mssCaCbCc\msNb\msNc\Ed; °ttamaḥ \msNa\msL
        \ \vc °daśadhyānaṁ\lem  \msCa\msCb\msNa\msNb\msNc; °daśadhyāna° \msCc\Ed, °datadhyānan \msL
        \ \vd °kṣaram\lem  \msCb\msNb\msNc; °kṣara{\lost} \msCa, °kṣaraḥ \msCc\msNa\msL\Ed}

vaiśampāyana uvāca~{\dandab}\dontdisplaylinenum 
\varr{
        \ \vo vaiśampāyana uvāca\lem  \msCb\msCc\msNa\msNb\msNc\msL\Ed; {\lost}{\lost}{\lost}{\lost}{\lost}{\lost}vāca \msCa}

brahmadevapit\textsubring{r}yajño yajño bhūtātitheś ca ha\thinspace{\danda} \dontdisplaylinenum
            \paral{\textit{\vb {\normalfont cf.\ Śatapathabrāhmana 11.5.6: } aharaharbhūtebhyo baliṁ haret tathaitam bhūtayajñaṁ\oo
                {\normalfont Garuḍapurāṇa 1.50.71cd: } bhūtayajñaḥ sa vai jñeyo bhūtebhyo yastvayaṁ baliḥ}} 

japo yogas tapo dhyānaṁ svādhyāyaś ca daśa sm\textsubring{r}taḥ \veg\dontdisplaylinenum
\varr{
        \ \va °deva°\lem  \msCa\msCc\msNa\msNb\msNc\msL\Ed; °daiva° \msCb\oo
                 °yajño\lem  \msCa\msCb\msNa\msNb\Ed; °yojño \msNc, °yajña \msCc\msL
        \ \vb yajño\lem  \msCa\msCb\msNa\msL; yajña° \msCc\msNb\msNc\Ed\oo
                 °titheś ca ha\lem  \msCb; °tithiś ca ha \msCa\msCc\msNa\msNb\msNc\msL, °tithiñ ca yaḥ \Ed
        \ \vc yogas tapo dhyānaṁ\lem  \mssCaCbCc\msNb\msNc\Ed; yoga{\lost}{\lost}\uncl{dhānaṁ} \msNa,
                                                yoga \gap\gap\ pānaṁ \msL
        \ \vd svādhyāyaś ca\lem  \mssCaCbCc\msNb\msNc\Ed; \uncl{sādhyā}yaś ca \msNa, 
                                                sādhutapaś ca \msL}

patnīputrapaśubh\textsubring{r}tyadhanadhānyayaśaḥśriyaḥ\thinspace{\dandab} \dontdisplaylinenum

māna bhoga daśa rājan daśakāma udāh\textsubring{r}taḥ \veg\dontdisplaylinenum
\varr{
        \ \va °yaśaḥ°\lem  \msCa\msCb\msNa\msNb\msNc\msL; °yaśa° \msCc\Ed
        \ \vc °bhoga\lem  \mssCaCbCc\msNa\msNb\msNc\msL; °bhogaṁ \Ed
        \ \vd °h\textsubring{r}taḥ\lem  \msCa\msCc\msNa\msNb\msNc\msL\Ed; °h\textsubring{r}tam \msCb}

mānaso yaugapadyaś ca saṁkṣiptaś ca viśāmpate\thinspace{\dandab} \dontdisplaylinenum
             \paral{\textit{\vo {\normalfont cf.\ Dharmaputrikā 1.56: } saṁkṣiptā prathamā jñeyā viśālā samanantaram{\thinspace\ketdanda}
                                        tato dvikaraṇī ceti trividho yoga ucyate{\thinspace\danda}}}

viśālā nāma yogaś ca tato dvikaraṇaḥ sm\textsubring{r}taḥ \veg\dontdisplaylinenum
\varr{
        \ \va yaugapadyaś ca\lem  \corr; yaugapadyañ ca \msCa\msCb\msNb,
                        yogapadyaṁ ca \msCc\msNa\msNc\msL, yogapadyaś ca \Ed
        \ \vb °kṣiptaś ca\lem  \Ed; °ksiptaṁ ca \mssCaCbCc\msNa\msNb\msNc\msL
        \ \vc viśālā nāma yogaś ca\lem  \Ed; vi{\lost}{\lost}{\lost}{\lost} yogañ ca \msCa, 
                                        viśālā nāma yogaṁ ca \msCb\msCc\msNa\msNb\msNc\msL
        \ \vd dvikaraṇaḥ\lem  \msCa\msCb\msNa\msL; vikaraṇaḥ \msCc\Ed, dvikaraṇī \msNb, dvikaraṇa \msNc}

raviḥ somo hutāśaś ca sphaṭikāmbaram eva ca\thinspace{\dandab} \dontdisplaylinenum
            \paral{\textit{\vab {\normalfont cf. Dharmaputikā 4:5cd: } sūryacandrahutāśārciḥsphāṭikāmbarasannibhāḥ}}

daśayogāsanāsīno nityam eva tapodhanaḥ \veg\dontdisplaylinenum
\varr{
        \ \va raviḥ\lem  \msCa; ravi° \msCb\msCc\msNa\msNb\msNc\msL\Ed
        \ \vb sphaṭikāmbara°\lem  \mssCaCbCc\msNb\msNc\Ed; sphaṭikāṁ{\lost}ra° \msNa, sphaṭikāṁsata° \msL
        \ \vc daśayogāsanāsīno\lem  \msCa\msCc\msNa\msNb\msNc; daśayogasamāsīno \msCb,
                                       devayogāsatāsīno \msL, daśayogāsanāsīnau \Ed
        \ \vd °dhanaḥ\lem  \msCa\msCb\msNa\msL; °dhana \msCc\msNb\msNc\Ed}

anirodhamanāḥ sūkṣmaṁ dhyāyed yogaḥ sa mānasaḥ\thinspace{\dandab} \dontdisplaylinenum
            \paral{\textit{\vab {\normalfont cf.\ Dharmaputrikā 1.54: } ak\textsubring{r}tvā prāṇasaṁrodhaṁ manasaikena kevalam{\thinspace\danda}
                                dhyāyeta paramaṁ sūkṣmaṁ sa yogo mānasaḥ sm\textsubring{r}taḥ{\thinspace\ketdanda}}}

prāṇāyāmair mano ruddhvā yaugapadyaḥ sa ucyate \veg\dontdisplaylinenum
            \paral{\textit{\vcd {\normalfont cf.\ Dharmaputrikā 1.55:} saṁyamya manasā prāṇaṁ prāṇāyāmair manas tathā{\thinspace\danda}
                                        evaṁ dhyāyet paraṁ sūkṣmaṁ yaugapadyaḥ sa ucyate{\thinspace\ketdanda}}}
\varr{
        \ \va anirodha°\lem  \mssCaCbCc\msNa\msNb\msNc\msL; anilādha° \Ed\oo
                 °manāḥ\lem  \mssCaCbCc\msNa\msNc\msL\Ed; °manā \msNb
        \ \vb dhyāyed yo°\lem  \msCa\msCb\msNa\msNb\msNc\msL; dhyāyo° \msCc, dhyānaṁ yo° \Ed
        \ \vc °yāmair ma°\lem  \msCa\msNa\msNb\msNcpcorr\msL\Ed;
                                 °yāmai ma° \msCb, °yāmai mma° \msCc, °yāmer ma° \msNcacorr\oo
                 ruddhvā\lem  \mssCaCbCc\msNa\msNb\msNc\msL; ruddhā \Ed
        \ \vd yauga°\lem  \msCa\msCb\msNa\msNcpcorr\msL; yoga° \msCc\msNb\msNcacorr\Ed}

brahmādistambaparyantaṁ sarvaṁ sthāvarajaṅgamam\thinspace{\dandab} \dontdisplaylinenum
            \paral{\textit{\vab {\normalfont \kb\ Dharmaputrikā 1.57cd: } brahmādistambhaparyantāḥ sarve sthāvarajaṅgamāḥ}}

pralīyamānaṁ dhyāyeta kramāt sūkṣmaṁ vicintayet \veg\dontdisplaylinenum
            \paral{\textit{\vcd {\normalfont \kb\ Dharmaputrikā 1.59ab: } pralīyamānan dhyāyeta kramāc chūnyaṁ bhavej jagat}}
\varr{
        \ \vo (brahmādi°{\normalfont ...} vicintayet)\lem  \mssCaCbCc\msNa\msNc\msL\Ed; \om\ \msNb
        \ \va °stamba°\lem  \mssCaCbCc\msNa\msNc\Ed; \om\ \msNb, °staṁbha° \msL\oo
                 °paryantaṁ\lem  \msCb\msCc\msNa\msL; °\uncl{dviya}{\lost}° \msCa, \om\ \msNb, °paryanta° \msNc\Ed
        \ \vb sarvaṁ\lem  \msCb\msNa; {\lost}{\lost} \msCa, sarva° \msCc\msNc\msL\Ed, \om\ \msNb
        \ \vc pralīya°\lem  \mssCaCbCc\msNa\msNc\Ed; \om\ \msNb, praṇīya° \msL
        \ \vd kramāt sū°\lem  \msCa\msCb\msNa\msNc\msL\Ed; kramā sū° \msCc, \om\ \msNb}

saṁkṣipta eṣa ākhyāto viśālāṁ ch\textsubring{r}ṇu tattvataḥ\thinspace{\dandab} \dontdisplaylinenum
            \paral{\textit{\vab {\normalfont cf.\ Dharmaputrikā 1.60ab: } eṣa yogavidhiḥ proktaḥ saṁkṣipto nāma nāmataḥ}}

brahmādisūkṣmaparyantaṁ cintayīta vicakṣaṇaḥ \veg\dontdisplaylinenum
\varr{
        \ \vo (saṁkṣipta{\normalfont ...} vicakṣaṇaḥ)\lem  \mssCaCbCc\msNa\msNc\msL\Ed; \om\ \msNb
        \ \va saṁkṣipta\lem  \mssCaCbCc\msNa\msNc\Ed; \om\ \msNb, saṁkṣiptaḥ \msL\oo
                 eṣa\lem  \mssCaCbCc\msNa\msNc\msL; \om\ \msNb, eva \Ed\oo
                 ākhyāto\lem  \msCb\msNc; ākhyātaḥ \msCa\msCc\msNa\msL\Ed, \om\ \msNb
        \ \vc °sūkṣma°\lem  \mssCaCbCc\msNc\Ed; °staṁba° \msNa, \om\ \msNb, tava \msL\oo
                 °paryantaṁ\lem  \mssCaCbCc\msNa\msL; \om\ \msNb, °paryanta \msNc\Ed
        \ \vd cintayīta\lem  \msCa\msCbpcorr\msCc\msNa\msNc\msL\Ed; \om\ \msNb, ciyīta \msCbacorr}

saṁkṣiptāṁ ca viśālāṁ ca cintayīta parasparam\thinspace{\dandab} \dontdisplaylinenum

eṣā dvikaraṇī nāma yogasya vidhir ucyate \veg\dontdisplaylinenum
            \paral{\textit{\vo {\normalfont \kb\ Dharmaputrikā 1.62cd--63ab: } etau saṁhāravargau dvau pāramparyeṇa cintayet{\thinspace\ketdanda}
                                                       eṣā dvikaraṇī nāma yogasya vidhir iṣyate{\thinspace\danda}}}
\varr{
        \ \vo (saṁkṣiptāṁ{\normalfont ...} vidhir ucyate)\lem  \mssCaCbCc\msNa\msNc\msL\Ed; \om\ \msNb
        \ \va saṁkṣiptāṁ\lem  \msCb\msNc; saṁkṣiptā \msCapcorr\msCc\msNa\msL\Ed, \om\ \msCaacorr\msNb\oo
                 viśālāṁ\lem  \msCapcorr\msCb\msNc; \om\ \msCaacorr, viśālā \msCc\msNa\msL\Ed, \om\ \msNb
        \ \vc dvi°\lem  \msCa\msCb\msNa\msNc\msL; vi° \msCc\Ed, \om\ \msNb}

dehamadhye h\textsubring{r}di jñeyaṁ h\textsubring{r}dimadhye tu paṅkajam\thinspace{\dandab} \dontdisplaylinenum

paṅkajasya ca madhye tu karṇikāṁ viddhi gopate \veg\dontdisplaylinenum
\varr{
        \ \va jñeyaṁ\lem  \msCa\msCb\msNa\msNc\Ed; jñeya \msCc\msL, jñe \msNbacorr, jñe{\lost} \msNbpcorr
        \ \vb tu paṅkajam\lem  \msCb\msCc\msNa\msNb\msNc\msL\Ed; \uncl{tu} pa{\lost}{\lost} \msCa
        \ \vc paṅkajasya ca\lem  \msCb\msCc\msNa\msNc\Ed; {\lost}ṅkajasya ca \msCa,
                                                        kaṅkasya tu \msNb, pankajaṁsya ca \msL
        \ \vd karṇikāṁ viddhi gopate\lem  \msCa\msCb\msNa\msNb\msNc\msL; karṇiddhiddhi gopate \msCc, 
                                                               karṇikāṁ ca viṁśāpate \Ed}

karṇikāyās tu madhye tu pañcabinduṁ vidur budhāḥ\thinspace{\dandab} \dontdisplaylinenum

ravisomaśikhāṁ caiva sphaṭikāmbaram eva ca \veg\dontdisplaylinenum
    \paral{\textit{\vcd {\normalfont cf.\ Dharmaputrikā 4.5cd: } sūryacandraprakā\-śārcisphāṭikāmbarasannibhāḥ}}
\varr{
        \ \vb °binduṁ\lem  \msCa\msNc; °bindu \msCb\msCc\msNa\msNb\msL\Ed
        \ \vc °śikhāṁ\lem  \msCa\msNa\msL; °śikhā \msCb\msCc\msNb\msNc\Ed
        \ \vd sphaṭi°\lem  \msCa\msCc\msNa\msNb\msNc\msL\Ed; sphāṭi° \msCb}

ravimaṇḍalamadhye tu bhāvayec candramaṇḍalam\thinspace{\dandab} \dontdisplaylinenum

tasya madhye śikhāṁ dhyāyen nirdhūmajvalanaprabhām \veg\dontdisplaylinenum
\varr{
        \ \vb bhāvayec candramaṇḍalam\lem  \msCa\msCb\msNa\msNb\msNc\msL\Ed; bhāvaye candramaṇḍalaḥ \msCc
        \ \vc °śikhāṁ\lem  \msCa\msCb\msNa\msNb\msNc\msL; °śikhā \msCc\Ed}

agnimadhye maṇiṁ dhyāyec chuddhadhārājalaprabham\thinspace{\dandab} \dontdisplaylinenum

tasya madhye 'mbaraṁ dhyāyet susūkṣmaṁ śivam avyayam \veg\dontdisplaylinenum
\varr{
        \ \vab maṇiṁ dhyāyec chuddha°\lem  \msCb\msNa\msNb\msNc\msL\Ed; {\lost}{\lost}{\lost}{\lost}{\lost}{\lost} \msCa, 
                                                maniṁ dhyāyec chuddha° \msCc
        \ \vb °dhārā°\lem  \msCa\msCb\msNa\msNb\msNc\msL; °dhāra° \msCc\Ed\oo
                  °prabham\lem  \msCc\msNa\msNb\msNc\msL\Ed; °prabhām \msCa\msCb
        \ \vc 'mbaraṁ\lem  \msCa\msCb\msNa\msNb\msNc; 'mbara \msCc, baraṁ \msL, 'kṣaraṁ \Ed
        \ \vd susūkṣmaṁ\lem  \msCc\msNa\msNc\msL; sūkṣmaṁ \msCa, susūkṣma° \msCb, 
                                        \uncl{sva}sūkṣma° \msNb, sasūkṣmaṁ \Ed}

daśayogam idaṁ rājan kathitaṁ ca mayā tava\thinspace{\dandab} \dontdisplaylinenum

daśadhyānaṁ samāsena kīrtitaṁ ś\textsubring{r}ṇu tad yathā \veg\dontdisplaylinenum
\varr{
        \ \vc °dhyānaṁ\lem  \mssCaCbCc\msNa\msNc; °dhyāna \msNb\msL\Ed}

ghoṣaṇī piṅgalā caiva vaidyutī candramālinī\thinspace{\dandab} \dontdisplaylinenum

candrā mano'nugā caiva suk\textsubring{r}tā ca tathāparā \veg\dontdisplaylinenum
\varr{
        \ \va ghoṣaṇī\lem  \mssCaCbCc\msNa\msNb\msNc\msL; ghoṣaṇā \Ed
        \ \vb vaidyutī\lem  \msCa\msCb\msNa\msNb\msNc\msL\Ed; vidyuta \msCc, vidyutī \Ed
        \ \vc candrā mano'nugā\lem  \msCb\msNa\msNb\msNc\msL; 
                                candrā manānugā \msCa, candramanonugā \msCc, candro mano'nugā \Ed
        \ \vd suk\textsubring{r}tā ca tathāparā\lem  \msCa\msCc\msNa\msNc\msL; suk\textsubring{r}tā tathāparā \msCb, \om\ \msNb,
                                                                suk\textsubring{r}tā ca tathāpara \Ed}

saumyā nirañjanā caiva nirālambā ca kīrtitā\thinspace{\dandab} \dontdisplaylinenum

supiṣitvāṅgulau śrotre dhvanim ākarṇayen naraḥ \veg\dontdisplaylinenum
\varr{
        \ \va saumyā nirañjanā caiva\lem  \msCb\msCc\msNa\msL\Ed; saumyā nirañjanā {\lost}{\lost} \msCa, \om\ \msNb,
                                                                saumyā ṇirañjanā caiva \msNc
        \ \vb kīrtitā\lem  \mssCaCbCc\msNa\msNb\msNc\Ed; kīrtitāḥ \msL
        \ \vc supiṣitvāṅgulau\lem  \msCa\msCb\msNa\msNb\msNc; su{\lost}{i}{}ṣicāṅgulau \msCc,
                                               supithitvāṅgulau \msL, suśiṣi cāṅgulau \Ed
        \ \vd °karṇaye°\lem  \msNb; °karṣaye° \mssCaCbCc\msNa\msNc\Ed, °karṣaya° \msL}

tat tad akṣaram ākarṇya am\textsubring{r}tatvāya kalpyate\thinspace{\dandab} \dontdisplaylinenum

piṅgalāṁ tu śikhādhūmāṁ dhyāyen nityam atandritaḥ \veg\dontdisplaylinenum
\varr{
        \ \va °karṇya\lem  \mssCaCbCc\msNb\msNc\msL\Ed; °kaṇṇya \msNa
        \ \vc piṅgalāṁ tu śikhādhūmāṁ\lem  \msCa\msCb\msNb\msL; piṅgalā tu śikhādhūmaṁ \msCc\Ed,
                                        piṅgalāṁn tu śikhādhūmāṁ \msNa, piṅgalān tu śikhādhūmā \msNc
        \ \vd °tandritaḥ\lem  \mssCaCbCc\msNa\msNb\msNc\Ed; °tendritaḥ \msL}

vimuktaḥ sarvapāpebhyo nirdvandvapadam āpnuyāt\thinspace{\dandab} \dontdisplaylinenum

vaidyutī tu niśāmadhye lakṣate 'jam anāmayam \veg\dontdisplaylinenum
\varr{
        \ \va vimuktaḥ\lem  \msCa\msCb\msNa\msNb\msNc\msL\Ed; vimukta \msCc
        \ \vb nirdvandva°\lem  \mssCaCbCc\msNc; nidvanda° \msNa\msNb\msL, nirdvanda° \Ed
        \ \vc vaidyutī tu\lem  \mssCaCbCc\msNa\msNb\msNc\Ed; vaidyutīnta \msL
        \ \vd lakṣate 'jam a°\lem  \msCc\Ed; lakṣye teja a° \msCa\msCb, lakṣyateja a° \msNa\msNb\msL,
                                                                                 lakṣateja a° \msNc}

pañcamāsasadābhyāsād divyacakṣur bhaven naraḥ\thinspace{\dandab} \dontdisplaylinenum

bindumālāṁ tataḥ paśyet tarucchāyāsamāśritām \veg\dontdisplaylinenum
\varr{
        \ \va pañcamāsasadā°\lem  \msCb\msNa\msNb\msL; \uncl{pa}{\lost}{\lost}sasadā° \msCa, pañcamāsassadā° \msCc, 
                                                      pañcamāsasamā° \Ed, pañcamāsaṁ sadā° \msNc
        \ \vab °sād di\lem  \mssCaCbCc\msNa\msNb\msL\Ed; °sā di° \msNc
        \ \vb °kṣur bhaven na°\lem  \msCa\msCb\msNa\Ed; °kṣur bhave na° \msCc,
                                        °kṣu bhaven na° \msNb\msL, °rkṣu bhaven na \msNc
        \ \vc tataḥ paśyet\lem  \mssCaCbCc\msNa\msNb\msNc\msL; tu yaḥ paśyen \Ed
        \ \vd tarucchāyā°\lem  \mssCaCbCc\msNa\msNb\msNc\msL; naracchāyāṁ \Ed\oo
                 °śritām\lem  \mssCaCbCc\msNb; °śritāḥ \msNa\msL, °śritam \msNc\Ed}

jātyasphaṭikasaṁkāśaṁ d\textsubring{r}ṣṭvā mucyati bandhanaiḥ\thinspace{\dandab} \dontdisplaylinenum

dhyāyen mano'nugā nāma pakṣmīr āpīḍya locane \veg\dontdisplaylinenum
\varr{
        \ \va °kasaṁkāśaṁ\lem  \mssCaCbCc\msNa\msNb\msNc\msLpcorr\Ed; °saṁkakāśaṁ \msLpcorr
        \ \vb bandhanaiḥ\lem  \msCa\msNa\msNc; bandhavaiḥ \msCb, bandhanāt \msCc\msNb\Ed,
                                                                                vaṁcanaiḥ \msL
        \ \vd pakṣmī°\lem  \mssCaCbCc\msNa\msL; yakṣmī \msNb, yakṣmo° \msNc, pakṣī° \Ed\oo
                 locane\lem  \msCa\msCb\msNa\msL; locanaḥ \msNb, locanaiḥ \msCc\Ed, locanai \msNc}

śvetapītāruṇaṁ binduṁ d\textsubring{r}ṣṭvā bhūyo na jāyate\thinspace{\dandab} \dontdisplaylinenum

mano'nugādi ṣaṭ tv ete dhyānam uktaṁ mayā tava \veg\dontdisplaylinenum


\alalfejezet{paramāṇuḥ}\varr{
        \ \vc °ṣaṭ tv ete\lem  \msCa\msNa\msNb\msNc\msL; °ṣaṭ tv etā \msCb, °ṣaṭkena \msCc\Ed
        \ \vd °ktaṁ mayā tava\lem  \msCc\msNa\msNc\msL\Ed; \uncl{ka}{\lost}{\lost} tava \msCa, °ktaṁ samāsataḥ \msCb,
                                                                                °kta mayā tava \msNb}

adhunānyat pravakṣyāmi paramāṇu caturvidham\thinspace{\dandab} \dontdisplaylinenum

pārthivādicaturbhūtaṁ yair vyāptaṁ nikhilaṁ jagat \danda\dontdisplaylinenum

lakṣaṇaṁ tasya rājendra ś\textsubring{r}ṇu vakṣyāmi sāmpratam \veg\dontdisplaylinenum
\varr{
        \ \vb °vidham\lem  \msCa\msCc\msNa\msNb\msNc\Ed; °vidhaḥ \msCb
        \ \vcd °bhūtaṁ yair vyāptaṁ\lem  \msNa; °bhūtaṁ yair vyāptin \msCa,
                                °bhūtaṁ yai vyāptaṁ \msCb\msCc\msNb, °bhūtaṁ yai vyāpta \msNc, °bhūtair yair vyāptaṁ \Ed}

pārthivordhvagatiḥ sūkṣmaḥ paramāṇu narādhipa\thinspace{\dandab} \dontdisplaylinenum

pratyakṣadarśanaṁ dhyānaṁ lakṣayen niyataṁ śuciḥ \veg\dontdisplaylinenum
\varr{
        \ \va pārthivordhva°\lem  \mssCaCbCc\msNa\msNb\msNc; pārthivorddha° \Ed
        \ \vb paramāṇu narādhipa\lem  \msCa\msCb\msNapcorr; paramāṇu narādhipaḥ \msCc,
                                                paramāṇu narādhinarādhipa \msNaacorr, paramānu narādhipa \msNb,
                                                paramāṇur narādhipa \msNc\Ed
        \ \vc pratyakṣadarśanaṁ\lem  \mssCaCbCc\msNb\Ed; pratyakṣaṁ darśanaṁ \msNa\msNc
        \ \vd lakṣayen niyataṁ\lem  \msCa\msNa\msNb\msNc; lakṣayen niyataḥ \msCb,
                                                 lakṣayen niyata \msCc, lakṣayan niyataḥ \Ed}

mucyate sarvapāpebhyo rāhunā candramā yathā\thinspace{\dandab} \dontdisplaylinenum

tena yo 'bhyasate nityaṁ sa yogī bhuvaneśvaraḥ \veg\dontdisplaylinenum
\varr{
        \ \va sarvapāpebhyo\lem  \msCb\msCc\msNa\msNb\msNc\Ed; \uncl{sarvapāpebhyo} \msCa
        \ \vb rāhunā\lem  \msCb\msCc\msNa\msNb\msNc\Ed; {\il}{\il}nā \msCa
        \ \vc 'bhyasate\lem  \msCa\msCc\msNa\msNb\msNc\Ed; labhyate \msCb
        \ \vd °śvaraḥ\lem  \mssCaCbCc\msNa\msNb\msNc; °śvara \Ed}

adhogati mahārāja paramāṇu jalodbhavaḥ\thinspace{\dandab} \dontdisplaylinenum

abhyased yad idaṁ rājan sarvapātakanāśanam \veg\dontdisplaylinenum
\varr{
        \ \vb paramāṇu ja°\lem  \mssCaCbCc\msNa\msNb\Ed; paramānur ja° \msNc
        \ \vc abhyased yad idaṁ\lem  \msCa\msCb\msNa\msNb\msNc\Ed; abhyased idaṁ \msCc}

āgneyaparamāṇūni tiryagūrdhvagatiḥ sm\textsubring{r}tā\thinspace{\dandab} \dontdisplaylinenum

ya idaṁ dhyāyate nityam uttamāṁ gatim āpnuyāt \veg\dontdisplaylinenum
\varr{
        \ \va āgneya°\lem  \mssCaCbCc\msNa\msNc\Ed; agneya° \msNb\oo
                 °paramāṇūni\lem  \mssCaCbCc\msNa\msNc; °paramānūni \msNb, paramāṇuś ca \Ed
        \ \vb tiryagūrdhva°\lem  \mssCaCbCc\msNa\msNb\msNc; tiryagūrddha° \Ed\oo
                 °gatiḥ\lem  \mssCaCbCc\msNa\msNc\Ed; °mitiḥ \msNb\oo
                 sm\textsubring{r}tā\lem  \msCa\msNa; sm\textsubring{r}tāḥ \msCb\msCc\msNb\msNc\Ed
        \ \vd gatim āpnu°\lem  \msCa\msCc\msNa\msNb\msNc\Ed; phalam āpnu° \msCb}

vāyavyaparamāṇūni adhordhvatiryag āsm\textsubring{r}tā\thinspace{\dandab} \dontdisplaylinenum

na sa muhyati taṁ d\textsubring{r}ṣṭvā vāyusambhava bhūpate \veg\dontdisplaylinenum
\varr{
        \ \va vāyavyaparamāṇūni\lem  \msCb\msNc; vāya{\il}{\il}ramāṇūni \msCa, vāyavyaṁ paramāṇūni \msCc\msNa,
                                                        vāyavyā paramāṇūni \msNb, vāyavyaṁ paramāṇuś ca \Ed
        \ \vb °rdhvatirya°\lem  \mssCaCbCc\msNb\msNc\Ed; °rdhvantirya° \msNa}

catvāra ete rājendra paramāṇu nirīkṣate\thinspace{\dandab} \dontdisplaylinenum

tena sarvamakhair iṣṭaṁ tena taptaṁ tapas tathā \veg\dontdisplaylinenum
\varr{
        \ \vb paramāṇu nirīkṣate\lem  \msCa\msCc\msNa\msNb; paramāṇur rīkṣate \msCb, paramāṇuṁ nirīkṣate \msNc, 
                                                                paramāṇur nirīkṣate \Ed
        \ \vc °makhair i°\lem  \msCa\msCb\msNa\msNb\Ed; °mayair i° \msCc, °makher i° \msNc
        \ \vd taptaṁ tapas tathā\lem  \msCa\msCb\msNa\msNb; taptaṁ tapan tathā \msCc, 
                                                       saptan tapas tathā \msNc, taptan taptaṁ tathā \Ed}

tena dattā mahī k\textsubring{r}tsnā saptasāgarasaṁv\textsubring{r}tā\thinspace{\dandab} \dontdisplaylinenum

sarvatīrthābhiṣekaś ca sarvavratakriyā tathā \veg\dontdisplaylinenum
\varr{
        \ \vc °bhiṣekaś ca\lem  \mssCaCbCc\msNb\msNc\Ed; °bhiṣeka \msNaacorr, °bhiṣekaṁ ca \msNapcorr}

anenaiva vidhānena daśadhyānaṁ narādhipa\thinspace{\dandab} \dontdisplaylinenum

kurute avyavacchinnaṁ sarvakāmaphalapradam \veg\dontdisplaylinenum


\alalfejezet{daśākṣaramantraḥ}\varr{
        \ \va anenaiva vidhānena\lem  \msCb\msCc\msNa\msNb\msNc\Ed; a{\il}{\lost}{\lost}{\lost}dhānena \msCa
        \ \vc °cchinnaṁ\lem  \mssCaCbCc\msNa\msNc\Ed; °cchinna \msNb}

daśākṣaraṁ mahārāja yogīndrasya mahātmanaḥ\thinspace{\dandab} \dontdisplaylinenum

kathayāmi samāsena ś\textsubring{r}ṇuṣvāvahito bhava \veg\dontdisplaylinenum
\varr{
        \ \va daśākṣaraṁ\lem  \msCa\msCb\msNa\msNb\msNc; daśākṣara° \msCc\Ed}

praṇavādisvarā trīṇi dīrghabindusamāyutam\thinspace{\dandab} \dontdisplaylinenum

pañca pañca cavarge tu vāyubījam adhaḥsthitam \veg\dontdisplaylinenum
\varr{
        \ \va trīṇi\lem  \mssCaCbCc\msNa\msNb\Ed; trīni \msNc
        \ \vc pañca pañca\lem  \mssCaCbCc\msNa\msNb\msNc\Ed; pañca{\il}{\il} \Ed\oo
                 tu\lem  \mssCaCbCc\msNa\msNc\Ed; ca \msNb
        \ \vd °dhaḥsthitam\lem  \mssCaCbCc\msNb\msNc; °dhasthitaṁ \msNa\Ed}

trayodaśasvarāyuktaṁ pañcame parikīrtitam\thinspace{\dandab} \dontdisplaylinenum

pañcaviṁśatimaḥ ṣaṣṭhaḥ akṣaraḥ parikīrtitaḥ \veg\dontdisplaylinenum
\varr{
        \ \va trayodaśa°\lem  \msCb\msCc\msNa\msNb\Ed; {\lost}yodaśa° \msCa\msNc\oo
                 °yuktaṁ\lem  \mssCaCbCc\msNa\msNb\Ed; °yukta \msNc
        \ \vb pañcame \lem \mssCaCbCc\msNa\msNb\msNc; pañcama \Ed
        \ \vc ṣaṣṭhaḥ\lem  \mssCaCbCc\msNa\msNb\msNc; ṣaṣṭha \Ed}

yād\textsubring{r}śaṁ pañcame proktaṁ saptame ca prayojayet\thinspace{\dandab} \dontdisplaylinenum

ākārasvarasaṁyuktaṁ sarvapātakanāśanam \veg\dontdisplaylinenum
\varr{
        \ \va pañcame\lem  \msCb\msNa\msNc; pañcameḥ \msCa\msCc\msNb, pañcamaḥ \Ed
        \ \vc ākāra°\lem  \msCa\msCb\msNa\msNb\msNc; akāra° \msCc\Ed}

prathamaṁ pañcame varge t\textsubring{r}tīyasvarayojitam\thinspace{\dandab} \dontdisplaylinenum

uttarekārasaṁyuktaṁ navamaṁ parikīrtitam \veg\dontdisplaylinenum
\varr{
        \ \vc uttarekārasaṁyuktaṁ\lem  \msCb\msCc\msNa\msNb\msNc; uttarekā{\lost}{\lost}{\il}{\il} \msCa, uktarekārasaṁyuktaṁ \Ed
        \ \vb navamaṁ parikīrtitam\lem  \msCb\msCc\msNa\msNb\msNc\Ed; {\il}{\il}{\lost}{\lost}rikīrtitam \msCa}

daśamaḥ punar oṁkāraḥ mantraśreṣṭho daśākṣaraḥ\thinspace{\dandab} \dontdisplaylinenum

japato dhyāyato vāpi pārthivādikrameṇa tu \danda\dontdisplaylinenum

mucyate so 'pi saṁsāre saṁśayo nāsti bhūpate \veg\dontdisplaylinenum


\alalfejezet{ācāravidhiḥ}\varr{
        \ \va °kāraḥ\lem  \msCa\msCc\msNa\msNb\msNc\Ed; °kārau \msCb
        \ \va saṁsāre\lem  \mssCaCbCc\msNb\msNc\Ed; saṁsāra \msNa}

ācāramūlo dharmas tu dharmamūlo janārdanaḥ\thinspace{\dandab} \dontdisplaylinenum

tena sarvajagad vyāptaṁ trailokyaṁ sacarācaraṁ \veg\dontdisplaylinenum

\ujvers\nemsloka 
ācārāl labhatīha āyur atulam akṣapyavittaṁ tathā
\dontdisplaylinenum

\nemslokab 
ācārāt sutam īpsitaṁ ca labhate śrīkīrtiprajñāyaśaḥ \danda\dontdisplaylinenum

\nemslokac 
ācārāl labhate ca lakṣmim atulāṁ khyātiṁ tathaivottamām
\dontdisplaylinenum

\nemslokad 
ācārād iha mantradharmaparamaṁ prāpnoti niḥsaṁśayam \veg\dontdisplaylinenum

\vers
\varr{
        \ \va ācārāl labhatīha\lem  \msCb\msNa\msNb\msNc\Ed; {\lost}{\lost}{\lost}{\lost}bhatīha \msCa, ācārā labhatīha \msCc\oo
                 °m akṣapyavittaṁ tathā\lem  \mssCaCbCc\msNa\msNb\msNc; °m aiśvaryyavittan tathā \Ed
        \ \vb ācārāt sutam īpsitaṁ ca\lem  \msCb\msCc\msNa\msNb\msNc\Ed; ācārāt su\uncl{tam īpsita}{\lost} \msCa\oo
                 śrīkīrtiprajñāyaśaḥ\lem  \msCa\msCb\msNapcorr\msNb\msNc\Ed; śrīkīrtiprajñāṁ yaśaḥ \msCc, \om\ \msNaacorr
        \ \vc ācārāl labhate\lem  \mssCaCbCc\msNapcorr\msNb\msNc\Ed; \om\ \msNaacorr\oo
                 lakṣmim atulāṁ\lem  \msCb\msCc; latulaṁ \msCa, lakṣmim atulaṁ \msNa\msNb\Ed, {\il}kṣmim atulaṁ \msNc\oo
                 °ttamām\lem  \msCb\msCc\msNa; °ttamam \msCa\msNb\msNc\Ed
        \ \vd ācārād iha\lem  \mssCaCbCc\msNa\msNc\Ed; ācārādiṣu \msNb\oo
                 °paramaṁ\lem  \mssCaCbCc\msNa\msNb\Ed; °parama \msNc\oo
                 °saṁśayam\lem  \mssCaCbCc\msNc\Ed; °saṁśayaḥ \msNa\msNb}

janamejaya uvāca~{\dandab}\dontdisplaylinenum 

\nemsloka 
ācārāt prabhavānusaṁśakathitaṁ suśliṣṭadharmākaram
\dontdisplaylinenum

\nemslokab 
ācārāt kativaṁśa kīrtaya punas t\textsubring{r}ptir na me jāyate \danda\dontdisplaylinenum

\nemslokac 
sarvajñaḥ tvam ahaṁ ś\textsubring{r}ṇomi varadaṁ kiñcin na me śāśvatam
\dontdisplaylinenum

\nemslokad 
tan me kīrtaya dharmasāraśubhadam ācāramūlāśrayam \veg\dontdisplaylinenum

\vers
\varr{
        \ \va °saṁśakathitaṁ suśliṣṭa°\lem  \msCb; °saṁśa{\lost}{\lost}{\lost}{\lost}śliṣṭa \msCa, °saṁsayathitaṁ suśliṣva° \msCc, °saṅgakathitaṁ suśliṣṭa° \Ed
        \ \vb °tivaṁśa\lem  \msCa\msCb; °thivaṁsa \msCc, °tidhā'ṅga \Ed\oo
                 punas t\textsubring{r}ptir na\lem  \msCa\msCb\Ed; punas t\textsubring{r}pti nna \msCc
        \ \vc °jñaḥ\lem  \msCa\msCb\Ed; °jña \msCc\oo
                 kiñcin na\lem  \msCc\Ed; kiñcana \msCa, kiñcān \msCb
        \ \vd °sāraśubhadam ā°\lem  \msCa\msCb\Ed; °sābhadaṁ \msCc}

vaiśampāyana uvāca~{\dandab}\dontdisplaylinenum 

\nemsloka 
nityaṁ namraśirodvijātiguruṣu śuśrūṣaṇaṁ devatā
\dontdisplaylinenum

\nemslokab 
tiṣṭhetācamanena cāśanakaraṁ vāmāsthinānodade \danda\dontdisplaylinenum

\nemslokac 
sūryāgniśaśibandhur āryapurataḥ kuryān na cāvaśyakam
\dontdisplaylinenum

\nemslokad 
śasye bhasmani govraje dvija jalaṁ kuryān na cārkaṁ naraḥ \veg\dontdisplaylinenum
\varr{
        \ \va devatā\lem  \msCb; deva{\il} \msCa, daivatā \msCc, daivatam \Ed
        \ \vb tiṣṭhetācamanena\lem  \msCb\msCc; {\lost}{\lost}tācamanena \msCa, tiṣṭhenācamanena \Ed\oo
                 vāmāsthinānodade\lem  \msCa\msCb; vāmāsthinānoodare \msCc, vāmāsthi mānādaram \Ed
        \ \vc bandhur āryapurataḥ\lem  \Ed; bandhu āryapurutaḥ \msCa, bandhu āryapurataḥ \msCb\msCc
        \ \vd bhasmani govraje\lem  \msCa\msCc\Ed; \uncl{bhasmani govra}je \msCb}

\ujvers\nemsloka 
pādenāgnijalaṁ sp\textsubring{r}śen na ca guruṁ pādena pādaṁ tathā
\dontdisplaylinenum

\nemslokab 
śaucaṁ kārya jalādinā ca niyataṁ nādho jalaṁ kārayet \danda\dontdisplaylinenum

\nemslokac 
kuryān nityabhivādanaṁ dvijaguror mātāpit\textsubring{r} devatām
\dontdisplaylinenum

\nemslokad 
etācāravidhiḥ samāsaniyamas tubhyaṁ mayā kīrtitam \veg\dontdisplaylinenum


\alalfejezet{striyaḥ}
\vers
\varr{
        \ \va °gnijalaṁ\lem  \msCa\msCc\Ed; °gniṁ jalaṁ \msCb
        \ \vb jalaṁ kārayet\lem  \msCb\msCc\Ed; {\lost}{\lost}{\lost}{\lost}{\lost} \msCa\oo
                 niyataṁ\lem  \msCa\msCb\Ed; niniyataṁ \msCc
        \ \vc °vādanaṁ\lem  \msCa\msCc\Ed; °vādaṁ na \msCb\oo
                 °pit\textsubring{r}devatām\lem  \mssCaCbCc; °pitr\textsubring{r} ddaivatam \Ed
        \ \vd samāsa°\lem  \msCa\msCc\Ed; samā° \msCb}

janamejaya uvāca~{\dandab}\dontdisplaylinenum 

\nemsloka 
strīṇāṁ kiṁ priyam asti tad vada vibho saṁsārasārastriyām
\dontdisplaylinenum

\nemslokab 
kiṁ sadbhāva na vedmi tasya viṣaye kiṁ dveṣya kiṁ tātpriyam \danda\dontdisplaylinenum

\nemslokac 
paśyāmi na ca tasya garbhakalayā prāpnoti niḥsaṁśayam
\dontdisplaylinenum

\nemslokad 
māyājālasahasragāpi yuvatī kurvanti bhartā priyam \veg\dontdisplaylinenum

\vers

vaiśampāyana uvāca~{\dandab}\dontdisplaylinenum 

\nemsloka 
rājan kiṁ priyam asti arthaparataḥ paśyāmi nānyan n\textsubring{r}pe
\dontdisplaylinenum

\nemslokab 
putrārthaikaprayojanaṁ yuvatayaḥ svāyambhuvoktāmaraiḥ \danda\dontdisplaylinenum

\nemslokac 
kāntā nityakalā pravartanakarī dharmasakhāyā satī
\dontdisplaylinenum

\nemslokad 
māyā vāpi karoti sadya manujātyaktānya vā sevate \veg\dontdisplaylinenum

\ujvers\nemsloka 
strīsaṅgaṁ parivarjayen narapate āyāsadaṁ duḥkhadam
\dontdisplaylinenum

\nemslokab 
m\textsubring{r}tyudvārabhayākaraṁ viṣag\textsubring{r}ham āpat sughorālayam \danda\dontdisplaylinenum

\nemslokac 
agnir mārutamattavāraṇasama tasyānugāmī sadā
\dontdisplaylinenum

\nemslokad 
strīhetor hatarāvaṇas tridaśapa indro 'pi visthāpitaḥ \veg\dontdisplaylinenum

\ujvers\nemsloka 
strīhetor api candramāstribhuvane dhiktāṁ gataś cāmaro
\dontdisplaylinenum

\nemslokad 
daṇḍakṣo hatarāṣṭrapaurasahitaḥ kiṁ bhūya vakṣyāmy aham \veg\dontdisplaylinenum


\alalfejezet{vipra-muni-bhikṣu-nirgranthi-parivrājaka-rṣyādayaḥ}
\vers

janamejaya uvāca~{\dandab}\dontdisplaylinenum 

\nemsloka 
vipre kīd\textsubring{r}śalakṣaṇaṁ bhavati bho kīd\textsubring{r}g muniś cocyate
\dontdisplaylinenum

\nemslokab 
tenārthena bhaveta bhikṣu bhagavan nigranthi ko vā dvija \danda\dontdisplaylinenum

\nemslokac 
kenārthena bhaved dvijendra bhagavan jñeyaḥ parivrājakaḥ
\dontdisplaylinenum

\nemslokad 
! jñeyāḥ kim \textsubring{r}ṣayaś ca lakṣaṇa muner icchāmi jñātuṁ punaḥ \veg\dontdisplaylinenum

\vers

vaiśampāyana uvāca~{\dandab}\dontdisplaylinenum 

\nemsloka 
satyaṁ śaucam ahiṁsatā damaśamau bhūtānukampī sadā
\dontdisplaylinenum

\nemslokab 
ātmārāmajito svadharmanirataḥ sattvastha nityaṁ manaḥ \danda\dontdisplaylinenum

\nemslokac 
kāmakrodhayamasvadāranirataḥ saṁtyajya lobhaḥ śanaiḥ
\dontdisplaylinenum

\nemslokad 
evaṁ yaḥ kurute dvijātisuvaraḥ śūdro 'pi yaḥ saṁyamī \veg\dontdisplaylinenum

\ujvers\nemsloka 
tasmāc chadmakavarjitaḥ sa bhagavān saṁsārabhībhidyakaḥ
\dontdisplaylinenum

\nemslokab 
yat tat sthānaparaṁ vrajanti puruṣāḥ tasmāt parivrājakaḥ \danda\dontdisplaylinenum

\nemslokac 
granthidārasutaṁ dhanaṁś ca virati nirgranthika socyate
\dontdisplaylinenum

\nemslokad 
ramyante \textsubring{r}ṣir āśrame dh\textsubring{r}timanas tasmād \textsubring{r}ṣiḥ socyate \veg\dontdisplaylinenum

\ujvers\nemsloka 
kāyavāṅmanadaṇḍatatparataras te daṇḍikarūcyate
\dontdisplaylinenum

\nemslokab 
saddharmaśravaṇaṁ vadanti śravaṇaḥ saddharmabrahmākṣaraḥ \danda\dontdisplaylinenum

\nemslokac 
pāśaprakṣipataṁ paśutvasakalaṁ pāśūpatās te sm\textsubring{r}tāḥ
\dontdisplaylinenum

\nemslokad 
vipre pāśupatādibhikṣusakalaṁ p\textsubring{r}ṣṭo 'smy ahaṁ lakṣaṇam \veg\dontdisplaylinenum

\ujvers\nemsloka 
sarvaṁ tat kathito 'si lakṣaṇa mayā sandhiśvanirnāśanam
\dontdisplaylinenum

\nemslokab 
prajñāsaṁgrahaśītavardhanaparaṁ saṁsāranirmūlanam \danda\dontdisplaylinenum

\nemslokac 
 
\dontdisplaylinenum

\nemslokad 
etaj jñānaparaṁ prabodham atulaṁ nityaṁ śivaṁ dhāryate \veg\dontdisplaylinenum

\vers


\jump
\begin{center}
\ketdanda iti v\textsubring{r}ṣasārasaṁgrahe dvāviṁśatitamo 'dhyāyaḥ\ketdanda
\end{center}
\dontdisplaylinenum\vers 
\bekveg\szamveg\vfill\phpspagebreak\szam\bek\versno=0\fejno=23
\thispagestyle{empty}



\alfejezet{\textbf{23 nidrotpattiḥ}}\jump\jump
\vers

janamejaya uvāca~{\dandab}\dontdisplaylinenum 

devānāṁ dānavānāṁ ca uttarāraṇim eva ca\thinspace{\danda} \dontdisplaylinenum

vidviṣanti ca te 'nyonyaṁ kāraṇaṁ tasya kīrtaya \veg\dontdisplaylinenum
\varr{
        \ \vab dānavānāṁ ca uttarāraṇim eva\lem  \msNa\Ed;
                               dā{\lost}{\lost}{\lost}{\lost}{\lost}{\lost}{\lost}ṇim eva \msCa
        \ \vd tasya\lem  \msCa\Ed; ta\uncl{sya} \msNa}

vaiśampāyana uvāca~{\dandab}\dontdisplaylinenum 

pāpapuṇyasvabhāvābhyāṁ devadaityasya bhūpate\thinspace{\danda} \dontdisplaylinenum

dharmapakṣasm\textsubring{r}to devo dānavo 'dharmapakṣataḥ \veg\dontdisplaylinenum
\varr{
        \ \vc dharmapakṣa°\lem  \msNa; dharme pakṣaḥ \msCa, 
                                         dharmapakṣaḥ \Ed\oo
                 devo\lem  \msCa\msNa; devā \Ed
        \ \vd 'dharma°\lem  \Ed; darppa° \msCa, darpa° \msNa}

hetunā tena rājendra anyonyaṁ vidviṣanti te\thinspace{\dandab} \dontdisplaylinenum

devadveṣṭāsurāḥ sarve vibudhāś cāsuradviṣaḥ \veg\dontdisplaylinenum


\alalfejezet{dharmādharmavipakṣatā}\varr{
        \ \vc devadveṣṭāsurāḥ sarve\lem  \eme;
                      devadveṣṭāsuraḥ sarve \msNa\Ed,
                       \uncl{de}vadve\uncl{ṣṭā}suras {\lost}{\lost} \msCa
        \ \vd vibudhāś\lem  \msNa\Ed; {\lost}{\lost}dhāś \msCa}

\ujvers\nemsloka 
dharmādharmavipakṣatāṁ ś\textsubring{r}ṇu parāṁ bhūtānukampodayām
\dontdisplaylinenum

\nemslokab 
satyaṁ śaucam ahiṁsatā damaśamo nirmānam īrṣyāruṣā \danda\dontdisplaylinenum

\nemslokac 
t\textsubring{r}ṣṇālobharatasya kāmaviṣayaḥ sarvendriyāṇāṁ jayaḥ
\dontdisplaylinenum

\nemslokad 
ādhyātmeṣu ratiḥ prasannamanaso nirdvandvasarvālayaḥ \veg\dontdisplaylinenum
\varr{
        \ \va °vipakṣatāṁ\lem  \Ed; °vivakṣatāṁ \msCa\msNa\oo
                 °kampodayām\lem  \msCa\msNa; °kampādayām \Ed
        \ \vb īrṣā°\lem  \msCa\msNa; īrṣyā° \Ed
        \ \vd prasannamanaso nirdvandva°\lem  \msNa\Ed; 
                        prasanna{\lost}{\lost}{\lost}{\lost}{\lost} \msCa}

\ujvers\nemsloka 
pāpopekṣaṇaśaśvapuṇyamudito dīneṣu kāruṇyatā
\dontdisplaylinenum

\nemslokab 
dānaṁ śīladh\textsubring{r}tikṣamājapatapaḥ svādhyāyamaune ratiḥ \danda\dontdisplaylinenum

\nemslokac 
yogābhyāsaratir divaukasagaṇe jñāne ca sāṁkhye tathā
\dontdisplaylinenum

\nemslokad 
akrodhārjavatejayajñam abhayaṁ saṁtoṣa adrohatā \veg\dontdisplaylinenum
\varr{
        \ \va pāpo°\lem  \msCa\msNa; pāpā° \Ed\oo
                °śaśva°\lem  \msCa\msNa; °śaśca° \Ed
        \ \vc yogābhyāsaratir divaukasa°\lem  \msCa; 
                       yogābhyāsaratidivaukasa° \msNa\ \unmetr, 
                 yogabhyāsaratidivaikasa° \Ed\ \unmetr
        \ \vd °bhayaṁ\lem  \Ed; °bhayas \msCa, °bhayaḥ \msNa}

\ujvers\nemsloka 
tyāgo mārdavahrīr acāpalaratir nyāsābhimāno grahāt
\dontdisplaylinenum

\nemslokab 
maitrībhāvasadārapaiśunamatir brāhmaṇyaśraddhānvitaḥ \danda\dontdisplaylinenum

\nemslokac 
etācāra sadā narendra vibudhāḥ saṁkṣepataḥ kīrtitāḥ
\dontdisplaylinenum

\nemslokad 
daityānāṁ ś\textsubring{r}ṇu kīrtaye svavahito 'sambhāvya teṣāṁ nijam \veg\dontdisplaylinenum
\varr{
        \ \va °hrīr acāpalaratinyāsā°\lem  \msNa\Ed;
                        \uncl{hrī}{\lost}{\lost}{\lost}{\lost}ratir nyāsā° \msCa
        \ \vb °nvitaḥ\lem  \msNa; °nvitā \msCa\Ed
        \ \vc kīrtitāḥ\lem  \msCa\msNa; kīrtitaḥ \Ed
        \ \vd daityānāṁ\lem  \msNa\Ed; daityānā \msCa\oo
                 kīrtaye\lem  \msCa\Ed; kīrtaya \msNa\oo
                 svavahito\lem  \msCa; svavahisaṁ \msNa,
                                         tv avahito \Ed\oo
                 nijam\lem  \msCa\Ed; nijaḥ \msNa}

\ujvers\nemsloka 
daityāḥ pāparatisvabhāvacapalā nirlajjadarpālasāḥ
\dontdisplaylinenum

\nemslokab 
kāmakrodhavaśāḥ suduṣṭamanasas t\textsubring{r}ṣṇādhikā nirdayāḥ \danda\dontdisplaylinenum

\nemslokac 
śaucācāravivarjitā gurugirānnānitya kuryuḥ kriyāḥ
\dontdisplaylinenum

\nemslokad 
jīvākarṣaṇajīvanaḥ pratidinaṁ mohāndharāgānvitāḥ \veg\dontdisplaylinenum
\varr{
        \ \va daityāḥ\lem  \msCa; daityā \msNa\Ed
        \ \vb kāmakrodhavaśāḥ\lem  \msNa\Ed; {\il}{\lost}{\lost}{\lost}{\lost}śās \msCa
        \ \vd jīvākarṣaṇa°\lem  \msCa\msNa; naivākarṣaṇa° \Ed}

\ujvers\nemsloka 
nidrā nitya divā prasaktam aśuciḥ sūryodaye supyate
\dontdisplaylinenum

\nemslokab 
āśāpāśaśatair nibaddhah\textsubring{r}dayo h\textsubring{r}tvā parasvaṁ punaḥ \danda\dontdisplaylinenum

\nemslokac 
mātsaryāt parapākabhedanirato mūlasya duṣpūratā
\dontdisplaylinenum

\nemslokad 
! nāstīkatvaparāṅganāsvabhirata utkocakāmaḥ sadā \veg\dontdisplaylinenum
\varr{
        \ \vb h\textsubring{r}tvā parasvaṁ punaḥ\lem  \msNa\Ed; 
                                \uncl{h\textsubring{r}}{\lost}{\lost}{\lost}{\lost}{\lost}naḥ \msCa
        \ \vc mātsaryā\lem  \msCa\msNa; māṁsaryā° \Ed
        \ \vd °parāṅganāsvabhirata\lem  \msCa;
                                °parāṅganās tv abhirata \msNa,
                                °parāṅganāpy abhirato \Ed\oo
                 utkoca°\lem  \msCa\msNa; uktā ca \Ed}

\ujvers\nemsloka 
devabrāhmaṇa vidviṣanti satataṁ lobhāc ca kāryakriyā
\dontdisplaylinenum

\nemslokab 
dharmaṁ dūṣayate ca mūḍhamanasā āryaṁ ca tīrthaṁ tathā \danda\dontdisplaylinenum

\nemslokac 
hantavyāny ahatāś ca manyabahavo visphūrjitam adruvan
\dontdisplaylinenum

\nemslokad 
daityānāṁ kathitaṁ ca cihna katicit sadbhāvataḥ kīrtitam \veg\dontdisplaylinenum
\varr{
        \ \vc °hatāś\lem  \msCa; °hatāṁś \msNa, °hatāṁ \Ed\oo
                 manya°\lem  \msCa\msNa; yanya \Ed\oo
                 visphūrjitam adruvan\lem  \msNa;
                                        visphurjjite nakravat \Ed, 
                                        vi{\lost}{\lost}{\lost}{\lost}druvan \msCa
        \ \vd kathitaṁ\lem  \msCa\msNa; kathitaś \Ed}

\ujvers\nemsloka 
martyeṣv eva narendra mānuṣam abhūd devāsurāṇāṁ n\textsubring{r}paḥ
\dontdisplaylinenum

\nemslokab 
yo yaṁ proktaḥ svabhāvatām ubhayato mānuṣyaloke tathā \danda\dontdisplaylinenum

\nemslokac 
yan me p\textsubring{r}cchitavān narendra kathitaṁ yat tvaṁ purā gopitam
\dontdisplaylinenum

\nemslokad 
vidveṣobhayakāraṇaṁ narapate kiṁ bhūya vakṣyāmy aham \veg\dontdisplaylinenum


\alalfejezet{nidrottpattiḥ}
\vers
\varr{
        \ \vb °loke\lem  \msCa\msNa; °lokan \Ed
        \ \vc p\textsubring{r}cchitavān\lem  \msNa\Ed; p\textsubring{r}cchitavā \msCa
        \ \vd vidveṣobhayakāraṇaṁ narapate kiṁ\lem  \msNa\Ed; 
                                vi\uncl{dveṣobhaya}{\lost}{\lost}{\lost}{\lost}{\lost}pate ki \msCa}

janamejaya uvāca~{\dandab}\dontdisplaylinenum 

asti kautūhalaṁ cānyaṁ p\textsubring{r}cchāmi tvāṁ dvijottama\thinspace{\danda} \dontdisplaylinenum

kathaṁ nidrā samutpannā sarvabhūtavimohanī \veg\dontdisplaylinenum
\varr{
        \ \va kautūhalaṁ\lem  \msCa\msNa; kautuhalaṁś \Ed
        \ \vd °mohanī\lem  \msCapcorr\msNa\Ed; °mohinī \msCaacorr}

rātrau prajāyate kasmād divā kasmān na jāyate\thinspace{\dandab} \dontdisplaylinenum

kasmāc ca kurute jantor nidrā netrapramīlanam \danda\dontdisplaylinenum

etan me saṁśayaṁ chindhi sarvajño 'si dvijottama \veg\dontdisplaylinenum
\varr{
        \ \vc jantor\lem  \msCa\msNa; janto \Ed
        \ \vf sarvajño 'si\lem  \msNa\Ed; {\lost}{\lost}{\lost}{\lost} \msCa}

vaiśampāyana uvāca~{\dandab}\dontdisplaylinenum 

devī hy eṣā mahābhāgā nidrā netrāśrayā n\textsubring{r}ṇām\thinspace{\danda} \dontdisplaylinenum

tasyā vaśaṁ gataṁ sarvaṁ jagatsthāvarajaṅgamam \veg\dontdisplaylinenum
\varr{
        \ \vb °śrayā\lem  \msCa\msNa; °śrayo \Ed}

sadevadānavagaṇā gandharvoragarākṣasāḥ\thinspace{\dandab} \dontdisplaylinenum

yakṣabhūtapiśācāś ca paśupakṣisarīs\textsubring{r}pāḥ \veg\dontdisplaylinenum
\varr{
        \ \va °dānava°\lem  \msCa\Ed; °dānavā° \msNa
        \ \vd °sarīs\textsubring{r}pāḥ\lem  \msCa\msNa; °śarīs\textsubring{r}paḥ \Ed}

guhyakāś ca m\textsubring{r}gā nāgā kiṁnarā jalajoragāḥ\thinspace{\dandab} \dontdisplaylinenum

nidrāvaśagatāḥ sarve pāpmanā tv abhilaṅghitāḥ \veg\dontdisplaylinenum
\varr{
        \ \va guhyakāś ca\lem  \eme;
                                 guhyakaś ca \Ed,
                                 guhyavastra° \msCa\msNa\oo
                 nāgāḥ\lem  \msCa\msNa; nāgā \Ed
        \ \vb kiṁnarā jalajoragāḥ\lem  \eme;
                        kiṁnarā jalajā nagāḥ \msNa\Ed,
                        kinna{\lost}{\lost}{\lost}{\lost}{\lost}gāḥ \msCa}

devadānavakarmānte tasminn am\textsubring{r}tasambhave\thinspace{\dandab} \dontdisplaylinenum

mandarotthāpane viṣṇur devāsurasamāgame \veg\dontdisplaylinenum
\varr{
        \ \va °karmānte\lem  \msCa\msNa; °karmāt te \Ed
        \ \vb °m\textsubring{r}ta°\lem  \msCa\msNa; °n\textsubring{r}ta° \Ed
        \ \vc °tthāpane\lem  \Ed; °tpādane \msCa\msNa}

jāyate vigrahe tv eṣā k\textsubring{r}te hy am\textsubring{r}tamanthane\thinspace{\dandab} \dontdisplaylinenum

rajas tamaś cāsuraṁ vai sattvaṁ devak\textsubring{r}taiḥ śubhaiḥ \veg\dontdisplaylinenum

tataḥ sattvamayī devī rajastamanivāsinī\thinspace{\dandab} \dontdisplaylinenum

krodhajā vai sthitā madhye devadānavapakṣayoḥ \veg\dontdisplaylinenum
\varr{
        \ \vab sattvamayī devī rajas tamasi vāsinī\lem  \msNa;
                       sattvamayī \uncl{de}{\lost}{\lost}{\lost}{\lost}masi vāsinī \msCa,
                       sattvamayī devī rajas tamanivāsinī \Ed}

tām adbhutamayīṁ d\textsubring{r}ṣṭvā vismitā devadānavāḥ\thinspace{\dandab} \dontdisplaylinenum

tasyāḥ prabhāvābhihatā dudruvas te diśo daśa \veg\dontdisplaylinenum
\varr{
        \ \va °bhuta°\lem  \msCa\msNa; °bhūta° \Ed
        \ \vd daśa\lem  \msCa\msNa; daśaḥ \Ed}

tatra pītāmbaradharo viṣṇur ekas tu tiṣṭhati\thinspace{\dandab} \dontdisplaylinenum

sābhigatvā viśālākṣī nārāyaṇam athābravīt \veg\dontdisplaylinenum
\varr{
        \ \va pītā°\lem  \msCa\msNa; pitā° \Ed
        \ \vc sābhi°\lem  \msCa\msNa; sobhi° \Ed
        \ \vd °bravīt\lem  \msNa\Ed; °\uncl{bra}{\lost} \msCa}

devadānavanāthas tvaṁ tvayi sarvaṁ pratiṣṭhitam\thinspace{\dandab} \dontdisplaylinenum

dehi deva mamāvāsaṁ yatrāhaṁ nivase sukham \veg\dontdisplaylinenum
\varr{
        \ \va deva°\lem  \msNa\Ed; {\lost}{\lost} \msCa
        \ \vb sarvaṁ\lem  \msCa\msNa; sarva° \Ed}

tato nārāyaṇas tuṣṭas tāṁ devīṁ pratyabhāṣata\thinspace{\dandab} \dontdisplaylinenum

śarīre mama vastavyaṁ viṣṇur enām athābravīt \veg\dontdisplaylinenum
\varr{
        \ \vc vastavyaṁ\lem  \Ed; vāstavyam \msCa\msNa}

tatas tāṁ vaiṣṇavaṁ tejaḥ pāpmanā samatiṣṭhata\thinspace{\dandab} \dontdisplaylinenum

tataḥ śete sa vaikuṇṭhaḥ pāpmanā tv abhilaṅghitaḥ \veg\dontdisplaylinenum
\varr{
        \ \va vaiṣṇavaṁ\lem  \msCa\msNa; viṣṇuvat \Ed
        \ \vd pāpmanā tv abhilaṅghitaḥ\lem  \msNa;
                                pāpmanā tv abhilaṅghitāḥ \Ed,
                                pāpma{\lost}{\lost}{\lost}{\lost}ghitaḥ \msCa}

tasmin śayāne vitrastā devāsuragaṇās tathā\thinspace{\dandab} \dontdisplaylinenum

ūcus te paramodvignāḥ śayānaṁ viṣṇum acyutam \veg\dontdisplaylinenum
\varr{
        \ \va tasmin\lem  \msCa\Ed; tasmi \msNa}

trātāraṁ nābhigacchāma uttiṣṭhottiṣṭha keśava\thinspace{\dandab} \dontdisplaylinenum

tataḥ śaṅkhagadāpāṇir uttiṣṭhata mahābhujaḥ \veg\dontdisplaylinenum

utthitaś ca viśālākṣaḥ pāpmanā tasya p\textsubring{r}ṣṭhataḥ\thinspace{\dandab} \dontdisplaylinenum

tataḥ sā vigrahavatī sthitā nārāyaṇālaye \veg\dontdisplaylinenum
\varr{
        \ \va utthita°\lem  \msCa\msNa; uttiṣṭha° \Ed\oo
                 viśālākṣaḥ\lem  \msCa\msNa; viśālākṣiḥ \Ed
        \ \vc tataḥ sā vigrahavatī\lem  \msNa\Ed;
                        tata{\lost}{\lost}{\lost}{\lost}\uncl{va}tī \msCa}

viṣṇur devāsuragaṇān idaṁ vacanam abravīt\thinspace{\dandab} \dontdisplaylinenum

asmākaṁ vai śarīreṣu iyaṁ pāpmā viniḥs\textsubring{r}tā \veg\dontdisplaylinenum
\varr{
        \ \va viṣṇur\lem  \msCa\Ed; viṣṇu \msNa\oo
                °gaṇān\lem  \msCa\Ed; °gaṇā \msNa
        \ \vd viniḥs\textsubring{r}tā\lem  \eme; vinis\textsubring{r}tā \msCa\msNa\Ed\ \unmetr}

eṣābhisattvārasatā satyena bhaginī mama\thinspace{\dandab} \dontdisplaylinenum

viśrutāṁ triṣu lokeṣu tāṁ pūjayatha māṁ yathā \veg\dontdisplaylinenum
\varr{
        \ \va eṣābhisattvārasatā\lem  \msCa;
                        eṣātisatvānasatā \msNa,
                        eṣātisattvāmasatī \Ed
        \ \vc °śrutāṁ\lem  \msCa; °śrutā \msNa, °śruto \Ed}

tato devāsuragaṇāḥ saptalokāḥ samānuṣāḥ\thinspace{\dandab} \dontdisplaylinenum

vibhaktā vaiṣṇavī pāpmā teṣu sarveṣu devatā \veg\dontdisplaylinenum
\varr{
        \ \vb °lokāḥ samānuṣāḥ\lem  \msNa\Ed; °\uncl{lo}{\lost}{\lost}{\lost}nuṣāḥ \msCa}

parvateṣv atha v\textsubring{r}kṣeṣu sāgareṣu saritsu ca\thinspace{\dandab} \dontdisplaylinenum

tato nidrāvaśagataṁ jagat sthāvarajaṅgamam \veg\dontdisplaylinenum

eṣotpattiś ca nidrāyā yathā vasati tac ch\textsubring{r}ṇu\thinspace{\dandab} \dontdisplaylinenum

trīṇi sthānāni yasyā vai śarīreṣu śarīriṇām \veg\dontdisplaylinenum

śleṣmapittānilasthāne trīṇi pakṣāṇi vāsinaḥ\thinspace{\dandab} \dontdisplaylinenum

tamaḥ śleṣmāśrayā nidrā rajonidrā tu vātikā \veg\dontdisplaylinenum
\varr{
        \ \vab °nilasthāne trīṇi\lem  \Ed; °nilasthāna trīṇi \msNa, 
                                        ni{\lost}{\lost}{\lost}{\lost}ṇi \msCa
        \ \vb pakṣāṇi\lem  \msCa; pakṣā ni° \msNa\Ed
        \ \vc tamaḥ\lem  \msCa\msNa; tama° \Ed
        \ \vd nidrā tu\lem  \msCa\msNa; nidrāti° \Ed}

pittāśrayāṁ sm\textsubring{r}tāṁ nidrāṁ sāttvikāṁ viddhi bhūpate\thinspace{\dandab} \dontdisplaylinenum

ādityaprabhavaṁ tejas tasmin sattvaṁ pratiṣṭhati \veg\dontdisplaylinenum
\varr{
        \ \va sm\textsubring{r}tāṁ\lem  \msCa\Ed; sm\textsubring{r}tā \msNa
        \ \vd sattvaṁ pratiṣṭhati\lem  \msCa\msNa; sarva pratiṣṭhitaṁ \Ed}

nidrā divā na bhavati tasmāt sattvaguṇātmikā\thinspace{\dandab} \dontdisplaylinenum

yasmāt somodbhavā nidrā tamāṁsi ca rajāṁsi ca \veg\dontdisplaylinenum
\varr{
        \ \vc yasmā°\lem  \msCa\msNa; tasmā° \Ed
        \ \vd tamāṁsi ca rajāṁsi ca\lem  \msNa\Ed;
                        ta\uncl{māṁ}si ca \uncl{ra}{\lost}{\lost}{\lost} \msCa}

tasmād rātrau bhaven nidrā tāmasī harajātmikā\thinspace{\dandab} \dontdisplaylinenum

yadā hi sarvāṅgagatau śrotāṁsi pratipadyate \veg\dontdisplaylinenum
\varr{
        \ \va bhaven\lem  \msCa\Ed; bhavan \msNa
        \ \vc sarvā°\lem  \msNa\Ed; satvā° \msCa}

rajas tamaś ca niyatas tadā nidrā pravartate\thinspace{\dandab} \dontdisplaylinenum

tamasy ūrdhvagataśroto hy akṣipakṣmāsamāśritā \veg\dontdisplaylinenum
\varr{
        \ \va niyatas\lem  \msCa\msNa; niyataṁs \Ed
        \ \vc °gataśroto\lem  \msCa\msNa; °gate śrotro \Ed
        \ \vd hy akṣi°\lem  \msCa\msNa; hy ākṣi° \Ed}

tamaḥ pravartate jantos tatas tv akṣnor nimīlanam\thinspace{\dandab} \dontdisplaylinenum

nāsākṣikarṇaśrotāṁsi prayujyante kaphena tu \veg\dontdisplaylinenum
\varr{
        \ \vab jantos tata°\lem  \msCa\msNa; janto tama° \Ed
        \ \vb tv akṣṇor ni°\lem  \msCa; tv akṣṇo ni° \msNa\Ed
        \ \vc °śrotāṁsi\lem  \msNa\Ed; śro\uncl{tā}{\lost} \msCa
        \ \vd prayujyante kaphena\lem  \msNa\Ed; {\lost}{\lost}{\lost}{\lost}{\lost}phena \msCa}

h\textsubring{r}dayaṁ muhyate cāpi tamasā cāv\textsubring{r}taṁ manaḥ\thinspace{\dandab} \dontdisplaylinenum

sparśaṁ na vedayaty eva na ś\textsubring{r}ṇoti na paśyati \veg\dontdisplaylinenum

nocchvāsayati nāsābhyāṁ viv\textsubring{r}tākṣimukho naraḥ\thinspace{\dandab} \dontdisplaylinenum

eṣā n\textsubring{r}ṇām antakarī nidrā vai tāmasī sm\textsubring{r}tā \veg\dontdisplaylinenum
\varr{
        \ \vb °mukho naraḥ\lem  \msCa\msNa; °mukhena ca \Ed
        \ \vc °ntakarī\lem  \msNa\Ed; °nakarī \msCa}

akarmaṇy aprav\textsubring{r}ttiś ca m\textsubring{r}tavat svapate kṣitau\thinspace{\dandab} \dontdisplaylinenum

nidrotpattiṁ vikāraṁ ca kathito 'smi narādhipa \danda\dontdisplaylinenum

tasmān nidrāṁ na seveta tamomohapravardhanīm \veg\dontdisplaylinenum


\jump
\begin{center}
\ketdanda iti v\textsubring{r}ṣasārasaṁgrahe nidrotpattis trayoviṁśatimo 'dhyāyaḥ\ketdanda
\end{center}
\dontdisplaylinenum\vers 
\varr{
        \ \va °tpattiṁ vikāraṁ ca\lem  \msNa;
                      °tpattiṁ vikāraś ca \Ed, 
                      °tpa\uncl{tti}{\lost}{\lost}{\lost}{\lost} \msCa
        \ \vd °vardhanīm\lem  \msCa; °vardhanī \msNa\Ed
        \ {\normalfont \Colo: }  °viṁśatimo\lem  \msCa\msNa; °viṁśatitamo \Ed}
\bekveg\szamveg\vfill\phpspagebreak\szam\bek\versno=0\fejno=24
\thispagestyle{empty}



\alfejezet{\textbf{24 śāstravarṇanam}}\jump\jump
\vers

janamejaya uvāca~{\dandab}\dontdisplaylinenum 

devānāṁ dānavānāṁ ca vaiṣamyāni śrutāni me\thinspace{\danda} \dontdisplaylinenum

nidrāsambhavam āścaryaṁ tvatprasādena veditam \veg\dontdisplaylinenum
\varr{
        \ \vb vaiṣamyāni\lem  \eme;  vaiśamyāni \msCa\msCb\Ed\oo
                 me\lem  \msCa\msCb; vai \Ed
        \ \vd tvatprasādena veditam\lem  \msCb\Ed; tvatpra{\lost}{\lost}{\lost}{\lost}{\lost}{\lost}tam \msCa}

trailokyavistarāyāmaṁ śrotum icchāmi bho dvija\thinspace{\dandab} \dontdisplaylinenum

kasmiṁścin narakaṁ jñeyaṁ pātālaṁ ca dvijottama \veg\dontdisplaylinenum
\varr{
        \ \va °lokya°\lem  \msCa\msCb; °lokyā° \Ed
        \ \vb bho\lem  \msCa\msCb; vai \Ed
        \ \vc kasmiṁścin narakaṁ\lem  \eme; kasmiṁścin narake \msCa\msCb, kasmiścin narakaṁ \Ed}

saptadvīpaṁ samicchāmi saptasāgaram eva ca\thinspace{\dandab} \dontdisplaylinenum

merumūrdhaṁ ca viprendra devālayaṁ nibodha mām \veg\dontdisplaylinenum


\alalfejezet{trailokyaṁ narakāṇi ca}\varr{
        \ \vc °mūrdhaṁ\lem  \msCa\msCb; °mūrdhaś \Ed
        \ \vd devālayaṁ\lem  \corr; devālaya \msCa\msCb\Ed}

vaiśampāyana uvāca~{\dandab}\dontdisplaylinenum 

ś\textsubring{r}ṇu saṁkṣepato rājan trailokyāyāmavistaram\thinspace{\danda} \dontdisplaylinenum

kālāgniḥ prathamo jñeyaḥ sarvādhastān nareśvara \veg\dontdisplaylinenum
\varr{
        \ \vb °vistaram\lem  \msCb\Ed; {\lost}{\lost}{\lost} \msCa}

tasyopari n\textsubring{r}paśreṣṭha jñeyā narakakoṭayaḥ\thinspace{\dandab} \dontdisplaylinenum

rauravādi avīcyantaṁ yātanāsthānam ucyate \veg\dontdisplaylinenum


\alalfejezet{sapta pātālāḥ}\varr{
        \ \va n\textsubring{r}pa°\lem  \msCa\Ed; n\textsubring{r}° \msCb}

upariṣṭāt tu vijñeyāḥ pātālāḥ sapta eva tu\thinspace{\dandab} \dontdisplaylinenum
            \paral{\textit{{\normalfont Niśv Kārikā 149:} upariṣṭāt tu deveśi pātālās sapta eva tu}}

ābhāsatālaḥ prathamaḥ svatālaś ca tataḥ param \veg\dontdisplaylinenum
\varr{
        \ \vd svatālaś ca\lem  \Ed; svalālañ ca \msCaacorr, svatālañ ca \msCapcorr, 
                                        sutālañ ca \msCb}

śītalaś ca gabhastiś ca śarkaraś ca śilātalam\thinspace{\dandab} \dontdisplaylinenum

saptamaṁ tu mahātālaṁ śeṣanāgak\textsubring{r}tālayaḥ \veg\dontdisplaylinenum
\varr{
        \ \va śītalaś ca\lem  \msCa\Ed; śrītalaś ca \msCb
        \ \vb śarkaraś ca śilātalam\lem  \eme; {\lost}{\lost}{\lost}{\lost}{\lost}lātalam \msCa,
                                         śilātalam \msCb,
                                         śarkaraś ca śilāv\textsubring{r}tam \Ed
        \ \vc saptamaṁ\lem  \msCa\msCb; saptamas \Ed
        \ \vd °layaḥ\lem  \msCa\Ed; °layam \msCb}

baliś ca daityarājendro rākṣasaś ca viśaṁkhaṇaḥ\thinspace{\dandab} \dontdisplaylinenum

ity evam ādayaḥ sarve nāgadānavarākṣasāḥ \veg\dontdisplaylinenum


\alalfejezet{sapta dvīpāḥ priyavratasutāś ca}\varr{
        \ \vb viśaṁkhaṇaḥ\lem  \Ed; visaṁśanaḥ \msCa, visaṁśayaḥ \msCb}

sapta dvīpās tato jñeyāḥ saptasāgarasaṁv\textsubring{r}tāḥ\thinspace{\dandab} \dontdisplaylinenum

priyavratasya putro 'bhūd daśa rājaparākramaḥ \veg\dontdisplaylinenum 
            \paral{\textit{{\normalfont For a similar enumeration of Priyavrata's ten sons and the seven islands,
                see, e.g., Vāyupurāṇa 33.1 ff.}}}
\varr{
        \ \vo (sapta{\normalfont ...}°parākramaḥ)\lem  \msCa\msCb; \om\ \Ed}

agnīdhraś cāgnibāhuś ca medhā medhātithir vasuḥ\thinspace{\dandab} \dontdisplaylinenum

jyotiṣmān dyutimān havyaḥ savanaḥ patra eva ca \veg\dontdisplaylinenum
            \paral{\textit{\vo {\normalfont \kb\ Brahmapurāṇa 5.9:}
                āgnīdhraś cāgnibāhuś ca medhyo medhātithir vasuḥ{\thinspace\danda}
                jyotiṣmān dyutimān havyaḥ savalaḥ putrasaṁjñakaḥ{\thinspace\ketdanda}
                {\normalfont \kb\ Brahmāṇḍapurāṇa 1.13.104 and 1.14.9 
                \kb\ Padmapurāṇa 1.7.83 etc.}}}
\varr{
        \ \vab agnīdhraś cāgnibāhuś ca medhā medhātithir vasuḥ\lem  \corr;
                 agnīndhraś cāgnibāhuś ca medhā medhātithir vasuḥ \msCb,
                                agninvraścāgnivā{\il}{\lost} {\lost}{\lost}{\lost}dhātithir vvasuḥ \msCa, \om\ \Ed
        \ \vcd havyaḥ savanaḥ patra eva ca\lem  \msCa\Ed; 
                                \uncl{havyaḥ savanaḥ patra eva ca} \msCb}

agnibāhuś ca medhā ca patraś caiva trayo janāḥ\thinspace{\dandab} \dontdisplaylinenum

saṁsārabhayabhītena mokṣamārgasamāśritāḥ \veg\dontdisplaylinenum
\varr{
        \ \va medhā ca\lem  \msCa\Ed; medhāś ca \msCb
        \ \vd °mārga°\lem  \msCb\Ed; °mārgaṁ \msCa}

agnīdhraṁ prathamadvīpe abhyaṣiñcat priyavrataḥ\thinspace{\dandab} \dontdisplaylinenum

plakṣadvīpeśvaraṁ cakre nāmnā medhātithiṁ tathā \veg\dontdisplaylinenum
\varr{
        \ \va agnīdhraṁ\lem  \eme; agnindhraṁ \msCa, agnīndhra \msCb, agnindhaṁ \Ed\oo
                 prathama°\lem  \Ed; prathamaṁ \msCa\msCb
        \ \vb abhyaṣiñcat\lem  \msCa\msCb; abhyaṣiñcata \Ed
        \ \vd medhātithiṁ tathā\lem  \msCa\Ed; medhātithitan tathā \msCb}

vasuś ca śālmalīdvīpe abhiṣikto mahīpatiḥ\thinspace{\dandab} \dontdisplaylinenum

jyotiṣmantaṁ kuśadvīpe rājānam abhiṣecayet \veg\dontdisplaylinenum
\varr{
        \ \va vasuś ca śālmalī°\lem  \msCb; {\lost}{\lost}{\lost}{\lost}{\lost}{\lost} \msCa, vasuñ ca śālmalī \Ed}

krauñcadvīpeśvaraṁ cakre dyutimantaṁ nareśvara\thinspace{\dandab} \dontdisplaylinenum

śākadvīpeśvaraṁ havyaṁ puṣkare savanaḥ sm\textsubring{r}taḥ \veg\dontdisplaylinenum
\varr{
        \ \vb dyutimantaṁ nareśvara\lem  \msCa; dyutimantan nareśvaram \msCbpcorr, 
                        śvarañ cakre dyutimantan nareśvaram \msCbacorr,
                        dyutimantaṁ nareśvaraḥ \Ed
        \ \vd savanaḥ\lem  \msCa\msCb; savana \Ed}

madhye puṣkaradvīpasya parvato mānasottaraḥ\thinspace{\dandab} \dontdisplaylinenum

lokapālāḥ sthitās tatra caturbhiś caturo diśaḥ \veg\dontdisplaylinenum
\varr{
        \ \vd caturo diśaḥ\lem  \msCb\Ed; {\lost}{\lost}{\lost}{\lost}{\lost} \msCa}

mahāvītaḥ sm\textsubring{r}to varṣo dhātakī ca narādhipa\thinspace{\dandab} \dontdisplaylinenum

tasya bāhyaḥ samudro 'bhūt svādūdaka iti sm\textsubring{r}taḥ \veg\dontdisplaylinenum
\varr{
        \ \vo (mahāvītaḥ{\normalfont ...}sm\textsubring{r}taḥ)\lem  \msCa\msCb; \om\ \Ed
        \ \va mahāvītaḥ\lem  \msCa\msCb; mahānītaḥ \Ed\oo
                 sm\textsubring{r}to\lem  \msCa\Ed; sm\textsubring{r}tā \msCb
        \ \vc bāhyaḥ\lem  \msCa\Ed; bāhya \msCb
        \ \vd °dūdaka\lem  \msCa\Ed; °dūka \msCb}

catuḥṣaṣṭi sm\textsubring{r}to lakṣo yojanānāṁ narādhipa\thinspace{\dandab} \dontdisplaylinenum

puṣkaradvīpam antaś ca kṣīrodo nāma sāgaraḥ \veg\dontdisplaylinenum
\varr{
        \ \va catuḥ°\lem  \msCb; catu° \msCa\oo       
                 lakṣo\lem  \msCa; lakṣā \msCb, \om\ \Ed
        \ \vb narādhipa\lem  \msCa; narādhipaḥ \msCb, \om\ \Ed}

dvātriṁśallakṣavistāraḥ śākadvīpabahirv\textsubring{r}taḥ\thinspace{\dandab} \dontdisplaylinenum

jaladaś ca kumāraś ca sukumāramaṇīcakaḥ \veg\dontdisplaylinenum
\varr{
        \ \va °vistāraḥ\lem  \msCa\msCb; °vistāraiḥ \Ed
        \ \vb °bahirv\textsubring{r}taḥ\lem  \conj; °vahav\textsubring{r}ṇaḥ \msCa, °bahuv\textsubring{r}taḥ \msCb, °vahav\textsubring{r}ṇe \Ed
        \ \vcd kumāraś ca sukumāramaṇīcakaḥ\lem  \msCb\Ed;
                                           kumā{\il}{\lost}{\lost}{\lost}{\lost}{\il}{\il}ṇīcakaḥ \msCa}

kusumottaramodaś ca saptamaṁ ca mahādrumam\thinspace{\dandab} \dontdisplaylinenum

havyaputrāḥ sm\textsubring{r}tāḥ sapta varṣanāma tathā sm\textsubring{r}taḥ \veg\dontdisplaylinenum
\varr{
        \ \vb saptamaṁ\lem  \msCa\msCb; saptamaś \Ed}

dvīpāntaṁ dadhimaṇḍodakṣīrodārdhaṁ vinirdiśet\thinspace{\dandab} \dontdisplaylinenum

krauñcadvīpasamudrānte sapta varṣās tu te sm\textsubring{r}tāḥ \veg\dontdisplaylinenum
\varr{
        \ \va °maṇḍoda°\lem  \msCb; °maṇḍādi° \msCa\Ed
        \ \vb vinirdiśet\lem  \msCa\msCb; nirdiśet \Ed
        \ \vc °dvīpa°\lem  \msCa\msCb; °dvīpe \Ed
        \ \vd varṣās\lem  \msCa\msCb; varṣan \Ed}

kuśalo manonugaś coṣṇaḥ yāvanaś cāndhakārakaḥ\thinspace{\dandab} \dontdisplaylinenum

muniś ca dundubhiś caiva sutā dyutimatas tu vai \veg\dontdisplaylinenum
\varr{
        \ \va kuśalo manonugaś coṣṇaḥ\lem  \msCb;
                                kuśalo manonugaś co\uncl{ṣṇaḥ} \msCa,
                                kuśalomnonugaś coṣṇaḥ \Ed
        \ \vb yāvanaś cāndhakārakaḥ\lem  \msCb;
                                \uncl{yā}vana\uncl{ś cā}{\il}{\lost}{\lost}{\lost} \msCa,
                                yavanaś cāndhakārakaḥ \Ed
        \ \vd sutā dyutimatas\lem  \msCa\msCb; sutadyutimanas \Ed}

dadhyardhe gh\textsubring{r}tamaṇḍodaḥ kuśadvīpasamāv\textsubring{r}taḥ\thinspace{\dandab} \dontdisplaylinenum

tatrāpi saptavarṣe ca nāmataḥ ś\textsubring{r}ṇu bhārata \veg\dontdisplaylinenum
\varr{
        \ \va gh\textsubring{r}ta°\lem  \msCa\msCb; dh\textsubring{r}ta° \Ed
        \ \vb °dvīpa°\lem  \msCa\msCb; °dvīpaḥ \Ed
        \ \vc °varṣe\lem  \msCa\msCb; °varṣaṁ \Ed
        \ \vd bhārata\lem  \msCa\Ed; bhārataḥ \msCb}

udbhimān veṇumāṁś caiva svairannālambano dh\textsubring{r}tiḥ\thinspace{\dandab} \dontdisplaylinenum

ṣaṣṭhaḥ prabhākaraś caiva kapilaḥ saptamaḥ sm\textsubring{r}taḥ \veg\dontdisplaylinenum
            \paral{\textit{{\normalfont Cf. Brahmapurāṇa 20.36--37ab: }
                 jyotiṣmataḥ kuśadvīpe ś\textsubring{r}ṇudhvaṁ tasya putrakān{\thinspace\danda}
                 udbhido veṇumāṁś caiva svairatho randhano dh\textsubring{r}tiḥ{\thinspace\ketdanda}
                 prabhākaro 'tha kapilas tannāmnā varṣapaddhatiḥ{\thinspace\danda}}}
\varr{
        \ \va veṇumāṁś caiva\lem  \msCa; veṇumāṁ va \msCb, dhenusāś caiva \Ed
        \ \vb svaira°\lem  \msCa; svairā° \Ed}

gh\textsubring{r}tamaṇḍas tadardhena tasyānte madirodadhiḥ\thinspace{\dandab} \dontdisplaylinenum

samantāc chālmalīdvīpo varṣāḥ saptaiva kīrtitāḥ \veg\dontdisplaylinenum
\varr{
        \ \va °maṇḍas tadardhena\lem  \msCb; maṇḍotadardhena \msCa, maṇḍotardhena \Ed
        \ \vb tasyānte madiro°\lem  \Ed;
                        \uncl{ta}{\lost}{\lost}{\lost}diro° \msCa, tasyāntemadhiro° \msCb
        \ \vd varṣāḥ\lem  \msCb\Ed; varṣoḥ \msCa}

śvetaś ca haritaś caiva jīmūto rohitas tathā\thinspace{\dandab} \dontdisplaylinenum

vaidyuto mānasaś caiva suprabhaḥ saptamaḥ sm\textsubring{r}taḥ \veg\dontdisplaylinenum
\varr{
        \ \vb rohita°\lem  \msCa\msCb; lohita° \Ed}

madirodadhito 'rdhena jñeyas tv ikṣurasodadhiḥ\thinspace{\dandab} \dontdisplaylinenum

plakṣadvīpo v\textsubring{r}tas tena saptavarṣasamanvitaḥ \veg\dontdisplaylinenum
\varr{
        \ \va °dadhito\lem  \msCa\msCb; °dadhino \Ed
        \ \vb jñeyas tv i°\lem  \msCa\msCb; jñeya tv i° \Ed}

śāntaś ca śiśiraś caiva sukhadānanda eva ca\thinspace{\dandab} \dontdisplaylinenum

śivakṣemo dhruvaś caiva sapta medhātitheḥ sutāḥ \veg\dontdisplaylinenum
\varr{
        \ \va śāntaś ca śiśiraś\lem  \msCb\Ed; {\lost}{\lost}{\lost}{\lost}{\lost}raś \msCa
        \ \vc śiva°\lem  \msCa; śivaśiva° \Ed}

lavaṇodas tu tasyānte jambūdvīpasamāv\textsubring{r}taḥ\thinspace{\dandab} \dontdisplaylinenum

lakṣayojanavistāra upadvīpasamanvitaḥ \veg\dontdisplaylinenum
\varr{
        \ \va °das tu tasyānte\lem  \msCa\msCb; °dadhisyānte \Ed
        \ \vb jambū°\lem  \msCa\Ed; ja\uncl{mbu}° \msCb\oo 
                 °dvīpa°\lem  \msCa\msCb; °dvīpā° \Ed\oo
                 °v\textsubring{r}taḥ\lem  \msCa\Ed; °v\textsubring{r}tāḥ \msCb
        \ \vc °vistāra\lem  \msCa\msCb; °vistāro \Ed
        \ \vd °dvīpa°\lem  \msCa\msCb; °dvipa° \Ed}

aṅgadvīpo yavadvīpo malayadvīpa eva ca\thinspace{\dandab} \dontdisplaylinenum

śaṅkhadvīpakamudvīpo varāhadvīpa eva ca \veg\dontdisplaylinenum
\varr{
        \ \vo (aṅgadvīpo{\normalfont ...}eva ca)\lem  \msCa\Ed; \om\ \eyeskip{to 24.32cd} \msCb
        \ \vd eva ca\lem  \Ed; {\lost}{\lost} \msCa, \om\ \msCb}

siṁha barhiṇadvīpaṁ ca padmaś cakras tathaiva ca\thinspace{\dandab} \dontdisplaylinenum

vajraratnākaradvīpo haṁsakaḥ kumudas tathā \veg\dontdisplaylinenum
\varr{
        \ \vo (siṁha{\normalfont ...}tathā)\lem  \msCa\Ed; \om\ \msCb
        \ \va siṁha barhiṇa°\lem  \Ed; {\lost}{\lost}{\lost}rhiṇa° \msCa, \om\ \msCb
        \ \vb padmaś cakra\lem  \msCa; \om\ \msCb, padmacakra° \Ed}

lāṅgalo v\textsubring{r}ṣadvīpaś ca dvīpo bhadrākaras tathā\thinspace{\dandab} \dontdisplaylinenum

candradvīpaś ca sindhuś ca candanadvīpa eva ca \danda\dontdisplaylinenum

upadvīpasahasrāṇi evamādīni kīrtitam \veg\dontdisplaylinenum


\alalfejezet{agnīdhraputrā jambudvīpe}\varr{
        \ \vo (lāṅgalo{\normalfont ...}kīrtitam)\lem  \msCa\Ed; \om\ \msCb
        \ \vd candana°\lem  \msCa; \om\ \msCb, nandana° \Ed
        \ \vab (upadvīpa°{\normalfont ...}kīrtitam)\lem  \msCa\Ed; \om\ \msCb}

agnīdhro navavarṣeṣu navaputrān asiñcayat\thinspace{\dandab} \dontdisplaylinenum

nābhiḥ kiṁpuruṣaś caiva harivarṣa ilāv\textsubring{r}taḥ \veg\dontdisplaylinenum
\varr{
        \ \vc agnīdhro\lem  \eme; agnīndhra \msCa\msCb, agnīndhro \Ed
        \ \vd °siñcayat\lem  \msCb; si{\il}{\lost} \msCa, °bhiṣiñcayat \Ed
        \ \va nābhiḥ\lem  \Ed; {\lost}{\lost} \msCa, nābhi \msCb}

pañcamaṁ ramyakaṁ varṣaṁ ṣaṣṭhaṁ caiva hiraṇmayam\thinspace{\dandab} \dontdisplaylinenum

kuravaḥ saptamo jñeyo bhadrāśvaś cāṣṭamaḥ sm\textsubring{r}taḥ \veg\dontdisplaylinenum
\varr{
        \ \vo (pañcamaṁ°{\normalfont ...}prakīrtitāḥ)\lem  \msCa\msCb; \om\ \Ed}

navamaḥ ketumālo 'bhūn navavarṣāḥ prakīrtitāḥ\thinspace{\dandab} \dontdisplaylinenum

himavaddakṣiṇe pārśve varṣo bhāratasaṁjñitaḥ \veg\dontdisplaylinenum
\varr{
        \ \vc °mālo\lem  \msCa; °māno \msCb, \om\ \Ed
        \ \vo (navamaḥ{\normalfont ...}°sambhavaḥ)\lem  \msCa\msCb; \om\ \Ed}

atrāpi navabhedo 'bhūd bhāratātmajasambhavaḥ\thinspace{\dandab} \dontdisplaylinenum

indradvīpaḥ kaśeruś ca tāmravarṇo gabhastimān \veg\dontdisplaylinenum
\varr{
        \ \vcd 'bhūd bhāratātmaja°\lem  \msCb\Ed; {\lost}{\lost}{\lost}{\lost}{\lost}ja° \msCa}

nāgadvīpas tathā saumyo gāndharvaś cātha vāruṇaḥ\thinspace{\dandab} \dontdisplaylinenum

ayaṁ ca navamo dvīpaḥ kumārīdvīpasaṁjñitaḥ \danda\dontdisplaylinenum

dakṣiṇe hemakūṭasya varṣaḥ kiṁpuruṣaḥ sm\textsubring{r}taḥ \veg\dontdisplaylinenum
\varr{
        \ \vc saumyo\lem  \msCb\Ed; saumyā \msCa
        \ \vd gāndharva°\lem  \msCa\msCb; gandharva° \Ed}

niṣadho dakṣiṇapārśve harivarṣa iti sm\textsubring{r}taḥ\thinspace{\dandab} \dontdisplaylinenum

merumūle tu rājendra jñeyo varṣa ilāv\textsubring{r}taḥ \veg\dontdisplaylinenum

uttaraṇeṇa (uttareṇa?) tu nīlasya varṣa ramyaka ucyate\thinspace{\dandab} \dontdisplaylinenum

śveta-uttarato jñeyo varṣaramyahiraṇmayaḥ \veg\dontdisplaylinenum

tasya uttarato jñeyas triś\textsubring{r}ṅgavaraparvataḥ\thinspace{\dandab} \dontdisplaylinenum

tasya cottarapārśve tu varṣaḥ kuruvale sm\textsubring{r}taḥ \veg\dontdisplaylinenum

pūrvaṁ bhadrāśvato jñeyaḥ ketumālas tu paścime\thinspace{\dandab} \dontdisplaylinenum

himaṁvān hemakūṭaś ca niṣadho nīla eva ca \veg\dontdisplaylinenum

śvetaś ca ś\textsubring{r}ṅgavantaś ca ṣaḍ ete varṣaparvatāḥ\thinspace{\dandab} \dontdisplaylinenum

aśītinavatīlakṣaḥ - varṣaparvatam āyatam \veg\dontdisplaylinenum

himavān hemakūṭaś ca niṣadhaś ceti dakṣiṇa\thinspace{\dandab} \dontdisplaylinenum

śvetaś caivatriś\textsubring{r}ṅgaś ca nīlaś caiva tathottare \veg\dontdisplaylinenum

niṣadho nīlamadhye tu meruḥ śailamanoramaḥ\thinspace{\dandab} \dontdisplaylinenum

praviṣṭaṣoḍaśādhas tāṁ caturāśītim ucch\textsubring{r}taḥ \veg\dontdisplaylinenum

yojanānāṁ sahasrāṇi dvātriṁśad ūrdha ! vist\textsubring{r}taḥ\thinspace{\dandab} \dontdisplaylinenum

brahmāmanovatī nāma pureva satimadhyame \veg\dontdisplaylinenum

devarājo 'marāvatyām agnis tejovatī pure \veg\dontdisplaylinenum

yamaḥ saṁyamanī nāma nityaṁ vasati bhūpate\thinspace{\dandab} \dontdisplaylinenum

nai\textsubring{r}tir vasati nityaṁ ramye śuddhavatī pure \veg\dontdisplaylinenum

varuṇo bhogavatyāṁ tu vāyor gandhavatī purī\thinspace{\dandab} \dontdisplaylinenum

mahodayāpurī ramyā somasyālayaraṁ sm\textsubring{r}tam \veg\dontdisplaylinenum

yaśovatī purī ramyānnityam āste triśūlinaḥ\thinspace{\dandab} \dontdisplaylinenum

tatragaṅgā catuḥbhinnā nipatantī mahītale \veg\dontdisplaylinenum

uttare paścime caiva pūrvadakṣiṇatas tathā\thinspace{\dandab} \dontdisplaylinenum

pūrvaṁ gaṅgā sravatyāccālakānandā ca dakṣiṇe \veg\dontdisplaylinenum

śītā paścimagā gaṅgā bhadrasomā tathottare\thinspace{\dandab} \dontdisplaylinenum

ṣaṣṭiyojanasāhasraṁ nirālambā nipatya ca \veg\dontdisplaylinenum

bhadrāśvaṁ plāvayitvā tu vanāny upavanāni ca\thinspace{\dandab} \dontdisplaylinenum

droṇasthalī girīṇāṁ ca atikramyārṇavaṁ gatā \veg\dontdisplaylinenum

tathaivālakanandā ca gatāśailenanimnagā\thinspace{\dandab} \dontdisplaylinenum

gaṅgā bhāratavarṣaṁ ca praviṣṭālavaṇo dadhim \veg\dontdisplaylinenum

plāvayitvā sthalīn sarvān mānuṣākaluṣāpahā\thinspace{\dandab} \dontdisplaylinenum

paścimena gatāgaṅgā sītānāmā ca bhārataḥ \veg\dontdisplaylinenum

plāvayet ketumālāṁ ca kṣetraśaivavanasthalīm\thinspace{\dandab} \dontdisplaylinenum

atikramyārṇavagatā sthalīdroṇī ca nimnagā \veg\dontdisplaylinenum

bhadrasomanadīty evaṁ plāvayitvottaraṁ kurun\thinspace{\dandab} \dontdisplaylinenum

sthalī prasravaṇadroṇīm atikramyārṇavaṁ gatā \veg\dontdisplaylinenum

mero vai dakṣiṇe pārśve jambūv\textsubring{r}kṣaḥ sanātanaḥ\thinspace{\dandab} \dontdisplaylinenum

tena nāmāṅkito rājan jambūdvīpa iti śrutam \veg\dontdisplaylinenum

koṭīṣoḍaśabhiś caiva ayutāni trayodaśa\thinspace{\dandab} \dontdisplaylinenum

adhordhayāma rājendra kṣityāvaraṇam antataḥ \veg\dontdisplaylinenum

navalakṣādhikaṁ rājan pañcakoṭī mahī sm\textsubring{r}tā\thinspace{\dandab} \dontdisplaylinenum

yojanānāṁ tu vijñeyaḥ p\textsubring{r}thivyāyām avistarāt \veg\dontdisplaylinenum

svādūdakasya ca bahir lokāloko mahāgiriḥ\thinspace{\dandab} \dontdisplaylinenum

kañcanidviguṇābhūmi tasmād giribahi sm\textsubring{r}taḥ \veg\dontdisplaylinenum

tasmād bāhyaḥ samudro bhūd garbhādeti samudrarāṭ\thinspace{\dandab} \dontdisplaylinenum

aṣṭāviṁśatikaṁ lakṣaṁ śatalakṣāṇi vistaram \veg\dontdisplaylinenum

etad bhūrlokavistāro hy ata ūrdhva bhuvaḥ sm\textsubring{r}taḥ\thinspace{\dandab} \dontdisplaylinenum

svarlokāsyapareṇaiva maharlokam ataḥ param \veg\dontdisplaylinenum

janalokas tapaḥ satyaṁ kramaśaḥ parikīrtitam\thinspace{\dandab} \dontdisplaylinenum

brahmalokaḥ sm\textsubring{r}taḥ satyaṁ viṣṇulokam ataḥ param \veg\dontdisplaylinenum


\alalfejezet{śivalokaḥ}
tasmāt pareṇa bodhavyaṁ divyadhyānapuraṁ mahat\thinspace{\dandab} \dontdisplaylinenum

sahasrabhaumaprāsādaṁ vaidūryamaṇitoraṇam \veg\dontdisplaylinenum

nānāratnavicitrāṇi nānābhūtagaṇākulam\thinspace{\dandab} \dontdisplaylinenum

sarvakāmasam\textsubring{r}ddhāni pūrṇaṁ tatra manoharaiḥ \veg\dontdisplaylinenum

tatra siṁhāsane divye sarvaratnavibhūṣite\thinspace{\dandab} \dontdisplaylinenum

tatrāste bhagavān rudraḥ somāṅkitajaṭādharaḥ \veg\dontdisplaylinenum

tryakṣatribhuvanaśreṣṭhas triśūlī tridaśādhipaḥ\thinspace{\dandab} \dontdisplaylinenum

devyā saha mahābhāgo gaṇaiś ca parivāritaḥ \veg\dontdisplaylinenum

skandanandipurogaś ca gaṇakoṭiśatākulaḥ\thinspace{\dandab} \dontdisplaylinenum

anekarudrakanyābhi rūpiṇībhir alaṅkitaḥ \veg\dontdisplaylinenum
\varr{
        \ \vc °kanyābhi°\lem  \corr; °kanyabhi° \Ed}

tatra puṇyanadī sapta sarvapāpāpanodanī\thinspace{\dandab} \dontdisplaylinenum

suvarṇavālukādivyā ratnapāṣāṇaśobhitā \veg\dontdisplaylinenum

pāvanī ca vareṇyā ca varārhāvaradā varā\thinspace{\dandab} \dontdisplaylinenum

vareśāvarabhadrā ca suprasannā jalāśivā \veg\dontdisplaylinenum

anekakusumārāmā ratnapuṣpaphaladrumāḥ\thinspace{\dandab} \dontdisplaylinenum

anekaratnaprākārā yojanāyutam ucchritāḥ \veg\dontdisplaylinenum

ahiṁsāsatyaniratāḥ kāmakrodhavivarjitāḥ\thinspace{\dandab} \dontdisplaylinenum

dhyānayogaratānityaṁ tatra modanti te narāḥ \veg\dontdisplaylinenum 

tatra gomātaras sarvā nivasanti yatavratāḥ\thinspace{\dandab} \dontdisplaylinenum

golokaḥ śivalokaś ca eka eva vidhīyate \veg\dontdisplaylinenum

tasmād ūrdhaṁ paraṁ jñeyaṁ sthānatrayam anuttamam\thinspace{\dandab} \dontdisplaylinenum

kandagaurī maheśānaṁ nityaśuddhaṁ paraṁ śivam \veg\dontdisplaylinenum

dinak\textsubring{r}t koṭisaṅkāsam anopamyaṁ sanātanam\thinspace{\dandab} \dontdisplaylinenum

ādityāda ! śivāntaś ca dvistheṇordhvakramaiḥ m\textsubring{r}staḥ (sm\textsubring{r}taḥ) \veg\dontdisplaylinenum


\alalfejezet{śāstravarṇanā}
\ujvers\nemsloka 
abhyantare tat kathito 'dya sāraṁ
\dontdisplaylinenum

\nemslokab 
kim anya rājan kathayāmi sāram \danda\dontdisplaylinenum

\nemslokac 
jñānārṇavaṁ kīrtita dharmasāram
\dontdisplaylinenum

\nemslokad 
purāṇavedopaniṣatsusāram \veg\dontdisplaylinenum
\varr{
        \ \va abhyantare tat ka°\lem  \msCa\msCb; atyantaretka° \Ed
        \ \vb kim anya rā°\lem  \msCa\Ed; kim anyad rā° \msCb
        \ \vc jñānārṇavaṁ kīrtita dharmasāram\lem  \msCb\Ed;
                jñānārṇṇa\uncl{ṅkīrti}{\lost}{\lost}{\lost}{\lost}{\lost} \msCa}

\ujvers\nemsloka 
yathā hi rājā parivāramadhye
\dontdisplaylinenum

\nemslokab 
yathāntavartī bahivartin eva \danda\dontdisplaylinenum

\nemslokac 
bhuñjanti bhogān satatāntavartī
\dontdisplaylinenum

\nemslokad 
kleśādhikaṁ nitya bahiḥsthitānām \veg\dontdisplaylinenum
\varr{
        \ \va °vāramadhye\lem  \msCa; °vāraṇai \msCb, °cāramadhye \Ed
        \ \vb yathāntava°\lem  \msCb\Ed; yathāntarvva \msCa\oo
                 °vartin eva\lem  \msCb\Ed; vartti\uncl{ne}va \msCa
        \ \vc bhuñjanti bhogān\lem  \msCb\Ed; \uncl{bhuñja} {\lost}{\lost}{\lost} \msCa\oo
                 satatāntavartī\lem  \msCa\Ed; satatānnavartī \msCb
        \ \vd bahiḥ°\lem  \msCa\Ed; bahi° \Ed}

\ujvers\nemsloka 
yathaiva rājā kariṇo 'ntadantam
\dontdisplaylinenum

\nemslokab 
bhuñjanti bhogān satataṁ narendra \danda\dontdisplaylinenum

\nemslokac 
yudhyeta rājā bahirdantabhogair
\dontdisplaylinenum

\nemslokad 
yadantaraṁ paśya samānajātam \veg\dontdisplaylinenum
\varr{
        \ \va kariṇo 'ntadantam\lem  \msCb; kariṇo 'ntardantam \msCa, kariṇāntadantadattam \Ed
        \ \vb bhuñjanti\lem  \msCb\Ed; bhujanti \msCa
        \ \vc rājā\lem  \msCa\Ed; rāja \msCb\oo
                 bahirdantabhogair\lem  \msCa\msCb; bahidattabhogair \Ed
        \ \vd yadantaraṁ paśya samānajātam\lem  \msCb; yadantare paśya samānajātam \Ed,
                                 yadanta\uncl{re} {\lost}{\lost}{\lost}{\lost}najātam \msCa}

\ujvers\nemsloka 
na dānatulyaṁ tv abhayapradasya
\dontdisplaylinenum

\nemslokab 
na yajñatulyaṁ jita-indriyasya \danda\dontdisplaylinenum

\nemslokac 
na cārthatulyaṁ jitakāminaś ca
\dontdisplaylinenum

\nemslokad 
na dharmatulyaṁ damakāmitasya \veg\dontdisplaylinenum
\varr{
        \ \vc °kāminaś ca\lem  \Ed; kāmina{\lost}{\lost} \msCa
        \ \vd na dharmatulyaṁ\lem  \Ed; {\lost}{\lost}{\lost}{\lost}{\lost} \msCa, \om\ \msCb\oo
                 damakāmitasya\lem  \msCa; \om\ \msCb, damakāminasya \Ed }

\ujvers\nemsloka 
bahvantaraṁ naiva hi dharmayoś ca
\dontdisplaylinenum

\nemslokab 
kleśādhikaṁ bāhyaphalālpasāram \danda\dontdisplaylinenum

\nemslokac 
yad atra dharmaṁ phalanaiṣṭhikasya
\dontdisplaylinenum

\nemslokad 
na tulya koṭīśatayājināpi \veg\dontdisplaylinenum
\varr{
        \ \vc °naiṣṭhikasya\lem  \Ed; °naiśikasya \msCb}

\ujvers\nemsloka 
etat pavitraṁ paramaṁ sadharmam
\dontdisplaylinenum

\nemslokab 
purā yathoktaṁ parameśvareṇa \danda\dontdisplaylinenum

\nemslokac 
mayāpi tulyaṁ kathitaṁ yathāvat
\dontdisplaylinenum

\nemslokad 
purāṇavedopaniṣatsusāram \veg\dontdisplaylinenum

\ujvers\nemsloka 
sadojasaubhāgyam atīva medhā
\dontdisplaylinenum

\nemslokab 
nirutsukaḥ saumyam anuttamaṁ ca \danda\dontdisplaylinenum

\nemslokac 
suputrapautraṁ na vichinnagotram
\dontdisplaylinenum

\nemslokad 
bhavanti vidyādharalokapūjyam \veg\dontdisplaylinenum
\varr{
        \ \va sadoja°\lem  \Ed; sadojaḥ \msCb
        \ \vb nirutsukaḥ\lem  \Ed; nirutsuka° \msCb}

\ujvers\nemsloka 
yaśaśriyaṁ kīrtir atīva tejo
\dontdisplaylinenum

\nemslokab 
janapriyo dhānyadhanāyuv\textsubring{r}ddhim \danda\dontdisplaylinenum

\nemslokac 
prabodhaprajñārujadharmav\textsubring{r}ddhim
\dontdisplaylinenum

\nemslokad 
bhavanti taṁ śāstrasadābhiyogī \veg\dontdisplaylinenum
\varr{
        \ \va yaśaḥ°\lem  \msCb; yaśa° \Ed
        \ \vb °v\textsubring{r}ddhim\lem  \msCb; °v\textsubring{r}ddhiḥ \Ed
        \ \vc °v\textsubring{r}ddhim\lem  \Ed; °v\textsubring{r}ddhi \msCb
        \ \vd taṁ\lem  \msCb; te \Ed}

\ujvers\nemsloka 
yaśasvinī āryasuvarṇaś\textsubring{r}ṅgī
\dontdisplaylinenum

\nemslokab 
vedāntavipradvijagāyaneṣu \danda\dontdisplaylinenum

\nemslokac 
dattvā phalaṁ tīrtham anuttameṣu
\dontdisplaylinenum

\nemslokad 
ś\textsubring{r}ṇvanti ye tasya bhavet sapuṇyam \veg\dontdisplaylinenum
\varr{
        \ \vc °nuttameṣu\lem  \Ed; °numeṣu \msCb}

\ujvers\nemsloka 
daśādhikaṁ vācayituś ca puṇyam
\dontdisplaylinenum

\nemslokab 
śatādhikaṁ yaḥ paṭhati prabhāte \danda\dontdisplaylinenum

\nemslokac 
sahasraśaḥ pustak\textsubring{r}tasya puṇyam
\dontdisplaylinenum

\nemslokad 
pare 'bhyaste kīrtayate 'yutāni \veg\dontdisplaylinenum
\varr{
        \ \va vācayituś ca\lem  \msCb; vāca catuś ca \Ed
        \ \vd pare\lem  \msCb; paro \Ed\oo
                 kīrta°\lem  \msCb\pcorr; kīrti° \msCb\acorr}

\ujvers\nemsloka 
adhītya yasyoragataṁ suśāstram
\dontdisplaylinenum

\nemslokab 
samastamadhyāyam anukramena \danda\dontdisplaylinenum

\nemslokac 
daśāyutāṅgo dadatuś ca puṇyam
\dontdisplaylinenum

\nemslokad 
labhaty asaṁdigdhayathādinaikaṁ \veg\dontdisplaylinenum
\varr{
        \ \vc daśāyutāṅgo dadatu°\lem  \Ed; daśāyu\uncl{taṅga de}datu° \msCb
        \ \vd °yathādinaikaṁ\lem  \Ed; °ya\uncl{thādi}naikaṁ \msCb}

\ujvers\nemsloka 
yenedaṁ śāstrasāram avikalamanasā yo 'bhyaset tatprayatnāt
\dontdisplaylinenum

\nemslokab 
vyakto 'sau siddhayogī bhavati ca niyataṁ yas tu cittaprasannaḥ \danda\dontdisplaylinenum

\nemslokac 
pitryaṁ yo gītapūrvaṁ pratidina śataśa uddhriyante ca sarve
\dontdisplaylinenum

\nemslokad 
ātmānaṁ nirvikalpaṁ śivapadam asamaṁ prāpnuvantīha sarve \veg\dontdisplaylinenum

\vers


\jump
\begin{center}
\ketdanda iti v\textsubring{r}ṣasārasaṁgrahe śāstravarṇanā nāma caturviṁśatitamo 'dhyāyaḥ samāptaḥ\ketdanda
\end{center}
\dontdisplaylinenum\vers 


\jump
\begin{center}
\ketdanda v\textsubring{r}ṣasārasaṁgrahaḥ samāpta iti\ketdanda
\end{center}
\dontdisplaylinenum\vers 
