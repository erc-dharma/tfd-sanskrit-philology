\renewcommand{\dnapp}[1]{}\renewcommand{\rmapp}[1]{#1}
\fejno=0\versno=0
\begin{center}{\Huge  V\textsubring{r}ṣasārasa\.mgrahaḥ}\end{center}

\alfejezet{\textbf{prathamo 'dhyāyaḥ}}\jump\jump
\vers
\szam\bek\versno=0\fejno=1
\thispagestyle{empty}


\ujvers\nemsloka 
anādimadhyāntam anantapāra\.m
\dontdisplaylinenum
            \var{\va °ntam ananta°\lem  \msCa\msCbpcorr\msCc\msNa\msNb\msNc\Ed; °ntam anta° \msCbacorr\oo
                 °pāra\.m\lem  \mssCaCbCc\msNc\Ed; °pāraga\.m \msNa\msNb}%

\nemslokab 
susūkṣmam avyaktajagatsusāram \danda\dontdisplaylinenum
            \var{\vb susūkṣmam\lem  \msCa\msCb\msNa\msNb\msNc\Ed; śusukṣmam \msCc\oo
                °jagatsusāra\.m\lem  \msCa\msCb\msNa\msNc\Ed; °jagaśusāra\.m \msCc, °jagatsurāsura\.m \msNb}%

\nemslokac 
harīndrabrahmādibhir āptam agra\.m
\dontdisplaylinenum
            \var{\vc °bhir āpta°\lem  \conj; °bhir āsa° \mssCaCbCc\msNa\msNb\msNc\Ed}%

\nemslokad 
praṇamya vakṣye v\textsubring{r}ṣasārasa\.mgraham \veg\dontdisplaylinenum
            \paral{\textit{{\normalfont Testimonia for this chapter:    \msCa\ ff.\thinspace 193v--195v,
                                                \msCb\ ff.\thinspace 201v--203v,
                                                \msCc\ ff.\thinspace 267r--270r,
                                                \msNa\ ff.\thinspace 1v--3v,
                                                \msNb\ exp.\thinspace 43 (lower then upper leaf; 1.62cd--2.22 are missing),
                                                \msNc\ ff.\thinspace 209v--211v;
                                                \mssCaCbCc\ = \msCa + \msCb + \msCc }}}


\alalfejezet{janamejayavaiśampāyanasa\.mvādaḥ}
\vers

śatasāhasrika\.m grantha\.m sahasrādhyāyam uttamam\thinspace{\dandab} \dontdisplaylinenum
            \var{\vb sahasrādhyāyam u°\lem  \msCa\msCb\msNa\msNb\msNc; sahaśradhyāyam u° \msCc,
                                                        sahasrādhyāyar u° \Ed}%

parva cāsya śata\.m pūrṇa\.m śrutvā bhāratasa\.mhitām \veg\dontdisplaylinenum
            \var{\vc parva cāsya\lem  \msCa\msNa\msNb\msNc; parvañ cāsya \msCb,
                                                parvvam asya \msCc, pūrvam asya \Ed\oo
                 śata\.m pūrṇa\.m\lem  \msCa\msCb\msNa\msNb\msNc\Ed; ta \msCc}%
            \var{\vd śrutvā\lem  \msCa\msCc\msNa\msNb\msNc\Ed; śraddhā \msCb\oo
                 bhāratasa\.mhitām\lem  \msCa\msCb\msNa\msNb\msNc; bhārasa\.mhitā \msCc, nārādasa\.mhitām \Ed}%
            \paral{\textit{\vc {\normalfont  cf. MBh 1.2.70ab: } etat parvaśata\.m pūrṇa\.m vyāsenokta\.m mahātmanā}}

\vers

at\textsubring{r}ptaḥ puna papraccha vaiśampāyanam eva hi\thinspace{\dandab} \dontdisplaylinenum
            \var{\va at\textsubring{r}ptaḥ puna papraccha\lem  \eme;
                a\uncl{t\textsubring{r}ptaḥ}pu{\il}{\il}praccha \msCa, 
                at\textsubring{r}ptaḥ punaḥ papraccha \msCb\msNa\msNb\msNc,
                at\textsubring{r}ptaḥ punar apracche \msCc,
                at\textsubring{r}ptā punaḥ papraccha \Ed}% 
            \var{\vb vaiśampāyana°\lem  \msCa\msCb\msNa\msNb\msNc\Ed; vesampāyana° \msCc}%

janamejaya yat pūrva\.m tac ch\textsubring{r}ṇu tvam atandritaḥ \veg\dontdisplaylinenum
            \var{\vc janamejaya yat\lem  \eme;
                  janamejayena yat \msCapcorr\msCb\msNa\msNb\msNc\Ed,
                  janamejaye yat \msCaacorr,
                  janmejayena yam \msCc\oo
                 pūrva\.m\lem  \mssCaCbCc\msNc\Ed; pūrva \msNa, pūr\uncl{va} \msNb}%
              \var{\vd tac ch\textsubring{r}ṇu\lem  \msCa\msCb\msNa\msNc\Ed; tac ch\textsubring{r}ṇa \msCc, {\lost}{\lost}{\lost} \msNb\oo
                   °tandritaḥ\lem  \msCc; °tandritam \msCa\msCb\msNc\Ed, °tandri\uncl{tam} \msNa, {\lost}{\lost}{\lost} \msNb}%

janamejaya uvāca~{\dandab}\dontdisplaylinenum 
            \var{\vo janamejaya\lem  \msCa\msCb\msNa\msNb\msNc\Ed; janmejaya \msCc}%

bhagavan sarvadharmajña sarvaśāstraviśārada\thinspace{\danda} \dontdisplaylinenum
           \var{\va bhagavan sa°\lem  \msCa\msCb\msNa\msNb\msNc\Ed; bhacāva\.m sa° \msCc\oo
                °dharma°\lem  \mssCaCbCc\msNb\msNc\Ed; \om\ \msNa}%
           \var{\vb °viśārada\lem  \msCa\msNb\msNc; °visāradaḥ \msCb\msCc\msNa\Ed}%

asti dharma\.m para\.m guhya\.m sa\.msārārṇavatāraṇam \veg\dontdisplaylinenum
            \var{\vd dharma\.m\lem  \msCa\msNa\msNb\msNc\Ed; dharmaḥ \msCb, dharma \msCc\oo
                 para\.m guhya\.m\lem  \msCa\msNb\Ed; paro guhya \msCb, para\.m guhya \msCc\msNa, paraguhya\.m \msNc}%

dvaipāyanamukhodgīrṇa\.m dharma\.m yat tad dvijottama\thinspace{\dandab} \dontdisplaylinenum
            \var{\va dvaipāyana°\lem  \msCa\msCb\msNa\msNb\msNc\Ed; dvepāyana° \msCc\oo
                 °mukhodgīrṇa\.m\lem  \msCa\msCb\msNa\msNb\msNc; °mukhodgīrṇa \msCc, mukhād gīrṇa\.m \Ed}%
             \var{\vb dharma\.m yat tad dvi°\lem  \msCb; dharma\.m vā yad dvi° \msCa\msNa\msNb\msNc\Ed, 
                                                                dharmavat ya dvi° \msCc\oo 
                  °ttama\lem  \msCa\msCb\msNa\msNb\msNc\Ed; °ttamaḥ \msCc}%  

kathayasva hi me t\textsubring{r}pti\.m kuru yatnāt tapodhana  \veg\dontdisplaylinenum
            \var{\vd yatnāt tapodhana\lem  \msCb\msNa\msNb\msNc\Ed; yannāt ta{\il}{\il}na \msCa,
                                                        yatnā tapodhanaḥ \msCc}%

vaiśampāyana uvāca~{\dandab}\dontdisplaylinenum 

ś\textsubring{r}ṇu rājann avahito dharmākhyānam anuttamam\thinspace{\danda} \dontdisplaylinenum
            \var{\vb °khyānam anuttamam\lem  \msCa\msNa\msNb\msNc\Ed; °khyānam uttamam \msCb,
                                               °khyānam utamam \msCc}%

vyāsānugrahasamprāpta\.m guhyadharma\.m ś\textsubring{r}ṇotu me \veg\dontdisplaylinenum
            \var{\vc °prāpta\.m\lem  \msCa\msCb\msNa\msNb\msNc\Ed; °prāpta \msCc}%
            \var{\vd °dharma\.m\lem  \msCa\msCb\msNa\msNb\msNc\Ed; °rma\.m \msCc\oo
                 ś\textsubring{r}ṇotu\lem  \msCa\msCb\msNa\msNb\msNc\Ed; ś\textsubring{r}ṇota \msCc\oo
                 me\lem  \msCa\msCc\msNa\msNb\msNc\Ed; mai \msCb}%

anarthayajñakartāra\.m tapovrataparāyaṇam\thinspace{\dandab} \dontdisplaylinenum
            \var{\va °kartāra\.m\lem  \mssCaCbCc\msNa\msNc\Ed; °karttanta\.m \msNb}%
            \var{\vb °yaṇam\lem  \msCa\msCb\msNb\msNc\Ed; °yana \msCc, °yaṇaḥ \msNa}%

śīlaśaucasamācāra\.m sarvabhūtadayāparam \veg\dontdisplaylinenum
            \var{\vd °param\lem  \msCa\msCb\msNa\msNc\Ed; °nvitam \msCc, °\uncl{pa}ra\.m \msNb}%

jijñāsanārtha\.m praśnaika\.m viṣṇunā prabhaviṣṇunā\thinspace{\dandab} \dontdisplaylinenum
            \var{\va °rtha\.m praśnaika\.m\lem  \msCb\msNa\msNb\msNc; °rtha\.m praśneka\.m \msCa, 
                                        °rthapraśnekam \msCc\Ed}%
            \var{\vb prabha°\lem  \msCa\msCb\msNa\msNb\Ed; prabhu° \msCc, prābha° \msNc}%

dvijarūpadharo bhūtvā papraccha vinayānvitaḥ \veg\dontdisplaylinenum
            \var{\vc °dharo\lem  \msCb\msCc\msNa\msNc\Ed; °{\il}ro \msCa, °dharā \msNb}%
            \var{\vd °nvitaḥ\lem  \msCa\msCb\msNa\msNb\msNc\Ed; °nvita\.m \msCc}%


\alalfejezet{brahmavidyā}
[vigatarāga uvāca~{\dandab}\dontdisplaylinenum ]

brahmavidyā katha\.m jñeyā rūpavarṇavivarjitā\thinspace{\danda} \dontdisplaylinenum
            \var{\va jñeyā\lem  \msCa\msNa\msNb\msNc; jñeya\.m \msCb\msCc, bhūyo \Ed}%
            \var{\vb °varṇa°\lem  \mssCaCbCc\msNa\msNb\msNc; °varṇā° \Ed\oo
                 °varjitā\lem   \msCa\msCb\msNa\msNb\Ed; °varjita\.m \msCc, °varjitāḥ \msNc}%

svaravyañjananirmuktam akṣara\.m kimu tat param \veg\dontdisplaylinenum
            \var{\vc °vyañjana°\lem  \mssCaCbCc\msNa\msNb\msNc; °vyajjana° \Ed\oo
                 °muktam akṣa°\lem  \msCa\msCc\msNa\msNb\msNc\Ed; °mukta akṣa° \msCb}%
            \var{\vd kimu tat param\lem  \msCa\msNa\msNc\Ed; kim ataḥ param \msCb\msCc, kim atat para\.m \msNb}%

anarthayajña uvāca~{\dandab}\dontdisplaylinenum 

anuccāryam asandigdham avicchinnam anākulam\thinspace{\danda} \dontdisplaylinenum
            \var{\va °ccārya°\lem  \msCa\msCb\msNa\msNb\Ed; °cārya° \msCc\msNc}%
            \var{\vab °sandigdham avicchinnam anākulam\lem  \msCa\msCb\msNa\msNc\Ed;
                                °vicchinnasandigdham anākuna \msCc, °sandigdha\-m anacchinnam anākulam \msNb}%

nirmala\.m sarvaga\.m sūkṣmam akṣara\.m kimu tatparam \veg\dontdisplaylinenum
            \var{\vd kimu tatparam\lem  \msCa\msNa\msNb\msNc\Ed; kim ataḥ param \msCb,
                                                kim atatpara\.m \msCc}%


\alalfejezet{kālapāśaḥ}
vigatarāga uvāca~{\dandab}\dontdisplaylinenum 

dehī dehe kṣaya\.m yāte bhūjalāgniśivādibhiḥ\thinspace{\danda} \dontdisplaylinenum
            \var{\va dehe kṣa°\lem  \msCa\msCc\msNc; dehāt kṣa° \msCb, dehakṣa° \msNa\msNb\Ed}%
            \var{\vb °śivādibhiḥ\lem   \msCa\msCb\msNa\msNb\msNc\Ed; °śivādibhi \msCc}%
            \paral{\textit{\vb {\normalfont cf.\ Kūrmapurāṇa 2.23.74: } 
                atha kaścit pramādena mriyate 'gniviṣādibhiḥ{\thinspace\danda} 
                tasyāśauca\.m vidhātavya\.m kārya\.m caivodakādikam{\thinspace\ketdanda}}}

yamadūtaiḥ katha\.m nīto nirālambo nirañjanaḥ \veg\dontdisplaylinenum
            \var{\vc °dūtaiḥ\lem  \msCa\msCb\msNa\msNb\msNc\Ed; °dūte \msCc\oo
                 nīto\lem  \msCa\msCb\msNa\msNb\msNc; nītvā \msCc, nītā \Ed}%
            \var{\vd nirañjanaḥ\lem  \msCa\msCb\msNa\msNb\msNc\Ed; nirañjana \msCc}%

kālapāśaiḥ katha\.m baddho nirdehaś ca katha\.m vrajet\thinspace{\dandab} \dontdisplaylinenum
            \var{\va °pāśaiḥ\lem  \msCa\msCb\msNa\msNb\msNc\Ed; °pāśe \msCc\oo
                 baddho\lem  \msCa\msCc\msNa\msNb\msNc\Ed; ba\uncl{ddho} \msCb}%
            \var{\vb nirdehaś ca\lem  \msCa\msCb\msNa\msNb\msNc\Ed; nirdahaḥ sa \msCc\oo
                 vrajet\lem  \mssCaCbCc\msNa\msNc\Ed; bhavet \msNb}%

svarga\.m vā sa katha\.m yāti nirdeho bahudharmak\textsubring{r}t \danda\dontdisplaylinenum
            \var{\vc svarga\.m\lem  \msCa\msCb\msNa\msNb\msNc\Ed; svarga \msCc\oo
                 sa\lem  \mssCaCbCc\msNa\msNc\Ed; sa\.m \msNb\oo
                 yāti\lem  \msNa\msNb\msNc; yānti \mssCaCbCc\Ed}%

etan me sa\.mśaya\.m brūhi jñātum icchāmi tattvataḥ \veg\dontdisplaylinenum
            \var{\ve sa\.mśaya\.m\lem  \mssCaCbCc\msNc\Ed; sa\.mśaye \msNa, sa\.mśayo \msNb}%
            \var{\vf °tum icchāmi\lem  \msCa\msCc\msNa\msNb\msNc\Ed; °tum i \msCb}%

anarthayajña uvāca~{\dandab}\dontdisplaylinenum 
            \var{\vo anarthayajña uvāca\lem  \mssCaCbCc\msNapcorr\msNb\msNc\Ed; \om\ \msNaacorr}%

atisa\.mśayakaṣṭa\.m te p\textsubring{r}ṣṭo 'ha\.m dvijasattama\thinspace{\danda} \dontdisplaylinenum
            \var{\va atisa\.mśayakaṣṭa\.m te\lem  \msCb\msNa\msNb\msNc;
                atiśa\.msa\uncl{ya}kaṣṭan te \msCa, atiśa\.msayakaṣṭam me \msCc\Ed}%
            \var{\vb dvijasattama\lem  \msCa\msCb\msNa\msNb\msNc\Ed; ca dvijottamaḥ \msCc}%

durvijñeya\.m manuṣyais tu devadānavapannagaiḥ \veg\dontdisplaylinenum
            \var{\vc °jñeya\.m\lem  \msCa\msCb\msNa\msNc; °jñeya \msCc\msNb\Ed\oo
                 manuṣyais tu\lem  \msCa\msNa\msNb\msNc\Ed; manuṣaiś ca \msCb, maṇukṣe\uncl{ptu} \msCc}%

karmahetuḥ śarīrasya utpattir nidhana\.m ca yat\thinspace{\dandab} \dontdisplaylinenum
            \var{\va karma°\lem  \msCa\msCb\msNa\msNb\msNc; anarthayajña uvāca{\danda}{\danda} karma° \msCc\Ed\oo
                 °hetuḥ\lem  \msCb; °hetu \msCa\msNa\msNb\msNc\Ed, °he\.mtu \msCc\oo
                 śarīrasya\lem  \msCa\msCb\msNa\msNb\msNc\Ed; śarīrasya\.m \msCc}%
            \var{\vb utpattir ni°\lem  \corr; utpattini° \msCa\msCb\msNa\msNb\msNc\Ed, utpatini° \msCc\oo
                 yat\lem  \mssCaCbCc\msNa\msNc\Ed; yaḥ \msNb}%

suk\textsubring{r}ta\.m duṣk\textsubring{r}ta\.m caiva pāśadvayam udāh\textsubring{r}tam \veg\dontdisplaylinenum
            \var{\vc suk\textsubring{r}ta\.m\lem  \msCa\msCb\msNa\msNb\msNc\Ed; suk\textsubring{r}tak\textsubring{r}tan \msCc}%
            \var{\vd °h\textsubring{r}tam\lem  \msCa\msCb\msNa\msNb\msNc\Ed; °h\textsubring{r}taḥ \msCc}% 

tenaiva saha sa\.myāti naraka\.m svargam eva vā\thinspace{\dandab} \dontdisplaylinenum
            \var{\va tenaiva\lem  \msCa\msCb\msNa\msNb\msNc\Ed; teneva \msCc\oo
                 sa\.myāti\lem  \msCa\msCb\msNa\msNb\msNc\Ed; sā yānti \msCc}%
            \var{\vb vā\lem  \mssCaCbCc\msNb\msNc\Ed; ca \msNa}%

sukhaduḥkha\.m śarīreṇa bhoktavya\.m karmasambhavam \veg\dontdisplaylinenum
            \var{\vc °duḥkha\.m\lem  \msCa\msCb\msNa\msNc; °duḥkha \msCc\msNb\Ed}%
            \var{\vd °sambhavam\lem  \msCa\msCb\msNa\msNb\msNc; °sambhavaḥ \msCc\Ed}%

hetunānena viprendra dehaḥ sambhavate n\textsubring{r}ṇām\thinspace{\dandab} \dontdisplaylinenum
            \var{\va °ndra\lem  \mssCaCbCc\msNa\msNc\Ed; °ndraḥ \msNb}%
            \var{\vb dehaḥ\lem  \msCa\msCb\msNa\msNc\Ed; dehe \msCc, deha \msNb\oo
                 n\textsubring{r}ṇām\lem  \msCa\msNa\msNb\msNc\Ed; n\textsubring{r}ṇā \msCb\msCc}%

ya\.m kālapāśam ity āhuḥ ś\textsubring{r}ṇu vakṣyāmi suvrata \veg\dontdisplaylinenum
            \var{\vc ya\.m kālapāśam ity āhuḥ\lem  \eme; ya\.m kālapāśam ity āha \msCa\msCb\msNa, 
                kālapāseti satvāha \msCc, ya\.m kālapāśam ity āhu \msNb\msNc,
                                        ya\.m kālapāśam ity āhu \Ed}%
            \var{\vd °vrata\lem  \msCa\msNa\msNb\msNc\Ed; °vrataḥ \msCb\msCc}%

na tvayā vidita\.m kiñcij jijñāsyasi katha\.m dvija\thinspace{\dandab} \dontdisplaylinenum
            \var{\va vidita\.m\lem  \msCa\msCb\msNa\msNb\msNc\Ed; vidita \msCc}%
            \var{\vab kiñcij ji°\lem  \msCb; kiñcid vi° \msCapcorr\msNa\msNb\msNc\Ed, kid vi° \msCaacorr, 
                                                        kiñci ji° \msCc}%
            \var{\vb katha\.m dvija\lem  \msCa\msCb\msNa\msNb\msNc\Ed;
                                        {\il}{\il}{\il}{\il}{\il}{\il}{\il}{\il}{\il} \uncl{ma tvayā vidita\.m kiñcid vijñāsyasi} 
                                        \cancelled\ katha\.m dvija \msCc}%

kālapāśa\.m ca viprendra sakala\.m vettum arhasi \veg\dontdisplaylinenum
            \var{\vd vettum arhasi\lem  \mssCaCbCc\msNa\msNb; vettum ūhasi \msNc, vaktum arhasi \Ed}%

kalākalitakāla\.m ca kālatattvakalā\.m ś\textsubring{r}ṇu\thinspace{\dandab} \dontdisplaylinenum
            \var{\va kalā°\lem  \msCa\msCb\msNapcorr\msNb\msNc\Ed; kālā° \msCc\msNaacorr\oo
                 °kāla\.m\lem  \mssCaCbCc\msNa\msNb\msNc; °kālaś \Ed}%
            \var{\vb °kalā\.m\lem  \msCa\msCc\msNb\Ed; °kalā \msCb\msNc, °vidhi\.m \msNa}%

truṭidvaya\.m nimeṣas tu nimeṣadviguṇā kalā \veg\dontdisplaylinenum
            \var{\vc truṭidvaya\.m\lem  \msCa\msCc\msNc\Ed; tuṭidvaya \msCb\msNb, tuṭidvaya\.m \msNa\oo
                 °meṣas tu\lem  \msCb\msCc\msNb\msNc\Ed; °mevas tu \msCa, °nimeṣadvi° \msNa}%

kalādviguṇitā kāṣṭhā kāṣṭhā vai tri\.mśatiḥ kalā\thinspace{\dandab} \dontdisplaylinenum
            \var{\vb kāṣṭhā vai tri\.mśatiḥ\lem  \msCa\msNa\msNb\msNc\Ed; vai tri\.mśatā \msCb, kāṣṭhā vai tri\.mśati \msCc}%

tri\.mśatkalā muhūrtaś ca mānuṣena dvijottama \veg\dontdisplaylinenum
            \var{\vc muhūrtaś ca\lem  \msCa\msCc\msNa\msNb\msNc; muhūrtta \msCb, muhūrtañ ca \Ed}%
            \var{\vd mānuṣena\lem  \msCa\msCb\msNa\msNb\msNc\Ed; mānu\uncl{ṣaś ca} \msCc\oo
                 °ttama\lem  \mssCaCbCc\msNa\msNcpcorr\Ed; °tamaḥ \msNb, °ttamaḥ \msNc}%

muhūrtatri\.mśakenaiva ahorātra\.m vidur budhāḥ\thinspace{\dandab} \dontdisplaylinenum
            \var{\va muhūrta°\lem  \mssCaCbCc\msNa\msNb\msNc; muhūrta\.m \Ed}%

ahorātra\.m punas tri\.mśan māsam āhur manīṣiṇaḥ \veg\dontdisplaylinenum

samā dvādaśa māsāś ca kālatattvavido janāḥ\thinspace{\dandab} \dontdisplaylinenum
            \var{\va samā\lem  \msCa\msCb\msNa\msNb\msNc\Ed; māsa \msCc\oo
                 °māsā°\lem  \msCa\msCb\msNa\msNb\msNc; °māsa° \msCc\Ed}%
            \var{\vb kāla°\lem  \mssCaCbCc\msNa\msNb\Ed; kalā° \msNc}%

śata\.m varṣasahasrāṇi trīṇi mānuṣasa\.mkhyayā \veg\dontdisplaylinenum
            \var{\vc śata\.m\lem  \mssCaCbCc\msNa\msNb\msNc; śata° \Ed}%
            \var{\vb mānuṣa°\lem  \msCa\msNa\msNb\msNc\Ed; māṇuṣya° \msCb\msCc\ \unmetr}%

ṣaṣṭi\.m caiva sahasrāṇi kālaḥ kaliyugaḥ sm\textsubring{r}taḥ\thinspace{\dandab} \dontdisplaylinenum
            \var{\vo \om\ \msNb\ \eyeskip{from 21d to 24d}}%
            \var{\va ṣaṣṭi\.m caiva\lem  \mssCaCbCc\msNc; ṣaṣṭi\.m varṣa° \msNa, \om\ \msNb, ṣaṣṭiś caiva \Ed}%
            \var{\vb °yugaḥ\lem  \mssCaCbCc\msNa\msNc; \om\ \msNb, °yuga \Ed}%

dviguṇaḥ kalisa\.mkhyāto dvāparo yuga sa\.mjñitaḥ \veg\dontdisplaylinenum
            \var{\vc dviguṇaḥ\lem  \mssCaCbCc\msNa\msNc; \om\ \msNb, dviguṇā \Ed}%
            \var{\vd dvāparo\lem  \mssCaCbCc\msNa\msNc; \om\ \msNb, dvāpare \Ed}%

tretā tu triguṇā jñeyā catuḥ k\textsubring{r}tayugaḥ sm\textsubring{r}taḥ\thinspace{\dandab} \dontdisplaylinenum
            \var{\vo \om\ \msNb\ \eyeskip{from 21d to 24d}}%
            \var{\va tretā\lem  \msCa\msCb\msNa\Ed; tetrā \msCc, \om\ \msNb, tretrā \msNc}%
            \var{\vb °yugaḥ\lem  \mssCaCbCc\msNa\msNc; \om\ \msNb, °yuga \Ed}%

eṣā caturyugāsa\.mkhyā k\textsubring{r}tvā vai hy ekasaptatiḥ \veg\dontdisplaylinenum
            \var{\vd hy e°\lem  \mssCaCbCc\msNa\msNb\Ed; he° \msNc}%

manvantarasya caikasya jñānam ukta\.m samāsataḥ\thinspace{\dandab} \dontdisplaylinenum
            \var{\vo \om\ \msNb\ \eyeskip{from 21d to 24d}}%
            \var{\va caikasya\lem  \mssCaCbCc\msNapcorr\msNc\Ed; \om\ \msNaacorr\msNb}%

kalpo manvantarāṇā\.m tu caturdaśa tu sa\.mkhyayā \veg\dontdisplaylinenum
            \var{\vc kalpo\lem  \msCb; kalpa \msCa\msCc\msNa\msNc\Ed, \om\ \msNb}%
            \var{\vd °daśa\lem  \msCa\msCc\msNa\msNc\Ed; °daśa\.m \msCb, \om\ \msNb}%

daśa kalpasahasrāṇi brahmāhaḥ parikalpitam\thinspace{\dandab} \dontdisplaylinenum
            \var{\vb °āhaḥ\lem  \msCb\msCc\msNa\msNb\msNc\Ed; °āha \msCa\oo
                 parikalpitam\lem  \msCa\msNc; karikalpitam \msCb, parikalpitaḥ \msCc\msNb\Ed, parikīrtitāḥ \msNa}%

rātrir etāvatī proktā munibhis tattvadarśibhiḥ \veg\dontdisplaylinenum

rātryāgame pralīyante jagat sarva\.m carācaram\thinspace{\dandab} \dontdisplaylinenum
            \var{\va pralīyante\lem  \msCa\msCc\msNa\msNb\msNc\Ed; pralīyate \msCb}%

ahāgame tathaiveha utpadyante carācaram \veg\dontdisplaylinenum
            \var{\vd ahāgame\lem  \mssCaCbCc\msNa\msNc; ahāga{\lost} \msNb, ahnāgame \Ed}%

parārdhaparakalpāni atītāni dvijottama\thinspace{\dandab} \dontdisplaylinenum
            \var{\va °rdha°\lem  \mssCaCbCc\msNa\msNc\Ed; °rdha\.m \msNb}%

anāgata\.m tathaivāhur bh\textsubring{r}gurādimaharṣayaḥ \veg\dontdisplaylinenum
            \var{\vcd °vāhur bh\textsubring{r}°\lem  \msCa\msCb\msNa\msNc\Ed; °vāhu bh\textsubring{r}° \msCc\msNb}%
            \var{\vd maharṣayaḥ\lem  \mssCaCbCc\msNapcorr\msNb\Ed; mahayaḥ \msNaacorr, marhaṣayaḥ \msNc}%

yathārkagrahatārendu bhramato d\textsubring{r}śyate tv iha\thinspace{\dandab} \dontdisplaylinenum
            \var{\vb d\textsubring{r}śyate tv iha\lem  \msCa\msNa\msNb\msNc\Ed; d\textsubring{r}śyandiha \msCb, d\textsubring{r}syate tv ihaḥ \msCc}%

kālacakra\.m bhramatvaiva viśrama\.m na ca vidmahe \veg\dontdisplaylinenum
            \var{\vc °cakra\.m\lem  \mssCaCbCc\msNa\msNc\Ed; °cakra \msNb\oo
                 °tvaiva\lem  \msCa\msNa\msNc\Ed; °tveva \msCb\msNb, °tveha \msCc}%
            \var{\vd °śrama\.m\lem  \mssCaCbCc\msNapcorr\msNc\Ed; °śramo \msNaacorr, °śrāman \msNb\oo
                 vidmahe\lem  \msCa\msCc\msNa\msNb\msNc\Ed; vigrahe \msCb}%

kālaḥ s\textsubring{r}jati bhūtāni kālaḥ sa\.mharate punaḥ\thinspace{\dandab} \dontdisplaylinenum
            \var{\vb kālaḥ\lem  \mssCaCbCc\msNa\msNb\msNc; kāla \Ed}%

kālasya vaśagāḥ sarve na kālavaśak\textsubring{r}t kvacit \veg\dontdisplaylinenum
            \var{\vc vaśagāḥ\lem  \mssCaCbCc\msNa\msNb\msNc; vaśagā \Ed}%
            \paral{\textit{\vo \kb\ {\normalfont Kūrmapurāṇa 1.11.32: } 
                kālaḥ s\textsubring{r}jati bhūtāni kālaḥ sa\.mharate prajāḥ{\thinspace\danda}
                sarve kālasya vaśagā na kālaḥ kasyacid vaśe{\thinspace\ketdanda}}}

caturdaśaparārdhāni devarājā dvijottama\thinspace{\dandab} \dontdisplaylinenum
            \var{\vb devarājā\lem  \mssCaCbCc\msNa\msNb\msNc; devarāja \Ed}%

kālena samatītāni kālo hi duratikramaḥ \veg\dontdisplaylinenum
            \paral{\textit{\vd {\normalfont  = MBh 12.220.41d = Garuḍapurāṇa 1.108.7 } }}

eṣa kālo mahāyogī brahmā viṣṇuḥ paraḥ śivaḥ\thinspace{\dandab} \dontdisplaylinenum
            \var{\va kālo\lem  \msCa\msCb\msNa; kāla \msCc\msNb\msNc\Ed}%
            \var{\vb brahmā viṣṇuḥ paraḥ\lem  \msCb; brahmaviṣṇuparaḥ \msCa\msNc, brahmā viṣṇu paraḥ \msCc\msNa\msNb,
                                                        brahmaviṣṇupara \Ed\ \unmetr}%

anādinidhano dhātā sa mahātmā namaskuru \veg\dontdisplaylinenum


\alalfejezet{parārdhādi}
vigatarāga uvāca~{\dandab}\dontdisplaylinenum 

śruta\.m vai kālacakra\.m tu mukhapadmaviniḥs\textsubring{r}tam\thinspace{\danda} \dontdisplaylinenum
            \var{\va °cakra\.m tu\lem  \msCa\msCb\msNa\msNb\msNc\Ed; °cakrasya \msCc}%
            \var{\vb viniḥs\textsubring{r}tam\lem  \corr; vinis\textsubring{r}tam \mssCaCbCc\msNa\msNb\msNc\Ed\ \unmetr}%

parārdha\.m ca para\.m caiva śrotu\.m vaḥ pratidīpitam \veg\dontdisplaylinenum
            \var{\vc parārdha\.m\lem  \msCb\msCc\msNa\msNb\msNc\Ed; \uncl{pa}rārddha\.m \msCa}%
            \var{\vd vaḥ\lem  \mssCaCbCc\msNa\msNb\msNc; yaḥ \Ed}%

anarthayajña uvāca~{\dandab}\dontdisplaylinenum 
            \var{\vo anarthayajña uvāca\lem  \mssCaCbCc\msNapcorr\msNb\msNc\Ed; \om\ \msNaacorr}%

eka\.m daśa\.m śata\.m caiva sahasram ayuta\.m tathā\thinspace{\danda} \dontdisplaylinenum
            \var{\vb °yuta\.m\lem  \mssCaCbCc\msNa\msNc\Ed; °tan \msNb}%

prayuta\.m niyuta\.m koṭim arbuda\.m v\textsubring{r}ndam eva ca \veg\dontdisplaylinenum
            \var{\vcd koṭim a°\lem  \mssCaCbCc\msNa\msNb\Ed; koṭir a° \msNc}%
            \var{\vd °rbuda\.m\lem  \mssCaCbCc\msNa\msNb\Ed; °buda\.m \msNc}%

kharva\.m caiva nikharva\.m ca śaṅkuḥ padma\.m tathaiva ca\thinspace{\dandab} \dontdisplaylinenum
            \var{\va ca\lem  \mssCaCbCc\msNa\msNc\Ed; tu \msNb}%
            \var{\vb śaṅkuḥ\lem  \corr; śaṅku \mssCaCbCc\msNa\msNb\msNc, śa\.mkha \Ed}%
            \paral{\textit{\vab {\normalfont  = Brahmāṇḍapurāṇa 3.2.101 } }}

samudro madhyam anta\.m ca parārdha\.m ca para\.m tathā \veg\dontdisplaylinenum
            \var{\vc madhyam anta\.m ca\lem  \mssCaCbCc\msNaacorr; madhyamānta\.m ca \msNapcorr, 
                        madhya\uncl{mantañ ca} \msNb, madhyam antaś ca \msNc, \om\ \Ed}%
            \var{\vcd \om\ \Ed}%

sarve daśaguṇā jñeyāḥ parārdha\.m yāvad eva hi\thinspace{\dandab} \dontdisplaylinenum
            \var{\vb parārdha\.m\lem  \msNc; parārdha \msCb\msCc\msNa\msNb,
                                         parā\uncl{rdha} \msCa, \om\ \Ed}%

parārdhadviguṇenaiva parasa\.mkhyā vidhīyate \veg\dontdisplaylinenum
            \var{\vo \om\ \Ed}%
            \var{\vc parārdha°\lem  \mssCaCbCc\msNa\msNb\Ed; parārdha\.m \msNc}%

parāt paratara\.m nāsti iti me niścitā matiḥ\thinspace{\dandab} \dontdisplaylinenum
            \var{\vab parāt paratara\.m nāsti iti me niścitā matiḥ\lem  \mssCaCbCc\msNb\msNcpcorr;
                parāt paratara\.m nāsti iti me niścitā mati \msNa\msNcacorr,
                v\textsubring{r}ndañ caiva mahāv\textsubring{r}nda dviparānantam eva ca{\danda}
                parāt paratara\.m nāsti iti me niścitā matiḥ{\danda}{\danda} \Ed}%

purāṇavedapaṭhitā mayākhyātā dvijottama \veg\dontdisplaylinenum
            \var{\ve °veda°\lem  \msCa\Ed; °vede \msCb\msCc\msNb\msNc\ \unmetr, °vedā \msNa}%
            \var{\vf °ākhyātā\lem  \msCa\msCb\msNa; °ākhyāta\.m \msCc\msNb\msNc\Ed}%


\alalfejezet{brahmāṇḍam}
vigatarāga uvāca~{\dandab}\dontdisplaylinenum 

brahmāṇḍa\.m kati vijñeya\.m pramāṇa\.m prāpita\.m kvacit\thinspace{\danda} \dontdisplaylinenum
            \var{\va brahmāṇḍa\.m\lem  \msCa\msCb\msNa\msNb\msNc\Ed; brahmāṇḍa \msCc}%
            \var{\vb pramāṇa\.m prāpita\.m\lem  \conj; pramāṇañ cāpita\.m \mssCaCbCc\msNa\msNb\Ed,
                                                                pramāñ cāpitat \msNc}%

kati cāṅgulimūrdheṣu sūryas tapati vai mahīm \veg\dontdisplaylinenum
            \var{\vd mahīm\lem  \msCb\msCc\msNa; mahī\uncl{m} \msCa, mahī \msNb\msNc\Ed}%

anarthayajña uvāca~{\dandab}\dontdisplaylinenum 

brahmāṇḍānā\.m prasa\.mkhyātu\.m mayā śakya\.m katha\.m dvija\thinspace{\danda} \dontdisplaylinenum
            \var{\va prasa\.mkhyātu\.m\lem  \mssCaCbCc\msNa\msNc; prasa\.msā tu \msNb, ca sa\.mkhyātu\.m \Ed}%
            \var{\vb śakya\.m\lem  \msNa\msNb\Ed; śakyā \mssCaCbCc\msNc}%

devās te 'pi na jānanti mānuṣāṇā\.m ca kā kathā \veg\dontdisplaylinenum

paryāyeṇa tu vakṣyāmi yathāśakya\.m dvijottama\thinspace{\dandab} \dontdisplaylinenum

brahmaṇā yat purākhyāto mātariśvā yathā tathā \veg\dontdisplaylinenum
            \var{\vc purā°\lem  \mssCaCbCc\msNa\msNb\msNc; mamā° \Ed}%
            \paral{\textit{\vcd {\normalfont cf. Brahmāṇḍapurāṇa 3.4.58cd: } 
                        brahmā dadau śāstram ida\.m purāṇa\.m mātariśvane}} 

śivāṇḍābhyantareṇaiva \crux{sarveṣām iva bhūritāḥ}\thinspace{\dandab} \dontdisplaylinenum
            \var{\vb °ṣām iva bhūritāḥ\lem  \msCa\msCb\msNc; °ṣām eva bhūritāḥ \msCc,
                                                        °ṣām iva bhūritā \msNa, °ṣām eva bhūriṇām \msNb,
                                                                         °ṣām eva bhūr imā\.m \Ed}%

daśanāma diśāṣṭānā\.m brahmāṇḍe kīrtita\.m ś\textsubring{r}ṇu \veg\dontdisplaylinenum
            \var{\vc diśā°\lem  \mssCaCbCc\msNa\msNc\Ed; śivā° \msNb}%
            \var{\vd kīrtita\.m ś\textsubring{r}ṇu\lem  \msCa\msCc\msNa\msNb\msNc\Ed; ya ca kīrtitam \msCb}%


\alalfejezet{daśa nāmāni digaṣṭakānām}
sahāsaha sahaḥ sahyo visahaḥ sa\.mhato 'sabhā\thinspace{\dandab} \dontdisplaylinenum
            \var{\va sahāsaha\lem  \mssCaCbCc\msNa\msNb\Ed; sahāsahaḥ \msNc\oo
                 sahyo\lem  \msCa\msCc\msNa\msNb\msNc; sa\uncl{hyo} \msCb, sajño \Ed}%
            \var{\vb visahaḥ\lem  \msCa\msCb\msNa\msNb\msNc\Ed; visaha \msCc\oo
                 sabhā\lem  \msCa\msCc\msNa\msNb\msNc; sabhāḥ \msCb, satā \Ed}%

prasaho 'prasahaḥ sānuḥ pūrvato daśa nāyakāḥ \veg\dontdisplaylinenum
            \var{\vc prasaho\lem  \mssCaCbCc\msNa\msNb\msNc; prasaheḥ \Ed\oo
                 prasahaḥ\lem  \msCa\msCb\msNa\msNb\msNc; prasa\uncl{vaḥ} \msCc, saprahaḥ \Ed\oo
                 sānuḥ\lem  \mssCaCbCc\msNa\msNb; sānu \msNc\Ed}%
            \var{\vd pūrvato\lem  \mssCaCbCc\msNa\msNb\msNc; parvato \Ed}%

prabhāso bhāsano bhānuḥ pradyoto dyutimo dyutiḥ\thinspace{\dandab} \dontdisplaylinenum
            \var{\va bhāsano\lem  \msCa\msCb\msNa\msNb\msNc; bhāsa{\lost} \msCc, bhāsato \Ed\oo
                 bhānuḥ\lem  \msCa\msCc\msNa\msNb\msNc\Ed; bhānu \msCb}%
            \var{\vb dyutimo\lem  \mssCaCbCc\msNa\msNb; dyutino \msNc\Ed}%

dīptatejāś ca tejāś ca tejā tejavaho daśa \veg\dontdisplaylinenum
            \var{\vc dīptatejā°\lem  \mssCaCbCc\msNa\msNb\msNc; dīptateja° \Ed\oo
                 tejāś ca\lem  \msCa\msCb\msNa\msNb\msNc\Ed; tejaś ca \msCc}%

āgneye tv etad ākhyāta\.m yāmye ś\textsubring{r}ṇv atha bho dvija\thinspace{\dandab} \dontdisplaylinenum
            \var{\va āgneye\lem  \mssCaCbCc\msNa\msNb\Ed; āgneya \msNc}%
            \var{\vb dvija\lem  \mssCaCbCc\msNa\msNc\Ed; dvijaḥ \msNb}%

yamo 'tha yamunā yāmaḥ sa\.myamo yamuno 'yamaḥ \veg\dontdisplaylinenum
            \var{\vc yamunā\lem  \mssCaCbCc\msNa\msNb\Ed; yamanā \msNc}%
            \var{\vd yamuno\lem  \msCa\msCb\msNb;
                        yamano \msCc\msNc, yumunā \msNa, yamunā° \Ed}%

sa\.myano yamanoyāno yaniyugmā yanoyanaḥ\thinspace{\dandab} \dontdisplaylinenum
            \var{\va sa\.myano\lem  \msNa; sa\.myamo \mssCaCbCc\msNb\Ed, sa\.myamā \msNc\oo
                 yamano°\lem  \msCa\msCc\msNa\msNc\Ed; yamuno° \msCb\msNb\oo
                 °yāno\lem  \mssCaCbCc\msNa\msNb\Ed; °yāmo \msNc}%
            \var{\vb yano yanaḥ\lem  \msNb; nayo yanaḥ \msCa\msCc\msNa, nayo nayaḥ \msCb,
                                nayo yamaḥ \msNc, nayonaya \Ed}%

nagajo naganā nando nagaro naga nandanaḥ \veg\dontdisplaylinenum
            \var{\vc naganā nando\lem  \msCa\msCc\msNa\msNb\msNc;
                                nagajā nando \msCb, nagano nado \Ed}%
            \var{\vd nagaro naganandanaḥ\lem  \msNb; nagaroraganandanaḥ \msCa\msNc,
                                   nagaro\uncl{nagananda}naḥ \msCb,
                                   naga\uncl{ro}{\lost}{\lost}nandanaḥ \msCc,
                                   nagarogaranandanaḥ \msNa, nagaronnaganandanaḥ \Ed}%

nagarbho gahano guhyo gūḍhajo daśa tatparaḥ\thinspace{\dandab} \dontdisplaylinenum
            \var{\va nagarbho\lem  \mssCaCbCc\msNa\msNc\Ed; n\textsubring{r}gabho \msNb\oo 
                 guhyo\lem  \mssCaCbCc\msNa\msNb\msNc; guhye \Ed}%

vāruṇena pravakṣyāmi ś\textsubring{r}ṇu vipra nibodha me \veg\dontdisplaylinenum
            \var{\vc vāruṇena\lem  \mssCaCbCc\msNa\msNb\msNc; vāruṇe ca \Ed}%
            \var{\vd ś\textsubring{r}ṇu\lem  \msNb; ś\textsubring{r}ṅge \msCa\msCb\msNa\msNc, ś\textsubring{r}\uncl{ṅge} \msCc, m\textsubring{r}ddhe \Ed}%

babhraḥ setur bhavodbhadraḥ prabhavodbhavabhājanaḥ\thinspace{\dandab} \dontdisplaylinenum
            \var{\va babhraḥ setur bha°\lem  babhra\.m setur bha° \msCa\msCb, babhra\.m setu bha° \msCc,
                      babhraḥ setu bha° \msNa, babhra\.m sotur bha° \msNb, babhra setur bha° \msNc,
                                        babhrūn satur bha° \Ed}%
            \var{\vb °bhājanaḥ\lem  \mssCaCbCc\msNa\msNb\msNc; °bhājana \Ed}%

bharaṇo bhuvano bhartā daśaite varuṇālayāḥ \veg\dontdisplaylinenum
            \var{\vc bharaṇo\lem  \msCb\msNc; bharaṇa \msCa\msNa, bharaṇā\.m \msCc\Ed,
                                        bharaṇā \msNb}%
            \var{\vd daśaite\lem  \mssCaCbCc\msNa\msNb\Ed; daśete \msNc\oo
                °ālayāḥ\lem  \mssCaCbCc\msNa\msNb\msNc; °ālayā \Ed}%

n\textsubring{r}garbho 'suragarbhaś ca devagarbho mahīdharaḥ\thinspace{\dandab} \dontdisplaylinenum
            \var{\va °garbhaś ca\lem  \msCa\msCb\msNb\msNc; °garbhāś ca \msCc\msNa\Ed}%

v\textsubring{r}ṣabho v\textsubring{r}ṣagarbhaś ca v\textsubring{r}ṣāṅko v\textsubring{r}ṣabhadhvajaḥ \veg\dontdisplaylinenum
            \var{\vc °garbhaś ca\lem  \mssCaCbCc\msNb\msNc\Ed; °garbhāś ca \msNa}%
            \var{\vd v\textsubring{r}ṣabha°\lem  \msCa\msCb\msNa\msNb\msNc\Ed; v\textsubring{r}ṣa{\il}° \msCc}%

jñātavyaś ca tathā samyag v\textsubring{r}ṣajo v\textsubring{r}ṣanandanaḥ\thinspace{\dandab} \dontdisplaylinenum
            \var{\va jñātavyaś\lem  \mssCaCbCc\msNa\msNb\msNc; jñānavāñ \Ed}%
            \var{\ab samyag v\textsubring{r}ṣajo\lem  \mssCaCbCc\msNb\msNc;
                        samyag \msNa, satyav\textsubring{r}ṣajo \Ed}%

nāyakā daśa vāyavye kīrtitā ye mayā dvija \veg\dontdisplaylinenum
            \var{\vd ye\lem  \mssCaCbCc\msNa\msNb\Ed; ya \msNc\oo
                 dvija\lem  \msCa\msCb\msNa\msNc\Ed; dvijaḥ \msCc\msNb}%

sulabhaḥ sumanaḥ saumyaḥ suprajaḥ sutanuḥ śivaḥ\thinspace{\dandab} \dontdisplaylinenum
            \var{\va sulabhaḥ\lem  \mssCaCbCc\msNa\msNb\msNc; surabhaḥ \Ed\oo
                 sumanaḥ\lem  \mssCaCbCc\msNa\msNb\Ed; sumanāḥ \msNc}%

sata satya layaḥ śambhur daśanāyakam uttare \veg\dontdisplaylinenum
            \var{\vc sata satya\lem  \mssCaCbCc\msNc; satyasatya \msNa, suta satya \msNb, sata satyā° \Ed\oo
                 layaḥ\lem  \mssCaCbCc\msNa\msNb\Ed; laya\.m \msNc}%
            \var{\vcd śambhur da°\lem  \msCa\msCb\msNb\Ed; śambhu da° \msCc\msNa\msNc}%
            \var{\vd °nāyakam u°\lem  \mssCaCbCc\msNa\msNb\msNc; °nāyaka u° \Ed}%

indu bindu bhuvo vajra varado vara varṣaṇaḥ\thinspace{\dandab} \dontdisplaylinenum
            \var{\vb °varṣaṇaḥ\lem  \mssCaCbCc\msNa\msNb; °{\il}\uncl{rśaṇam} \msNc, °daryya ca \Ed}%

ilano valino brahmā daśeśāneṣu nāyakāḥ \veg\dontdisplaylinenum
            \var{\vd daśe°\lem  \msCa\msNa\msNc\Ed; daśai° \msCb\msCc\msNb}%

aparo vimalo moho nirmalo mana mohanaḥ\thinspace{\dandab} \dontdisplaylinenum
            \var{\vb nirmalo ma°\lem  \eme; nimalo ma° \msCa, nirmalonma° \msCb\msNc,
                                                nirmalotma° \msCc\Ed, nimalorma° \msNa\msNb}%

akṣayaś cāvyayo viṣṇur varado madhyame daśa \veg\dontdisplaylinenum
            \var{\vc akṣayaś cā°\lem  \msCa\msCb\msNa\msNb\msNc; akṣayāś cā° \msCc, akṣayañ cā° \Ed}%
            \var{\vcd viṣṇur va°\lem  \msCa\msCb\msNc\Ed; viṣṇu va° \msCc\msNa, rviṣṇur va \msNb}%
            \var{\vd madhyame daśa\lem  \msCa\msCb\msNc; madhyamo daśa \msCc\msNa, 
                                        varavarṣaṇaḥ \msNb, madhyame daśaḥ \Ed}%

sarveṣā\.m daśam īśānā\.m parivāraśata\.m śata\.m\thinspace{\dandab} \dontdisplaylinenum
            \var{\va sarveṣā\.m\lem  \mssCaCbCc\msNa\msNb\Ed; sarveṣā \msNc\oo
                 daśam īśānā\.m\lem  \mssCaCbCc\msNa\msNb\msNc; daśarīśānā\.m \Ed}%
            \var{\vb parivāra°\lem  \msCa\msCc\msNb\msNc\Ed; pari° \msCb, parivāra\.m \msNa}%

śatānā\.m p\textsubring{r}thag ekaika\.m sahasraiḥ parivāritam \veg\dontdisplaylinenum
            \var{\vd °vāritam\lem  \msCa\msCb\msCcpcorr\msNa\msNb\msNc; °vāritā \msCcacorr, °vāritāḥ \Ed}%

sahasreṣu ca ekaikam ayutaiḥ parivāritam\thinspace{\dandab} \dontdisplaylinenum
            \var{\vab ekaikam a°\lem  \msCa\msCb\msNb\msNc\Ed; ekaika\.m ma° \msCc\msNa}%
            \var{\vb parivāritam\lem  \mssCaCbCc\msNa\msNb\msNc; parivāritamāḥ \Ed}%

ayuta\.m prayutair v\textsubring{r}ndaiḥ prayuta\.m niyutair v\textsubring{r}taḥ \veg\dontdisplaylinenum
            \var{\vc ayuta\.m\lem  \Ed; ayutaiḥ \mssCaCbCc\msNa\msNc, ayutai \msNb\oo
                 prayutair v\textsubring{r}°\lem  \mssCaCbCc\msNa\msNb\Ed; prayutai v\textsubring{r}° \msNc}%
            \var{\vd prayuta\.m niyutair v\textsubring{r}tam\lem  \eme; 
                        prayuta\.m niyutair v\textsubring{r}taḥ \Ed; prayutair niyutair v\textsubring{r}taḥ \msCa\msCb\msNa\msNc,
                        prayuter niyutair v\textsubring{r}taḥ \msCc, prayutai niyutai v\textsubring{r}taḥ \msNb}%

ekaikasya parīvāro niyutaḥ p\textsubring{r}thag eva ca\thinspace{\dandab} \dontdisplaylinenum
            \var{\va parīvāro\lem  \mssCaCbCc\msNa\msNb\msNc; parivāro \Ed}%
            \var{\vb niyutaḥ\lem  \msCa\msCb\msNa\msNb\msNc\Ed; niyuta \msCc\oo
                 ca\lem  \mssCaCbCc\msNa\msNb\msNcpcorr\Ed; caḥ \msNcacorr}%

koṭibhir daśakoṭyena ekaikaḥ parivāritaḥ \veg\dontdisplaylinenum
            \var{\vc koṭibhir da°\lem  \msCa\msCc\msNa\msNb\msNc\Ed; koṭibhi \msCb\oo
                 °koṭyena\lem  \mssCaCbCc\Ed; °koṭyona \msNa\msNc, °koṭyenaḥ \msNb}%
            \var{\vd ekaikaḥ\lem  \msCa\msCb\msNa\Ed; ekaika \msCc\msNb\msNc\oo
                 parivāritaḥ\lem  \msCb\msCc\msNa\msNb\msNc\Ed; parivāri\uncl{taḥ} \msCa}%

daśakoṭiṣu ekaika\.m v\textsubring{r}ndav\textsubring{r}ndabh\textsubring{r}tair v\textsubring{r}tam\thinspace{\dandab} \dontdisplaylinenum
            \var{\va °koṭiṣu\lem  \msCb\msCc\msNb\Ed; °koṭīṣu \msCa\msNa\msNc}%
            \var{\vb °v\textsubring{r}ndabh\textsubring{r}tair v\textsubring{r}tam\lem  \mssCaCbCc\msNb; °v\textsubring{r}ndav\textsubring{r}tair v\textsubring{r}ta\.m \msNa,
                                °v\textsubring{r}ndabh\textsubring{r}tai v\textsubring{r}ta\.m \msNc, °v\textsubring{r}nda\.m v\textsubring{r}tair v\textsubring{r}taḥ \Ed}%

v\textsubring{r}ndavargeṣu ekaika\.m kharvabhiḥ parivāritam \veg\dontdisplaylinenum
            \var{\vd kharvabhiḥ\lem  \mssCaCbCc\msNa\msNb\Ed; kharvarbhiḥ \msNc\oo
                 °vāritam\lem  \mssCaCbCc\msNa\msNb\msNc; °vāritaḥ \Ed}%

kharvavargeṣu ekaika\.m daśakharvagaṇair v\textsubring{r}tam\thinspace{\dandab} \dontdisplaylinenum
            \var{\vb °gaṇair v\textsubring{r}tam\lem  \msCa\msCc\msNa\msNb; °gaṇai v\textsubring{r}tam \msCb, °gaṇe v\textsubring{r}tta\.m \msNc, 
                                                                        °gaṇair v\textsubring{r}taḥ \Ed}%

daśakharveṣu ekaika\.m śaṅkubhiḥ parivāritam \veg\dontdisplaylinenum
            \var{\vc °kharveṣu\lem  \mssCaCbCc\msNa\msNb\Ed; °garveṣu \msNc}%
            \var{\vd °vāritam\lem  \mssCaCbCc\msNa\msNb\msNc; °vāritaḥ \Ed}%

śaṅkubhiḥ p\textsubring{r}thag ekaika\.m padmena parivāritam\thinspace{\dandab} \dontdisplaylinenum
            \var{\va p\textsubring{r}thag ekaika\.m\lem  \eme; p\textsubring{r}thag enaiva \msCa\msCc\msNa\msNb\msNc\Ed, p\textsubring{r}thag ainaiva \msCb}%
            \var{\vb °vāritam\lem  \msNapcorr; °vāritaḥ \mssCaCbCc\msNb\msNc\Ed, °ta\.m \msNaacorr}%

padmavargeṣu ekaika\.m samudraiḥ parivāritam \veg\dontdisplaylinenum
            \var{\vd samudraiḥ\lem  \msCc\msNa\msNb\msNc\Ed; samudaiḥ \msCa, damu\uncl{daiḥ} \msCb\oo
                 °vāritam\lem  \mssCaCbCc\msNa\msNb\msNc; °vāritaḥ \Ed}%

samudreṣu tathaikaika\.m madhyasa\.mkhyais tu tair v\textsubring{r}tam\thinspace{\dandab} \dontdisplaylinenum
            \var{\va tathai°\lem  \msCa\msCb\msNa\msNb\msNc\Ed; tathe° \msCc}%
            \var{\vb madhyasa\.mkhyais tu tair v\textsubring{r}tam\lem  \mssCaCbCc\msNa;
                        madhyasakhyais tu tai v\textsubring{r}tam \msNb,
                        madhyasakhyais tu ter v\textsubring{r}ta\.m \msNc,
                        madhye śaṅkhyāyutair v\textsubring{r}taḥ \Ed}%

madhyasa\.mkhyeṣu ekaikam anantaiḥ parivāritam \veg\dontdisplaylinenum
            \var{\vc madhyasa\.mkhyeṣu\lem  \mssCaCbCc\msNa\msNb\msNc; madhye śa\.mkheṣu \Ed}%
            \var{\vcd ekaikam anantaiḥ\lem  \mssCaCbCc\msNa\msNb\Ed; ekaika\.m manataiḥ \msNc}%
            \var{\vd °vāritam\lem  \mssCaCbCc\msNa\msNb\msNc; °vāritaḥ \Ed}%

ananteṣu ca ekaika\.m parārdhaparivāritam\thinspace{\dandab} \dontdisplaylinenum
            \var{\vb parārdhaparivāritam\lem  \msCa\msCb\msNa\msNb\msNc; parārdha{\lost}{\lost}{\lost}ritam \msCc,
                                                            parārdhaiḥ parivāritaḥ \Ed}%

parārdheṣu ca ekaika\.m pareṇa parivāritam \danda\dontdisplaylinenum
            \var{\vd °vāritam\lem  \mssCaCbCc\msNa\msNc; °vārivārita\.m \msNb, °vāritaḥ \Ed}%

eṣa vai kathito vipra śakya\.m sā\.mkhyam udīritam \veg\dontdisplaylinenum
            \var{\ve kathito\lem  \mssCaCbCc\msNa\msNc; \uncl{kathito} \msNb, kathitā \Ed}%
            \var{\vf śakya\.m\lem  \msCa\msCb\msNa\msNb\msNc\Ed; śakya \msCc\oo
                 sā\.mkhyam u°\lem  \msCa\msCc\msNb; sākhyam u° \msCb, syakhyam u° \msNa,
                                                sa\.mkhyam u \msNc, sa\.mkhyām u° \Ed}%


\alalfejezet{pramāṇam}
pramāṇa\.m ś\textsubring{r}ṇu me vipra sa\.mkṣepād bruvato mama\thinspace{\dandab} \dontdisplaylinenum
            \var{\va pramāṇa\.m\lem  \msCc\msNa\msNc\Ed; praṇāma\.m \msCa\msCb, pramāṇa \msNb}%
            \var{\vb sa\.mkṣepād bruvato\lem  \msCa\msCc\msNa\msNb\Ed; sa\.mkṣepād vadato \msCb,
                                        sa\.mkhyepād bruvato \msNc}%

candrodaye pūrṇamāsyā\.m vapur aṇḍasya tād\textsubring{r}śam \veg\dontdisplaylinenum

koṭikoṭisahasra\.m tu yojanānā\.m samantataḥ\thinspace{\dandab} \dontdisplaylinenum

aṇḍānā\.m ca parīmāṇa\.m brahmaṇā parikīrtitam \veg\dontdisplaylinenum
            \var{\va parī°\lem  \msCa\msCc\msNa\msNb\msNc\Ed; pari° \msCb\ \unmetr}%
            \var{\vb brahmaṇā\lem  \msCa\msCb\msNa\msNb\msNc\Ed; {\lost}{\lost}{\lost} \msCc\oo
                 °kīrtitam\lem  \msCa\msCb\msNb\msNc\Ed; °kīrti\uncl{tāḥ} \msCc, °kīrtitaḥ \msNa}%

saptakoṭisahasrāṇi saptakoṭiśatāni ca\thinspace{\dandab} \dontdisplaylinenum

vi\.mśakoṭiṣu gulmeṣu ūrdhvatas tapate raviḥ \veg\dontdisplaylinenum
            \var{\vd raviḥ\lem  \mssCaCbCc\msNa\msNc; ravi \Ed}%
            \paral{\textit{\vcd {\normalfont The folio in \msNb\ ends with } ūrdhva°, {\normalfont and the folios that 
                may have contained verses 1.62cd--2.22 are missing.}}}

pramāṇa\.m nāma sa\.mkhyā ca kīrtitāni samāsataḥ\thinspace{\dandab} \dontdisplaylinenum
            \var{\va pramāṇa\.m\lem  \msCa\msCc\msNa\msNc\Ed; praṇāma\.m \msCb}%

brahmāṇḍa\.m cāprameyāṇā\.m lakṣaṇa\.m parikīrtitam \veg\dontdisplaylinenum
            \var{\vc brahmāṇḍa\.m cā\lem  \msNa; brahmāṇḍaś ca \msCa\msCb\msNc, \uncl{brahmāṇḍāś cā}° \msCc
                 °meyāṇā\.m\lem  \msCa\msNa\Ed; °meyāṇā \msCb\msCc\msNc}%
            \var{\vd °kīrtitam\lem  \msCa\msCb\msNa\msNc\Ed; °kīrtitāḥ \msCc}%


\alalfejezet{vyāsāḥ}
purāṇāśīsahasrāṇi śatāni dvijasattama\thinspace{\dandab} \dontdisplaylinenum
            \var{\vb °sattama\lem  \msCa\msCb\msNa\msNc\Ed; {\lost}{\lost}maḥ  \msCc}%

brahmaṇā kathita\.m pūrṇa\.m mātariśvā yathātatham \veg\dontdisplaylinenum
            \var{\vc pūrṇa\.m\lem  \msCa\msCc\msNa\Ed; pūrve \msCb, pūrṇṇa \msNc}%
            \var{\vd °tatham\lem  \msCa\msCb\msNa\msNc\Ed; °tathā \msCc}%

vāyunā pāda sa\.mkṣipya prāpta\.m cośanasa\.m purā\thinspace{\dandab} \dontdisplaylinenum
            \var{\vb prāpta\.m cośanasa\.m\lem  \msCb\msNa\msNc; prāpta\.m causanasa\.m \msCa, prāpta{\il}ausanasa\.m \msCc,
                                                           prāptaś cośanasa\.m \Ed}%

tenāpi pāda sa\.mkṣipya prāptavā\.mś ca b\textsubring{r}haspatiḥ \veg\dontdisplaylinenum

b\textsubring{r}haspatis tu provāca sūrya\.m tri\.mśatsahasrikam\thinspace{\dandab} \dontdisplaylinenum
            \var{\vb sūrya\.m\lem  \msCc\Ed; sūryas \msCa\msNa\msNc, sūrya \msCb\oo
                 tri\.mśatsa°\lem  \msCa\msCb\msNa\msNc\Ed; tri\.mśasa° \msCc}%

pañcavi\.mśatsahasrāṇi m\textsubring{r}tyu\.m prāha divākaraḥ \veg\dontdisplaylinenum
            \var{\vc °vi\.mśatsahasrāṇi\lem  \corr; °vi\.mśahasrāṇi \msCa,
                                            °vi\.mśasahasrāṇi \msCb\msCc\msNa\msNc, °viśatsahasrāṇi \Ed}%

ekavi\.mśatsahasrāṇi m\textsubring{r}tyunendrāya kīrtitam\thinspace{\dandab} \dontdisplaylinenum 
            \var{\va °vi\.mśat°\lem  \Ed; °vi\.mśa° \mssCaCbCc\msNa\msNc}%
            \var{\vb kīrtitam\lem  \Ed; kīrtitaḥ \msCa\msCb\msNa\msNcpcorr, kīrtitāḥ \msCc, kīrttita \msNc}%

indreṇāha vasiṣṭhāya vi\.mśatślokasahasrikam \veg\dontdisplaylinenum
            \var{\vc vasiṣṭhāya\lem  \msCa\msCc\msNa\Ed; viśiṣṭhāya \msCb, vahiṣṭhāya \msNc}%
            \var{\vd vi\.mśatślo°\lem  \corr; vi\.mśaślo° \msCa\msCc\msNa\msNc\Ed, viśaślo° \msCb}%

aṣṭādaśasahasrāṇi tena sārasvatāya tu\thinspace{\dandab} \dontdisplaylinenum
            \var{\va aṣṭā°\lem  \mssCaCbCc\msNa\Ed; āṣṭā° \msNc}%

sārasvatas tridhāmāya sahasradaśa sapta ca \veg\dontdisplaylinenum
            \var{\vc sārasvatas tri°\lem  \eme; sārasvatā tri° \msCa\msCc\msNa\msNc\Ed, sārasvatās tri° \msCb\oo
                 °dhāmāya\lem  \mssCaCbCc\msNapcorr\msNc\Ed; \om\ \msNaacorr}%

ṣoḍaśānā\.m sahasrāṇi bharadvājāya vai tataḥ\thinspace{\dandab} \dontdisplaylinenum
            \var{\vb bhara°\lem  \msCa\msCb\msNa\msNc\Ed; bhāra° \msCc}%

daśa pañcasahasrāṇi triv\textsubring{r}ṣāya abhāṣata \veg\dontdisplaylinenum
            \var{\vd abhāṣata\lem  \msCa\msCb\msNa; a\uncl{bhāṣata} \msCc, abhāṣataḥ \msNc\Ed}%

caturdaśasahasrāṇi antarīkṣāya vai tataḥ\thinspace{\dandab} \dontdisplaylinenum

trayyāruṇi\.m sahasrāṇi trayodaśa abhāṣata \veg\dontdisplaylinenum
            \var{\vc trayyāruṇi\.m\lem  \corr; tryaiyāruṇi \msCa\msCb\msNa, traiyāruṇi \msCc\Ed, tryaiyārūpini \msNc}%
            \var{\vd abhāṣata\lem  \msCa\msCc\msNc; abhāṣataḥ \msCb, svabhāvata \msNa, hy abhāṣata \Ed}%

trayyāruṇis tu viprendro dhana\.mjayam abhāṣata\thinspace{\dandab} \dontdisplaylinenum
            \var{\va trayyāruṇi°\lem  \corr; tryaiyāruṇi° \mssCaCbCc\msNc, traiyāruṇi° \msNa\Ed\oo
                 viprendro\lem  \msCa\msCb\msNa\msNc\Ed; viprenda \msCc}%
            \var{\vb dhana\.mjaya°\lem  \mssCaCbCc\msNapcorr\msNc\Ed; dhana° \msNaacorr\oo
                 °bhāṣata\lem  \msCa\msCc\msNa\msNc; °bhāṣataḥ \msCb\Ed}%

dvādaśāni sahasrāṇi sa\.mkṣipya punar abravīt \veg\dontdisplaylinenum

k\textsubring{r}ta\.mjayāya samprāpto dhana\.mjayamahāmuniḥ\thinspace{\dandab} \dontdisplaylinenum

k\textsubring{r}ta\.mjayād dvijaśreṣṭha \textsubring{r}ṇa\.mjayamahātmane \veg\dontdisplaylinenum
            \var{\vc °jayād dvi°\lem  \msCa\msNa\Ed; °jayā dvi° \msCb\msCc\msNc\oo
                 °śreṣṭha\lem  \mssCaCbCc\msNa\msNc; °śreṣṭho \Ed}%
            \var{\vd \textsubring{r}ṇa\.mjaya°\lem  \msCa\msCc\msNa\msNc\Ed; \textsubring{r}ṇa\.mjāya° \msCb}%

\textsubring{r}ṇañjayāt punaḥ prāpto gautamāya maharṣiṇe\thinspace{\dandab} \dontdisplaylinenum
            \var{\va prāpto\lem  \mssCaCbCc\msNa\msNc; prāptau \Ed}%

gautamāc ca bharadvājas tasmād dharyadvatāya tu \veg\dontdisplaylinenum
            \var{\vc gautamāc ca\lem  \mssCaCbCc\msNa\Ed; gautamāś ca \msNc\oo
                 bharadvāja°\lem  \msCa\msCc\msNa\msNc\Ed; bharadvāra° \msCb}%
            \var{\vd tasmād dharyadvatāya\lem  \mssCaCbCc\msNa\msNc; tasmād damyāddamāya \Ed}%

rājaśravās tataḥ prāptaḥ somaśuṣmāya vai tataḥ\thinspace{\dandab} \dontdisplaylinenum
            \var{\va rājaśravās\lem  \eme; rājaśrava \mssCaCbCc\msNa\Ed, rājaśrava \msNc}%

somaśuṣmāt tataḥ prāptas t\textsubring{r}ṇabindus tu bho dvija \veg\dontdisplaylinenum
            \var{\vc °śuṣmāt ta°\lem  \mssCaCbCc\msNc\Ed; °śuṣmā ta° \msNa}%
            \var{\vcd prāptas t\textsubring{r}°\lem  \msCa\msCb\msNa\msNc\Ed; prā\uncl{pta t\textsubring{r}}° \msCc}%
            \var{\vd bho\lem  \msCa\msCc\msNa\msNc\Ed; \om\ \msCb}%

t\textsubring{r}ṇabindus tu v\textsubring{r}kṣāya v\textsubring{r}kṣaḥ śaktim abhāṣata\thinspace{\dandab} \dontdisplaylinenum
            \var{\vb °bhāṣata\lem  \msCa\msCb\msNa\msNc; °bhāṣataḥ \msCc\Ed}%

śaktiḥ parāśara\.m prāha jātūkarṇāya vai tataḥ \veg\dontdisplaylinenum
            \var{\vd jātū°\lem  \corr; jatu° \msCa\msCc\msNa\msNc\Ed, tu° \msCb}%

dvaipāyana\.m tu provāca jātūkarṇo maharṣiṇam\thinspace{\dandab} \dontdisplaylinenum
            \var{\va dvaipāyana\.m tu\lem  \eme; dvaipāyanas tu \mssCaCbCc\msNa\msNc, dvaipāyanāya \Ed}%
            \var{\vb jātūkarṇo maharṣiṇam\lem  \eme;
                         jatukarṇo maharṣiṇam \msCa\msCb\msNapcorr\msNc, jatukarṇā maharṣiṇaḥ \msCc, 
                         jakarṇo maharṣiṇa\.m \msNaacorr, jatukarṇamaharṣiṇā \Ed}%

romaharṣāya samprāpto dvaipāyanamahāmuniḥ \veg\dontdisplaylinenum
            \var{\vd °muniḥ\lem  \mssCaCbCc\msNa\msNc; °muni \Ed}%

romaharṣāya provāca putrāyāmitabuddhaye\thinspace{\dandab} \dontdisplaylinenum
            \var{\va °harṣāya\lem  \mssCaCbCc\msNa\msNc; °harṣaṇāya \Ed}%

daśadve ca sahasrāṇi purāṇa\.m samprakāśitam \danda\dontdisplaylinenum
            \var{\vb purāṇa\.m samprakāśitam\lem  \msCa\msCb\msNa\msNc\Ed; purāṇa samprakāśitā\.m \msCc}%

mānuṣāṇā\.m hitārthāya ki\.m bhūyaḥ śrotum icchasi \veg\dontdisplaylinenum
            \var{\ve mānuṣāṇā\.m\lem  \msCa\msCc\msNa\msNc\Ed; manuṣāṇā\.m \msCb}%
            \var{\vf bhūyaḥ\lem  \mssCaCbCc\msNa\msNc; bhūya \Ed\oo
                 °cchasi\lem  \msCa\msCb\msNa\msNc\Ed; °cchasīti \msCc}%


\jump
\begin{center}
\ketdanda iti v\textsubring{r}ṣasārasa\.mgrahe brahmāṇḍasa\.mkhyā nāmādhyāyaḥ prathamaḥ\ketdanda
\end{center}
\dontdisplaylinenum\vers 
            \var{{\normalfont  Colophon: } nāmādhyāyaḥ prathamaḥ\lem  \mssCaCbCc\msNa\msNc; 
                                nāma prathamo 'dhyāya \Ed}%
\bekveg\szamveg\vfill\phpspagebreak\szam\bek\versno=0\fejno=2
\thispagestyle{empty}



\alfejezet{\textbf{dvitīyo 'dhyāyaḥ}}\jump\jump
\vers

vigatarāga uvāca~{\dandab}\dontdisplaylinenum 

śruta\.m mayā janāgreṇa brahmāṇḍasya tu nirṇayam\thinspace{\danda} \dontdisplaylinenum
            \var{\va janāgreṇa\lem  \msCb\msCc\msNa\msNc\Ed; janā{\lost}{\lost} \msCa}%

pramāṇa\.m varṇarūpa\.m ca sa\.mkhyā tasya samāsataḥ \veg\dontdisplaylinenum
            \paral{\textit{{\normalfont Testimonia for this chapter: \msCa\ ff.\thinspace 195v--197r, 
                                             \msCb\ ff.\thinspace 203v--204v, 
                                             \msCc\ ff.\thinspace 270r--270v (it breaks off at 2.21 and resumes at 3.30b),
                                             \msNa\ ff.\thinspace 3v--4v, 
                                             \msNb\ exp.\thinspace 43 and 42 (sic!) (it broke off at 1.62d and resumes at 2.23),
                                             \msNc\ ff.\thinspace 211v--213r;
                \mssCaCbCc\ = \msCa + \msCb + \msCc }}}

śivāṇḍeti tvayā prokto brahmāṇḍālayakīrtitaḥ\thinspace{\dandab} \dontdisplaylinenum
            \var{\vb brahmāṇḍā°\lem  \mssCaCbCc\msNa\msNc; brahmāṇḍa \Ed}%

kīd\textsubring{r}śa\.m lakṣaṇa\.m jñeya\.m pramāṇa\.m tasya vā kati \veg\dontdisplaylinenum
            \var{\vc jñeya\.m\lem  \msCa\msCb\msNa\msNc\Ed; jñeyā \msCc}%
            \var{\vd kati\lem  \msCa\msCb\msNa\msNc\Ed; katiḥ \msCc}%

kasya vālayana\.m jñeya\.m pramāṇa\.m vātra vāsinaḥ\thinspace{\dandab} \dontdisplaylinenum
            \var{\va layana\.m jñeya\.m\lem  \msCa\msCc\msNa\msNc; layana\.m \msCb, lakṣaṇa\.m jñeya\.m \Ed}%
            \var{\vb vāsinaḥ\lem  \msCa\msCc\msNa\msNc\Ed; vāsirānaḥ \msCb}%

kā vā tatra prajā jñeyā ko vā tatra prajāpatiḥ \veg\dontdisplaylinenum
            \var{\vc kā\lem  \eme; ko \mssCaCbCc\msNa\msNc, ki\.m \Ed\oo
                 prajā jñeyā\lem  \msCb\msCc\msNa\msNc\Ed; pra\uncl{jā}{\lost}yā \msCa}%


\alalfejezet{śivāṇḍasa\.mkhyā}
anarthayajña uvāca~{\dandab}\dontdisplaylinenum 

śivāṇḍalakṣaṇa\.m vipra na tva\.m praṣṭum ihārhasi\thinspace{\danda} \dontdisplaylinenum
            \var{\vb na tva\.m\lem  \mssCaCbCc\msNa\msNc; tatva\.m \Ed\oo
                 °rhasi\lem  \mssCaCbCc\msNa\Ed; °hasi \msNc}%

daivatair api kā śaktir jñātu\.m draṣṭu\.m ca tattvataḥ \veg\dontdisplaylinenum
            \var{\vc daivatai°\lem  \msCa\msCb\msNa; devatai° \msCc\msNc\Ed\oo
                 śaktir\lem  \msCa; śakti \msCb\msCc\msNa\msNc\Ed}%

agamyagamana\.m guhya\.m guhyād api samuddhitam\thinspace{\dandab} \dontdisplaylinenum
            \var{\va agamyagamana\.m\lem  \msCa\msCb\msNa\Ed; agamyagagahana\.m \msCc, agamyagagamana\.m \msNc}%
            \var{\vb guhyā°\lem  \msNc\Ed; guhā° \mssCaCbCc\msNa\oo
                 samuddhitam\lem  \mssCaCbCc\msNa; samraddhita\.m \msNc, sam\textsubring{r}ddhidam \Ed}%
             \paral{\textit{\vab {\normalfont cf. Liṅgapurāṇa 1.21.71ab: } namo guṇyāya guhyāya agamyagamanāya ca}}

na prabhur netaras tatra na daṇḍyo na ca daṇḍakaḥ \veg\dontdisplaylinenum
            \var{\vc prabhur ne\lem  \msCa\msCb\msNa\msNc\Ed; prane° \msCc}%
            \var{\vd daṇḍyo\lem  \msCc\msNa\msNc; daṇḍo \msCa\msCb, daṇḍyā \Ed\oo
                 daṇḍakaḥ\lem  \msCa\msCc\msNa\msNc\Ed; ṇḍakaḥ \msCbacorr, paṇḍakaḥ \msCbpcorr}%

na satyo nān\textsubring{r}tas tatra suśīlo no duḥśīlavān\thinspace{\dandab} \dontdisplaylinenum
            \var{\va satyo\lem  \mssCaCbCc\msNa\msNc; satyau \Ed\oo
                 tatra\lem  \mssCaCbCc\msNa\msNc; tatrā \Ed}%
            \var{\vb no\lem  \msCb\msCc\msNa\msNc\Ed; {\lost} \msCa}%

nān\textsubring{r}jur na ca dambhitva\.m na t\textsubring{r}ṣṇā na ca īrṣyatā \veg\dontdisplaylinenum
            \var{\vc nān\textsubring{r}jur na\lem  \eme; nā\textsubring{r}jur nna \msCa\Ed, nā\textsubring{r}jur na \msCb\msNc, nā\textsubring{r}jun na \msNa,
                                                \uncl{nā\textsubring{r}ju na} \msCc}%
            \var{\vd na t\textsubring{r}ṣṇā na ca\lem  \mssCaCbCc\msNc\Ed;  na ca t\textsubring{r}ṣṇā na \msNa\oo
                īrṣyatā\lem  \msCa\msCb\msNa\msNc; īrṣyatāḥ \msCc, irṣyatā \Ed}%

na krodho na ca lobho 'sti na māno 'sti na sūyakaḥ\thinspace{\dandab} \dontdisplaylinenum
            \var{\va krodho\lem  \msCa\msCb\msNa\msNc\Ed; krodhau \msCc}%
            \var{\vb sūyakaḥ\lem  \msCa\msCc\msNa\msNc; sūcakaḥ \msCb, steyakaḥ \Ed\ \unmetr}%

īrṣyā dveṣo na tatrāsti na śaṭho na ca matsaraḥ \veg\dontdisplaylinenum
             \var{\vd śaṭho\lem  \msCa\msCb\msNa\msNc; ṣaṭho \msCc, śaṭhe \Ed\oo
                  matsaraḥ\lem  \mssCaCbCc\msNa\msNc; matsarāḥ \Ed}%

na vyādhir na jarā tatra na śoko 'sti na viklavaḥ\thinspace{\dandab} \dontdisplaylinenum
            \var{\va vyādhir na\lem  \msCa\msCb\msNa\Ed; vyādhi na \msCc\msNc\oo
                 jarā tatra\lem  \msCb\msNc; jarās tatra \msCa\msCc\msNa\Ed}%
            \var{\vb viklavaḥ\lem  \mssCaCbCc\msNa\msNc; viklava \Ed}%

nādhamaḥ puruṣas tatra nottamo na ca madhyamaḥ \veg\dontdisplaylinenum

notk\textsubring{r}ṣṭo mānavas tasmin striyaś caiva śivālaye\thinspace{\dandab} \dontdisplaylinenum
            \var{\va mānava°\lem  \msCb\msCc\msNa\msNc\Ed; mā{\lost}va° \msCa}%

na nindā na praśa\.msāsti matsarī piśuno na ca \veg\dontdisplaylinenum
            \var{\vc praśa\.msāsti\lem  \mssCaCbCc\msNa\msNc; praśa\.msāś ca \Ed}%

garvadarpa\.m na tatrāsti krūramāyādika\.m tathā\thinspace{\dandab} \dontdisplaylinenum

yācamāno na tatrāsti dātā caiva na vidyate \veg\dontdisplaylinenum
            \var{\vc tatrāsti\lem  \mssCaCbCc\msNapcorr\msNc\Ed; tatrā \msNaacorr}%

anarthī vraja tatrasthaḥ kalpav\textsubring{r}kṣasamāśritaḥ\thinspace{\dandab} \dontdisplaylinenum
            \var{\va vraja ta°\lem  \mssCaCbCc\msNa\Ed; vrajas ta° \msNc}%

na karma nāpriyas tatra na kaliḥ kalaho na ca \veg\dontdisplaylinenum
            \var{\vc karma nā°\lem  \eme; karma na \mssCaCbCc\msNa\msNc, karmaṇā \Ed}%
            \var{\vd kaliḥ\lem  \mssCaCbCc\msNa\msNcpcorr; kali \msNcacorr\Ed}%

dvāparo na ca na tretā k\textsubring{r}ta\.m cāpi na vidyate\thinspace{\dandab} \dontdisplaylinenum
            \var{\va ca na tretā\lem  \msCc\msNa\msNc\Ed; ca na tretrā \msCa, ca tretā na \msCb}%
            \var{\vb k\textsubring{r}ta\.m cā°\lem  \msCc\msNa; k\textsubring{r}taś cā° \msCa\msCb\msNc\Ed}%

manvantara\.m na tatrāsti kalpaś caiva na vidyate \veg\dontdisplaylinenum
            \var{\vc manvantara\.m na tatrāsti\lem  \msCa\msCb\msNa\Ed; 
                                manvantatrāsti \msCc,
                                manvantarananta tatrāsti \msNc}%
            \var{\vd kalpaś caiva\lem  \mssCaCbCc\msNc\Ed; kalpa\.m caiva \msNa}%

āhūtasamplava\.m nāsti brahmarātridina\.m tathā\thinspace{\dandab} \dontdisplaylinenum
            \var{\va āhūta°\lem  \mssCaCbCc\msNa\msNc; ābhūta° \Ed}%
            \var{\vb brahmarātridina\.m\lem  \mssCaCbCc\msNa\msNc; brahmarātridivas \Ed}%

na janmamaraṇa\.m tatra āpada\.m nāpnuyāt kvacit \veg\dontdisplaylinenum
            \var{\vc janmamaraṇa\.m tatra\lem  \msCc\msNa\Ed; janmaraṇa\.m tatra \msCa\msCb,
                                                janmamaraṇantrata \msNc}%
            \var{\vd āpada\.m\lem  \mssCaCbCc\msNa\msNc; apada\.m \Ed}%

na cāśāpāśabaddho 'sti rāgamoha\.m na vidyate\thinspace{\dandab} \dontdisplaylinenum
            \var{\va cāśāpāśa°\lem  \msCb\msNcpcorr; ca sāyāśa° \msCa\msCc\msNa\msNcacorr\Ed\oo 
                 °baddho\lem  \msCa\msCb\msNa\msNc; °ddho \msCc, °v\textsubring{r}ddho \Ed}%
            \var{\vb °moha\.m\lem  \msCb\msCc\msNa\msNc\Ed; °moho \msCa}%

na devā nāsurās tatra na yakṣoragarākṣasāḥ \veg\dontdisplaylinenum
            \var{\vc devā\lem  \msCa\msCc\msNa\msNc\Ed; devo \msCb}%

na bhūtā na piśācāś ca gandharvā \textsubring{r}ṣayas tathā\thinspace{\dandab} \dontdisplaylinenum
            \var{\vb gandharvā\lem  \mssCaCbCc\msNa\msNc;  gandharvo \Ed}%

tārā graha\.m na tatrāsti nāgaki\.mnaragāruḍam \veg\dontdisplaylinenum

na japo nāhnikas tatra nāgnihotrī na yajñak\textsubring{r}t\thinspace{\dandab} \dontdisplaylinenum
            \var{\va japo\lem  \msCb\msCc\msNa\msNc\Ed; jayo \msCa\oo
                 nāhnikas ta°\lem  \msCa\msCc\msNa\msNc\Ed; nāhnika ta° \msCb}%

na vrata\.m na tapaś caiva na tiryaṅnaraka\.m tathā \veg\dontdisplaylinenum
            \var{\vd na tiryaṅnaraka\.m\lem  \eme; nātiryannarakas \msCa\msCc\msNa,
                                 nātiryanarakan \msCb, nātrirya\.m narakas \msNc, na tīrthannarakan \Ed}%
            \paral{\textit{\vd {\normalfont Cf. 19.48cd: }viśiṣṭhe tv indriyagrāme tiryannarakasādhanam}}

tasyeśānasya devasya aiśvaryaguṇavistaram\thinspace{\dandab} \dontdisplaylinenum

api varṣaśatenāpi śakya\.m vaktu\.m na kenacit \veg\dontdisplaylinenum

harecchāprabhavāḥ sarve paryāyeṇa bravīmi te\thinspace{\dandab} \dontdisplaylinenum
            \var{\va harecchāprabhavāḥ\lem  \msNc; harecchaprabhavāḥ \mssCaCbCc\msNa, harecchāprabhavā \Ed}%

devamānuṣavarjyāni v\textsubring{r}kṣagulmalatādayaḥ \veg\dontdisplaylinenum
            \var{\vc varjyāni\lem  \mssCaCbCc\msNa\msNc; vajjñāni \Ed}%

parārdhadviguṇotsedhā vistāraś ca tathāvidhaḥ\thinspace{\dandab} \dontdisplaylinenum
            \var{\va °guṇotsedhā\lem  \conj; °guṇocchedhā \msCa\msCb\msNa\msNc, °guṇecchedhā \msCc, °guṇācchredhā \Ed}%
            \var{\vb vistāraś ca\lem  \msNc; vistāra\.m ca \mssCaCbCc\msNa\Ed\oo
                 °vidhaḥ\lem  \msNc; °vidhā \mssCaCbCc\msNa\Ed}%

anekākārapuṣpāṇi phalāni ca manoharam \veg\dontdisplaylinenum
            \var{\vc anekākāra°\lem  \msCb\msCc\msNa\msNc\Ed; anekāra° \msCa}%

anye kāñcanav\textsubring{r}kṣāṇi maṇiv\textsubring{r}kṣāṇy athāpare\thinspace{\dandab} \dontdisplaylinenum
            \var{\va anye\lem  \mssCaCbCc\msNa\msNc; bahu° \Ed}%

pravālamaṇiṣaṇḍāś ca padmarāgaruhāni ca \veg\dontdisplaylinenum
            \var{\vc ṣaṇḍāś ca\lem  \mssCaCbCc\msNa\msNc; ghaṇṭāś ca \Ed}%
            \var{\vd °ruhāni\lem  \msCa\msCb\msNa\msNc; °ruhāṇi \msCc, °sahāni \Ed}%

svādumūlaphalāskandalatāviṭapapādapāḥ\thinspace{\dandab} \dontdisplaylinenum
            \var{\va svādu°\lem  \msCb\msCc\msNa\msNc\Ed; svādhu° \msCa\oo
                 °mūla°\lem  \mssCaCbCc\msNc\Ed; °mūlā \msNa}%

kāmarūpāś ca te sarve kāmadāḥ kāmabhāṣiṇaḥ \veg\dontdisplaylinenum

tatra vipra prajāḥ sarve anantaguṇasāgarāḥ\thinspace{\dandab} \dontdisplaylinenum

tulyarūpabalāḥ sarve sūryāyutasamaprabhāḥ \veg\dontdisplaylinenum
            \var{\vc °bālāḥ\lem  \msCa\msCb\msNa\msNc; °varāḥ \Ed}%

parārdhadvayavistāra\.m parārdhadvayam āyatam\thinspace{\dandab} \dontdisplaylinenum

parārdhadvayavikṣepā yojanānā\.m dvijottama \veg\dontdisplaylinenum
            \var{\vc °dvaya°\lem  \msCa\msCb\msNapcorr\msNb\msNc\Ed; °dva° \msNaacorr\oo
                 vikṣepā\lem  \msCa\msCb\msNa\msNb\msNc; vijñeyā \Ed}%
            \var{\vd °ttama\lem  \msCa\msCb\msNb\msNc\Ed; °ttamaḥ \msNa}%

aiśvaryatva\.m na sa\.mkhyāsti balaśaktiś ca bho dvija\thinspace{\dandab} \dontdisplaylinenum
            \var{\vb balaśaktiś ca bho dvija\lem  \msCa\msCb\msNapcorr\msNb\msNc; 
                                        \om\ \msNaacorr, tava śaktiś ca bho dvija \Ed}%

adhordhvo na ca sa\.mkhyāsti na tiryañ caiti kaścana \veg\dontdisplaylinenum
            \var{\vc adhordhvo na ca sa\.mkhyāsti\lem  \msCa\msCb\msNapcorr\msNb\msNc\Ed; \om\ \msNaacorr}%
            \var{\vd na tiryañ caiti kaścana\lem  \msNapcorr\msNc;
                                                na tiryañ ceti kaścana \msCa\msCb\msNb\Ed,
                                                na tirya\.m ceti kaścana \msNaacorr}%

śivāṇḍasya ca vistāram āyāma\.m ca na vedmy aham\thinspace{\dandab} \dontdisplaylinenum

bhogam akṣayas tatraiva janmam\textsubring{r}tyur na vidyate \veg\dontdisplaylinenum
            \var{\vc bhogam akṣayas ta°\lem  \msCa\msCb\msNa\msNb\msNc; bhogamayās tu ta° \Ed}%
            \var{\vd °m\textsubring{r}tyur na\lem  \msCa\msCb\msNa\msNc\Ed; °m\textsubring{r}tyu na \msNb}%

śivāṇḍamadhyam āśritya gokṣīrasad\textsubring{r}śaprabhāḥ\thinspace{\dandab} \dontdisplaylinenum
            \var{\vb prabhāḥ\lem  \msCa\msCb\msNa\msNb\msNc; prabhā \Ed}%

parārdhaparakoṭīnām īśānānā\.m sm\textsubring{r}tālayaḥ \veg\dontdisplaylinenum
            \var{\vd °śānānā\.m\lem  \msCa\msCb\msNa\Ed; °śānānā \msNb, °gānānā\.m \msNc\oo
                 sm\textsubring{r}tālayaḥ\lem  \msCa\msNb\msNc; sm\textsubring{r}tālaya \msCb, sm\textsubring{r}tālaya\.m \msNa, sm\textsubring{r}tālayā \Ed}%

bālasūryaprabhāḥ sarve jñeyās tatpuruṣālaye\thinspace{\dandab} \dontdisplaylinenum
            \var{\va °bhāḥ\lem  \msCa\msCb\msNa\msNb\msNc; °bhā \Ed}%
            \var{\vb jñeyās ta°\lem  \msCa\msCb\msNb\msNc; jñeyā ta° \msNa\Ed\oo
                 °ālaye\lem  \msCa\msCb\msNa\msNb\msNc; °ālaya\.m \Ed}%

parārdhaparakoṭīnā\.m pūrvasyā\.m diśam āśritāḥ \veg\dontdisplaylinenum
            \var{\vd diśa°\lem  \msCa\msCb\msNa\msNc\Ed; diśi° \msNb}%

bhinnāñjanaprabhāḥ sarve dakṣiṇā\.m diśam āśritāḥ\thinspace{\dandab} \dontdisplaylinenum
            \var{\va °prabhāḥ\lem  \msCa\msCb\msNa\msNb\msNc; °prabhā \Ed}%
            \var{\vb dakṣiṇā\.m\lem  \msCa\msCb\msNa\msNb\msNc; dakṣiṇa° \Ed\oo
                 diśam\lem  \msCa\msNa\msNb\msNc; diśim \msCb\Ed}%

parārdhaparakoṭīnām aghorālayam āśritāḥ \veg\dontdisplaylinenum
            \var{\vd °ghorā°\lem  \msCa\msCb\msNa\msNb\msNc; °dhorā° \Ed\oo
                 °śritāḥ\lem  \msCa\msCb\msNa\msNb\msNc; °śritā \Ed}%

kundenduhimaśailābhāḥ paścimā\.m diśam āśritāḥ\thinspace{\dandab} \dontdisplaylinenum
            \var{\vb paścimā\.m\lem  \msCa\msNa\msNb\msNc\Ed; paścimā \msCb\oo
                 diśa°\lem  \msCa\msCb\msNa\msNb\Ed; diśi° \msNc\oo
                 °śritāḥ\lem  \msCa\msCb\msNa\msNb\msNc; °śritā \Ed}%

parārdhaparakoṭīnā\.m sadyamiṣṭālayaḥ sm\textsubring{r}taḥ \veg\dontdisplaylinenum
            \var{\vd sadyamiṣṭā°\lem  \msCa\msCb\msNb\msNc\Ed; sadyamiṣṭvā° \msNa\oo
                 sm\textsubring{r}taḥ\lem  \msCa\msNa\msNb\msNc\Ed; sm\textsubring{r}tāḥ \msCb}%

kuṅkumodakasa\.mkāśā uttarā\.m diśam āśritāḥ\thinspace{\dandab} \dontdisplaylinenum
            \var{\vb uttarā\.m\lem  \msCa\msNa\msNb\msNc\Ed; uttarā \msCb\oo
                 diśam\lem  \msCb\msNa\msNb\msNc\Ed; diśim \msCa}%

parārdhaparakotīnā\.m vāmadevālayaḥ sm\textsubring{r}taḥ \veg\dontdisplaylinenum
            \var{\vd °layaḥ\lem  \msCa\msCb\msNa\msNb\Ed; °laya \msNc}%

īśānasya kalāḥ pañca vaktrasyāpi catuṣ kalāḥ\thinspace{\dandab} \dontdisplaylinenum
            \var{\va kalāḥ\lem  \msCa\msCb\msNa\msNb\msNc; kalā \Ed}%
            \var{\vb catuṣ kalāḥ\lem  \msCa\msCb\msNa\msNb\msNc; catuṣtake \Ed}%

aghorasya kalā aṣṭau vāmadevās trayodaśa \veg\dontdisplaylinenum
            \var{\vd vāmadevā°\lem  \msCa\msCb\msNa\msNc\Ed; vāmadeva° \msNb}%

sadyaś cāṣṭau kalā jñeyāḥ sa\.msārārṇavatārakāḥ\thinspace{\dandab} \dontdisplaylinenum
            \var{\va jñeyāḥ\lem  \msCa\msCb\msNa\msNb\msNc; jñeyā \Ed}%
            \var{\vb sa\.msārā°\lem  \msCa\msCbpcorr\msNa\msNb\msNc\Ed; sa\.msā° \msCbacorr}%

aṣṭatri\.mśat kalā hy etāḥ kīrtitā dvijasattama \veg\dontdisplaylinenum
            \var{\vc °tri\.mśat ka°\lem  \corr; °tri\.mśaka° \msCa\msCb\msNa\msNb\msNc\Ed\oo
                 hy etāḥ\lem  \msCa\msCb\msNa\msNb\msNc; jñeyāḥ \Ed}%
            \var{\vd °sattama\lem  \msCa\msCb\msNa\msNc; °sattamaḥ \msNb\Ed}%

sa\.mkhyā varṇā diśaś caiva ekaikasya p\textsubring{r}thak p\textsubring{r}thak\thinspace{\dandab} \dontdisplaylinenum
            \var{\va sa\.mkhyā varṇā\lem  \msCb\msNc; sa\.mkhyā varṇṇo \msCa\msNb, sa\.mkhyā vaṇṇā \msNa, sa\.mdhyā varṇā \Ed}%
            \var{\vb ekaikasya\lem  \msCa\msNb\msNc\Ed; aikaikasya \msCb\msNa}%

pūrvoktena vidhānena bodhavyās tattvacintakaiḥ \veg\dontdisplaylinenum
            \var{\vd bodhavyās ta°\lem  \eme; bodhavyā ta° \msCa\msCb\msNa\msNb\msNc\Ed}%

śivāṇḍagamanāk\textsubring{r}ṣṭyā śivayoga\.m sadābhyaset\thinspace{\dandab} \dontdisplaylinenum
            \var{\va °k\textsubring{r}ṣṭyā\lem  \msCa\msCb\msNb\Ed; k\textsubring{r}ṣṭā \msNa\msNc}%
            \var{\vb yoga\.m sadābhyaset\lem  \msCa\msCb\msNa\msNc\Ed; yoga samabhyaset \msNb}%

śivayoga\.m vinā vipra tatra gantu\.m na śakyate \veg\dontdisplaylinenum
            \var{\vc °yoga\.m\lem  \msCa\msCb\msNa\msNb\msNc; °yoga \Ed}%

aśvamedhādiyajñānā\.m koṭyāyutaśatāni ca\thinspace{\dandab} \dontdisplaylinenum

k\textsubring{r}cchrāditapa sarvāṇi k\textsubring{r}tvā kalpaśatāni ca \danda\dontdisplaylinenum
            \var{\vc °tapa\lem  \Ed; °tapaḥ \msCa\msCb\msNa\msNb\msNc\ \unmetr}%

tatra gantu\.m na śakyeta devair api tapodhana \veg\dontdisplaylinenum
            \var{\ve śakyeta\lem  \msCa\msNa\msNb\msNc; śakyaita \msCb, śakyete \Ed}%
            \var{\vf devai°\lem  \msCa\msCb\msNa\msNb\Ed; deve° \msNc\oo
                 °dhana\lem  \msCa\msNa\msNb\msNc\Ed; °dhanam \msCb}%

gaṅgādisarvatīrtheṣu snātvā taptvā ca vai punaḥ\thinspace{\dandab} \dontdisplaylinenum

tatra gantu\.m na śakyeta \textsubring{r}ṣibhir vā mahātmabhiḥ \veg\dontdisplaylinenum
            \var{\va gantu\.m\lem  \msCa\msCb\msNa\Ed; gantu \msNb\msNc\oo
                 śakyeta\lem  \msCa\msCb\msNa\msNb\msNc; śakyante \Ed}%

saptadvīpasamudrāṇi ratnapūrṇāni bho dvija\thinspace{\dandab} \dontdisplaylinenum
            \var{\va °dvīpa°\lem  \msCa\msCb\msNa\msNb\Ed; °dīpa° \msNc\oo
                 °samudrāṇi\lem  \msCa\msCb\msNa\msNc\Ed; °samudrāya \msNb}%
            \paral{\textit{\vab {\normalfont Cf. ŚDhU 2.104: } triḥ pradatvā mahī\.m pūrṇā\.m{\normalfont ...}}}

dattvā vā vedaviduṣe śraddhābhaktisamanvitaḥ \danda\dontdisplaylinenum

tatra gantu\.m na śakyeta vinā dhyānena niścayaḥ \veg\dontdisplaylinenum
            \var{\vc gantu\.m\lem  \msCa\msCb\msNa\Ed; gantu \msNb, ga\.mntu \msNc\oo
                 śakyeta\lem  \msCa\msCb\msNa\msNb\msNc; śakyante \Ed}%

svadehān mā\.msam uddh\textsubring{r}tya dattvārthibhyaś ca niścayāt\thinspace{\dandab} \dontdisplaylinenum
            \var{\va svadehān mā\.msa°\lem  \msCa\msCb\msNa\msNb; svadehāt mā\.msa° \msNc, svadehātmā\.m sa° \Ed}%

svadāraputrasarvasva\.m śiro 'rthibhyaś ca yo dadet \danda\dontdisplaylinenum
            \var{\va °sva\.m\lem  \msCa\msCb\msNa\msNc\Ed; °sva \msNb}%

na tatra gantu\.m śakyeta anyair vāpi suduṣkaraiḥ \veg\dontdisplaylinenum
            \var{\ve na tatra gantu\.m\lem  \msCa\msNa\msNb\msNc\Ed; na tatra gantu\.m na \msCb}%
            \var{\vf °duṣkaraiḥ\lem  \msCa\msCb\msNa\msNc\Ed; °duṣk\textsubring{r}taḥ \msNb}%

yajñatīrthatapodānavedādhyayanapāragaḥ\thinspace{\dandab} \dontdisplaylinenum
            \var{\vc °dāna°\lem  \msCa\msCb\msNc\Ed; °dāna\.m \msNa, °dānai \msNb}%
            \var{\vd °pāragaḥ\lem  \msCb\msNa\msNc\Ed; °pāragāḥ \msCa\msNb}%

brahmāṇḍāntasya bhogā\.ms tu bhuṅkte kālavaśānugaḥ \veg\dontdisplaylinenum
            \var{\va brahmāṇḍāntasya bhogā\.ms tu\lem  \msCa\msCb\msNa\msNc; 
                                        brahmāṇḍāntasya bhogās tu \msNb,
                                        brahmāṇḍāt tasya bhogās tu \Ed}%
            \var{\vb bhuṅkte\lem  \msCa\msCb\msNa\msNb; \uncl{bhuṅkte} \msNc, bhuktvā \Ed\oo
                 °gaḥ\lem  \msCa\msCb\msNapcorr\msNb\msNc\Ed; °gāḥ \msNaacorr}%

kālena samapreṣyeṇa dharmo yāti parikṣayam\thinspace{\dandab} \dontdisplaylinenum
            \var{\vb dharmo\lem  \msCa\msCb\msNa\msNb\Ed; dharme \msNc}%

alātacakravat sarva\.m kālo yāti paribhraman \danda\dontdisplaylinenum

traikālyakalanāt kālas tena kālaḥ prakīrtitaḥ \veg\dontdisplaylinenum
            \var{\ve °kalanāt kāla°\lem  \msCa\msCb\msNa\msNc\Ed; °kalanā kāla° \msNb}%


\jump
\begin{center}
\ketdanda iti v\textsubring{r}ṣasārasa\.mgrahe śivāṇḍasa\.mkhyā nāmādhyāyo dvitīyaḥ\ketdanda
\end{center}
\dontdisplaylinenum\vers 
            \var{{\normalfont Colophon:} nāmādhyāyo dvitīyaḥ\lem  \msCa\msCb\msNa\msNc;
                                                                nāmādhyāya dvitīyaḥ \msNb,
                                                                nāma dvitīyo 'dhyāyaḥ \Ed}%
\bekveg\szamveg\vfill\phpspagebreak\szam\bek\versno=0\fejno=3
\thispagestyle{empty}



\alfejezet{\textbf{t\textsubring{r}tīyo 'dhyāyaḥ}}\jump\jump

\alalfejezet{dharmapravacanam}
\vers

vigatarāga uvāca~{\dandab}\dontdisplaylinenum 

kimartha\.m dharmam ity āhuḥ katimūrtiś ca kīrtyate\thinspace{\danda} \dontdisplaylinenum
            \var{\va āhuḥ\lem  \msCa\msCb\msNa\msNb\msNc; āhu \Ed}%

katipādav\textsubring{r}ṣo jñeyo gatis tasya kati sm\textsubring{r}tāḥ \veg\dontdisplaylinenum
            \var{\vd sm\textsubring{r}tāḥ\lem  \msCa\msNa\msNb\msNc; sm\textsubring{r}tā \msCb, sm\textsubring{r}taḥ \Ed}%
            \paral{\textit{{\normalfont Testimonia for this chapter: \msCa\ ff.\thinspace 197r--198v, 
                                             \msCb\ ff.\thinspace 204v--206r, 
                                             \msCc\ ff.\thinspace 273r--273v (it broke off at 2.21 and resumes at 3.30b; f. 272 is missing),
                                             \msNa\ ff.\thinspace 4v--6r, 
                                             \msNb\ exp.\thinspace 42, 47--48 (sic!),
                                             \msNc\ ff.\thinspace 213r--214v;
                                \mssCaCbCc\ = \msCa + \msCb + \msCc }}}

kautūhala\.m mamotpanna\.m sa\.mśaya\.m chindhi tattvataḥ\thinspace{\dandab} \dontdisplaylinenum
            \var{\va kautūhala\.m\lem  \msCa\msCb\msNa\msNb\msNc; kautuhala \Ed\oo
                 mamotpanna\.m\lem  \msCa\msCb\msNa\msNb\Ed; samotpanna\.m \msNc}%
            \var{\vb sa\.mśaya\.m\lem  \msCb\msNa\msNb\msNc\Ed; saśaya\.m \msCa}%

kasya putro muniśreṣṭha prajās tasya kati sm\textsubring{r}tāḥ \veg\dontdisplaylinenum

anarthayajña uvāca~{\dandab}\dontdisplaylinenum 

dh\textsubring{r}tir ity eṣa dhātur vai paryāyaḥ parikīrtitaḥ\thinspace{\danda} \dontdisplaylinenum

ādhāraṇān mahattvāc ca dharma ity abhidhīyate \veg\dontdisplaylinenum 
            \var{\vc ādhāraṇān ma°\lem  \msCa\msNb; ādhāraṇāt pa° \msCb, ādhāraṇāt ma° \msNa\msNc, ādhāreṇa ma° \Ed}%
            \var{\vd °bhidhīyate\lem  \msCa\msNa\msNc\Ed; °vidhīyate \msCb\msNb}%
            \paral{\textit{\vo\ \kb\ {\normalfont Matsyapurāṇa 145.27: }  dharmeti dhāraṇe dhātur mahatve caiva ucyate{\thinspace\danda}
                                                  ādhāraṇe mahattve vā dharmaḥ sa tu nirucyate{\thinspace\danda}}}

śrutism\textsubring{r}tidvayor mūrtiś catuṣpādav\textsubring{r}ṣaḥ sthitaḥ\thinspace{\dandab} \dontdisplaylinenum
            \var{\vab °sm\textsubring{r}tidvayor mūrtiś ca°\lem  \msCa; °sm\textsubring{r}tidvayo mūrttiś ca° \msCb\msNb, °sm\textsubring{r}tidvayo mūrtti ca° \msNa\msNc, 
                                                                        °sm\textsubring{r}tir dvayo mūrtiś ca \Ed}%
            \var{\vb °v\textsubring{r}ṣaḥ\lem  \msCa\msCb\msNa\msNb\Ed; °v\textsubring{r}ṣa \msNc}%

caturāśrama yo dharmaḥ kīrtitāni manīṣibhiḥ \veg\dontdisplaylinenum
            \var{\vc caturā°\lem  \msCb\msNa\msNb\Ed; cāturā° \msCa\msNc}%

gatiś ca pañca vijñeyāḥ ś\textsubring{r}ṇu dharmasya bho dvija\thinspace{\dandab} \dontdisplaylinenum
            \var{\va vijñeyāḥ\lem  \eme; vijñeyaḥ \msCa\msNa\msNb\msNc\Ed, \om\ \msCb}%
            \paral{\textit{\vab {\normalfont \msCb\ reads here } gatiś ca pautrāś ca anekāś ca babhūva ha,
                        {\normalfont skipping to 3.7cd, omitting 3.5--7ab.}}}

devamānuṣatirya\.m ca narakasthāvarādayaḥ \veg\dontdisplaylinenum

brahmaṇo h\textsubring{r}daya\.m bhittvā jāto dharmaḥ sanātanaḥ\thinspace{\dandab} \dontdisplaylinenum
            \var{\va brahmaṇo\lem  \msCa\msNa\msNb\msNc; \om\ \msCb, brāhmaṇo \Ed\oo
                 bhittvā\lem  \msCa\msCb\msNa\msNc\Ed; vittvā \msNb}%
            \var{\vb dharmaḥ\lem  \msCa\msCb\msNa\msNc\Ed; dharma \msNb}%
            \paral{\textit{\vab {\normalfont cf.\ Devīpurāṇa 4.59cd: } brahmaṇo h\textsubring{r}dayāj jātaḥ putro dharma iti sm\textsubring{r}taḥ \oo 
                    {\normalfont cf. also MBh 1.60.40ab: } brahmaṇo h\textsubring{r}daya\.m bhittvā niḥs\textsubring{r}to bhagavān bh\textsubring{r}guḥ}}

tasya patnī mahābhāgā trayodaśa sumadhyamāḥ \veg\dontdisplaylinenum
            \var{\vd °madhyamāḥ\lem  \msCa\msNa\msNb\msNc\Ed, \om\ \msCb}%

dakṣakanyā viśālākṣī śraddhādyāḥ sumanoharāḥ\thinspace{\dandab} \dontdisplaylinenum
            \var{\va °ākṣī\lem  \msCa\msNa\msNb\msNc; \om\ \msCb, °ākṣi \Ed}%
            \var{\vb °ādyāḥ\lem  \eme; °ādyā \msNb\msNc\Ed, °āḍhyāḥ \msNa, \om\ \msCb, °āḍhyā \msCa\oo
                 °harāḥ\lem  \msNb\Ed; °harā \msCa\msNc,  \om\ \msCb, °{\il}\uncl{mā}ḥ \msNa}%

tasya putrāś ca pautrāś ca anekāś ca babhūva ha \danda\dontdisplaylinenum
            \var{\vcd tasya putrāś ca pautrāś ca anekāś ca babhūva ha\lem  \msCa\msNb;
                gatiś ca pautrāś ca anekāś ca babhūva ha \msCb  \eyeskip{from 3.5a},
                tasya putrāś ca yotrāś ca anekāś ca babhūva ha \msNa\msNc,
                tasya putrā anekāś ca tathā pautrā babhūvahaḥ \Ed}%

eṣa dharmanisargo 'ya\.m ki\.m bhūyaḥ śrotum icchasi \veg\dontdisplaylinenum

vigatarāga uvāca~{\dandab}\dontdisplaylinenum 
            \var{\vo vigatarāga uvāca\lem  \msCb\msNapcorr\msNc\Ed; vigatarāga u \msCa\msNb, \om\ \msNaacorr}%

dharmapatnī viśeṣeṇa putras tebhyaḥ p\textsubring{r}thak p\textsubring{r}thak\thinspace{\danda} \dontdisplaylinenum

śrotum icchāmi tattvena kathayasva tapodhana \veg\dontdisplaylinenum

anarthayajña uvāca~{\dandab}\dontdisplaylinenum 

śraddhā lakṣmīr dh\textsubring{r}tis tuṣṭiḥ puṣṭir medhā kriyā lajjā\thinspace{\danda} \dontdisplaylinenum
            \var{\va lakṣmīr dh\textsubring{r}tis tuṣṭiḥ\lem  \msCa; 
                                lakṣmīr dh\textsubring{r}tis tuṣ \msCb,  
                                lakṣmī ddh\textsubring{r}tir ddh\textsubring{r}tis tuṣṭiḥ \msNaacorr, 
                                lakṣmīr ddh\textsubring{r}tis tuṣṭiḥ \msNapcorr, 
                                lakṣmī\.m dh\textsubring{r}ti tuṣṭiḥ \msNb,
                                lakṣmī dh\textsubring{r}tis tuṣṭiḥ \msNc,
                                lakṣmī dh\textsubring{r}tis tuṣṭī \Ed}%
            \var{\vb puṣṭir me°\lem  \msCa\msCb\msNa\msNb\msNc; puṣṭi me° \Ed\oo
                 lajjā\lem  \msCa\msCb\msNb\msNc\Ed; lajā \msNa}%

buddhiḥ śāntir vapuḥ kīrtiḥ siddhiḥ prasūtisambhavāḥ \veg\dontdisplaylinenum
            \var{\vc buddhiḥ\lem  \msCb\msNa\msNb\msNc\Ed; buddhi \msCa}%
            \var{\vd siddhiḥ prasūtisambhavāḥ\lem  \conj; siddhiś cābhūtisambhavā \msCa\msNa\msNb\msNc, 
                                        siddhiś cātisambhavā \msCb, siddhiś ca bhūtisambhavā \Ed}%

śraddhā kāmaḥ suto jāto darpo lakṣmīsutaḥ sm\textsubring{r}taḥ\thinspace{\dandab} \dontdisplaylinenum
            \var{\va kāmaḥ\lem  \msNa; kāma° \msCa\msCb\msNb\msNc, dharma° \Ed}%

dh\textsubring{r}tyās tu niyamaḥ putraḥ sa\.mtoṣas tuṣṭijaḥ sm\textsubring{r}taḥ \veg\dontdisplaylinenum
            \paral{\textit{\vo {\normalfont For 3.10--13, see a rather similar 
        passage e.g.\ in Kūrmapurāṇa 1.8.20 ff.: }
        śraddhāyā ātmajaḥ kāmo darpo lakṣmīsutaḥ sm\textsubring{r}taḥ{\thinspace\danda}
        dh\textsubring{r}tyās tu niyamaḥ putras tuṣṭyāḥ sa\.mtoṣa ucyate{\thinspace\ketdanda} 
        puṣṭyā lābhaḥ sutaś cāpi medhāputraḥ śrutas tathā{\thinspace\danda} 
        kriyāyāś cābhavat putro daṇḍaḥ samaya eva ca{\thinspace\ketdanda}  
        buddhyā bodhaḥ sutas tadvad apramādo vyajāyata{\thinspace\danda} 
        lajjāyā vinayaḥ putro vapuṣo vyavasāyakaḥ{\thinspace\ketdanda}  
        kṣemaḥ śāntisutaś cāpi sukha\.m siddhir ajāyata{\thinspace\danda}
        yaśaḥ kīrtisutas tadvad ity ete dharmasūnavaḥ{\thinspace\ketdanda}   
        kāmasya harṣaḥ putro 'bhūd devānando vyajāyata{\thinspace\danda}  
        ity eṣa vai sukhodarkaḥ sargo dharmasya kīrtitaḥ{\thinspace\ketdanda}}}

puṣṭyā lābhaḥ suto jāto medhāputraḥ śrutas tathā\thinspace{\dandab} \dontdisplaylinenum
            \var{\va lābhaḥ\lem  \msCa\msCb\msNb\msNc; lābha° \msNa\Ed}%
            \var{\vb °putraḥ\lem  \eme; °putra \msCa\msCb\msNa\msNb\msNc\Ed\oo
                 śruta°\lem  \msCa\msNa\msNb\msNc\Ed; śrata° \msCb}%

kriyāyās tv abhayaḥ putro daṇḍaḥ samaya eva ca \veg\dontdisplaylinenum
            \var{\vc tv abhayaḥ putro\lem  \msCa\msCb\msNa\msNb\msNc; tūbhayaḥ putrau \Ed}%
            \var{\vd daṇḍaḥ\lem  \corr; daṇḍe \msCa\msNaacorr, daṇḍo \msCb, daṇḍa° \msNapcorr\msNb\msNc\Ed\oo
                 ca\lem  \msCa\msCb\msNa\msNb\msNc; tu \Ed}%

lajjāyā vinayaḥ putro buddhyā bodhaḥ sutaḥ sm\textsubring{r}taḥ\thinspace{\dandab} \dontdisplaylinenum
            \var{\va lajjāyā vinayaḥ\lem  \msCa\msCb\msNa\msNb\msNc; lajjāyāḥ vinaya° \Ed}%
            \var{\vb sutaḥ sm\textsubring{r}taḥ\lem  \msNa\Ed\msNb\msNc; sutaḥ {\il}{\il} \msCa, sutaḥs tathā \msCb}%

lajjāyāḥ sudhiyaḥ putra apramādaś ca tāv ubhau \veg\dontdisplaylinenum
            \var{\vc sudhiyaḥ\lem  \Ed; sudhiya \msCa\msCb\msNa\msNb\msNc\oo
                 putra\lem  \msCa\msCb\msNa\msNb\msNc; putraḥ \Ed}%
            \var{\vd apramāda°\lem  \msCa\msCb\Ed\msNb\msNc; apramādā° \msNa}%

kṣemaḥ śāntisuto vindyād vyavasāyo vapoḥ sutaḥ\thinspace{\dandab} \dontdisplaylinenum
            \var{\vb vapoḥ\lem  \msCa\msCb\msNb\msNc\Ed; vapo \msNa}%

yaśaḥ kīrtisuto jñeyaḥ sukha\.m siddher vyajāyata \danda\dontdisplaylinenum
            \var{\vd siddhe°\lem  \msCb\msNa\msNb; siddhi \msCa\msNc\Ed\oo
                 vyajāyata\lem  \msCa\msCb\msNa; vyajāyate \msNb\Ed, vyajāyataḥ \msNc}%

svāyambhuve 'ntare tv āsan kīrtitā dharmasūnavaḥ \veg\dontdisplaylinenum
            \var{\ve svāyambhuve\lem  \msCa\msNa\msNc; svāyambhuvo \msCb, svayambhuve \msNb\Ed\oo
                 'ntare tv āsan\lem  \conj; 'ntare tvāsi \msCa\msCb\msNa, 
                                'ntare tv āsīt \msNb, 'ntare tv āsa\.m \msNc, 'ntar evāsi \Ed}%

vigatarāga uvāca~{\dandab}\dontdisplaylinenum 

mūrtidvaya\.m katha\.m dharma\.m kathayasva tapodhana\thinspace{\danda} \dontdisplaylinenum
            \var{\va dharma\.m\lem  \msCa\msCb\msNa\msNb; ddharma \msNc, dharmaḥ \Ed}%

kautūhalam atīva\.m me kartaya jñānasa\.mśayam \veg\dontdisplaylinenum
            \var{\vc kautūhala°\lem  \msCa\msNa\msNb\msNc\Ed; kotūhala° \msCb\oo
                 °tīva\.m me\lem  \msCa\msNa\msNb\msNc\Ed; °tīva me \msCb}%
            \var{\vd kartaya\lem  \eme; kīrtaya \msCa\msCb\msNa\msNb\msNc\Ed\oo
                 °sa\.mśayam\lem  \msCa\msNa\msNc\Ed; °sa\.mśayaḥ \msCb\msNb}%

anarthayajña uvāca~{\dandab}\dontdisplaylinenum 

śrutism\textsubring{r}tidvayī mūrtir dharmasya parikīrtitā\thinspace{\danda} \dontdisplaylinenum
            \var{\va śruti°\lem  \msCa\msNa\msNb\msNc; śrutiḥ \msCb\Ed\oo
                 dvayī mūrti\lem  \msNc; °dvayor mūrttir \msCa, °dvayor mūrti \msCb\Ed, dvayo mūrtti \msNa\msNb}%
            \var{\vb °kīrtitā\lem  \msCa\msCb\msNa\Ed; °kīrttitaḥ \msNb, kīrttitāḥ \msNc}%

dārāgnihotrasambandham ijyā śrautasya lakṣaṇam \danda\dontdisplaylinenum
            \var{\vcd °bandham i°\lem  \eme; °baddha i° \msCa\msCb\msNa\msNc, °bandha i° \msNb\Ed}%
            \var{\vd śrautasya\lem  \eme; śrotasya \msCa\msCb\msNc, śrautrasya \msNa, srotrasya \msNb, śrutasya \Ed}%
            \paral{\textit{\vcd {\normalfont cf.\ Manu 3.171ab: }dārāgnihotrasa\.myoga\.m kurute yo 'graje sthite; 
               {\normalfont and also Matsyapurāṇa 142.41: } 
                        dārāgnihotrasambandham \textsubring{r}gyajuḥsāmasa\.mhitāḥ{\thinspace\danda}
                        ityādibahula\.m śrauta\.m dharma\.m saptarṣayo 'bruvan{\thinspace\ketdanda}}}

smārto varṇāśramācāro yamaiś ca niyamair yutaḥ \veg\dontdisplaylinenum
            \var{\ve smārto\lem  \eme; smārta \msCa\msCb\msNa\msNb\msNc\Ed}%
            \paral{\textit{\vcdef\ \kb\ {\normalfont Matsyapurāṇa 145.31: }dārāgnihotrasambandham ijyā śrautasya lakṣaṇam{\thinspace\danda}
                                smārto varṇāśramācāro yamaiś ca niyamair yutaḥ{\thinspace\ketdanda}
        {\normalfont  \kb\ \MBh\ Indeces 1.36.10: }dānāgnihotram ijyā ca śrautasyaitad dhi lakṣaṇam{\thinspace\danda}
                                       smārto varṇāśramācāro yamaiś ca niyamair yutaḥ{\thinspace\ketdanda} }}


\alalfejezet{yamaniyamabhedaḥ}
yamaś ca niyamaś caiva dvayor bhedam ataḥ ś\textsubring{r}ṇu\thinspace{\dandab} \dontdisplaylinenum
            \var{\va niyama°\lem  \msCa\msCb\msNb\msNc\Ed; niyamai° \msNa}%

ahi\.msā satyam asteyam ān\textsubring{r}śa\.msya\.m damo gh\textsubring{r}ṇā \veg\dontdisplaylinenum
            \var{\vd °m ān\textsubring{r}śa\.msya\.m\lem  \eme; °m an\textsubring{r}śa\.msyo \msCa\msCb\msNa\msNb\Ed, °m ān\textsubring{r}śa\.msyā \msNc}%

dhanyāpramādo mādhuryam ārjava\.m ca yamā daśa\thinspace{\dandab} \dontdisplaylinenum
            \var{\va dhanyā°\lem  \Ed; dhanyaḥ \msCa\msCb\msNb\msNc, dhyanya\.m \msNa\oo
                 mādhurya°\lem  \Ed; mādhūrya° \msCa\msCb\msNa\msNb\msNc}%
            \var{\vb ārjava\.m ca\lem  \msCa\msCb\msNa\msNb\msNc; ārjavaś ca \Ed}%

ekaikasya punaḥ pañcabhedam āhur manīṣiṇaḥ \veg\dontdisplaylinenum
            \var{\vd °m āhur ma°\lem  \msCa\msCb\msNa\msNb\Ed; °m āhu ma° \msNc}%


\alalfejezet{yameṣv ahi\.msā (1)}
ahi\.msādi pravakṣyāmi ś\textsubring{r}ṇuṣvāvahito dvija\thinspace{\dandab} \dontdisplaylinenum
            \var{\vb ś\textsubring{r}ṇuṣvā°\lem  \msCa\msCb\msNc\Ed; ś\textsubring{r}ṇuṣva° \msNa\msNb}%

trāsana\.m tāḍana\.m bandho māraṇa\.m v\textsubring{r}ttināśanam \danda\dontdisplaylinenum
            \var{\vc bandho\lem  \msCa\msCb\msNa\msNc; baddho \msNb, bandha \Ed}%

hi\.msā\.m pañcavidhām āhur munayas tattvadarśinaḥ \veg\dontdisplaylinenum
            \var{\ve hi\.msā\.m\lem  \msCa\msNa\msNc; hi\.msā \msCb\msNb\Ed\oo
                 °vidhām āhu°\lem  \msCb\msNa\msNc; °vidham āhu° \msCa, °vidhāny āhu° \msNb, °vidha prāhu° \Ed}%

kāṣṭhaloṣṭakaśādyais tu tāḍayantīha nirdayāḥ\thinspace{\dandab} \dontdisplaylinenum
            \var{\va kāṣṭhaloṣṭa°\lem  \msCa\msCb\msNa\msNc\Ed; kā\uncl{ṣṭha}{\lost}{\lost} \msNb}%
            \var{\vb nirdayāḥ\lem  \msCa\msCb\msNa\msNb\msNc; nirdayā \Ed}%

tatprahāravibhinnāṅgo m\textsubring{r}tavadhyam avāpnuyāt \veg\dontdisplaylinenum
            \var{\vc °bhinnāṅgo\lem  \msCa\msCb\msNa\msNb\msNc; °bhinnāṅgā \Ed}%

baddhvā pādau bhujoraś ca śirorukkaṇṭhapāśitāḥ\thinspace{\dandab} \dontdisplaylinenum
            \var{\va bhujoraś ca\lem  \msCa\msCb\msNb\msNc; bhujauraś ca \msNa\Ed}%
            \var{\vb śirorukkaṇṭha°\lem  \corr; śirorukaṇṭha° \msCa\msCb\msNa\msNb\msNc; śiroruḥ kaṇṭha° \Ed}%

anāhatā mriyanty eva\.m vadhabandhajanāḥ sm\textsubring{r}tāḥ \veg\dontdisplaylinenum
            \var{\vc anāhatā mriyanty eva\.m\lem  \msCa\msCb\msNa\msNc\Ed; anāhata mriya\.mty eṣa \msNb}%
            \var{\vd °janāḥ sm\textsubring{r}tāḥ\lem  \conj; °najāḥ sm\textsubring{r}tāḥ \msCa\msCb\msNa\msNb, °najāḥ sm\textsubring{r}tā \msNc, °naja sm\textsubring{r}taḥ \Ed}%

śatrucaurabhayair ghoraiḥ si\.mhavyāghragajoragaiḥ\thinspace{\dandab} \dontdisplaylinenum
            \var{\va °caurabhayair ghoraiḥ\lem  \msCa\msCb\msNa\msNc\Ed; °corabhayai ghorai \msNb}%

trāsanād vadham āpnoti anyair vāpi suduḥsahaiḥ \veg\dontdisplaylinenum
            \var{\vd anyair vāpi\lem  \msCa\msCb\msNa\msNb\Ed; anye cāpi \msNc}%

yasya yasya hared vitta\.m tasya tasya vadhaḥ sm\textsubring{r}taḥ\thinspace{\dandab} \dontdisplaylinenum
            \var{\va hared vi°\lem  \msCa\msCb\msNa\msNc\Ed; hare vi° \msNb}%
            \var{\vb vadhaḥ\lem  \msCa\msCb\msNa\msNb\msNc; vadha \Ed}%

v\textsubring{r}ttijīvābhibhūtānā\.m taddvārā nihataḥ sm\textsubring{r}taḥ \veg\dontdisplaylinenum
            \var{\va °bhibhūtānā\.m\lem  \msCa\msCb\msNa\msNc\Ed; °vibhūtānā\.m \msNb}%
            \var{\vb taddvārā ni°\lem  \conj; tadvārān ni° \msCa\msCb\msNa\msNb\msNc, taddvārān ni° \Ed}%

viṣavahniśaraśastrair māyāyogabalena vā\thinspace{\dandab} \dontdisplaylinenum
            \var{\vab °śastrair māyā°\lem  \msCa\msCb\msNa\msNb; °śastrai mā° \msNc, °śastrair mmayā° \Ed}%

hi\.msakāny āhu viprendra munayas tattvadarśinaḥ \veg\dontdisplaylinenum
            \var{\vc hi\.msakāny āhu\lem  \msCb\msNb\msNc; hi\.msakāny āhur \msCa\msNa\ \unmetr, hi\.msakety āhu \Ed}%

ahi\.msā parama\.m dharma\.m yas tyajet sa durātmavān\thinspace{\dandab} \dontdisplaylinenum
            \var{\vc parama\.m dharma\.m\lem  \msCa\msCb\msNa\Ed; parama\.m dharma \msNb, paramo dharma\.m \msNc}%
            \var{\vd tyajet sa durātmavān\lem  \msCb\msNc\Ed; tyajec cha durātma{\il} \msCa, tyajet sudurātmavān \msNa,
                                tyajet sa durātmanam \msNb}%

kleśāyāsavinirmukta\.m sarvadharmaphalapradam \veg\dontdisplaylinenum

nātaḥ parataro mūrkho nātaḥ paratara\.m tamaḥ\thinspace{\dandab} \dontdisplaylinenum
            \var{\vb °tara\.m\lem  \msCa\msCbpcorr\msNa\msNb\msNc; °tan \msCbacorr\Ed}%

nātaḥ paratara\.m duḥkha\.m nātaḥ parataro 'yaśaḥ \veg\dontdisplaylinenum

nātaḥ paratara\.m pāpa\.m nātaḥ paratara\.m viṣam\thinspace{\dandab} \dontdisplaylinenum

nātaḥ paratarāvidyā nātaḥ para\.m tapodhana \veg\dontdisplaylinenum
            \var{\vd para\.m tapodhana\lem  \msCa\msCb\msNa\msNb\msNc; para tapodyamāḥ \Ed}%

yo hinasti na bhūtāni udbhijjādi caturvidham\thinspace{\dandab} \dontdisplaylinenum
            \var{\va yo hinasti na\lem  \msCa\msCb\msNa\msNc; yo na hinsanti \msNb, yo hi nāsti na \Ed}%
            \var{\vb udbhijjādi\lem  \eme; udbhijādi \msCa\msCb\msNb\msNc\Ed, udbhijāni \msNa\oo
                 °vidham\lem  \msCa\msCb\msNa\msNb\Ed; °vidhi\.m \msNc}%

sa bhavet puruṣaḥ śreṣṭhaḥ sarvabhūtadayānvitaḥ \veg\dontdisplaylinenum
            \var{\vc puruṣaḥ\lem  \msCa\msCb\msNa\msNb\msNc; puruṣa° \Ed}%

sarvabhūtadayā\.m nitya\.m yaḥ karoti sa paṇḍitaḥ\thinspace{\dandab} \dontdisplaylinenum
            \var{\va °dayā\.m nitya\.m\lem  \msCa\msNa\Ed; °dayā nitya\.m \msCb\msNb, °dayā nitya \msNc}%

sa yajvā sa tapasvī ca sa dātā sa d\textsubring{r}ḍhavrataḥ \veg\dontdisplaylinenum
            \var{\vc yajvā\lem  \msCa\msCb\msNa\msNc\Ed; yajyā \msNb}%

ahi\.msā parama\.m tīrtham ahi\.msā parama\.m tapaḥ\thinspace{\dandab} \dontdisplaylinenum
            \var{\va parama\.m tī°\lem  \msCa\msNa\msNb\msNc\Ed; paran tī° \msCb}%

ahi\.msā parama\.m dānam ahi\.msā parama\.m sukham \veg\dontdisplaylinenum
            \paral{\textit{\vo {\normalfont  This and the following verses are similar to MBh 13.117.37--38\oo
        \msCc\ resumes here in 189.jpg, f. 273r (sic!) with } rama\.m sukham {\normalfont  (3.30b) }}} 

ahi\.msā paramo yajñaḥ ahi\.msā parama\.m vratam\thinspace{\dandab} \dontdisplaylinenum
            \var{\va yajñaḥ\lem  \msCb\msCc\msNb\Ed; yajñar \msCa, yajña \msNa\msNc}%

ahi\.msā parama\.m jñānam ahi\.msā paramā kriyā \veg\dontdisplaylinenum
            \var{\vc parama\.m\lem  \mssCaCbCc\msNa\msNb\msNc; paramo \Ed}%
            \var{\vd paramā\lem  \mssCaCbCc\msNa\msNc\Ed; paramā\.m \msNb}%

ahi\.msā parama\.m śaucam ahi\.msā paramo damaḥ\thinspace{\dandab} \dontdisplaylinenum
            \var{\vab \om\ \Ed}%

ahi\.msā paramo lābhaḥ ahi\.msā parama\.m yaśaḥ \veg\dontdisplaylinenum
            \var{\vc lābhaḥ\lem  \msNc; lābha \msCa\msCb\msNa\msNb\Ed, lābho \msCc}%
            \var{\vd parama\.m\lem  \mssCaCbCc\msNb\msNc\Ed; paramā \msNa}%
            \paral{\textit{{\normalfont After pādas cd, \Ed\ inserts this: }
        ahi\.msā paramā kīrti ahi\.msā paramo damaḥ, 
        {\normalfont which is not to be found in \mssCaCbCc\msNa\msNb\msNc}}} 

ahi\.msā paramo dharmaḥ ahi\.msā paramā gatiḥ\thinspace{\dandab} \dontdisplaylinenum
            \var{\va dharmaḥ\lem  \msNa\msNc; dharma \msCa\msCb\Ed, dharmo \msCc, dha{\lost} \msNb}%
            \var{\vb ahi\.msā paramā gatiḥ\lem  \mssCaCbCc\msNa\msNc; {\lost}{\lost}{\lost}{\lost}{\lost}{\lost}{\lost}{\lost} \msNb, ahi\.msā paramo gatiḥ \Ed}%

ahi\.msā parama\.m brahma ahi\.msā paramaḥ śivaḥ \veg\dontdisplaylinenum
            \var{\ve ahi\.msā parama\.m brahma\lem  \mssCaCbCc\msNa\Ed; \uncl{ahi\.msā parama\.m brahma} \msNb, ahi\.msā para\.mma\.m brahma \msNc}%

mā\.msāśanān nivarteta manasāpi na kāṅkṣayet\thinspace{\dandab} \dontdisplaylinenum
            \var{\va mā\.msāśanān ni°\lem  \msCa\msCb\Ed; mānsāśana ni° \msCc, mā\.msāśanan ni° \msNa, mansāsanan ni° \msNb,
                                                        \uncl{mā\.msaśānān ni°} \msNc}%

sa mahat phalam āpnoti yas tu mā\.msa\.m vivarjayet \veg\dontdisplaylinenum
            \var{\vd mā\.msa\.m\lem  \mssCaCbCc\msNa; mā\.msa \msNb\Ed, māsa\.m \msNc}%

svamā\.msa\.m paramā\.msena yo vardhayitum icchati\thinspace{\dandab} \dontdisplaylinenum
            \var{\va °mā\.msena\lem  \mssCaCbCc\msNa\msNb\Ed; °māsena \msNc}%
            \var{\vb vardhayitu°\lem  \mssCaCbCc\msNa\msNc\Ed; varddhayati \msNb}%
            \paral{\textit{\vab {\normalfont  = MBh 13.116.14ab and 13.116.34ab }}}

anabhyarcya pit\textsubring{\=r}n devān na tato 'nyo 'sti pāpak\textsubring{r}t \veg\dontdisplaylinenum
            \var{\vc pit\textsubring{\=r}n\lem  \msCa\msCb\msNa\msNc; pit\textsubring{r}n \msCc\Ed, \uncl{pit\textsubring{\=r}n} \msNb}%
            \var{\vd tato 'nyo\lem  \mssCaCbCc\msNa\msNb\msNc; tad anyo \Ed}%
            \paral{\textit{\vo {\normalfont \kb\ Manu 5.52}}}

madhuparke ca yajñe ca pit\textsubring{r}daivatakarmaṇi\thinspace{\dandab} \dontdisplaylinenum
            \var{\vb °daivata°\lem  \msCa\msCb\msNa\msNc\Ed; °devata° \msCc\msNb}%

atraiva paśavo hi\.msyā nānyatra manur abravīt \veg\dontdisplaylinenum
            \var{\vc atraiva paśavo hi\.msyā\lem  \msCa\msCc\msNc\Ed; atraiva paśavo hi\.msā \msCb, atraiva paśavo hi\.msyān \msNa,
                                                        {\lost}{\lost}{\lost}{\lost}{\lost}{\lost}{\lost}{\lost} \msNb}%
            \var{\vd nānyatra manur abravīt\lem  \mssCaCbCc\msNa\msNc\Ed; {\lost}{\lost}\uncl{tra manur abravīt} \msNb}%
            \paral{\textit{\vo {\normalfont \kb\ Manu 5.41}}}

krītvā svaya\.m vāpy utpādya paropah\textsubring{r}tam eva vā\thinspace{\dandab} \dontdisplaylinenum
            \var{\va krītvā\lem  \mssCaCbCc\msNa\msNb\msNc; k\textsubring{r}tvā \Ed\oo
                 °py utpādya\lem  \mssCaCbCc\msNa\msNb\msNc; °py utpādyā° \Ed}%
            \var{\vb °h\textsubring{r}ta°\lem  \mssCaCbCc\msNa\msNb\msNc; °hita° \Ed\oo
                 vā\lem  \mssCaCbCc\msNa\msNb\msNc; ca \Ed}%

devān pit\textsubring{\=r}\.mś cārcayitvā khādan mā\.msa\.m na doṣabhāk \veg\dontdisplaylinenum
            \var{\vc pit\textsubring{\=r}\.mś cārcayitvā\lem  \mssCaCbCc\msNa\msNc; pit\textsubring{\=r}ś cārcayitvā \msNb, pit\textsubring{r}ś cārpayitvā \Ed}%
            \var{\vd mā\.msa\.m\lem  \mssCaCbCc\msNa\msNb\Ed; māsa\.m \msNc}%
            \paral{\textit{\vo {\normalfont = Manu 5.32 (in Olivelle's critical edition; other editions read } paropak\textsubring{r}ta°{\normalfont ) }}}

vedayajñatapastīrthadānaśīlakriyāvrataiḥ\thinspace{\dandab} \dontdisplaylinenum
            \var{\vb °śīla°\lem  \msCa\msCb\msNa\msNb\msNc\Ed; °śala° \msCc}%

mā\.msāhāraniv\textsubring{r}ttānā\.m ṣoḍaśā\.mśa\.m na pūryate \veg\dontdisplaylinenum
            \var{\vc °v\textsubring{r}ttānā\.m\lem  \mssCaCbCc\msNa\msNc; °v\textsubring{r}ttānā \msNb, °v\textsubring{r}ttīnā\.m \Ed}%
            \var{\vd na\lem  \msCa\msCc\msNa\msNb\msNc\Ed; ta \msCb}% 

m\textsubring{r}gāḥ parṇat\textsubring{r}ṇāhārād ajameṣagavādibhiḥ\thinspace{\dandab} \dontdisplaylinenum
            \var{\va  parṇa°\lem  \mssCaCbCc\msNb\msNc; paṇṇa° \msNa, parṇā° \Ed}%
            \var{\vab °hārād a\lem  \msCa\msCc\msNbpcorr\msNc\Ed; °hād a \msNbacorr, °hārā a° \msCb\msNa}%

sukhino balavantaś ca vicaranti mahītale \veg\dontdisplaylinenum

vānarāḥ phalam āhārā rākṣasā rudhirapriyāḥ\thinspace{\dandab} \dontdisplaylinenum
            \var{\vab °hārā rā°\lem  \msCb\msNa\msNb; °hārād rā° \msCa\msCc\msNc\Ed}%

nihatā rākṣasāḥ sarve vānaraiḥ phalabhojibhiḥ \veg\dontdisplaylinenum
            \var{\vd °bhojibhiḥ\lem  \mssCaCbCc\msNa\msNb\msNc; °bhogibhiḥ \Ed}%

tasmān mā\.msa\.m na hīheta balakāmena bho dvija\thinspace{\dandab} \dontdisplaylinenum
            \var{\va mā\.msa\.m\lem  \mssCaCbCc\msNa\msNb\Ed; māsa\.m \msNc}%
            \var{\vb hīheta\lem  \mssCaCbCc\msNc\Ed; hīyeta \msNa\msNb}%

balena ca guṇākāśāt parato bhayabhīruṇā \veg\dontdisplaylinenum
            \var{\vc guṇākāśā°\lem  \mssCaCbCc\msNa\msNb\msNc; guṇā kuryā° \Ed}%

ahi\.msakasamo nāsti dānayajñasamīhayā\thinspace{\dandab} \dontdisplaylinenum
            \var{\vb °yajñasamīhayā\lem  \msCa\msCb\msNa\msNb; °dharmasamīhayā \msCc, °yajñasamīhayāḥ \msNc, °dharmasamīhaya \Ed}%

iha loke yaśaḥ kīrtiḥ paratra ca parā gatiḥ \veg\dontdisplaylinenum
            \var{\vc yaśaḥ\lem  \msCa\msCb\msNa\msNb\msNc\Ed; ya\uncl{śa\.m} \msCc}%
            \var{\vd parā gatiḥ\lem  \msCc\msNa\msNc; \uncl{parā gatiḥ} \msCa, 
                                                parāṅgatim \msCb\msNb, parā\.m gatiḥ \Ed}%

\ujvers\nemsloka 
trailokya\.m maṇiratnapūrṇam akhila\.m dattvottame brāhmaṇe
\dontdisplaylinenum
            \var{\va trailokya\.m\lem  \mssCaCbCc\msNa\msNc\Ed; trailokya \msNb\oo
                 akhila\.m dattvottame brāhmaṇe\lem  \msCb\msCc\msNb\msNc\Ed;
                                        a\uncl{khila\.m}{\il}{\il}{\il}{\il}{\il}{\il}{\il} \msCa, akhila\.m dattottame brāhmaṇe \msNa}%

\nemslokab 
koṭīyajñasahasrapadmam ayuta\.m dattvā mahī\.m dakṣiṇām \danda\dontdisplaylinenum
            \var{\vb koṭīyajñasahasrapadmam\lem  \msCb\msCc\msNa\msNb\msNc\Ed; {\il}{\il}{\il}{\il}{\il}{\il}{\il}{\il}{\il} \msCa\oo
                 mahī\.m\lem  \msCa\msCb\msNa\msNb\msNc\Ed; mahī \msCc}%

\nemslokac 
tīrthānā\.m ca sahasrakoṭiniyuta\.m snātvā sak\textsubring{r}n mānavaḥ
\dontdisplaylinenum
            \var{\vc °koṭi°\lem  \mssCaCbCc\msNa\msNb\msNc; °koṭī° \Ed\ \unmetr\oo
                 snātvā\lem  \msCa\msCc\msNa\msNb\msNc\Ed; snā ' \msCb}%

\nemslokad 
etatpuṇyaphalam ahi\.msakajanaḥ prāpnoti niḥsa\.mśayaḥ \veg\dontdisplaylinenum
            \var{\vd °phalam ahi\.msa°\lem  \mssCaCbCc\msNa\msNb\Ed; °phala\.m tv ahi\.msa° \msNc\oo
                 niḥsa\.mśayaḥ\lem  \msCc\msNa\msNb\msNc; {\il}{\il}{\il}{\il} \msCa, niḥsa\.mśaya{\il} \msCb, niḥsa\.mśaya\.m \Ed}%

\vers


\jump
\begin{center}
\ketdanda iti v\textsubring{r}ṣasārasa\.mgrahe ahi\.msāpraśa\.msā nāmādhyāyas t\textsubring{r}tīyaḥ\ketdanda
\end{center}
\dontdisplaylinenum\vers 
            \var{{\normalfont Colophon:} nāmādhyāyas t\textsubring{r}tīyaḥ\lem  \mssCaCbCc\msNa\msNb; nāmādhyāyas t\textsubring{r}tīya \msNc,
                        nāmas t\textsubring{r}tīyo 'dhyāyaḥ \Ed}%
\bekveg\szamveg\vfill\phpspagebreak\szam\bek\versno=0\fejno=4
\thispagestyle{empty}



\alfejezet{\textbf{caturtho 'dhyāyaḥ}}\jump\jump

\alalfejezet{yameṣu satyam (2)}
\vers

anarthayajña uvāca~{\dandab}\dontdisplaylinenum 

sadbhāvaḥ satyam ity āhur d\textsubring{r}ṣṭapratyayam eva vā\thinspace{\danda} \dontdisplaylinenum
            \var{\va sadbhāvaḥ\lem  \mssCaCbCc\msNa\msNc; sadbhāva° \msNb\Ed}%
            \var{\vab satyam ity āhur d\textsubring{r}°\lem  \msCb\msNa\msNc\Ed;
        satya\uncl{m i}ty āhu d\textsubring{r}° \msCa, satyam ity āhu d\textsubring{r}° \msCc, 
        satyām ity āhur d\textsubring{r}° \msNb}%
            \var{\vb °pratyaya°\lem  \msCa\msCb\msNa\msNb; °pratya° \msCc, °pratyeya° \msNc, pratyakṣa° \Ed}%

yathābhūtārthakathana\.m tat satyakathana\.m sm\textsubring{r}tam \veg\dontdisplaylinenum
            \var{\vc yathābhūtārthakathana\.m\lem  \msCa\msCb\msNa\msNb\msNc\Ed; 
        yathābhūtārtha \msCcacorr, yathābhūtārtha{\il}kta kathana\.m \msCcpcorr}%
            \var{\vd tat satyakathana\.m\lem  \msCa\msCc\msNa\msNb\msNc\Ed; tat satyakathaka\.m \msCb, kathana\.m sm\textsubring{r}ta\.m \msCcacorr, 
        satyakakathana\.m sm\textsubring{r}ta\.m \msCcpcorr}%

ākrośatāḍanādīni yaḥ saheta suduḥsaham\thinspace{\dandab} \dontdisplaylinenum
            \var{\va °tāḍanā°\lem  \msCa\msCc\msNa\msNb\msNc\Ed; °nāḍanā° \msCb}%
            \var{\vb suduḥsaham\lem  \msCa\msCb\msNa\msNb\msNc\Ed; sudusaha\.m \msCc}%

kṣamate yo jitātmā tu sa ca satyam udāh\textsubring{r}tam \veg\dontdisplaylinenum
            \var{\vd satyam udāh\textsubring{r}tam\lem  \msCb\msCc\msNa\msNb\msNc\Ed; 
        \uncl{satya}m u\uncl{dā}h\textsubring{r}tam \msCa}%

vadhārtham udyataḥ śastra\.m yadi p\textsubring{r}ccheta karhicit\thinspace{\dandab} \dontdisplaylinenum
            \var{\va °dyataḥ\lem  \mssCaCbCc\msNb\msNc\Ed; °dyata \msNa\oo
        śastra\.m\lem  \msCa\msCb\msNa\msNb\msNc; śastra \msCc, satya \msCb\Ed}%
            \var{\vb karhicit\lem  \mssCaCbCc\Ed; karhacit \msNa\msNb\msNc}%

na tatra satya\.m vaktavyam an\textsubring{r}ta\.m satyam ucyate \veg\dontdisplaylinenum
            \var{\vc satya\.m\lem  \msCa\msCc\msNa\msNb\msNc; satya \msCb\Ed}%

vadhārhaḥ puruṣaḥ kaścid vrajet pathi bhayāturaḥ\thinspace{\dandab} \dontdisplaylinenum
            \var{\vb °turaḥ\lem  \msCa\msCc\msNa\msNb\msNc\Ed; °tura \msCb}%

p\textsubring{r}cchato 'pi na vaktavya\.m satya\.m tad vāpi ucyate \veg\dontdisplaylinenum
            \var{\vc p\textsubring{r}cchato\lem  \mssCaCbCc\msNa\msNb\msNc; p\textsubring{r}cchate \Ed}%
            \var{\vd tad vāpi\lem  \mssCaCbCc\msNa\msNc\Ed; tad api \msNb}%

\ujvers\nemsloka 
na narmayuktam an\textsubring{r}ta\.m hinasti
\dontdisplaylinenum
            \var{\va hinasti\lem  \msCa\msCb\msNb\msNc; hi nāsti \msCc\msNa\Ed}%

\nemslokab 
na strīṣu rājan na vivāhakāle \danda\dontdisplaylinenum
            \var{\vb rājan na\lem  \msCa\msCb\msNb\msNc\Ed; rāja na \msCc, rājya\.m na \msNa}%

\nemslokac 
prāṇātyaye sarvadhanāpahāre
\dontdisplaylinenum
            \var{\vc °tyaye\lem  \mssCaCbCc\msNa\msNc\Ed; °tyaje \msNb\oo
                 °pahāre\lem  \msCa\msCb\msNa\msNc\Ed; °prahāre \msCc\msNb}%

\nemslokad 
pañcān\textsubring{r}ta\.m satyam udāharanti \veg\dontdisplaylinenum
            \paral{\textit{\vo {\normalfont cf.\ \MBh\ 1.77.16: } na narmayukta\.m vacana\.m hinasti na strīṣu rājan na vivāhakāle{\thinspace\danda}
        prāṇātyaye sarvadhanāpahāre pañcān\textsubring{r}tāny āhur apātakāni{\thinspace\ketdanda};
        {\normalfont \MBh\ 12.159.28: } na narmayukta\.m vacana\.m hinasti na strīṣu rājan na vivāhakāle{\thinspace\danda}
        na gurvarthe nātmano jīvitārthe pañcān\textsubring{r}tāny āhur apātakāni{\thinspace\ketdanda};
        {\normalfont \MP\ 31.16: } na narmayukta\.m vacana\.m hinasti na strīṣu rājan na vivāhakāle{\thinspace\danda}
        prāṇātyaye sarvadhanāpahāre pañcān\textsubring{r}tāny āhur apātakāni{\thinspace\ketdanda};
        {\normalfont Kauṇḍinya's commentary ad \PS\ 1.9: }
        gobrāhmaṇārthe 'vacana\.m himasti na strīṣu rājan na vivāhakāle{\thinspace\danda}
        prāṇātyaye sarvadhanāpahāre pañcān\textsubring{r}tāni āhur apātakāni{\thinspace\ketdanda};
        {\normalfont Abhidharmakośabhāṣya  24114--24117 }:
        na narmayuktam an\textsubring{r}ta\.m hi nāsti na strīṣu rājan na vivāhakāle{\thinspace\danda}
        prāṇātyaye sarvadhanāpahāre pañcān\textsubring{r}tāñ ? āhur apātakāni{\thinspace\ketdanda} }}

\vers

devamānuṣatiryeṣu satya\.m dharmaḥ paro yataḥ\thinspace{\dandab} \dontdisplaylinenum
            \var{\vb °mānuṣa°\lem  \mssCaCbCc\msNa\msNb\Ed; °mānuṣya° \msNc}%

satya\.m śreṣṭha\.m variṣṭha\.m ca satya\.m dharmaḥ sanātanaḥ \veg\dontdisplaylinenum
            \var{\vc śreṣṭha\.m\lem  \mssCaCbCc\msNa\msNc; śreṣṭha \msNb\Ed\oo
        variṣṭha\.m ca\lem  \msCa\msCbpcorr\msCc\msNa\msNb\msNc\Ed; variṣṭhamvariṣṭhamvañ ca \msCbacorr}%
            \var{\vd satya\.m\lem  \msCa\msCc\msNa\msNc\Ed; satya° \msCb\msNb\oo
        dharmaḥ\lem  \msCa\msCb\msNa\msNb\msNc; dharma \msCc\Ed}%

satya\.m sāgaram avyakta\.m satyam akṣayabhogadam\thinspace{\dandab} \dontdisplaylinenum
            \var{\va satya\.m\lem  \msCa\msCb\msNa\msNb\msNc\Ed; satya \msCc}%
            \var{\vb satyam akṣayabhogadam\lem  \msCa\msNa\msNb\msNc; satya\.mm akṣayabhogadam \msCb\msCc, satyam akṣayate nara\.m \Ed}%

satya\.m potaḥ paratrārtha\.m satya\.m panthāna vistaram \veg\dontdisplaylinenum
            \var{\vc potaḥ\lem  \mssCaCbCc\msNb\msNc; pota \msNa, proktaḥ \Ed}%
            \var{\vd panthāna vistaram\lem  \mssCaCbCc\msNa\msNb\msNc; yaj jñānavistaram \Ed}%

satyam iṣṭagatiḥ prokta\.m satya\.m yajñam anuttamam\thinspace{\dandab} \dontdisplaylinenum
            \var{\va °ṣṭagatiḥ\lem  \mssCaCbCc\msNa\msNc\Ed; °\uncl{ṣṭā}gatiḥ \msNb}%

satya\.m tīrtha\.m para\.m tīrtha\.m satya\.m dānam anantakam \veg\dontdisplaylinenum
            \var{\vc tīrtha\.m\lem  \mssCaCbCc\msNa; tīrtha \msNb\msNc, tīrthāt \Ed}%

satya\.m śīla\.m tapo jñāna\.m satya\.m śauca\.m damaḥ śamaḥ\thinspace{\dandab} \dontdisplaylinenum
            \var{\va satya\.m\lem  \msCa\msCc\msNa\msNb\msNc\Ed; satya \msCb}%
            \var{\vb śamaḥ\lem  \mssCaCbCc\msNa\msNc\Ed; śamam \msNb}%

satya\.m sopānam ūrdhvasya satya\.m kīrtir yaśaḥ sukham \veg\dontdisplaylinenum
            \var{\vc satya\.m\lem  \msCa\msCc\msNa\msNb\Ed; sa\.mtya\.m \msCb, satya \msNc}%
            \var{\vd sukham\lem  \mssCaCbCc\msNa\msNb\msNc; sukhaḥ \Ed}%
            \paral{\textit{\vc {\normalfont cf.\ Varāhapurāṇa 193.36cd: } satya\.m svargasya sopāna\.m pārāvārasya naur iva}}

aśvamedhasahasra\.m ca satya\.m ca tulayā dh\textsubring{r}tam\thinspace{\dandab} \dontdisplaylinenum
            \var{\va °sahasra\.m ca\lem  \msCa\msCb\msNa\msNb\msNc\Ed; °sahasrasya \msCc}%
            \var{\vb tulayā\lem  \msCa\msCb\msNa\msNb\msNc\Ed; tulyayā \msCc}%

aśvamedhasahasrād dhi satyam eva viśiṣyate \veg\dontdisplaylinenum
            \var{\vc °sahasrād dhi\lem  \msCa\msCb\msNa\msNb\msNc\Ed; °sahasrā hi \msCc}%
            \var{\vd eva\lem  \msCa\msCb\msNa\msNb\msNc; eva\.m \msCc\Ed}%
            \paral{\textit{\vo {\normalfont  \kb\ Mārkaṇḍeyapurāṇa 8.42: }
        aśvamedhasahasra\.m ca satya\.m ca tulayā dh\textsubring{r}tam{\thinspace\danda}
        aśvamedhasahasrād dhi satyam eva viśiṣyate{\thinspace\ketdanda}}}
            \paral{\textit{\vcd {\normalfont  = MBh 1.69.22cd and 13.74.29cd }}}

satyena tapate sūryaḥ satyena p\textsubring{r}thivī sthitā\thinspace{\dandab} \dontdisplaylinenum
            \var{\vab sūryaḥ satyena p\textsubring{r}thivī sthitā\lem  \msNa\msNc;
        sū\uncl{ryaḥ sa}tyena p\textsubring{r}thi sthitāḥ \msCa,
        sūryaḥ satyaina p\textsubring{r}thivī sthitā \msCb,
        sūrya  satyena p\textsubring{r}thivī sthitāḥ \msCc,
        sūrya \uncl{satye} {\lost}{\lost}{\lost} vī sthitā \msNb,
        sūryaḥ satyena p\textsubring{r}thivī sthitāḥ \Ed}%

satyena vāyavo vānti satye toya\.m ca śītalam \veg\dontdisplaylinenum
            \var{\vc vāyavo\lem  \mssCaCbCc\msNa\msNc\Ed; vātyavo \msNb}%
            \var{\vd satye\lem  \mssCaCbCc\msNa\msNb\msNc; satyāt \Ed}%
            \paral{\textit{\vc {\normalfont \kb\ Varāhapurāṇa 193.37: } 
        sūryas tapati satyena vātaḥ satyena vāti ca{\thinspace\danda}  
        agnir dahati satyena satyena p\textsubring{r}thivī sthitā{\thinspace\ketdanda}}}

tiṣṭhanti sāgarāḥ satye samayena priyavrataḥ\thinspace{\dandab} \dontdisplaylinenum
            \var{\va sāgarāḥ\lem  \msCa\msCb\msNa\msNb\msNc\Ed; sāgarā \msCc}%
            \var{\vb samayena\lem  \mssCaCbCc\msNa\msNb\msNc; satyena ca \Ed}%

satye tiṣṭhati govindo balibandhanakāraṇāt \veg\dontdisplaylinenum

agnir dahati satyena satyena śaśir ācaraḥ\thinspace{\dandab} \dontdisplaylinenum
            \var{\vab satyena satyena\lem  \mssCaCbCc\msNapcorr\msNb\Ed; satyena \msNaacorr\msNc}%
            \var{\vb śaśir ācaraḥ\lem  \msNa\msNb\msNc; saśi\uncl{bhācaraḥ} \msCa,
        śa\uncl{si}{\il}caraḥ \msCb, sa śirā varaḥ \msCc, śaśibhāṣkaraḥ \Ed}%
            \paral{\textit{\vc {\normalfont \kb\ Varāhapurāṇa 193.37cd: } 
        agnir dahati satyena satyena p\textsubring{r}thivī sthitā}}

satyena vindhyās tiṣṭhanti vardhamāno na vardhate \veg\dontdisplaylinenum
            \var{\vc vindhyās tiṣṭhanti\lem  \msCa\msNa\msNc;
        vindhyas tiṣṭhanti \msCb\msNb, vindhyā tiṣṭhanti \msCc, tiṣṭhate vindhyo \Ed}%

lokālokaḥ sthitaḥ satye meruḥ satye pratiṣṭhitaḥ\thinspace{\dandab} \dontdisplaylinenum
            \var{\va °lokaḥ\lem  \Ed; °loka \mssCaCbCc\msNa\msNb\msNc\oo
        sthitaḥ\lem  \mssCaCbCc\msNa\msNb\Ed; sthiḥ \msNc\oo
        satye\lem  \mssCaCbCc\msNa\msNb\msNc; satya\.m \Ed}%
            \var{\vb meruḥ\lem  \msCa\msCb\msNa\msNb\msNc; meru \msCc\Ed}%

vedās tiṣṭhanti satyeṣu dharmaḥ satye pratiṣṭhati \veg\dontdisplaylinenum
            \var{\vc vedās ti°\lem  \msCa\msCc\msNa\msNb\msNc; devās ti° \msCb, vedā ti° \Ed}%
            \var{\vd satye\lem  \msCa\msCb\msNa\msNb\msNc\Ed; dharme \msCc\oo
        pratiṣṭhati\lem  \mssCaCbCc\msNa\msNb\Ed; pratiṣṭhiti \msNcacorr, pratiṣṭhitaḥ \msNcpcorr}%

satya\.m gauḥ kṣarate kṣīra\.m satya\.m kṣīre gh\textsubring{r}ta\.m sthitam\thinspace{\dandab} \dontdisplaylinenum
            \var{\va gauḥ\lem  \msCa\msCb\msNa\msNc\Ed; gau \msCc\msNb}%
            \var{\vab kṣīra\.m satya\.m\lem  \msCa\msCc\msNa\msNb\msNc\Ed; kṣītya\.m \msCbacorr, ksī{\il} nitya\.m \msCb}%
            \var{\vb kṣīre gh\textsubring{r}ta\.m sthitam\lem  \msCa\msCb\msNa\msNc; kṣīra\.m gh\textsubring{r}ta\.m sthitam \msCc, kṣīre gh\textsubring{r}ta sthitam \msNb,
        kṣīra\.m sthita\.m gh\textsubring{r}tam \Ed}%

satye jīvaḥ sthito dehe satya\.m jīvaḥ sanātanaḥ \veg\dontdisplaylinenum
            \var{\vc satye jīvaḥ\lem  \mssCaCbCc\msNa\msNb; satye jīva \msNc, satya\.m jīva \Ed}%
            \var{\vd jīvaḥ\lem  \msCa\msCb\msNa\msNb\msNc\Ed; jīva \msCc}%

satyam ekena samprāpto dharmasādhananiścayaḥ\thinspace{\dandab} \dontdisplaylinenum
            \var{\va satyam ekena\lem  \msCa\msCc\msNa\msNc\Ed; satyem ekena \msNb, satyam ekaina \msCb}%
            \var{\vb dharma°\lem  \Ed; dharmaḥ \mssCaCbCc\msNa\msNb\msNc\oo
        °niścayaḥ\lem  \msCb\msCc\msNa\msNb\msNc\Ed; °niścaḥ \msCa}%

rāmarāghavavīryeṇa satyam eka\.m surakṣitam \veg\dontdisplaylinenum
            \var{\vd satyam eka\.m\lem  \mssCaCbCc\msNa\msNc\Ed; satyem eka\.m \msNb\oo
        surakṣitam\lem  \msCa\msCc\msNb\msNc\Ed; surakṣitaḥ \msNa, surikṣitam \msCb}%

etat satyavidhānasya kīrtita\.m tava suvrata\thinspace{\dandab} \dontdisplaylinenum
            \var{\va etat satya°\lem  \msCa\msCc\msNa\msNb\msNc\Ed; eva\.m satya° \msCb}%
            \var{\vb suvrata\lem  \msCa\msNa\msNc; suvrate \msCb\msNb, suvrata\uncl{ḥ} \msCc, suvrata\.m \Ed}%

sarvalokahitārthāya kim anyac chrotum icchasi \veg\dontdisplaylinenum

vigatarāga uvāca~{\dandab}\dontdisplaylinenum 

na hi t\textsubring{r}pti\.m vijānāmi śrutvā dharma\.m tavāpy aham\thinspace{\danda} \dontdisplaylinenum
            \var{\va t\textsubring{r}pti\.m\lem  \msCa\msCb\msNa\msNb\msNc\Ed; t\textsubring{r}pti \msCc\oo
                 vijānāmi\lem  \mssCaCbCc\msNa\msNc\Ed; vināmi \msNb}%
            \var{\vb śrutvā dharma\.m tavāpy aham\lem  \msCb\msCc\msNa\msNb\msNc; śru dharman tavāpy aham \msCa,
                                                         dharma\.m śrutvā tathāpy aham \Ed}%

upariṣṭād ato bhūyaḥ kathayasva tapodhana \veg\dontdisplaylinenum
            \var{\vd °dhana\lem  \msCc\msNa\msNb\Ed; °dhūna \msCa, °dhanaḥ \msCb\msNc}%


\alalfejezet{yameṣv asteyam (3)}
anarthayajña uvāca~{\dandab}\dontdisplaylinenum 

steya\.m ś\textsubring{r}ṇv atha viprendra pañcadhā parikīrtitam\thinspace{\danda} \dontdisplaylinenum
            \var{\vb °kīrtitam\lem  \msCa\msCc\msNa\msNb\msNc\Ed; °kīrttitām \msCb}%

adattādānam ādau tu utkoca\.m ca tataḥ param \danda\dontdisplaylinenum
            \var{\vd utkoca\.m ca tataḥ\lem  \msCa\msCc\msNa\msNb\msNc; tkoca tataḥ \msCb, utkoca\.m cān\textsubring{r}taḥ \Ed}%

prasthavyājas tulāvyājaḥ prasahyasteya pañcamam \veg\dontdisplaylinenum
            \var{\vc tulāvyājaḥ\lem  \msCb\msNc\Ed; tulāvyāja \msCa\msCc\msNa\msNb}%
            \var{\vd °sahya°\lem  \mssCaCbCc\msNa\msNc\Ed; °sahye \msNb\oo
                 °steya\lem  \msCb\msCc\msNa\msNb\Ed; °stena \msCa\msNc\oo
                 pañcamam\lem  \msCa\msCb\msNa\msNb\msNc; pañcamaḥ \msCc\Ed}%

dh\textsubring{r}ṣṭaduṣṭaprabhāvena paradravyāpakarṣaṇam\thinspace{\dandab} \dontdisplaylinenum
            \var{\va dh\textsubring{r}ṣṭaduṣṭa°\lem  \msCa\msNa\msNc\Ed; dh\textsubring{r}ṣṭadumna° \msCb, dh\textsubring{r}taduṣṭa° \msCc, d\textsubring{r}ṣtaduṣṭa° \msNb}%
            \var{\vb °karṣaṇam\lem  \mssCaCbCc\msNb\msNc\Ed; °karṣaṇa \msNa}%

vāryamāṇo 'pi durbuddhir adattādānam ucyate \veg\dontdisplaylinenum
            \var{\vb vāryamāṇo 'pi\lem  \msCa\msCc\msNa\msNb\msNc\Ed; vāryamāno vi° \msCb}%

utkoca\.m ś\textsubring{r}ṇu viprendra dharmasa\.mkarakārakam\thinspace{\dandab} \dontdisplaylinenum
            \var{\va utkoca\.m\lem  \msCb\msCc\msNa\msNb\msNc\Ed; utkoca \msCa\oo
                 viprendra\lem  \mssCaCbCc\msNa\msNc\Ed; vidrendra \msNb}%
            \var{\vb °sa\.mkara°\lem  \msCc\msNa; °śaṅkara° \msCa\msCb\msNb, °sakara° \msNc, °sa\.mhāra° \Ed\oo
                 °kārakam\lem  \mssCaCbCc\msNb\msNc\Ed; °kārakaḥ \msNa}%

mūlakāryavināśārtham utkocaḥ parig\textsubring{r}hyate \veg\dontdisplaylinenum
            \var{\vc °vināśārtha°\lem  \mssCaCbCc\msNapcorr\msNb\msNc\Ed; °vinārtha° \msNaacorr}%
            \var{\vd °tkocaḥ\lem  \mssCaCbCc\msNa\msNc; °tkoca\.m \msNb, °tkoca \Ed}%

tena cāsau vijānīyād dravyalobhabalāt k\textsubring{r}tam\thinspace{\dandab} \dontdisplaylinenum
            \var{\vab vijānīyād dra°\lem  \msCa\msCb\msNa\msNb\msNc\Ed; vijānīyā dra° \msCc}%

prasthavyāja-upāyena kuṭumba\.m trātum icchati \veg\dontdisplaylinenum

ta\.m ca stenam vijānīyāt paradravyāpahārakam\thinspace{\dandab} \dontdisplaylinenum
            \var{\va ta\.m ca stenam\lem  \msCa; tañ ca stena \msCb, ta\.m ca steya\.m \msNa, tañ ca teya \msNb, so 'pi tena \msCc\Ed,
                                                                                tañ ca tena \msNc}%
            \var{\vb °hārakam\lem  \msCa\msCb\msNapcorr\msNc\Ed; °hārakaḥ \msCc, °hārakā \msNaacorr °hārakāḥ \msNb}%

tulāvyāja-upāyena parasvārtha\.m hared yadi \veg\dontdisplaylinenum
            \var{\vd parasvārtha\.m\lem  \msCa\msCc\msNa\msNc; parasvārtha \msCb\msNb, parasyārtha\.m \Ed\oo
                 hared yadi\lem  \msCa\msCc\msNa\msNb\msNc\Ed; hared yati \msCb}%

cauralakṣaṇakāś cānye kūṭakā yaṭikā narāḥ\thinspace{\dandab} \dontdisplaylinenum
            \var{\vb kūṭakā yaṭikā\lem  \msCb\msCc\msNaacorr\msNc; \uncl{ku}ṭakā yaṭikā \msCa, kūṭakāryaṭikā \msNapcorr\Ed, 
                                                kūṭakā paṭikā \msNb}%

durbalārjavabāleṣu cchadmanā vā balena vā \veg\dontdisplaylinenum
            \var{\vc °rjava°\lem  \mssCaCbCc\msNa\msNc\Ed; °java° \msNb}%
            \var{\vd cchadmanā\lem  \Ed; cchanmanā \mssCaCbCc\msNa\msNb, cchatmānā \msNc}%

apah\textsubring{r}tya dhana\.m mūḍhaḥ sa coraś cora ucyate\thinspace{\dandab} \dontdisplaylinenum
            \var{\vab mūḍhaḥ sa\lem  \mssCaCbCc\msNa\msNc\Ed; mūḍhās sa \msNb}%
            \var{\vb coraś cora\lem  \msCa\msCc\msNb\Ed; caura cora \msCb, cauraś caura \msNa, cauraś cora \msNc}%

nāsti steyasama\.m pāpa\.m nāsty adharmaś ca tatsamaḥ \veg\dontdisplaylinenum
            \var{\vcd \om\ \Ed}%
            \var{\vc steya°\lem  \msNa\msNc; tena \msCa, stena° \msCb\msCc\msNb, \om\ \Ed}%
            \var{\vd °samaḥ\lem  \msCa\msCb\msNa\msNb\msNc; °sama\.m \msCc, \om\ \Ed}%

nāsti stenasamākīrtir nāsti stenasamo 'nayaḥ\thinspace{\dandab} \dontdisplaylinenum
            \var{\vab \om\ \Ed}%
            \var{\va stena°\lem  \msCa\msCb\msNa\msNb; tena \msCc, steya° \msNc, \om\ \Ed}%
            \var{\vb stena°\lem  \mssCaCbCc\msNb\Ed; steya° \msNa\msNc}%

nāsti steyasamāvidyā nāsti stenasamaḥ khalaḥ \veg\dontdisplaylinenum
            \var{\vc steya°\lem  \msNa\msNc\Ed; stena° \mssCaCbCc\msNb\oo
                 °samā\lem  \msCc\msNb; °samo \msCa\msCb\msNa\msNc\Ed}%
            \var{\vd stena°\lem  \mssCaCbCc\msNb; steya° \msNa\msNc, tena \Ed}%

nāsti stenasama ajño nāsti stenasamo 'lasaḥ\thinspace{\dandab} \dontdisplaylinenum
            \var{\va stena°\lem  \msCa\msCb\msNb\msNc; steya° \msCc\msNa\Ed\oo
                 °sama\lem  \mssCaCbCc\msNa\msNc\Ed\ \unmetr; °sama\.m \msNb}%
            \var{\vb stena°\lem  \msCa\msCb\msNb; steya° \msCc\msNa\msNc, tena \Ed}%

nāsti stenasamo dveṣyo nāsti stenasamo 'priyaḥ \danda\dontdisplaylinenum
            \var{\vc stena°\lem  \msCa\msCb\msNb; steya° \msCc\msNa\msNc, tena \Ed}%
            \var{\vd stena°\lem  \msNb; steya° \mssCaCbCc\msNa\msNc\Ed}%

nāsti steyasama\.m duḥkha\.m nāsti stenasamo 'yaśaḥ \veg\dontdisplaylinenum
            \var{\ve steya°\lem  \msCc; stena° \msCa\msCb\msNa\msNb, stenya° \msNc, tena \Ed}%
            \var{\vf stena°\lem  \msCa\msCb\msNa\msNb; steya° \msCc\msNc, tena \Ed}%

\ujvers\nemsloka 
pracchanno hriyate ca vittam athavā pratyakṣam anyo haret
\dontdisplaylinenum
            \var{\va pracchanno\lem  \msCa\msCc\msNa\msNb\msNc\Ed; prasthanno \msCb\oo
                 ca vittam athavā\lem  \msNapcorr\Ed; vittam \msCa\msNaacorr\msNb, 'rtham anyapuruṣaḥ \msCb\msNc, citta \msCc\oo
                 pratyakṣam anyo\lem  \msCa\msCc\msNa\msNb\msNc; pratyakṣam ano \msCb, pratyakṣyam anye \Ed}%

\nemslokab 
nikṣepād dhanahāriṇo 'nyam adhamo vyājena cānyo haret \danda\dontdisplaylinenum
            \var{\vb nikṣepād dhana°\lem  \msCa\msCb\msNa; nikṣepā dhana° \msCc\msNb\msNc, nikṣepātraya° \Ed\oo
                 °hāriṇo\lem  \msCa\msCc\msNa\msNc\Ed; °hāriṇyo \msCb, °hāriṇā \msNb\oo
                 'nyam adhamo\lem  \msCa\msCb\msNa\msNb\msNc; 'nyam adhano \msCc, 'nyavidhayo \Ed\oo
                 cānyo\lem  \mssCaCbCc\msNa\msNb\msNc; cānyā \Ed\oo
                 haret\lem  \mssCaCbCc\msNb\msNc\Ed; hare \msNa}%

\nemslokac 
anye lekhyavikalpanāh\textsubring{r}tadhanā anyo h\textsubring{r}tād vai h\textsubring{r}tā
\dontdisplaylinenum
            \var{\vc anye lekhya°\lem  \corr; anyā lekha° \msCb\msCc; anyo le\uncl{khya}° \msCa, anyo lekhya° \msNa\msNb\msNc,
                                                                                                 anyollekhya \Ed\oo
                 °dhanā anyo\lem  \msCa\msCc\msNa\msNb\msNc\Ed; °dhanyo \msCb\oo
                 h\textsubring{r}tād vai\lem  \mssCaCbCc\msNc\Ed; h\textsubring{r}tad vai \msNa, h\textsubring{r}tād ve \msNb}%

\nemslokad 
anyaḥ krītadhano paro dhayah\textsubring{r}ta  ete jaghanyāḥ sm\textsubring{r}tāḥ \veg\dontdisplaylinenum
            \var{\vd anyaḥ krītadhano\lem  \mssCaCbCc\msNa\msNb; anya krītadhano \msNc, anāśrītadhana\.m \Ed\oo
                 paro dhayah\textsubring{r}ta\lem  \msCa\msCc\msNb; paro dhayahyata \msCb, paro dhana\uncl{h\textsubring{r}ta} \msNa, 
                                             parodhaprah\textsubring{r}ta \msNc, madā hy apah\textsubring{r}ta\.m \Ed\oo
                 jaghanyāḥ\lem  \mssCaCbCc\msNa\msNb\msNc; jaghanyaḥ \Ed}%

\ujvers\nemsloka 
stenastulya na mūḍham asti puruṣo dharmārthahīno 'dhamaḥ
\dontdisplaylinenum
            \var{\va stenastulya\lem  \Ed; stenatulya \msCa\msCb\msNc\ \unmetr, steyastulya \msCc, steyatulya \msNa\ \unmetr,
                                                                 tena tulya \msNb\ \unmetr}%

\nemslokab 
yāvaj jīvati śaṅkayā narapateḥ sa\.mtrasyamāno raṭan \danda\dontdisplaylinenum
            \var{\vb yāvaj jīvati\lem  \mssCaCbCc\msNa\msNb\msNc; yāvat taj jīvati \Ed\oo
                 °pateḥ\lem  \msCb\msNb\msNc; °patiḥ \msCa\msCc\msNa\Ed\oo
                 sa\.mtrasyamāno raṭan\lem  \mssCaCbCc\msNa\msNb\msNc; sa\.mtrāsyamāno śaṭhaḥ \Ed}%
            \paral{\textit{{\normalfont The lower folio side in exposure 49 in \msNb\ is rather blurred and seems to be partly erased,
                        therefore all the readings in this MS for verses 4.29--46 are rather uncertain,
                        even if not indicated explicitly.}}}

\nemslokac 
prāptaḥ śāsana tīvrasahyaviṣama\.m prāpnoti karmeritaḥ
\dontdisplaylinenum
            \var{\vc prāptaḥ\lem  \mssCaCbCc\msNb\msNc\Ed; prāpta° \msNa\oo
                 °sahya°\lem  \mssCaCbCc\msNa\msNc; {\lost}{\lost} \msNb, °sadya° \Ed\oo
                 °viṣama\.m\lem  \eme; °viṣamaḥ \mssCaCbCc\msNa\msNc\Ed, {\lost}{\lost}{\lost} \msNb\oo
                 karmeritaḥ\lem  \msCb\msCc\msNa\msNc\Ed; karme\uncl{rita} \msCa, {\lost}{\lost}\uncl{ritaḥ} \msNb}%

\nemslokad 
! kālena mriyate sa yāti nirayam ākrandamāno bh\textsubring{r}śam \veg\dontdisplaylinenum
            \var{\vd nirayam ākrandamāno\lem  \mssCaCbCc\msNa; \uncl{nira}yam ākrandamā\uncl{no} \msNb, 
                                                               niraya\.m sa krandamāno \msNc, niyamam ākrandramāno \Ed}%

\ujvers\nemsloka 
nītvā durgatikoṭikalpa nirayāt tiryatvam āyānti te
\dontdisplaylinenum
            \var{\va nirayāt tiryatva°\lem  \msCb\msNa; nirayān tiryatva° \msCa, nirayā tiryatva° \msCc, 
                       ni\uncl{rayāt tiryatva}° \msNb, nirayān tiryakṣa° \msNc, nirayān tiryaktva° \Ed}%

\nemslokab 
tiryatve ca tathaivam ekaśatika\.m prabhramya varṣārbudam \danda\dontdisplaylinenum
            \var{\vb tiryatve\lem  \mssCaCbCc\msNa\msNc; \uncl{tiryatve} \msNb, tiryaktva\.m \Ed\oo
                 tathaivam ekaśatika\.m\lem  \msCb; tathaikam ekaśatika\.m \msCa\msNa\msNc, tathaikam ekaśatika \msCc, 
                                                       \uncl{tathai}kam ekaśatika\.m \msNb, tathaikam ekasakika\.m \Ed\oo
                 °bhramya°\lem  \mssCaCbCc\msNc\Ed; °bhrāmya \msNa, °{\lost}{ā}mya \msNb\oo
                 varṣārbudam\lem  \msNcpcorr; varṣāmbudam \msCa\msCb\msNa\msNb\msNcacorr, varṣāmbudaḥ \msCc\Ed}%

\nemslokac 
mānuṣya\.m tad avāpnuvanti vipule dāridryarogākulam
\dontdisplaylinenum
            \var{\vc  mānuṣya\.m\lem  \msCa\msCc\msNa\msNc\Ed; mānuṣya \msCb\ \unmetr, \uncl{mānuṣya} \msNb\ \toplost\oo
                 vipule\lem  \mssCaCbCc\msNa\msNc; vipu\uncl{la} \msNb\ \toplost, vipula\.m \Ed\oo
                 dāridrya°\lem  \mssCaCbCc\msNa\msNc; {\il}ri{\il} \msNb, dāridhra° \Ed}%

\nemslokad 
tasmād durgatihetukarma sakala\.m tyaktvā śiva\.m cāśrayet \veg\dontdisplaylinenum
            \var{\vd tasmād du°\lem  \msCa\msCb\msNa\msNc\Ed; tasmā du° \msCc, \uncl{tasmā du°} \msNb\oo
                 cāśrayet\lem  \mssCaCbCc\msNb\msNc\Ed; cāśrat \msNa}%


\alalfejezet{yameṣv ān\textsubring{r}śa\.msyam (4)}
\vers

aṣṭamūrtiśivadveṣṭā pitur mātuś ca yo dviṣet\thinspace{\dandab} \dontdisplaylinenum
            \var{\va °śiva°\lem  \mssCaCbCc\msNa\msNb\Ed; °śiva\.m \msNc}%

gavā\.m vā atither dveṣṭā n\textsubring{r}śa\.msāḥ pañca eva te \veg\dontdisplaylinenum
            \var{\vc gavā\.m vā\lem  \msCa\msCc\msNa\msNc\Ed; avām vā \msCb, {\il}{\il}\uncl{m vā} \msNb\oo
                 atither dve°\lem  \msCa\msCb\msNb\msNc\Ed; atithidve° \msCc, atithe dve° \msNa}%
            \var{\vd n\textsubring{r}śa\.msāḥ\lem  \msCa\msCc\msNa\msNb; n\textsubring{r}śa\.msā \msCb\msNc\Ed}%

aṣṭamūrtiḥ śivaḥ sākṣāt pañcavyomasamanvitaḥ\thinspace{\dandab} \dontdisplaylinenum
            \var{\va °mūrtiḥ\lem  \mssCaCbCc\msNa\msNb\msNc; °mūrti° \Ed}%
            \var{\vb °nvitaḥ\lem  \msCa\msCb\msNa\msNc\Ed; °nvitāḥ \msCc\msNb}%

sūryaḥ somaś ca dīkṣaś ca dūṣakaḥ sa n\textsubring{r}śa\.msakaḥ \veg\dontdisplaylinenum
            \var{\vc sūryaḥ\lem  \mssCaCbCc\msNa; \uncl{sūrya°} \msNb\msNc, sūrya° \Ed\oo
                 dīkṣa°\lem  \mssCaCbCc\msNa\msNc; \uncl{dī}{\il} \msNb, dīkṣu° \Ed}%

pitākāśasamo jñeyo janmotpattikaraḥ pitā\thinspace{\dandab} \dontdisplaylinenum
            \var{\vb °karaḥ pitā\lem  \msCa\msCb\msNa\msNc\Ed;  °\uncl{karaḥ pitā} \msNb, °karapitāḥ \msCc}%

pit\textsubring{r}daivatam ādityam ān\textsubring{r}śa\.msa tamanvitaḥ \veg\dontdisplaylinenum
            \var{\vc °daivata°\lem  \msCa\msCc\msNa\msNc\Ed; °devata° \msCb, {\il}vata° \msNb}%
            \var{\vcd °dityam ān\textsubring{r}śa\.msa tamanvitaḥ\lem  \eme; °diścam ān\textsubring{r}śa\.msa tamanvitaḥ \msCa\msCb,
                                 °dityam an\textsubring{r}śa\.msa tamanvitaḥ \msCc\msNb,
                                 °diśca an\textsubring{r}śa\.msa tamānvitaḥ \msNa, 
                                 °diścam an\textsubring{r}śa\.msa tamānvitaḥ \msNc, 
                                 °dityam mān\textsubring{r}śa\.msa tato 'nvitaḥ \Ed}%

p\textsubring{r}thvyā\.m gurutarī mātā ko na vandeta mātaram\thinspace{\dandab} \dontdisplaylinenum
            \var{\va p\textsubring{r}thvyā\.m\lem  \Ed; p\textsubring{r}thvyā \msCa\msCb\msNc, \uncl{p\textsubring{r}thvyā} \msCc\msNa, p\textsubring{r}thvī \msNb}%
            \var{\vb vandeta\lem  \msCa\msNa\msNb\msNc\Ed; vandena vandeta \msCb, vandyeta \msCc}%

yajñadānatapovedās tena sarvak\textsubring{r}ta\.m bhavet \veg\dontdisplaylinenum

gāvaḥ pavitra\.m maṅgalya\.m devatānā\.m ca devatāḥ\thinspace{\dandab} \dontdisplaylinenum
            \var{\va pavitra\.m\lem  \mssCaCbCc\msNa\msNc\Ed; \uncl{pavitra} \msNb\oo
                 maṅgalya\.m\lem  \msCa\msCb\msNa; \uncl{maṅgalya\.m} \msNb, māṅgalya\.m \msCc\msNc\Ed\oo
                 devatāḥ\lem  \mssCaCbCc\msNc; daivatāḥ \msNa, \uncl{devatāḥ} \msNb, devatā \Ed}%
            \paral{\textit{\va {\normalfont \kb\ Viṣṇusm\textsubring{r}ti 23.57c: } gāvaḥ pavitramaṅgalya\.m (goṣu lokāḥ pratiṣṭhitāḥ)
                {\normalfont  cf.\ also MBh Indices 13.15.33: } gāvaḥ pavitra\.m parama\.m goṣu lokāḥ pratiṣṭhitāḥ 
                {\normalfont and Agnipurāṇa 291.1cd: } gāvaḥ pavitrā māṅgalyā goṣu lokāḥ pratiṣṭhitāḥ}}

sarvadevamayā gāvas tasmād eva na hi\.msayet \veg\dontdisplaylinenum
            \var{\vd °smād eva\lem  \msCa\msCc\msNa\msNb\msNc; °smād uva \msCb, °smād gāva\.m \Ed}%

jātamātrasya lokasya gāvas trātā na sa\.mśayaḥ\thinspace{\dandab} \dontdisplaylinenum
            \var{\va jātamātrasya lokasya\lem  \msCa\msCc\msNa\msNc\Ed; jātamātra\uncl{sya lokasya} \msNb, 
                                                        satasātasya \msCbacorr, satasātasya nokasya \msCbpcorr}%

gh\textsubring{r}ta\.m kṣīra\.m dadhi mūtra\.m śak\textsubring{r}t karṣaṇam eva ca \veg\dontdisplaylinenum
            \var{\vd śak\textsubring{r}t ka°\lem  \msCa\msCc\msNa\msNc\Ed; \uncl{śak\textsubring{r}t ka°} \msNb, kṣat ka° \msCb}%

\ujvers\nemsloka 
pañcām\textsubring{r}ta\.m pañcapavitrapūta\.m
\dontdisplaylinenum
            \var{\va °pavitrapūtam\lem  \msCc\msNa\Ed; °pavitrapūtana \msCa\ \unmetr, °pavitra\.m \msCb\ \unmetr, °pavitrapūta \msNb,
                                                                                                °pavitrapūta\.mna\.m \msNc\ \unmetr}%

\nemslokab 
ye pañcagavya\.m puruṣāḥ pibanti \danda\dontdisplaylinenum
            \var{\vb °gavya\.m\lem  \msCa\msCb\msNa\msNc\Ed; °gavyā \msCc, \uncl{°gavyā\.m} \msNb\oo
                 puruṣāḥ\lem  \msCa\msCb\msNa\msNb\msNc; puruṣā \msCc, puruṣaḥ \Ed\oo
                 pibanti\lem  \msCa\msCb\msNa\msNb\msNc\Ed; vivanti \msCc}%

\nemslokac 
te vājimedhasya phala\.m labhanti
\dontdisplaylinenum
            \var{\vc labhanti\lem  \msCa\msCb\msNa\msNb\msNc\Ed; bhavanti \msCc}%

\nemslokad 
tad akṣaya\.m svargam avāpnuvanti \veg\dontdisplaylinenum
            \var{\vd svarga°\lem  \msCa\msCc\msNa\msNb\msNc\Ed; sva° \msCb}%

\ujvers\nemsloka 
gobhir na tulya\.m dhanam asti ki\.mcid
\dontdisplaylinenum
            \var{\va gobhir na\lem  \msNc; gobhis tu° \mssCaCbCc\msNa\msNb\ \unmetr, gāvatu° \Ed}%
            \paral{\textit{\va {\normalfont cf. MBh 13.51.26cd: } gobhis tulya\.m na paśyāmi dhana\.m ki\.m cid ihācyuta}}

\nemslokab 
duhyanti vāhyanti bahiścaranti \danda\dontdisplaylinenum

\nemslokac 
t\textsubring{r}ṇāni bhuktvā am\textsubring{r}ta\.m sravanti
\dontdisplaylinenum

\nemslokad 
vipreṣu dattāḥ kulam uddharanti \veg\dontdisplaylinenum
            \var{\vd dattāḥ\lem  \msCa\msCb\msNa\msNb\msNc; \uncl{dattā} \msCc, dattā \Ed}%

\ujvers\nemsloka 
gavāhnika\.m yaś ca karoti nityam
\dontdisplaylinenum
            \var{\va gavāhnika\.m\lem  \msCb\msCc\msNa\msNb\msNc\Ed; gavā\.mhnika\.m \msCa\oo
                 yaś ca karoti\lem  \mssCaCbCc\msNa\msNb\msNc; yaḥ prakaroti \Ed}%

\nemslokab 
śuśrūṣaṇa\.m yaḥ kurute gavā\.m tu \danda\dontdisplaylinenum
            \var{\vb gavā\.m tu\lem  \msCb\msNc; gavān tu \msCa\msCc\msNa\msNb, gavānām \Ed}%

\nemslokac 
aśeṣayajñatapadānapuṇyam
\dontdisplaylinenum
            \var{\vc °tapa°\lem  \mssCaCbCc\msNa\msNc; \uncl{°tapa°} \msNb, °japa° \Ed}%

\nemslokad 
labhaty asau tam an\textsubring{r}śa\.msakartā \veg\dontdisplaylinenum
            \var{\vd labhaty asau tam an\textsubring{r}śa\.msakartā\lem  \msCb\msNa\msNb\msNc; labhaty asau bham an\textsubring{r}śa\.msakarttā \msCa,
                                 labhaty asau tam an\textsubring{r}ta\.m sa karttā \msCc,
                                 bhavaty asau dharmam aśeṣakartā \Ed}%

\vers

atithi\.m yo 'nugaccheta atithi\.m yo 'numanyate\thinspace{\dandab} \dontdisplaylinenum

atithi\.m yo 'nupūjyeta atithi\.m yaḥ praśa\.msate \veg\dontdisplaylinenum
            \var{\vd praśa\.msate\lem  \msCa\msCb\msNa\msNb\msNc\Ed; praśa\.msyate \msCc}%

atithi\.m yo na pīḍyeta atithi\.m yo na duṣyati\thinspace{\dandab} \dontdisplaylinenum
            \var{\va na pīḍyeta\lem  \msCa\msCb\msNa\Ed; na gaccheta \msCc\ \eyeskip{to 4.40c}, \uncl{na pī}{\il}{\il} \msNb,
                                                                                nipīḍyeta \msNc}%
            \var{\vb atithi\.m\lem  \msCa\msCb\msNa\msNc\Ed; ati\.m \msCc, {\il}{\il}{\il} \msNb\oo
                 na duṣyati\lem  \msCa\msCc\msNa\msNc\Ed; nuduṣyati \msCb, {\il}duṣyati \msNb}%

atithipriyakartā yaḥ atitheḥ paricārakaḥ \veg\dontdisplaylinenum
            \var{\vc atithi°\lem  \msCa\msNa; atithi\.m \msCb\msCc\msNc\Ed, ati\uncl{thi\.m} \msNb\oo
                 °priya°\lem  \msCa\msCb\msNa\msNb\msNc\Ed; priyaḥ \msCc\oo
                 yaḥ\lem  \msCb\msCc\msNb\msNc\Ed; yar \msCa, ya \msNa}%

atitheḥ k\textsubring{r}tasa\.mtoṣas tasya puṇyam anantakam\thinspace{\dandab} \dontdisplaylinenum
            \var{\va atitheḥ\lem  \msCb\msCc\msNc; atithi° \msCa\msNa\msNb, atithi\.m \Ed}%
            \var{\vab °sa\.mtoṣas tasya\lem  \msCa\msCc\msNa\msNb\msNc\Ed; °sa\.mtā yasya \msCb}%
            \var{\vb puṇya°\lem  \mssCaCbCc\msNa\msNb\Ed; pūna° \msNc}%

āsanenārghapādyena pādaśaucajalena ca \veg\dontdisplaylinenum
            \var{\vc °ārgha°\lem  \mssCaCbCc\msNa\msNb\msNc; °ārdhya° \Ed}%

annavastrapradānair vā sarva\.m vāpi nivedayet\thinspace{\dandab} \dontdisplaylinenum
            \var{\va annava°\lem  \msCa\msCb\msNa\msNc\Ed; annam va° \msCc, \uncl{anna}va° \msNb}%
            \var{\vb nivedayet\lem  \mssCaCbCc\msNa\msNb\msNc; pradāpayet \Ed}%

putradārātmano vāpi yo 'tithim anupūjayet \veg\dontdisplaylinenum
            \var{\vc °dārātmano\lem  \msCb\msCc\msNa\msNb\msNc; °\uncl{dārā}tmano \msCa, °dārātmako \Ed}%
            \var{\vd °pūjayet\lem  \msCa\msNa\Ed; °pūjyate \msCb\msCc\msNb, °pūjate \msNc}%

śraddhayā cāvikalpena aklībamānasena ca\thinspace{\dandab} \dontdisplaylinenum
            \var{\va śraddhayā\lem  \msCa\msCb\msNa\msNb\msNc\Ed; śraddhāyā \msCc\oo
                 cāvikalpena\lem  \msCb\msCc\msNa\msNb\msNc\Ed; cāpi kalpena \msCa}%

na p\textsubring{r}cched gotracaraṇa\.m svādhyāya\.m deśajanmanī \veg\dontdisplaylinenum
            \var{\vc °caraṇa\.m\lem  \mssCaCbCc\msNa\msNb\msNc; °pravara\.m \Ed}%
            \var{\vd deśajanmanī\lem  \msCb\msCc\msNa\msNb\msNc\Ed; deśajanmanā \msCa}%
            \paral{\textit{ {\normalfont \vcd cf.\ \MBh\ 13.62.18ab:
                } na p\textsubring{r}cched gotracaraṇa\.m svādhyāya\.m deśam eva vā}}

cintayen manasā bhaktyā dharmaḥ svayam ihāgataḥ\thinspace{\dandab} \dontdisplaylinenum
            \var{\va cintayen ma°\lem  \msCa\msCc\msNa\msNb\Ed; cittayet ma° \msCb, cintayet ma° \msNc}%
            \var{\vb °gataḥ\lem  \msCa\msCb\msNa\msNc\Ed; °gatāḥ \msCc, ga\uncl{tam} \msNb}%

aśvamedhasahasrāṇi rājasūyaśatāni ca \veg\dontdisplaylinenum
            \var{\vd °sūya°\lem  \msCa\msNa\msNc\Ed; °sūrya° \msCb\msCc, °sū\uncl{rya}° \msNb}%

puṇḍarīkasahasra\.m ca sarvatīrthatapaḥphalam\thinspace{\dandab} \dontdisplaylinenum
            \var{\vb °tapaḥ°\lem  \mssCaCbCc\msNa\msNb\Ed; °tapa° \msNc\ \unmetr}%

atithir yasya tuṣyeta n\textsubring{r}śa\.msamatam uts\textsubring{r}jet \veg\dontdisplaylinenum
            \var{\vd n\textsubring{r}śa\.msamatam uts\textsubring{r}jet\lem  \msCa\msNa\msNc; n\textsubring{r}śa\.msamata uts\textsubring{r}jet \msCb, 
                                     n\textsubring{r}śa\.msakamamam uts\textsubring{r}jet \msCc, n\textsubring{r}sasamatam uts\textsubring{r}jet \msNb, na sa\.mśaya samaśnute \Ed}%

sa tasya sakala\.m puṇya\.m prāpnuyān nātra sa\.mśayaḥ\thinspace{\dandab} \dontdisplaylinenum

na gatim atithijñasya gatim āpnoti karhicit \veg\dontdisplaylinenum
            \var{\vc na gatim a°\lem  \msCa\msCb\msNb\msNc; na gati nā° \msNa, na tithim a° \msCc\Ed}%
            \var{\vd karhicit\lem  \msCa\Ed; karhacit \msCb\msCc\msNa\msNb\msNc}%

tasmād atithim āyāntam abhigacchet k\textsubring{r}tāñjaliḥ\thinspace{\dandab} \dontdisplaylinenum
            \var{\va °yānta°\lem  \msCa\msCb\msNa\msNb\msNc\Ed; °yānti° \msCc}%

saktuprasthena caikena yajña āsīn mahādbhutaḥ \veg\dontdisplaylinenum
            \var{\vc saktu°\lem  \eme; śanku° \msCa\msCb, śa\.mktu° \msCc, śaktu° \msNa\msNc, śakthu° \msNb, śakti° \Ed\oo
                 caikena\lem  \mssCaCbCc\msNa\msNb\Ed; cekena \msNc}%
            \var{\vd āsīn mahādbhutaḥ\lem  \corr; āsīn mahadbhutaḥ \msCa\msCb\msNa\msNb, āsī mahadbhutaḥ \msCc, 
                                               āsīt mahādbhutaḥ \msNc, āsīn mahadbhutam \Ed}%

atithiprāptadānena svaśarīra\.m diva\.mgatam\thinspace{\dandab} \dontdisplaylinenum
            \var{\va °dānena\lem  \msCa\msCb\msNa\msNb\msNc\Ed; °prādānena \msCc}%
            \var{\vb sva°\lem  \mssCaCbCc\msNa\msNb; \uncl{sa°} \msNc, sa° \Ed\oo
                 °gatam\lem  \msCa\msCb\msNa\msNb\msNc\Ed; °gataḥ \msCc}%

nakulena purādhīta\.m vistareṇa dvijottama \danda\dontdisplaylinenum
            \var{\vd °ttama\lem  \msCa\msCb\msNa\msNb\msNc; °ttamam \msCc, °ttamaḥ \Ed}%

vidita\.m ca tvayā pūrva\.m prasthavārttā ca kīrtitā \veg\dontdisplaylinenum
            \var{\vf kīrtitā\lem  \msCa\msCb\msNa\msNb\msNc; kīrtitam \msCc, kīrtitāḥ \Ed}%


\alalfejezet{yameṣu damaḥ (5)}
dama eva manuṣyāṇā\.m dharmasārasamuccayaḥ\thinspace{\dandab} \dontdisplaylinenum
            \var{\vb dharmasāra°\lem  \eme; dharmaḥ sāra° \mssCaCbCc\msNa\msNb\msNc, dharmabhāra° \Ed}%

damo dharmo damaḥ svargo damaḥ kīrtir damaḥ sukham \veg\dontdisplaylinenum
            \var{\vc svargo\lem  \msCa\msCb\msNa\msNb\msNc\Ed; svarga \msCc}%
            \var{\vd kīrtir da°\lem  \msCa\msCb\msNb\Ed; kīrti da° \msCc\msNa\msNc}%

damo yajño damas tīrtha\.m damaḥ puṇya\.m damas tapaḥ\thinspace{\dandab} \dontdisplaylinenum
            \var{\va damas tī°\lem  \msCa\msCc\msNa\msNb\msNc\Ed; dama tī° \msCb}%

damahīnam adharmaś ca damaḥ kāmakulapradaḥ \veg\dontdisplaylinenum
            \var{\vd damaḥ\lem  \msCa\msCb\msNa\msNb\msNc; dama \msCc, dama\.m \Ed\oo
                 kāma°\lem  \mssCaCbCc\msNa\msNb\Ed; kāma\.m \msNc}%

nirdamaḥ kari mīnaś ca pataṅgabhramaram\textsubring{r}gāḥ\thinspace{\dandab} \dontdisplaylinenum
            \var{\va °damaḥ\lem  \msCa\msCb\msNa\msNb\msNc\Ed; °dama \msCc}%
            \var{\vb °bhramara°\lem  \mssCaCbCc\msNa\msNb\Ed\ \unmetr; °bhrama\uncl{rā}° \msNc}%

tvag jihvā ca tathā ghrāṇā cakṣuḥ śravaṇam indriyāḥ \veg\dontdisplaylinenum
            \var{\vc ghrāṇā\lem  \msCa\msNa\msNb\msNc\Ed; ghrāṇa\.m \msCb, ghrāṇa \msCc}%
            \var{\vd °ndriyāḥ\lem  \mssCaCbCc\msNa\msNb\msNc; °ndriyaḥ \Ed}%

durjayendriyam ekaika\.m sarve prāṇaharāḥ sm\textsubring{r}tāḥ\thinspace{\dandab} \dontdisplaylinenum
            \var{\vb sarve\lem  \msCa\msCc\msNa\msNb\msNc\Ed; sarva° \msCb\oo
                 °harāḥ\lem  \mssCaCbCc\msNa\msNb\msNc; °harā \Ed}%

dama\.m yo jayate samyag nirdamo nidhana\.m vrajet \veg\dontdisplaylinenum

m\textsubring{r}ge śrotravaśān m\textsubring{r}tyuḥ pataṅgāś cakṣuṣor m\textsubring{r}tāḥ\thinspace{\dandab} \dontdisplaylinenum
            \var{\va m\textsubring{r}ge\lem  \mssCaCbCc\msNa\msNc; m\textsubring{r}go \msNb\Ed\oo
                 śrotra°\lem  \msCa\msCb\msNa\msNb\Ed; śotra° \msCc, śrota° \msNc\oo
                 °vaśā°\lem  \msCa\msCc\msNa\msNb\msNc\Ed; °vacaśā° \msCb}%
            \var{\vb pataṅgāś ca°\lem  \mssCaCbCc\msNa\msNb\msNc; pataṅgā ca° \Ed\oo
                 °ṣor m\textsubring{r}tāḥ\lem  \msCa\msCb\msNa\msNb\Ed; °so m\textsubring{r}tāḥ \msCc, °ṣo m\textsubring{r}tāḥ \msNc}%

ghrāṇayā bhramaro naṣṭo naṣṭo mīnaś ca jihvayā \veg\dontdisplaylinenum
            \var{\vc ghrāṇayā\lem  \msCa\msCc\msNa\msNb\msNc\Ed; ghrātayā \msCb}%
            \var{\vcd naṣṭo naṣṭo\lem  \msCa\msCc\msNa\msNb\msNc\Ed; naṣṭo \msCb}%

sparśena ca karī naṣṭo bandhanāvāsaduḥsahaḥ\thinspace{\dandab} \dontdisplaylinenum
            \var{\vb °saduḥsahaḥ\lem  \msCa\msCc\msNa\msNc\Ed; °saduḥsaha \msCb, °sudussahaḥ \msNb}%

ki\.m punaḥ pañcabhuktānā\.m m\textsubring{r}tyus tebhyaḥ kim adbhutam \veg\dontdisplaylinenum
            \var{\vc punaḥ\lem  \msCapcorr\msCb\msCc\msNa\msNb\msNc\Ed; puna \msCaacorr}%
            \var{\vd tebhyaḥ\lem  \mssCaCbCc\msNa\msNb\msNc; tebhya \Ed}%

purūravātilobhena atikāmena daṇḍakaḥ\thinspace{\dandab} \dontdisplaylinenum
            \var{\va purūravā°\lem  \corr; purūravo \msCa\msCb\msNa\msNb\msNc, purorave \msCc, pururavā° \Ed\oo
                 °tilobhena atikāmena\lem  \mssCaCbCc\msNa\msNb\msNc; °tikāmena atilobhena \Ed}%
            \var{\vb daṇḍakaḥ\lem  \mssCaCbCc\msNa\msNb\msNc; puṇḍakaḥ \Ed}%

sagaraś cātidarpeṇa atimānena rāvaṇaḥ \veg\dontdisplaylinenum
            \var{\va sagara°\lem  \msCa\msCb\msNa\msNb\msNc\Ed; sāgara° \msCc}%

atikrodhena saudāsa atipānena yādavāḥ\thinspace{\dandab} \dontdisplaylinenum
            \var{\vb atipānena\lem  \mssCaCbCc\msNa\msNb\msNc; atipāpena \Ed}%

atit\textsubring{r}ṣṇāc ca māndhāto nahuṣo dvijavajñayā \veg\dontdisplaylinenum
            \var{\vc atit\textsubring{r}ṣṇāc ca māndhātā\lem  \conj;
                atit\textsubring{r}ṣṇā ca māndāto \msCa,
                atit\textsubring{r}ṣṇā ca māndhāto \msCb\msCc\msNa\msNc,
                atit\textsubring{r}ṣṇā ca mandhāto \msNb,
                atit\textsubring{r}ṣṇā ca mānāc ca ca \Ed}%
            \var{\vd nahuṣo\lem  \mssCaCbCc\msNa\msNc\Ed; naghuṣo \msNb}%

atidānād balir naṣṭa atiśauryeṇa arjunaḥ\thinspace{\dandab} \dontdisplaylinenum
            \var{\va °r naṣṭa\lem  \msCa\msNa\msNb\msNc\Ed; °r naṣṭo \msCb, naṣṭo \msCc}%

atidyūtān nalo rājā n\textsubring{r}go goharaṇena tu \veg\dontdisplaylinenum
            \var{\vc atidyūtān nalo\lem  \msCa\msCc\msNb\msNc; atidyūtān naro \msCb\msNa, tikhyātān nalo \Ed}%
            \var{\vd n\textsubring{r}go go°\lem  \Ed; n\textsubring{r}gaṅ go° \msCa\msCc\msNb\msNc, n\textsubring{r}ga\.m go° \msCb\msNa}%
            \paral{\textit{\vo {\normalfont After this verse, \Ed\ adds: }
                        tasmād dama sadā sa rakṣet ati sarvatra varjayet
                       {\normalfont (understand: } tasmād dama\.m sadā rakṣet ati sarvatra varjayet{\normalfont )}}}

\ujvers\nemsloka 
damena hīnaḥ puruṣo dvijendra
\dontdisplaylinenum
            \var{\va hīnaḥ puruṣo dvijendra\lem  \mssCaCbCc\msNa\msNc; hīna puruṣo dvijendra \msNb, hīna\.m puruṣa\.m dvijendraḥ \Ed}%

\nemslokab 
svarga\.m ca mokṣa\.m ca sukha\.m ca nāsti \danda\dontdisplaylinenum

\nemslokac 
vijñānadharmakulakīrtināśo
\dontdisplaylinenum
            \var{\vc °nāśo\lem  \Ed; °nāma \msCa\msCc\msNa, °nāśa \msCb, °naśca \msNb, °nāgā \msNc}%

\nemslokad 
bhavanti viprā damayā vihīnāḥ \veg\dontdisplaylinenum
            \var{\vd viprā\lem  \msNapcorr\msNc; vipra \mssCaCbCc\msNaacorr\msNb\Ed\oo
                 damayā\lem  \msCa\msCbpcorr\msCc\msNa\msNb\msNc\Ed; dayā \msCbacorr}%


\alalfejezet{yameṣu gh\textsubring{r}ṇā (6)}
\vers

nirgh\textsubring{r}ṇo na paratrāsti nirgh\textsubring{r}ṇo na ihāsti vai\thinspace{\dandab} \dontdisplaylinenum
            \var{\va nirgh\textsubring{r}ṇo\lem  \msCa\msCb\msNb; nigh\textsubring{r}ṇo \msCc\msNc, nirgh\textsubring{r}ṇa \msNaacorr, 
                                nirgh\textsubring{r}\uncl{ṇe} \msNapcorr, nirgh\textsubring{r}ṇe \Ed}%
            \var{\vb nirgh\textsubring{r}ṇo\lem  \msCa\msCb\msNaacorr\msNb; nigh\textsubring{r}ṇo \msCc\msNc, nirgh\textsubring{r}ṇe \msNapcorr\Ed}%

nirgh\textsubring{r}ṇe na ca dharmo 'sti nirgh\textsubring{r}ṇe na tapo 'sti vai \veg\dontdisplaylinenum
            \var{\vc nirgh\textsubring{r}ṇe\lem  \msCa\msCb\msNb\Ed; nigh\textsubring{r}ṇe \msCc\msNa\msNc}%
            \var{\vd nirgh\textsubring{r}ṇe\lem  \msCa\msCb\msNa\msNb\Ed; nigh\textsubring{r}ṇe \msCc\msNc}%

parastrīṣu parārtheṣu parajīvāpakarṣaṇe\thinspace{\dandab} \dontdisplaylinenum
            \var{\vb °jīvāpakarṣaṇe\lem  \msCa\msCc\msNa\msNb\msNc; °jīvāparkaṇe \msCb, °jīvopakarṣaṇe \Ed}%

paranindāparānneṣu gh\textsubring{r}ṇā\.m pañcasu kārayet \veg\dontdisplaylinenum
            \var{\vc paranindā°\lem  \msCb\msCc\msNa\msNb\msNc\Ed; paranind{\il}° \msCa\oo
                 °parānneṣu\lem  \mssCaCbCc\msNa\msNc\Ed; °parā\.mneṣu \msNb}%
            \var{\vd gh\textsubring{r}ṇā\.m\lem  \msCa\msCb\msNa\msNc; gh\textsubring{r}ṇā \msCc\msNb\Ed}%

parastrī ś\textsubring{r}ṇu viprendra gh\textsubring{r}ṇīkāryā sadā budhaiḥ\thinspace{\dandab} \dontdisplaylinenum
            \var{\va gh\textsubring{r}ṇī°\lem  \msCa\msCc\msNa\msNb\msNc\Ed; gh\textsubring{r}ṇā \msCb}%

rājñī viprī parivrājā svayoniparayoniṣu \veg\dontdisplaylinenum
            \var{\vc °vrājā\lem  \mssCaCbCc\msNc; °vrājī \msNa\msNb, °vrājyā \Ed}%
            \var{\vd °para°\lem  \mssCaCbCc\msNa\msNc\Ed; °paśu° \msNb}%

parārthe ś\textsubring{r}ṇu bhūyo 'nya anyāyārtham upārjanam\thinspace{\dandab} \dontdisplaylinenum
            \var{\vb anyāyā°\lem  \mssCaCbCc\msNa\msNc\Ed; anyayā° \msNb\oo
                 °rjanam\lem  \mssCaCbCc\msNa\msNc\Ed; °rjjavam \msNb}%

āḍhaprasthatulāvyājaiḥ parārtha\.m yo 'pakarṣati \veg\dontdisplaylinenum
            \var{\vc °tulā°\lem  \mssCaCbCc\msNa\msNc\Ed; °tula° \msNb}%
            \var{\vd °rtha\.m\lem  \msCa\msCb\msNa\Ed; °rtha \msCc, \uncl{°rtha} \msNb, °rthe \msNc}%

jīvāpakarṣaṇe vipra gh\textsubring{r}ṇīkurvīta paṇḍitaḥ\thinspace{\dandab} \dontdisplaylinenum
            \var{\va vipra\lem  \msCb\msNa\msNb\msNc\Ed; vi\uncl{pra} \msCa, vipre \msCc}%
            \var{\vb gh\textsubring{r}ṇī°\lem  \mssCaCbCc\msNa\msNb\msNc; gh\textsubring{r}ṇā\.m \Ed}%

vanajā vanajā jīvā vihagācaraṇācarāḥ \veg\dontdisplaylinenum
            \var{\vc vanajā vanajā\lem  \msCa\msCc\msNa\msNb\Ed; vanajā va{\il}jā \msCbacorr, vanajā va\uncl{ni}jā \msCbpcorr,
                                                vanaja vinajā \msNc}%
            \var{\vd vihagācaraṇācarāḥ\lem  \conj; vilagācaraṇācarāḥ \msCa\msCb\msNc;
                                                        vilagocaragocaraḥ \msCc\Ed, vilagocaragocarāḥ \msNa,
                                                        \uncl{vilagācara}ṇācarāḥ \msNb}%

paranindā ca kā vipra ś\textsubring{r}ṇu vakṣye samāsataḥ\thinspace{\dandab} \dontdisplaylinenum
            \var{\vb vakṣye\lem  \mssCaCbCc\msNa\msNb\msNc; vakṣyā \Ed}%

devānā\.m brāhmaṇānā\.m ca gurumātātithidviṣaḥ \veg\dontdisplaylinenum
            \paral{\textit{\vcd {\normalfont These two pādas are illegible in \msNb.}}}

parānneṣu gh\textsubring{r}ṇā kāryā abhojyeṣu ca bhojanam\thinspace{\dandab} \dontdisplaylinenum
            \var{\vb abhojyeṣu\lem  \msCa\msCc\msNa\msNb\msNc\Ed; abhojye \msCb}%

sūtake m\textsubring{r}take śauṇḍe varṇabhraṣṭakule naṭe \veg\dontdisplaylinenum
            \var{\vc śauṇḍe\lem  \msNa; sauṇḍye \msCa\msCc\msNc, śoṇḍye \msCb, \uncl{sauṇḍe} \msNb, sauṇḍo \Ed}%
            \paral{\textit{\vo {\normalfont This verse is mostly illegible in \msNb.}}}

\ujvers\nemsloka 
ete pañcagh\textsubring{r}ṇāsu saktapuruṣāḥ svargārthamokṣārthinaḥ
\dontdisplaylinenum
            \var{\va °puruṣāḥ\lem  \msNc; °puruṣaḥ \mssCaCbCc\msNa\msNb\Ed\oo
                 °rthinaḥ\lem  \msNcpcorr; °rthinā\.m \mssCaCbCc\msNa\msNb\Ed, °rthinā \msNcacorr}%

\nemslokab 
loke 'nindanam āpnuvanti satata\.m kīrtir yaśo'la\.mk\textsubring{r}tam \danda\dontdisplaylinenum
            \var{\vb 'nindanam āpnuvanti\lem  \msCa\msCb\msNa\msNb\msNc; 'nindanavāpnuvanti \msCc, nandanavāyuvānti \Ed}%

\nemslokac 
prajñābodhaśruti\.m sm\textsubring{r}ti\.m ca labhate māna\.m ca nitya\.m labhet
\dontdisplaylinenum
            \var{\vc °śruti\.m\lem  \msNc; °śruti° \mssCaCbCc\msNa\msNb\Ed\oo
                nitya\.m\lem  \msCa\msCc\msNa\msNb\msNc\Ed; nitya \msCb}%

\nemslokad 
dākṣiṇya\.m sa bhavet sa āyuṣa para\.m prāpnoti niḥsa\.mśayaḥ \veg\dontdisplaylinenum
            \var{\vd sa āyuṣa\lem  \eme; samāyuṣa \mssCaCbCc\msNc, samāyuṣaḥ \msNa\ \unmetr, \uncl{samāyuṣa} \msNb, sa mānuṣa \Ed\oo
                 niḥsa\.mśayaḥ\lem  \mssCaCbCc\msNb\msNc\Ed; nisa\.mśayaḥ \msNa}%


\alalfejezet{yameṣu pañcadhanyavidhiḥ (7)}
\vers

caturmaunaś catuḥśatruś caturāyatana\.m tathā\thinspace{\dandab} \dontdisplaylinenum
            \var{\va caturmauna°\lem  \msCa\msCb\msNa\msNc\Ed; caturmoṇa° \msCc, \uncl{caturmauna°} \msNb}%
            \var{\vab °tuḥ śatruś ca°\lem  \msCa\msCb\msNa\msNb\msNc; °tuśatru ca° \msCc, °tuḥ śatru ca° \Ed}%
            \var{\vb °turāyatana\.m\lem  \msCb\msCc\msNa\msNc\Ed; °\uncl{tu}rāyatana\.m \msCa, \uncl{caturāyatanam} \msNb}%

catur dhyāna\.m catuṣpāda\.m pañcadhanyavidhocyate \veg\dontdisplaylinenum
            \var{\vc °pāda\.m\lem  \mssCaCbCc\msNc\Ed; °pādaḥ \msNa, {\il}{\il} \msNb}%
            \var{\vd pañcadhanya°\lem  \mssCaCbCc\msNa\msNb\msNc; dhanyapañca° \Ed}%

caturmaunasya vakṣyāmi ś\textsubring{r}ṇuṣvāvahito bhava\thinspace{\dandab} \dontdisplaylinenum
            \var{\va °maunasya\lem  \msCa\msCc\msNa\msNb\msNc\Ed; °monasya \msCb}%

pāruṣyapiśunāmithyāsambhinnāni ca varjayet \veg\dontdisplaylinenum
            \var{\vc pāruṣya°\lem  \mssCaCbCc\msNb\msNc\Ed; pāruṣya\.m \msNa\oo
                 °piśunā°\lem  \mssCaCbCc\msNa\msNb\msNc; °piṇḍānā° \Ed}%

kāmaḥ krodhaś ca lobhaś ca mohaś caiva caturvidhaḥ\thinspace{\dandab} \dontdisplaylinenum 

catuḥśatrur nihantavyaḥ so 'rihā vītakalmaṣaḥ \veg\dontdisplaylinenum
            \var{\vc catuḥśatrur ni°\lem  \msCa\msCb\Ed; catuśatru ni° \msCc\msNa\msNb\msNc}%
            \var{\vd so 'rihā\lem  \msCa\msCc\msNa\msNb\msNc; srorihā \msCb, sarvathā \Ed}%

caturāyatana\.m vipra kathayiṣyāmi tac ch\textsubring{r}ṇu\thinspace{\dandab} \dontdisplaylinenum

karuṇāmuditopekṣāmaitrī cāyātana\.m sm\textsubring{r}tam \veg\dontdisplaylinenum
            \var{\vc mudito°\lem  \mssCaCbCc\msNa\msNb\msNc; muditau° \Ed}%
            \var{\vd cāyatana\.m\lem  \msCc\msNa\msNb\msNc\Ed; cāyatana \msCa, cāyata\uncl{na} \msCb}%

catur dhyānādhunā vakṣye sa\.msārārṇavatāraṇam\thinspace{\dandab} \dontdisplaylinenum

ātmavidyābhavaḥ sūkṣma\.m dhyānam ukta\.m caturvidham \veg\dontdisplaylinenum
            \var{\vc  °bhavaḥ\lem  \msCb\msCcpcorr\msNa\msNb\msNc; °bhava \msCa\msCcacorr, °bhava\.m \Ed}%
            \var{\vcd sūkṣma\.m dhyā°\lem  \msCa\msNa\msNc\Ed; 
                                        sūkṣmā\uncl{nyā}° \msCb, sū\uncl{kṣma}dhyā° \msCc, sūkṣmadhyāna° \msNb}%
            \var{\vd  °nam ukta\.m caturvidham\lem  \msCc\msNb; °nam uktaś caturvidham \msCa, °nam uktaś caturvidhaḥ \msCb\msNa, 
                                                                      °nam ukta\.m caturvidhi\.m \msNc, °nayajñaś ca \Ed}%

ātmatattvaḥ sm\textsubring{r}to dharmo vidyā pañcasu pañcadhā\thinspace{\dandab} \dontdisplaylinenum
            \var{\va sm\textsubring{r}to\lem  \msCa\msCb\msNa\msNb\msNc; sm\textsubring{r}tā \msCc\Ed\oo
                 dharmo\lem  \mssCaCbCc\msNa\msNb\msNc; dhanyā \Ed}%

ṣaṭtri\.mśākṣaram ityāhuḥ sūkṣmatattvam alakṣaṇam \veg\dontdisplaylinenum
            \var{\vcd āhuḥ sū°\lem  \msCb\msCc\msNa\msNb\msNc\Ed; ā{\il}{\il} \msCa}%

catuṣpādaḥ sm\textsubring{r}to dharmaś caturāśramam āśritaḥ\thinspace{\dandab} \dontdisplaylinenum
            \var{\vab dharmaś ca°\lem  \msCa\msCb\msNa\msNc\Ed; dharma ca° \msCc\msNb}%
            \var{\vb °śritaḥ\lem  \mssCaCbCc\msNa\msNb\Ed; °śritāḥ \msNc}%

g\textsubring{r}hastho brahmacārī ca vānaprastho 'tha bhaikṣukaḥ \veg\dontdisplaylinenum
            \var{\vd bhaikṣukaḥ\lem  \mssCaCbCc\msNa\msNb\msNc; bhakṣakaḥ \Ed}%
            \paral{\textit{\vcd {\normalfont  = MBh 12.234.13ab \kb\ MBh 14.4513ab etc. } }}

dhanyās te yair ida\.m vetti nikhilena dvijottama\thinspace{\dandab} \dontdisplaylinenum
            \var{\va yair ida\.m\lem  \msCa\msNa\msNb\msNc\Ed; yer ida\.m \msCb\msCc\oo
                 vetti\lem  \msCa\msCb\msNa\msNb\msNc\Ed; veti \msCc}%

pāvana\.m sarvapāpānā\.m puṇyānā\.m ca pravardhanam \veg\dontdisplaylinenum
            \var{\vd pravardhanam\lem  \mssCaCbCc\msNa\msNb\msNc; pravardhanaḥ \Ed}%

āyuḥ kīrtir yaśaḥ saukhya\.m dhanyād eva pravardhate\thinspace{\dandab} \dontdisplaylinenum
            \var{\vb dhanyād eva\lem  \mssCaCbCc\msNa\msNb\msNc; dharmād eva \Ed}%

śāntiḥ puṣṭiḥ sm\textsubring{r}tir medhā jāyate dhanyamānavaḥ \veg\dontdisplaylinenum
            \var{\vc puṣṭiḥ\lem  \msCb\msCc\msNa\msNb\msNc\Ed; {\il}ṣṭiḥ \msCa\oo
                 sm\textsubring{r}tir medhā\lem  \msCa\msCb\msNb\msNc\Ed; sm\textsubring{r}ti medhā \msCc\msNa}%


\alalfejezet{yameṣv apramādaḥ (8)}
pramādasthāna pañcaiva kīrtayiṣyāmi tac ch\textsubring{r}ṇu\thinspace{\dandab} \dontdisplaylinenum
            \var{\va °sthāna\lem  \msCa\msCc\msNa\msNb; °sthāna\.m \msCb\msNc\Ed\ \unmetr\oo
                 pañcaiva\lem  \mssCaCbCc\msNa\msNb\msNc; pañcaiva\.m \Ed}%
            \var{\vb kīrtayiṣyāmi\lem  \mssCaCbCc\msNa\msNc\Ed; kīrtiyiṣyāmi \msNb}%

brahmahatyā surāpāna\.m steyo gurvaṅganāgamam \danda\dontdisplaylinenum

mahāpātakam ity āhus tatsa\.myogī ca pañcamaḥ \veg\dontdisplaylinenum
            \paral{\textit{\vcdef {\normalfont \kb\ \MBh\ Indeces 12.30: }
                    brahmahatyā\.m surāpāna\.m steya\.m gurvaṅganāgamam{\thinspace\danda}
                    mahānti pātakāny āhuḥ sa\.myoga\.m caiva taiḥ saha{\thinspace\ketdanda}
                    {\normalfont  \kb\ also Manu 11.54: }
                    brahmahatyā surāpāna\.m steya\.m gurvaṅganāgamaḥ{\thinspace\danda}
                    mahānti pātakāny āhuḥ sa\.msargaś cāpi taiḥ saha{\thinspace\ketdanda}}}

an\textsubring{r}ta\.m ca samutkarṣa\.m rājagāmī ca paiśunaḥ\thinspace{\dandab} \dontdisplaylinenum
            \var{\va samutkarṣa\.m\lem  \msCa\msNa; samutkarṣa \msCc\msNb\msNc\Ed, samutka\uncl{rṣa} \msCb}%
            \var{\vb rāja°\lem  \mssCaCbCc\msNa\msNb\msNc; rājñī° \Ed}%

guroś cālīkanirbaddhaḥ samāni brahmahatyayā \veg\dontdisplaylinenum
            \var{\vc °nirbaddhaḥ\lem  \msCa\msCb\msNc; nibaddhas \msCa\msCc\msNa\msNb, nirvaddhas \Ed}%
            \var{\vd brahmahatyayā\lem  \msCb\msCc\msNa\msNb\msNc\Ed; bra{\il}{\il}{\il}yā \msCa}%
            \paral{\textit{\vo \kb\ {\normalfont MBh 5.40.3: }
                 an\textsubring{r}ta\.m ca samutkarṣe rājagāmi ca paiśunam{\thinspace\danda}
                 guroś cālīkanirbandhaḥ samāni brahmahatyayā{\thinspace\ketdanda}
                        {\normalfont = Manu 11.55 \kb\ Viṣṇusm\textsubring{r}ti 37.1--4 \kb\ Agnipurāṇa 168.25} }}

brahmo \textsubring{r}gvedanindā ca kūṭasākṣī suh\textsubring{r}d vadhaḥ\thinspace{\dandab} \dontdisplaylinenum
            \var{\va brahmo\lem  \mssCaCbCc\msNa\msNb\msNc; brahma \Ed}%
            \var{\vb suh\textsubring{r}d vadhaḥ\lem  \mssCaCbCc\msNa\msNb\msNc; sak\textsubring{r}d budhaḥ \Ed}%

garhitānādyayor jagdhiḥ surāpānasamāni ṣaṭ \veg\dontdisplaylinenum
            \var{\vc °nādyayor jagdhiḥ\lem  \eme; °nnañ cayo jagdhis \msCa, °nnañ cayo jagdhi \msCb,
                                 °nnañ cayodvignaḥ \msCc, °nna\.mcayo jagdhiḥ \msNa, °nna\.mcayo jagdhiḥ \msNb,
                                 °nnañ cayojave \msNc, °nnaś ca yo vipraḥ \Ed}%
            \paral{\textit{\vo \kb\ {\normalfont Manu 11.56: }
                brahmojjhatā vedanindā kauṭasākṣya\.m suh\textsubring{r}dvadhaḥ{\thinspace\danda}
                garhitānādyayor jagdhiḥ surāpānasamāni ṣaṭ{\thinspace\ketdanda}}}

retotsekaḥ svayonyāsu kumārīṣv antyajāsu ca\thinspace{\dandab} \dontdisplaylinenum
            \var{\va svayonyāsu\lem  \msCa\msCc\msNa\msNb\msNc\Ed; sutonyāsu \msCb}%

sakhyuḥ putrasya ca strīṣu gurutalpasamaḥ sm\textsubring{r}taḥ \veg\dontdisplaylinenum
            \var{\vc sakhyuḥ\lem  \eme; sakhya \mssCaCbCc\msNa\Ed, {\il}{\il} \msNb, sa\uncl{khyu} \msNc\oo
                 putrasya ca strīṣu\lem  \mssCaCbCc\msNa\msNc; {\il}{\il}{\il}{\il}{\il}{\il} \msNb, putrīṣu cāstrīṣu \Ed}%
            \var{\vd °samaḥ\lem  \mssCaCbCc\msNa\msNc; {\il}{\il} \msNb, °sama \Ed}%
            \paral{\textit{\vo \kb\ {\normalfont Manu 11.58: }
                                retaḥsekaḥ svayonīṣu kumārīṣv antyajāsu ca{\thinspace\danda}
                                sakhyuḥ putrasya ca strīṣu gurutalpasama\.m viduḥ{\thinspace\ketdanda}}}

nikṣepasyāpaharaṇa\.m narāśvarajatasya ca\thinspace{\dandab} \dontdisplaylinenum
            \var{\va nikṣepa°\lem  \msCa\msCc\msNa\msNc\Ed; \uncl{nikṣepa°} \msNb, nikhepa° \msCb}%
            \var{\vb narāśvarajatasya\lem  \msCa\msCc\msNa\msNc\Ed; \uncl{narāśvarajatasya} \msNb, narāṇā\.m svajanasya \msCb}%

bhūmivajramaṇīnā\.m ca rukmasteyasamaḥ sm\textsubring{r}taḥ \veg\dontdisplaylinenum
            \var{\vd rukmasteya°\lem  \eme; \uncl{rūgya}{\il}ya° \msCa, rugmasteya° \msCb\msCc\msNa\msNc, {\il}{\il}{\il}{\il} \msNb, h\textsubring{r}tasteya° \Ed\oo
                 °samaḥ\lem  \msCa\msCbpcorr\msCc\msNa\msNb\msNc; saḥ \msCbacorr, °sama \Ed}%
            \paral{\textit{\vo  {\normalfont = Manu 11.57: }
                        nikṣepasyāpaharaṇa\.m narāśvarajatasya ca{\thinspace\danda}
                        bhūmivajramaṇīnā\.m ca rukmasteyasama\.m sm\textsubring{r}tam{\thinspace\ketdanda}}}

catvāra ete sambhūya yat pāpa\.m kurute naraḥ\thinspace{\dandab} \dontdisplaylinenum
            \var{\va ete\lem  \mssCaCbCc\msNa\msNc; \uncl{ete} \msNb, eva \Ed\oo
                 sambhūya\lem  \msCa\msCb\msNa\msNc\Ed; sa\.mbhūyo \msCc, \uncl{sa\.mbhūyo} \msNb}%

mahāpātakapañcaitan tena sarva\.m prakāśitam \danda\dontdisplaylinenum
            \var{\vc °pañcaitan\lem  \mssCaCbCc\Ed; °pañcaitam \msNb, °pañcetan \msNc, °pañcaite \msNa}%

pañcapramādam etāni varjanīya\.m dvijottama \veg\dontdisplaylinenum
            \var{\ve °mādam\lem  \mssCaCbCc\msNa\msNb\msNc; °māda \Ed}%
            \var{\vf varjanīya\.m\lem  \msCa\msCb\msNa\msNb\msNc\Ed; varjanīyo \msCc}%


\alalfejezet{yameṣu mādhuryam (9)}
kāyavāṅmanamādhuryaś cakṣur buddhiś ca pañcamaḥ\thinspace{\dandab} \dontdisplaylinenum
            \var{\vab manamādhurya\.m ca°\lem  \eme; °manasā dhūryaś ca° \mssCaCbCc\msNa\msNc,
                                                       °mana{\il}dhūrya{\il}° \msNb, °manasā bhūyaś ca° \Ed}%
            \var{\vb °kṣur buddhi°\lem  \msCa\msCb\msNc\Ed; °kṣu buddhi° \msCc\msNa, {\il}{\il}{\il} \msNb}%

saumyad\textsubring{r}ṣṭipradāna\.m ca krūrabuddhi\.m ca varjayet \veg\dontdisplaylinenum
            \var{\vc °dāna\.m\lem  \mssCaCbCc\msNa\msNc; {\il}{\il} \msNb, °dānaś \Ed}%
            \var{\vd °buddhi\.m ca\lem  \msCa\msNa\msNc; buddhiś ca \msCb, °d\textsubring{r}ṣṭi\.m ca \msCc\Ed, {\il}{\il}{\il} \msNb}%

prasannamanasā dhyāyet priyavākyam udīrayet\thinspace{\dandab} \dontdisplaylinenum
            \var{\va prasanna°\lem  \mssCaCbCc\msNa\Ed; \uncl{prasanna}° \msNb, prasa\.mna° \msNc}%

yathāśaktipradāna\.m ca svāśramābhyāgato guruḥ \veg\dontdisplaylinenum
            \var{\vc yathā°\lem  \mssCaCbCc\msNa\msNb\msNc; yasya \Ed\oo
                 °dāna\.m\lem  \mssCaCbCc\msNa\msNb\msNc; °dātaś \Ed}%
            \var{\vd svāśramā°\lem  \msCa\msCb\msNa\msNb\msNc\Ed; svāsamā° \msCc\oo
                 °gato\lem  \mssCaCbCc\msNa\msNb\Ed; °sato \msNc}%

indhanodakadāna\.m ca jātavedam athāpi vā\thinspace{\dandab} \dontdisplaylinenum
            \var{\vb indhano°\lem  \mssCaCbCc\msNa\msNb\Ed; itvano° \msNc\oo
                 jāta°\lem  \msCa\msCc\msNa\msNb\msNc\Ed; jā° \msCb}%

sulabhāni na dattāni indhanāgnyudakāni ca \danda\dontdisplaylinenum
            \var{\vc sulabhāni na\lem  \mssCaCbCc\msNa\msNb\msNc; surabhāni ca \Ed}%
            \var{\vd °dakāni\lem  \mssCaCbCc\msNa\msNc\Ed; °\uncl{ta}kāni \msNb}%

kṣute jīveti vā nokta\.m tasya ki\.m parataḥ phalam \veg\dontdisplaylinenum
            \var{\ve kṣute\lem  \conj; kṣuta\.m \mssCaCbCc\msNa\msNb\msNc, śata\.m \Ed}%


\alalfejezet{yameṣv ārjavam (10)}
pañcārjavāḥ praśa\.msanti munayas tattvadarśinaḥ\thinspace{\dandab} \dontdisplaylinenum
            \var{\va pañcārjavāḥ\lem  \msCa\msCb\msNa\msNc; pañcārjavaḥ \msCc, {\il}{\il}{\il}{\il} \msNb, pañcārjavā \Ed\oo
                 praśa\.msanti\lem  \mssCaCbCc\msNc; praśasanti \msNa\Ed, \uncl{prasasanti} \msNb}%

karmav\textsubring{r}ttyābhiv\textsubring{r}ddhi\.m ca pāratoṣikam eva ca \veg\dontdisplaylinenum
            \var{\vc karma°\lem  \msCb\msCc\msNa\msNc\Ed; {\il}rmma° \msCa, \uncl{kammā}° \msNb\oo
                 °v\textsubring{r}ttyābhiv\textsubring{r}ddhi\.m ca\lem  \mssCaCbCc\msNa\msNc; °v\textsubring{r}ttibhiv\textsubring{r}ddhiñ ca \msNb, °v\textsubring{r}tyābhiv\textsubring{r}ddhiś ca \Ed}%

strīdhanotkocavitta\.m ca ārjavo nābhinandati\thinspace{\dandab} \dontdisplaylinenum
            \var{\va strīdhanotkoca°\lem  \mssCaCbCc\msNa\msNb\msNc; strīdhanaṅgo ca \Ed\oo
                 °vitta\.m ca\lem  \mssCaCbCc\msNa\msNc\Ed; °vittiñ ca \msNb}%
            \var{\vb ārjavo nā°\lem  \msCa\msCb\msNa\msNb\msNc; ārjavañ ca \msCc, rjave nā° \Ed}%

ārjavo na v\textsubring{r}thā yajña ārjavo na v\textsubring{r}thā tapaḥ \veg\dontdisplaylinenum
            \var{\vcd ārjavo na v\textsubring{r}thā yajña ārjavo na v\textsubring{r}thā tapaḥ\lem  \mssCaCbCc\msNb\msNc; \om\ \msNaacorr,
                                                 ārjavo na v\textsubring{r}thā yajña ārjavo na v\textsubring{r}thā tapa \msNapcorr,                  
                                                 ārjavo na v\textsubring{r}thā yajñaś cārrjavo na v\textsubring{r}thā tapaḥ \Ed}%

ārjavo na v\textsubring{r}thā dānam ārjavo na v\textsubring{r}thāgnayaḥ\thinspace{\dandab} \dontdisplaylinenum
            \var{\vab \om\ \Ed}%

ārjavasyendriyagrāmaḥ suprasanno 'pi tiṣṭhati \danda\dontdisplaylinenum
            \var{\vcd \om\ \Ed}%
            \var{\vc °grāmaḥ\lem  \msCa\msCb\msNc\Ed; °grāmāt \msCc\msNb, °grāmāḥ \msNa}%

ārjavasya sadā devāḥ kāye tasya caranti te \veg\dontdisplaylinenum
            \var{\vf tasya caranti\lem  \msCb\msCc\msNa\msNb\msNc; tasya ramanti \Ed, ta{\il}{\lost}{\lost}nti \msCa}%

\ujvers\nemsloka 
iti yamapravibhāgaḥ kīrtito 'ya\.m dvijendra
\dontdisplaylinenum
            \var{\va yamapravibhāgaḥ\lem  \msCa\msCb\msNb\msNc; yamavibhāgaḥ \msCc, yamapraribhāgaḥ \msNa, niyamaparibhāgaḥ \Ed\oo
                 dvijendra\lem  \mssCaCbCc\msNa\msNb\msNc; narendra \Ed}%

\nemslokab 
iha parata sukhārtha\.m kārayet tan manuṣyaḥ \danda\dontdisplaylinenum

\nemslokac 
duritamalapahārī śaṅkarasyājñayāste
\dontdisplaylinenum
            \var{\vc durita°\lem  \mssCaCbCc\msNa\msNb\msNc; irita° \Ed\oo
                 °pahārī\lem  \msCa\msCb\msNa\msNb\msNc\Ed; °palapahārī \msCc\oo
                 °jñayāste\lem  \mssCaCbCc\msNb\msNc\Ed; °jñayāte \msNa}%

\nemslokad 
bhavati p\textsubring{r}thivibhartā hy ekachatraprav\textsubring{r}ttā \veg\dontdisplaylinenum
            \var{\vd °v\textsubring{r}ttā\lem  \mssCaCbCc\msNb\msNc; °v\textsubring{r}ttāḥ \msNa\Ed}%

\vers


\jump
\begin{center}
\ketdanda iti v\textsubring{r}ṣasārasa\.mgrahe yamavibhāgo nāmādhyāyaś caturthaḥ\ketdanda
\end{center}
\dontdisplaylinenum\vers 
            \var{{\normalfont Colophon: } nāmādhyāyaś caturthaḥ\lem  \mssCaCbCc\msNa\msNb\msNc;
                                                         nāmaś caturtho 'dhyāyaḥ \Ed}%
\bekveg\szamveg\vfill\phpspagebreak\szam\bek\versno=0\fejno=5
\thispagestyle{empty}



\alfejezet{\textbf{pañcamo 'dhyāyaḥ}}\jump\jump

\alalfejezet{niyamāḥ}
\vers

vigatarāga uvāca~{\dandab}\dontdisplaylinenum 
            \var{\vo vigatarāga uvāca\lem  \msCb\msCc\msNa\msNb\msNc\Ed; vigata\uncl{rāga uvā}ca \msCa}%

\nemsloka 
kathaya niyamatattva\.m sāmprata\.m tva\.m viśeṣād
\dontdisplaylinenum
            \var{\va kathaya ni°\lem  \mssCaCbCc\msNa\msNb\msNc; kathayati \Ed\oo
                 °tattva\.m\lem  \msCa\msCc\msNa\msNb\msNc\Ed; ta\.m \msCb\oo
                 sāmprata\.m tva\.m viśeṣād\lem  \msCa\msNa\msNc\Ed; 
                      tvā\.m vaśeṣāt \msCb, sā\.mprata tva\.m viseṣāt \msCc\msNb}%

\nemslokab 
am\textsubring{r}tavadanatulya\.m śrotukāmo gato 'smi \danda\dontdisplaylinenum
            \var{\vb °tulya\.m śro°\lem  \msCa\msCc\msNapcorr\msNb\msNc\Ed; °tulyā\.m śro° \msCb, 
                                                \uncl{°tulya\.m śro} tulya\.m sro° \msNaacorr\oo
                 °kāmo\lem  \mssCaCbCc\msNa\msNb\msNc; °kāmā \Ed}%

\nemslokac 
prak\textsubring{r}tidahanadagdha\.m jñānatoyair niṣiktam
\dontdisplaylinenum
            \var{\vc °dahana°\lem  \mssCaCbCc\msNa\msNb\msNc; °vadana° \Ed\oo
                 °r niṣiktam\lem  \msCa\msCc\msNa\msNb\msNc\Ed; °r vimuktam \msCb}%

\nemslokad 
\crux{apara vada matajñā} nāsti dharmeṣu t\textsubring{r}ptiḥ \veg\dontdisplaylinenum
            \var{\vd apara°\lem  \mssCaCbCc\msNb\msNc\Ed; apara\.m \msNa\ \unmetr\oo
                 °vadama°\lem  \msCapcorr\msCb\msCc\msNa\msNb\msNc; °vada° \msCaacorr, °vadana° \Ed\oo
                 °tajñā nāsti\lem  \msCa\msCb\msNa\msNc; °tajñā\uncl{nn}āsti \msCc, {\il}{\il}{\il}{\il} \msNb, °tajjñān nāsti \Ed}%

\vers

anarthayajña uvāca~{\dandab}\dontdisplaylinenum 

\nemsloka 
śravaṇasukham ato 'nyat kīrtayiṣye dvijendra
\dontdisplaylinenum
            \var{\va °sukha°\lem  \mssCaCbCc\msNapcorr\msNb\msNc\Ed; °mukha° \msNaacorr\oo 
                 °m ato 'nyat\lem  \mssCaCbCc\msNa\msNc; °m ato 'nya \msNb, °m ano 'nyat \Ed\oo
                 kīrta°\lem  \mssCaCbCc\msNc\Ed; kīrti° \msNa\msNb}%

\nemslokab 
niyamakalaviśeṣaḥ pañca pañca prakāraḥ \danda\dontdisplaylinenum
            \var{\vb °viśeṣaḥ\lem  \msCc\msNa\msNb\msNc\Ed; viśe{\il} \msCa, °viśeṣa \msCb\oo
                 prakāraḥ\lem  \mssCaCbCc\msNa\msNb\Ed; pakāraḥ \msNc}%

\nemslokac 
hariharamunibhīṣṭa\.m dharmasāra\.m dvijendra
\dontdisplaylinenum

\nemslokad 
kalikaluṣavināśa\.m prāyamokṣaprasiddham \veg\dontdisplaylinenum
            \var{\vd °vināśa\.m\lem  \msCa\msCb\msNa\msNb\msNc; °vināśa° \msCc\Ed}%

\vers

śaucam ijyā tapo dāna\.m svādhyāyopasthanigrahaḥ\thinspace{\dandab} \dontdisplaylinenum
            \var{\va ijyā\lem  \msCa\msCb\msNa\msNc\Ed; ījyā \msCc\msNb\oo
                 dāna\.m\lem  \mssCaCbCc\msNa\msNc\Ed; dāna° \msNb}%

vratopavāsamauna\.m ca snāna\.m ca niyamā daśa \veg\dontdisplaylinenum
            \paral{\textit{\vo = {\normalfont  \LP\ (Liṅgapurāṇa) 1.8.29\cd--30\ab } }}


\alalfejezet{niyameṣu śaucam (1)}
tatra śaucādinirdeśa\.m vakṣyāmīha dvijottama\thinspace{\dandab} \dontdisplaylinenum
            \var{\va °nirdeśa\.m\lem  \mssCaCbCc\msNc\Ed; °niyama\.m \msNa, °rddeśa\.m \msNb}%

śārīraśaucam āhāro mātrā bhāvaś ca pañcamaḥ \veg\dontdisplaylinenum
            \var{\vc śārīra°\lem  \mssCaCbCc\msNa\msNc\Ed; śarīra° \msNb\oo
                 °śaucam āhāro\lem  \msCb\msCc\msNa\msNb\msNc\Ed; °śauca{\il}hāro \msCa}%
            \var{\vd mātrā bhāvaś ca\lem  \msCb\msCc\msNa\msNc\Ed; mātrā bhāva\.m ca \msCa, \uncl{sātrābhā}vaś ca \msNb}%


\alalalfejezet{śarīraśaucam}

tāḍayen na ca bandheta na ca prāṇair viyojayet\thinspace{\dandab} \dontdisplaylinenum
            \var{\va tāḍayen na\lem  \mssCaCbCc\msNa\msNb\Ed; tāḍaye na \msNc}%

parastrīparadravyeṣu śauca\.m kāyikam ucyate \veg\dontdisplaylinenum
            \var{\vd śauca\.m\lem  \mssCaCbCc\msNa\msNb\Ed; śauca \msNc\oo
                 kāyikam ucyate\lem  \mssCaCbCc\msNa\msNb\Ed; kāyikam umucyete \msNc}%

śrotraśauca\.m dvijaśreṣṭha gudopasthamukhādayaḥ\thinspace{\dandab} \dontdisplaylinenum
            \var{\va śrotra°\lem  \eme; śrota° \mssCaCbCc\msNa\msNb\msNc\Ed}%
            \var{\vb gudopastha°\lem  \mssCaCbCc\msNa\msNb; gudoprastha° \msNc, gudāpastha° \Ed}%

mukhasyācamana\.m śaucam āhāravacaneṣu ca \veg\dontdisplaylinenum
            \var{\vc mukhasyā°\lem  \msCa\msCc\msNa\msNb\msNc\Ed; mukhasthā° \msCb}%
            \var{\vcd śaucam ā°\lem  \msCa\msCc\msNa\msNc\Ed; śauca\.mm ā° \msCb\msNb}%

mūtraviṣṭāsamutsarge devatārādhaneṣu ca\thinspace{\dandab} \dontdisplaylinenum
            \var{\va °viṣṭā°\lem  \mssCaCbCc\msNa\msNc\Ed; °viṣṭa° \msNb}%

m\textsubring{r}ttoyais tu gudopastha\.m śaucayīta vicakṣaṇaḥ \veg\dontdisplaylinenum
            \var{\vc m\textsubring{r}ttoyais tu\lem  \msCc\msNa\msNb\Ed; \uncl{m\textsubring{r}}{\il}{\il}{\il} \msCa, 
                                                      m\textsubring{r}toyais tu \msCb, m\textsubring{r}ttoyes tu \msNc\oo
                 °pastha\.m\lem  \msCa\msCb\msNa\msNb\msNc; °pastha \msCc\Ed}%

ekopasthe gude pañca tathaikatra kare daśa\thinspace{\dandab} \dontdisplaylinenum
            \var{\va °pasthe\lem  \msCa\msCb\msNa\msNc\Ed; °pastha° \msCc\msNb\oo
                 gude\lem  \msCa\msCb\msNa\msNc\Ed; °gudo \msCc\msNb}%
            \var{\vb tathaikatra\lem  \msCa\msCc\msNa\msNb\msNc; tathaika\uncl{tra} \msCb, tathaikaś ca \Ed\oo
                 daśa\lem  \msCa\msCb\msNa\msNb\msNc\Ed; daśaḥ \msCc}%
            \paral{\textit{\vab {\normalfont  \kb\ Manu 5.136ab: } ekā liṅge gude tisras tathaikatra kare daśa}}

ubhayoḥ sapta dātavyā m\textsubring{r}daḥ śuddhi\.m samīhatā \veg\dontdisplaylinenum
            \var{\vc dātavyā\lem  \msCa\msCb\msNa\msNb\msNc; dātavyo \msCc\Ed\
             \vd m\textsubring{r}daḥ\lem  \mssCaCbCc\msNc\Ed; m\textsubring{r}taḥ \msNa, m\textsubring{r}dā \msNb\oo
                 śuddhi\.m samīhatā\lem  \msCa\msCb\msNa; śuddhisamīhayā \msCc, śu\uncl{ddhi} samīhatā \msNb,
                                                                śuddhiḥ samīhatā \msNc, śuddhi\.m samāhitā \Ed}%
            \paral{\textit{\vcd {\normalfont  \kb\ Manu 5.136cd: } ubhayoḥ sapta dātavyā m\textsubring{r}daḥ śuddhim abhīpsatā}}

etac chauca\.m g\textsubring{r}hasthānā\.m dviguṇa\.m brahmacāriṇām\thinspace{\dandab} \dontdisplaylinenum
            \var{\va etac chauca\.m\lem  \msCa\msCb\msNa\msNc; cetac hauca \msCc\Ed, eta{\il}{\il} \msNb}%
            \var{\vb °guṇa\.m\lem  \msCa\msCb\msNa\msNb\msNc\Ed; °guṇa \msCc}%
            \paral{\textit{\vab {\normalfont  = Manu 5.137ab } }}

vānaprasthasya triguṇa\.m yatīnā\.m tu caturguṇam \veg\dontdisplaylinenum
            \var{\vc tri°\lem  \msCa\msCb\msNa\msNb\msNc\Ed; dvi° \msCc}%
            \paral{\textit{\vcd {\normalfont  \kb\ Manu 5.137cd: } triguṇa\.m syād vanasthānā\.m yatīnā\.m tu caturguṇam }}


\alalalfejezet{āhāraśaucam}

āhāraśauca\.m vakṣyāmi ś\textsubring{r}ṇuṣvāvahito bhava\thinspace{\dandab} \dontdisplaylinenum
            \var{\vb ś\textsubring{r}ṇuṣvāvahito\lem  \msCb\msCc\msNa\msNc\Ed; ś\textsubring{r}ṇu\uncl{ṣvāva}{\il}{\il} \msCa, ś\textsubring{r}ṇuṣvavahito \msNb}%

bhāgadvaya\.m tu bhuñjīta bhāgam eka\.m jala\.m pibet \danda\dontdisplaylinenum
            \var{\vd pibet\lem  \msCa\msCc\msNa\msNb\msNc\Ed; pibe \msCb}%

vāyusa\.mcāradānārtha\.m caturtham avaśeṣayet \veg\dontdisplaylinenum
            \var{\ve °cāradānārtha\.m\lem  \mssCaCbCc\msNa\msNb\msNc; °cāraṇārthāya \Ed}%

snigdhasvādurasaiḥ ṣaḍbhir āhāraṣaḍrasair budhaḥ\thinspace{\dandab} \dontdisplaylinenum
            \var{\va °svādu°\lem  \mssCaCbCc\msNa\msNc; °svā{\il} \msNb, °svāda° \Ed}%
            \var{\vb °ṣaḍrasair bu°\lem  \msCb\Ed; °sadravair bu° \msCa\msNa\msNc,
                                        °sadravai bu° \msCc, °ṣaḍrasai bu° \msNb}%

dhātuvaiṣamyanāśo 'sti na ca rogāḥ sudāruṇāḥ \veg\dontdisplaylinenum
            \var{\vc °vaiṣamyanāśo 'sti\lem  \msCa\msCc\msNa\msNb\msNc;
                      °\uncl{dai}ṣamyanāśāsti \msCb, °vaiṣamya naśyanti \Ed}%
            \var{\vd sudāruṇāḥ\lem  \mssCaCbCc\msNa\msNb\msNc; sudāruṇaḥ \Ed}%

abhakṣya\.m ca na bhakṣeta apeya\.m na ca pāyayet\thinspace{\dandab} \dontdisplaylinenum
            \var{\va abhakṣya\.m\lem  \mssCaCbCc\msNa\msNc; {\il}{\il}{\il} \msNb, abhakṣa\.m \Ed}%
            \var{\vb na ca\lem  \mssCaCbCc\msNa\msNb; ca na \msNc\Ed}%

agamya\.m na ca gamyeta avācya\.m na ca bhāṣayet \veg\dontdisplaylinenum
            \var{\vd avācya\.m\lem  \msCa\msCb\msNa\msNb\msNc\Ed; avāca\.m \msCc}%

laśuna\.m ca palāṇḍu\.m ca g\textsubring{r}ñjana\.m kacakāni ca\thinspace{\dandab} \dontdisplaylinenum
            \var{\va palāṇḍu\.m\lem  \Ed; palaṇḍu\.m \mssCaCbCc\msNb\msNc, palaḍu\.m \msNa}%
            \var{\vb kavakāni\lem  \mssCaCbCc\msNa\msNb\msNc; ca kacāni \Ed}%
            \paral{\textit{{\normalfont \vab cf.\ Manu 5.5ab: } laśuna\.m g\textsubring{r}ñjana\.m caiva palāṇḍu\.m kavakāni ca}}

gaura\.m ca śūkara\.m mā\.msa\.m varjayec ca vidhānataḥ \veg\dontdisplaylinenum
            \var{\vc gaura\.m ca\lem  \eme; gorasva \msCa\msNb, goraś ca \msCb\msCc\msNa\msNc, gauraś ca \Ed\oo
                 mā\.msa\.m\lem  \mssCaCbCc\msNa\msNb\msNc; māsa\.m \Ed}%

chattrāka\.m viḍvarāha\.m ca gomā\.msa\.m ca na bhakṣayet\thinspace{\dandab} \dontdisplaylinenum
            \var{\va chattrāka\.m\lem  \msNa\msCa\msCb\msNb\msNc\Ed; chattrāka \msCc\oo
                 viḍva°\lem  \mssCaCbCc\msNb\Ed; vidva° \msNa\msNc}%
            \var{\vb gomā\.msa\.m\lem  \msNa\msCa\msCbpcorr\msCc\msNb\msNc\Ed; gomāñ \msCbacorr}%
            \paral{\textit{\vab {\normalfont Cf. Manu 5.19ab: } chatrāka\.m viḍvarāha\.m ca laśuna\.m grāmakukkuṭam}}

caṭaka\.m ca kapota\.m ca jālapādā\.mś ca varjayet \veg\dontdisplaylinenum
            \var{\vc caṭaka\.m\lem  \msCa\msCb\msNa\msNc\Ed; caṭakām \msCc\msNb}%

ha\.msasārasacakrāhvakukkuṭān śukaśyenakān\thinspace{\dandab} \dontdisplaylinenum
            \var{\vb °kukkuṭān śu°\lem  \mssCaCbCc\msNc\Ed; °kukkuṭā śu° \msNa, °kukkuṭā\.m śu° \msNb\oo
                 °śyenakān\lem  \msCa\msCc\msNc\Ed; °śonakān \msCb, °śyenakā \msNa, °śyenakā\.m \msNb}%

kākolūka\.m balāka\.m ca matsyādī\.mś cāpi varjayet \veg\dontdisplaylinenum
            \var{\vc kākolūka\.m balāka\.m ca\lem  \msCb\msNc; kākolūka\uncl{sva}{\il}{\il}ñ ca \msCa,
                                                        kākolūkabalāka\.m ca \msCc\msNa\Ed, 
                                                        \uncl{kākolūka\.m balāka\.m ca} \msNb}%

amedhyā\.mś cāpavitrā\.mś ca sarvān eva vivarjayet\thinspace{\dandab} \dontdisplaylinenum
            \var{\va amedhyā\.mś cā°\lem  \mssCaCbCc\msNa\msNc; \uncl{amedhyāś cā°} \msNb, amedhyaś cā° \Ed}%

śākamūlaphalānā\.m ca abhakṣya\.m parivarjayet \veg\dontdisplaylinenum

mānaveṣu purāṇeṣu śaivabhāratasa\.mhite\thinspace{\dandab} \dontdisplaylinenum

kīrtitāni viśeṣeṇa śaucācāram aśeṣataḥ \veg\dontdisplaylinenum

tvayā jijñāsito 'smy adya sa\.mkṣiptaḥ kathito mayā\thinspace{\dandab} \dontdisplaylinenum
            \var{\va jijñāsito\lem  \mssCaCbCc\msNa\msNb; jijñāsano \msNc, jijñāsato \Ed}%
            \var{\vb °kṣiptaḥ\lem  \msCa\msCc\msNa\msNc\Ed; °kṣipya \msCb, °kṣipta \msNb\oo
                  kathito\lem  \mssCaCbCc\msNa\msNb\msNc; kathita\.m \Ed}%

satyavādī śucir nitya\.m dhyānayogarataḥ śuciḥ \veg\dontdisplaylinenum
            \var{\vc śucir\lem  \msCa\msCb\Ed; śuci \msCc\msNc, śucin \msNa\msNb}%

ahi\.msakaḥ śucir dānto dayābhūtakṣamā śuciḥ\thinspace{\dandab} \dontdisplaylinenum
            \var{\va ahi\.msakaḥ\lem  \msCa\msCc\msNa\msNb\msNc\Ed; ahi\.msaka \msCb\oo 
                 śucir dānto\lem  \msCa\msCb\msNa\msNb\Ed; śuci dānto \msCc\msNc, śucir dāntau \Ed}%

sarveṣām eva śaucānām arthaśauca\.m para\.m sm\textsubring{r}tam \veg\dontdisplaylinenum
            \var{\vd °śauca\.m para\.m sm\textsubring{r}tam\lem  \msCa\msNa\msNb\msNc; °śauca\.m para sm\textsubring{r}tam \msCb\msCc,
                                                      °śaucayana\.m sm\textsubring{r}taḥ \Ed}%
            \paral{\textit{\vcd {\normalfont  = Manu 5.106ab}}}

yo 'rthe hi śuciḥ sa śucir na m\textsubring{r}dvāriśuciḥ śuciḥ\thinspace{\dandab} \dontdisplaylinenum
            \var{\vab yo 'rthe hi śuciḥ sa śucir na\lem  \mssCaCbCc\msNc\ \unmetr; 
                                yo 'rthe hi śuciḥ sa śuci na \msNa\msNb, yo 'rthe hi suśucir vipra na \Ed}%
            \var{\vb °śuciḥ śuciḥ\lem  \mssCaCbCc\msNa\msNc; śuci śuciḥ \msNb, °śuciḥ śuci \Ed}%
            \paral{\textit{\vab {\normalfont \kb\ Manu 5.106cd: } yo 'rthe śucir hi sa śucir na m\textsubring{r}dvāriśuciḥ śuciḥ}}

kāyavāṅmanasā\.m śauca\.m sa śuciḥ sarvavastuṣu \veg\dontdisplaylinenum
            \var{\vd śuciḥ\lem  \msCa\msCb\msNa\msNb\msNc\Ed; śuci \msCc\oo
                 vastuṣu\lem  \mssCaCbCc\msNa\msNb\Ed; vastuṣuḥ \msNc}%
            \var{\vcd {\normalfont  \Ed\ adds here, after pādas cd: } śaucāśaucavidhir jñātvā mucyate sarvakilbiṣāt}%

\ujvers\nemsloka 
śaucāśaucavidhijña mānava yadi kālakṣaye niścayaḥ
\dontdisplaylinenum
            \var{\va śaucāśauca°\lem  \msCa\msCc\msNa\msNb\msNc\Ed; śaucāśuca \msCb\oo
                 kālakṣaye niścayaḥ\lem  \msNaacorr\msNc;
                      kālakṣayair niścayaḥ \msCa\msCb\msNapcorr,
                      kālakṣayen niścayaḥ \msCc\msNb,
                      kālakṣayetiś ca yaḥ \Ed}%

\nemslokab 
saubhāgyatvam avāpnuvanti satata\.m kīrtir yaśo'laṅk\textsubring{r}taḥ \danda\dontdisplaylinenum
            \var{\vb kīrtir ya°\lem  \msCb\msNa\msNb\msNc\Ed; kīrtiya° \msCa\msCc \unmetr\oo
                °laṅk\textsubring{r}taḥ\lem  \msCa\msCc\msNa\msNb\msNc\Ed; °lak\textsubring{r}taḥ \msCb}%

\nemslokac 
prāpta\.m tena ihaiva puṇyasakala\.m saddharmaśāstreritam
\dontdisplaylinenum
            \var{\vc °eritam\lem  \mssCaCbCc\msNa\msNb\msNc; °oditaḥ \Ed}%

\nemslokad 
jīvānte ca paratra-m-īhitagati\.m prāpnoti niḥsa\.mśayam \veg\dontdisplaylinenum
            \var{\vd paratra-m-ī°\lem  \mssCaCbCc\msNa\msNb\msNc; pavitram ī° \Ed\oo
                 °gati\.m\lem  \eme; °gatiḥ \mssCaCbCc\msNa\msNb\msNc\Ed\oo
                 niḥsa\.mśayam\lem  \msCa\msNb\msNc; niḥsa\.mśayaḥ \msCb\msCc\msNa\Ed}%

\vers


\jump
\begin{center}
\ketdanda iti v\textsubring{r}ṣasārasa\.mgrahe śaucācāravidhir nāmādhyāyaḥ pañcamaḥ\ketdanda
\end{center}
\dontdisplaylinenum\vers 
    \var{{\normalfont Colophon:} °vidhir\lem  \msCa\Ed; °vidhi° \msCb\msCc\msNa\msNc, \uncl{vi\.mdhi} \msNb\oo
                        nāmādhyayaḥ pañcamaḥ\lem  \mssCaCbCc\msNa\msNb\msNc; 
                                                nāma pañcamo 'dhyāyaḥ \Ed}%
\bekveg\szamveg\vfill\phpspagebreak\szam\bek\versno=0\fejno=6
\thispagestyle{empty}



\alfejezet{\textbf{ṣaṣṭho 'dhyāyaḥ}}\jump\jump

\alalfejezet{niyameṣu ijyā (2)}
\vers

[anarthayajña uvāca~{\dandab}\dontdisplaylinenum ]

atha pañcavidhām ijyā\.m pravakṣyāmi dvijottama\thinspace{\danda} \dontdisplaylinenum
            \var{\va °m ijyā\.m\lem  \msCb; °m ījyā\.m \msCa\msCc\msNa\msNb\msNc\Ed}%
            \var{\vb °ttama\lem  \mssCaCbCc\msNa\Ed; °ttamaḥ \msNb\msNc}%

dharmamokṣaprasiddhyartha\.m ś\textsubring{r}ṇuṣvāvahito dvija \veg\dontdisplaylinenum
            \var{\vc °mokṣaprasiddhyartha\.m\lem  \mssCaCbCc\msNc; °mokṣaprasiddhyartha \msNa\msNb, 
                                                        °mokṣeśasiddhyaartha\.m \Ed}%
            \var{\vd dvija\lem  \mssCaCbCc\msNa\msNb\msNc; bhava \Ed}%

arthayajñaḥ kriyāyajño japayajñas tathaiva ca\thinspace{\dandab} \dontdisplaylinenum
            \var{\va arthayajñaḥ\lem  \msCa\msCc\msNa; anarthayajñaḥ \msCb, arthayajña \msNb\msNc, arthayajña° \Ed}%

jñāna\.m dhyāna\.m ca pañcaitat pravakṣyāmi p\textsubring{r}thak p\textsubring{r}thak \veg\dontdisplaylinenum
            \var{\vc jñāna\.m\lem  \msCa\msCb\msNa\msNb\Ed; jñāna \msCc\msNc}%


\alalalfejezet{arthayajñaḥ}

agnyupāsanakarmādi agnihotrakratukriyā\thinspace{\dandab} \dontdisplaylinenum
            \var{\vb agni°\lem  \msCb\msCc\msNa\msNc\Ed; \uncl{a}{\lost}° \msCa, {\il}{\il} \msNb\oo
                 °kriyā\lem  \msCa\msNa\msNb\msNc\Ed; °kriyāḥ \msCb\msCc}%

aṣṭakāḥ pārvaṇī śrāddha\.m dravyayajñaḥ sa ucyate \veg\dontdisplaylinenum
            \var{\vc aṣṭakāḥ\lem  \eme; aṣṭakā \mssCaCbCc\msNa\msNb\msNc\Ed\oo
                 pārvaṇī\lem  \msCa\msCc\msNa\msNc\Ed; parvaṇī \msCb, \uncl{parvaṇī} \msNb}%
            \var{\vd °yajñaḥ\lem  \msCa\msCb\msNa\msNc\Ed; °yajña \msCc, {\il}{\il} \msNb}%


\alalalfejezet{kriyāyajñaḥ}

ārāmodyānavāpīṣu devatāyataneṣu ca\thinspace{\dandab} \dontdisplaylinenum
            \var{\vb °yataneṣu\lem  \msCb\msCc\Ed; °layaneṣu \msCa\msNa\msNc, °yata{\il}{\il} \msNb}%

svahastak\textsubring{r}tasa\.mskāraḥ kriyāyajña sa ucyate \veg\dontdisplaylinenum
            \var{\vc °hasta°\lem  \mssCaCbCc\msNa\msNc; {\il}{\il} \msNb, °hastaiḥ \Ed}%


\alalalfejezet{japayajñaḥ}

japayajña\.m tato vakṣye svargamokṣaphalapradam\thinspace{\dandab} \dontdisplaylinenum
            \var{\va °yajña\.m tato\lem  \msCa\msNa\msNb\msNc\Ed; °yajña\.m tapo \msCb °yajñas tato \msCc}%

vedādhyayana kartavya\.m śivasa\.mhitam eva ca \veg\dontdisplaylinenum
            \var{\vc vedā°\lem  \mssCaCbCc\msNa\msNc\Ed; adā° \msNb}%

itihāsapurāṇa\.m ca japayajñaḥ sa ucyate\thinspace{\dandab} \dontdisplaylinenum
            \var{\va °purāṇa\.m ca\lem  \mssCaCbCc\msNa\msNb\msNc; °purāṇaś ca \Ed}%
            \var{\vb °yajñaḥ\lem  \msCa\msCb\msNa\msNb\msNc\Ed; °yajña \msCc}%


\alalalfejezet{jñānayajñaḥ}

ida\.m karma akarmedam ūhāpohaviśāradaḥ \veg\dontdisplaylinenum
            \var{\vc karma\lem  \mssCaCbCc\msNa\msNb\msNc; kramam \Ed}%

śāstracakṣuḥ samālokya jñānayajñaḥ sa ucyate\thinspace{\dandab} \dontdisplaylinenum
            \var{\va °cakṣuḥ\lem  \msCa\msCb\msNa\msNb\msNc\Ed; °cakṣu \msCc}%
            \var{\vb °yajñaḥ\lem  \msCa\msCb\msNa\msNc\Ed; °yajña \msCc, °\uncl{yajñas} \msNb}%

dhyānayajña\.m samāsena kathayiṣyāmi te ś\textsubring{r}ṇu \veg\dontdisplaylinenum
            \var{\vc °yajña\.m\lem  \msCa\msCb\msNa\msNc\Ed; °yajña \msCc\msNb}%


\alalalfejezet{dhyānayajñaḥ}

dhyāna\.m pañcavidha\.m caiva kīrtita\.m hariṇā purā\thinspace{\dandab} \dontdisplaylinenum
            \var{\va dhyāna\.m\lem  \mssCaCbCc\msNb\Ed; dhyāna \msNa\msNc}%

sūryaḥ somo 'gni sphaṭikaḥ sūkṣma\.m tattva\.m ca pañcamam \veg\dontdisplaylinenum
            \var{\vc somo\lem  \msCa\msCc\msNa\msNc; somā° \msCb\msNb\Ed}%
            \var{\vd sūkṣma\.m tattva\.m ca pañcamam\lem  \msCb;
                                sūkṣma\.m ta\uncl{tva}{\lost}{\lost}{\lost}ñcamam \msCa, 
                                sūkṣmatattva\.m ca pañcamaḥ \msCc\msNa\msNb,
                                sūkṣma\.m tattvañ ca pañcamaḥ \msNc,
                                sūkṣmā\.m tattvaś ca pañcamam \Ed}%

sūryamaṇḍalam ādau tu tattva\.m prak\textsubring{r}tir ucyate\thinspace{\dandab} \dontdisplaylinenum

tasya madhye śaśi\.m dhyāyet tattva\.m puruṣa ucyate \veg\dontdisplaylinenum
            \var{\vc śaśi\.m\lem  \mssCaCbCc\msNa\Ed; śaśi \msNb, śaśi\.mn \msNc}%
            \var{\vcd dhyāyet ta°\lem  \msCa\msCb\msNa\msNb\msNc\Ed; dhyāye ta° \msCc}%

candramaṇḍalamadhye tu jvālām agni\.m vicintayet\thinspace{\dandab} \dontdisplaylinenum
            \var{\vb jvālām agni\.m\lem  \mssCaCbCc\msNa\msNb\Ed; jālām agni \msNc}%

prabhutattvaḥ sa vijñeyo janmam\textsubring{r}tyuvināśanaḥ \veg\dontdisplaylinenum
            \var{\vc °tattvaḥ\lem  \mssCaCbCc\msNc; °tatva \msNa, °tatva\.m \msNb\Ed}%
            \var{\vd °nāśanaḥ\lem  \msCa\msCb\msNa\msNb\msNc; °nāśanam \msCc\Ed}%

agnimaṇḍalamadhye tu dhyāyet sphaṭika nirmalam\thinspace{\dandab} \dontdisplaylinenum
            \var{\vb dhyāyet sphaṭika\lem  \msCapcorr\msCb\msNa\msNb\msNc; dhyāyet sphaṭi \msCaacorr,
                                        dhyāye sphaṭika \msCc\Ed\oo
                 °malam\lem  \mssCaCbCc\msNb\Ed; °malaḥ \msNa, \uncl{°malaḥ}}%

vidyātattvaḥ sa vijñeyaḥ kāraṇam ajam avyayam \veg\dontdisplaylinenum
            \var{\vc tattvaḥ sa\lem  \msCb\msNa\msNb\msNc; ta\uncl{tvan}{\lost} \msCa, tatva sa \msCc, tatva\.m sa \Ed}%
            \var{\vd °jam avyayam\lem  \msCa\msCb\msNa\msNb\msNc\Ed; °m avyaya\.m \msCc}%

vidyāmaṇḍalamadhye tu dhyāyet tattvam anuttamam\thinspace{\dandab} \dontdisplaylinenum
            \var{\vab dhyāyet ta°\lem  \msCa\msCb\msNa\msNb\msNc\Ed; dhyāye ta° \msCc}%

akīrtitam anaupamya\.m śivam akṣayam avyayam \danda\dontdisplaylinenum

pañcama\.m dhyānayajñasya tattvam ukta\.m samāsataḥ \veg\dontdisplaylinenum
            \var{\vc °yajñasya\lem  \msCa\msCb\msNa\msNb\msNc; °yajñañ ca \msCc\Ed}%
            \var{\vd samāsataḥ\lem  \mssCaCbCc\msNa\msNb\msNc; sanātanaḥ \Ed}%

vigatarāga uvāca~{\dandab}\dontdisplaylinenum 

ekaikasya hi tattvasya phala\.m kīrtaya kīd\textsubring{r}śam\thinspace{\danda} \dontdisplaylinenum
            \var{\va hi\lem  \Ed; tri° \mssCaCbCc\msNa\msNb\msNc}%

kāni lokāḥ prapadyante kāla\.m vāsya tapodhana \veg\dontdisplaylinenum
            \var{\vc lokāḥ\lem  \msCa\msNa\msNc; lokā \msCb\msCc\msNb\Ed\oo
                 prapadyante\lem  \msCb\msCc\msNa\msNb\msNc\Ed; pra{\il}{\il}{\il} \msCa}%
            \var{\vd °dhana\lem  \msCa\msCc\msNa\msNb\Ed; °dhanaḥ \msCb\msNc}%

anarthayajña uvāca~{\dandab}\dontdisplaylinenum 

brahmaloka\.m tu prathama\.m tattva\.m prak\textsubring{r}ticintayā\thinspace{\danda} \dontdisplaylinenum
            \var{\vab prathama\.m tattva\.m\lem  \Ed; prathama\.m tatva \mssCaCbCc\msNapcorr\msNb\msNc, 
                                                                    \om\ \msNaacorr\oo
                  prak\textsubring{r}ticintayā\lem  \mssCaCbCc\msNa\msNb\msNc; ca k\textsubring{r}ticintaya \Ed}%

kalpakoṭisahasrāṇi śivavan modate sukhī \veg\dontdisplaylinenum
            \var{\vd sukhī\lem  \mssCaCbCc\msNa\msNb\msNc; sukham \Ed}%

dvitīya\.m tattva puruṣa\.m dhyāyamāno m\textsubring{r}to yadi\thinspace{\dandab} \dontdisplaylinenum

viṣṇulokam ito yāti kalpakoṭyayuta\.m sukhī \veg\dontdisplaylinenum
            \var{\vc yāti\lem  \mssCaCbCc\msNa\msNb\msNc; yānti \Ed}%

prabhutattva\.m t\textsubring{r}tīya\.m tu dhyāyamāno mariṣyati\thinspace{\dandab} \dontdisplaylinenum
            \var{\va °tattva\.m\lem  \msCa\msCb\msNa\msNb\msNc\Ed; °tatva \msCc\oo
                 t\textsubring{r}tīya\.m\lem  \mssCaCbCc\msNa\msNb\msNc; t\textsubring{r}tīyas \Ed}%
            \var{\vb dhyāyamāno mariṣyati\lem  \msCb\msCc\msNa\msNb\msNc; dhyāya{\il}{\il}{\il}riṣyati \msCa,
                                                        dhayāyāmāno mariṣyati \Ed}%

śivaloke vasen nitya\.m kalpakotyayuta\.m śatam \veg\dontdisplaylinenum
            \var{\vc śivaloke\lem  \msCa\msCc\msNa\msNb\msNc; śivaloka \msCb, rudraloke \Ed\oo
                 vasen ni°\lem  \msCa\msCb\msNa\msNb\msNc\Ed; vase ni° \msCc}% 
            \var{\vd °yuta\.m\lem  \mssCaCbCc\msNa\msNc\Ed; °yuta \msNb}%

vidyātattvām\textsubring{r}ta\.m dhyāyet sadāśivam anāmayam\thinspace{\dandab} \dontdisplaylinenum
            \var{\va °tattvām\textsubring{r}ta\.m\lem  \msCa\msCb\msNa\msNb\msNc; °tatvam\textsubring{r}tan \msCc, °tattvāmata\.m \Ed}%

akṣaya\.m lokam āpnoti kalpānāntapara\.m tathā \veg\dontdisplaylinenum 
            \var{\vc akṣaya\.m\lem  \mssCaCbCc\msNa\msNb\msNc; akṣaya° \Ed}%

pañcama\.m śivatattva\.m tu sūkṣma\.m cātmani sa\.msthitam\thinspace{\dandab} \dontdisplaylinenum

na kālasa\.mkhyā tatrāsti śivena saha modate \veg\dontdisplaylinenum

\ujvers\nemsloka 
pañcadhyānābhiyukto bhavati ca na punarjanmasa\.mskārabandhaḥ
\dontdisplaylinenum
            \var{\va °yukto\lem  \msCb\msCc\msNa\msNb\msNc; °yu{\il} \msCa\ \toplost, °yuktau \Ed\oo
                 ca\lem  \msCa\msCc\msNa\msNb\msNc; \om\ \msCb\Ed\oo
        punarjanma°\lem  \msCb\msCc\msNa\msNb\msNc\Ed; puna\uncl{ja}nma° \msCa\ \toplost, punajanma° \msCc}%

\nemslokab 
jijñāsyantā\.m dvijendra bhavadahanakaraḥ prārthanākalpav\textsubring{r}kṣaḥ \danda\dontdisplaylinenum
            \var{\vb jijñāsyantā\.m\lem  \msCa\msNb\msNc\Ed; jijñāsyatā\.m \msCb\msNa\ \unmetr, jijñāsyantā \msCc}%

\nemslokac 
janmenaikena muktir bhavati kimu na vā mānavāḥ sādhayantu
\dontdisplaylinenum
            \var{\vc janmenaikena\lem  \msCb\msNb\msNc\Ed; janmanaikena \msCa\msCc\msNa\ \unmetr\oo
                 muktir bh°\lem  \msCa\msCb\msNa\msNb\msNc\Ed; mukti bh° \msCc\oo
                 na vā\lem  \mssCaCbCc\msNb\msNc\Ed; bhavā \msNa\oo
                 mānavāḥ\lem  \msCa\msNa\msNb\msNc; mānamānavāḥ \msCb, mānavā \msCc, mānava \Ed}%

\nemslokad 
pratyakṣān nānumāna\.m sakalamalahara\.m svātmasa\.mvedanīyam \veg\dontdisplaylinenum
            \var{\vd pratyakṣā°\lem  \mssCaCbCc\msNb\msNc\Ed; pratyakṣa° \msNa\oo
                 °vedanīyam\lem  \msCb\msNa\msNb; °vedanīyaḥ \msCa\msCc\msNc, °vedanīya \Ed}%

\vers


\alalfejezet{niyameṣu tapaḥ (3)}
mānasa\.m tapa ādau tu dvitīya\.m vācika\.m tapaḥ\thinspace{\dandab} \dontdisplaylinenum
            \var{\va °tapa\lem  \mssCaCbCc\msNa\msNb\msNc; °tapam \Ed}%

kāyika\.m ca t\textsubring{r}tīya\.m tu manovākkarma tatparam \danda\dontdisplaylinenum
            \var{\vc kāyika\.m ca t\textsubring{r}tīya\.m tu\lem  \mssCaCbCc\msNa\msNc\Ed; mānasa\.m tapa ādau tu \msNb\ {\normalfont (eyeskip)}}%
            \var{\vd manovākkarma\lem  \msCa\msNc\Ed; manokkarma \msCb, mmanovākarma° \msCc, manovākkāya° \msNa\msNb\oo
                 °param\lem  \msCc; °paraḥ \msCa\msCb\msNa\msNb\msNc\Ed}%

kāyika\.m vācika\.m caiva tapo miśraka pañcamam \veg\dontdisplaylinenum
            \var{\vc kāyika\.m\lem  \mssCaCbCc\msNb\msNc\Ed; kāyika \msNa}%

manaḥsaumya\.m prasādaś ca ātmanigraham eva ca\thinspace{\dandab} \dontdisplaylinenum
            \var{\va °saumya\.m\lem  \msNc; °saumya° \msCa\msCb\msNa\msNb\Ed, °saum\uncl{ya}° \msCc\ \toplost\oo 
                 prasādaś ca\lem  \msCa\msCc\msNa\msNc; prasāda\.m ca \msCb\Ed, pradānaś ca \msNb}%

mauna\.m bhāvaviśuddhiś ca pañcaitat tapa mānasam \veg\dontdisplaylinenum
            \var{\vc mauna\.m\lem  \mssCaCbCc\msNa\msNb\msNc; mauna{\il} \Ed\oo
                 °śuddhiś ca\lem  \msCa\msCb\msNa\msNb\msNc; °śuddhi\.m ca \msCc\Ed}%
            \var{\vd pañcaitat\lem  \msCa\msNb\msNc; pañcaite \msCb\msNa, pañcetat \msCc, pañcaitan \Ed}%

anudvegakarā vāṇī priya\.m satya\.m hita\.m ca yat\thinspace{\dandab} \dontdisplaylinenum

svādhyāyābhyasana\.m caiva vācika\.m tapa ucyate \veg\dontdisplaylinenum
            \var{\vc °bhyasana\.m caiva\lem  \msCb\msCc\msNa\msNc\Ed; °bhyasana{\il}{\il} \msCa, °bhyasa\uncl{na\.m} caiva \msNb}%
            \paral{\textit{\vcd {\normalfont  \kb\ MBh 6.39.15cd: } svādhyāyābhyasana\.m caiva vāṅmaya\.m tapa ucyate}}

ārjava\.m ca ahi\.msā ca brahmacarya\.m surārcanam\thinspace{\dandab} \dontdisplaylinenum
            \var{\va ārjava\.m ca ahi\.msā ca\lem  \mssCaCbCc\msNa\msNb\msNc;
                                        ārjavatvam ahi\.msāś ca \Ed}%
            \var{\vb °carya\.m\lem  \msCa\msCb\msNa\msNb\msNc; °carya \msCc\Ed}%

śauca\.m pañcamam ity etat kāyika\.m tapa ucayate \veg\dontdisplaylinenum
            \var{\vc śauca\.m\lem  \mssCaCbCc\msNa\msNb\msNc; śauca \Ed}%

iṣṭa\.m kalyāṇabhāva\.m ca dhanya\.m pathya\.m hita\.m vadet\thinspace{\dandab} \dontdisplaylinenum
            \var{\va iṣṭa\.m\lem  \msCa\msCb\msNa\msNc\Ed; iṣṭa \msCc\msNb\oo
                 °bhāva\.m\lem  \mssCaCbCc\msNa\msNb\msNc; °bhāvaś \Ed}%
            \var{\vb pathya\.m\lem  \mssCaCbCc\msNa\msNb\msNc; satya\.m \Ed}%
            \paral{\textit{\vc {\normalfont MBh 5.145.6: } 
                mayā nāgapura\.m gatvā sabhāyā\.m dh\textsubring{r}tarāṣṭrajaḥ{\thinspace\danda}
                tathya\.m pathya\.m hita\.m cokto na ca g\textsubring{r}hṇāti durmatiḥ{\thinspace\ketdanda}}}

manomiśraka pañcaitat tapa ukta\.m maharṣibhiḥ \veg\dontdisplaylinenum
            \var{\vc mano°\lem  \mssCaCbCc\msNa\msNb\msNc; mana° \Ed\oo
                 pañcaitat\lem  \mssCaCbCc\msNa\msNb; pañcetat \msNc, pañcaitān \Ed}%
            \var{\vd tapa ukta\.m maharṣibhiḥ\lem  \mssCaCbCc\msNa\msNb\msNc; tapam ukta\.m mahirṣibhiḥ \Ed}%

svastimaṅgalam āśīrbhir atithigurupūjanam\thinspace{\dandab} \dontdisplaylinenum
            \var{\va °śīrbhi°\lem  \msCa\Ed; °śībhi° \msCb\msCc\msNa\msNb\msNc}%
            \var{\vb °tithi°\lem  \mssCaCbCc\msNa\msNb\msNc; °tithi\.m \Ed}%

kāyamiśraka pañcaitat tapa ukta\.m mahātmabhiḥ \veg\dontdisplaylinenum
            \var{\vc °miśraka\lem  \msCc\msNa\msNb\msNc\Ed; °{\il}{\il}ka \msCa, °mityaśraka \msCb\oo
                 pañcaitat\lem  \mssCaCbCc\msNa\msNb\msNc; pañcaitan \Ed}%
            \var{\vd tapa ukta\.m\lem  \mssCaCbCc\msNa\msNb\msNc; tapam ukta\.m \Ed}%

maṇḍūkayogī hemante grīṣme pañcatapās tathā\thinspace{\dandab} \dontdisplaylinenum
            \var{\vb grīṣme\lem  \mssCaCbCc\msNa\msNb\msNc; g\textsubring{r}ṣme \Ed}%
            \paral{\textit{\vab {\normalfont See MBh and Lalitavistara passages in Umā Playful article (p.627): }
               maṇḍūkayogī hemante grīṣmapañcā tapās bhavet {\normalfont ...
                also Umā 6.26ab :}  maṇḍūkayogo hemante grīṣme pañcatapāstathā{\thinspace\danda}}}

abhrāvakāśe varṣāsu tapaḥ sādhanam ucyate \veg\dontdisplaylinenum
            \var{\vd tapaḥ\lem  \msCa\msCb\msNa\msNb\msNc\Ed; tapa \msCc\oo
                 sādhanam u°\lem  \msCa\msNa\msNc\Ed; sādhana u° \msCb\msCc\msNb}%

svamā\.msoddh\textsubring{r}tya dāna\.m ca hastapādaśiras tathā\thinspace{\dandab} \dontdisplaylinenum
            \var{\va dāna\.m\lem  \mssCaCbCc\msNa\msNc; \uncl{dāna} \msNb\ \toplost, dānaś \Ed}%
            \paral{\textit{\vab {\normalfont Cf. 2.38 and 17.xx ff. }}}

puṣpam utpādya dāna\.m ca sarve te tapa sādhanāḥ \veg\dontdisplaylinenum
            \var{\vc dāna\.m\lem  \mssCaCbCc\msNa\msNb\msNc; dānaś \Ed}%
            \var{\vd tapa\lem  \Ed; tapaḥ \mssCaCbCc\msNa\msNb\msNc\ \unmetr}%

k\textsubring{r}cchrātik\textsubring{r}cchra\.m nakta\.m ca taptak\textsubring{r}cchram ayācitam\thinspace{\dandab} \dontdisplaylinenum
            \var{\va k\textsubring{r}cchrātik\textsubring{r}cchra\.m\lem  \msCa\msCb\msNa\Ed; k\textsubring{r}cchrādik\textsubring{r}cchra \msCc, k\textsubring{r}cchrātik\textsubring{r}cchra \msNb, k\textsubring{r}cchātik\textsubring{r}ccha\.m \msNc}%
            \var{\vb °yācitam\lem  \mssCaCbCc\msNa\msNb\msNc; °yācitaḥ \Ed}%

cāndrāyaṇa\.m parāka\.m ca tapaḥ sā\.mtapanādayaḥ \veg\dontdisplaylinenum
            \var{\vc cāndrāyaṇa\.m parāka\.m\lem  \msCa\msCc\msNb\msNc; cāndrāyana\.m parāka\.m \msCb, 
                                                        candrāyaṇa\.m parāka\.m \msNa, cāndrāyaṇavarākaś \Ed}%
            \var{\vd tapaḥ sā\.mtapanādayaḥ\lem  \msCa\msCb\msNa\msNb\msNc; tapasāntapanādayaḥ \msCc\Ed}%

\ujvers\nemsloka 
yeneda\.m tapa tapyate sumanasā sa\.msāraduḥkhacchidam
\dontdisplaylinenum
            \var{\va tapa ta°\lem  \Ed; tapas ta° \mssCaCbCc\msNa\msNb\msNc\ \unmetr\oo
                 °manasā\lem  \eme; °manasaḥ \mssCaCbCc\msNa\msNb\msNc\Ed}%

\nemslokab 
āśāpāśa vimucya nirmalamatis tyaktvā jaghanya\.m phalam \danda\dontdisplaylinenum
            \var{\vb nirmalamati°\lem  \msCa\msCc\msNa\msNb\msNc\Ed; nirmalarmati° \msCb\oo
                 jaghanya\.m\lem  \mssCaCbCc\msNa\msNb\msNc; jagat ya\.m \Ed}%

\nemslokac 
svargākāṅkṣyan\textsubring{r}patvabhogaviṣaya\.m sarvāntika\.m tat phalam
\dontdisplaylinenum
            \var{\vc °kāṅkṣya°\lem  \mssCaCbCc\msNa\msNb\msNc; °kā\.mkṣa° \Ed\oo
                 sarvāntika\.m\lem  \msCa\msCc\msNa\msNb\msNc\Ed; sarvārttika\.m \msCb}%

\nemslokad 
jantuḥ śāśvatajanmam\textsubring{r}tyubhavane tanniṣṭhasādhya\.m vahet \veg\dontdisplaylinenum
            \var{\vd °bhavane\lem  \mssCaCbCc\msNa\msNb\Ed; °bhavene \msNc\oo
                 °sādhya\.m vahet\lem  \msCc\msNa\msNb\msNc; °\uncl{sādhyam}{\il}{\il} \msCa, 
                                                  °sādhya vahet \msCb, °sādhya\.m vadet \Ed}%

\vers


\jump
\begin{center}
\ketdanda iti v\textsubring{r}ṣasārasa\.mgrahe ṣaṣṭho 'dhyāyaḥ\ketdanda
\end{center}
\dontdisplaylinenum\vers 
\bekveg\szamveg\vfill\phpspagebreak\szam\bek\versno=0\fejno=7
\thispagestyle{empty}


\vers


\alfejezet{\textbf{saptamo 'dhyāyaḥ}}\jump\jump

\alalfejezet{niyameṣu dānam (4)}
dānāni ca tathety āhuḥ pañcadhā munibhiḥ purā\thinspace{\dandab} \dontdisplaylinenum
            \var{\va tathety āhuḥ\lem  \msCa\msCc\msNb\msNc\Ed; tathaity āhuḥ \msCb\msNa}%

anna\.m vastra\.m hiraṇya\.m ca bhūmi godāna pañcamam \veg\dontdisplaylinenum
            \var{\vc vastra\.m\lem  \msCa\msCb\msNa\msNc\Ed; vastra \msCc\msNb}%


\alalalfejezet{annadānam}

annāt tejaḥ sm\textsubring{r}tiḥ prāṇaḥ annāt puṣṭir vapuḥ sukham\thinspace{\dandab} \dontdisplaylinenum
            \var{\va annāt tejaḥ sm\textsubring{r}tiḥ prāṇaḥ\lem  \mssCaCbCc\msNapcorr\msNb; annāt tejaḥ sm\textsubring{r}tiḥ prāṇa \msNaacorr, 
                                                        annāt tejaḥ sm\textsubring{r}ti prāṇaḥ \msNc,
                                                        annād bhavanti bhūtāni \Ed}%

annāc chrīḥ kānti vīrya\.m ca annāt sattva\.m ca jāyate \veg\dontdisplaylinenum
            \var{\vc annāc chrīḥ\lem  \mssCaCbCc\msNa\msNc; annāc chrī \msNb\Ed\oo
                 kānti vīrya\.m ca\lem  \msCb\msCc\msNa\msNb\Ed; kāntir vīryañ ca \msCa\msNc\ \unmetr, 
                                                                        kāntivīrśyañ ca \Ed}%
            \var{\vd annāt sattva\.m ca\lem  \msCa\msCb\msNa\msNb\msNc; annā satvañ ca \msCc, annāt sattvaś ca \Ed\oo
                 jāyate\lem  \msCb\msCc\msNa\msNb\msNc\Ed; jāya{\il} \msCa}%

annāj jīvanti bhūtāni anna\.m tuṣṭikara\.m sadā\thinspace{\dandab} \dontdisplaylinenum
            \var{\va annāj jī°\lem  \msCa\msNa\msNb\Ed; annā jī° \msCb\msCc\msNc}%
            \var{\vb anna\.m\lem  \msCa\msCb\msNa\msNc\Ed; annā\.m \msCc, annā \msNb\oo
                 °kara\.m\lem  \msCa\msCb\msNa\msNb\msNc; °karaḥ \msCc\Ed}%

ānnāt kāmo mado darpaḥ annāc chaurya\.m ca jāyate \veg\dontdisplaylinenum
            \var{\vc darpaḥ\lem  \msCa\msCc\msNa\msNb; darppa \msCb\msNc, darppo \Ed}%
            \var{\vd annāc chaurya\.m ca\lem  \msCa\msCc\msNc; annāt sauryañ ca \msCb\msNa\msNb, annāc chauryaś ca \Ed}%

anna\.m kṣudhāt\textsubring{r}ṣāvyādhīn sadya eva vināśayet\thinspace{\dandab} \dontdisplaylinenum
            \var{\va anna\.m kṣu°\lem  \msCa\msCb\msNapcorr\msNc; annā kṣu° \msCc\msNaacorr, annāt kṣu° \msNb\Ed}%
            \var{\vab °vyādhīn sa°\lem  \msCb\msNc; °vyādhān sa° \msCa\msCc\msNb, °vādhān sa° \msNa, °vyādhā sa° \Ed}%
            \var{\vb vināśayet\lem  \msCa\msCc\msNa\msNb\msNc\Ed; viśayet \msCb}%

annadānāc ca saubhāgya\.m khyātiḥ kīrtiś ca jāyate \veg\dontdisplaylinenum

annadaḥ prāṇadaś caiva prāṇadaś cāpi sarvadaḥ\thinspace{\dandab} \dontdisplaylinenum
            \var{\va annadaḥ\lem  \mssCaCbCc\msNa\msNb\msNc; annada \Ed}%
            \var{\vb prāṇadaś cāpi\lem  \mssCaCbCc\msNa\msNc\Ed; prāṇaś cāpi \msNb\oo
                 sarvadaḥ\lem  \msCa\msCb\msNa\msNb\msNc\Ed; sarvadāḥ \msCc}%

tasmād annasama\.m dāna\.m na bhūta\.m na bhaviṣyati \veg\dontdisplaylinenum
            \var{\vd bhūta\.m\lem  \msCc\msNa\msNb\msNc; {\lost}tan \msCa, bhūte \msCb, bhūto \Ed}%
            \paral{\textit{\vcd {\normalfont  \kb\ MBh 13.62.6ab: }annena sad\textsubring{r}śa\.m dāna\.m na bhūta\.m na bhaviṣyati}}


\alalalfejezet{vastradānam}

vastrābhāvān manuṣyasya śriyād api parityajet\thinspace{\dandab} \dontdisplaylinenum
            \var{\va °bhāvān ma°\lem  \mssCaCbCc\msNb\Ed; °bhāvāt ma° \msNa\msNc}%
            \var{\vb śriyād api\lem  \msCa\msCc\msNa\msNb\Ed; priyād api \msCb, śriyā vāpi \msNc}%

vastrahīno na pūjyeta bhāryāputrasakhādibhiḥ \veg\dontdisplaylinenum

vidyāvān sukulīno 'pi jñānavān guṇavān api\thinspace{\dandab} \dontdisplaylinenum

vastrahīnaḥ parādhīnaḥ paribhūtaḥ pade pade \veg\dontdisplaylinenum

apamānam avajñā\.m ca vastrahīno hy avāpnuyāt\thinspace{\dandab} \dontdisplaylinenum
            \var{\va °vajñā\.m\lem  \mssCaCbCc\msNa\msNb\msNc; °vajña\.m \Ed}%
            \var{\vb °hīno\lem  \msCa\msCc\msNa\msNb\msNc\Ed; °hī \msCb}%

jugupsati mahātmāpi sabhāstrījanasa\.msadi \veg\dontdisplaylinenum

tasmād vastrapradānāni praśa\.msanti manīṣiṇaḥ\thinspace{\dandab} \dontdisplaylinenum

na jīrṇa\.m sphuṭita\.m dadyād vastra\.m kutsitam eva vā \veg\dontdisplaylinenum
            \var{\vc jīrṇa\.m sphuṭita\.m\lem  \mssCaCbCc\msNa\msNc; jīrṇasphaṭita\.m \msNb\Ed}%
            \var{\vd kutsitam eva vā\lem  \msCa\msCb\msNa\msNb\Ed; kutsitam eva ca \msCc, kutsitmeva vā \msNc}%

nava\.m purāṇarahita\.m m\textsubring{r}du sūkṣma\.m suśobhanam\thinspace{\dandab} \dontdisplaylinenum
            \var{\vb sūkṣma\.m\lem  \msCa\msCb\msNa\msNb\msNc; sūkṣma \msCc, śukla\.m \Ed}%

susa\.msk\textsubring{r}tya pradātavya\.m śraddhābhaktisamanvitam \veg\dontdisplaylinenum
            \var{\vc °dātavya\.m\lem  \msCa\msCb\msNa\msNb\msNc\Ed; °dātavya \msCc}%
            \var{\vd °samanvitam\lem  \mssCaCbCc\msNapcorr\msNb\msNc\Ed; °ta\.m \msNaacorr}%

śraddhāsattvaviśeṣeṇa deśakālavidhena ca\thinspace{\dandab} \dontdisplaylinenum
            \var{\va °sattva°\lem  \mssCaCbCc\msNa\msNb\msNc; °sa ca° \Ed}%

pātradravyaviśeṣeṇa phalam āhuḥ p\textsubring{r}thak p\textsubring{r}thak \veg\dontdisplaylinenum
            \paral{\textit{\vo {\normalfont cf. Manu 7.86--87 (the latter usually labelled as an additional verse): }
                        pātrasya hi viśeṣeṇa śraddadhānatayāiva ca{\thinspace\danda} 
                        alpa\.m vā bahu vā pretya dānasya phalam aśnute{\thinspace\ketdanda}
                        deśakālavidhānena dravya\.m śraddhāsamanvitam{\thinspace\danda}
                        pātre pradīyate yat tu tad dharmasya prasādhanam{\thinspace\ketdanda}}}

yād\textsubring{r}śa\.m dīyate vastra\.m tād\textsubring{r}śa\.m prāpyate phalam\thinspace{\dandab} \dontdisplaylinenum

jīrṇavastrapradānena jīrṇavastram avāpnuyāt \danda\dontdisplaylinenum

śobhana\.m dīyate vastra\.m śobhana\.m vastram āpnuyāt \veg\dontdisplaylinenum

\ujvers\nemsloka 
dadyād vastra suśobhana\.m dvijavare kāle śubhe sādaram
\dontdisplaylinenum
            \var{\va dvijavare kāle śubhe\lem  \mssCaCbCc\msNa\msNb\msNc; dvijayine ekāśubha\.m \Ed}%

\nemslokab 
saubhāgyam atula\.m labheta sa naro rūpa\.m tathā śobhanam \danda\dontdisplaylinenum
            \var{\vb naro\lem  \msCa\msCc\msNa\msNb\msNc\Ed; daro \msCb}%

\nemslokac 
tasmin yāti suvastrakoṭi śataśaḥ prāpnoti niḥsa\.mśayam
\dontdisplaylinenum
            \var{\vc tasmin yāti\lem \mssCaCbCc\msNb\msNc\Ed; ta\uncl{smā}n yāti \msNa\oo
                 suvastra°\lem  \mssCaCbCc\msNa\msNb\msNc; sa vastra° \Ed\oo
                 °sa\.mśayam\lem  \msCa\msCb\msNc; °sa\.mśayaḥ \msCc\msNa\msNb\Ed}%

\nemslokad 
tasmāt tva\.m kuru vastradānam asak\textsubring{r}t pāratrikotkarṣaṇam \veg\dontdisplaylinenum
            \var{\vd dānam asak\textsubring{r}t pā°\lem  \mssCaCbCc\msNa\msNc\Ed; dānasat pā° \msNb}%


\alalalfejezet{suvarṇadānam}

\vers

suvarṇadāna\.m viprendra sa\.mkṣipya kathayāmy aham\thinspace{\dandab} \dontdisplaylinenum
            \var{\va °dāna\.m\lem  \mssCaCbCc\msNa\msNc; °dāna \msNb\Ed}%

pavitra\.m maṅgala\.m puṇya\.m sarvapātakanāśanam \veg\dontdisplaylinenum
            \var{\vd °pātaka°\lem  \msCb\msCc\msNa\msNb\msNc\Ed; °pāpaka° \msCa}%

dhārayet satata\.m vipra suvarṇakaṭakāṅgulim\thinspace{\dandab} \dontdisplaylinenum
            \var{\vb °kaṭakāṅgulim\lem  \msCb\msCc\msNa\msNc\Ed; °ka{\il}{\il}gulim \msCa, °kaṭakāṅgulam \msNb}%

mucyate sarvapāpebhyo rāhunā candramā yathā \veg\dontdisplaylinenum
            \paral{\textit{\vcd {\normalfont  = 22.38 CHECK }}}

dattvā suvarṇa\.m viprebhyo devebhyaś ca dvijarṣabha\thinspace{\dandab} \dontdisplaylinenum
            \var{\va suvarṇa\.m\lem  \mssCaCbCc\msNa\msNc\Ed; suvarṇa \msNb}%
            \var{\vb °rṣabha\lem  \msCa\msCb\msNa\msNc\Ed; °rṣabhaḥ \msCc\msNb}%

tuṭimātre 'pi yo dadyāt sarvapāpaiḥ pramucyate \veg\dontdisplaylinenum
            \var{\vc tuṭi°\lem  \mssCaCbCc\msNa\msNb\msNc; truṭi° \Ed\oo
                 °mātre\lem  \mssCaCbCc\msNb\msNc; °mātro \msNa\Ed}%
            \var{\vd sarvapāpaiḥ pramucyate\lem  \msCb\msCc\msNa\msNb\msNc;
                                 sarvapāpaiḥ sa mucyate \msCa, sarvapāpai pramucyate \Ed}%

raktimāṣakakarṣa\.m vā palārdha\.m palam eva vā\thinspace{\dandab} \dontdisplaylinenum
            \var{\va raktimāṣaka°\lem  \msNcacorr; rantimāṣaka° \msCa, rattimāṣaka° \msCb\msNa\msNcpcorr, 
                                                        rantimmānsaka° \msCc, rattimānsaka° \msNb, rattimāṣaka° \Ed}%
            \var{\vb °rdha\.m\lem  \msCa\msCb\msNc\Ed; °ddha \msCc\msNa\msNb}%

evam eva phala\.m v\textsubring{r}ddhir jñeyā dānaviśeṣataḥ \veg\dontdisplaylinenum
            \var{\vcd v\textsubring{r}ddhir jñeyā\lem  \msCa\Ed; v\textsubring{r}ddhi jñeyā \msCb\msCc\msNa\msNb, v\textsubring{r}rddhi jñeyā \msNc}%


\alalalfejezet{bhūmidānam}

sarvādhāra\.m mahīdāna\.m praśa\.msanti manīṣiṇaḥ\thinspace{\dandab} \dontdisplaylinenum
            \var{\va °dhāra\.m\lem  \msCb; °dhāra° \msCa\msCc\msNa\msNb\msNc\Ed}%
            \var{\vab °dāna\.m praśa\.msanti\lem  \msCb\msCc\msNa\msNb\msNc\Ed; dā{\il}\uncl{nam pra}{\lost}santi \msCa}%

annavastrahiraṇyādi sarva\.m vai bhūmisambhavam \veg\dontdisplaylinenum
            \var{\vd sarva\.m vai\lem  \msCb\msCc\msNa\msNb\msNc\Ed; sarva\.m \uncl{ve} \msCa\ \toplost}%

bhūmidānena viprendra sarvadānaphala\.m labhet\thinspace{\dandab} \dontdisplaylinenum
            \var{\vb °phala\.m labhet\lem  \mssCaCbCc\msNa\msNb\Ed; °lala\.m bhavet \msNcacorr, °la\.m bhavet \msNc}%

bhūmidānasama\.m vipra yady asti vada tattvataḥ \veg\dontdisplaylinenum

māt\textsubring{r}kukṣivimuktas tu dharaṇīśaraṇo bhavet\thinspace{\dandab} \dontdisplaylinenum
            \var{\va °muktas tu\lem  \mssCaCbCc\msNa\msNb\msNc; °muktis tu \Ed}%
            \var{\vb °śaraṇo\lem  \mssCaCbCc\msNa\msNb; °śaraṇa \msNc, °śaraṇā\.m \Ed}%

carācarāṇā\.m sarveṣā\.m bhūmiḥ sādhāraṇā sm\textsubring{r}tā \veg\dontdisplaylinenum

ekahasta\.m dvihasta\.m vā pañcāśac chatam eva vā\thinspace{\dandab} \dontdisplaylinenum
            \var{\va ekahasta\.m\lem  \msCb\msNa\msNb\msNc; ekahasta° \msCa\msCc\Ed}%

sahasrāyutalakṣa\.m vā bhūmidāna\.m praśasyate \veg\dontdisplaylinenum
            \var{\vd bhūmidāna\.m praśasyate\lem  \msCa\msCc\msNa\msNc\Ed; bhūmidāna praśasyate \msCb, 
                        pañcāśac chatam eva vā{\danda} sahāyutalakṣam vā bhūmida\.m praśasyate \msNb\ {\normalfont (eyeskip)}}%

ekahastā\.m ca yo bhūmi\.m dadyād dvijavarāya tu\thinspace{\dandab} \dontdisplaylinenum
            \var{\va °hastā\.m ca\lem  \msCa\msCc\msNa\msNc\Ed; °hastañ ca \msCb\msNb}%
            \var{\vb dadyād dvi°\lem  \mssCaCbCc\msNa\msNb\msNc; dadyā dvi° \Ed}%

varṣakoṭiśata\.m divya\.m svargaloke mahīyate \veg\dontdisplaylinenum

eva\.m bahuṣu hasteṣu guṇāguṇi phala\.m sm\textsubring{r}tam\thinspace{\dandab} \dontdisplaylinenum
            \var{\vb guṇāguṇi°\lem  \mssCaCbCc\msNa\msNb\msNc; guṇāgaṇi° \Ed}%

śraddhādhika\.m phala\.m dāna\.m kathita\.m te dvijottama \veg\dontdisplaylinenum
            \var{\vc °dhika\.m\lem  \msCb\msCc\msNa\msNb; °dhika° \msCa\msNc\Ed}%
            \var{\vd °ttama\lem  \mssCaCbCc\msNa\msNb\Ed; °ttamaḥ \msNc}%

jāmadagnyena rāmeṇa bhūmi\.m dattvā dvijāya vai\thinspace{\dandab} \dontdisplaylinenum
            \var{\va jāmadagnyena\lem  \msCb\msNa\msNc; jāmadagnye{\il} \msCa; jāmadagnena \msCc\msNb\Ed\oo
                 rāmeṇa\lem  \msCb\msNc\Ed; rāmena \msCc\msNa\msNb, {\il}{\il}ṇa \msCa}%
            \var{\vb dattvā dvi°\lem  \msCa\msCc\msNa\msNb\msNc\Ed; dadyād dvi° \msCb}%

āyur akṣayam āpta\.m tu ihaiva ca dvijottama \veg\dontdisplaylinenum
            \var{\vd ca\lem  \mssCaCbCc\msNa\msNb\msNc; hi \Ed}%


\alalalfejezet{godānam}

hemaś\textsubring{r}ṅgā\.m raupyakhurā\.m cailaghaṇṭā\.m dvijottama\thinspace{\dandab} \dontdisplaylinenum
            \var{\vab \om\ \msNb}%
            \var{\va °ś\textsubring{r}ṅgā\.m\lem  \mssCaCbCc\msNc\Ed; °ś\textsubring{r}ṅga\.m \msNa, \om\ \msNb\oo
                 raupya°\lem  \mssCaCbCc\msNa\msNb\Ed; ropya\.m \msNc\oo
                 °khurā\.m\lem  \msCc\Ed; °kṣurā\.m \msCa\msCb\msNa\msNc, \om\ \msNb}%

viprāya vedaviduṣe dattvānantaphala\.m sm\textsubring{r}tam \veg\dontdisplaylinenum
            \var{\vd dattvānanta°\lem  \mssCaCbCc\msNa\msNb\msNc; dattvānta° \Ed}%
            \paral{\textit{\vo {\normalfont cf. e.g. MBh 7.58.18: }
                tathā gāḥ kapilā dogdhrīḥ sarṣabhāḥ pāṇḍunandanaḥ{\thinspace\danda}
                hemaś\textsubring{r}ṅgī rūpyakhurā dattvā cakre pradakṣiṇam{\thinspace\ketdanda}
                      {\normalfont and Bhaviṣyapurāṇa Uttara 12.25 CHECK: }
                hemaś\textsubring{r}\.mgī\.m raupyakhurā\.m sagha\.mṭā\.m kā\.msyadohanām{\thinspace\danda} 
                mahādevāya gā\.m dadyād dīkṣitāya dvijāya vai{\thinspace\ketdanda}}}


\alalalfejezet{dānapraśa\.msā}

\ujvers\nemsloka 
dānābhyāsarataḥ pravartanabhavā\.m śakyānurūpa\.m sadā
\dontdisplaylinenum
            \var{\va °rūpa\.m\lem  \mssCaCbCc\msNa\msNc\Ed; °rūpa \msNb}%

\nemslokab 
anna\.m vastrahiraṇyaraupyam udaka\.m gāvas tilān medinīm \danda\dontdisplaylinenum
            \var{\vb °raupya°\lem  \msCa\msCc\msNa\msNb\Ed; °ropya° \msCb, °\uncl{raupya}° \msNc\oo
                 gāvas tilān me°\lem  \eme; gāvas tilām me° \msCa\msCc\msNc, gāvas tilā me° \msCb\msNa,
                                                gāvan tilā me° \msNb, gāvas tila\.m me° \Ed}%

\nemslokac 
dadyāt pādukachattrapīṭhakalaśa\.m pātrādyam anyac ca vā
\dontdisplaylinenum
            \var{\vc dadyāt pā°\lem  \mssCaCbCc\msNa\msNc\Ed; dadyā pā° \msNb\oo
                 pātrādyam anyac ca vā\lem  \msCa\msCc\msNa\msNb\msNc;
                                         patrādyam anyac ca vā \msCb, pātreṣu labdheṣu vai \Ed}% 

\nemslokad 
śraddhādānam abhinnarāgavadana\.m k\textsubring{r}tvā mano nirmalam \veg\dontdisplaylinenum
            \var{\vd śraddhādāna°\lem  \mssCaCbCc\msNa\msNb\msNc; dattvādāna° \Ed}%

\ujvers\nemsloka 
dānād eva yaśaḥ śriyaḥ sukhakarāḥ khyāti\.m ca tulyā\.m labhet
\dontdisplaylinenum
            \var{\va yaśaḥ\lem  \msCb\msNc\Ed; yaśa \msCa\msCc\msNa\msNb\oo
                 sukhakarāḥ\lem  \mssCaCbCc\msNa\msNb\msNcacorr\Ed; sukhakara \msNcpcorr\oo
                 khyāti\.m ca tulyā\.m\lem  \eme; khyātiś ca tulya\.m \mssCaCbCc\msNa\msNb\msNc\Ed\oo
                 labhet\lem  \mssCaCbCc\msNa\msNb; bhavet \msNc\Ed}%

\nemslokab 
dānād eva nigarhaṇa\.m ripugaṇe ānandada\.m saukhyadam \danda\dontdisplaylinenum
            \var{\vb nigarhaṇa\.m\lem  \msCapcorr\msCc\msNa\Ed; nirhaṇa\.m \msCaacorr, nivarhaṇa\.m \msCb\msNc,
                                                                 nigarhana \msNb\oo
                 °gaṇe ānandada\.m saukhyadam\lem  \msCa\msCb\msNa\msNb\msNc; °gaṇai ānandada\.m saukhyadam \msCc,
                                                        °gaṇaiś cānandasaukhyapradam  \Ed}%

\nemslokac 
dānād durjayatā prasādam atula\.m saubhāgya dānāl labhet
\dontdisplaylinenum
            \var{\vc dānād du°\lem  \Ed; dānādū° \mssCaCbCc\msNa\msNb\msNc\oo
                 °rjayatā\lem  \mssCaCbCc\msNb\msNc\Ed; °rjayatām \msNa\oo
                 prasāda°\lem  \mssCaCbCc\msNb\msNc\Ed; prāsāda° \msNa\oo
                 saubhāgya\lem  \msCa\msCc\msNa\msNb\msNc; saugāgya \msCb, saubhāgya\.m \Ed\ \unmetr\oo
                 dānāl labhet\lem  \msCb\Ed; dāna\.m labhet \msCa\msCc\msNa\msNb\msNc}%

\nemslokad 
dānād eva anantabhoga niyata\.m svarga\.m ca tasmād bhavet \veg\dontdisplaylinenum
            \var{\vd dānād eva\lem  \msCa\msCb\msNa\msNb\msNc\Ed; dānād ova \msCc\oo
                 °niyata\.m\lem  \msCa\msCb\msNa\msNb\msNc\Ed; °niyata \msCc}%

\ujvers\nemsloka 
dānād eva ca śakralokasakala\.m dānāj janānandanam
\dontdisplaylinenum
            \var{\va śakralokasakala\.m\lem  \mssCaCbCc\msNb\msNc; śatrulokasakala\.m \msNa, śakralokam atula\.m \Ed\oo
                 dānāj ja°\lem  \msCc\msNa\msNb\msNc\Ed; dānā ja° \msCa, dānārja° \msCb}%

\nemslokab 
dānād eva mahī\.m samasta bubhuje samrāḍ mahīmaṇḍale \danda\dontdisplaylinenum
            \var{\vb dānād eva\lem  \msCa\msCc\msNa\msNb\msNc\Ed; dāned eva \msCb\oo
                 mahī\.m samasta\lem  \conj; mahīsamāsu \msCb\msCc, mahī\.m samā\.msu \msCa\msNa\msNc,
                                                 mahī samasta \msNb, mahīyasā\.m sa \Ed\oo
                 samrāḍ ma°\lem  \msCa\msCc\msNa\msNb\msNc\Ed; sa\.mmrāḍ ma° \msCb}%

\nemslokac 
dānād eva surūpayonisubhagaś candrānano vīkṣyate
\dontdisplaylinenum
            \var{\vc surūpa°\lem  \mssCaCbCc\msNa\msNc\Ed; svarūpa° \msNb\oo
                 °yonisu°\lem  \msNb\Ed; °yonis su° \msCa °yoniḥ su° \msCb\msCc\msNa\msNc\oo
                 °bhagaś ca°\lem  \msCa\msCc\msNb\msNc; °bhaga ca° \msCb\msNa\Ed\oo
                 °ndrānano\lem  \msCa\msCb\msNa\Ed; °ndrānane \msCc\msNb, °ndrānanau \msNc\oo
                 vīkṣyate\lem  \msCb\msCc; vīkṣate \msCa\msNa\msNb\msNc, vikṣate \Ed}%

\nemslokad 
dānād eva anekasambhavasukha\.m prāpnoti niḥsa\.mśayam \veg\dontdisplaylinenum
            \var{\vd niḥsa\.mśayam\lem  \msCa\msCb\msNc; nisa\.mśayaḥ \msCc, niḥsa\.mśayaḥ \msNa\Ed, nissayaḥ \msNb}%

\vers


\jump
\begin{center}
\ketdanda iti v\textsubring{r}ṣasārasa\.mgrahe dānapraśa\.msādhyāyaḥ saptamaḥ\ketdanda
\end{center}
\dontdisplaylinenum\vers 
            \var{{\normalfont Colophon:} °praśa\.msādhyāyaḥ saptamaḥ\lem  \msCa\msCc\msNa\msNb\msNc; 
                                °praśa\.msādhyāyaḥ samāptaḥ \msCb,
                                °praśa\.msā saptamo 'dhyāyaḥ \Ed}%
\bekveg\szamveg\vfill\phpspagebreak\szam\bek\versno=0\fejno=8
\thispagestyle{empty}



\alfejezet{\textbf{aṣṭamo 'dhyāyaḥ}}\jump\jump

\alalfejezet{niyameṣu svādhyāyaḥ (5)}
\vers

pañcasvādhyāyana\.m kāryam ihāmutra sukhārthinā\thinspace{\dandab} \dontdisplaylinenum
            \var{\vb °mutra\lem  \msCa; °mūtra \Ed}%

śaiva\.m sā\.mkhya\.m purāṇa\.m ca smārta\.m bhāratasa\.mhitām \veg\dontdisplaylinenum

śaive tattva\.m vicinteta śaivaḥ pāśupatadvaye\thinspace{\dandab} \dontdisplaylinenum
            \var{\va śaive\lem  \msCa; śaiva\.m \Ed}%
            \var{\vb śaivaḥ\lem  \msCa; śaivāḥ \Ed\oo
                 °dvaye\lem  \msCa\msCc\Ed; °ye \msCb}%

atra vistarataḥ prokta\.m tattvasārasamuccayam \veg\dontdisplaylinenum

sa\.mkhyātattva tu sā\.mkhyeṣu bodhavya\.m tattvacintakaiḥ\thinspace{\dandab} \dontdisplaylinenum
            \var{\va sa\.mkhyātattva tu\lem  \Ed; sa\.m\uncl{khyā}{\il}{\il}{\il} \msCa}%

pañcatattvavibhāgena kīrtitāni maharṣibhiḥ \veg\dontdisplaylinenum

purāṇeṣu mahīkoṣo vistareṇa prakīrtitaḥ\thinspace{\dandab} \dontdisplaylinenum

adhordhvamadhyatirya\.m ca yatnataḥ sampraveśayet \veg\dontdisplaylinenum
            \var{\vd sampraveśayet\lem  \msCa; samprabodhayet \Ed}%

smārta\.m varṇāśramācāra\.m dharmanyāyapravartanam\thinspace{\dandab} \dontdisplaylinenum
            \var{\va smārta\.m\lem  \msCa; smārta° \Ed}%
            \var{\vb °vartanam\lem  \msCa; °vartana \Ed}%

śiṣṭācāro vikalpena grāhyas tatra aśaṅkitaḥ \veg\dontdisplaylinenum
            \var{\vc °cāro\lem  \msCa; °cāra° \Ed}%
            \var{\vd grāhyas tatra aśaṅkitaḥ\lem  \Ed; grāhyas ta{\il}{\il}{\il}ṅkitaḥ \msCa}%

itihāsam adhīyānaḥ sarvajñaḥ sa naro bhavet\thinspace{\dandab} \dontdisplaylinenum

dharmārthakāmamokṣeṣu sa\.mśayas tena chidyate \veg\dontdisplaylinenum


\alalfejezet{niyameṣv upasthanigrahaḥ (6)}
ś\textsubring{r}ṇuṣvāvahito vipra pañcopasthavinigraham\thinspace{\dandab} \dontdisplaylinenum

striyo vā garhitotsargaḥ svaya\.mmuktiś ca kīrtyate \veg\dontdisplaylinenum
            \var{\vc garhitotsargaḥ\lem  \msCa; garhito svargaḥ \Ed}%

svapnopaghāta\.m viprendra divāsvapna\.m ca pañcamaḥ\thinspace{\dandab} \dontdisplaylinenum
            \var{\va °ghāta\.m\lem  \msCa; °ghāta \Ed}%

agamyā strī divā parve dharmapatny api vā bhavet \veg\dontdisplaylinenum
            \var{\vc strī divā parve\lem  \eme; 
                {\il} divāparvve \msCa, strī divāpūrve \Ed}%

viruddhastrī na seveta varṇabhraṣṭādhikāsu ca\thinspace{\dandab} \dontdisplaylinenum
            \var{\va viruddhastrī\lem  \msCa; dviruddhāstrīn \Ed}%
            \var{\vb °dhikāsu\lem  \msCa; °pikāsu \Ed}%

ajameṣagavādīnā\.m vaḍavā mahiṣīṣu ca \veg\dontdisplaylinenum

garhitotsargam ity etad yatnena parivarjayet\thinspace{\dandab} \dontdisplaylinenum

anyonyakaṣaṇā vāpi apānakaṣaṇāpi vā \veg\dontdisplaylinenum
            \var{\vc °kaṣaṇā\lem  \msCa; °karṣaṇā \Ed}%
            \var{\vd °kaṣaṇāpi\lem  \msCa; °karṣaṇāpi \Ed}%

svaya\.mmuktir iya\.m jñeyā tasmāt tā\.m parivarjayet\thinspace{\dandab} \dontdisplaylinenum
            \var{\vb tā\.m\lem  \msCa; strī \Ed}%

svapnaghāta\.m dvijaśreṣṭha aniṣṭa\.m paṇḍitaiḥ sadā \veg\dontdisplaylinenum

svapne strīṣu ramante ca retaḥ prakṣarate tataḥ\thinspace{\dandab} \dontdisplaylinenum
            \var{\vb prakṣarate\lem  \msCa; praskhalatas \Ed}%

divāśaya\.m na kartavya\.m nitya\.m dharmapareṇa tu \veg\dontdisplaylinenum 
            \var{\vd °pareṇa\lem  \Ed; °parena \msCa}%

svargamārgārgalā hy etā striyo nāma prakīrtitāḥ\thinspace{\dandab} \dontdisplaylinenum
            \var{\vb striyo\lem  \msCa; strīyo \Ed}%


\alalfejezet{niyameṣu vratapañcakam (7)}
mārjārakabakaśvānagomahīvratapañcakam \veg\dontdisplaylinenum
            \var{\vcd mārjārakavakaśvānagomahīvrata°\lem  \msCa; 
                mārjārakaś ca śvānāś ca gomahīvaka \Ed}%

svaviṣṭāmūtra\.m bhūmīṣu chādayed dvijasattama\thinspace{\dandab} \dontdisplaylinenum

sūryasomānumodanti mārjāravratikeṣu ca \veg\dontdisplaylinenum
            \var{\vc °modanti\lem  \msCa; °ṣādanti \Ed}%

bakavac cendriyagrāma\.m suniyamya tapodhana\thinspace{\dandab} \dontdisplaylinenum
            \var{\vb tapodhana\lem  \msCa; tapodhanam \Ed}%

sādhayec ca manas tuṣṭi\.m mokṣasādhanatatparaḥ \veg\dontdisplaylinenum

mūtraviṣṭena bhūmīṣu kurute chādana\.m sadā\thinspace{\dandab} \dontdisplaylinenum
            \var{\va mūtraviṣṭena\lem  \msCa; mūtraviṣṭe ca \Ed}%
            \var{\vb chādana\.m\lem  \Ed; dhanada\.m \msCa}%

tuṣyate bhagavān śarvaḥ śvānavratacaro yadi \veg\dontdisplaylinenum

mūtravarco na ruddhyeta sadā govratiko naraḥ\thinspace{\dandab} \dontdisplaylinenum
            \var{\va varco\lem  \msCa; varcā \Ed}%
            \var{\vb govratiko\lem  \Ed; {\il}{\il}tiko \msCa}%

bhīmas tuṣṭikaraś caiva purāṇeṣu nigadyate \veg\dontdisplaylinenum
            \var{\vc bhīmas\lem  \Ed; bhīma \msCa}%

kuddālair dārayanto 'pi kīlakoṭiśataiś citaḥ\thinspace{\dandab} \dontdisplaylinenum
            \var{\vb kīlakoṭiśataiś citaḥ\lem  \msCa; kīṭakoṭīśatair api \Ed}%

kṣamate p\textsubring{r}thivī devī evam eva mahīvrataḥ \veg\dontdisplaylinenum

vratapañcakam ity etad yaś careta jitendriyaḥ\thinspace{\dandab} \dontdisplaylinenum


\alalfejezet{niyameṣv upavāsaḥ (8)}
sa cottamam ida\.m loka\.m prāpnoti na ca sa\.mśayaḥ \veg\dontdisplaylinenum

śeṣānnām antarānnā\.m ca naktāyācitam eva ca\thinspace{\dandab} \dontdisplaylinenum
            \var{\va śeṣānnām antarānnā\.m\lem  \corr;
                śe\uncl{ṣā}{\il}\uncl{nnam antarānnañ} \msCa,
                śeṣāṇām antarāṇā\.m \Ed}%
            \var{\vb ca\lem  \msCa; vā \Ed}%

upavāsa\.m ca pañcaitat kathayiṣyāmi tac ch\textsubring{r}ṇu \veg\dontdisplaylinenum

vaiśvadevātithiśeṣa\.m pit\textsubring{r}śeṣa\.m ca yad bhavet\thinspace{\dandab} \dontdisplaylinenum

bh\textsubring{r}tyaputrakalatrebhyaḥ śeṣāśī vighasāśanaḥ \veg\dontdisplaylinenum
            \var{\vd vighasāśanaḥ\lem  \msCa; viṣasāsanaḥ \Ed}%

antasamprāntarāśī ca sāyamāśī tathaiva ca\thinspace{\dandab} \dontdisplaylinenum
            \var{\va antarāprāntarāśī\lem  \msCa; antasamprāntarāśī \Ed}%
            \var{\vb sāyamāśī\lem  \corr; sāyamāśīn \msCa niya° \Ed}%

sadopavāsī bhavati yo na bhuṅkte kadācana \veg\dontdisplaylinenum
            \paral{\textit{\vcd \kb\ {\normalfont  MBh 13.93.10cd: } sadopavāsī bhavati yo na bhuṅkte 'ntarā punaḥ}}

na divā bhojana\.m kārya\.m rātrau naiva ca bhojayet\thinspace{\dandab} \dontdisplaylinenum

naktavele ca bhoktavya\.m naktadharmaḥ samīhitā \veg\dontdisplaylinenum
            \var{\va ca\lem  \msCa; va \Ed}%
            \var{\vb dharma\.m samīhitā\lem  \msCa; dharmaḥ samīhitaḥ \Ed}%

anārambhasya āhāra\.m kuryān nityam ayācitam\thinspace{\dandab} \dontdisplaylinenum

parair danta\.m tu yo bhuṅkte tam ayācitam ucyate \veg\dontdisplaylinenum

bhakṣya\.m bhojya\.m ca lehya\.m ca coṣya\.m peya\.m ca pañcamam\thinspace{\dandab} \dontdisplaylinenum

na kāṅkṣen nopabhuñjīta upavāsaḥ sa ucyate \veg\dontdisplaylinenum
            \var{\vc °bhuñjīta\lem  \Ed; °{\il}{\il}ta \msCa}%


\alalfejezet{niyameṣu maunavratam (9)}
mithyā piśunapāruṣya\.m p\textsubring{r}ṣṭavāgapralāpanam\thinspace{\dandab} \dontdisplaylinenum
            \var{\va °pāruṣya\.m\lem  \msCa; °yābhinnā \Ed}%
            \var{\vb p\textsubring{r}ṣṭavāga°\lem  \msCa; p\textsubring{r}ṣtevāka° \Ed}%

maunapañcakam ity etad dhārayen niyatavrataḥ \veg\dontdisplaylinenum
            \var{\vc mauna°\lem  \msCa; mauna\.m \Ed}%
            \var{\vd dhārayen\lem  \msCa; dhārayan \Ed}%

asambhūtam ad\textsubring{r}ṣṭa\.m ca dharmāc cāpi bahiṣk\textsubring{r}taḥ\thinspace{\dandab} \dontdisplaylinenum
            \var{\vb dharmāc\lem  \msCa; dharma\.m \Ed}%

anarthaproyavākya\.m yat tan mithyāvacana\.m sm\textsubring{r}tam \veg\dontdisplaylinenum
            \var{\vc anarthā°\lem  \msCa; anartha° \Ed\oo
                yat tan\lem  \msCa; yan tan \Ed}%

parastrī\.m nābhinandanti parasyaiśvaryam eva ca\thinspace{\dandab} \dontdisplaylinenum
            \var{\va parastrī\.m\lem  \msCa; parastrī \Ed}%

aniṣṭadarśanākāṅkṣī piśunaḥ samudāh\textsubring{r}taḥ \veg\dontdisplaylinenum

m\textsubring{r}tamātā pitā caiva hānisthāna\.m katha\.m bhavet\thinspace{\dandab} \dontdisplaylinenum

bhuktvā kāmam am\textsubring{r}ṣṭānā\.m pāruṣya\.m samudāh\textsubring{r}tam \veg\dontdisplaylinenum
            \var{\vc bhuktvā\lem  \conj; bhuktva \msCa, bhuktā \Ed}%

h\textsubring{r}di na sphuṭase mūḍha śiro vā na vidāryase\thinspace{\dandab} \dontdisplaylinenum

evam ādīny anekāni tīkṣṇavādī sa ucyate \veg\dontdisplaylinenum

dyūtabhojanayuddha\.m ca madyastrīkarṣam eva ca\thinspace{\dandab} \dontdisplaylinenum
            \var{\va °yuddha\.m\lem  \msCa; °yuddhaś \Ed}%
            \var{\vb °karṣam\lem  \Ed; °kaṣam \msCa}%

asatpralāpaḥ pañcaitat kīrtita\.m te dvijottama \veg\dontdisplaylinenum
            \var{\vd te\lem  \Ed; me \msCa}%

maunam eva sadā kārya\.m vākyasaubhāgyam icchatā\thinspace{\dandab} \dontdisplaylinenum

apāruṣyam asa\.mbhinna\.m vākya\.m satyam udīrayet \veg\dontdisplaylinenum
            \var{\vc °bhinna\.m\lem  \msCa; °digdha\.m \Ed}%

yas tu maunasya no kartā dūṣitaḥ sa kulādhamaḥ\thinspace{\dandab} \dontdisplaylinenum
            \var{\vb dūṣitaḥ\lem  \msCa; bhūṣitaḥ \Ed}%

janma janma ca durgandho mūkaś caivopajāyate \veg\dontdisplaylinenum
            \var{\vc janma janma\lem  \msCa; janme janme \Ed\oo
                 durgandho\lem  \msCa; d\textsubring{r}gandho \Ed}%

\ujvers\nemsloka 
tasmān maunavrata\.m sadaiva sud\textsubring{r}ḍha\.m kurvīta yo niścitam
\dontdisplaylinenum
            \var{\va tasmān\lem  \Ed; {\il}{\il}n \msCa}%

\nemslokab 
vācā tasya alaṅghyatā ca bhavati sarvā\.m sabhyā\.m nandati \danda\dontdisplaylinenum
            \var{\vb alaṅghyatā\lem  \msCa; ala\.mghyatāñ \Ed}%

\nemslokac 
vaktrāc cotpalagandham asya satata\.m vāyanti gandhotkaṭāḥ
\dontdisplaylinenum
            \var{\vc cotpala°\lem  \msCa; cotara° \Ed}%

\nemslokad 
śāstrānekasahasraśo girinaraḥ proccāryate nirmalaḥ \veg\dontdisplaylinenum
            \var{\vd °malaḥ\lem  \Ed; °malam \msCa}%

\vers


\alalfejezet{niyameṣu snānam (10)}
snāna\.m pañcavidha\.m caiva pravakṣyāmi yathātatham\thinspace{\dandab} \dontdisplaylinenum
            \var{\vb yathātatham\lem  \Ed; {\il}{\il}tatham \msCa}%

āgneya\.m vāruṇa\.m brāhmya\.m vāyavya\.m divyam eva ca \veg\dontdisplaylinenum
            \var{\vc vāruṇa\.m\lem  \msCa; brāhmaṇa\.m \Ed}%

āgneya\.m bhasmanā snāna\.m toyāc chataguṇa\.m phalam\thinspace{\dandab} \dontdisplaylinenum

bhasmapūta\.m pavitra\.m ca bhasma pāpapraṇāśanam \veg\dontdisplaylinenum

tasmād bhasma prayuñjīta dehinā\.m tu malāpaham\thinspace{\dandab} \dontdisplaylinenum

sarvaśāntikara\.m bhasma bhasma rakṣakam uttamam \veg\dontdisplaylinenum

bhasmanā tryāyuṣa\.m k\textsubring{r}tvā brahmacaryavrate sthitam\thinspace{\dandab} \dontdisplaylinenum
            \var{\va tryāyuṣa\.m k\textsubring{r}tvā\lem  \Ed; tryāyu{\il}{\il}{\il} \msCa}%
            \var{\vb °vrate\lem  \msCa; °vrata° \Ed}%

bhasmanā \textsubring{r}ṣayaḥ sarve pavitrīk\textsubring{r}tam ātmanaḥ \veg\dontdisplaylinenum
            \var{\vc \textsubring{r}ṣayaḥ sarve\lem  \msCa; \textsubring{r}ṣibhir sarvaiḥ \Ed}%

bhasmanā vibudhā muktā vīrabhadrabhayārditāḥ\thinspace{\dandab} \dontdisplaylinenum
            \var{\va muktā\lem  \msCa; muktāḥ \Ed}%

bhasmānusa\.msa\.md\textsubring{r}ṣṭyaiva brahmaṇānumatā k\textsubring{r}taḥ \veg\dontdisplaylinenum
            \var{\vc bhasmānusa\.msa\.md\textsubring{r}ṣṭyaiva\lem  \msCa;
                        bhasmanā sa\.mprad\textsubring{r}śyāiva\.m \Ed}%
            \var{\vd brahmaṇānumatā\lem  \msCa; brāhmaṇānumato \Ed}%

cāturāśramato 'dhikya\.m vrata\.m pāśupata\.m k\textsubring{r}tam\thinspace{\dandab} \dontdisplaylinenum
            \var{\va cāturā°\lem  \msCa; caturā° \Ed}%

tasmāt pāśupata\.m śreṣṭha\.m bhasmadhāraṇahetavaḥ \veg\dontdisplaylinenum

vāruṇa\.m salilasnāna\.m kartavya\.m vividha\.m naraiḥ\thinspace{\dandab} \dontdisplaylinenum
            \var{\va vāruṇa\.m\lem  \Ed; vā{\il}{\il} \msCa}%
            \var{\vb vividha\.m\lem  \msCa; vidhivan \Ed}%

nadītoyataḍāgeṣu prasraveṣu hradeṣu ca \veg\dontdisplaylinenum

brahmasnāna\.m ca viprendra āpohiṣṭha\.m vidur budhāḥ\thinspace{\dandab} \dontdisplaylinenum

trisa\.mdhyam eva kartavya\.m brahmasnāna\.m tad ucyate \veg\dontdisplaylinenum

goṣu sa\.mcāramārgeṣu yatra godhūlisambhavaḥ\thinspace{\dandab} \dontdisplaylinenum

tatra gatvāvasīdeta snānam ukta\.m manīṣibhiḥ \veg\dontdisplaylinenum

varṣatoyāmbudhārābhiḥ plāvayitvā svakā\.m tanum\thinspace{\dandab} \dontdisplaylinenum

snāna\.m divya\.m vadaty eva jagadādimaheśvaraḥ \veg\dontdisplaylinenum

\ujvers\nemsloka 
iti niyamavibhāgaḥ pañcabhedena vipra
\dontdisplaylinenum

\nemslokab 
nigadita tava p\textsubring{r}ṣṭaḥ sarvalokānukampya \danda\dontdisplaylinenum
            \var{\vb nigadita\lem  \Ed; nigaditas \msCa\ \unmetr\oo
                 °kampya\lem  \msCa; °kampyaḥ \Ed}%

\nemslokac 
sakalamalapahāre dharmapañcāśad etat
\dontdisplaylinenum
            \var{\vc °hāre\lem  \Ed; °hāri \msCa\ \unmetr\oo
                 °pañcāśad\lem  \msCa; °pañcāśam \Ed}%

\nemslokad 
! na bhavati punarjanma kalpakoṭyāyute 'pi \veg\dontdisplaylinenum

\vers


\jump
\begin{center}
\ketdanda iti v\textsubring{r}ṣasārasa\.mgrahe niyamapraśa\.msā nāmādhyāyo 'ṣṭamo\ketdanda
\end{center}
\dontdisplaylinenum\vers 
                    \var{{\normalfont Colophon: } nāmādhyāyo 'ṣṭamo\lem  \corr;
                                nāmādhyāya aṣṭamo \msCa, nāma aṣṭamo 'dhyāyaḥ \Ed}%
\bekveg\szamveg\vfill\phpspagebreak\szam\bek\versno=0\fejno=9
\thispagestyle{empty}



\alfejezet{\textbf{9 traiguṇyam}}\jump\jump
\vers

[anarthayajña uvāca~{\dandab}\dontdisplaylinenum ]

trikālaguṇabhedena bhinna\.m sarvacarācaram\thinspace{\danda} \dontdisplaylinenum

tasmāt triguṇabandhena veṣṭita\.m nikhila\.m jagat \veg\dontdisplaylinenum

vigatarāga uvāca~{\dandab}\dontdisplaylinenum 

traikālyam iti ki\.m jñeya\.m traidhātukaśarīriṇaḥ\thinspace{\danda} \dontdisplaylinenum
            \var{\va °kālyam\lem  \Ed; °kālam \msCa\oo
                 ki\.m jñeya\.m\lem  \msCa; vijñeya\.m \Ed}%
            \var{\vb °dhātuka°\lem  \msCa; °dhāyukta° \Ed}%

ki\.mcid vistaram eveha kathayasva tapodhana \veg\dontdisplaylinenum
            \var{\vc eveha\lem  \msCa; etad dhi \Ed}%
            \var{\vd kathayasva\lem  \Ed; ka{\il}{\il}{\il} \msCa}%

anarthayajña uvāca~{\dandab}\dontdisplaylinenum 

traikālya\.m triguṇa\.m jñeya\.m vyāpī prak\textsubring{r}tisambhavaḥ\thinspace{\danda} \dontdisplaylinenum

anyonyam upajīvanti anyonyam anuvartinaḥ \veg\dontdisplaylinenum

sattva\.m rajas tamaś caiva rajaḥ sattva\.m tamas tathā\thinspace{\dandab} \dontdisplaylinenum
            \var{\va rajas\lem  \msCa; raja° \Ed}%
            \var{\vb rajaḥ\lem  \msCa; raja° \Ed\oo
                 sattva\.m\lem  \msCa; sattva° \Ed}%

tamaḥ sattva\.m rajaś caiva anyonyamithunāḥ sm\textsubring{r}tāḥ \veg\dontdisplaylinenum
            \var{\vc sattva\.m\lem  \msCa; sattva° \Ed}%

sāttviko bhagavān viṣṇu rājasaḥ kamalodbhavaḥ\thinspace{\dandab} \dontdisplaylinenum
            \var{\vb rājasaḥ kamalodbhavaḥ\lem  \Ed;
                        \uncl{rāja}{\il}{\il}{\il}{\il}{\il}{\il} \msCa}%

tāmaso bhagavān īśaḥ sakala\.m vikaleśvaraḥ \veg\dontdisplaylinenum
            \var{\vcd tāmaso bhagavān īśaḥ sakala\.m\lem  \Ed; 
                        {\il}{\il}{\il}{\il}{\il}{\il}{\il}{\il}\uncl{sakalam} \msCa}%

sattva\.m kundenduvarṇābha\.m padmarāganibha\.m rajaḥ\thinspace{\dandab} \dontdisplaylinenum

tamaś cāñjanaśailābha\.m kīrtitāni manīṣibhiḥ \veg\dontdisplaylinenum
            \var{\vc °bha\.m\lem  \msCa; °bhā \Ed}%

sattva\.m jala\.m rajo 'ṅgāra\.m tamo dhūmasamākulam\thinspace{\dandab} \dontdisplaylinenum
            \var{\va 'ṅgāra\.m\lem  \msCa; 'ṅgaran \Ed}%

etadguṇamayair baddhāḥ pacyante sarvadehinaḥ \veg\dontdisplaylinenum

vigatarāga uvāca~{\dandab}\dontdisplaylinenum 

kena kena prakāreṇa guṇapāśena badhyate\thinspace{\danda} \dontdisplaylinenum
            \var{\vb guṇa°\lem  \msNa\Ed; \om\ \msCa}%

cihnam eṣā\.m p\textsubring{r}thaktvena kathayasva tapodhana \veg\dontdisplaylinenum

anarthayajña uvāca~{\dandab}\dontdisplaylinenum 

anekākārabhāvena badhyante guṇabandhanaiḥ\thinspace{\danda} \dontdisplaylinenum

mohitā nābhijānanti jānanti śivayoginaḥ \veg\dontdisplaylinenum

ūrdhva\.mgo nityasattvastho madhyago rajasāv\textsubring{r}taḥ\thinspace{\dandab} \dontdisplaylinenum
            \var{\va ūrdhva\.mgo\lem  \conj;
                      ūrdhvāṅgo ni° \msCa\msNapcorr\Ed,
                      ūrdhvāṅgā na° \msNaacorr\oo
                 °sattva°\lem  \msCa\msNa; °satya° \Ed}%
            \var{\vb madhyago\lem  \msCa; madhyamo \Ed\oo
                 °v\textsubring{r}taḥ\lem  \msCa\msNa; °v\textsubring{r}tam \Ed}%

adhogatis tamo'vasthā bhavanti puruṣādhamāḥ \veg\dontdisplaylinenum

svarge 'pi hi trayo vaite bhāvanīyās tapodhana\thinspace{\dandab} \dontdisplaylinenum

mānuṣeṣu ca tiryeṣu guṇabhedās trayas trayaḥ \veg\dontdisplaylinenum
            \var{\vc tiryeṣu\lem  \msCa\msNa; tīryeṣu \Ed}%

brahmā viṣṇuś ca rudraś ca dharma indraḥ prajāpatiḥ\thinspace{\dandab} \dontdisplaylinenum
            \var{\vb dharma indraḥ\lem  \msCa\msNa; dharmar indra° \Ed}%

somo 'gni varuṇaḥ sūryo daśa sattvottamāḥ sm\textsubring{r}tāḥ \veg\dontdisplaylinenum
            \var{\vd daśa\lem  \msCa\msNa; daśaḥ \Ed}%

rudrādityā vasusādhyāḥ viśveśamaruto dhruvaḥ\thinspace{\dandab} \dontdisplaylinenum
            \var{\vab °dityā vasusādhyāḥ vi°\lem  \msNa;
                                °dityāvasusā{\il}{\il} \msCa,
                                °dityavasusādhyāḥ vi° \Ed}%

\textsubring{r}ṣayaḥ pitaraś caiva daśaite sattvamadhyamāḥ \veg\dontdisplaylinenum

tārā grahā surā yakṣā gandharvāḥ ki\.mnaroragāḥ\thinspace{\dandab} \dontdisplaylinenum
            \var{\vb gandharvāḥ\lem  \msCa\Ed; gandharvā \msNa}%

rakṣobhūtapiśācāś ca daśaite sāttvikādhamāḥ \veg\dontdisplaylinenum

\textsubring{r}tvik purohitācāryayajvāno 'tithivijñanī\thinspace{\dandab} \dontdisplaylinenum
            \var{\vb °vijñanī\lem  \msCa\msNa; °vijñakau \Ed}%

rājamantrī vrato vedī daśaite rājasottamāḥ \veg\dontdisplaylinenum
            \var{\vc °mantrī\lem  \msCa\msNa; °mantri \Ed}%

sūto 'mbaṣṭavaṇik cograḥ śilpakārukamāgadhāḥ\thinspace{\dandab} \dontdisplaylinenum
            \var{\va 'mbaṣṭa°\lem  \Ed; {\il}ṣṭa° \msCa\oo
                °vaṇik co°\lem  \corr;
                     °vaṇiś co° \msCa, °vaṇiśvo° \Ed}%

veṇavaidehakāmātyā daśaite rajamadhyamāḥ \veg\dontdisplaylinenum
            \var{\vc vaidehakāmātyā\lem  \msCa; vaidecakau mātyā \Ed}%

carmak\textsubring{r}tkumbhak\textsubring{r}tkolī lohak\textsubring{r}ttrapunīlikāḥ\thinspace{\dandab} \dontdisplaylinenum
            \var{\va °kolī\lem  \msCa; °kālī \Ed}%
            \var{\vb °nīlikāḥ\lem  \msCa; °tīlikā \Ed}%

naṭamuṣṭikacaṇḍālā daśaite rajasādhamāḥ \veg\dontdisplaylinenum
            \var{\vc °caṇḍālā\lem  \msCa; °cāṇḍālaḥ \Ed}%

gogajagavayā aśvam\textsubring{r}gacāmaraki\.mnarāḥ\thinspace{\dandab} \dontdisplaylinenum
            \var{\va °gavayā\lem  \msCa; °gavayo \Ed}%
            \var{\vb °cāmara°\lem  \msCa; °vānara° \Ed}%

si\.mhavyāghravarāhāś ca daśaite tamasottamāḥ \veg\dontdisplaylinenum
            \var{\vc °varāhāś\lem  \msCa; °varāhaś \Ed}%

ajameṣamahiṣyāś ca mūṣikānakulādayaḥ\thinspace{\dandab} \dontdisplaylinenum

uṣṭraraṅkuśaśagaṇḍā daśaite tamamadhyamāḥ \veg\dontdisplaylinenum
            \var{\vc uṣṭra°\lem  \msCa; da\.mṣṭri° \Ed\oo
                 °śaśagaṇḍā\lem  \msCa; °śagaṇḍāś ca \Ed}%
            \var{\vd tamamadhyamāḥ\lem  \Ed; tamadhyamāḥ \msCa}%

\textsubring{r}kṣagodhām\textsubring{r}gaś\textsubring{r}ṅgibakavānaragardabhāḥ\thinspace{\dandab} \dontdisplaylinenum
            \var{\vb °gardabhāḥ\lem  \msCa; °gardabhaḥ \Ed}%

sūkaraśvānagomāyur daśaite tamasādhamāḥ \veg\dontdisplaylinenum

krauñcaha\.msaśukaśyenabhāsavāruṇḍasārasāḥ\thinspace{\dandab} \dontdisplaylinenum

cakrāṅgaśukamāyūrā daśaite tamasāttvikāḥ \veg\dontdisplaylinenum
            \var{\vc °ṅgaśukamāyūrā\lem  \Ed; °\uncl{ṅga}{\il}{\il}{\il}yūrā \msCa}%
            \var{\vd tamasāttvikāḥ\lem  \Ed; tamassāttvikāḥ \msCa}%

valākāḥ kukkuṭāḥ kākāś cillalāvakitittirāḥ\thinspace{\dandab} \dontdisplaylinenum
            \var{\va valākāḥ\lem  \corr; valākā \msCa; valāka° \Ed\oo
                 kukkuṭāḥ kākāś\lem  \corr;
                      kukkuṭakākāś \msCa\ \unmetr; kukkuṭo kākā \Ed}%
            \var{\vb °tittirāḥ\lem  \msCa; °tittiriḥ \Ed}%

g\textsubring{r}dhrakaṅkabakaśyena daśaite tamarājasāḥ \veg\dontdisplaylinenum

kokilolūkakiñjalkakapotāḥ pañca eva ca\thinspace{\dandab} \dontdisplaylinenum
            \var{\va °kiñjalka°\lem  \Ed; °kiñjalya° \msCa}%

śārikāś ca kuliṅgāś ca daśaite tamasādhamāḥ \veg\dontdisplaylinenum
            \var{\vc śārikāś\lem  \corr; śārikā \msCa, śālikā \Ed\oo
                 kuliṅgāś\lem  \corr; kuliṅgā \msCa\Ed}%

makaragohanakrāś ca \textsubring{r}ṣā ca tamasāttvikāḥ\thinspace{\dandab} \dontdisplaylinenum
            \var{\vb tamasāttvikāḥ\lem  \Ed; tama\uncl{ssā}{\il}{\il} \msCa}%

kacchapaśuśukumbhīramaṇḍukās tamarājasāḥ \veg\dontdisplaylinenum
            \var{\vc °kumbhīra°\lem  \msCa; °kambhīrā \Ed}%

śa\.mkhaśuktikaśambūkakabandhyās tamatāmasāḥ\thinspace{\dandab} \dontdisplaylinenum
            \var{\va °śambūka°\lem  \corr; °śambūkā \msCa\Ed}%

candanāgarupadma\.m ca plakṣodumbarapippalāḥ \veg\dontdisplaylinenum
            \var{\vc °garu°\lem  \msCa; °guru° \Ed}%

vaṭadāruśamībilvā daśaite tamasāttvikāḥ\thinspace{\dandab} \dontdisplaylinenum
            \var{\vb tamasāttvikāḥ\lem  \Ed; tamassātvikāḥ \msCa}%

jāmbīralakucāmrātadāḍimākolavetasāḥ \veg\dontdisplaylinenum

nimbinīpo dhuvāvaś ca daśaite tamarājasāḥ\thinspace{\dandab} \dontdisplaylinenum
            \var{\vc dhuvāvaś\lem  \Ed; dhravāvaś \msCaacorr, 
                      dhavāvaś \msCapcorr}%
            \var{\vd daśaite\lem  \Ed; {\il}{\il}{\il} \msCa}%

v\textsubring{r}kṣavallīlatāveṇutvaksārat\textsubring{r}ṇabhūruhāḥ \veg\dontdisplaylinenum
            \var{\vd °sāra°\lem  \msCa; °sāras \Ed}%

mīrajā ca śilāśasyā daśaite tamasāttvikāḥ\thinspace{\dandab} \dontdisplaylinenum

bhramarādipataṅgāś ca krimikīṭajalaukasaḥ \veg\dontdisplaylinenum
            \var{\vc pataṅgānāś ca\lem  \msCa; pataṅgānā\.m \Ed}%
            \var{\vd krimikīṭajalaukasaḥ\lem  \msCa; kimikīṭajalaukasā\.m \Ed}%

yūkodda\.mśamaśānā\.m ca viṣṭajās tamasāttvikāḥ\thinspace{\dandab} \dontdisplaylinenum
            \var{\va yūkodda\.mśamaśānā\.m\lem  \msCa;
                      yuktoda\.mśamaśānāś \Ed}%
                  \var{\vb viṣṭajās tamasāttvikāḥ\lem  \corr; 
                             viṣṭajās tamassātvikāḥ \msCa,
                             viṣṭajā tamasāttvikāḥ \Ed}%

dayā satya\.m damaḥ śauca\.m jñāna\.m mauna\.m tapaḥ kṣamā \veg\dontdisplaylinenum

śila\.m ca nābhimāna\.m ca sāttvikāś cottamā janāḥ\thinspace{\dandab} \dontdisplaylinenum
            \var{\va nābhimāna\.m\lem  \msCa; nābhimānā\.m \Ed}%

kāmat\textsubring{r}ṣṇāratidyūtamāno yuddhamadaḥ sp\textsubring{r}hā \veg\dontdisplaylinenum

nirgh\textsubring{r}ṇāḥ kalikartāro rājaseṣūttamo janāḥ\thinspace{\dandab} \dontdisplaylinenum
            \var{\va nirgh\textsubring{r}ṇāḥ\lem  \msCa; nirgh\textsubring{r}ṇā \Ed}%
            \var{\vb rājaseṣūttamo\lem  \msCa;
                      rājase hy uttamo \Ed}%

hi\.msāsūyāgh\textsubring{r}ṇāmūḍhanidrātandrībhayālasāḥ \veg\dontdisplaylinenum
            \var{\vd °tandrī°\lem  \msCa; °tantrī° \Ed}%

krodho matsaramāyī ca tāmaseṣūttamā janāḥ\thinspace{\dandab} \dontdisplaylinenum
            \var{\va krodho\lem  \msCa; krodha° \Ed}%
            \var{\vb tāmaseṣūttamo\lem  \msCa;
                      tāmase hy uttamo \Ed}%

laghuprītiprakāśī ca dhyānayoge sadotsukaḥ \veg\dontdisplaylinenum
            \var{\vd °yoge\lem  \Ed; °\uncl{yoge} \msCa}%

prajñābuddhivirāgī ca sāttvika\.m guṇalakṣaṇam\thinspace{\dandab} \dontdisplaylinenum

bālako nipuṇo rāgī māno darpaś ca lobhakaḥ \veg\dontdisplaylinenum
            \var{\vc nipuṇo\lem  \Ed; nipuno \msCa}%

sp\textsubring{r}hā īrṣyā pralāpī ca rājasa\.m guṇalakṣaṇam\thinspace{\dandab} \dontdisplaylinenum
            \var{\vb rājasa\.m\lem  \msCa; tāmasa\.m \Ed}%

udvega ālaso mohaḥ krūras taskaranirdayaḥ \danda\dontdisplaylinenum
            \var{\vc krūras\lem  \msCa; krūra° \Ed}%

krodhaḥ piśunanidrā ca tāmasa\.m guṇalakṣaṇam \veg\dontdisplaylinenum
            \var{\vc piśuna°\lem  \Ed; piśuno \msCa}%

vigatarāga uvāca~{\dandab}\dontdisplaylinenum 

kena cihnena vijñeya āhāraḥ sarvadehinām\thinspace{\danda} \dontdisplaylinenum
            \var{\vab kena cihnena vijñeya āhāraḥ sarvadehinām\lem  \Ed;
                      {\il}{\il}{\il}{\il}{\il}{\il}{\il}{\il}{\il}{\il}{\il}{\il}{\il} dehinām \msCa}%

traiguṇyasya p\textsubring{r}thaktvena kathayasva tapodhana \veg\dontdisplaylinenum

anarthayajña uvāca~{\dandab}\dontdisplaylinenum 

āyuḥ kīrtiḥ sukha\.m prītir balārogyavivardhanam\thinspace{\danda} \dontdisplaylinenum
            \var{\va kīrtiḥ\lem  \msCa; kirtiḥ \Ed\oo
                 prītir\lem  \corr; prīti \msCa, priti \Ed}%

h\textsubring{r}dasvādurasa\.m snigdha āhāraḥ sāttvikapriyaḥ \veg\dontdisplaylinenum
            \var{\vc °rasa\.m\lem  \msCa; °rasā \Ed}%
            \var{\vd sāttvikapriyaḥ\lem  \msCa; sāttvikaḥ kiyāḥ \Ed}%

atyuṣṇam āmlalavaṇa\.m rūkṣa\.m tīkṣṇa\.m vidāhikaḥ\thinspace{\dandab} \dontdisplaylinenum
            \var{\va āmla°\lem  \msCa; alla° \Ed}%
            \var{\vb tīkṣṇa\.m\lem  \corr; tī\uncl{kṣṇa} \msCa,
                stīkṣa\.m \Ed\oo
                 vidāhikaḥ\lem  \Ed; {\il}\uncl{dāhika} \msCa}%

rājase śreṣṭham āhāro duḥkhaśokābhayapradaḥ \veg\dontdisplaylinenum

abhakṣyamedhyapūtī ca pūti paryuṣita\.m ca yat\thinspace{\dandab} \dontdisplaylinenum
            \var{\va abhakṣyamedhyapūtī ca\lem  \msCa;
                                abhakṣamadyapūtī vai \Ed}%

āyāsarasavisvāda āhāras tāmasapriyaḥ \veg\dontdisplaylinenum
            \var{\vc āyāsa°\lem  \Ed; āyāma° \msCa}%
            \var{\vd tāmasa°\lem  \msCa; tāmasaḥ \Ed}%

vigatarāga uvāca~{\dandab}\dontdisplaylinenum 

guṇātīta\.m katha\.m jñeya\.m sa\.msāraparapāragam\thinspace{\danda} \dontdisplaylinenum

guṇapāśanibaddhānā\.m mokṣa\.m kathaya tattvataḥ \veg\dontdisplaylinenum
            \var{\vc °baddhānā\.m\lem  \msCa; °baddhāmo \Ed}%

anarthayajña uvāca~{\dandab}\dontdisplaylinenum 

ātmavat sarvabhūtāni samyak paśyeta bho dvija\thinspace{\danda} \dontdisplaylinenum

guṇātītaḥ sa vijñeyaḥ sa\.msāraparapāragaḥ \veg\dontdisplaylinenum
            \var{\vc guṇātītaḥ\lem  \msCa; guṇātīta\.m \Ed}%

īrṣyādveṣasamo yas tu sukhaduḥkhasamāś ca ye\thinspace{\dandab} \dontdisplaylinenum

stutinindāsamā ye ca guṇātītaḥ sa ucyate \veg\dontdisplaylinenum

tulyapriyāpriyo yaś ca arimitrasamas tathā\thinspace{\dandab} \dontdisplaylinenum
            \var{\va tulya°\lem  \Ed; tulyaḥ \msCa}%

mānāpamānayos tulyo guṇātītaḥ sa ucyate \veg\dontdisplaylinenum 

eṣa te kathito vipra guṇasadbhāvanirṇayaḥ\thinspace{\dandab} \dontdisplaylinenum
            \var{\vb °sadbhāva°\lem  \msCa; °madbhāva° \Ed}%

guṇayuktas tu sa\.msārī guṇātītaḥ parā\.mgatiḥ \veg\dontdisplaylinenum
            \var{\vd guṇātītaḥ\lem  \msCa; guṇātīta \Ed\oo
                 °gatiḥ\lem  \eme; °gatim \msCa, °gati \Ed}%


\jump
\begin{center}
\ketdanda iti v\textsubring{r}ṣasārasa\.mgrahe traiguṇyaviśeṣanīyo nāmādhyāyo navamaḥ\ketdanda
\end{center}
\dontdisplaylinenum\vers 
            \var{{\normalfont Colophon: } nāmādhyāyo navamaḥ\lem  \msCa; 
                                nāma navamo 'dhyāyaḥ \Ed}%
\bekveg\szamveg\vfill\phpspagebreak\szam\bek\versno=0\fejno=10
\thispagestyle{empty}



\alfejezet{\textbf{10 kāyatīrthopavarṇanam}}\jump\jump
\vers

vigatarāga uvāca~{\dandab}\dontdisplaylinenum 

katama\.m sarvatīrthānā\.m śreṣṭham āhur manīṣinaḥ\thinspace{\danda} \dontdisplaylinenum
            \var{\vab tīrthānā\.m śreṣṭham\lem  \Ed; tīrthā{\il}{\il}ṣṭham \msCa}%
            \var{\vb manīṣinaḥ\lem  \msCa; manīṣibhiḥ \Ed}%

kathayasva muniśreṣṭha yady asti bhuvi kāmadam \veg\dontdisplaylinenum
            \var{\vd bhuvi\lem  \msCa; bhūri \Ed}%

anarthayajña uvāca~{\dandab}\dontdisplaylinenum 

atiguhyam ida\.m praśna\.m p\textsubring{r}ṣṭaḥ snehād dvijottama\thinspace{\danda} \dontdisplaylinenum

bravīmi vaḥ purāv\textsubring{r}tta\.m nandinā kathito 'smy aham \veg\dontdisplaylinenum

nandikeśvara uvāca~{\dandab}\dontdisplaylinenum 

kailāsaśikhare ramye siddhacāraṇasevite\thinspace{\danda} \dontdisplaylinenum
            \var{\va kailāsa°\lem  \msCa; kailāśe \Ed}%
            \paral{\textit{\vab {\normalfont  cf.\ MBh 12.327.18cd: } merau girivare ramye siddhacāraṇasevite }}

tatrāsīna\.m śiva\.m sākṣād devī vacanam abravīt \veg\dontdisplaylinenum

devy uvāca~{\dandab}\dontdisplaylinenum 

bhagavan devadeveśa sarvabhūtajagatpate\thinspace{\danda} \dontdisplaylinenum

praṣṭum icchāmy aha\.m tv eka\.m dharmaguhya\.m sanātanam \veg\dontdisplaylinenum

atitīrthapara\.m guhya\.m sa\.msārād yena mucyate\thinspace{\dandab} \dontdisplaylinenum

manuṣyāṇā\.m hitārthāya brūhi tattva\.m maheśvara \veg\dontdisplaylinenum

maheśvara uvāca~{\dandab}\dontdisplaylinenum 

ko mā\.m p\textsubring{r}cchati tat praśna\.m muktvā tvām eva sundari\thinspace{\danda} \dontdisplaylinenum
            \var{\va praśna\.m\lem  \Ed; praśna \msCa}%
            \var{\vb muktvā\lem  \msCa; muktā \Ed}%

ś\textsubring{r}ṇu vakṣyāmi tat praśna\.m devair api sudurlabham \veg\dontdisplaylinenum

kurukṣetra\.m prayāga\.m ca vārāṇasīm ataḥ param\thinspace{\dandab} \dontdisplaylinenum

gaṅgāgnisomatīrtha\.m ca sūryapuṣkaramānasam \veg\dontdisplaylinenum

naimiṣa\.m bindusāra\.m ca setubandha\.m surahradam\thinspace{\dandab} \dontdisplaylinenum
            \var{\vb °bandha\.m\lem  \msCa; °bandha° \Ed}%

ghaṇṭikeśvaravāgīśa\.m jñātvā niścayapāpahā \veg\dontdisplaylinenum
            \var{\vd niścayapāpahā\lem  \Ed; niśca\uncl{ya}{\il}{\il}{\il} \msCa}%

umovāca~{\dandab}\dontdisplaylinenum 

evamādi mahādeva pūrvavat kathitā 'smy aham\thinspace{\danda} \dontdisplaylinenum
            \var{\vb kathitā\lem  \msCa; kathito \Ed}%

svargabhogaprada\.m tīrtham eteṣā\.m suranāyaka \veg\dontdisplaylinenum
            \var{\vd suranāyaka\lem  \msCapcorr; suranāka \msCaacorr,
                                        suranāyakam \Ed}%

katha\.m mucyate sa\.msārāj jñānamātreṇa īśvara\thinspace{\dandab} \dontdisplaylinenum

kautūhala\.m mahaj jāta\.m chindhi sa\.mśayakārakam \veg\dontdisplaylinenum
            \var{\vd °kārakam\lem  \Ed; °kāraka \msCa}%

rudra uvāca~{\dandab}\dontdisplaylinenum 

ki\.m na jānāsi tat tīrtha\.m sulabha\.m durlabha\.m ca yat\thinspace{\danda} \dontdisplaylinenum

sulabha\.m gurusevīnā\.m durlabha\.m tadvivarjanāt \veg\dontdisplaylinenum
            \var{\vc sulabha\.m gurusevīnā\.m\lem  \Ed;
                      {\il}{\il}{\il}{\il}{\il}{\il}vīnā\.m \msCa}%
            \var{\vd °varjanāt\lem  \Ed; °varjayet \msCa}%

kuruḥ puruṣa vijñeyaḥ śarīra\.m kṣetra ucyate\thinspace{\dandab} \dontdisplaylinenum
            \var{\va puruṣa\lem  \Ed; puruṣaḥ \msCa\ \unmetr}%
            \var{\vb śarīra\.m\lem  \Ed; śarī\uncl{ra} \msCa}%

śarīrastha\.m kurukṣetra\.m sarvatīrthaphalapradam \veg\dontdisplaylinenum

sarvayajñaphalāvāptiḥ sarvadānaphalāni ca\thinspace{\dandab} \dontdisplaylinenum

sarvavratatapaś cīrṇa\.m tatphala\.m sakala\.m bhavet \veg\dontdisplaylinenum

evam eva phala\.m teṣā\.m tīrthapañcadaśeṣu ca\thinspace{\dandab} \dontdisplaylinenum

snānadhyāna\.m mahāpuṇya\.m mahātīrtha\.m mahāsukham \veg\dontdisplaylinenum
            \var{\vc snānadhyāna\.m mahāpuṇya\.m\lem  \Ed; {\il}{\il}{\il}{\il}{\il}{\il}puṇya \msCa}%

devy uvāca~{\dandab}\dontdisplaylinenum 

atīva romaharṣo me jāto 'sti tridaśeśvara\thinspace{\danda} \dontdisplaylinenum

sulabha\.m sukara\.m sūkṣma\.m śrutvā tuṣṭiś ca me gatā \veg\dontdisplaylinenum

caturdaśaparo bhūyaḥ kathayasva manoharam\thinspace{\dandab} \dontdisplaylinenum

prayāgādi p\textsubring{r}thaktvena tattvatas tu sureśvara \veg\dontdisplaylinenum

rudra uvāca~{\dandab}\dontdisplaylinenum 

suṣumṇā bhavatī gaṅgā iḍā ca yamunā nadī\thinspace{\danda} \dontdisplaylinenum
            \var{\va bhavatī gaṅgā\lem  \Ed; bhagavatī ga{\il} \msCa}%

etā śrotavahā nadyaḥ prayāgaḥ sa vidhīyate \veg\dontdisplaylinenum

dakṣiṇā vāruṇī nāsā vāmanāsā asi sm\textsubring{r}tā\thinspace{\dandab} \dontdisplaylinenum
            \var{\va dakṣiṇā\lem  \Ed; dakṣi\uncl{ṇa\.m} \msCa\oo
                 vāruṇī\lem  \Ed; varuṇī \msCa}%

vāruṇā-asimadhyena tena vārāṇasī sm\textsubring{r}tā \veg\dontdisplaylinenum

ākāśagaṅgā vikhyātā tasyāḥ sravati cām\textsubring{r}tam\thinspace{\dandab} \dontdisplaylinenum

ahorātram avicchinna\.m gaṅgā sā tena ucyate \veg\dontdisplaylinenum

somatīrtham iḍānāḍī kiṅkiṇīravacihnitā\thinspace{\dandab} \dontdisplaylinenum
            \var{\vb °rava°\lem  \msCa; °rāva° \Ed}%

ta\.m tu śrutvā na sa\.mdehaḥ sarvapāpakṣayo bhavet \veg\dontdisplaylinenum
            \var{\vc ta\.m tu\lem  \corr; \uncl{tantu} \msCa; tantu \Ed}%

sūryatīrtha\.m suṣumṇā ca nīravāravasa\.myutā\thinspace{\dandab} \dontdisplaylinenum
            \var{\vb nīravā°\lem  \Ed; vīravā° \msCa}%

śrutimātrād vimucyeta pāparāśir mahān api \veg\dontdisplaylinenum

agnitīrthārjunā nāḍī brahmaghoṣamanoramā\thinspace{\dandab} \dontdisplaylinenum
            \var{\va °rjunā\lem  \msCa; °rjuna\.m \Ed}%
            \var{\vb °ramā\lem  \msCa; °ramāḥ \Ed}%

tat tad akṣaram ākarṇya am\textsubring{r}tattvāya kalpate \veg\dontdisplaylinenum

puṣkara\.m h\textsubring{r}di madhyastham aṣṭapattra\.m sakarṇikam\thinspace{\dandab} \dontdisplaylinenum
            \var{\vb °karṇikam\lem  \corr; {\il}{\il}{\il} \msCa, °karṇikām \Ed}%

cintayet sūkṣma tanmadhye janmam\textsubring{r}tyuvināśanam \veg\dontdisplaylinenum
            \var{\vc sūkṣma\lem  \corr; \uncl{sūkṣma} \msCa, sūkṣma\.m \Ed}%

mānasa\.m saramadhyastha\.m saha\.msakamalopari\thinspace{\dandab} \dontdisplaylinenum
            \var{\va mānasa\.m\lem  \Ed; \uncl{mānasa} \msCa}%

salīlo līlayācārī parataḥ parapāragaḥ \veg\dontdisplaylinenum
            \var{\vc salīlo\lem  \msCa; salīlā \Ed}%

naimiṣa\.m ś\textsubring{r}ṇu deveśi nimiṣā pratyayo bhavet\thinspace{\dandab} \dontdisplaylinenum

samyag chāyā\.m nirīkṣeta svātmāno vā parasya vā \veg\dontdisplaylinenum
            \var{\vd svātmāno\lem  {\il}nmano \msCa\oo
                 vā\lem  \msCa; ca \Ed}%

āyātapy aṅgulīmātra\.m nimiṣākṣi sa paśyati\thinspace{\dandab} \dontdisplaylinenum
            \var{\va °mātra\.m\lem  \msCa; °madhye \Ed}%

d\textsubring{r}ṣṭvā pratyayam eva\.m hi naimiṣajñas sa ucyate \veg\dontdisplaylinenum

tīrtha\.m bindusara\.m nāma ś\textsubring{r}ṇu vakṣyāmi sundari\thinspace{\dandab} \dontdisplaylinenum
            \var{\va tīrtha\.m bindu°\lem  \msCa; tīrtham indu° \Ed}%

dehamadhye h\textsubring{r}di jñeya\.m h\textsubring{r}di madhye tu paṅkajam \veg\dontdisplaylinenum

karṇikā padmamadhye tu binduḥ karṇikamadhyataḥ\thinspace{\dandab} \dontdisplaylinenum
            \var{\va °madhye\lem  \Ed; °dhye \msCa}%

bindumadhye sthito nādaḥ sa nādaḥ kena bhidyate \veg\dontdisplaylinenum
            \var{\vc bindumadhye\lem  \Ed; \uncl{bindu}{\il}{\il} \msCa}%
            \var{\vd bhidyate\lem  \Ed; \uncl{vi}dyate \msCa}%

ukāra\.m ca makāra\.m ca bhitvā nādo vinirgataḥ\thinspace{\dandab} \dontdisplaylinenum
            \var{\va ukāra\.m ca makāra\.m\lem  \msCa;
                ukāraś ca makāraś \Ed}%

ta\.m viditvā viśālākṣi so 'm\textsubring{r}tatva\.m labheta vā \veg\dontdisplaylinenum
            \var{\vd so 'm\textsubring{r}tatva\.m\lem  \msCa; somatatva\.m \Ed}%

\ujvers\nemsloka 
vakṣye te setubandha\.m duritamalahara\.m nādatoyapravāham
\dontdisplaylinenum
            \var{\va te\lem  \msCapcorr\Ed; \om\ \msCaacorr}%

\nemslokab 
jihvākaṇṭhorukūlāsuragaṇapulināvartaghoṣā taraṅgā \danda\dontdisplaylinenum

\nemslokac 
kumbhīrāghoṣamīnā daśagaṇamakarā bhīmanakrāvisargāḥ
\dontdisplaylinenum
            \var{\vc °mīnā\lem  \msCa; °mānā \Ed\oo
                 daśa°\lem  \Ed; {\il}{\il} \msCa}%

\nemslokad 
sānusvāre gabhīre madasukharamaṇa\.m setubandha\.m vrajasva \veg\dontdisplaylinenum
            \var{\vd °ramaṇa\.m\lem  \Ed; °ramana\.m \msCa\oo
                 vrajasva\lem  \msCa; ramasva \Ed}%

\ujvers\nemsloka 
saptadvīpāntamadhye ś\textsubring{r}ṇu śaśivadane sarvaduḥkhāntalābham
\dontdisplaylinenum

\nemslokab 
īśānenābhijuṣṭa\.m h\textsubring{r}di hradavimala\.m nāma śītāmbupūrṇam \danda\dontdisplaylinenum
            \var{\vb °juṣṭa\.m\lem  \Ed; °duṣṭa\.m \msCa\oo
                 nāma\lem  \Ed; nāda \msCa}%

\nemslokac 
tatraika\.m jātapadma\.m prak\textsubring{r}tidalayuta\.m keśara\.m śaktibhinnam
\dontdisplaylinenum

\nemslokad 
pañcavyomapraśasta\.m gatiparamapada\.m prāptukāmena sevyam \veg\dontdisplaylinenum
            \var{\vd sevyam\lem  \msCa; sarvam \Ed}%

\ujvers\nemsloka 
! nāḍyaikāsaṅgatāni nipatitam am\textsubring{r}ta\.m ghaṇṭikāpārakeṇa
\dontdisplaylinenum
            \var{\va °pārakeṇa\lem  \msCa; °yāṅkareṇa \Ed}%

\nemslokab 
t\textsubring{r}pyante tena nitya\.m h\textsubring{r}dikamalapuṭa\.m sthānabhūtāntarātmā \danda\dontdisplaylinenum

\nemslokac 
ya\.m paśyantīśabhaktā kalikaluṣahara\.m vyāpina\.m niṣprapañcam
\dontdisplaylinenum
            \var{\vc paśyantīśabhaktā\lem  \msCa; paśyannīśamakṣā \Ed\oo
                 °prapañcam\lem  \msCa; °prapañca \Ed}%

\nemslokad 
! deveśa\.m ghaṇṭikeśamarabhavam abhavantīrtham ākāśabindum \veg\dontdisplaylinenum
            \var{\vd deveśa\.m\lem  \Ed; devyeśa\.m \msCa\oo
                 ghaṇṭikeśa°\lem  \msCa; ghāṇṭakeśā° \Ed\oo
                 °bhavantīrtham\lem  \msCa; bhava{\il}{\il}rtham \Ed}%

\ujvers\nemsloka 
mīmā\.msāratnakūlā kramapadapulinā śaivaśāstrārthatoyā
\dontdisplaylinenum
            \var{\va śaiva°\lem  \msCa; śarva° \Ed}%

\nemslokab 
mīnaughā pañcarātra\.m śrutikuṭilagatismārtavegā taraṅgā \danda\dontdisplaylinenum
            \var{\vb mīnaughā°\lem  \Ed;  mīnoghā° \Ed\oo
        % CHECK app
                 pañcarātra\.m\lem  \msCa; pañcaśatra\.m \Ed\oo
                 °vegā\lem  \msCa; °vegās \Ed}%

\nemslokac 
yogāvartātiśobhā upaniṣadivahā bhāratāvartaphenā
\dontdisplaylinenum

\nemslokad 
pañcāsadvyomarūpī rasabhavananadī tīrthavāgīśvarīyam \veg\dontdisplaylinenum

\ujvers\nemsloka 
yas ta\.m vetti sa vetti vedanikhila\.m sa\.msāraduḥkhacchidam
\dontdisplaylinenum

\nemslokab 
janmavyādhiviyogatāpamaraṇa\.m kleśārṇava\.m duḥsaham \danda\dontdisplaylinenum
            \var{\vb °rṇava\.m\lem  \msCa; °rṇava \Ed}%

\nemslokac 
garbhāvāsam atīva sahyaviṣaya\.m dustīrya duḥkhālayam
\dontdisplaylinenum
            \var{\vc garbhāvāsam\lem  \msCa; garbhovāsam \Ed\oo
                 °viṣaya\.m\lem  \msCa; °viṣama\.m \Ed}%

\nemslokad 
prāpta\.m tena na sa\.mśayaḥ śivapada\.m duṣprāpya devair api \veg\dontdisplaylinenum
            \var{\vd sa\.mśayaḥ\lem  \msCa; sa\.mśaya\.m \Ed}%

\vers


\jump
\begin{center}
\ketdanda iti v\textsubring{r}sasārasa\.mgrahe kāyatīrthopavarṇano nāma daśamo 'dhyāyaḥ\ketdanda
\end{center}
\dontdisplaylinenum\vers 
    \var{\v kāyatīrthopavarṇano\lem  \Ed; kāyatī{\il}{\il}{\il}rṇṇano \msCa\oo
         nāmādhyāyo daśamaḥ\lem  \msCa; nāma daśamo 'dhyāyaḥ \Ed}%
\bekveg\szamveg\vfill\phpspagebreak\szam\bek\versno=0\fejno=11
\thispagestyle{empty}



\alfejezet{\textbf{ekādaśamo 'dhyāyaḥ}}\jump\jump 
\vers


\alalfejezet{caturāśramadharmavidhānaḥ}
devy uvāca~{\dandab}\dontdisplaylinenum 

sarvayajñaḥ paraśreṣṭha asti anyaḥ surottama\thinspace{\danda} \dontdisplaylinenum 
            \var{\vb anyaḥ\lem  \msCb\msNa\msNc; anya \msCa\msCc\msNb, cānyā \Ed\oo
                 °ttama\lem  \mssCaCbCc\msNa\msNb\Ed; °ttamaḥ \msNc}%
            \paral{\textit{{\normalfont Testimonia for this chapter:    \msCa\ ff.\thinspace 208v--210r,
                                                \msCb\ ff.\thinspace 214r--215v,
                                                \msCc\ ff.\thinspace 285v--287v,
                                                \msNa\ ff.\thinspace 15v--17v,
                                                \msNb\ ff.\thinspace 221v--223v,
                                                \msNc\ ff.\thinspace 223v--225v;
                                                        \mssCaCbCc\ = \msCa + \msCb + \msCc }}}

alpakleśam anāyāsa arthaprāya\.m vineśvara \veg\dontdisplaylinenum
            \var{\vc °nāyāsa\lem  \mssCaCbCc\msNc\Ed; °nāyā\uncl{sa\.m} \msNa, °\uncl{nāyā}sa\.m \msNb}%
            \var{\vd °rthaprāya\.m\lem  \msNapcorr\msNc; °rthaprāya \mssCaCbCc,
                      °rthaprārthaprāya\.m \msNaacorr, \uncl{°rthaprāya} \msNb, °thāmnāya \Ed\oo
                 vineśvara\lem  \mssCaCbCc\msNa\msNc; \uncl{vineśvara} \msNb, sureśvara \Ed}%

sarvayajñaphalāvāpti daivataiś cāpi pūjitam\thinspace{\dandab} \dontdisplaylinenum
            \var{\va daivatai°\lem  \msCa\msCb\msNa\Ed; devatai° \msCc\msNc, \uncl{devatai} \msNb}%      

kathayasva suraśreṣṭha mānuṣāṇā\.m hitāya vai \veg\dontdisplaylinenum
            \var{\vcd °śreṣṭha mānuṣāṇā\.m hitāya vai\lem  \mssCaCbCc\msNa\msNc\Ed; °śre{\lost}{\lost}{\lost}{\lost}{\lost}{\lost}{\lost}{\lost}{\lost}{\lost} \msNb}%

maheśvara uvāca~{\dandab}\dontdisplaylinenum 
            \var{\vo mahe°\lem  \mssCaCbCc\msNa\msNb\Ed; mehe° \msNc}%

na tulya\.m tava paśyāmi dayā bhūteṣu bhāmini\thinspace{\danda} \dontdisplaylinenum
            \var{\va tulya\.m tava\lem  \msNa\msCb\msCc\msNb\msNc\Ed; {\lost}{\lost}{\lost}{\lost} \msCa}%
            \var{\vb bhāmini\lem  \msCa\msCb\msNa\msNb\msNc\Ed; bhāmi \msCc}%

kim anyat kathayiṣyāmi dayā yatra na vidyate \veg\dontdisplaylinenum
            \var{\vc kim anya°\lem  \mssCaCbCc\msNa\msNc\Ed; kimyanya° \msNb}%

sadāśivamukhāt pūrva\.m śruta\.m me varasundari\thinspace{\dandab} \dontdisplaylinenum

ś\textsubring{r}ṇu devi pravakṣyāmi dharmasāram anuttamam \veg\dontdisplaylinenum
            \var{\vc devi pravakṣyāmi\lem  \msCb\msCc\msNa\msNb; te devi vakṣyāmi \msCa\msNc\Ed}%
            \var{\vd °sāram anuttamam\lem  \msCa\msCb\msNa\msNb\msNc\Ed; °sārasamuccayam \msCc}%


\alalfejezet{g\textsubring{r}hasthaḥ(?)}
vinārthena tu yo yajñaḥ sa yajñaḥ sārvakāmikaḥ\thinspace{\dandab} \dontdisplaylinenum
            \var{\vb yajñaḥ\lem  \mssCaCbCc\msNa\msNb\msNc; yajña \Ed\oo
                 sārvakāmikaḥ\lem  \msCb\Ed; sarvakālikaḥ \msCa\msNc,
                                   sarvakāmika \msCc, sārvakālikaḥ \msNa, sārvakāmikāḥ \msNb}%
             \paral{\textit{\vab {\normalfont  See a sequence or list of the four āśramas in 4.75: }
                g\textsubring{r}hastho brahmacārī ca vānaprastho 'tha bhaikṣukaḥ;
                {\normalfont  see also 5.9: } 
                etac chauca\.m g\textsubring{r}hasthānā\.m dviguṇa\.m brahmacāriṇām{\thinspace\danda}
                vānaprasthasya triguṇa\.m yatīnā\.m tu caturguṇam{\thinspace\ketdanda}}}

akṣayaś cāvyayaś caiva sarvapātakanāśanaḥ \veg\dontdisplaylinenum
            \var{\vc akṣayaś cāvyayaś\lem  \msCb\msNb\msNc\Ed; akṣaya\.m cāvyaya\.m \msCa\msCc\msNa}%
            \var{\vd °nāśanaḥ\lem  \msCa\msNa\msNb\msNc; °nāśanam \msCb\Ed, °nāśana \msCc}%

bahuvighnakaro hy artho bahvāyāsakaras tathā\thinspace{\dandab} \dontdisplaylinenum
            \var{\va °karo\lem  \msCa\msCb\msNa\msNb\msNc; °karā \msCc\Ed\oo
                 hy artho\lem  \mssCaCbCc\msNa\msNb\msNc; hy ertho \Ed}%
            \var{\vb karas tathā\lem  \mssCaCbCc\msNa\msNb\msNc; karatasthā \Ed}%

brahmahatyā ivendrasya pravibhāgaphalā sm\textsubring{r}tā \veg\dontdisplaylinenum
            \var{\vd pravibhāga°\lem  \msCb; pravibhoga° \msCa\msCc(?)\msNa\msNc\Ed, pratibhoga° \msNb\oo
                 °phalā sm\textsubring{r}tā\lem  \msCc; °phalaḥ sm\textsubring{r}taḥ \msCapcorr\msCb\msNa\msNb\msNc; 
                                                °phala sm\textsubring{r}taḥ \msCaacorr, °pradaḥ sm\textsubring{r}taḥ \Ed}%
                     \paral{\textit{\vcd {\normalfont See e.g.\ \BhP\ 6.9.6: } brahmahatyām añjalinā jagrāha yad apīśvaraḥ{\thinspace\danda}
                                    sa\.mvatsarānte tad agha\.m bhūtānā\.m sa viśuddhaye{\thinspace\danda}
                                    bhūmyambudrumayoṣidbhyaś caturdhā vyabhajad dhariḥ{\thinspace\ketdanda}}}

pañcaśodhyena śodhyeta arthayajño varānane\thinspace{\dandab} \dontdisplaylinenum
            \var{°yajño\lem  \msCa\msCb\msNa\msNb\msNc\Ed; °yajña \msCc}%

śodhite tu phala\.m śuddham aśuddhe niṣphala\.m bhavet \veg\dontdisplaylinenum
            \var{\vcd śuddham aśuddhe\lem  \mssCaCbCc\msNb\msNc; śuddha\.mm aśuddhe \msNa, śuddham aśuddha\.m \Ed}%

devy uvāca~{\dandab}\dontdisplaylinenum 
            \var{\vo devy uvāca\lem  \mssCaCbCc\msNa\msNbpcorr\msNc\Ed; \om\ \msNbacorr}%

pañcaśodhye suraśreṣṭha sa\.mśayo 'tra bhaven mama\thinspace{\danda} \dontdisplaylinenum
            \var{\va °śodhye\lem  \mssCaCbCc\msNa; °śodhya \msNb\msNc, °śodhyaḥ \Ed\oo
                 °śreṣṭha\lem  \msCa\msCb\msNa\msNb\msNc\Ed; °sre\uncl{mna} \msCc}%
            \var{\vb 'tra bhave°\lem  \mssCaCbCc\msNa\msNb\msNc; 'trā bhava° \Ed}%

kathayasva vibhāgena śrotum icchāmi tattvataḥ \veg\dontdisplaylinenum

rudra uvāca~{\dandab}\dontdisplaylinenum 

manaḥśuddhis tu prathama\.m dravyaśuddhir ataḥ param\thinspace{\danda} \dontdisplaylinenum
            \var{\vb °śuddhir ataḥ\lem  \mssCaCbCc\msNa\msNc\Ed; °śuddhigataḥ \msNb}%

mantraśuddhis t\textsubring{r}tīyā tu karmaśuddhir ataḥ param \danda\dontdisplaylinenum
            \var{\va mantraśuddhis t\textsubring{r}tīyā\lem  \mssCaCbCc\msNa\msNb\Ed; mantraddhi t\textsubring{r}tīyā \msNc}%
            \var{\vb karmaśuddhi°\lem  \mssCaCbCc\msNa\msNb\Ed; karmasiddhi \msNc}%

pañcamī sattvaśuddhis tu kratuśuddhiś ca pañcadhā \veg\dontdisplaylinenum
            \var{\vc pañcamī\lem  \mssCaCbCc\msNa\msNb\msNc; pañcama\.m \Ed\oo
                 °śuddhis tu\lem  \mssCaCbCc\msNb\msNc; °śuddhiś ca \msNa\Ed}%
            \var{\vd °śuddhiś ca pañcadhā\lem  \msCa\msCb\msNb\msNc\Ed; °śuddhis tu pañcadhā \msCc,
                                                                °śuddhir ataḥ param \msNa}%

manaḥśuddhir nāma aviparītabhāvanayā\thinspace{\dandab} \dontdisplaylinenum 
            \var{\vab °śuddhir nā°\lem  \msCa\msCb\msNa\msNb\msNc\Ed; °śuddhi nā° \msCc\oo
                  °bhāvanayā\lem  \mssCaCbCc\msNc\Ed; °bhāvanavā \msNa, °bhāvanatayā \msNb}%

dravyaśuddhir nāma ananyāyopārjitadravyena \veg\dontdisplaylinenum 
            \var{\vcd °śuddhir nā°\lem  \msCa\msCb\msNa\msNb\Ed; °śuddhi nā° \msCc\msNc\oo
                  ananyāyo°\lem  \msCb\msNa\msNb\msNc; ana{\lost}yo° \msCa, anyāyo° \msCc, svalponyāyo° \Ed\oo
                  °dravyena\lem  \mssCaCbCc\msNa\msNc\Ed; °vyena \msNb}%

mantraśuddhir nāma svaravyañjanayuktatayā\thinspace{\dandab} \dontdisplaylinenum 
            \var{\vab mantraśuddhir °nā\lem  \msCa\msCb\msNb\Ed; mantraśuddhi nā° \msCc\msNc, mantras tuddinā° \msNa\oo
                  °yuktatayā\lem  \msCa\msCc\msNa\msNb\msNc\Ed; °yuktayā \msCb}%

kriyāśuddhir nāma yathākramāviparītatayā \danda\dontdisplaylinenum 
            \var{\vcd °śuddhir nā°\lem  \msCa\msCb\msNa\msNc\Ed; °śuddhi nā° \msCc\msNb\oo
                  °kramā°\lem  \msCa\msCb\msNa\msNb\msNc\Ed; °krama° \msCc\oo
                  °rītatayā\lem  \msCa\msCc\msNa\msNb\Ed; °rītayā \msCb, °{\lost}{\lost}tayā \msNc}%

sattvaśuddhir nāma rajastama-apradhānatayā \veg\dontdisplaylinenum 
            \var{\vef °śuddhir nā°\lem  \msCb\msNa\msNb\msNc\Ed; °śuddhi nā° \msCa\msCc\oo
                  °dhānatayā\lem  \mssCaCbCc\msNa\msNb\Ed; °dhānata \msNc}%

\vers

vidhim eva\.m yadā śudhyed yadi yajña\.m karoti hi\thinspace{\dandab} \dontdisplaylinenum
            \var{\va °dhim eva\.m yadā\lem  \msCb\Ed; °dhim eva yadā \msCa\msCc\msNa, °dhim eva ya \msNb,
                                                              °dhim eva\.m yathā \msNc}%
            \var{\vab śudhyed yadi\lem  \conj; sūyed yadi \msCa\msNa\msNc, pūrya yadi \msCb,  
                                 sūryed yadi \msCc, sūyed yati \msNb, śuddhya ya° \Ed}%
            \var{\vb yajña\.m\lem  \msCa\msCb\msNa\Ed; yajña \msCc\msNc, sa\.mjña \msNb\oo
                 hi\lem  \mssCaCbCc\msNa\msNc\Ed; \om\ \msNb}%

tasya yajñaphalāvāptir janmam\textsubring{r}tyuś ca no bhavet \veg\dontdisplaylinenum
            \var{\vcd °vāptir ja°\lem  \msCa\msCb\Ed; °vāpti ja \msCc\msNb\msNc, °vāpi ja° \msNa}%

vinārthena tu yo yajña\.m karoti varasundari\thinspace{\dandab} \dontdisplaylinenum
            \var{\vb °sundari\lem  \mssCaCbCc\msNa\msNb\msNc; °sundarī \Ed}%

na tasya tatphalāvāptiḥ sarvayajñeṣv aśeṣataḥ \veg\dontdisplaylinenum
            \var{\vd °yajñeṣv aśeṣataḥ\lem  \mssCaCbCc\msNa\msNb\msNc; °yajñeṣu śeṣataḥ \Ed}%

yajñavāṭa kurukṣetra\.m sattvāvāsak\textsubring{r}tālayaḥ\thinspace{\dandab} \dontdisplaylinenum
            \var{\va °vāṭa kuru°\lem  \msCa\msCc\msNa\msNb\msNc; °vāṭaṅ kuru° \msCb, °vāṭak\textsubring{r}ta° \Ed\oo
                 °kṣetra\.m\lem  \mssCaCbCc\msNa\msNb\Ed; °kṣetra \msNc}%
            \var{\vb sattvā°\lem  \msCa\msCbpcorr\msCc\msNa\msNb\msNc\Ed; satvāsatvā° \msCbacorr\oo
                 °layaḥ\lem  \msCa\msCb\msNa\msNb\msNc\Ed; °layam \msCc}%

pratyāhāra mahāvediḥ kuśaprastarasa\.myamaḥ \veg\dontdisplaylinenum
            \var{\vc °vediḥ\lem  \eme; °vedi \mssCaCbCc\msNa\msNb\msNc; °devi \Ed}%

vidhi niyamavistāro dhyānavahnipradīpitaḥ\thinspace{\dandab} \dontdisplaylinenum
            \var{\va vidhi\lem  \mssCaCbCc\msNa\msNb\msNc; vidhir \Ed\oo
                 °vistāro\lem  \msCa\msCc\msNa\msNb\msNc\Ed; °vistārau \msCb}%
            \var{\vb dhyānavahnipradīpitaḥ\lem  \msCa\msNa; dhyāna\.m vahnipradīpitaḥ \msCb,
                       dhyānam agnipradīpitaḥ \msCc, dhyāna agnipradīpanaḥ \msNb,
                       dhyānavahniḥ pradīpitaḥ \msNc, dhyānav\textsubring{r}ddhir pradīpinaḥ \Ed}%

yogendhanasamijjvālatapodhūmasamākulaḥ \veg\dontdisplaylinenum
            \var{\vcd °ndhanasamijjvālatapodhūma°\lem  \msNb\msNc; °ndhanasamijjvālatapodhūpa° \msCa,
               °\uncl{ndha}satvamijjvālatapodhūma° \msCb, °ndhanasamijvālatapodhūma° \msCc,
               °ndhanaśami\uncl{ta}jvālatayodhūya° \msNa, °ndhanasamijjvālā tapodhūma° \Ed}%

pātranyāsa śivajñāna\.m sthālīpāka śivātmakaḥ\thinspace{\dandab} \dontdisplaylinenum
            \var{\va pātra°\lem  \mssCaCbCc\msNa\msNb\Ed; pātrā° \msNc}%

ājyāhutim avicchinna\.m lambakaśruvapātitaḥ \veg\dontdisplaylinenum
            \var{\vc °cchinna\.m\lem  \mssCaCbCc\msNa\msNb\Ed; °cchinna \msNc}%
            \var{\vd lambaka°\lem  \msCa\msCb\msNa\msNb\msNc; \uncl{la}mbaka° \msCc, tryambaka° \Ed\oo
                 °pātitaḥ\lem  \mssCaCbCc\msNa\msNb\msNc; °pātitam \Ed}%

dhāraṇādhvaryuvat k\textsubring{r}tvā prāṇāyāmaś ca \textsubring{r}tvijaḥ\thinspace{\dandab} \dontdisplaylinenum
            \var{\va °dhvaryuva°\lem  \msNb; °dhvaryava° \mssCaCbCc, °\uncl{dhva}ryava° \msNa,
                                                   dhva{\il}{\il} \msNc, dharmava° \Ed}%

tarkayuktaḥ savistāraḥ samādhir vayatāpanaḥ \veg\dontdisplaylinenum
            \var{\vc °yuktaḥ\lem  \msCa\msCb\msNb\msNc\Ed; °yukta \msCc, °yuktiḥ \msNa\oo
                 °vistāraḥ\lem  \msCa\msCb\msNa\msNb\msNc\Ed; °vistāro \msCc}%

brahmavidyāmayo yūpaḥ paśubandho manonmanaḥ\thinspace{\dandab} \dontdisplaylinenum
            \var{\vb °nmanaḥ\lem  \msCa\msNa\msNb\Ed; °tmanaḥ \msCb\msCc\msNc}%

śraddhā patnī viśālākṣi sa\.mkalpaḥ pada śāśvatam \veg\dontdisplaylinenum
            \var{\vc patnī\lem  \msCb\msCc\msNa\msNb\msNc\Ed; \uncl{patnī} \msCa\oo
                 viśālākṣi\lem  \mssCaCbCc\msNa\msNb; viśālākṣī \msNc\Ed}%
            \var{\vd °kalpaḥ\lem  \eme; °kalpa \mssCaCbCc\msNa\msNb\msNc\Ed\oo
                 pada śāśvatam\lem  \msCb\msCc\msNa\msNb\msNc\Ed; pa\uncl{da}{\lost}śvatam \msCa}%

pañcendriyajayotpannaḥ puroḍāśo 'm\textsubring{r}tāśanaḥ\thinspace{\dandab} \dontdisplaylinenum
            \var{\vb °ḍāśo\lem  \mssCaCbCc\msNb\msNc; °bhā \msNaacorr, °bhāse \msNapcorr, °bhāge \Ed\oo
                 m\textsubring{r}tā°\lem  \msCa\msCb\msNa\msNb\msNc\Ed; m\textsubring{r}gā° \msCc}%

brahmanādo mahāmantraḥ prāyaścittānilo jayaḥ \veg\dontdisplaylinenum
            \var{\vd °ttānilo\lem  \msCa\msCb\msNa\msNc\Ed; °ttanilo \msCc\msNb\oo
                 jayaḥ\lem  \mssCaCbCc\msNa\msNb\msNc; jalāḥ \Ed}%

somapāna parijñānam upākarma caturyamaḥ\thinspace{\dandab} \dontdisplaylinenum
            \var{\va pari°\lem  \msCa\msCb\msNa\msNb\msNc\Ed; para° \msCc}%

itihāsa jalasnāna\.m purāṇak\textsubring{r}tam ambaraḥ \veg\dontdisplaylinenum
            \var{\vc °snāna\.m\lem  \msCa\msCc\msNa\msNb\msNc\Ed; °snāna \msCb}%
            \var{\vd purāṇa°\lem  \mssCaCbCc\msNa\msNb\msNc; purāṇa\.m \Ed\oo
                 °k\textsubring{r}tam ambaraḥ\lem  \msCa\msCc\msNa\msNb\msNc\Ed; °k\textsubring{r}tambaram \msCb\ \unmetr}%

iḍāsuṣumnāsa\.mvedye snānam ācamana\.m sak\textsubring{r}t\thinspace{\dandab} \dontdisplaylinenum
            \var{\va °suṣumnā°\lem  \msCa\msCb\msNa\msNb\msNc\Ed; °suṣumna° \msCc\oo
                 °vedye\lem   \msCa\Ed; °vedya \msCb\msNb, °vedyeḥ \msCc, °vaidya \msNa, °bhedo \msNc}%
            \var{\vb sak\textsubring{r}t\lem  \msCa\msCb\msNa\msNb\msNc\Ed; viduḥ \msCc}%

sa\.mtoṣātithim ād\textsubring{r}tya dayābhūtadvijārcitaḥ \veg\dontdisplaylinenum
            \var{\vc °toṣātithim ād\textsubring{r}tya\lem  \mssCaCbCc\msNa\msNc\Ed; °toṣatithim āv\textsubring{r}tya \msNb}%
            \var{\vd °dvijā°\lem  \msCa\msCc\msNa\msNb\msNc\Ed; °dayā° \msCb}%

brahmakūrca guṇātīta havirgandha nirañjanaḥ\thinspace{\dandab} \dontdisplaylinenum
            \var{\vb °havir ga°\lem  \msCa\msCc\msNb\msNc\Ed; °havi\uncl{r ga}° \msCb, °haviga \msNa}%

brahmasūtra\.m trayas tattva\.m bodhanā muṇḍita\.m śiraḥ \veg\dontdisplaylinenum
            \var{\vc °sūtra\.m trayas\lem  \msCb\msNb\msNc\Ed; °sūtran trayastayas \msCa,
                                        °sūtra\.m traya \msCc, °sūtratraya\.m \msNa}%
            \var{\vd muṇḍita\.m\lem  \msCa\msCc\msNa\msNb\Ed; muṇḍita° \msCb\msNc\unmetr}%

niv\textsubring{r}ttyādi caturvedaś catuḥprakaraṇāsanaḥ\thinspace{\dandab} \dontdisplaylinenum
            \var{\va niv\textsubring{r}ttyā°\lem  \eme; niv\textsubring{r}tyā° \mssCaCbCc\msNa\msNb\msNc; nirv\textsubring{r}tyā° \Ed}%
            \var{\vb °prakaraṇāsanaḥ\lem  \msCa\msCb\msNa\msNb\msNc; prakaranāśanaḥ \msCc, prakaraśāsanaḥ \Ed}%

dakṣiṇām abhaya\.m bhūte dattvā yajña\.m yajet sadā \veg\dontdisplaylinenum
            \var{\vc °bhaya\.m bhūte\lem  \msCa\msCc\msNa\msNb\msNc\Ed; °bhakṣayam bhūtai \msCb}%
            \var{\vd yajña\.m yajet\lem  \mssCaCbCc\msNa\msNb\msNc; yajña dadat \Ed}%
            \paral{\textit{\vc {\normalfont cf.\ 22.14ab:} dakṣiṇābhaya bhūtebhyaḥ paśubandhaḥ svaya\.mk\textsubring{r}taḥ}}

vinārtha\.m yajñasamprāptiḥ kathitā te varānane\thinspace{\dandab} \dontdisplaylinenum
            \var{\va vinārtha\.m\lem  \msCa\msCb\msNa\msNb\msNc\Ed; vinārtha \msCc}%
            \var{\vb kathitā te\lem  \msCa\msCb\msNa\msNb\msNc; kathi\uncl{to} smi \msCc, kathitas te \Ed\oo
                 varānane\lem  \msCa\msCb\msNa\msNb\msNc\Ed; va\uncl{rā}nane \msCc}%

āsahasrasya yajñānā\.m phala\.m prāpnoti nityaśaḥ \veg\dontdisplaylinenum
            \var{\vd prāpnoti\lem  \msCb\msCc\msNa\msNb\msNc\Ed; prā{\lost}ti \msCa\oo
                 nityaśaḥ\lem  \mssCaCbCc\msNa\msNc\Ed; mānavaḥ \msNb}%

āśramaḥ prathamas tubhya\.m kathito 'sti varānane\thinspace{\dandab} \dontdisplaylinenum
            \var{\va āśramaḥ\lem  \msCa\msNa\msNb\msNc\Ed; āśrama \msCb\msCc\oo
                 °s tubhya\.m\lem  \msCa\msCb\msNa\msNb\msNc; °syeṣa \msCc, °syaiva\.m \Ed}%
            \var{\vb 'sti\lem  \msCa\msCb\msNa\msNc; smi \msCc\msNb\Ed}%

sadāśivena saddharma\.m daivatair api pūjitam \veg\dontdisplaylinenum
            \var{\vc °dharma\.m\lem  \msCa\msCc\msNa\msNb\msNc; °dha\uncl{rma\.m} \msCb, °dharme \Ed}%
            \var{\vd daiva°\lem  \mssCaCbCc\msNa\msNc; deva° \msNb\Ed\oo
                 pūjitam\lem  \msCa\msCc\msNa\msNb\msNc\Ed; pūpūjitam \msCb}%


\alalfejezet{brahmacaryam}
brahmacarya\.m nibodheda\.m ś\textsubring{r}ṇuṣvāvahitā śubhe\thinspace{\dandab} \dontdisplaylinenum
            \var{\va °carya\.m\lem  \mssCaCbCc\msNb\msNc\Ed; °carya \msNa}%
            \var{\vb °vahitā śubhe\lem  \msCa\msCb\msNa\msNc\Ed; °vahito bhava \msCc, °vahito śubhe \msNb}%

dvitīyam āśrama\.m devi sarvapāpavināśanam \veg\dontdisplaylinenum
            \var{\vd °vināśanam\lem  \mssCaCbCc\msNa\msNc\Ed; °pranāśanam \msNb}%
            \paral{\textit{\vcd {\normalfont  cf.\ MBh 12.184.10A: } gārhasthya\.m khalu dvitīyam āśrama\.m vadanti}}

vrata\.m brahmapara\.m dhyāna\.m sāvitrī prak\textsubring{r}tau layaḥ\thinspace{\dandab} \dontdisplaylinenum
            \var{\va °para\.m dhyāna\.m\lem  \mssCaCbCc\msNa\msNb\msNc; °parijñāna\.m \Ed}%
            \var{\vb °k\textsubring{r}tau layaḥ\lem  \eme; °k\textsubring{r}tir layam \msCa\msNa\msNc\Ed,
                         °k\textsubring{r}tālayam \msCb, °k\textsubring{r}tīlayam \msCc, °k\textsubring{r}tilaḥ \msNb}%
            \paral{\textit{\vab {\normalfont cf.\ 16.8cd: ! }}}

brahmasūtrākṣara\.m sūkṣma\.m triguṇālaya mekhalam \veg\dontdisplaylinenum
            \var{\vd °laya\lem  \msCb\msCc\msNa\msNb\msNc\Ed; °la{\lost} \msCa\oo
                 mekhalam\lem  \mssCaCbCc\msNa\msNb\msNc; yat phalam \Ed}%

dama daṇḍa dayā pātra\.m bhikṣā sa\.msāramocanam\thinspace{\dandab} \dontdisplaylinenum
            \var{\va daṇḍa dayā\lem  \mssCaCbCc\msNb\msNc; daṇḍādayā \msNa, daṇḍādayo \Ed\oo
                 pātra\.m\lem  \mssCaCbCc\msNa\msNc\Ed; pātra \msNb}%

tryāyuṣa\.m dvyakṣarātīta\.m jñānabhaṣma-alaṅk\textsubring{r}tam \veg\dontdisplaylinenum
            \var{\vc °yuṣa\.m\lem  \mssCaCbCc\msNb\msNc\Ed; °yuṣa \msNa}%
            \var{\vd bhasma\lem  \mssCaCbCc\msNa\msNb\msNc; bhasmam \Ed}%

snānavrata\.m sadāsatya\.m śīlaśaucasamanvitam\thinspace{\dandab} \dontdisplaylinenum
            \var{\va °vrata\.m\lem  \msCa\msCc\msNa\msNb; °vrata \msCb\msNc\Ed}%

agnihotra trayas tattva\.m japa brahmabilasvaraḥ \veg\dontdisplaylinenum
            \var{\vc °hotra trayas tattva\.m\lem  \msNa\msNc\Ed; °hotran trayas tatva\.m \msCa,
                        °hotra\uncl{ta}yas tatva\.m \msCb, °hotratraya\.m tatvā \msCc, °hotra\.m traya\.ms tatva\.m \msNb}%
            \var{\vd °bilasvaraḥ\lem  \corr; °bilaśvaraḥ \mssCaCbCc\msNa\msNb, °bileśvara \msNc\Ed}%

dvitīya āśramo devi yathāha bhagavān śivaḥ\thinspace{\dandab} \dontdisplaylinenum
            \var{\va dvitīya āśramo\lem  \msCa\msCb\msNa\msNb\msNc; dvitīyam āśramo \msCc, dvitīyam āśrama\.m \Ed}%
            \var{\vb yathāha\lem  \msCa\msCb\msNa\msNc; yathāha\.m \msCc\msNb, yad āha \Ed}%

mayāpi kathita\.m tubhya\.m janmam\textsubring{r}tyuvināśanam \veg\dontdisplaylinenum
            \var{\vc mayāpi kathita\.m tu°\lem  \eme; mamāpi kathita\.m tu° \mssCaCbCc\msNa\msNb,
                                mamāpi kathitas tu° \msNc, mayāpi kathito tu° \Ed}%
            \var{\vd °m\textsubring{r}tyu°\lem  \msCb\msCc\msNa\msNb\msNc\Ed; °m\textsubring{r}{\lost}° \msCa\oo
                 °nāśana\.m\lem  \mssCaCbCc\msNa\msNb\Ed; °nāśanaḥ \msNc}%


\alalfejezet{vānaprasthaḥ}
vānaprasthavidhi\.m vakṣye ś\textsubring{r}ṇuṣvāyatalocane\thinspace{\dandab} \dontdisplaylinenum
            \var{\va °vidhi\.m\lem  \msCa\msCc\msNa\msNb\msNc\Ed; °vidhi \msCb}%

yathāśruta\.m yathātathyam \textsubring{r}ṣidaivatapūjitam \veg\dontdisplaylinenum
            \var{\vd °daivata°\lem  \msCa\msCb\msNa\msNb\msNc\Ed; °devata° \msCc}%

vairāgyavanam āśritya niyamāśramam āharet\thinspace{\dandab} \dontdisplaylinenum
            \var{\va vairāgya°\lem  \mssCaCbCc\msNa\msNb\msNc; vairāgyā \Ed}%
            \var{\vb niyamā°\lem  \mssCaCbCc\msNapcorr\msNb\msNc\Ed; mā° \msNaacorr\oo
                 °śramam ā°\lem  \msCb\msCc\msNa\msNb\msNc\Ed; °śramano haret \msCa}%

śīlaśailad\textsubring{r}ḍhadvāre prākāre vijitendriyaḥ \veg\dontdisplaylinenum
            \var{\vc °d\textsubring{r}ḍha°\lem  \mssCaCbCc\msNa\msNb\msNc; °d\textsubring{r}ṣa° \Ed}%
            \var{\vd °kāre\lem  \msCa\msCb\msNa\msNb\msNc\Ed; °kāra° \msCc}%

adhibhūtaḥ sm\textsubring{r}to mātā adhyātmaś ca pitā tathā\thinspace{\dandab} \dontdisplaylinenum
            \var{\va sm\textsubring{r}to\lem  \msCa\msCc\msNa\msNb\msNc; {\lost}{\lost} \msCb, sm\textsubring{r}tau \Ed}%
            \paral{\textit{\vab {\normalfont cf.\ 22.10ab:} adhyātmanagarasphītaḥ adhibhūtajanākulaḥ}}

adhidaivika-m-ācāryo vyavasāyāś ca bhrātaraḥ \veg\dontdisplaylinenum
            \var{\vc adhidaivika°\lem  \eme\ \Goodall; 
                \uncl{a}{\lost}\uncl{bhau}{\lost}ka° \msCa, adhibhautika° \msCb\msCc\msNa\msNc\Ed, adhibhauktika° \msNb}%
            \var{\vd vyavasāyāś ca\lem  \mssCaCbCc\msNa\msNb\msNc; vyavasāyaś ca \Ed}%

śrutiḥ sm\textsubring{r}tiḥ sm\textsubring{r}tā bhāryā prajñā putraḥ kṣamānujaḥ\thinspace{\dandab} \dontdisplaylinenum
            \var{\va sm\textsubring{r}tā\lem  \msCa\msCc\msNa\msNb\msNc\Ed; sm\textsubring{r}to \msCb}%

maitrī bandhur jaṭā cāpa\.m karuṇā supavitrakam \veg\dontdisplaylinenum
            \var{\vc bandhur ja°\lem  \msCa\msCb\msNa\msNc\Ed; bandhu ja° \msCc\msNb}%

muditā mauna catvāraḥ sarvakāryam upekṣakā\thinspace{\dandab} \dontdisplaylinenum
            \var{\va mauna catvāraḥ\lem  \msCa\msNa\msNb\msNc\Ed; maunaś catvāraḥ \msCb, mauna catvāra \msCc}%
            \var{\vb °kāryam u°\lem  \mssCaCbCc\msNb\msNc\Ed; °kāryām u° \msNa\oo 
                 °pekṣakā\lem  \mssCaCbCc\msNa\msNb\msNc; °pekṣayā \Ed}%

yamavalkalasa\.mvītas tapaḥk\textsubring{r}ṣṇājinādharaḥ \veg\dontdisplaylinenum
            \var{\vc °sa\.mvīta°\lem  \mssCaCbCc\msNa\msNb\msNc; °sānvīta° \Ed}%
            \var{\vd °k\textsubring{r}ṣṇā°\lem  \msCa\msCb\msNa\msNb\msNc\Ed; °k\textsubring{r}ṣṇā\.m \msCc\oo
                 °jinādharaḥ\lem  \msNc; °jinadharaḥ \mssCaCbCc\msNa\msNb\ \unmetr, °jina\.m puraḥ \Ed}%
 
uttarāsaṅgam āsīno yogapaṭṭad\textsubring{r}ḍhavrataḥ\thinspace{\dandab} \dontdisplaylinenum
            \var{\vb °d\textsubring{r}ḍha°\lem  \mssCaCbCc\msNa\msNc\Ed; °d\textsubring{r}ṣṭa° \msNb\oo 
                 °vrataḥ\lem  \msCb\msCc\msNa\msNb\msNc\Ed; {\lost}{\lost} \msCa}%

vedaghoṣeṇa ghoṣeṇa prāṇāyāmo 'gnihāvanam \veg\dontdisplaylinenum
            \var{\vc veda°\lem  \msCb\msCc\msNa\msNb\msNc\Ed; {\lost}da° \msCa\oo
                 °ṇa ghoṣeṇa\lem  \msCa\msCb\msNa\msNb\msNc\Ed; °ṇa ghoṣīṇa \msCc}%
            \var{\vd °hāvanam\lem  \msCa\msNa\msNb\msNc\Ed; °hāvana \msCc; °\uncl{hāvanam} \msCb}%

jitaprāṇam\textsubring{r}gākūlo dh\textsubring{r}ti yajñaḥ kriyā japaḥ\thinspace{\dandab} \dontdisplaylinenum
            \var{\vb °japaḥ\lem  \msCa\msCb\msNa\msNb\msNc\Ed; °jiṇaḥ \msCc}%

arthasa\.mgraha śāstreṣu sakhā damadayādayaḥ \veg\dontdisplaylinenum
            \var{\vd sakhā\lem  \mssCaCbCc\msNa\msNc\Ed; sakho \msNb\oo
                 damada°\lem  \msCapcorr\msCb\msNa\msNb\msNc\Ed; dayada° \msCc, dama° \msCaacorr}%

śivayajña\.m prayuñjīta sādhanāṣṭakapūjanam\thinspace{\dandab} \dontdisplaylinenum
            \var{\va °yajña\.m\lem  \msCa\msCb\msNa\msNb\Ed; °yajña \msCc\msNc}%
            \var{\vb °pūjanam\lem  \msCa\msCb\msNa\msNb\msNc\Ed; °pūjika\.m \msCc}%
            \paral{\textit{\vb {\normalfont cf.\ Dharmaputrikā 2.1:} 
                aṣṭabhiḥ sādhanair ebhiś citta\.m kāyañ ca yatnataḥ{\thinspace\danda}
                śodhayitvā tato yogī yogābhyāsa\.m samācaret{\thinspace\ketdanda}}}

pañcabrahmajalaiḥ pūtaḥ satyatīrthaśivahrade \veg\dontdisplaylinenum
            \var{\vc °brahmajalaiḥ pūtaḥ\lem  \mssCaCbCc\msNa\msNc\Ed; bra{\lost}{\lost}{\lost}{\lost}{\lost} \msNb}%
            \var{\vd °tīrtha\lem  \mssCaCbCc\msNa\msNb\msNc; °tīrtha\.m \Ed}%

snānam ācamana\.m k\textsubring{r}tvā sa\.mdhyātrayam upāśrayet\thinspace{\dandab} \dontdisplaylinenum
            \var{\va °camana\.m\lem  \msCa\msCc\msNa\msNb\msNc\Ed; °cana\.m \msCb}%
            \paral{\textit{\vb {\normalfont See 11.59cd:} śivasya h\textsubring{r}daya\.m sa\.mdhyā tasmāt sa\.mdhyām upāsayet}}

akṣamālā purāṇārtha\.m japaśānta\.m divāniśam \veg\dontdisplaylinenum
            \var{\vc akṣamālā\lem  \msCb\msCc\msNa\msNb\msNc\Ed; \uncl{akṣa}{\lost}lā \msCa\oo
                 purāṇārtha\.m\lem  \mssCaCbCc\msNa\Ed; purāṇāñ ca \msNb, purāṇā\uncl{rthā} \msNc}%
            \var{\vd °śānta\.m\lem  \msCapcorr\msCb\msCc\msNb\msNc\Ed; °śanti \msCaacorr\msNa}%

jñānasalilasampūrṇamitihāsakamaṇḍaluḥ\thinspace{\dandab} \dontdisplaylinenum
            \var{\va °salila°\lem  \mssCaCbCc\msNa\msNb\msNc; °salīla° \Ed}%
            \var{\vb °kamaṇḍaluḥ\lem  \mssCaCbCc\msNa\msNb\msNc; °kamaṇḍalu \Ed}%

pañcakarmakriyotkrānti japa pañcavidhaḥ sukham \veg\dontdisplaylinenum
            \var{\vc °tkrāntija°\lem  \msCa\msCb\msNb; °krāntija° \msCc, °tkrāntir ja° \msNa, 
                                        °tkāntija° \msNc, 'krānti ja° \Ed}%

sādhana\.m śivasa\.mkalpo yogasiddhiphalapradaḥ\thinspace{\dandab} \dontdisplaylinenum
            \var{\vd °daḥ\lem  \mssCaCbCc\msNa\msNb\msNc; °dam \Ed}%

sa\.mtoṣaphalam āhāraḥ kāmakrodhaparājitaḥ \veg\dontdisplaylinenum

āśāpāśajayābhyāso dhyānayogaratipriyaḥ\thinspace{\dandab} \dontdisplaylinenum
            \var{\va °bhyāso\lem  \mssCaCbCc\msNa\msNb\msNc; °bhyāsa \Ed}%
            \var{\vb °rati°\lem  \msCc\msNa\msNb\msNc; {\lost}{\lost} \msCa, °riti° \msCb, °ratiḥ \Ed}%

atithibhyo 'bhaya\.m dattvā vānaprasthaś cared vratam \danda\dontdisplaylinenum
            \var{\va atithibhyo 'bhaya\.m\lem  \mssCaCbCc\msNa\msNb\msNc; ārtibhyaś cābhaya\.m \Ed\oo
                 dattvā\lem  \msCa\msCb\msNa\msNb\msNc\Ed; dārā \msCc}%
            \var{\vb °prasthaś ca°\lem  \msCa\msCb\msNa\msNc\Ed; °prastha ca° \msCc\msNb}%

vānaprastham aya\.m dharma\.m yat pūrvam avadhāritam \veg\dontdisplaylinenum
            \var{\vd yat pūrvam avadhāritam\lem  \msCc\Ed; 
                      gadita\.m pūrvadhārita\.m \msCa\msCb,
                      gadita pūrvadhārita\.m \msNb,
                      gadita\.m yat pūrvadhārita\.m \msNaacorr\ \unmetr,
                      gadita\.m yat pūrvam avadhārita\.m \msNapcorr\ \unmetr,
                      gadita\.m yat pūrvamedhārita\.m \msNc\ \unmetr}%

\ujvers\nemsloka 
! sa\.msāroddharaṇam anityaharaṇam ajñānanirmūlanam 
\dontdisplaylinenum
            \var{\va °haraṇam anityaharaṇam ajñā°\lem  \msCa\msCb\msNaacorr\msNb\msNc; 
                                °haraṇa\.m anityaharaṇan tajñā° \msNapcorr, 
                                °haraṇa\.mm anityaharaṇam ajñā° \msCc\Ed}%

\nemslokab 
! prajñāv\textsubring{r}ddhikaram amoghakaraṇa\.m kleśārṇavottāraṇam \danda\dontdisplaylinenum
            \var{\vb \om\ \msNb\oo
                 °karam amogha°\lem  \mssCaCbCc\msNa\ \unmetr; \om\ \msNb, °kam amogha° \msNc,
                                                                         °kara\.m prabodha° \Ed\oo
                 kleśārṇavo°\lem  \mssCaCbCc\msNc; kleśāṇṇavo° \msNa, \om\ \msNb, śokārṇavo° \Ed}%

\nemslokac 
! janmavyādhiharam akarmadahana\.m sevet sa dharmottamam
\dontdisplaylinenum
            \var{\vc sevet sa\lem  \msCa\msCb\msNa\msNc\Ed; seve sa \msCc, sevet ta \msNb}%

\nemslokad 
? śraddhāpūrvakam eva yaḥ saniyama\.m sākṣāc ca jīvan śivaḥ \veg\dontdisplaylinenum
            \var{\vd \om\ \mssCaCbCc\msNa\msNb\msNc}%


\alalfejezet{parivrājakaḥ}
\vers

parivrājakadharmo 'ya\.m kīrtayiṣyāmi tac ch\textsubring{r}ṇu\thinspace{\dandab} \dontdisplaylinenum
            \var{\vb kīrtayiṣyāmi\lem  \msCb\msCc\msNa\msNb\msNc\Ed; kīrtayi{\lost}mi \msCa}%

sukhaduḥkha\.m sama\.m k\textsubring{r}tvā lobhamohavivarjitaḥ \veg\dontdisplaylinenum
            \var{\vc °duḥkha\.m\lem  \msCb; °duḥkha \msCa\msCc\msNa\msNb\msNc\Ed}%
            \var{\vd lobhamoha°\lem  \msCb; lābhālobha° \msCa\msNa\msNb\msNc, lābhalobha° \msCc, lābhālābha° \Ed\oo
                 °varjitaḥ\lem  \mssCaCbCc\msNa\msNc\Ed; °varjitāḥ \msNb}%
            \paral{\textit{\vd {\normalfont  cf.\ 4.71: }  
                     kāmaḥ krodhaś ca lobhaś ca mohaś caiva caturvidhaḥ{\thinspace\danda}
                     catuḥśatrur nihantavyaḥ sarvathā vītakalmaṣaḥ{\thinspace\ketdanda}}}

varjayen madhu mā\.msāni paradārā\.mś ca varjayet\thinspace{\dandab} \dontdisplaylinenum
            \var{\va varjayen\lem  \msCa\msNb; varjayet \msCb\msCc\msNa\msNc\Ed}%
            \paral{\textit{\va = {\normalfont Kūrmapurāṇa 2.27.12a etc.}}}

varjayec ciravāsa\.m ca paravāsa\.m ca varjayet \veg\dontdisplaylinenum
            \var{\vc °vāsa\.m\lem  \mssCaCbCc\msNa\msNb\msNc; °vāsaś \Ed}%
            \var{\vd °vāsa\.m\lem  \mssCaCbCc\msNa\msNb\msNc; °vāsaś \Ed}%

varjayet s\textsubring{r}ṣṭabhojyāni bhikṣām ekā\.m ca varjayet\thinspace{\dandab} \dontdisplaylinenum
            \var{\vab \om\ \msCb}%
            \var{\va varjayet s\textsubring{r}ṣṭa°\lem  \msCc(?)\msNa\msNc; varjayet m\textsubring{r}ṣṭa° \msCa, \om\ \msCb,
                varjjan m\textsubring{r}ṣṭa° \msNb, varjayen m\textsubring{r}ṣṭa° \Ed\oo
                 °bhojyāni\lem  \mssCaCbCc\msNa\msNb\Ed; °bhojāli(?) \msNc}%
             \var{\vb °kṣām ekā\.m\lem  \msCa\msNb; \om\ \msCb, °kṣām eka\.m \msCc\msNa,
                                                °kṣam ekañ \msNc, °kṣām ekaś \Ed}%

varjayet sa\.mgraha\.m nityam abhimāna\.m ca varjayet \veg\dontdisplaylinenum

susūkṣma\.m manasā dhyātvā śucau pāda\.m vinikṣipet\thinspace{\dandab} \dontdisplaylinenum
            \var{\vb pāda\.m\lem  \msCb\msCc\msNa\msNc; pā\uncl{da\.m} \msCa, pāda \msNb\Ed\oo
                 vinikṣi°\lem  \msCb\msCc\msNa\msNb\Ed; {\lost}nikṣi° \msCa, vinikṣa° \msNc}%

na kupyeta anālābhe lābhe vāpi na harṣayet \veg\dontdisplaylinenum
            \var{\vc kupyeta\lem  \msCa\msCb\msNa\msNb\msNc\Ed; kupeta \msCc\oo
                 anālābhe\lem  \msNa; manolābhe \msCa\msCb\msNb\msNc, manolābho \msCc, manālābhe \Ed}%

arthat\textsubring{r}ṣṇāsv anudvigno roṣe vāpi sudāruṇe\thinspace{\dandab} \dontdisplaylinenum
            \var{\va artha°\lem  \msCb\msCc\msNc; arthā° \msCa\msNa\msNb, atha \Ed\oo
                 °nudvigno\lem  \msCa\msCb\msNa\msNb\msNc\Ed; °nudigno \msCc}%

stutinindā sama\.m k\textsubring{r}tvā priya\.m vāpriyam eva vā \veg\dontdisplaylinenum

niyamās tu parīdhāna\.m sa\.myamāv\textsubring{r}tamekhalaḥ\thinspace{\dandab} \dontdisplaylinenum
            \var{\va °dhāna\.m\lem  \msCa\msCb\msNa\msNb\Ed; °\uncl{dhāna\.m} \msNc, °dhānā \msCc}%
            \var{\vb °v\textsubring{r}ta°\lem  \mssCaCbCc\msNa\msNc; °m\textsubring{r}ta° \msNb, °n\textsubring{r}ta° \Ed\oo
                 °mekhalaḥ\lem  \msCa\msCb\msNa\msNc\Ed; °mekhalāḥ \msCc, °mekhalā \msNb}%

nirālamba\.m manaḥ k\textsubring{r}tvā buddhi\.m k\textsubring{r}tvā nirañjanām \veg\dontdisplaylinenum
            \var{\vc °ba\.m manaḥ k\textsubring{r}tvā\lem  \msNc; °bam asatk\textsubring{r}tvā \msCa\msNa,
                                °bam asa\.mk\textsubring{r}tvā \msCb, °bam ana\.mk\textsubring{r}tvā \msCc,
                                °ba manas k\textsubring{r}tvā \msNb, °bam anaṅk\textsubring{r}tvā \Ed}%
            \var{\vd buddhi\.m\lem  \msCa\msCc\msNa\msNb\msNc; buddhi \msCb\Ed\oo
                 nirañjanām\lem  \eme; nirañjanam \mssCaCbCc\msNb\msNc\Ed, nirañjanaḥ \msNa}%

ātmāna\.m p\textsubring{r}thivī\.m k\textsubring{r}tvā kha\.m ca k\textsubring{r}tvā manonmanam\thinspace{\dandab} \dontdisplaylinenum
            \var{\vab k\textsubring{r}tvā kha\.m ca\lem  \msCb\msCc\msNa\msNb\msNc\Ed; k\textsubring{r}\uncl{tvā}{\lost}ñca \msCa}%
            \var{\vb manonmanam\lem  \mssCaCbCc\msNa\msNb; manonmanaḥ \msNc, manonmanaiḥ \Ed}%

tridaṇḍa\.m triguṇa\.m k\textsubring{r}tvā pātra\.m k\textsubring{r}tvākṣaro 'vyayaḥ \veg\dontdisplaylinenum
            \var{\vd °kṣaro\lem  \mssCaCbCc\msNa\msNc\Ed; °karo \msNb\oo
                 vyayaḥ\lem  \msCa\msCb\msNa\msNb; vyaya\.m \msCc, vyaya \msNc, dvayam \Ed}%

nyased dharmam adharma\.m ca īrṣyādveṣa\.m parityajet\thinspace{\dandab} \dontdisplaylinenum
            \var{\va °dharma\.m ca\lem  \mssCaCbCc\msNb\msNc\Ed; °dharma\.m vā \msNa}%
            \var{\vb īrṣyā°\lem  \msNa\msNc\Ed; īrṣā° \mssCaCbCc\msNb\oo
                 °dveṣa\.m\lem  \msCa\msCb\msNa\msNb\msNc\Ed; °dveṣa \msCc}%

nirdvandvo nityasatyastho nirmamo niraha\.mk\textsubring{r}taḥ \veg\dontdisplaylinenum
            \var{\vc nirdvandvo\lem  \msCa\msCb\msNa\msNb\msNc\Ed; niva\.mdvo \msCc\oo
                 °satya°\lem  \msCa\msCb\msNa\msNb\msNc\Ed; °sa\.mtya° \msCc}%
            \var{\vd nirmamo\lem  \msNc\Ed; nirmā\.mso \mssCaCbCc\msNa, nirma\.mso \msNb\oo
                 °k\textsubring{r}taḥ\lem  \mssCaCbCc\msNb\msNc; °k\textsubring{r}ta\.m \msNa, °k\textsubring{r}tiḥ \Ed}%
                     \paral{\textit{\vcd {\normalfont cf. BhG 2.45cd: }nirdvandvo nityasatvastho niryogakṣema ātmavān}}

divasasyāṣṭame bhāge bhikṣā\.m saptag\textsubring{r}ha\.m caret\thinspace{\dandab} \dontdisplaylinenum
            \var{\va divasasyā°\lem  \msCa\msCc\msNa\msNb\msNc\Ed; divasatyā° \msCb}%
            \var{\vb bhikṣā\.m\lem  \mssCaCbCc\msNa\msNc\Ed; bhikṣā \msNb}%

na cāsīta na tiṣṭheta na ca dehīti vā vadet \veg\dontdisplaylinenum

yathālābhena varteta aṣṭau piṇḍān dine dine\thinspace{\dandab} \dontdisplaylinenum
            \var{\va yathālābhena\lem  \msCb\msCc\msNa\msNb\msNc\Ed; yathālā{\lost}{\lost} \msCa}%
            \var{\vb aṣṭau\lem  \mssCaCbCc\msNa\msNb\msNc; aṣṭa \Ed}%

vastrabhojanaśayyāsu na prasajyeta vistaram \veg\dontdisplaylinenum
            \var{\vc °śayyāsu\lem  \mssCaCbCc\msNa\msNc; °śayyāñca \msNb, °śaiyyāsu \Ed}%
            \var{\vd °sajyeta\lem  \msCa\msCc\msNa\msNb; °yujye \msCb, °saheta \msNc, °sahyeta \Ed\oo
                 vistaram\lem  \mssCaCbCc\msNa\msNb\msNc; vistaraḥ \Ed}%

nābhinandeta maraṇa\.m nābhinandeta jīvitam\thinspace{\dandab} \dontdisplaylinenum
            \paral{\textit{\vab {\normalfont = MBh 12.237.15ab, Manu 6.45ab, Nāradaparivrājakopaniṣad 3.61cd. }}}

indriyāṇi vaśa\.mk\textsubring{r}tvā kāma\.m hatvā yatavrataḥ \veg\dontdisplaylinenum
            \var{\vc vaśa\.mk\textsubring{r}°\lem  \msCa\msCb\msNa\msNb\msNc\Ed; vasa\.mtk\textsubring{r}° \msCc}%
            \var{\vd hatvā yatavrataḥ\lem  \mssCaCbCc\msNa\msNc\Ed; k\textsubring{r}tvā yataḥ vrataḥ \msNb}%

atīta\.m ca bhaviṣya\.m ca na bhikṣuś cintayet sadā\thinspace{\dandab} \dontdisplaylinenum
            \var{\vb bhikṣuś ci°\lem  \mssCaCbCc\msNb\msNc; bhikṣu\.mś ci° \msNa, bhikṣu ci° \Ed\oo 
                 sadā\lem  \msCa\msCc\msNa\msNb\msNc\Ed; \om\ \msCb}%

! krodhamānamadadarpān parivrāḍ varjayet sadā \veg\dontdisplaylinenum
            \var{\vcd °darpān pa°\lem  \msCa\msCc\msNa\msNb\msNc\Ed; °darpāt pa° \msCb}%

virāga\.m tu dhanuḥ k\textsubring{r}tvā prāṇāyāmaguṇair yutam\thinspace{\dandab} \dontdisplaylinenum
            \var{\va dhanuḥ\lem  \mssCaCbCc\msNa\msNb\msNc; dhanuṣ \Ed}%
            \var{\vb prāṇāyāmagu°\lem  \msCb\msCc\msNa\msNb\msNc\Ed; prāṇāyāmaṅgu° \msCa\oo
                 yutam\lem  \mssCaCbCc\msNb\msNc; yutaḥ \msNa, v\textsubring{r}ta\.m \Ed}%

dhāraṇāśaratīkṣṇena m\textsubring{r}ga\.m hatvā manendriyam \veg\dontdisplaylinenum
            \var{\va °tīkṣṇena\lem  \msNb\Ed; °tīkṣṇeṇa \mssCaCbCc\msNc; °tīkṣeṇa \msNa}%

maitrīkhaḍgasutīkṣṇena sa\.msārāri\.m nik\textsubring{r}ntayet\thinspace{\dandab} \dontdisplaylinenum
            \var{\va sutīkṣṇena\lem  \msNb\msCa\msNc\Ed; sutīkṣṇeṇa \msCb\msCc\msNapcorr, ṇa \msNaacorr}%
            \var{\vb °sārāri\.m\lem  \msCa\msCb\msNa\msNb\Ed; °sārāri \msCc\msNc}%

karuṇāvartacakreṇa krodhamattagaja\.m jayet \veg\dontdisplaylinenum

muditāvarmabaddhāṅgas tūṇa\.m pūrṇam upekṣayā\thinspace{\dandab} \dontdisplaylinenum
            \var{\vb tūṇa\.m pūrṇam u°\lem  \eme\ \Goodall; tūṇṇāpūrṇṇam u° \msCa,
                      tūṇāpūrṇṇam u° \msCb, tū\uncl{na}pūrṇṇam u° \msCc,
                      tūṇṇāpūṇṇām u° \msNa, tūrṇṇāpūrṇṇam u° \msNb\msNc, tūṇīpūrṇam u° \Ed}% 
            \paral{\textit{\vo {\normalfont Cf.\ 4.72: } caturāyatana\.m vipra kathayiṣyāmi tac ch\textsubring{r}ṇu{\thinspace\danda}
                                 karuṇāmuditopekṣāmaitrī cāyatana\.m sm\textsubring{r}tam{\thinspace\ketdanda}}}

anakṣara\.m para\.m brahma cintayet satata\.m dvija \veg\dontdisplaylinenum
            \var{\vc anakṣara\.m\lem  \msCb; anākṣara\.m \msCa\msNa, anākṣara° \msCc\msNc\Ed, anakṣara° \msNb\oo
                 para\.m\lem  \msCa\msCc\msNa\msNb\Ed; para \msCb\msNc}%

brahmaṇo h\textsubring{r}daya\.m viṣṇur viṣṇoś ca h\textsubring{r}daya\.m śivaḥ\thinspace{\dandab} \dontdisplaylinenum
            \var{\va h\textsubring{r}daya\.m\lem  \msCb\msCc\msNa\msNb\Ed; {\lost}daya\.m \msCa, h\textsubring{r}daye \msNc}%
            \var{\vab viṣṇur vi°\lem  \msCa\msNa\Ed; viṣṇum vi° \msCb, viṣṇu vi° \msCc\msNb\msNc}%
            \var{\vb śivaḥ\lem  \Ed; śiva\.m \mssCaCbCc\msNa\msNb\msNc}%

śivasya h\textsubring{r}daya\.m sa\.mdhyā tasmāt sa\.mdhyām upāsayet \veg\dontdisplaylinenum
            \var{\vd °sayet\lem  \msCa\msCc\msNb; °śayet \msCb\msNa, °śrayet \msNc\Ed}%
            \paral{\textit{\vo \kb\ {\normalfont  Saubhāgyabhāskara of Bhāskararāya ad Lalitāsahasranāmastotra 302: }
                brahmaṇo h\textsubring{r}daya\.m viṣṇur viṣṇor api śivaḥ sm\textsubring{r}taḥ{\thinspace\danda}
                śivasya h\textsubring{r}daya\.m sandhyā tenopāsyā dvijātibhiḥ{\thinspace\ketdanda}
                iti kaśyapādivacanaiḥ kaurmapādmaskāndādinikhilapurāṇeṣu ca tatra 
                tatra devīkālikābrahmāṇḍamārkaṇḍeyādipurāṇeṣu bahuśaḥ 
                śaktirahasyadevībhāgavatat\textsubring{r}tīyaskandhādiṣu{\thinspace\danda} }}

\ujvers\nemsloka 
sa\.msārārṇavatāraṇa\.m śubhagatiḥ sa brahma sa\.mdhyākṣara\.m
\dontdisplaylinenum
           \var{\va °gatiḥ\lem  \msCc\Ed; °gati \msCa\msCb\msNa\msNb\ \unmetr, °gati\.m \msNc\oo 
                °kṣara\.m\lem  \msCa\msCc\msNa\msNb\msNc\Ed; °kṣara \msCb}%

\nemslokab 
dhyāyen nityam atandrito hy anupama\.m vyaktātmavedya\.m śivam \danda\dontdisplaylinenum
            \var{\vb °tandrito\lem  \msCa\msNa\msNc\Ed; °nandrito \msCb, °tandriya \msCc, °tandriya\.m \msNb\oo
                 °vedya\.m\lem  \mssCaCbCc\msNa\msNc\Ed; °vedya \msNb\ \unmetr}%

\nemslokac 
rūpair varṇaguṇādibhiś ca vihita\.m durlakṣyalakṣyottama\.m
\dontdisplaylinenum
            \var{\vc rūpair va°\lem  \msCa\msNa\msNc\Ed; rūpai va° \msCb\msCc\msNb\oo
                 vihita\.m\lem  \mssCaCbCc\msNaacorr(?)\msNb\msNc; rahita\.m \msNapcorr(?)\Ed\oo
                 durlakṣyalakṣyottamam\lem  \msCa\msNb; dulakṣyalakṣyottamam \msNa;
                                              durlakṣyalakṣottamam \msCb\msCc\msNc\Ed}%

\nemslokad 
yatnoddh\textsubring{r}tya samāśrayet suraguru\.m sarvārtihartā haram \veg\dontdisplaylinenum
            \var{\vd yatnoddh\textsubring{r}tya\lem  \mssCaCbCc\msNa\msNb\msNc; yatnād dh\textsubring{r}tya \Ed\oo
                 samāśraye°\lem  \mssCaCbCc\msNa\msNc\Ed; maṇāśraye° \msNb\oo
                 sarvārtihartā haram\lem  \mssCaCbCc\msNb; sarvārttiha\uncl{rttā} hara\.m \msNa,
                                         sarvāttiharttā hara\.m \msNc,
                                         sarvārtihan śaṅkaram \Ed}%

\vers


\jump
\begin{center}
\ketdanda iti v\textsubring{r}ṣasārasa\.mgrahe caturāśramadharmavidhāno nāmādhyāya ekādaśamaḥ\ketdanda
\end{center}
\dontdisplaylinenum\vers 
    \var{{\normalfont Colophon: } nāmādhyāya ekādaśamaḥ\lem  \mssCaCbCc\msNa\msNb; nāmādhyāya ekādaśa \msNc, 
                                                                          nāma ekādaśo 'dhyāyaḥ \Ed}%
\bekveg\szamveg\vfill\phpspagebreak\szam\bek\versno=0\fejno=12
\thispagestyle{empty}



\alfejezet{\textbf{dvādaśamo 'dhyāyaḥ}}\jump\jump 
\vers


\alalfejezet{ātithyadharmaḥ} 
devy uvāca\thinspace{\dandab} \dontdisplaylinenum
            \var{\vo devy uvāca\lem  \msCa\msCb\Ed; \om\ \msBod}%

ahi\.msā paramo dharmaḥ satata\.m parikīrtyate \danda\dontdisplaylinenum

ātithyakānā\.m dharma\.m ca kathayasva yad uttamam \veg\dontdisplaylinenum
            \var{\vc ātithya°\lem  \msCa\Ed; atithya° \msCb}%

maheśvara uvāca~{\dandab}\dontdisplaylinenum 

ahi\.msātithyakānā\.m ca ś\textsubring{r}ṇu dharma\.m yad uttamam\thinspace{\danda} \dontdisplaylinenum
            \var{\vb ś\textsubring{r}ṇu\lem  \msCb\Ed; {\lost}ṇu \msCa\oo
                 dharma\.m\lem  \msCa\msCb; dharma \Ed\oo
                 °ttamam\lem  \msCa\msCb; °ttamā\.m \Ed}%

trailokyam akhila\.m devi ratnapūrṇa\.m sulocane \veg\dontdisplaylinenum
            \var{\vd °pūrṇa\.m\lem  \msCa\msCb; °pūrṇā\.m \Ed\oo
                 °locane\lem  \msCa\Ed; °locana\.m \msCb}%

caturvedavide dāna\.m na tattulyam ahi\.msakaḥ\thinspace{\dandab} \dontdisplaylinenum
            \var{\va dāna\.m\lem  \msCa\Ed; nāna\.m \msCb}%

ś\textsubring{r}ṇu dharmam atithyānā\.m kīrtayiṣyāmi sundari \veg\dontdisplaylinenum


\alalfejezet{vipulopākhyānam}
āsīd v\textsubring{r}tta\.m purākhyāna\.m nagare kusumāhvaye\thinspace{\dandab} \dontdisplaylinenum
            \var{\va āsīd v\textsubring{r}tta\.m\lem  \msCa\Ed; āśīdatta\.m \msCb\oo 
                 °khyāna\.m\lem  \msCa\msCb; °khyāta\.m \Ed}%

kapilasya suto vidvān vipulo nāma viśrutaḥ \veg\dontdisplaylinenum

dharmanityo jitakrodhaḥ satyavādī jitendriyaḥ\thinspace{\dandab} \dontdisplaylinenum
            \paral{\textit{\vb {\normalfont  = MBh 12.218.13b }}}

brahmaṇyaś ca k\textsubring{r}tajñaś ca madbhaktaḥ k\textsubring{r}taniścayaḥ \veg\dontdisplaylinenum
            \var{\vc brahmaṇya°\lem  \msCb\Ed; brāhmaṇya° \msCa\oo
                 °jñaś ca\lem  \msCa\Ed; °jña \msCb\
             \vd °bhaktaḥ\lem  \msCa\msCb; °bhakta° \Ed}%

dhanāḍhyo 'tithipūjyaś ca dātā dānto dayālukaḥ\thinspace{\dandab} \dontdisplaylinenum
            \var{\va °pūjyaś ca\lem  \msCa\Ed; °pūjya \msCb\
             \vb dānto\lem  \msCbacorr\Ed; dānta \msCa, dāntom{\normalfont (?)} \msCbpcorr}%

nyāyārjitadhano nityam anyāyaparivarjitaḥ \veg\dontdisplaylinenum
            \var{\vc nyāyā°\lem  \Ed; nyāyo° \msCa\msCb}%

bhāryā ca rūpiṇī tasya candrabimbaśubhānanā\thinspace{\dandab} \dontdisplaylinenum

pīnottuṅgastanī kāntā sakalānandakāriṇī \danda\dontdisplaylinenum
            \var{\vd sakalā°\lem  \msCb\Ed; {\lost}{\lost}{\lost} \msCa}%

pativratā patiratā patiśuśrūṣaṇe ratā \veg\dontdisplaylinenum
            \var{\ve pativratā\lem  \msCa\Ed; prativratā \msCb}%

atha kenāpi kālena sūryarāgam abhūt tataḥ\thinspace{\dandab} \dontdisplaylinenum

grastabhāgatrayas tv āsīt k\textsubring{r}ṣṇamādhavamāsike \veg\dontdisplaylinenum

snātukāmāvatīryante sarve pauran\textsubring{r}pādayaḥ\thinspace{\dandab} \dontdisplaylinenum
            \var{\va °vatīryante\lem  \msCa\msCb; ca tīrthante \Ed}%

devāś ca pitaraś caiva tarpyante vidhivat tathā \veg\dontdisplaylinenum
            \var{\vd tarpyante\lem  \msCa\Ed; tapyante \msCb}%

kecij juhvati tatrāgni\.m kecid viprā\.mś ca tarpayet\thinspace{\dandab} \dontdisplaylinenum
            \var{\va °cij juhvati\lem  \msCa\Ed; °cij juti \msCb\
             \vb viprā\.mś ca\lem  \msCa\Ed; viprāś ca \msCb}%

kecid dānopatiṣṭhanti kecit stunvanti devatām \veg\dontdisplaylinenum
            \var{\vc dāno°\lem  \msCa\msCb; dhyāno° \Ed\
             \vd stunvanti\lem  \msCa\Ed; stuvanti \msCb\oo
                 devatām\lem  \msCa\Ed; devatā \msCb}%

dhyānayogaratāḥ kecit kecit pañcatape ratāḥ\thinspace{\dandab} \dontdisplaylinenum

eva\.m pravartamāneṣu rājanādiṣu sarvaśaḥ \veg\dontdisplaylinenum
            \var{\vd rājanā°\lem  \msCa\msCb; rājānā° \Ed}%

vipulo 'pi ca tatraiva gaṅgāgaṇḍakisa\.mgame\thinspace{\dandab} \dontdisplaylinenum
            \var{\va 'pi ca\lem  \msCa\Ed; 'pi \msCb}%

bhāryayā saha tatraiva snātvā kṣomavibhūṣaṇaḥ \veg\dontdisplaylinenum
            \var{\vc bhāryayā\lem  \msCapcorr\msCb; bhāryāyā \msCaacorr\Ed}%

devatāguruviprāṇām anyeṣā\.m tarpaṇe rataḥ\thinspace{\dandab} \dontdisplaylinenum
            \var{\va °viprā°\lem  \msCb\Ed; °vi{\lost} \msCa}%
            \var{\vb tarpaṇe\lem  \msCa\msCb; tarpaṇā \Ed}%

tatrāvasarasamprāpto brāhmaṇo 'tithir āgataḥ \veg\dontdisplaylinenum

bhāryā tasyātirūpeṇa mohitā brahmaṇas tadā\thinspace{\dandab} \dontdisplaylinenum
            \var{\vb mohitā\lem  \msCa\Ed; mohito \msCb\oo
                 brahmaṇas tadā\lem  \msCa\msCb; brāhmaṇasya ca \Ed}%

brāhmaṇo 'pi tathaiveha rūpeṇāpratimo bhavet \veg\dontdisplaylinenum
            \var{\vc brāhmaṇo\lem  \msCa\Ed; brahmaṇo \msCb\oo
                 tathaiveha\lem  \msCb\Ed; ta\uncl{the}veha \msCa}%

anyonyad\textsubring{r}ṣṭisa\.msaktau jātau tau tu parasparam\thinspace{\dandab} \dontdisplaylinenum
            \var{\va °sa\.msaktau\lem  \msCc\Ed; °sa\.mśaktau \msCa, °śaktau \msCb}%

vipulenāñjali\.m k\textsubring{r}tvā brāhmaṇa sa\.mśitavrata \veg\dontdisplaylinenum
            \var{\vd brāhmaṇa\lem  \msCb; brāhmaṇaḥ \msCa\Ed\oo 
                 °śita°\lem  \eme; °śrita° \msCa\msCb\Ed\oo
                 °vrata\lem  \conj; °vra{\il} \msCa, °vrataḥ \msCb\Ed}%
            \paral{\textit{\vd {\normalfont  = MBh 12.213.18d and 12.347.1d }}}

ājñāpaya dvijaśreṣṭha adya me 'nugraha\.m kuru\thinspace{\dandab} \dontdisplaylinenum
            \var{\vb °graha\.m\lem  \msCa\Ed; °graha \msCb}%

bhāryābh\textsubring{r}tyapaśugrāma ratnāni vividhāni ca \veg\dontdisplaylinenum

vipulenaivam uktas tu g\textsubring{r}hīto brāhmaṇo 'bravīt\thinspace{\dandab} \dontdisplaylinenum

yadi satya\.m pradātāsi suprasanna\.m manas tava \veg\dontdisplaylinenum
            \var{\vf °sanna\.m\lem  \msCa\msCb; °sanna° \Ed}%

vipula uvāca~{\dandab}\dontdisplaylinenum 

suprasanna\.m mano me 'dya suprasanna\.m tapaḥphalam\thinspace{\danda} \dontdisplaylinenum

śīghram ājñāpaya vipra yac cābhilaṣita\.m tava \danda\dontdisplaylinenum

adeya\.m nāsti viprasya svaśiraḥprabh\textsubring{r}ti dvija \veg\dontdisplaylinenum
            \var{\vf °bh\textsubring{r}ti\lem  \msCa\msCb; °bh\textsubring{r}tir \Ed}%

brāhmaṇa uvāca~{\dandab}\dontdisplaylinenum 
            \var{\vo brāhmaṇa\lem  \msCapcorr\Ed; brāhmaṇā \msCaacorr}%

yady eva\.m vadase bhadra bhāryā\.m me dehi rūpiṇīm\thinspace{\danda} \dontdisplaylinenum

svasti bhavatu bhadra\.m vaḥ kalyāṇa\.m bhava śāśvatam \veg\dontdisplaylinenum
            \var{\vc svasti\lem  \msCa\msCb; svastir \Ed}%
            \var{\vd bhava\lem  \msCa\msCb; tava \Ed}%

vipula uvāca~{\dandab}\dontdisplaylinenum 
            \var{\vo vipula\lem  \msCa; vipra \Ed}%

pratīccha bhāryā\.m suśroṇī\.m rūpayauvanaśālinīm\thinspace{\danda} \dontdisplaylinenum

akutsitā\.m viśālākṣī\.m pūrṇacandranibhānanām \veg\dontdisplaylinenum

bhāryovāca\thinspace{\dandab} \dontdisplaylinenum

parityājyā katha\.m nātha apāpā\.m tyajase katham \danda\dontdisplaylinenum
            \var{\va °tyājyā\lem  \msCa\Ed; °tyājya \msCb}%

atīva hi priyā\.m bhāryā\.m nirdoṣā\.m sa katha\.m tyajeḥ \veg\dontdisplaylinenum
            \var{\vd tyajeḥ\lem  \msCa; tyajet \Ed}%

sakhā bhāryā manuṣyāṇām iha loke paratra ca\thinspace{\dandab} \dontdisplaylinenum

dāna\.m vā sumahad dattvā yajño vā subahuḥ k\textsubring{r}taḥ \veg\dontdisplaylinenum
            \var{\vd °bahuḥ\lem  \eme; °bahu \msCa\msCb\ \unmetr; °bahūn \Ed}%

aputro nāpnuyāt svarga\.m tapobhir vā suduṣkaraiḥ\thinspace{\dandab} \dontdisplaylinenum
            \var{\vab svarga\.m tapobhir vā\lem  \Ed; sva\uncl{rggan} {\lost}{\lost}{\lost}r vvā \msCa}%

śruto me pit\textsubring{r}bhiḥ prokto brāhmaṇaiś ca mamāntike \veg\dontdisplaylinenum
            \var{\vd °ntike\lem  \msCa\Ed; °ntikaiḥ \msCb}%

aputro nāpnuyāt svarga\.m śruta\.m me bahuśaḥ purā\thinspace{\dandab} \dontdisplaylinenum
            \var{\va svarga\.m\lem  \msCa\Ed; svarga \msCb}%

mandapālo dvijaśreṣṭho gataḥ svarga\.m tapobalāt \veg\dontdisplaylinenum
            \var{\vc °pālo\lem  \Ed; °pāla \msCa\msCb}%
            \paral{\textit{\vc {\normalfont See Mandapāla's story in MBh 1.220.5ff.}}}

dānāni ca bahūn dattvā yajñā\.mś ca vividhā\.ms tathā\thinspace{\dandab} \dontdisplaylinenum
            \var{\vb yajñā\.mś ca vividhā\.ms\lem  \msCa; yatvā yajñā\.mś ca vividhā\.m \msCb, syajñāś ca vividhās \Ed}%

vedā\.mś ca japa yajñā\.mś ca k\textsubring{r}tvā tad dvijasattamaḥ \veg\dontdisplaylinenum
            \var{\vc vedā\.mś ca japayajñā\.mś ca\lem  \msCa; vedāś ca japayajñā\.mś ca \msCb,
                                                 vedāś ca japayajñāś ca \Ed}%

prāptadvāro 'pi yasyāpi devadūtair nivāritaḥ\thinspace{\dandab} \dontdisplaylinenum
            \var{\va yasyāpi\lem  \msCa\msCb; yasyāhi \Ed}%

aputro nāpnuyāt svarga\.m yadi yajñaśatair api \veg\dontdisplaylinenum

ity uktas tu cyutaḥ svargān mandapālo mahān \textsubring{r}ṣiḥ\thinspace{\dandab} \dontdisplaylinenum

putrān utpādayām āsa śāraṅgāś caturo dvijaḥ \veg\dontdisplaylinenum
            \var{\vd śāraṅgāś ca\lem  \msCa; śāraṅga\.mś ca \msCb, śāraṅgāc ca \Ed}%

tena puṇyaprabhāveṇa svarga\.m prāpto hy avāritaḥ\thinspace{\dandab} \dontdisplaylinenum

kulatrāṇā\.m kalatrāsmi bharaṇād bhārya eva ca \veg\dontdisplaylinenum
            \var{\vc kula°\lem  \msCb; kala° \msCa\Ed\
             \vd bhārya eva\lem  \msCa\Ed; bhāryam eva \msCb}%

dārasa\.mgraha putrārthe kriyate śāstradarśanāt\thinspace{\dandab} \dontdisplaylinenum
            \var{\va °graha\lem  \Ed; °grahaḥ \msCa\msCb\oo
                 putrā°\lem  \msCa\msCb; pātrā° \Ed\
             \vb kriyate\lem  \msCa\Ed; kriyāte \msCb}%

yāni santi g\textsubring{r}he dravya\.m grāmaghoṣag\textsubring{r}hāṇi ca \veg\dontdisplaylinenum

dātum arhasi viprāya na mā\.m dātum ihārhasi\thinspace{\dandab} \dontdisplaylinenum

bhāryāyā vacana\.m śrutvā vipulaḥ punar abravīt \veg\dontdisplaylinenum

vipula uvāca~{\dandab}\dontdisplaylinenum 
            \var{\vo vipula uvāca\lem  \Ed; \om\ \msCa\msCb}%

sādhu bhāmini jānāmi sādhu sādhu pativrate\thinspace{\danda} \dontdisplaylinenum
            \var{\va jānāmi\lem  \msCb\Ed; jānāsi \msCa}%

jito 'smy anena vākyena anenāsmi hi toṣitaḥ \veg\dontdisplaylinenum

adya grahaṇakāle ca dvija āgatya yācate\thinspace{\dandab} \dontdisplaylinenum

dadāmīti pratijñāya adattvā naraka\.m vraje \veg\dontdisplaylinenum
            \var{\vd vraje\lem  \msCa; vrajet \msCb\Ed}%

naraka\.m yadi gacchāmi kulena saha sundari\thinspace{\dandab} \dontdisplaylinenum

kalpakoṭisahasre 'pi narakasthād yaśasvini \veg\dontdisplaylinenum
            \var{\vc °sahasre 'pi\lem  \msCa\msCb; °sahasrāṇi \Ed}%
            \var{\vd °sthād\lem  \msCa; sthā \msCb, °stho \Ed}%

muktim eva na paśyāmi janmakoṭiśatair api\thinspace{\dandab} \dontdisplaylinenum
            \var{\va muktim eva\lem  \msCa; muktim evan \Ed}%

adānāc cāśubha\.m devi paśyāmi varavarṇini \veg\dontdisplaylinenum

dānena tu śubha\.m paśye svargaloke yad akṣayam\thinspace{\dandab} \dontdisplaylinenum
            \var{\vb °loke\lem  \msCa; °loka\.m \Ed}%

nokta\.m mayān\textsubring{r}ta\.m pūrva\.m nitya\.m satyavrate sthitaḥ \veg\dontdisplaylinenum
            \var{\vd °vrate\lem  \msCa\msCb; °vrata° \Ed}%

satyadharmam atikramya nānyadharma\.m samācare\thinspace{\dandab} \dontdisplaylinenum
            \var{\vb °care\lem  \msCa\msCb; °caret \Ed}%

bhāryā dharmasakhety eva\.m tvayi pūrvam udāh\textsubring{r}tam \veg\dontdisplaylinenum

yadi dharmasakhāyāsi so 'dya kāla ihāgataḥ\thinspace{\dandab} \dontdisplaylinenum
            \var{\va °sakhāyā°\lem  \msCa\Ed; °sakhā° \msCb}%

dvijarūpadharo dharmaḥ svayam eva ihāgataḥ \veg\dontdisplaylinenum
            \var{\vc °dharo\lem  \msCa\Ed; °paro \msCb}%

jijñāsārtham aha\.m bhadre na vighna\.m kartum arhasi\thinspace{\dandab} \dontdisplaylinenum

mātāvyaktaḥ pitā brahmā buddhir bhāryā damaḥ sakhā \veg\dontdisplaylinenum
            \var{\vd buddhir\lem  \msCa\msCb; buddhi \Ed\oo
                 sakhā\lem  \msCb\Ed; samā \msCa}%

putro dharmaḥ kriyācārya ity ete mama bāndhavāḥ\thinspace{\dandab} \dontdisplaylinenum

kālaśreṣṭho grahaḥ sūryo gaṅgā śreṣṭhā nadīṣu ca \veg\dontdisplaylinenum
            \var{\vc °śreṣtho\lem  \msCb; °śreṣṭha° \msCa, °śreṣṭhaḥ \Ed}%
            \paral{\textit{\vd {\normalfont \kb\ 15.18b: } śreṣṭhā gaṅgā nadīṣu ca}}

candrakṣaye dina\.m śreṣṭha\.m naraśreṣṭho dvijottamaḥ\thinspace{\dandab} \dontdisplaylinenum
            \var{\va dina\.m\lem  \msCa\msCb; dina° \Ed}%

śuśrūṣaṇārtha\.m viprasya mayā dattāsi sundari \danda\dontdisplaylinenum
            \var{\va °rtha\.m\lem  \msCa\Ed; °rtha \msCb\
             \vb dattāsi\lem  \msCa\msCb; dattāni \Ed}%

sarvasva\.m brāhmaṇe dattvā vanam evāśrayāmy aham \veg\dontdisplaylinenum

śaṅkara uvāca~{\dandab}\dontdisplaylinenum 
            \var{\vo śaṅkara\lem  \msCa\msCb; maheśvara \Ed}%

tūṣṇīmbhūtā tato bhāryā aśrupūrṇākulekṣaṇā\thinspace{\danda} \dontdisplaylinenum
            \var{\va tūṣṇīmbhūtā\lem  \msCa; tūṣṇībhūtvā \msCb, tūṣṇībhūtā\.m \Ed\oo
                 bhāryā\lem  \msCa\msCb; bhāryā\.m \Ed}%
            \var{\vb °kṣaṇā\lem  \msCa; °kṣaṇām \Ed}%

kare g\textsubring{r}hya viśālākṣī brāhmaṇāya niveditā \veg\dontdisplaylinenum
            \var{\vc °kṣī\lem  \msCa\msCb; °kṣī\.m \Ed\
             \vd brāhmaṇāya niveditā\lem  \msCa\Ed; brāhmaya diveditā \msCb}%

yāni santi g\textsubring{r}he dravya\.m hiraṇya\.m paśavas tathā\thinspace{\dandab} \dontdisplaylinenum
            \var{\vb hiraṇya\.m\lem  \msCa\msCb; hiraṇya° \Ed}%

dadāmi te dvijaśreṣṭha grāmaghoṣag\textsubring{r}hādikam \veg\dontdisplaylinenum
            \var{\vc dadāmi\lem  \msCa\Ed; dadāni \msCb\
                 te dvija°\lem  \msCb\Ed; {\lost}{\lost}ja° \msCa}%

muktā vaiḍūryavāsā\.msi divyāṇy ābharaṇāni ca\thinspace{\dandab} \dontdisplaylinenum

sarvān g\textsubring{r}hāṇa viprendra śraddhayā dattasatk\textsubring{r}tām \veg\dontdisplaylinenum

prīyatā\.m bhagavān dharmaḥ prīyatā\.m ca maheśvaraḥ\thinspace{\dandab} \dontdisplaylinenum

prīyantā\.m pitaraḥ sarve yady asti suk\textsubring{r}ta\.m phalam \veg\dontdisplaylinenum
            \var{\vc prīyantā\.m\lem  \msCa; prīyatā\.m \msCb\Ed}%
            \var{\vd asti\lem  \msCb\Ed; asi \msCa}%

rudra uvāca~{\dandab}\dontdisplaylinenum 
            \var{\vo rudra\lem  \msCa\msCb; maheśvara \Ed}%

vipulasya vacaḥ śrutvā brāhmaṇena tapasvinā\thinspace{\danda} \dontdisplaylinenum
            \var{\va vacaḥ śrutvā\lem  \msCb\Ed; vaca\uncl{ś śru}{\lost} \msCa}%

āśīḥ suvipula\.m dattvā vipulāya mahātmane \veg\dontdisplaylinenum

vaset tatra g\textsubring{r}he ramye bhāryām ādāya tasya ca\thinspace{\dandab} \dontdisplaylinenum
            \var{\va vaset tatra g\textsubring{r}he\lem  \msCb\msNa; vasa tatra g\textsubring{r}he \msCa, vasate ca g\textsubring{r}ha\.m \Ed}%

vipulas tu namask\textsubring{r}tvā k\textsubring{r}tvā cāpi pradakṣiṇam \veg\dontdisplaylinenum
            \var{\vd cāpi\lem  \msCa\msCb; ca vi° \Ed}%

brāhmaṇam abhivādyaiva\.m gataḥ śīghra\.m vanāntaram\thinspace{\dandab} \dontdisplaylinenum
            \var{\va °dyaiva\.m\lem  \eme; °dyeva\.m \msCa\Ed, °dyena\.m \msCb}%

vane mūlaphalāhāro vicareta mahītale \veg\dontdisplaylinenum

ekākī vijane śūnye cintayā ca pariplutaḥ\thinspace{\dandab} \dontdisplaylinenum
            \var{\va ekākī\lem  \msCb\Ed; e\uncl{kā}{\lost} \msCa}%

kva gacchāmi kva bhokṣyāmi kutra vā ki\.m karomy aham \veg\dontdisplaylinenum
            \var{\vc kva bhokṣyāmi\lem  \msCa; kva bhojyāmi \msCb, ki\.m bhokṣyāmi \Ed\ \unmetr}%

na patha\.m viṣaya\.m vedmi grāma\.m vā nagarāṇi vā\thinspace{\dandab} \dontdisplaylinenum

kheṭakharvaṭadeśa\.m vā jānāmīha na ka\.mcana \veg\dontdisplaylinenum
            \var{\vc °kharvaṭa°\lem  \Ed; °karppaṭa° \msCa\msCb}%
            \var{\vd ka\.mcana\lem  \eme; kaścana \msCa\msCb\Ed}%

amu\.m suśaila\.m paśyāmi vipulodarakandaram\thinspace{\dandab} \dontdisplaylinenum

tam āruhya nirīkṣyāmi grāma\.m nagarapattanam \veg\dontdisplaylinenum

evam uktvā tu vipulaḥ śanaiḥ parvatam āruhat\thinspace{\dandab} \dontdisplaylinenum
            \var{\va evam u°\lem  \msCa\Ed; eka\.m u° \msCb\
             \vb °ruhat\lem  \Ed; °ruhet \msCa\msCb}%

v\textsubring{r}kṣacchāyā\.m samālokya niṣasāda śramānvitaḥ \veg\dontdisplaylinenum

etasminn eva kāle tu v\textsubring{r}kṣaśākhāvatārya ca\thinspace{\dandab} \dontdisplaylinenum
            \var{\va kāle tu\lem  \msCa\msCb; kālena \Ed}%

apūrva\.m ca surūpa\.m ca sugandhatva\.m ca śobhanam \veg\dontdisplaylinenum
            \var{\vc surūpa\.m\lem  \msCa\Ed; svarūpa\.m \msCb}%

phala\.m g\textsubring{r}hya vicitra\.m ca h\textsubring{r}dayānandana\.m śubham\thinspace{\dandab} \dontdisplaylinenum

vipulasyāgrataḥ k\textsubring{r}tvā punar v\textsubring{r}kṣa\.m samāruhat \veg\dontdisplaylinenum

vipulaś citravad d\textsubring{r}ṣṭvā vismaya\.m parama\.m gataḥ\thinspace{\dandab} \dontdisplaylinenum

aho vā svapnabhūto 'smi aho vā tapasaḥ phalam \veg\dontdisplaylinenum

na paśyāmi na jighrāmi na ca svāda\.m ca vedmy aham\thinspace{\dandab} \dontdisplaylinenum
            \var{\va jighrāmi\lem  \msCa\Ed; ca ghrāmi \msCb}%

vārtāpi na ca me śrotrā pratijānāmi ka\.mcana \veg\dontdisplaylinenum
            \var{\vc śrotrā\lem  \msCa; śrotā \msCb\Ed}%
            \var{\vd ka\.mcana\lem  \eme; kaścana \msCa\msCb\Ed}%

evam uktvā hy anekāni phala\.m g\textsubring{r}hya manoramam\thinspace{\dandab} \dontdisplaylinenum

sunirīkṣya punar jighra\.m punar jighra\.m nirīkṣya ca \veg\dontdisplaylinenum

phala\.m cātra nirūpyanto deśa\.m vāpy avalokayan\thinspace{\dandab} \dontdisplaylinenum
            \var{\va cātra\lem  \msCb\Ed; cā \msCaacorr, cā\uncl{tra} \msCapcorr\oo
                 nirūpyanto\lem  \Ed; nirūpyānti \msCa, nirūpyā\.m cā\
             \vb °lokayan\lem  \msCa\Ed; °lokayat \msCb}%

pātheyarahitaś cāsmi devadatta\.m phala\.m mama \veg\dontdisplaylinenum
            \var{\vd °datta\.m\lem  \msCa; °datta° \msCb\Ed}%

tatphala\.m pratig\textsubring{r}hyaiva nagara\.m praviśāmy aham\thinspace{\dandab} \dontdisplaylinenum
            \var{\va °g\textsubring{r}hyaiva\lem  \msCb\Ed; °g\textsubring{r}hyeva \msCa}%

prārthayitvā ca yat ki\.mcij jīvanārtha\.m carāmy aham \veg\dontdisplaylinenum

tataḥ śailam atikramya nagara\.m praviveśa ha\thinspace{\dandab} \dontdisplaylinenum

pathi kaścij janaḥ p\textsubring{r}ṣṭhaḥ ki\.mnāma nagara\.m tv idam \veg\dontdisplaylinenum
            \var{\vd nagara\.m\lem  \msCa\Ed; nagara \msCb}%

sa hovāca pathī kena kim apūrvam ihāgataḥ\thinspace{\dandab} \dontdisplaylinenum
            \var{\va sa ho°\lem  \msCa\Ed; aho° \msCb}%

dakṣiṇāpathadeśo 'ya\.m naravīrapura\.m tv adaḥ \veg\dontdisplaylinenum
            \var{\vc °patha°\lem  \msCa\Ed; °pathe \msCb\
             \vd tv adaḥ\lem  \msCb; tvayaḥ \msCa, svayam \Ed}%

rājā si\.mhajaṭo nāma rājñī tasya ca kekayī\thinspace{\dandab} \dontdisplaylinenum
            \var{\va rājā\lem  \msCa\msCb; rāja \Ed\oo
                 °jaṭo\lem  \msCa\msCb; °yato \Ed}%
            \var{\vb kekayī\lem  \msCb\Ed; kaikayī \msCa}%

ativ\textsubring{r}ddho jarāgrastaḥ kekayī ca tathaiva ca \veg\dontdisplaylinenum
            \var{\vd kekayī\lem  \msCb\Ed; kaikayī \msCa}%

dātā sarvakalājñaś ca yuddhe vīryabalānvitaḥ\thinspace{\dandab} \dontdisplaylinenum
            \var{\va dātā\lem  \msCb\Ed; {\lost}tā \msCa\oo
                 °kalā°\lem  \Ed; °kala° \msCa\msCb}%

brahmaṇyo vatsalo loke sarvaśāstraviśāradaḥ \veg\dontdisplaylinenum

vipula uvāca~{\dandab}\dontdisplaylinenum 

atra śreṣṭhim upāsyāmi nāma vā tasya ki\.m vada\thinspace{\danda} \dontdisplaylinenum
            \var{\vb nāma\lem  \msCa\msCb; nāma\.m \Ed\oo
                 vada\lem  \msCa\Ed; vadaḥ \msCb}%

katamo deśas tadvāsaḥ kathayasva na sa\.mśayaḥ \veg\dontdisplaylinenum
            \var{\vd kathayasva\lem  \msCa\Ed; kathayasya \msCb}%

vipulenaivam uktas tu pathikovāca ta\.m punaḥ\thinspace{\dandab} \dontdisplaylinenum

mama bhīmabalo nāma śreṣṭhikasya g\textsubring{r}hāgataḥ \veg\dontdisplaylinenum
            \var{\vc mama bhīmabalo nāma\lem  \msCb\msCc\msNa; mama bhī{\lost}balo nāma \msCa, \om\ \Ed}%

śreṣṭhikaḥ puṇḍako nāma khyātaḥ śreṣṭhika ucyate\thinspace{\dandab} \dontdisplaylinenum

kautuka\.m tava yady asti tad āgaccha mayā saha \veg\dontdisplaylinenum

evam astv iti tenokto vipulena mahātmanā\thinspace{\dandab} \dontdisplaylinenum
            \var{\vc °stv iti\lem  \msCa\msNa\Ed; °stiti \msCb\msCc\oo
                 °kto\lem  \mssCaCbCc\msNa; °ktau \Ed}%

tenaiva saha niryātaḥ śreṣṭhikasya g\textsubring{r}ha\.m prati \veg\dontdisplaylinenum
            \var{\vb prati\lem  \msCa\msCb\msNa; pratiḥ \Ed\msCc}%

śreṣṭhikaḥ svag\textsubring{r}hāsīno d\textsubring{r}ṣṭaḥ sa vipulena tu\thinspace{\dandab} \dontdisplaylinenum
            \var{\vc śreṣṭhikaḥ\lem  \Ed\msCb\msCc; śreṣṭhitaḥ \msCa, śreṣṭhikaḥ \msNa\
             \vd d\textsubring{r}ṣṭaḥ sa\lem  \msCb\Ed; \uncl{d\textsubring{r}}{\lost}{\lost} \msCa, d\textsubring{r}ṣṭa sa \msCc}%

tasyāntikam upāgamya tat phala\.m sa niveditaḥ \veg\dontdisplaylinenum

aho phalam ida\.m śreṣṭham aho phalam ihānitam\thinspace{\dandab} \dontdisplaylinenum

aho rūpam aho gandham aho phala\.m suśobhanam \veg\dontdisplaylinenum
            \var{\vc gandham\lem  \msCa\msCbpcorr\Ed; gandham aho gandham \msCbacorr\
             \vd phala\.m\lem  \corr; phala \msCa\msCb\Ed}%

tat phala\.m na mahījāta\.m na merau na ca kandare\thinspace{\dandab} \dontdisplaylinenum
            \var{\va tat pha°\lem  \msCa\msCb; yat pha° \Ed}%

devalokika suvyakta\.m na martya upajāyate \veg\dontdisplaylinenum
            \var{\vd martya upajāyate\lem  \eme;
                      martya\uncl{mupajā}{\lost}{\lost} \msCa, martya supajāyate \msCb, mahyām upajāyate \Ed}%

aho 'smi saphala\.m bhoktā rājārhaś ca na sa\.mśayaḥ\thinspace{\dandab} \dontdisplaylinenum
            \var{\va aho\lem  \msCb; {\lost}ho \msCa; adyo \Ed\oo
                 saphala\.m\lem  \msCb; \uncl{sa}phalam \msCa, tat phala\.m \Ed}%

ḍhaukayitvā phala\.m divya\.m rājāna\.m toṣayāmy aham \veg\dontdisplaylinenum

tatas tvarita gatvaiva phala\.m g\textsubring{r}hya manoharam\thinspace{\dandab} \dontdisplaylinenum
            \var{\va tvarita\lem  \Ed; tvarita\.m \msCa\msCb\ \unmetr}%
            \var{\vb g\textsubring{r}hya\lem  \msCa\Ed; g\textsubring{r}ha \msCb\
                 °haram\lem  \msCa\msCb; °ramam \Ed}%

ādareṇopas\textsubring{r}tyaiva rājāna\.m sa phala\.m dadau \veg\dontdisplaylinenum
            \var{\vd sa phala\.m\lem  \msCa\msCb; tat phala\.m \Ed}%

rājā ca sa phala\.m d\textsubring{r}ṣṭvā vismaya\.m parama\.m gataḥ\thinspace{\dandab} \dontdisplaylinenum
            \var{\va sa phala\.m\lem  \msCa\msCb; tat phala\.m \Ed}%

kutaḥ śreṣṭhi tvayā nīta\.m phala\.m sarvamanoharam \veg\dontdisplaylinenum
            \var{\vc śreṣṭhi\lem  \msCa\msCb; śreṣṭha \Ed}%
            \var{\vd phala\.m sarvamanoharam\lem  \Ed; phala{\lost}{\lost}{\lost}{\lost}haram \msCa\ phala\uncl{m ya}rvamanoharam \msCb}%

svādumūlaphalakanda\.m d\textsubring{r}ṣṭvā pūrva\.m na tād\textsubring{r}śam\thinspace{\dandab} \dontdisplaylinenum
            \var{\va °kanda\.m d\textsubring{r}ṣṭvā\lem  \msCa; °skanda d\textsubring{r}ṣṭvā \msCb,°skanda d\textsubring{r}ṣṭā \Ed}%
            \var{\vb tād\textsubring{r}śam\lem  \msCa\msCb; yād\textsubring{r}śam \Ed}%

rūpagandhaguṇopeta\.m h\textsubring{r}dayānandakārakam \veg\dontdisplaylinenum

sadya evopabhuñjāmi tvayā dattam ida\.m phalam\thinspace{\dandab} \dontdisplaylinenum
            \var{\va sadya evopayuñjāmi\lem  \msCa\msCb; satya eva prabhuñjāmi \Ed}%

kīd\textsubring{r}śa\.m svāda vijñātum icchāmi kuru māciram \veg\dontdisplaylinenum
            \var{\vc svādavijñānam\lem  \msCa\msCb; svādu vijñātum \Ed}%

tataḥ sa bhakṣayām āsa phala\.m cām\textsubring{r}tasa\.mnibham\thinspace{\dandab} \dontdisplaylinenum
            \var{\va tataḥ\lem  \msCa\Ed; tata \msCb}%

am\textsubring{r}topamasusvāda\.m sarva\.m ca bubhuje n\textsubring{r}paḥ \veg\dontdisplaylinenum
            \var{\vcd svāda\.m sarva\.m ca\lem  \msCb\Ed; svā{\lost}{\lost}{\lost}{\lost} \msCa}%

sadya ṣoḍaśavarṣasya yauvana\.m samapadyata\thinspace{\dandab} \dontdisplaylinenum
            \var{\vb °padyata\lem  \msCa\msCb; °padyate \Ed}%

na valīpalita\.m sadyo na jarā na ca durbalaḥ \veg\dontdisplaylinenum
            \var{\vc valī°\lem  \msCa\msCb; vali° \Ed}%

keśadantanakhasnigdho d\textsubring{r}ḍhadanto d\textsubring{r}ḍhendriyaḥ\thinspace{\dandab} \dontdisplaylinenum
            \var{\vb °danto\lem  \msCa; °deho \Ed}%

tejaścakṣurbalaprāṇān sadya sarvān avāptavān \veg\dontdisplaylinenum
            \var{\vc °cakṣurbalaprāṇān\lem  \msCa\msCb; °cakṣuvalaprāṇa\.m \Ed}%

mantrī purohitāmātya sarve bh\textsubring{r}tyajanās tathā\thinspace{\dandab} \dontdisplaylinenum
            \var{\vb sarve bh\textsubring{r}tyajanās tathā\lem  \msCa\Ed; janās tathās tathā \msCb}%

paurastrī bālav\textsubring{r}ddhāś ca sarve te vismaya\.m gatāḥ \veg\dontdisplaylinenum
            \var{\vc °strī\lem  \msCa\msCb; °stri \Ed}%
            \var{\vd sarve\lem  \msCb\Ed; {\lost}{\lost} \msCa}%

rājā si\.mhajaṭo nāma tuṣṭim eva parā\.m gataḥ\thinspace{\dandab} \dontdisplaylinenum

praharṣam atula\.m caiva prāptavān sa nareśvaraḥ \veg\dontdisplaylinenum

uvāca rājā ta\.m śreṣṭhi\.m svārthatatparanirdayaḥ\thinspace{\dandab} \dontdisplaylinenum
            \var{\va śreṣṭhi\.m\lem  \msCa\msCb; śreṣṭha\.m \Ed}%
            \var{\vb °dayaḥ\lem  \msCa\msCb; °daya \Ed}%

kuru bhīmabalas tv eva\.m phalam ānaya adya vai \veg\dontdisplaylinenum
            \var{\vc kuru\lem  \msCa\msCb; ś\textsubring{r}ṇu \Ed\oo
                 bhīmabalas tv eva\.m\lem  \msCb\msCc; bhīmavastv eva\.m \msCa\Ed}%

punar me yauvanaprāptis tvatprasādān narottama\thinspace{\dandab} \dontdisplaylinenum
            \var{\vb °ttama\lem  \msCa\msCb; °ttamaḥ \Ed}%

kekayī\.m durbalā\.m v\textsubring{r}ddhā\.m punaḥ prāpaya yauvanam \veg\dontdisplaylinenum
            \var{\vc kekayī\.m durbalā\.m\lem  \corr; kaikayīn durbalān \msCa, kekayī\.m \msCb, kekayī durbalā \Ed}%
            \var{\vcd v\textsubring{r}ddhā\.m punaḥ\lem  \msCb; v\textsubring{r}\uncl{ddhā}{\lost}{\lost} \msCa, v\textsubring{r}ddhā punaḥ \Ed}%

sa rājñā evam uktas tu śreṣṭhī bhīmabalas tathā\thinspace{\dandab} \dontdisplaylinenum
            \var{\vb śreṣṭhī\lem  \Ed; śreṣṭhi \msCa\msCb}%

pratyuvāca ha rājāna\.m prāñjaliḥ praṇataḥ sthitaḥ \veg\dontdisplaylinenum
            \var{\vc °vāca ha\lem  \msCa; °vācāha \Ed}%

na phaleda\.m vane rājan na vāṇijyak\textsubring{r}ṣeṇa vā\thinspace{\dandab} \dontdisplaylinenum
            \var{\va na phaleda\.m\lem  \Ed; na vane na \msCa}%

kenāpi kulaputreṇa tava darśanakā\.mkṣayā \veg\dontdisplaylinenum

datto 'smi tava rājendra mayā datto 'si bhūpate\thinspace{\dandab} \dontdisplaylinenum
            \var{\vb datto 'si\lem  \msCa; prāpto ṣi \Ed}%

na te śaknomy aha\.m rājan vaktu\.m vaideśina\.m naram \veg\dontdisplaylinenum
            \var{\vc te\lem  \msCa; ca \Ed}%
            \var{\vcd rājan vaktu\.m\lem  \Ed; rā{\lost}{\lost}ktum \msCa}%
            \var{\vd vaideśina\.m naram\lem  \corr; \uncl{vai}deśinan naram \msCa, ca dehi tannaraḥ \Ed}%

śrutvā bhīmabala\.m vākya\.m pratyuvāca tataḥ punaḥ\thinspace{\dandab} \dontdisplaylinenum
            \var{\va °bala\.m\lem  \Ed; °bala° \msCa\ \unmetr}%

amātyakulaputras tva\.m brūhi madvacana\.m punaḥ \veg\dontdisplaylinenum

yadi nāsti kim etat ta\.m mayā vā prārthito bhavān\thinspace{\dandab} \dontdisplaylinenum
            \var{\va kim etat\lem  \Ed; kim edat \msCa}%
            \var{\vb prārthito\lem  \Ed; mārjjito \msCa}%

yatra hy eko bahavo 'tra jāyante nātra sa\.mśayaḥ \veg\dontdisplaylinenum
            \var{\vc yatra hy eko bahavo 'tra\lem  \msCa\ \unmetr; 
                                yatraścaiko bahūn tatra \Ed}%

āgamopāyamārga\.m ca tenaiva sa tu gamyatām\thinspace{\dandab} \dontdisplaylinenum

avaśya\.m tena gantavya\.m tena mārgeṇa mārgaya \veg\dontdisplaylinenum
            \var{\vc avaśya\.m tena\lem  \Ed; ava\uncl{sya}{\lost}na \msCa}%
            \var{\vd mārgaya\lem  \msCa; mārgayaḥ \Ed}%

adattvā phalam anyac ca śiraś chedyāmi durmateḥ\thinspace{\dandab} \dontdisplaylinenum
            \var{\vb °mateḥ\lem  \eme; °mate \msCa\Ed}%

chedya caṇḍavicaṇḍābhyā\.m rakṣabhīmabalādhamaḥ \veg\dontdisplaylinenum
            \var{\vc chedya\lem  \Ed; chedye \msCa}%

tato bhīmabalaḥ kruddhaḥ khaḍga\.m g\textsubring{r}hya śaśiprabham\thinspace{\dandab} \dontdisplaylinenum
            \var{\vb śaśiprabham\lem  \msCa; śaśī pradam \Ed}%

alaṅghya vacana\.m rājñaḥ kulaputra vraja tvaram \veg\dontdisplaylinenum
            \var{\vc alaṅghya\lem  \msCa; uvāca \Ed}%
            \var{\vd kulaputra vraja tvaram\lem  \Ed; kulaputra\.m vrajatyaram \msCa}%

mā ruṣa kulaputra tva\.m mayā vadhyo bhaviṣyasi\thinspace{\dandab} \dontdisplaylinenum
            \var{\va °putra\lem  \msCa; °putras \Ed}%

yady asti phalam anyad vā dehi rājānam adya vai \veg\dontdisplaylinenum
            \var{\vc yady asti\lem  \Ed; {\lost}dyosti \msCa}%

yatra prāpta\.m phala\.m divya\.m tatra vā deśaya tava\thinspace{\dandab} \dontdisplaylinenum
            \var{\va prāpta\.m\lem  \msCa; prāpti \Ed}%
            \var{\vb deśaya\lem  \msCa; deśayan \Ed}%

tatphalena vinā bhadra durlabha\.m tava jīvitam \veg\dontdisplaylinenum

vipula uvāca~{\dandab}\dontdisplaylinenum 

jīvitāśām aha\.m prāpto vaideśi bhavana\.m tava\thinspace{\danda} \dontdisplaylinenum

k\textsubring{r}takartā katha\.m vadhyaḥ prāpnuyām aham adya vai \veg\dontdisplaylinenum
            \var{\vd prāpnuyām\lem  \msCa; prāpto 'yam \Ed}%

phala\.m vā na punas tv anyad dātu\.m śakya\.m na kenacit\thinspace{\dandab} \dontdisplaylinenum
            \var{\vb śakya\.m na kenacit\lem  \Ed; śakya{\lost}{\lost}nacit \msCa}%

sahya parvataśailāgre āśīnaḥ śrāntamānasaḥ \veg\dontdisplaylinenum

vānaras tatphala\.m g\textsubring{r}hya mama dattvā punar gataḥ\thinspace{\dandab} \dontdisplaylinenum
            \var{\vb mama\lem  \msCa; mahya\.m \Ed}%

mayā dattam ida\.m tubhya\.m tvayāpi ca narādhipe \veg\dontdisplaylinenum

tatra gacchāva bho śreṣṭhi d\textsubring{r}śyate yadi vānaraḥ\thinspace{\dandab} \dontdisplaylinenum

tvayā mayā ca gatvaiva yo vāsaḥ plavagādhipaḥ \veg\dontdisplaylinenum
            \var{\vc gatvaiva\lem  \Ed; gatveva \msCa}%

śreṣṭhinā ca tathety āha gacchāmaḥ sahitā vayam\thinspace{\dandab} \dontdisplaylinenum
            \var{\vb gacchāmaḥ\lem  \Ed; ga{\lost}mas \msCa}%

yatra prāpta\.m phala\.m tubhya\.m mokṣayāmo na sa\.mśayaḥ \veg\dontdisplaylinenum
            \var{\vc prāpta\.m\lem  \msCa; prāpta \Ed}%

rudra uvāca~{\dandab}\dontdisplaylinenum 

tam āruhya giri\.m sahya\.m mārgamāṇaḥ samantataḥ\thinspace{\danda} \dontdisplaylinenum
            \var{\vb °mānaḥ\lem  \msCa; °mānāḥ \Ed}%

vipulena tato d\textsubring{r}ṣṭo vānaraḥ plavagādhipaḥ \veg\dontdisplaylinenum
            \var{\vd plavagā°\lem  \Ed; plagā° \msCa}%

aya\.m sa vānaraśreṣṭho v\textsubring{r}kṣacchāyāsamāśritaḥ\thinspace{\dandab} \dontdisplaylinenum
            \var{\va vānara°\lem  \msCa; vānaraḥ \Ed}%
            \var{\vb °cchāyā°\lem  \Ed; °cchā\.myā° \msCa}%

mama puṇyabalenaiva d\textsubring{r}śyate 'dyāpi vānaraḥ \veg\dontdisplaylinenum

vānara kuru mitrārtha\.m sadyom\textsubring{r}tyur bhaven mama\thinspace{\dandab} \dontdisplaylinenum

pūrvadatta\.m phalam anyad dehi vānara jīvaya \veg\dontdisplaylinenum
            \var{\vd vānara jīvaya\lem  \msCa; vā na ca jīvaye \Ed}%

vānara uvāca~{\dandab}\dontdisplaylinenum 

gandharveṇa mama datta\.m phala\.m datta\.m tu te mayā\thinspace{\danda} \dontdisplaylinenum

punar anyat katha\.m dāsye tatra gaccha yadīcchasi \veg\dontdisplaylinenum

vipula uvāca~{\dandab}\dontdisplaylinenum 

adattvā tat phala\.m tubhya\.m jīvitu\.m sa\.mśayo bhavet\thinspace{\danda} \dontdisplaylinenum

athavā tatra gacchāmo yatra citrarathaḥ svayam \veg\dontdisplaylinenum
            \var{\vc athavā tatra\lem  \Ed; a{\lost}{\lost}{\lost}tra \msCa}%

vānaraḥ punar evāha eva\.m kurvāmahe vayam\thinspace{\dandab} \dontdisplaylinenum

tataś citrarathāvāsam upagamyedam abravīt \veg\dontdisplaylinenum
            \var{\vc tataś\lem  \msCa; tatra \Ed}%

gandharvarāja kāryārthī tva\.m hy aha\.m punar āgataḥ\thinspace{\dandab} \dontdisplaylinenum
            \var{\vb tva\.m\lem  \msCa; tvat \Ed}%

pūrvadattaphala\.m tv anyad dehi mā\.m yadi śakyate \veg\dontdisplaylinenum 

gandharvarājovāca\thinspace{\dandab} \dontdisplaylinenum

sūryalokagataś cāsmi tena datta\.m phalottamam \danda\dontdisplaylinenum
            \var{\va gataś cāsmi\lem  \Ed; gata\uncl{ś cā}{\lost} \msCa}%
            \var{\vb tena datta\.m\lem  \Ed; {\lost}{\lost}{\lost}ttam \msCa}%

mayā datta\.m phala\.m tubhyam atyantasuh\textsubring{r}do 'si me \veg\dontdisplaylinenum
            \var{\vc datta\.m\lem  \corr; datta° \msCa\Ed}%

kuto 'nyat phalam ādāsye mama nāsti plavaṅgama\thinspace{\dandab} \dontdisplaylinenum
            \var{\va 'nyat phalam ādāsye\lem  \msCa; 'nyaphala dāsyāmi \Ed}%
            \var{\vb mama nāsti plavaṅgama\lem  \msCa; matto 'sti plavaṅgamaḥ \Ed}%

sūryaloka\.m gamiṣyāmas tatra yācasva bhāskaram \veg\dontdisplaylinenum
            \var{\vc gamiṣyāmas\lem  \msCa; gamiṣyāmi \Ed}%

gandharvenaivam uktas tu tathety āha plavaṅgamaḥ\thinspace{\dandab} \dontdisplaylinenum

sūryaloka\.m tataḥ prāptā gandharvādaya sarvaśaḥ \veg\dontdisplaylinenum
            \var{\vd °daya\lem  \conj; °dayas \msCa, °dayaḥ \Ed}%

gandharva uvāca~{\dandab}\dontdisplaylinenum 
            \var{\vo gandharva uvāca\lem  \corr;
                                 gandharva \uncl{uvā}{\lost} \msCa, gandharvarājovāca \Ed}%

kāryārthena punaḥ prāptas tvatsakāśa\.m khageśvara\thinspace{\danda} \dontdisplaylinenum

pūrvadattaphala\.m tv anyad dehi jīvam anāśaya \veg\dontdisplaylinenum
            \var{\vc tv anya°\lem  \msCa; stv anya° \Ed}%
            \var{\vd °nāśaya\lem  \msCa; °nāśayaḥ \Ed}%

sūrya uvāca~{\dandab}\dontdisplaylinenum 

somalokagataś cāsmi tena datta\.m phalottamam\thinspace{\danda} \dontdisplaylinenum

saphala\.m dattam evāsi suh\textsubring{r}datvān mayā tava \veg\dontdisplaylinenum
            \var{\vc °vāsi\lem  \msCa; °vābhiḥ \Ed}%
            \var{\vd suh\textsubring{r}datvān\lem  \msCa; sa ca dattvā \Ed}%

anyad dātu\.m na śaknomi gaccha somapurādya vai\thinspace{\dandab} \dontdisplaylinenum
            \var{\va anyad\lem  \Ed; anya \msCa}%
            \var{\vb °purādya\lem  \msCa; °parādya \Ed}%

ta\.m prārthayāvikalpena atriputra\.m graheśvaram \veg\dontdisplaylinenum
            \var{\vc °vikalpena\lem  \Ed; °\uncl{vika}{\lost}{\lost} \msCa}%
            \var{\vd °putra\.m\lem  \Ed; °putra° \msCa}%

rudra uvāca~{\dandab}\dontdisplaylinenum 
            \var{\vo rudra\lem  \msCa; maheśvara \Ed}%

gataḥ sūryāgrataḥ k\textsubring{r}tvā somaloka\.m tathaiva hi\thinspace{\danda} \dontdisplaylinenum
            \var{\va gataḥ\lem  \Ed; gata \msCa}%

uvāca sūryaḥ somāya kāraṇāpekṣayā śaśim \veg\dontdisplaylinenum
            \var{\vd śaśim\lem  \msCa; śaśi \Ed}%

soma uvāca~{\dandab}\dontdisplaylinenum 

kimartham āgato bhūyaḥ kartavya\.m tatra bhāskara\thinspace{\danda} \dontdisplaylinenum
            \var{\vb tatra\lem  \msCa; tava \Ed\oo
                 °kara\lem  \msCa; °karaḥ \Ed}%

phala\.m dātu\.m punas tv anyan muktvā tv anyat karomy aham \veg\dontdisplaylinenum
            \var{\vc anyan\lem  \eme; anya \msCa, anyat \Ed}%
            \var{\vd muktvā\lem  \msCa; muktā \Ed\oo
                 anyat ka°\lem  \eme; anyaṅ ka° \msCa\Ed}%

sūrya uvāca~{\dandab}\dontdisplaylinenum 

yadi śakya\.m phala\.m dehi anyan na prārthayāmy aham\thinspace{\danda} \dontdisplaylinenum
            \var{\vb anyan\lem  \msCa; anyān \Ed}%

na dattāsi phalam anyan mayā vaddhyo bhaviṣyasi \veg\dontdisplaylinenum
            \var{\vc phalam anyan\lem  \msCa; phala\.m manye \Ed}%
            \var{\vd vaddhyo\lem  \msCa; vaddho \Ed}%

soma uvāca~{\dandab}\dontdisplaylinenum 

āgama\.m tasya vakṣyāmi ś\textsubring{r}ṇuṣvāvahito bhava\thinspace{\danda} \dontdisplaylinenum

indreṇāsmi phala\.m datta\.m saphala\.m datta me bhavān \veg\dontdisplaylinenum

gatvaivendrasadas tv anyat prārthayāmaḥ sahaiva tu\thinspace{\dandab} \dontdisplaylinenum
            \var{\va gatvaivendra°\lem  \msCa; gandharvendra° \Ed}%

eva\.m kurma iti prāha gatvendrasadana\.m prati \veg\dontdisplaylinenum
            \var{\vc kurma\lem  \msCa; soma \Ed}%

somenendram uvāceda\.m phalakāmā ihāgatāḥ\thinspace{\dandab} \dontdisplaylinenum

pūrvadattaphalam anyad dehi śakra mamādya vai \veg\dontdisplaylinenum
            \var{\vd śakra\lem  \msCa; śaka \Ed}%

indra uvāca~{\dandab}\dontdisplaylinenum 

yad artham iha samprāptaḥ sa ca nāsti niśākara\thinspace{\danda} \dontdisplaylinenum
            \var{\vb °kara\lem  \msCa; °karaḥ \Ed}%

viṣṇuhastān mayā prāptam ekam eva phala\.m śubham \veg\dontdisplaylinenum

sarva eva hi gacchāmo viṣṇuloka\.m graheśvara\thinspace{\dandab} \dontdisplaylinenum

sarva evopajagmus te phalārtha\.m madhusūdanam \veg\dontdisplaylinenum
            \var{\vc °jagmu°\lem  \Ed; °ñjagmu° \msCa}%

evam uktvā gatāḥ sarve devarājapurask\textsubring{r}tāḥ\thinspace{\dandab} \dontdisplaylinenum
            \var{\va °ktvā\lem  \msCa; °ktā \Ed}%

muhūrtenaiva samprāptā viṣṇuloka\.m yaśasvini \veg\dontdisplaylinenum

upas\textsubring{r}tya tata indraḥ praṇipatya janārdanam\thinspace{\dandab} \dontdisplaylinenum

sarveṣām uparodhena prārthayāmi yaśodhara \veg\dontdisplaylinenum
            \var{\vd °dhara\lem  \msCa; °dharam \Ed}%

viṣṇur uvāca~{\dandab}\dontdisplaylinenum 
            \var{\vo viṣṇur u°\lem  \msCa; viṣṇu u° \Ed}%

pūrvadattaphalasyārthe tac ca sarvam ihāgatāḥ\thinspace{\danda} \dontdisplaylinenum
            \var{\va °datta°\lem  \msCa; °datta\.m \Ed\oo
                 °rthe\lem  \msCa; °rthi \Ed}%

na śaknomi phala\.m dātu\.m ki\.m vā tv anyat karomy aham \veg\dontdisplaylinenum
            \var{\vd tv anyat\lem  \eme; tv anyaṅ \msCa\Ed}%

indra uvāca~{\dandab}\dontdisplaylinenum 

brahmāṇḍam api bhettu\.m tva\.m śaknoṣi garuḍadhvaja\thinspace{\danda} \dontdisplaylinenum
            \var{\va bhettu\.m tva\.m\lem  \msCa; bhartu\.mtva\.m \Ed}%

aśakya\.m tava nāstīti jānāmi puruṣottama \veg\dontdisplaylinenum
            \var{\vd °ttama\lem  \msCa; °ttamam \Ed}%

evam uktvā punar viṣṇuḥ pratyuvāca purandaram\thinspace{\dandab} \dontdisplaylinenum

phalam eka\.m parityajya sarva\.m śaknomi kauśika \veg\dontdisplaylinenum

upāyo 'tra pravakṣyāmi āgama\.m ś\textsubring{r}ṇu gopate\thinspace{\dandab} \dontdisplaylinenum

brahmaṇā ca mama datta\.m tat phalaika\.m purandara \veg\dontdisplaylinenum
            \var{\vc mama\lem  \msCa; mamā° \Ed}%

mayā dattaphala\.m tv eka\.m kim anyad dātum icchasi\thinspace{\dandab} \dontdisplaylinenum
            \var{\vb °cchasi\lem  \Ed; °cchati \msCa}%

prārthayāmo 'tra gatvaika\.m parameṣṭhiprajāpatim \veg\dontdisplaylinenum 
            \var{\vc prārthayāmo 'tra gatvaika\.m\lem  \msCa; prārthayā ca gatvaiva\.m \Ed}%
            \var{\vd °ṣṭhi°\lem  \msCa; °ṣṭhi\.m \Ed}%

tavoparādhād devendra prārthayāmi pitāmaham\thinspace{\dandab} \dontdisplaylinenum
            \var{\va tavo°\lem  \msCa; tato° \Ed}%

evam uktvā gatāḥ sarve purask\textsubring{r}tya janārdanam \veg\dontdisplaylinenum
            \var{\vc gatāḥ\lem  \msCa; gatā \Ed}%

indraḥ somaś ca sūryaś ca gandharvo vānaras tathā\thinspace{\dandab} \dontdisplaylinenum
            \var{\va sūryaḥ śaśī caiva\lem  \msCa; somaś ca sūryaś ca \Ed}%

vipulaḥ śreṣṭhikaś caiva rājadūtadvaya\.m tathā \veg\dontdisplaylinenum
            \var{\vd °dvaya\.m\lem  \Ed; °dvayas \msCa}%

brahmaloka\.m muhūrtena prāptavān surasundari\thinspace{\dandab} \dontdisplaylinenum

d\textsubring{r}ṣṭvā brahmasado ramya\.m sarvakāmaparicchadam \veg\dontdisplaylinenum
            \var{\vc °sado\lem  \msCa; °sada\.m \Ed}%

anekāni vicitrāṇi ratnāni vividhāni ca\thinspace{\dandab} \dontdisplaylinenum

mandārataruśobhāni vaidūryamaṇikuṭṭimam \veg\dontdisplaylinenum
            \var{\vc °taru°\lem  \Ed; °tala° \msCa}%
            \var{\vd °kuṭṭimam\lem  \Ed; °kuṭimām \msCa}%

pravālamaṇistambhāni vajrakāñcanavedikām\thinspace{\dandab} \dontdisplaylinenum
            \var{\vb °vedikām\lem  \msCa; °vedikā \Ed}%

pravālasphāṭiko jāla indranīlagavākṣakaḥ \veg\dontdisplaylinenum
            \var{\vc °sphāṭiko jāla\lem  \msCa; °sphaṭiko jālā \Ed}%

d\textsubring{r}śyate vipulas tatra nānāv\textsubring{r}kṣa manoramāḥ\thinspace{\dandab} \dontdisplaylinenum
            \var{\va d\textsubring{r}śyate\lem  \conj; paśyate \msCa, d\textsubring{r}śyante \Ed\oo
                 vipula°\lem  \msCa; vipulā° \Ed}%

puṣpānāmitav\textsubring{r}kṣāgrāḥ phalānāmitakā bhavet \veg\dontdisplaylinenum
            \var{\vc puṣpā°\lem  \msCa; puṣpa° \Ed\oo
                 °grāḥ\lem  \eme; °grā \msCa, °yā \Ed}%
            \var{\vd phalānāmitakā\lem  \msCa; phalanāmitakā\.m \Ed}%

sarve ratnamayā v\textsubring{r}kṣāḥ sarve ratnamaya\.m jalam\thinspace{\dandab} \dontdisplaylinenum

v\textsubring{r}kṣagulmalatāvallī kandamūlaphalāni ca \veg\dontdisplaylinenum

sarve ratnamayā d\textsubring{r}ṣṭā vipulo vipulekṣaṇaḥ\thinspace{\dandab} \dontdisplaylinenum
            \var{\va sarve\lem  \Ed; sarvai \msCa}%

anekabhauma\.m prāsāda\.m muktādāmavibhūṣitam \veg\dontdisplaylinenum

apsarogaṇakoṭībhiḥ sarvābharaṇabhūṣitam\thinspace{\dandab} \dontdisplaylinenum

vimānakoṭikoṭīśa\.m sarvakāmasamanvitam \veg\dontdisplaylinenum
            \var{\vcd \om\ \Ed}%

brahmalokasabhā ramyā sūryakoṭisamaprabhā\thinspace{\dandab} \dontdisplaylinenum

tatra brahmā sukhāsīno nānāratnopaśobhite \veg\dontdisplaylinenum

caturmūrtiś caturvaktraś caturbāhuś{}caturbhujaḥ\thinspace{\dandab} \dontdisplaylinenum

caturvedadharo devaś caturāśramanāyakaḥ \veg\dontdisplaylinenum

caturvedāv\textsubring{r}tas tatra mūrtimantam upāsate\thinspace{\dandab} \dontdisplaylinenum

gāyatrī vedamātā ca sāvitrī ca surūpiṇī \veg\dontdisplaylinenum

vyāh\textsubring{r}tiḥ praṇavaś caiva mūrtimān samupāsate\thinspace{\dandab} \dontdisplaylinenum
            \var{\va praṇavaś caiva\lem  \Ed; praṇa\uncl{va}{\lost}va \msCa}%

vauṣaṭkāro vaṣaṭkāro namaskāraḥ sa mūrtimān \veg\dontdisplaylinenum

śrutiḥ sm\textsubring{r}tiś ca nītiś ca dharmaśāstra\.m samūrtimān\thinspace{\dandab} \dontdisplaylinenum
            \var{\vb °śāstra\.m samūrtimat\lem  \msCa; °śāstrasamūrtimān \Ed}%

itihāsaḥ purāṇa\.m ca sā\.mkhyayogaḥ patañjalam \veg\dontdisplaylinenum
            \var{\vc purāṇa\.m\lem  \msCa; purāṇaś \Ed}%
            \var{\vd °jalam\lem  \msCa; °jali \Ed}%

āyurvedo dhanurvedo vedo gāndharva-m-eva ca\thinspace{\dandab} \dontdisplaylinenum
            \var{\vb gāndharva-m-eva\lem  \msCa; gāndharvar eva \Ed}%

arthavedo 'nyavedāś ca mūrtimān samupāsite \veg\dontdisplaylinenum
            \var{\vc arthavedo 'nyavedāś ca\lem  \Ed; arthavedānyavedañ ca \msCa}%

tato brahmā samutthāya abhigamya janārdinam\thinspace{\dandab} \dontdisplaylinenum

gā\.m ca argha\.m ca dattvaivam āsyatām iti cābravīt \veg\dontdisplaylinenum
            \var{\vc argha\.m\lem  \msCa; arghya\.m \Ed}%

maṇiratnamaye divye āsane garuḍadhvajaḥ\thinspace{\dandab} \dontdisplaylinenum

devarājo raviḥ somo gandharvaḥ plavageśvaraḥ \veg\dontdisplaylinenum
            \var{\vc raviḥ somo\lem  \msCa; śaśī sūryo \Ed}%

vipulaś ca mahāsattva āsyatā\.m ratna-āsane\thinspace{\dandab} \dontdisplaylinenum
            \var{\vb °āsane\lem  \msCa; °śāśane \Ed}%

sādhu bho vipulaśreṣṭha sādhu bho vipula\.m tapaḥ \veg\dontdisplaylinenum
            \var{\vd vipula\.m tapaḥ\lem  \Ed; \uncl{vi}{\lost}{\lost}{\lost}paḥ \msCa}%

sādhu bho vipulaprājña sādhu bho vipulaśriya\thinspace{\dandab} \dontdisplaylinenum
            \var{\vb °śriya\lem  \msCa; °śriyaḥ \Ed}%

toṣitāḥ sma vaya\.m sarve brahmaviṣṇumaheśvarāḥ \veg\dontdisplaylinenum
            \var{\vc toṣitāḥ\lem  \msCa; toṣitā \Ed}%

ādityā vasavo rudrā sādhyāśvinau marut tathā\thinspace{\dandab} \dontdisplaylinenum
            \var{\vb sādhyāśvinau\lem  \conj; sādhyāśvinyau \msCa, sādhyā yakṣo \Ed}%

bhuṅkṣva bhogān yathotsāha\.m mama loke yathāsukham \veg\dontdisplaylinenum
            \var{\vc bhuṅkṣva\lem  \msCa; bhu\.mkṣa \Ed}%

iya\.m vimānakoṭīṇā\.m tavārthāyopakalpitā\thinspace{\dandab} \dontdisplaylinenum
            \var{\vb kalpitā\lem  \msCa; kalpitān \Ed}%

sahasrāṇā\.m sahasrāṇi apsarā kāmarūpiṇī \veg\dontdisplaylinenum
            \var{\vd °rūpiṇī\lem  \msCa; °rūpiṇi \Ed}%

tavārthīyopasarpanti sarvāla\.mkārabhūṣitāḥ\thinspace{\dandab} \dontdisplaylinenum
            \var{\va v°rthīyo°\lem  \msCa; °rtheyo° \Ed}%

yāvat kalpasahasrāṇi parārdhāni tapodhana \danda\dontdisplaylinenum
            \var{\vd °dhana\lem  \msCa; °dhanāḥ \Ed}%

yatra yatra prayāsitva\.m tatra tatropabhujyatām \veg\dontdisplaylinenum

maheśvara uvāca~{\dandab}\dontdisplaylinenum 

iti śrutvā vacas tasya vipulo vipulekṣaṇaḥ\thinspace{\danda} \dontdisplaylinenum

vepamāno bhayatrasta aśrupūrṇākulekṣaṇaḥ \veg\dontdisplaylinenum
            \var{\vc bhayatrasta\lem  \Ed; bhayas tatra \msCa}%

praṇamya śirasā bhūmau praṇipatya punaḥ punaḥ\thinspace{\dandab} \dontdisplaylinenum

uvāca madhura\.m vākya\.m brahmaloke pitāmaham \veg\dontdisplaylinenum
            \var{\vd °loke\lem  \Ed; °loka \msCa}%

vipula uvāca~{\dandab}\dontdisplaylinenum 

bhagavan sarvalokeśa sarvalokapitāmaha\thinspace{\danda} \dontdisplaylinenum

svapnabhūtam ivāścarya\.m paśyāmi tridaśeśvara \veg\dontdisplaylinenum

sm\textsubring{r}tibhra\.mśaś ca me jāto buddhir jātāndhacetanā\thinspace{\dandab} \dontdisplaylinenum
            \var{\vb jātāndhacetanā\lem  \msCa; jāto 'ndhacetanaḥ \Ed}%

mūḍho 'ha\.m tvā\.m katha\.m staumi jñānātīta\.m parāt param \veg\dontdisplaylinenum
            \var{\vcd \om\ \msCa}%

\ujvers\nemsloka 
tubhya\.m trailokyabandho bhava mama śaraṇa\.m trāhi sa\.msāraghoram
\dontdisplaylinenum
            \var{\va tubhya\.m\lem  \msCa; namas \Ed}%

\nemslokab 
bhīto 'ha\.m garbhavāsāj jaramaraṇabhayāt trāhi mā\.m mohabandhāt \danda\dontdisplaylinenum
            \var{\vb °jara°\lem  \msCa; °janu° \Ed\oo
                 °bhayāt\lem  \Ed; bhayan \msCa}%

\nemslokac 
! nitya\.m rāgādhivāsam aniyatavapuṣa\.m trāhi mā\.m kālapāśāt
\dontdisplaylinenum
            \var{\vc rogā°\lem  \msCa; °rāgā° \Ed}%

\nemslokad 
tirya\.m cānyonyabhakṣa\.m bahuyugaśataśas trāhi mohāndhakārāt \veg\dontdisplaylinenum
            \var{\vd tirya\.m\lem  \msCa; tiryaś \Ed}%

\ujvers\nemsloka 
śrutvaivovāca brahmā vipulamati punar mānayitvā yathāvat
\dontdisplaylinenum
            \var{\va śrutvaivovāca\lem  \msCa; śrutvaiva vāca \Ed\oo
                 °mati\lem  \Ed; °matiḥ \msCa\oo
                 mānayitvā\lem  \msCa; mānaya\.mvā \Ed}%

\nemslokab 
! āhūta samplavante bhaviṣyasi tava me janmalobho na bhūyaḥ \danda\dontdisplaylinenum
            \var{\vb āhūta\lem  \msCa; ābhūta \Ed\oo
                 bhaviṣyasi\lem  \msCa; avipali \Ed\oo
                 me janmalobho na\lem  \msCa; yajanmalābhānu \Ed}%

\nemslokac 
garbhāvāsannacatvanna ca punamaraṇa\.m kleśam āyāsapūrṇam
\dontdisplaylinenum
            \var{\vc °vāsannacatvanna\lem  \msCa; °vāsānubandha\.m na \Ed\oo
                 puna\lem  \Ed; punar \msCa}%

\nemslokad 
chittvā mohāndhaśatru\.m vrajasi ca parama\.m brahmabhūyatvam esi \veg\dontdisplaylinenum

\vers

maheśvara uvāca~{\dandab}\dontdisplaylinenum 

brahmaṇā evam uktas tu viṣṇunā prabhaviṣṇunā\thinspace{\danda} \dontdisplaylinenum

eva\.m bhavatu bhadra\.m vo yathovāca pitāmahaḥ \veg\dontdisplaylinenum

indreṇa raviṇā caiva somena ca punaḥ punaḥ\thinspace{\dandab} \dontdisplaylinenum
            \var{\va raviṇā\lem  \msCa; śaśinā \Ed}%
            \var{\vb somena\lem  \msCa; sūryeṇa \Ed}%

sādhyādityair marudrudrair viśvebhir vasavais tathā \veg\dontdisplaylinenum
            \var{\vb viśvebhir\lem  \Ed; viśveśvi \msCa}%

aho tapaḥ phala\.m divya\.m vipulasya mahātmanaḥ\thinspace{\dandab} \dontdisplaylinenum

svaśarīra\.m diva\.m prāpta\.m śraddhayā tithipūjayā \veg\dontdisplaylinenum
            \var{\vc sva°\lem  \msCa; sa° \Ed}%
            \var{\vd °pūjayā\lem  \msCa; °pūjanāt \Ed}%

evam ādīny anekāni vipule parikīrtitam\thinspace{\dandab} \dontdisplaylinenum

brahmāṇa\.m punar evāha viṣṇur viśvajagatprabhuḥ \veg\dontdisplaylinenum


\jump
\begin{center}
\ketdanda iti v\textsubring{r}ṣasārasa\.mgrahe vipulopākhyāno nāmādhyāyo dvādaśamaḥ\ketdanda
\end{center}
\dontdisplaylinenum\vers 
            \var{{\normalfont Colophon: } nāmādhyāyo dvādaśamaḥ\lem  \msCa; nāma dvādaśo 'dhyāyaḥ \Ed}%

\vers
\bekveg\szamveg\vfill\phpspagebreak\szam\bek\versno=0\fejno=13
\thispagestyle{empty}



\alfejezet{\textbf{13 garbhotpattiḥ}}\jump\jump 
devy uvāca~{\dandab}\dontdisplaylinenum 

ahi\.msātithyakānā\.m ca śruto dharmaḥ suvistaraḥ\thinspace{\danda} \dontdisplaylinenum

ki\.m na kurvanti manujāḥ sukhopāya\.m mahat phalam \veg\dontdisplaylinenum

svaśarīrasthito yajñaḥ svaśarīre sthita\.m tapaḥ\thinspace{\dandab} \dontdisplaylinenum

svaśarīre sthita\.m tīrtha\.m śruto vistarato mayā \veg\dontdisplaylinenum

kimartha\.m bhagavan brūhi sukhopāya\.m mahat phalam\thinspace{\dandab} \dontdisplaylinenum

ki\.m niv\textsubring{r}ttās tu deveśa \textsubring{r}ṣidaivatamānuṣāḥ \veg\dontdisplaylinenum

mahādeva uvāca~{\dandab}\dontdisplaylinenum 
            \var{\vo mahādeva\lem  \msCa; bhagavān \Ed}%

adya p\textsubring{r}ṣṭena kathita\.m gopita\.m \textsubring{r}ṣi sundari\thinspace{\danda} \dontdisplaylinenum

mānuṣāṇā\.m hitārthāya tava ca varavarṇini \veg\dontdisplaylinenum

adyaprabh\textsubring{r}ti deveśi khyātir loke bhaviṣyati\thinspace{\dandab} \dontdisplaylinenum

dhanyā eva\.m cariṣyanti adhanyā na ramanti tam \veg\dontdisplaylinenum

triguṇena tu bandhena baddhā pāśad\textsubring{r}ḍhena tu\thinspace{\dandab} \dontdisplaylinenum

tenārthena ramanty atra jānanto 'pi vimohitāḥ \veg\dontdisplaylinenum

devy uvāca~{\dandab}\dontdisplaylinenum 

ki\.m vā triguṇabandheti brūhi sa\.mśayachedaka\thinspace{\danda} \dontdisplaylinenum

adyāpi mama deveśa mohotpannas tribandhanaiḥ \veg\dontdisplaylinenum

bhagavān uvāca~{\dandab}\dontdisplaylinenum 

prāk\textsubring{r}ta\.m vaik\textsubring{r}ta\.m caiva dakṣiṇābandham eva ca\thinspace{\danda} \dontdisplaylinenum

etenaiva tu bandhena baddhāḥ varṇāśramāḥ sadā \veg\dontdisplaylinenum

jñānahīnā nivartante parama\.m prāpya tatparam\thinspace{\dandab} \dontdisplaylinenum

iṣṭastrīṇā nivartante dhanadhānyasamuccaye \danda\dontdisplaylinenum

snehād āk\textsubring{r}ṣya manasā\.m bandhaḥ prāk\textsubring{r}ta ucyate \veg\dontdisplaylinenum

yogayuktena manasā yad yad aiśvaryam āpyate\thinspace{\dandab} \dontdisplaylinenum

tac ca vaik\textsubring{r}tabandhas tu yadi tatrānurajyate \veg\dontdisplaylinenum

ārāmodyānavāpīṣu dānakratuphaleṣu ca\thinspace{\dandab} \dontdisplaylinenum

āśaktamanasā vācā dakṣiṇābandhaḥ kathyate \veg\dontdisplaylinenum

anenaiva tu pāśena baddhāvānaravad yathā\thinspace{\dandab} \dontdisplaylinenum

mokṣita\.m na ca śaknoti itaś cetaś ca dhāvati \veg\dontdisplaylinenum

devāsuramanuṣyeṣu tiryeṣu narakeṣu ca\thinspace{\dandab} \dontdisplaylinenum

bhramante cakrayantreva ? yāvat tattva\.m na vindati \veg\dontdisplaylinenum

garbhavāsaparikleśau janmam\textsubring{r}tyu punaḥ punaḥ\thinspace{\dandab} \dontdisplaylinenum

vyādhiḥ śokabhayāyāsa cintayā jarayā hataḥ \veg\dontdisplaylinenum

devy uvāca~{\dandab}\dontdisplaylinenum 

garbhotpattiḥ katha\.m deva yogī labhati kīd\textsubring{r}śīm\thinspace{\danda} \dontdisplaylinenum

kīd\textsubring{r}śa\.m labhate garbhaḥ śrotu\.m naḥ pratyudīryatām \veg\dontdisplaylinenum

bhagavān uvāca~{\dandab}\dontdisplaylinenum 

ś\textsubring{r}ṇu devi pravakṣyāmi garbhotpattir yathākramam\thinspace{\danda} \dontdisplaylinenum

yathā sa\.mśayaviccheda\.m bhaviṣyasi varānane \veg\dontdisplaylinenum

akṣarāt prabhavo brahmā karmabaddhasamudbhavam\thinspace{\dandab} \dontdisplaylinenum

karmato yajñaprabhavo yajñato dhūmasambhavaḥ \veg\dontdisplaylinenum

parjanyād annam utpattir annād bhūtāni jajñire\thinspace{\dandab} \dontdisplaylinenum

annād rasasamutpatti rasāc choṇitasambhavaḥ \veg\dontdisplaylinenum

śoṇitāt - mā\.msa-m-utpatti mā\.msād medasamudbhavaḥ\thinspace{\dandab} \dontdisplaylinenum

medaso 'sthīni jāyante asthibhyo majjasambhavaḥ \veg\dontdisplaylinenum

majjāyās tu bhavec chukra\.m naraḥ śukrasamudbhavaḥ\thinspace{\dandab} \dontdisplaylinenum

śukraśoṇitasa\.myogād garbhotpattis tataḥ sm\textsubring{r}taḥ \veg\dontdisplaylinenum

agnisomātmaka\.m devi śarīradvayadhātutaḥ\thinspace{\dandab} \dontdisplaylinenum

somadhātusm\textsubring{r}ta\.m śukram agnidhāturajasm\textsubring{r}tam \danda\dontdisplaylinenum

agnisomāśraya\.m devi śarīram iti sa\.mjñitam \veg\dontdisplaylinenum

māsī māsī \textsubring{r}tuḥ strīṇā\.m bhavatīha na sa\.mśayaḥ\thinspace{\dandab} \dontdisplaylinenum

\textsubring{r}tukāle prasarpyeta na sukhārtha\.m varānane \veg\dontdisplaylinenum

putrakāmaprayuñjīta dharmārthaś ca yaśasvini\thinspace{\dandab} \dontdisplaylinenum

pumān strīpu\.m prayuñjīta araṇī bahutāśanaḥ \veg\dontdisplaylinenum

pumān śukrādhiko jñeyaḥ kanyā raktādhikā bhavet\thinspace{\dandab} \dontdisplaylinenum

samaśukre ca rakte ca sa ca jāyen napu\.msakaḥ \veg\dontdisplaylinenum


\alalfejezet{dviyamā triyamā ca gurviṇī}
devy uvāca~{\dandab}\dontdisplaylinenum 

dviyamā triyamā caiva katha\.m jāyeta gurviṇī\thinspace{\danda} \dontdisplaylinenum

katha\.m strīdviyamā jāyet katha\.m vā puruṣadvayam \veg\dontdisplaylinenum

bhagavān uvāca~{\dandab}\dontdisplaylinenum 

raktādhikā sm\textsubring{r}tā kanyā jāyate varavarṇini\thinspace{\danda} \dontdisplaylinenum

vāyunā ca dvidhā bhinnā kanyakadviyamā sm\textsubring{r}tā \veg\dontdisplaylinenum

śukrādhikās tu puruṣa dvidhā bhinnānilena tu\thinspace{\dandab} \dontdisplaylinenum

dviyamā puruṣo jñeyā triyamās tu tridhā k\textsubring{r}te \veg\dontdisplaylinenum

\textsubring{r}tusnātā yadā nārī yadi garbhādi g\textsubring{r}hyati\thinspace{\dandab} \dontdisplaylinenum

prathame ca dvitīye ca t\textsubring{r}tīye ca sa jīvati \veg\dontdisplaylinenum

sameṣu janayet putraḥ kanyakā viṣame dine\thinspace{\dandab} \dontdisplaylinenum

ṣaṣṭyāṣṭamau ca daśamī dvādaśī ca pumān bhavet \veg\dontdisplaylinenum

pañcamī saptamī caiva navamekādaśī striyaḥ\thinspace{\dandab} \dontdisplaylinenum

samarakte ca śukre ca śyāmaḥ sa\.mjāyate pumān \veg\dontdisplaylinenum

rudhira\.m tv ekarātreṇa kalala\.m pratipadyate\thinspace{\dandab} \dontdisplaylinenum

kalala\.m pañcarātreṇa arbudatva\.m prajāyate \veg\dontdisplaylinenum

arbudaḥ saptarātreṇa mā\.msapeśī samudbhavaḥ\thinspace{\dandab} \dontdisplaylinenum

dvitīya\.m saptarātreṇa tat sarva\.m mā\.msaśoṇitam \veg\dontdisplaylinenum

t\textsubring{r}tīya\.m saptarātreṇa h\textsubring{r}daya\.m jāyate tataḥ\thinspace{\dandab} \dontdisplaylinenum

tataḥ sarvāṇi gātrāṇi śiraś caivopajāyate \veg\dontdisplaylinenum

h\textsubring{r}daye jāyamāne tu mūrcchāntandrirarocakaḥ\thinspace{\dandab} \dontdisplaylinenum

striyāḥ dhardiḥ praśekaś ca daurbalya\.m copajāyate \veg\dontdisplaylinenum

tasyā hi h\textsubring{r}daya\.m nārī yadi bhakṣyati ki\.mcana\thinspace{\dandab} \dontdisplaylinenum

bhakṣya\.m lohya\.m tathā peyam upabhogās tathāyayat \veg\dontdisplaylinenum

śayanāsanayānāni vastrāṇy ābharaṇāni ca\thinspace{\dandab} \dontdisplaylinenum

yad yad ākā\.mkṣate ki\.mcit tat tad āsyai pradāpayet \veg\dontdisplaylinenum
            \paral{\textit{ \vo {\normalfont Cf.\ MBh 13.40.12:}
        śayyāsanam ala\.mkāram annapānam anāryatām
        durvāgbhāva\.m rati\.m caiva dadau stgrībhyaḥ prajāpatiḥ}}

nāyā sa\.mkārayec cāsyā na caivam avamānayet\thinspace{\dandab} \dontdisplaylinenum

mukham āpāṇḍura\.m snigdha\.m kapolastanakeśayoḥ \veg\dontdisplaylinenum

śarīraś ca śriyā jaṣṭu\.m pīnoruśroṇi vakṣasam\thinspace{\dandab} \dontdisplaylinenum

liṅgerebhir vijānīyā\.m garbhe jīva\.m pratiṣṭhitam \veg\dontdisplaylinenum

caturthe saptarātreṇa śiraś caivopajāyate\thinspace{\dandab} \dontdisplaylinenum

pañcamasaptarātreṇa grīvā tatropajāyate \veg\dontdisplaylinenum

ṣaṣṭhamasaptarātreṇa skandhagātra\.m prajāyate\thinspace{\dandab} \dontdisplaylinenum

saptamasaptarātreṇa p\textsubring{r}ṣṭhava\.mśa prajāyate \veg\dontdisplaylinenum

aṣṭamasaptarātreṇa pāṇī jāyate cobhayau\thinspace{\dandab} \dontdisplaylinenum

saptarātra\.m nava prāpya jāyate h\textsubring{r}di pañjaram \veg\dontdisplaylinenum

daśame saptarātre ca pādau jāyate cobhau\thinspace{\dandab} \dontdisplaylinenum

udaraś copajāyeta saptaikādaśarātrike \veg\dontdisplaylinenum

dvādaśasaptarātreṇa kukṣipārśveḥ prajāyate\thinspace{\dandab} \dontdisplaylinenum

saptatraidaśarātreṇa kuṭisutropajāyate \veg\dontdisplaylinenum

navaty aṣṭamarāteṇa jāyate sūtravi\.mśati\thinspace{\dandab} \dontdisplaylinenum

saptapañcadaśāhena sarvamedaḥ prajāyate \veg\dontdisplaylinenum

ṣoḍaśasaptarātreṇa athisarvāṇi jāyate\thinspace{\dandab} \dontdisplaylinenum

saptasaptadaśāhena jāyate snāyubandhanam \veg\dontdisplaylinenum

saptamāṣṭādaśāhena jāyate mukhamaṇḍalam\thinspace{\dandab} \dontdisplaylinenum

saptonavi\.mśarātreṇa ghrāṇava\.mśaḥ prajāyate \veg\dontdisplaylinenum

saptavi\.mśatirātreṇa naitranāli\.m prajāyate\thinspace{\dandab} \dontdisplaylinenum

saptaikavi\.mśarātreṇa karṇayugma\.m prajāyate \veg\dontdisplaylinenum

dvāvi\.mśasaptarātreṇa jāyate dvau bhruvau tataḥ\thinspace{\dandab} \dontdisplaylinenum

saptatrivi\.mśarātreṇa gaṇḍayugma\.m prajāyate \veg\dontdisplaylinenum

caturvi\.mśatisaptāhe oṣṭhayugma\.m prajāyate\thinspace{\dandab} \dontdisplaylinenum

pañcavi\.mśatisaptāhe jihvā jāyate sundari \veg\dontdisplaylinenum

ṣaḍvi\.mśasaptarātreṇa dantapaṅkti prajāyate\thinspace{\dandab} \dontdisplaylinenum

unavi\.mśatisaptāhe jāyate ca tvag eva ca \veg\dontdisplaylinenum

tri\.mśatasaptarātreṇa jāyate nābhimaṇḍalam\thinspace{\dandab} \dontdisplaylinenum

saptaikatri\.mśarātreṇa sarvarandhra\.m prajāyate \veg\dontdisplaylinenum

dvātri\.mśasaptarātreṇa nakhavi\.mśati jāyate\thinspace{\dandab} \dontdisplaylinenum

tritri\.mśasaptarātreṇa sarve sandhiḥ prajāyate \veg\dontdisplaylinenum

pañcatri\.mśati saptāhe sarvamarma prajāyate\thinspace{\dandab} \dontdisplaylinenum

ṣaḍtri\.mśasaptarātreṇa vedanā copajāyate \veg\dontdisplaylinenum

saptatri\.mśatisaptāhe īrṣyādveṣaḥ prajāyate\thinspace{\dandab} \dontdisplaylinenum

aṣṭatri\.mśatisaptāhe pañcātmakasamanvitam \veg\dontdisplaylinenum

sarvāṅgam aṅgasampūrṇaḥ paripakva(ḥ) sa tiṣṭhati\thinspace{\dandab} \dontdisplaylinenum

mātusvāśitapītaś ca nābhisūtrāganena tu \veg\dontdisplaylinenum

ajātasyopadhāryante garbhasthasyaiva jantavaḥ\thinspace{\dandab} \dontdisplaylinenum

tataḥ praviśate dehe nidrāsvapna yathā tathā \veg\dontdisplaylinenum

nopalabhyati sūkṣmatvād araṇy agnir yathā tathā\thinspace{\dandab} \dontdisplaylinenum

garbhodakena siktāṅgajarāyā pariveṣṭitaḥ \veg\dontdisplaylinenum

jāti smarati tatrastho jantuś cetaḥsamanvitaḥ\thinspace{\dandab} \dontdisplaylinenum

m\textsubring{r}taś cāha\.m punarjāto bhūyaś caiva punarm\textsubring{r}taḥ \veg\dontdisplaylinenum

sthāvarāṇā\.m sahasreṣu jāto 'smi vividheṣu ca\thinspace{\dandab} \dontdisplaylinenum

caturvarṇavivarṇeṣu mānuṣeṣu sahasraśaḥ \veg\dontdisplaylinenum

sāmprata\.m ca punar garbhaḥ kleśaḥ prāptaḥ suduḥsahaḥ\thinspace{\dandab} \dontdisplaylinenum

idānī\.m jātamātro 'ha\.m sa\.mskāraiś cāpi sa\.msk\textsubring{r}taḥ \veg\dontdisplaylinenum

yogam evābhisevāmi sā[\.m]khya\.m vā pañcavi\.mśakam\thinspace{\dandab} \dontdisplaylinenum

yatra janmajarā nāsti yatra m\textsubring{r}tyuś ca nāsti vai \veg\dontdisplaylinenum

yatra brahma para\.m vedya\.m cariṣyāmi yatavrataḥ\thinspace{\dandab} \dontdisplaylinenum

evam ādīny anekāni cintayitvā punaḥ punaḥ \veg\dontdisplaylinenum

yāvat tiṣṭhati garbhastho jāti smarati pūrvikām\thinspace{\dandab} \dontdisplaylinenum

tato jāyati kaṣṭena mahākleśena mānavaḥ \veg\dontdisplaylinenum

yoniyantrasutīvreṇa pīḍyamānasuduḥkhitaḥ\thinspace{\dandab} \dontdisplaylinenum

jātamātrosm\textsubring{r}tibhra\.mśo bhavatīha acetaneḥ \veg\dontdisplaylinenum

māyāmudgaratīvreṇa hataḥ ki\.m śubham ācaret\thinspace{\dandab} \dontdisplaylinenum

eṣa garbhasamutpattiḥ kathito 'smi varānane \danda\dontdisplaylinenum

duḥkhasa\.msārapraśama\.m ki\.m bhūyaḥ śrotum icchasi \veg\dontdisplaylinenum


\jump
\begin{center}
\ketdanda iti v\textsubring{r}ṣasārasa\.mgrahe garbhotpattir nāma trayadaśo ! 'dhyāyaḥ\ketdanda
\end{center}
\dontdisplaylinenum\vers 
\bekveg\szamveg\vfill\phpspagebreak\szam\bek\versno=0\fejno=14
\thispagestyle{empty}


\vers


\alfejezet{\textbf{14 praśnavyākaraṇam}}\jump\jump 
devy uvāca~{\dandab}\dontdisplaylinenum 

atidīrghātihrasvaś ca pumān kenopajāyate\thinspace{\danda} \dontdisplaylinenum

atigauro 'tik\textsubring{r}ṣṇaś ca naro bhavati ki\.m prabho \veg\dontdisplaylinenum

bhagavān uvāca~{\dandab}\dontdisplaylinenum 

g\textsubring{r}hītagarbhā yā nārī nityam uttānaśālinī\thinspace{\danda} \dontdisplaylinenum

prasāritavimuktātmā so 'tidīrghaḥ prajāyate \veg\dontdisplaylinenum

g\textsubring{r}hītagarbhā yā nārī śete sa\.mkucitā sadā\thinspace{\dandab} \dontdisplaylinenum

raso 'nnādīni kaṭuka\.m sevanāḥ hrasva jāyate \veg\dontdisplaylinenum

g\textsubring{r}hītagarbhā yā nārī nitya\.m kṣīropasevitā\thinspace{\dandab} \dontdisplaylinenum

varakodravaśālī ca bhuktā cāpi yavodanam \veg\dontdisplaylinenum

śuklavastrasrajā yuktā sātigaura\.m prajāyate\thinspace{\dandab} \dontdisplaylinenum

g\textsubring{r}hītagarbhā yā nārī bāladhānyāni sevate \veg\dontdisplaylinenum

k\textsubring{r}ṣṇakodravatailādi māṣak\textsubring{r}ṣṇayavodanam\thinspace{\dandab} \dontdisplaylinenum

k\textsubring{r}ṣṇavastrasrajādīni tasyāḥ k\textsubring{r}ṣṇaḥ prajāyate \veg\dontdisplaylinenum

devy uvāca~{\dandab}\dontdisplaylinenum 

jātyandho jāyate kasmānṣaṇḍhobhīrur hatendriyaḥ\thinspace{\danda} \dontdisplaylinenum

kujo vā vāmano vāpi paṅgavaḥ sthūlaśiraḥ katham \veg\dontdisplaylinenum

bhagavān uvāca~{\dandab}\dontdisplaylinenum 

g\textsubring{r}hītagarbhā yā nārī tīkṣṇoṣṇāny upasevate\thinspace{\danda} \dontdisplaylinenum

laśunānipalāṇḍūni karañjamūlakāni ca \veg\dontdisplaylinenum

pippalīś\textsubring{r}ṅgavera\.m ca sarṣapānmaricāni ca\thinspace{\dandab} \dontdisplaylinenum

āsavaś ca parikliṣṭā ye cānye kaṭutiktakāḥ \veg\dontdisplaylinenum

tīkṣṇa\.m tu sevamānā yā jātyandho jāyate sutaḥ\thinspace{\dandab} \dontdisplaylinenum

mithyāpacārāḥ strīpu\.mso vyāpanne śukraśoṇite \danda\dontdisplaylinenum

yadā garbhāśaye rakta\.m striyāḥ pūrva\.m niṣicyate \veg\dontdisplaylinenum

paścāc chukra\.m raktakāle tadāṣaṇḍaḥ prajāyate\thinspace{\dandab} \dontdisplaylinenum

trastodvigno yadā bhītastrīpu\.msā\.msūpajāyate \veg\dontdisplaylinenum

tatra yo jāyate garbhabhiruḥ krandanako bhavet\thinspace{\dandab} \dontdisplaylinenum

nisargakāle śukrasya vighna utpadyate yadā \veg\dontdisplaylinenum

indriyāvartavighne tu tadā jāyed atindriyaḥ\thinspace{\dandab} \dontdisplaylinenum 

g\textsubring{r}hītagarbhā yā nārī vātalāny upasevate \veg\dontdisplaylinenum

kaṭukāni kaṣāyāni tiktāni ca viśeṣataḥ\thinspace{\dandab} \dontdisplaylinenum

vātaḥ prakupitas tasyā garbham ātuhya tiṣṭhati \veg\dontdisplaylinenum

kubjas tu jāyate tasmād garbhād vātanipīḍanāt\thinspace{\dandab} \dontdisplaylinenum

nityasāsavaśīlāyā tathā cotkaṭukāśanā \veg\dontdisplaylinenum

tasyā sa\.mhanyate garbho vāmanas tena jāyate\thinspace{\dandab} \dontdisplaylinenum

ativyāyāmaśīlā tu ya nārī viṣamāsanī \veg\dontdisplaylinenum

garbhaḥ sa\.mkṣubhyate tasyāḥ paṣaṇḍas tenopajāyate\thinspace{\dandab} \dontdisplaylinenum

g\textsubring{r}hītagarbhā yā nārī rūkṣadhānyāni sevate \veg\dontdisplaylinenum

vātaśleṣmaśirastho vai tasyā garbhasya kupyate\thinspace{\dandab} \dontdisplaylinenum

tataḥ sthūlaśirās tena pumān jāyaty asa\.mśayaḥ \veg\dontdisplaylinenum

devy uvāca~{\dandab}\dontdisplaylinenum 

karālāṅgā hanuḥ paṅgūr mūko gadgadabhāṣakaḥ\thinspace{\danda} \dontdisplaylinenum

vik\textsubring{r}tākṣas tv anakṣo vā bhavadrasvagudaḥ katham \veg\dontdisplaylinenum

bhagavān uvāca~{\dandab}\dontdisplaylinenum 

karālas tena doṣeṇa jāyate mānavas tathā\thinspace{\danda} \dontdisplaylinenum

atha karāla\.m kurute nārī lamboticūcukā \danda\dontdisplaylinenum

tasmād anena doṣeṇa karālo jāyate pumān \veg\dontdisplaylinenum

g\textsubring{r}hītagarbhā yā nārī raktapittāmayārditā\thinspace{\dandab} \dontdisplaylinenum

gohanu\.m janayet yeṣā raktapittaprakopitaḥ \veg\dontdisplaylinenum

g\textsubring{r}hītagarbhā yā nārī vātaśūlair upadrutā\thinspace{\dandab} \dontdisplaylinenum

śukro dāvartanī cāpi paṅgū janayate sutam \veg\dontdisplaylinenum

kṣudhārtā vedanārtā ca satataś copavāsinī\thinspace{\dandab} \dontdisplaylinenum

mūka\.m janayate bāla\.m dauh\textsubring{r}daś ca vimānitā \veg\dontdisplaylinenum

g\textsubring{r}hītagarbhā yā nārī vis\textsubring{r}jet - māsa māsikam\thinspace{\dandab} \dontdisplaylinenum

anakṣo jāyate tasyā garbhaśoṇitasa\.mkṣayāt \veg\dontdisplaylinenum

atha grastā yadā nārī vāto dāvartapīḍitā\thinspace{\dandab} \dontdisplaylinenum

g\textsubring{r}hītagarbhā rukṣāṇi vātalāny upasevate \veg\dontdisplaylinenum

vātasthānantatas tasyā garbhasyāpīḍita\.m bhavet\thinspace{\dandab} \dontdisplaylinenum

agudo jāyate tasmāj jātaś cāpi na jīvati \veg\dontdisplaylinenum

devy uvāca~{\dandab}\dontdisplaylinenum 

hīnāṅgo jāyate kasmād adhikāṅgo 'pi vā katham\thinspace{\danda} \dontdisplaylinenum

śvetapiṅgekṣaṇaḥ kasmāt katha\.m lohitalocanaḥ \veg\dontdisplaylinenum

bhagavān uvāca~{\dandab}\dontdisplaylinenum 

garbhasya jāyamānasya - - -  jāyate nilaḥ\thinspace{\danda} \dontdisplaylinenum

vātābhyā\.m śleṣmaṇāt - - - tadaṅga\.m parihīyate \veg\dontdisplaylinenum

hīnāṅgo jāyate tasmāt pumān vātaprakopataḥ\thinspace{\dandab} \dontdisplaylinenum

g\textsubring{r}hītagarbhā yā nārī madhurāṇy upasevate \veg\dontdisplaylinenum

ś\textsubring{r}ṅgāṭakakalotyāni śālūkāni viśāni ca\thinspace{\dandab} \dontdisplaylinenum

moca\.m tālaphala\.m caiva nārikelaphala\.m tathā \veg\dontdisplaylinenum

atikṣṇa\.m sevamānā tu adhikāṅga\.mprasūyate\thinspace{\dandab} \dontdisplaylinenum

piṅgākṣaḥ śleṣmapittābhyā\.m śvetākṣaḥ śleṣmaṇā bhavet \veg\dontdisplaylinenum

devy uvāca~{\dandab}\dontdisplaylinenum 

katha\.m vā jāyate putraḥ kanyakā kena jāyate\thinspace{\danda} \dontdisplaylinenum

apumān kena jāyeta dviyamā triyamā tathā \veg\dontdisplaylinenum

bhagavān uvāca~{\dandab}\dontdisplaylinenum 

śukrādhikaḥ pumān jñeyaḥ kanyā raktādhikā bhavet\thinspace{\danda} \dontdisplaylinenum

raktaśukrasamatvena jāyate sa napu\.msakaḥ \veg\dontdisplaylinenum

piṇḍībhūto yadā garbha mārutau vibhaved dvidhā\thinspace{\dandab} \dontdisplaylinenum

eva\.m te dviyamā jñeyās triyamā ca tridhā k\textsubring{r}te \veg\dontdisplaylinenum

devy uvāca~{\dandab}\dontdisplaylinenum 

śoṇita\.m mā\.msa medaś ca asthi majjā ca pañcamī\thinspace{\danda} \dontdisplaylinenum

śarīrasthāni d\textsubring{r}śyante śukrasthāna\.m na d\textsubring{r}śyate \veg\dontdisplaylinenum

tasyotpattiś ca sthāna\.m ca jñātum icchāmi tattvataḥ\thinspace{\dandab} \dontdisplaylinenum

kathayasva trilokeśa cchettum arhasi sa\.mśayaḥ \veg\dontdisplaylinenum

bhagavān uvāca~{\dandab}\dontdisplaylinenum 

manaḥ śukrasya prabhava\.m ghrāṇa\.m śrotra\.m tathākṣiṇī\thinspace{\danda} \dontdisplaylinenum

sthāna\.m tu sarvāṅgasamasparśāt sparśaḥ pravartate \veg\dontdisplaylinenum

yathā niṣikta\.m kṣīra\.m tu payasād dadhi jāyate\thinspace{\dandab} \dontdisplaylinenum

pramathyamānadadhnas tu sarpiso 'pi tathāgamaḥ \veg\dontdisplaylinenum

eva\.m śarīra nirgaccet - śukra\.m śukravahā śirāḥ\thinspace{\dandab} \dontdisplaylinenum

pūrayitvānupūrveṇa asthayo pratipadyate \veg\dontdisplaylinenum

tatas tu tāḥ śukravahā meḍhranābhīm anus\textsubring{r}tāḥ\thinspace{\dandab} \dontdisplaylinenum

nāśukra\.m tat tu siñcanti tasmād garbhasya sambhavaḥ \veg\dontdisplaylinenum

devy uvāca~{\dandab}\dontdisplaylinenum 

katha\.m vedayate jāti katha\.m jātismaro bhavet\thinspace{\danda} \dontdisplaylinenum

etasmin sa\.mśaya\.m me 'dya chettum arhasi śaṅkara \veg\dontdisplaylinenum

bhagavān uvāca~{\dandab}\dontdisplaylinenum 

bhāvitātmā\.m ca yo jantur devi bhogādhika\.m ca yat\thinspace{\danda} \dontdisplaylinenum

brahmavid jñānasa\.myuktaḥ sa jāti\.m smarate pumān \veg\dontdisplaylinenum

devy uvāca~{\dandab}\dontdisplaylinenum 

katha\.m sadyo g\textsubring{r}hītasya liṅgagarbhasya d\textsubring{r}śyate\thinspace{\danda} \dontdisplaylinenum

etat kathaya deveśa rahaḥ kāle maheśvara \veg\dontdisplaylinenum

bhagavān uvāca~{\dandab}\dontdisplaylinenum 

pipāśāromaharṣa\.m ca vepana\.m gātrasīdanam\thinspace{\danda} \dontdisplaylinenum

nidrāsveda\.m ca tandrā ca muhūrtam upajāyate \veg\dontdisplaylinenum

nikledatva\.m kharatva\.m ca yonyāt samupajāyate\thinspace{\dandab} \dontdisplaylinenum

na cārdrava\.mvai d\textsubring{r}śyeta śukrasya rajaso 'pi vā \danda\dontdisplaylinenum 

sadyog\textsubring{r}hītagarbhāyā liṅgāny etāni tattvataḥ \veg\dontdisplaylinenum

devy uvāca~{\dandab}\dontdisplaylinenum 

kena liṅgena vijñeya\.m putrajanma maheśvara\thinspace{\danda} \dontdisplaylinenum

kanyakā kena liṅgena jñāyate kathayasva me \veg\dontdisplaylinenum

bhagavān uvāca~{\dandab}\dontdisplaylinenum 

pādorujaṅghapārśvaś ca dakṣiṇa\.m yadi hy unnataḥ\thinspace{\danda} \dontdisplaylinenum

dakṣiṇa\.m vipula\.m tatra tadā putraḥ prajāyate \veg\dontdisplaylinenum

vāmaś caiva yadā paśyet tadā jāyeta kanyakā\thinspace{\dandab} \dontdisplaylinenum

unnata\.m madhyamasthāś ca tadā jāyet - napu\.msakam \veg\dontdisplaylinenum

devy uvāca~{\dandab}\dontdisplaylinenum 

pu\.msā kapolaromāni khalita\.m kena jāyate\thinspace{\danda} \dontdisplaylinenum

katha\.m strīṇā\.m na jāyeta romāṇi khalita\.m tathā \veg\dontdisplaylinenum

bhagavān uvāca~{\dandab}\dontdisplaylinenum 

tathā v\textsubring{r}ṣaṇagā jantor yasya retovahā śiraḥ\thinspace{\danda} \dontdisplaylinenum

nibaddhā mastake tālu kapolās tu samāśritāḥ \veg\dontdisplaylinenum

taiḥ kapoleṣu romāṇi jāyante antaretasaḥ\thinspace{\dandab} \dontdisplaylinenum

khalita\.m śukradoṣeṇa narāṇām upajāyate \veg\dontdisplaylinenum

śirā śukravahā strīṇā\.m na śūnyasyānna jāyate\thinspace{\dandab} \dontdisplaylinenum

yātmāpālo ca kās tv agni d\textsubring{r}ṣṭimaṇḍalasa\.mśritaḥ ? \veg\dontdisplaylinenum

śoṇitai soktikoṣṭasthanniśoṣayati tattvataḥ\thinspace{\dandab} \dontdisplaylinenum

nibaddhanty akṣipakṣmāṇi tena romāṇi ca bhruvoḥ \veg\dontdisplaylinenum

aśukratvāc ca nārīṇā\.m khalita\.m nopajāyate\thinspace{\dandab} \dontdisplaylinenum

chāyāvyapagatasnehā rukṣāgātraśiroruhā \danda\dontdisplaylinenum

grasatosmābhajaṭharā m\textsubring{r}tagarbhaḥ prajāyate \veg\dontdisplaylinenum

devy uvāca~{\dandab}\dontdisplaylinenum 

somadhātu katha\.m jñeyā agnidhātus tatheśvara\thinspace{\danda} \dontdisplaylinenum

p\textsubring{r}thagbhāgaviśeṣeṇa kathayasva maheśvara \veg\dontdisplaylinenum

bhagavān uvāca~{\dandab}\dontdisplaylinenum 

śleṣmamedas tathā snāyuḥ asthidantanakhāni ca\thinspace{\danda} \dontdisplaylinenum

striyās tanyaś ca śukraś ca yac ca śveta\.m tathākṣiṣu \veg\dontdisplaylinenum

eteṣā\.m saumyabhāgatvāc chvetatvam upajāyate\thinspace{\dandab} \dontdisplaylinenum

āgneyabhāvād raktatva\.m k\textsubring{r}ṣṇatva\.m cāpi gacchati \veg\dontdisplaylinenum

tvagmā\.msarudhira\.m majjād\textsubring{r}ṣṭiroma tathaiva ca\thinspace{\dandab} \dontdisplaylinenum

āgneyadhātusomaś ca kathito 'smi varānane \danda\dontdisplaylinenum

brūhi brūhi viśālākṣi yady asti tava sa\.mśayaḥ \veg\dontdisplaylinenum


\jump
\begin{center}
\ketdanda iti v\textsubring{r}ṣasārasa\.mgrahe praśnavyākaraṇo nāmaś caturdaśo 'dhyāyaḥ\ketdanda
\end{center}
\dontdisplaylinenum\vers 
\bekveg\szamveg\vfill\phpspagebreak\szam\bek\versno=0\fejno=15
\thispagestyle{empty}



\alfejezet{\textbf{pañcadaśamo 'dhyāyaḥ}}\jump\jump

\alalfejezet{jīvavarṇanam}
devy uvāca~{\dandab}\dontdisplaylinenum 

jīvabhūteti yat prokta\.m lakṣaṇa\.m kīd\textsubring{r}śa\.m bhavet\thinspace{\danda} \dontdisplaylinenum
            \var{\vb lakṣaṇa\.m kī°\lem  \msNa\msNb\msNc\Ed; lakṣaṇāṅ kī° \msCa, laṇa\.m kī° \msCb}%

sthānam asya na jānāmi rūpa\.m varṇa\.m ca īśvara \veg\dontdisplaylinenum
            \var{\vc sthānam asya\lem  \msCb\msNa\msNb\msNc\Ed; {\il}\uncl{na}m asya \msCa}%
            \var{\vd rūpa\.m varṇa\.m\lem  \msCa\msCb\msNa\Ed; rūpavarṇa\.m \msNb\msNc}%
            \paral{\textit{{\normalfont Testimonia for this chapter: \msCa\ ff.\thinspace 219r--220r, 
                                             \msCb\ ff.\thinspace 222v--223v, 
                                             \msCc\ is not available for this chapter,
                                             \msNa\ ff.\thinspace 26r--27r, 
                                             \msNb\ ff.\thinspace 230v--231r, 
                                             \msNc\ ff.\thinspace 234r--235r;
                                               \mssCaCbCc\ = \msCa + \msCb + \msCc }}}

etat kautūhala\.m chindhi sa\.mśaya\.m parameśvara\thinspace{\dandab} \dontdisplaylinenum
            \var{\va etat kautūhala\.m\lem  \msCa\msCb\msNa\msNb\Ed; etat kautūla\.m \msNc\oo
                 chindhi\lem  \msCa\msCb\msNa\msNb\Ed; chitvāndhi \msNc}%
            \var{\vb sa\.mśaya\.m\lem  \msCa\msCb\msNa\msNc\Ed; sa\.mśaya \msNb}%

na cānyad eva paśyāmi jīvanirṇaya kīrtaya \veg\dontdisplaylinenum

īśvara uvāca~{\dandab}\dontdisplaylinenum 
            \var{\vo īśvara\lem  \msCa\msCb\msNa\msNb\msNc; bhagavān \Ed}%

jīvasya lakṣaṇa\.m devi kathitu\.m kena śakyate\thinspace{\danda} \dontdisplaylinenum
            \var{\va lakṣaṇa\.m\lem  \msCb\msNa\msNb\msNc\Ed; kathita\.m \msCa}%

na rūpavarṇa\.m jīvasya vidyate sthānam eva ca \veg\dontdisplaylinenum
            \var{\vc °varṇa\.m\lem  \msCb\msNa\msNc; °varṇa \msCa\msNb\Ed}%

vyāpi sarvagata\.m sūkṣma\.m sarvam āśritya tiṣṭhati\thinspace{\dandab} \dontdisplaylinenum
            \var{\va vyāpi\lem  \msCb\msNa\msNb\msNc; vyā\uncl{pi} \msCa, vyāpī \Ed}%
            \var{\va °śritya\lem  \msCb; °ś\textsubring{r}tya \msCa\msNa\msNb, °śrutya \msNc, °v\textsubring{r}tya \Ed}%

nirālambam anādhāram anaupamya\.m nirañjanam \veg\dontdisplaylinenum
            \var{\vd °pamya\.m\lem  \msCa\msCb\msNa\msNc\Ed; °pamya \msNb}%

araṇistho yathā vahniḥ kāṣṭheṣu nopalabhyate\thinspace{\dandab} \dontdisplaylinenum

tadvaj jīvo na paśyeta śarīrastho 'pi sundari \veg\dontdisplaylinenum
            \var{\vc jīvo na\lem  \msCb\msNa\msNb\msNc; jīvon na \msCa, jīva\.m na \Ed}%
            \var{\vd 'pi\lem  \msCa\msCb\msNa\msNc\Ed; hi \msNb}%

dadhivac ca yathā sarpir d\textsubring{r}śyate na ca d\textsubring{r}śyate\thinspace{\dandab} \dontdisplaylinenum

tadvaj jīvaḥ śarīrastho d\textsubring{r}śyate na ca d\textsubring{r}śyate \veg\dontdisplaylinenum
            \var{\vc tadvaj jīvaḥ\lem  \msCa\msCb\msNa\msNb; tadva jīvaḥ \msNc, tadvaj jīva \Ed}%

devy uvāca~{\dandab}\dontdisplaylinenum 

ad\textsubring{r}ṣṭapratyayo hy asti nāsti pratyayadarśanam\thinspace{\danda} \dontdisplaylinenum

vyāpī katha\.m mahādeva sarvatrāvasthitaḥ katham \veg\dontdisplaylinenum
            \var{\vd °sthitaḥ\lem  \msCb\msNc\Ed; °sthita\.m \msCa\msNa, °sthita \msNb}%

maheśvara uvāca~{\dandab}\dontdisplaylinenum 
            \var{\vo maheśvara\lem  \msCa\msCb\msNb\msNc; mahādeva \msNa, bhagavān \Ed}%

asa\.mśayo mahādevi vyāpī sarvagataḥ śivaḥ\thinspace{\danda} \dontdisplaylinenum

d\textsubring{r}śyatendriyasa\.myogāj jīvapratyayadarśanam \veg\dontdisplaylinenum
            \var{\vc d\textsubring{r}śyate°\lem  \msCb\msNa\msNb\msNc; d\textsubring{r}śyete° \msCa, d\textsubring{r}śyante \Ed}%
            \var{\vd °jīva°\lem  \msCa\msCb\msNa\msNb\Ed; °jī° \msNc}%

yathākāśasthito vāyuḥ śabdasparśaguṇānvitaḥ\thinspace{\dandab} \dontdisplaylinenum
            \var{\vcd vāyuḥ śabda°\lem  \msCb\msNa\msNb\msNc\Ed; vāyu\uncl{śśa}{\il} \msCa}%
            \var{\vd °nvitaḥ\lem  \msCa\msNa\msNb\msNc\Ed; °nvitam \msCb}%

tadvad dehī vijānīyād guṇaceṣṭena nānyathā \veg\dontdisplaylinenum
            \var{\vd °ceṣṭena\lem  \msCa\msCb\msNa\msNb; °veṣṭana \msNc, °veṣṭena \Ed}%

devy uvāca~{\dandab}\dontdisplaylinenum 

vyāpīti kathitaḥ pūrva\.m jīvaḥ sarvagato 'pi ca\thinspace{\danda} \dontdisplaylinenum
            \var{\va kathitaḥ\lem  \msCa\msNa\msNcpcorr\Ed; kathita\.m \msCb\msNb, kathatiḥ \msNc}%

ta\.m v\textsubring{r}thā kathito 'sy adya mriyate kena hetunā \veg\dontdisplaylinenum
            \var{\vc v\textsubring{r}thā\lem  \msCa\msCb\msNa\msNb\Ed; vyathā \msNc\oo
                 'sy adya\lem  \msCa\msCb\msNc; smy adya \msNa\Ed, sy a{\il} \msNb}%

īśvara uvāca~{\dandab}\dontdisplaylinenum 
            \var{\vo īśvara\lem  \msCa\msCb\msNb\msNc; bhagavān \msNa\Ed}%

na jīvo mriyate devi sarveṣā\.m surasundari\thinspace{\danda} \dontdisplaylinenum

ghaṭāntastho yathākāśo bahirākāśavad yathā \veg\dontdisplaylinenum

ghaṭabhinne viśālākṣi viśeṣo nopalakṣyate\thinspace{\dandab} \dontdisplaylinenum
            \var{\vb nopalakṣyate\lem  \msCa\msCb\msNb\msNc\Ed; nopalabhyate \msNa}%

dehabhinne yadā devi vināśo nopalabhyate \veg\dontdisplaylinenum
            \var{\vc deha°\lem  \msCa\msNa\msNb\msNc\Ed; dehe \msCb\oo
                 yadā devi\lem  \msCa\msCb\msNa\msNb\msNc; tathā dehī \Ed}%
            \paral{\textit{\vo {\normalfont cf.\ Bhāgavatapurāṇa 12.5.5: }
                         ghaṭe bhinne ghaṭākāśa ākāśaḥ syād yathā purā{\thinspace\danda}
                         eva\.m dehe m\textsubring{r}te jīvo brahma sampadyate punaḥ{\thinspace\ketdanda}}}

susūkṣmaḥ sarvago vyāpī paramātmānam avyayaḥ\thinspace{\dandab} \dontdisplaylinenum
            \var{\va susūkṣmaḥ\lem  \msCa\msCb\msNa\msNb; susūkṣma \msNc, sa sūkṣmaḥ \Ed}%

bahir antaś ca bhūtānām acaraś cara eva saḥ \veg\dontdisplaylinenum
            \var{\vd °caraś ca°\lem  \msCa\msCb\msNa\msNb\msNc; °caran ca° \Ed\oo
                 saḥ\lem  \msCa\msCb\msNa\msNb\msNc; sa \Ed}%

aprameyo 'vināśī ca aprapañcaḥ prapañcakaḥ\thinspace{\dandab} \dontdisplaylinenum
            \var{\vab \om\ \msNb}%

sarvendriyaguṇābhāsaḥ sarvendriyavivarjitaḥ \veg\dontdisplaylinenum

evam eṣa mahādevi jīvasya varavarṇini\thinspace{\dandab} \dontdisplaylinenum

kathito 'smi samāsena kim anyac chrotum icchasi \veg\dontdisplaylinenum
            \var{\vd icchasi\lem  \msCb\msNa\msNb\msNc\Ed; icchati \msCa}%


\alalfejezet{sāraśreṣṭham}
devy uvāca~{\dandab}\dontdisplaylinenum 

sāraśreṣṭha\.m mahādeva kathayeśāna īśvara\thinspace{\danda} \dontdisplaylinenum
            \var{\va °śreṣṭha\.m\lem  \msCb\msNa\Ed; °śreṣṭha \msCa\msNb\msNc}%  

śrotum icchāmi deveśa mānuṣāṇā\.m hita\.m vada \veg\dontdisplaylinenum
            \var{\vd vada\lem  \msCa\msCb\msNa\msNb; vadaḥ \msNc\Ed}%

īśvara uvāca~{\dandab}\dontdisplaylinenum 
            \var{\vo īśvara\lem  \msCa\msCb\msNa\msNb\msNc; bhagavān \Ed}%

āśramāṇā\.m g\textsubring{r}hī śreṣṭho varṇaśreṣṭhā dvijātayaḥ\thinspace{\danda} \dontdisplaylinenum
            \var{\va āśramāṇā\.m\lem  \msCa\msCb\msNa\msNc\Ed; āśramāṇā \msNb\oo
                 g\textsubring{r}hī\lem  \msCb\msNa\msNb\msNc\Ed; g\textsubring{r}\uncl{hī} \msCa}%
            \var{\vb °śreṣṭhā\lem  \msCa\msCb\msNc; °śreṣṭo \msNa\msNb\Ed}%

aśvamedhaḥ kratuśreṣṭho japaśreṣṭho 'ghamarṣaṇaḥ \veg\dontdisplaylinenum
            \var{\vd japa°\lem  \msCapcorr\msNa\msNb\msNc\Ed; ja° \msCaacorr, 'japa° \msCb\oo
                 'ghamarṣaṇaḥ\lem  \msCb\msNa\msNb\msNc\Ed; rghamarṣaṇaḥ \msCa}%

devatānā\.m hariḥ śreṣṭhaḥ śreṣṭhā gaṅgā nadīṣu ca\thinspace{\dandab} \dontdisplaylinenum
            \var{\vab śreṣṭhaḥ śreṣṭhā gaṅgā\lem  \msCa\msNa\msNb\msNc\Ed; śreṣṭhā gaṅgāṇāñ ca \msCb}%

anāśanas tapaḥśreṣṭhas tīrthaśreṣṭhaḥ surahradaḥ \veg\dontdisplaylinenum
            \var{\vc anāśana°\lem  \msCa\msCb\msNa\msNb\Ed; anaśana° \msNc}%
            \var{\vd °rthaśreṣṭhaḥ\lem  \msCa\msCb\msNa\msNb\Ed; °rthaśreṣṭha \msNc}%

kṣoma\.m vastreṣu ca śreṣṭha\.m yaśaḥ śreṣṭha\.m vibhūṣaṇam\thinspace{\dandab} \dontdisplaylinenum
            \var{\va kṣauma\.m\lem  \msNc\Ed; kṣoma\.m \msCa\msCb\msNa, kṣoma \msNb}%
            \var{\vb śreṣṭha\.m\lem  \msCa\msCb\msNa\msNc\Ed; śreṣṭha \msNb\oo
                 °bhūṣaṇam\lem  \msCa\msNa\msNb\msNc\Ed; °bhūṣiṇam \msCb}%

bhārata\.m śrutiṣu śreṣṭha\.m vrataśreṣṭho dayāparaḥ \veg\dontdisplaylinenum
            \var{\vd °śreṣṭho\lem  \msCa\msCb\msNa\msNc\Ed; śreṣṭha\.m \msNb\oo
                 dayāparaḥ\lem  \msCb\msNa\msNb\msNc\Ed; \uncl{dayāpa}raḥ \msCa}%

dāneṣu cābhaya\.m śreṣṭha\.m manaḥ śreṣṭhendriyeṣu ca\thinspace{\dandab} \dontdisplaylinenum

vidyā sa\.mgrahaṣu śreṣṭhā satya\.m śreṣṭha\.m vacaḥsu ca \veg\dontdisplaylinenum
            \var{\vc sa\.mgrahaṣu\lem  \msCa\msCb\msNa\msNb\Ed; sa\.mgraheṣu \msNc\ \unmetr\oo
                 śreṣṭhā\lem  \msCa\msCb\msNa\msNb\msNc; śreṣṭho \Ed}% 

āyudhānā\.m dhanuḥ śreṣṭha\.m bāndhaveṣu ca mātaraḥ\thinspace{\dandab} \dontdisplaylinenum
            \var{\va śreṣṭha\.m\lem  \msCa\msCb\msNa\msNc\Ed; śreṣṭha \msNb}%
            \var{\vb bāndhaveṣu ca mātaraḥ\lem  \msCa\msCb\msNaacorr\msNc\Ed; bāndhaveṣu ca mātara\.m \msNapcorr,
                grahaśreṣṭho divākaraḥ \msNb\ \eyeskip{to 15.24b}}%

jñānam auṣadhiṣu śreṣṭha\.m vaidyaśreṣṭhaḥ śivākṣaraḥ \veg\dontdisplaylinenum
            \var{\vcd \om\ \msNb}%
            \var{\vc jñānam oṣadhiṣu\lem  \msNc; jñānam auṣadhiṣu \msCa\msCb\msNa\msNb\Ed}%
            \var{\vd vaidya°\lem  \msCa\msCb\msNa; \om\ \msNb, vaidyaḥ \msNc, vaidyo \Ed\oo
                 °śreṣṭhaḥ\lem  \msCb\msNa\msNc\Ed; °śreṣṭha \msCa, \om\ \msNb}%

akāraś cākṣaraḥ śreṣṭho dharmaśreṣṭho hy ahi\.msakaḥ\thinspace{\dandab} \dontdisplaylinenum

paśuṣu saurabhī śreṣṭhā nareṣu ca narādhipaḥ \veg\dontdisplaylinenum
            \var{\vo \om\ \msNb}%

māsi mārgaśiraḥ śreṣṭha\.m k\textsubring{r}taḥ śreṣṭhaś caturyuge\thinspace{\dandab} \dontdisplaylinenum
            \var{\vo \om\ \msNb}%
            \var{\va māsi\lem  \msCa\msCb\msNa\msNc; \om\ \msNb, māsī \Ed\oo
                 °śiraḥ\lem  \msCa\msCb\msNa\msNb\Ed; °śira \msNc}%
            \var{\vb śreṣṭhaś caturyuge\lem  \msCa\msNa\Ed; śreṣṭha\.m caturyuge \msCb, \om\ \msNb, śreṣṭhaś caryuge \msNc}%

vasanta \textsubring{r}tuṣu śreṣṭhaḥ śreṣṭha\.m cāyanam uttaram \veg\dontdisplaylinenum
            \var{\vd śreṣṭha\.m cā°\lem  \msCa\msCb\msNc\Ed; śreṣṭhaś cā° \msNa, \om\ \msNb\oo
                 °ttaram\lem  \msCa\msNa\msNc\Ed; °tta\uncl{me}m \msCb, \om\ \msNb}%

amāvāsyā dinaśreṣṭhā grahaśreṣṭho divākaraḥ\thinspace{\dandab} \dontdisplaylinenum
            \var{\va amāvāsyā dinaśreṣṭhā\lem  \msCa\msCb\msNc\Ed; \om\ \msNb, amāvāsyā dinaśreṣṭho \msNa}%
            \var{\vb grahaśreṣṭho divākaraḥ\lem  \msCa\msCb\msNa\msNb; grahaḥ śreṣṭho divākaraḥ \msNc, vasuśreṣṭho hutāśanaḥ \Ed}%

strīṣu lakṣmīr dh\textsubring{r}tiḥ śreṣṭhā vasuśreṣṭho hutāśanaḥ \veg\dontdisplaylinenum
            \var{\vcd \om\ \Ed}%
            \var{\vc strīṣu\lem  \msCa\msNa\msNb\msNc\Ed; strī \msCb\oo
                 lakṣmīr dh\textsubring{r}tiḥ\lem  \msCa; lakṣmīdh\textsubring{r}tiḥ \msCb\msNa\msNb\msNc, \om\ \Ed}%

\textsubring{r}ṣiṣu uṣaṇā śreṣṭhaḥ kāntiśreṣṭho niśākaraḥ\thinspace{\dandab} \dontdisplaylinenum
            \var{\va uṣaṇā\lem  \corr; uśanāḥ \msCa\msCb\msNa\msNb\msNc, uśanaḥ \Ed}%
            \var{\vb kānti°\lem  \msCb\msNa\msNb\Ed; kāntiḥ \msNc, kā{\il} \msCa}%

nakṣatreṣv abhijit śreṣṭhaḥ kālaḥ śreṣṭhaḥ kaleṣu ca  \veg\dontdisplaylinenum
            \var{\vc °bhijit śre°\lem  \Ed; °bhijiḥ śre° \msCa\msCb\msNa\msNbpcorr\msNc, °bhiji \msNbacorr}%
            \var{\vd kālaḥ\lem  \msCa\msCb\msNa\msNb\msNc; kaliḥ \Ed}%

vedeṣu ca vara\.m sāma sthāvareṣu himālayaḥ\thinspace{\dandab} \dontdisplaylinenum

aśvattho vaṭa v\textsubring{r}kṣeṣu bhūteṣu vara cetanaḥ \veg\dontdisplaylinenum
            \var{\vc vaṭa\lem  \msCa\msCb\msNa\msNb; vara \msNc\Ed}%
            \var{\vd vara cetanaḥ\lem  \msCb\Ed; varaś cetanaḥ \msCa\msNa\msNc\ \unmetr, vaś cetanaḥ \msNb}%

adhyātma sarvavidyāsu vākya satya vara sm\textsubring{r}taḥ\thinspace{\dandab} \dontdisplaylinenum
            \var{\va adhyātma\lem  \msCb\msNb\Ed; adhyātmā \msCa\msNc, ādhyātma\.m \msNa\oo
                 sarvavidyāsu\lem  \msCa\msNa\msNb\msNc; sarvavidyānā\.m \msCb, varavidyāsu \Ed}%
            \var{\vb vākya\lem  \msCb; vāhu \msCa\msNa\msNb\msNc, vācaḥ \Ed\oo
                 vara\lem  \msCa\msCb\Ed; va\uncl{ra}ḥ \msNa, varaḥ \msNb\msNc}%

prahlādo vara daityeṣu yakṣarakṣo dhaneśvaraḥ \veg\dontdisplaylinenum
            \var{\vc prahlādo\lem  \msCa\msCb\msNa\Ed; prahrādo \msNb\msNc}%
            \var{\vd °śvaraḥ\lem  \msCa\msCb\msNa\msNc\Ed; °śvara \msNb}%

marīcir vara vāteṣu hariḥ śreṣṭho m\textsubring{r}geṣu ca\thinspace{\dandab} \dontdisplaylinenum
            \var{\va marīcir vara\lem  \msNc; marīci vara \msCb\msNa\msNb\Ed, ma{\il}{\il}{\il}{\il} \msCa}%
            \var{\vb hariḥ\lem  \msCa\msCb\msNb\msNc\Ed; hari \msNa}%

sādhya nārāyaṇaḥ śreṣṭhaḥ pit\textsubring{\=r}ṇā\.m ca pitāmahaḥ \veg\dontdisplaylinenum

etat samāsato devi kathito 'si varānane\thinspace{\dandab} \dontdisplaylinenum
            \var{\vb 'si\lem  \msCa\msCb\msNa\msNb; smi \msNc\Ed}%

sarvasāra\.m samuddh\textsubring{r}tya ki\.m bhūyaḥ kathayāmy aham \veg\dontdisplaylinenum
            \var{\vd ki\.m\lem  \msCb\msNa\msNb\msNc\Ed; ki \msCa}%


\jump
\begin{center}
\ketdanda iti v\textsubring{r}ṣasārasa\.mgrahe jīvanirṇayo nāmādhyāyaḥ pañcadaśamaḥ\ketdanda
\end{center}
\dontdisplaylinenum\vers 
            \var{{\normalfont Colophon:} nāmādhyāyaḥ pañcadaśamaḥ\lem  \msCa\msCb\msNa; nāmādhyāyaḥ pañcamaḥ \msNb,
                                nāmādhyāyaḥ pañcadaśama \msNc, nāma pañcadaśo 'dhyāyaḥ \Ed}%
\bekveg\szamveg\vfill\phpspagebreak\szam\bek\versno=0\fejno=16
\thispagestyle{empty}



\alfejezet{\textbf{ṣoḍaśamo 'dhyāyaḥ}}\jump\jump

\alalfejezet{yogasadbhāvanirṇayaḥ}
\vers

devy uvāca~{\dandab}\dontdisplaylinenum 

adhunā śrotum icchāmi yogasadbhāvanirṇayam\thinspace{\danda} \dontdisplaylinenum
            \paral{\textit{\vo {\normalfont   \msCa\ 435.jpg line 2; \msCb\ 448.jpg line 2;
                      This chapter is missing in \msCc.
                      \msNa\ 220.jpg lower image line 5; \msNb\ 65.jpg upper image line 6.}}}
            \var{\vb °nirṇayam\lem  \msCa\msCb\msNa\msNb; °nirṇayaḥ \Ed}%

karaṇa\.m ca yathānyāya\.m kathayasva sureśvara \veg\dontdisplaylinenum
            \var{\vc karaṇa\.m\lem  \msCa\msCb\msNa\msNb; karaṇaś \Ed}%
            \var{\vd sureśvara\lem  \msCa\msCb\msNa\msNb; sureśvaraḥ \Ed}%

īśvara uvāca~{\dandab}\dontdisplaylinenum 
            \var{\vo īśvara\lem  \msCa\msCb\msNb; sureśa \msNa, bhagavān \Ed}%

ś\textsubring{r}ṇu devi pravakṣyāmi yogasadbhāvam uttamam\thinspace{\danda} \dontdisplaylinenum
            \var{\vb °m uttamam\lem  \msCa\msCb\msNa\Ed; °nirṇayam \msNb}%

ya\.m viditvā na paśyanti janāḥ sa\.msārabandhanam \veg\dontdisplaylinenum

brahmahā gurutalpī vā surāpasteya eva vā\thinspace{\dandab} \dontdisplaylinenum
            \var{\vb vā\lem  \msCb\msNa\msNb; \uncl{vā} \msCa,  ca \Ed}%

athavā sa\.mkare jātas tat sarvam apanodati \veg\dontdisplaylinenum
            \var{\vc sa\.mkare\lem  \msNa; ś\textsubring{r}ṅkare \msCa, śaṅkare \msCb\Ed, \uncl{śa\.m}kare \msNb}%
            \var{\vd tat sarva°\lem  \msCa\msCb\msNa\msNb; tasarva° \Ed}%

muhūrtārdhe muhūrte vā prāṇāyāmaparāyaṇaḥ\thinspace{\dandab} \dontdisplaylinenum
            \var{\va muhūrtārdhe muhūrte vā\lem  \msCa\msNb;
                muhūrtārdhe vā \msCb,
                muhūrttārddha muhūrte vā \msNa,
                muhūrtārdhamuhūrta\.m ca \Ed}%
            \paral{\textit{\vo {\normalfont  cf.\ 16.10.  }}}

dhyeya\.m cintayamānasya tatpāpa\.m kṣīyate narāt \veg\dontdisplaylinenum
            \var{\vc dhyeya\.m\lem  \msCa\msNb; dheyañ \msCb, dhyeya \msNa\Ed}%
            \var{\vd narāt\lem  \msCb\msNa; narān \msCa\msNb\Ed}%
            \paral{\textit{\vo {\normalfont  \kb\ a citation in Kauṇḍinya's commentary ad \PS\ 5.24: }
                muhūrtārdha\.m muhūrta\.m vā prāṇāyāmāntare 'pi vā{\thinspace\danda} 
                dhyeya\.m cintayamānas tu pāpa\.m kṣapayate naraḥ{\thinspace\ketdanda} }}

na yamo nāntakaḥ kruddho na m\textsubring{r}tyur bhīmavigrahaḥ\thinspace{\dandab} \dontdisplaylinenum
            \var{\vb m\textsubring{r}tyur bhī°\lem  \msCa\msCb\Ed; m\textsubring{r}tyu bhī° \msNa\msNb\oo 
                 bhīmavigrahaḥ\lem  \msCa\msCb\msNa\msNb; nāpavigrahaḥ \Ed}%
                     \paral{\textit{\vb \kb\ {\normalfont  MBh 12.289.25ab: } na yamo nāntakaḥ kruddho na m\textsubring{r}tyur bhīmavikramaḥ}}

nāviśanti mahātmāno yogino balavattarāḥ \veg\dontdisplaylinenum
            \var{\vc nāviśanti\lem  \msCa\msCb\msNa\msNb; 
                        viśanti sma \Ed}%
            \var{\vd balavattarāḥ\lem  \msCa\msCb\msNa\msNb; 
                        varavattarā \Ed}%

yathā vai sarvadhātūnā\.m doṣā dahyanti dhāmyatām\thinspace{\dandab} \dontdisplaylinenum
            \var{\vb doṣā dahyanti\lem  \msNb; \uncl{doṣā\.m dahya}nti \msCa, doṣā\.m dahyanti \msCb\msNa\Ed}%

tathā pāpāḥ pradahyante dhruva\.m prāṇasya nigrahāt \veg\dontdisplaylinenum
            \var{\vc pāpāḥ\lem  \msCa\msCb\msNa\msNb; pāpaḥ \Ed}%
            \var{\vd nigrahāt\lem  \msCa\msCb\msNa\Ed; nigrahān \msNb}%
            \paral{\textit{\vo  {\normalfont  \kb\ Bhaviṣyapurāṇa 1.145.9: }
                                dhyāyamānasya dahyante cānte doṣā yathāgninā{\thinspace\danda}
                                tathendriyak\textsubring{r}tā doṣā dahyante prāṇanigrahāt{\thinspace\ketdanda}
                    {\normalfont  \kb\ Gheraṇḍasa\.mhitā (ed.\ Thomi) 4.11: }
                                yathā parvatadhātūnā\.m doṣā dahyanti dhāmyatām{\thinspace\danda}
                                tathendriyak\textsubring{r}tā doṣā dahyante prāṇanigrahāt{\thinspace\ketdanda}}}

aśvamedhasahasra\.m ca rājasūyaśata\.m tathā\thinspace{\dandab} \dontdisplaylinenum

prāṇāyāmaśata\.m caiva na tattulya\.m kadācana \veg\dontdisplaylinenum
            \var{\vd kadācana\lem  \msCa\msCb\msNa\msNbpcorr\Ed; kadāca \msNbacorr}%

yajñena devān āpnoti rājya\.m vai tapasaḥ phalam\thinspace{\dandab} \dontdisplaylinenum
            \var{\va devān āpnoti\lem  \msCa\msCb\msNa\msNbpcorr\Ed; devāpnoti \msNbacorr}%
            \paral{\textit{\vo \kb\ {\normalfont  Agnipurāṇa 378.1: } 
                yajñaiś ca devān āpnoti vairāja\.m(?)\ tapasā padam{\thinspace\danda}
                brahmaṇaḥ karmasannyāsād vairāgyāt prak\textsubring{r}tau layam{\thinspace\ketdanda}
                \kb\ {\normalfont  Maskarin's commentary CHECK ad Gautamadharmasūtra 3.1: }
                        yajñena devān āpnoti vairāja\.m(!)\ tapasā punaḥ{\thinspace\danda}
                        sa\.mnyāsād brahmaṇaḥ sthāna\.m vairāgyāt prak\textsubring{r}tau layam{\thinspace\ketdanda}}}

sa\.mnyāsād brahmaṇaḥ sthāna\.m vairāgyāt prak\textsubring{r}tau layam \veg\dontdisplaylinenum
            \var{\vc °d brahmaṇaḥ\lem  \msCa\msCb\msNa\Ed; °d brāhmaṇaḥ \msNb}%
            \var{\vd vairāgyāt\lem  \msCa\msNa\msNb\Ed; mahātmāno \msCb\ ({\normalfont eyeskip to 16.5c?})\oo
                 prak\textsubring{r}tau layam\lem  \eme; prak\textsubring{r}tālayam \msCa\msCb\msNa\msNb\Ed}%
                     \paral{\textit{\vcd {\normalfont cf.\ 11.27ab above: } sāvitrī prak\textsubring{r}tau layaḥ}}

jñānāt prāpnoti kaivalya\.m para\.m brahma sanātanam\thinspace{\dandab} \dontdisplaylinenum
            \var{\vb para\.m\lem  \msCa\msCb\msNa\Ed; para° \msNb}%

ity etā gatayaḥ pañca vidhivat parikīrtitāḥ \veg\dontdisplaylinenum

muhūrtārdha\.m muhūrta\.m vā yoga\.m yuñjīta yogavit\thinspace{\dandab} \dontdisplaylinenum
            \var{\va muhūrtārdha\.m muhūrta\.m\lem  \msCa\msCb\msNb;
                         muhūrtārddha muhū\uncl{rtta\.m} \msNa, 
                         muhūrtārdha muhūrta\.m \Ed}%
            \var{\vb yoga\.m\lem  \msCa\msCb\msNa\Ed; yoga \msNb}%
            \paral{\textit{\vo {\normalfont  cf.\ 16.4. }}}

nistaret sarvapāpāni am\textsubring{r}tatva\.m ca gacchati \veg\dontdisplaylinenum
            \var{\vc nistaret\lem  \msCb\msNa\msNb\Ed; vistaret \msCa}%
            \var{\vd am\textsubring{r}tatva\.m\lem  \msCa\msCb\msNa\Ed; am\textsubring{r}tatva \msNb}%

yuñjāno 'pi prayatnena yāvat tattva\.m na vindati\thinspace{\dandab} \dontdisplaylinenum
            \var{\vb yāvat tattva\.m na vindati\lem  \msNa\Ed;
                yāvantanna vindati \unmetr\ \msCa, yāva tatva\.m na vindati \msCb, yāvatta\.mn na vindati \msNb}%

brahmaloke dhruva\.m vāso viṣṇuloke ca sundari \veg\dontdisplaylinenum
            \var{\vc brahmaloke\lem  \msCa\msCb\msNa\Ed; brahmaloko \msNb\oo
                 vāso\lem  \msCa\msNa\msNb\Ed; vāsvā \msCb}%

bhuktvā karmasahasrāṇi sarvakāmasamanvitaḥ\thinspace{\dandab} \dontdisplaylinenum

kṣīṇapuṇye tato martye jāyate vipule kule \veg\dontdisplaylinenum
            \var{\vc °puṇye ta°\lem  \msCa\msCb; °puṇyas ta° \msNa\msNb\Ed\oo
                 martye\lem  \msCa\msCb\msNa\msNb; martyā\.m \Ed}%

yogam evābhiseveta pūrvajātismaro naraḥ\thinspace{\dandab} \dontdisplaylinenum

sa\.msārārṇavam uttīrya sa śivatvam avāpnuyāt \veg\dontdisplaylinenum


\alalfejezet{yogavidhiḥ}
devy uvāca~{\dandab}\dontdisplaylinenum 

yogasya vidhim icchāmi śrotu\.m me puruṣottama\thinspace{\danda} \dontdisplaylinenum
            \var{\vb śrotu\.m me\lem  \msCa\msCb\Ed; śrotu\.m vai \msNa, śrotu me \msNb\oo
                 °ttama\lem  \msCa\msCb\msNa\msNb; °ttamaḥ \Ed}%

dhyānadhāraṇasiddhīnā\.m kathayasva sureśvara \veg\dontdisplaylinenum
            \var{\vc °siddhīnā\.m\lem  \msCa\msCb\msNa\msNb; °siddhānā\.m \Ed}%
            \var{\vd sureśvara\lem  \msCa\msCb\msNa\msNb; sureśvaraḥ \Ed}%

maheśvara uvāca~{\dandab}\dontdisplaylinenum 
            \var{\va maheśvara\lem  \msCa\msCb\msNa\msNb; bhagavān \Ed}%

ś\textsubring{r}ṇu yogavidhi\.m vakṣye bhavapāśanik\textsubring{r}ntanam\thinspace{\danda} \dontdisplaylinenum

śucir ekāgracittas tu janaśabdavivarjite \danda\dontdisplaylinenum
            \var{\vd °cittas tu\lem  \msCa\msNa\msNb\Ed; °cittasyastu \msCb\oo  
                 jana°\lem  \msCa\msCb\msNa\msNb; dhyāna° \Ed\oo
                 °vivarjite\lem  \msNa; °vivarjitaḥ \msCa\msCb\msNb, °vivarjitam \Ed}%            

tatrāsīnāsane yogī paramātmāna cintayet \veg\dontdisplaylinenum
            \var{\vf °tmāna cintayet\lem  \msCb\msNa; °tmāna\.m cintayet \msCa\Ed\ \unmetr, °tmānā vicintayet \msNb}%

padmaka\.m svastika\.m caiva niṣkalam añjalis tathā\thinspace{\dandab} \dontdisplaylinenum
            \var{\va padmaka\.m\lem  \msCb\msNa\msNb\Ed; padmaka \msCa}%
            \var{\vb niṣkalam añjalis tathā\lem  \msCa\msCb\msNb; niṣkala\.m mañjalis tathā \msNa,
                                                                niṣkalamakañjalintathā \Ed}%

ardhacandra\.m ca daṇḍa\.m ca paryaṅka\.m bhadram eva ca \veg\dontdisplaylinenum
            \var{\vd paryaṅka\.m\lem  \msCb\msNa\msNb\Ed; pa{\il}ṅka\.m \msCa}%
            \paral{\textit{\vo {\normalfont  cf.\ Sarvajñānottara 27:9cd--10ab: } 
                padmaka\.m svastika\.m vāpi upasthāñjalika\.m tathā{\thinspace\ketdanda} 
                pīṭhārdham ardhacandra\.m vā sarvatobhadram eva vā{\thinspace\danda}}}

etadāsanabandhena baddhvā yoga\.m samabhyaset\thinspace{\dandab} \dontdisplaylinenum
            \var{\vb baddhvā yoga\.m\lem  \msCa\msCb\msNa\msNb; baddhā yoga \Ed}%

sama\.m kāyaśirogrīva\.m dhārayann acalasthitaḥ \veg\dontdisplaylinenum
            \var{\vc sama\.m\lem  \msCa\msCb\msNa\Ed; sama° \msNb}%
            \paral{\textit{\vcd \kb\ {\normalfont  MBh 6.28.13ab (BhG 6.13ab): } sama\.m kāyaśirogrīva\.m dhārayann acala\.m sthiraḥ}}

pratyāhāras tathā dhyāna\.m prāṇāyāmaś ca dhāraṇā\thinspace{\dandab} \dontdisplaylinenum
            \var{\va pratyāhāras ta°\lem  \msCa\msCb\msNa\msNb; pratyahāras ta° \Ed\oo
                 dhyāna\.m\lem  \msCa\msCb\msNb\Ed; dhyāna \msNa}%
            \var{\vb prāṇāyāmaś ca\lem  \msCa\msCb\msNa\msNb; prāṇāyāmañ ca \Ed}%

tarkaś caiva samādhiś ca ṣaḍaṅgo yoga ucyate \veg\dontdisplaylinenum
         \paral{\textit{\vo {\normalfont  = Dharmaputrikā 1.13 (with } prāṇāyāmo 'tha {\normalfont  ) }
             {\normalfont  cf.\ Sarvajñānottarav\textsubring{r}tti ad Yogapāda 27(?):1: } 
             yad ukta\.m śrīmanmataṅge{\thinspace\danda}
                prāṇāyāmas tathā dhyāna\.m pratyāhāro 'tha dhāraṇam{\thinspace\danda}
                tarkaś caiva samādhiś ca ṣaḍaṅgo yoga uccyate iti{\thinspace\ketdanda}}}

viṣayāsaktacittānām indriyāṇā\.m prati prati\thinspace{\dandab} \dontdisplaylinenum
            \var{\vb prati prati\lem  \msCb\msNa\msNb; pratisrati \msCa, pratiṣṭhati \Ed}%

manasākarṣayed yas tu pratyāhāraḥ sa ucyate \veg\dontdisplaylinenum
            \var{\vc manasā°\lem  \msCa\msCb\msNa\msNb; manamā° \Ed}%
            \var{\vd pratyāhāraḥ sa\lem  \msCa\msCb\msNa\msNb; pratyahāras tad \Ed}%
            \paral{\textit{\vo \kb\ {\normalfont  Dharmaputrikā 1.14: } 
                viṣayeṣv atisaktāni indriyāṇi prati prati{\thinspace\danda}
                cittenākarṣaṇa\.m yatra pratyāhāraḥ sa ucyate{\thinspace\ketdanda}}}

śabdādiviṣayān devi vartulīk\textsubring{r}tya dhārayet\thinspace{\dandab} \dontdisplaylinenum
            \var{\va °viṣayān de°\lem  \msCa\msNa\msNb\Ed; viṣayā de° \msCb}%
            \paral{\textit{\vo {\normalfont  cf.\ Dharmaputrikā 1.16cd: 
                } ekatra vartulīk\textsubring{r}tya dhyeye vastuni yojayet }}

vītarāgaḥ samādhistho dhyeye vastuni yojayet \veg\dontdisplaylinenum
            \var{\vc vītarāgaḥ\lem  \msCa\msNb; vītarāga° \msCb\msNa\Ed}%
            \var{\vd dhyeye vastuni\lem  \msCb\msNa; dhyeyastuni \msCa, dhyeyavastuni \msNb\Ed}%

ātmā dhyātā mano dhyāna\.m dhyeyaḥ śuddhaḥ paraḥ śivaḥ\thinspace{\dandab} \dontdisplaylinenum
            \var{\va ātmā\lem  \msCa\msCb\msNa\msNb; ātma \Ed\oo
                 dhyātā\lem  \msCa\msCb\msNa\Ed; dhyāta\.m \msNb}%
            \var{\vb paraḥ śivaḥ\lem  \msCb\msNa\msNb; paraśivaḥ \msCa\Ed\ \unmetr}%
            \paral{\textit{\vo \kb\ {\normalfont Sarvajñānottara Yogapāda 27(?):4: }
               ātmā dhyātā mano dhyāna\.m dhyeyaḥ sūkṣmo maheśvaraḥ{\thinspace\danda}
               yat para\.m paramaiśvaryam etad dhyānaprayojanam{\thinspace\ketdanda}
               \vab \kb\ {\normalfont  Agnipurāṇa 165.22cd: }
                ātmā dhyātā mano dhyāna\.m dhyeyo viṣṇuḥ phala\.m hariḥ
                {\normalfont  \kb\ Dharmaputrikā 1.18: }
                dhyeyaḥ śivo dhyāt\textsubring{r} mano dhyānam ekāgracittatā{\thinspace\danda}
                duḥkhahānir guṇaiśvarya\.m svātantryañ ca prayojanam{\thinspace\ketdanda}}}

yat para\.m paramaiśvaryam eka\.m tatra prayojanam \veg\dontdisplaylinenum
            \var{\vcd °m eka\.m tatra\lem  \msCa\msCb\msNa\Ed; °m etat tatra \msNb}%

pūrakaḥ kumbhakaś caiva recakas tadanantaram\thinspace{\dandab} \dontdisplaylinenum
            \paral{\textit{\vo = {\normalfont  Dharmaputrikā 1.19ab (with } caiva{\normalfont  for } ceti{\normalfont  )} }}

praśāntaś ceti vikhyātaḥ prāṇāyāmaś caturvidhaḥ \veg\dontdisplaylinenum
            \var{\vc praśānta°\lem  \msCb\msNa\msNb\Ed; \uncl{pra}śānta° \msCa\oo
                 vikhyātaḥ\lem  \msCa\msCb\msNb\Ed; vikhyātāḥ \msNa}%
            \var{\vd °vidhaḥ\lem  \msCa\msCb\msNa\msNb; °vidhāḥ \Ed}%
            \paral{\textit{\vcd {\normalfont  See NiśvāsaNaya 4:113: }
                nābhyā\.m h\textsubring{r}dayasa\.mcārān manaś cendriyagocarāt{\thinspace\danda}
                prāṇāyāmaś caturthas tu supraśāntas tu viśrutaḥ{\thinspace\ketdanda} 
                {\normalfont  See also Svaccandatantra 7.298ab: }
                prāṇāyāmaś caturthas tu supraśānta iti śrutaḥ}} 

pūrake sthāpayed vahni\.m pādāṅguṣṭhena buddhimān\thinspace{\dandab} \dontdisplaylinenum
            \var{\va pūrake\lem  \eme; pūrakaḥ \msCa\msCb\msNa\msNb\Ed\oo
                 vahni\.m\lem  \msCa\msCb\msNa;  vahni \msNb\Ed}%
            \var{\vb °ṣṭhena\lem  \msCa\msCb\msNb\Ed; °ṣṭheṣu \msNa}%

kumbhakena virudhyeta dahyamāna\.m vicintayet \veg\dontdisplaylinenum
            \var{\vcd virudhyeta dahyamāna\.m\lem  \msCa\msCb\msNa; nirudhyeta dahyamānam \msNb,
                                                                               nirudhyeta daihyamāna \Ed}%

bhasmībhūta\.m tathātmāna\.m recakena vicintayet\thinspace{\dandab} \dontdisplaylinenum

śuddhadehas tataś cātmā śuddhasphaṭikanirmalaḥ \veg\dontdisplaylinenum

tālaśabdās tu nirvāṇa\.m daśa dve ca prakīrtitaḥ\thinspace{\dandab} \dontdisplaylinenum
            \var{\va tālaśabdā°\lem  \eme; tālāśabda° \msCa\msCb\msNa\msNb\Ed\oo
                 nirvāṇa\.m\lem  \msCa\msCb\Ed; nirvāṇa \msNa, \uncl{nirvvā}na\.m \msNb}%

prāṇāyāmān na sa\.mdeho dviguṇā dhāraṇā sm\textsubring{r}tā \veg\dontdisplaylinenum
            \var{\vc prāṇāyāmān na\lem  \msCa\msNa\msNb\Ed; prāṇāyān na \msCb}%
            \var{\vd sm\textsubring{r}tā\lem  \msCa\msCb\msNa\msNb; sm\textsubring{r}tāḥ \Ed}%

yoge tu triguṇā proktā sa\.mkrame ca caturguṇā\thinspace{\dandab} \dontdisplaylinenum
            \var{\va °guṇā\lem  \msCa\msCb\msNb\Ed; °guṇāḥ \msNa}%
            \var{\vab proktā sa\.mkrame ca caturguṇā\lem  \msCa\msCb; 
                         proktāḥ sa\.mkrame ca caturguṇā \msNa,
                         proktā\.m sa\.mkrame ca caturguṇā \msNb,
                         proktāḥ sa\.mkrameṇa caturguṇāḥ \Ed}%

! tathotkrāntau pañcaguṇā yogasiddhis tu ṣaḍguṇā \veg\dontdisplaylinenum
            \var{\vc tathotkrāntau\lem  \msCa\msCb\msNa\msNb; tathākratau \Ed}%
            \var{\vd ṣaḍguṇā\lem  \eme; ṣaḍguṇāḥ \msCa\msCb\msNa\msNb\Ed}%

ṣaḍaṅgena samāyukto yogayuktas tu nityaśaḥ\thinspace{\dandab} \dontdisplaylinenum
            \var{\vb yogayuktas tu\lem  \msCa\msCb\msNa\msNb; yogamuktas tu \Ed}%

mānaso yaugapadyaś ca dvirūpo yoga ucyate \veg\dontdisplaylinenum
            \var{\vcd yaugapadyaś ca dvirūpo\lem  \msNa; 
                yaugapadyaś \uncl{ca} {\il}{i}{\il}{\il} \msCa,
                yogapadyaś ca dvirūpo \msCb\msNb,
                yogapadyañ ca dvirūpo \Ed}%
            \paral{\textit{\vcd = {\normalfont  Dharmaputrikā 1.54ab.} }}

ak\textsubring{r}tvā prāṇasa\.mrodha\.m manasaikena kevalam\thinspace{\dandab} \dontdisplaylinenum
            \var{\va °sa\.mrodha\.m\lem  \msCa\msNb\Ed; °sa\.mrodha \msCb\msNa}%
            \var{\vb manasaikena\lem  \msCb\msNb\Ed; manasekena \msCa\msNa}%

dhyāyeta parama\.m sūkṣma\.m sa yogo mānasaḥ sm\textsubring{r}taḥ \veg\dontdisplaylinenum
            \var{\vc dhyāyeta pa°\lem  \msCa\msCb\msNa\Ed; dhyāyetat pa \msNb}%
            \var{\vd sa yogo\lem  \msCa\msNa\msNb\Ed; sa\.myogo \msCb\oo
                 mānasaḥ\lem  \msCapcorr\msCb\msNb\Ed; mānasa \msCaacorr\msNa\oo
                 sm\textsubring{r}taḥ\lem  \msCa\msCb\msNa\msNb; sm\textsubring{r}tam \Ed}%
            \paral{\textit{\vo = {\normalfont  Dharmaputrikā 1.54cd--55ab.} }}

sa\.myamya manasā prāṇa\.m prāṇāyāmān manas tathā\thinspace{\dandab} \dontdisplaylinenum
            \var{\va sa\.myamya\lem  \msCa\msCb\msNa\Ed; sayamya \msNb\oo
                 prāṇa\.m\lem  \msCa\msNa\msNb\Ed; \om\ \msCb}%
            \var{\vb prāṇāyāmān ma°\lem  \eme;
                prāṇāyāmām ma° \msCa\msNb, prāṇāyāmā ma° \msCb, prāṇāyāma\.m ma° \msNa, prāṇāyāmātma° \Ed}%

eva\.m dhyāyet para\.m sūkṣma\.m yaugapadyaḥ sa ucyate \veg\dontdisplaylinenum
            \var{\vb yaugapadyaḥ\lem  \msCa\msCb\Ed; yogapadyaḥ \msNa, yogapadya \msNb}%
            \paral{\textit{\vo \kb\ {\normalfont  Dharmaputrikā 1.55cd--56ab: }
                sa\.myamya manasā prāṇa\.m prāṇāyāmair manas tathā{\thinspace\danda}
                eva\.m dhyāyet para\.m sūkṣma\.m yaugapadyaḥ sa ucyate{\thinspace\ketdanda}}}


\alalfejezet{siddhilakṣaṇam}
siddhilakṣaṇa yogasya ś\textsubring{r}ṇu vakṣyāmi sundari\thinspace{\dandab} \dontdisplaylinenum
            \var{\va siddhi°\lem  \msCa\msCb\msNa\msNb; siddhir \Ed}%

śaṅkhabherīm\textsubring{r}daṅga\.m ca veṇudundubhim eva ca \danda\dontdisplaylinenum
            \var{\vc śaṅkhabherīm\textsubring{r}daṅga\.m ca\lem  \msNb; śaṅkha{\il}{\il}{\il}{\il}{\il}ś ca \msCa, śaṅkhabherīm\textsubring{r}daṅgaś ca \msCb\msNa\Ed}%
            \var{\vd °dundubhim eva\lem  \msCa\msCb\msNa\msNb; °dundubhir eva \Ed}%

tāḍita\.m na ca vindeta yadā tanmayatā\.m gataḥ \veg\dontdisplaylinenum
            \paral{\textit{\vo \kb\ {\normalfont  Kulasāra f.\ 38r: }
                 śa\.mkhabherīm\textsubring{r}da\.mgaiś ca vīṇāveṇuśatair api{\thinspace\danda} 
                 tāḍyamānair na vindeta yadā tanmayatā\.m gataḥ{\thinspace\ketdanda}}}
            \paral{\textit{\vef {\normalfont  cf. NiśvāsaMukha 4:65: }
                tāḍitañ ca na vindeta cakṣuṣā na ca paśyati{\thinspace\danda}
                divyad\textsubring{r}ṣṭiḥ prajāyeta yadā tanmayatāṅ gataḥ{\thinspace\ketdanda}}}

śītoṣṇa\.m sukhaduḥkha\.m ca t\textsubring{r}ṣṇābhukṣa\.m tathaiva ca\thinspace{\dandab} \dontdisplaylinenum
            \var{\vb t\textsubring{r}ṣṇābhukṣa\.m\lem  \msCa\msCb\msNa\msNb; t\textsubring{r}ḍbubhukṣā\.m \Ed}%

vedanā\.m naiva jānāti yogasiddhas tu sundari \veg\dontdisplaylinenum
            \var{\va vedanā\.m\lem  \msNa; vedanān \msCa\msCb, vedanā \msNb\Ed}%
            \var{\vb °siddha°\lem  \msCa\msNa; °siddhi° \msCb\msNb, °yukta° \Ed}%

eṣa yogavidhir devi tava p\textsubring{r}ṣṭena sundari\thinspace{\dandab} \dontdisplaylinenum

kathito 'smi samāsena kim anyat kathayāmy aham \veg\dontdisplaylinenum

devy uvāca~{\dandab}\dontdisplaylinenum 

vinā yogena deveśa sa\.msāratāraṇa\.m mama\thinspace{\danda} \dontdisplaylinenum
            \var{\va deveśa\lem  \msCb\msNa\msNb\Ed; veśa \msCa}%
            \var{\vb sa\.msāratāraṇa\.m mama\lem  \msCa\msCb\Ed; sa\.msārāt tāraṇa\.m mama \msNa, sa\.msārārṇṇavatāraṇa \msNb}%

kathayasva mahādeva nirvikalpakara\.m manaḥ \veg\dontdisplaylinenum

maheśvara uvāca~{\dandab}\dontdisplaylinenum 
            \var{\vo maheśvara\lem  \msCa\msCb\msNb; deveśa \msNa; bhagavān \Ed}%

sadāśivas tu niśvāsa ūrdhvaśvāsaḥ paraḥ śivaḥ\thinspace{\danda} \dontdisplaylinenum
            \var{\vd ūrdhvaśvāsaḥ\lem  \msCa\msCb\msNa; ūrdhvaśvāsa \msNb, arddhaśvāsaḥ \Ed}%

tayor madhye tu vijñeyaḥ paramātmā śivo 'vyayaḥ \veg\dontdisplaylinenum

dhyānayoga\.m na tasyāsti karaṇa\.m ca na vidyate\thinspace{\dandab} \dontdisplaylinenum

jñātamātreṇa mucyante kim anyat parip\textsubring{r}cchasi \veg\dontdisplaylinenum
            \var{\vc jñāta°\lem  \msCa\msCb\msNa\msNb; jñāna° \Ed}%
            \var{\vcd mucyante kim anyat pa°\lem  \msCb\msNa\Ed; mucya\uncl{nte}{\il}m anyat pa° \msCa,
                                                \uncl{mucya}nte kim at pa° \msNb}%


\alalfejezet{pañca śāstrāṇi}
jñānam anyat pravakṣyāmi ś\textsubring{r}ṇu devi nibodha me\thinspace{\dandab} \dontdisplaylinenum

śāstrapañcasu yat prokta\.m ś\textsubring{r}ṇu sa\.mkṣepa nirṇayam \danda\dontdisplaylinenum
            \var{\vd sa\.mkṣepa\lem  \msCb\msNa\msNb\Ed; sa\.mkṣepe \msCa\ \unmetr}%

sā\.mkhye yoge pañcarātre śaive vede ca nirmitam \veg\dontdisplaylinenum
            \var{\ve sā\.mkhye\lem  \msCa\msCb\msNa\msNb; sā\.mkhya° \Ed\oo
                 pañca°\lem  \msCa\msCb\msNb\Ed; paca° \msNa}%
            \var{\vf śaive\lem  \msCa\msCb\msNa\msNb; śaiva° \Ed}%

\ujvers\nemsloka 
yat sā\.mkhyasiddha\.m kathayāmy aha\.m te
\dontdisplaylinenum
            \var{\va °siddha\.m\lem  \msCa\msCb\msNa\Ed; °siddhi\.m \msNb\oo
                 te\lem  \msCa\msCb\msNapcorr\msNb\Ed; \om\ \msNaacorr}%

\nemslokab 
sa\.msāraghorārṇavayogasāram \danda\dontdisplaylinenum
            \var{\vb °rṇava°\lem  \msCa\msCb\msNb\Ed; °ṇṇa° \msNaacorr, °ṇṇava° \msNapcorr\oo 
                 °sāram\lem  \msCa\msNa\msNb\Ed; sāgaram \msCb}%

\nemslokac 
yogeṣu sāreṣv atha pañcarātre
\dontdisplaylinenum
            \var{\vc °ṣv atha\lem  \msCa\msNa\msNb\Ed; °ṣv etha \msCb\oo
                 pañcarātre\lem  \msCb\msNa\msNb\Ed; pañca\uncl{rātre} \msCa}%

\nemslokad 
vedeṣu śaiveṣu ca niścayas te \veg\dontdisplaylinenum
            \var{\vd vedeṣu\lem  \msCb\msNa\Ed; {\il}deṣu \msCa, deveṣu \msNb\oo
                 niścayas te\lem  \msCa;  niścayan te \msCb\Ed, niścayās te \msNa, niścaya\uncl{sve} \msNb}%

\ujvers\nemsloka 
ghrāṇendriyādyeṣu ca yat samastam
\dontdisplaylinenum

\nemslokab 
manaś ca līna\.m bhavatīva yasya \danda\dontdisplaylinenum
            \var{\vb manaś ca\lem  \msCa\msCb\msNa\msNb; nabhaś ca \Ed}%

\nemslokac 
! buddhyā niyamya sakalān hi bhāvān
\dontdisplaylinenum
            \var{\vc sakalān hi\lem  \corr; sakalā\.m hi \msCa\msNa\msNb, sakalā hi \msCb, śakalā\.m hi \Ed}%

\nemslokad 
sa labdhalakṣyaḥ śivam abhyupaiti \veg\dontdisplaylinenum
            \var{\vd °lakṣyaḥ\lem  \msCa\msCb\msNb; °lakṣya° \msNa\Ed\oo
                 °paiti\lem  \msCa\msNa\msNb\Ed; °peti \msCb}%

\ujvers\nemsloka 
śrotrādisarvendriyaniścalatve
\dontdisplaylinenum
            \var{\va °calatve\lem  \emeHaru; °calatvam \msCa\msCb\msNa\msNb\Ed}%

\nemslokab 
ekāgracitta\.m manasā niyamya \danda\dontdisplaylinenum

\nemslokac 
svadehaśūnyaḥ sa bhavec cireṇa
\dontdisplaylinenum
            \var{\vc °śūnyaḥ\lem  \msCa\msNa\msNb\Ed; °śūnya\.m \msCb}%

\nemslokad 
sa\.myogasiddhi\.m pravadanti tajjñāḥ \veg\dontdisplaylinenum
            \var{\vd sa\.myogasiddhi\.m\lem  \msNa; sa\.myogasi{\il} \msCa, sa\.mgasiddhi\.m \msCb, sa yogasiddhi\.m \msNb\Ed}%

\nemslokalong


\ujvers\nemsloka 
ādāv eva manaḥ śanair uparamet k\textsubring{r}tvā ca vaśyendriya\.m
\dontdisplaylinenum
    \var{\va uparamet k\textsubring{r}°\lem  \msCa\msCb\msNb\Ed; uparame k\textsubring{r}° \msNa\oo
                 °ndriyam\lem  \msCa\msNa\msNb\Ed; °ndriyaḥ \msCb}%

\nemslokab 
yāvat tal layatā\.m vrajeta manasā niḥsa\.mjñadehas tathā \danda\dontdisplaylinenum
            \var{\vb tallayatā\.m\lem  \msCa\msCb\msNa\msNb; tattapatā\.m \Ed\oo
                 manasā niḥsa\.mjña°\lem  \Ed; manasān nissa\.mjña° \msCa, manasā\.m niḥsa\.mjña° \msCb, manasān nisa\.mjña° \msNa,
                                                manasān nissajña° \msNb}%

\nemslokac 
etad dhyānasamādhiyogasakala\.m prāpnoti niḥsa\.mśaya\.m
\dontdisplaylinenum
            \var{\vc °sa\.mśayam\lem  \msCa\msNb\Ed; °sa\.mśayaḥ \msCb\msNa}%

\nemslokad 
ki\.m tac chāstrasahasrakoṭipaṭhita\.m sāra\.m na yo 'nviṣyati \veg\dontdisplaylinenum
            \var{\vd ki\.m ta°\lem  \msCa\msCb\msNa\msNb; citsa° \Ed\oo
                 °koṭi°\lem  \msCb\msNa\msNb\Ed; °ṭoki° \msCa\oo
                 °paṭhita\.m\lem  \msCa\msCb\msNa\msNb; °mathita\.m \Ed\oo 
                 na yo 'nviṣyati\lem  \msCa\msCb; na yo 'nviṣyate \msNa\msNb, tayer iṣyati \Ed}%

\ujvers\nemsloka 
ātmārāmajitaḥ samādhinirato vairāgyam apy āśritaḥ
\dontdisplaylinenum
            \var{\va ātmārāmajitaḥ\lem  \msCb\msNa\msNb; ātmārā\uncl{ma}{\il}{\il} \msCa, ātmārāmaḥ jitaḥ \Ed\oo
                 vairāgyam apy āśritaḥ\lem  \msCa\msCb\msNa\msNb; vairāgaśayyāśritaḥ \Ed}%

\nemslokab 
citta\.m yasya parikṣayo yadi bhavet tiṣṭhet tanutva\.m yathā \danda\dontdisplaylinenum
            \var{\vb pari°\lem  \msCa\msCb\msNb\Ed; parī° \msNa}%

\nemslokac 
taj jñeya\.m gatim uttama\.m śivapada\.m sa\.msāraduḥkhacchida\.m
\dontdisplaylinenum
            \paral{\textit{\vc {\normalfont  cf.\ 22.41d: } uttamā\.m gatim āpnuyāt}}

\nemslokad 
vedānteṣu ca niṣṭha eṣa kathitaḥ ki\.m śāstram anyad viśet \veg\dontdisplaylinenum
            \var{\vd anyad vi°\lem  \msCa\msCb\msNb\Ed; anya\.m vi° \msNa}%

\ujvers\nemsloka 
h\textsubring{r}tpadme karṇikāyām upari ravir avadyotayanto 'ntarālam
\dontdisplaylinenum
            \var{\va °padme\lem  \conj; °padma° \msCa\msCb\msNa\msNb\Ed\unmetr\oo
                 ravir ava°\lem  \eme; ravirava\.m° \msCa\msCb\msNa\msNb, ravirata° \Ed}%

\nemslokab 
yattejastejamārgair bahalatamaghanair dyotanād dīptadīpam \danda\dontdisplaylinenum
            \var{\vb yat te°\lem  \msCb; yas te° \msCa\msNa\msNb\Ed\oo
                 °mārgair bahala°\lem  \msCa\msCb\msNb; °mārgai bahala° \msNa, °mārgau bahula° \Ed\oo
                 °tamaghanair dyotanād dīptadīpam\lem  \conj; 
                                         °tamaghanair ghātanād dīptadīpam \msCa, 
                                         °maghanai ghāṭanādīptadīpam \msCb,     
                                         °tamaghanair ghāṭanādīptadīpam \msNa, 
                                         °tamaghanai ghāṭanādīptadīpam \msNb, 
                                         °tamaghanair dyotanād dīptadīpaḥ \Ed}%

\nemslokac 
bhittvā yat tāludeśe mukham uparigata\.m tāludeśena mūrdhni
\dontdisplaylinenum
            \var{\vc yat tālu°\lem  \Ed; gha\.mṭṭāla° \msCa, ghatola° \msCb ghaṇṭāla° \msNa\msNb\oo
                 °gata\.m\lem  \Ed; °gata° \msCa\msNa\msNb, °gatas° \msCb}%

\nemslokad 
! mūrdhni dvārāntareṇa śivaparamapada\.m yānti yogena yuktāḥ \veg\dontdisplaylinenum
            \var{\vd mūrdhni\lem  \msNa; mūrdhna \msCa\msCb\msNb, mūrdhnyā \Ed}%

\ujvers\nemsloka 
k\textsubring{r}ṣṇaḥ k\textsubring{r}ṣṇatamottamo 'timahato yas tejatejātmakaḥ
\dontdisplaylinenum
            \var{\va k\textsubring{r}ṣṇaḥ\lem  \eme\ \Kafle; k\textsubring{r}ṣṇa\.m \msCa\msCb\msNa\msNb; k\textsubring{r}tsna\.m \Ed\oo
                 °tamottamo\lem  \conj; °tamotamo \msCa\msCb\msNa\msNb\Ed\oo
                 'ti°\lem  \msCa\msNa\msNb\Ed; hi \msCb\oo
                 yas tejate°\lem  \Ed; yas tejaste° \msCa\msCb\msNa\msNb\ \unmetr}%

\nemslokab 
lokālokadharādharaḥ śriyapatiḥ prāṇapraviṣṭālayaḥ \danda\dontdisplaylinenum
            \var{\vb °dharādharaḥ śriyapatiḥ\lem  \Ed; 
                      °dharo dharādharadharaḥ \msCa\msCb\msNb,
                      °dharo dharadharadharaḥ \msNa\ \unmetr\oo
                 śriyapatiḥ\lem  \msCa\msCb\msNa\Ed; \om\ \msNb\oo
                 °praviṣṭālayaḥ\lem  \msCb\msNa\msNb; °\uncl{pra}viṣṭo layaḥ \msCa, pratiṣṭhālayaḥ \Ed}%

\nemslokac 
kartā kāraṇam avyayo 'vyayam asau vyāpī vibhaktāvidam
\dontdisplaylinenum

\nemslokad 
viṣṇur bhāvamayo vibhaktaviṣayair viśveśvaro viśvavit \veg\dontdisplaylinenum
            \var{\vd bhāvamayo\lem  \msCa\msCb\msNa\msNb; bhāvamayair \Ed}%

\ujvers\nemsloka 
! eṣa tattvavaraḥ parāparamayas tejaḥ parasthānadaḥ
\dontdisplaylinenum
            \var{\va parāparamaya°\lem  \conj; paraḥ paramaya° \msCa\msNa\msNb\Ed, paraḥ parama° \msCb\oo
                 °parasthā°\lem  \conj; °paraḥ sthā° \msCa\msCb\msNa\msNb\Ed}%

\nemslokab 
buddhyā bhāvanabhāvayendriyamano dehāntar ālokayan \danda\dontdisplaylinenum
            \var{\vb °bhāvayendriyamano\lem  \msCa\msNa\msNb; °bhāvayandriyamano \msCb, °bhāvayan niyamano \Ed\oo
                 dehāntar ālokayan\lem  \msCa\msNa\msNb; dehāntarālokayat \msCb, dehāntarostokayan \Ed}%

\nemslokac 
h\textsubring{r}tpadmāyatanasthitaḥ sa puruṣo niśvāsam ucchvāsadaḥ
\dontdisplaylinenum
            \var{\vc sa puruṣo ni°\lem  \msNa\msNb\Ed; \uncl{sa puruṣo} {\il} \msCa, puruṣau ni° \msCb\oo
                °cchvāsadaḥ\lem  \msCa\msCb\msNa\msNb; °cchvāsadām \Ed}%

\nemslokad 
nādas tasya sadā sadā nadati ta\.m nādopariṣṭhā haraḥ \veg\dontdisplaylinenum
            \var{\vd nādas tasya\lem  \msCa\msCb\msNa\msNb; nādantasya \Ed\oo
                 nadati ta\.m\lem  \msCa\msCb\msNa\msNb; na patita\.m \Ed\oo
                 °pariṣṭhā haraḥ\lem  \msCa\msCb\msNa\msNb; °pariṣṭadvaraḥ \Ed}%

\ujvers\nemsloka 
yas tejas tejate 'jo bahuniviḍaghano granthimālopagūḍhaḥ
\dontdisplaylinenum
            \var{\va yas tejas tejate 'jo\lem  \conj;
                        yas tejas tejas tejo \msCa\msCb\msNa\msNb\ \unmetr\ 
                        yas tejas tejaso vā \Ed\oo
                 °niviḍa°\lem  \msCa\msCb\msNa\msNb; °nividu° \Ed\oo
                 °ghano\lem  \msCb; °ghanaḥ \msCa\msNa\msNb\Ed\oo
                 granthimālo°\lem  \msCa\msNa\msNb; gratthimāno° \msCb\Ed}%

\nemslokab 
mūrtir mūrtānusārī bahukaraṇabh\textsubring{r}ta\.m kāraṇād dehabandhaḥ \danda\dontdisplaylinenum
            \var{\vb mūrtir mūrtā°\lem  \msCa; mūrtimūrtā° \msCb\msNa\msNb, mūrtir mūrtya° \Ed\oo
                 bahu°\lem  \msCa\msNa\msNb\Ed; bahya° \msCb\unmetr\oo
                 °bh\textsubring{r}ta\.m\lem  \msCa\msCb\msNa\Ed; °v\textsubring{r}ta\.m \msNb\oo
                 kāraṇād de°\lem  \msCa\msCb\msNa\msNb; kāraṇa\.m de° \Ed}%

\nemslokac 
bhittvā granthi\.m sapāśa\.m viṣam iva viṣaya\.m tyaktasaṅgaikabhāvāḥ
\dontdisplaylinenum
            \var{\vc sapāśa\.m\lem  \msNa\Ed; sapāśā\.m \msCa\msCb\msNb\oo
                 °saṅgaika°\lem  \msCb\msNa\msNb\Ed; °saṅsaika° \msCa}%

\nemslokad 
paśyanty ete tam īśa\.m guṇakalarahita\.m nirvikāra\.m prakāśam \veg\dontdisplaylinenum
            \var{\vb paśyanty ete tam ī°\lem  \msCb\msNa\msNb; paśyanty e{\il}{\il}m ī° \msCa, paśyanty etenam ī° \Ed}%

\ujvers\nemsloka 
yo 'sau tejāntarātmā kamalapuṭakuṭīsa\.mkaṭasthānalīnaḥ
\dontdisplaylinenum
            \var{\va \om\ \msCaacorr\oo
                yo 'sau tejāntarātmā\lem  \msCb\msNa\Ed; {\il}{\il}{\il}\uncl{jānta}rātmā \msCapcorr, \om\ \msCaacorr, 
                                                yo sau tejāntarāla° \msNb\oo
                °kuṭī°\lem  \msCapcorr\msCb\msNa\msNb; \om\ \msCaacorr, °kuṭi° \Ed}%

\nemslokab 
indor bhāsānurūpī vimaladalasadācchāditaḥ karṇikāyām \danda\dontdisplaylinenum
            \var{\vb indor bhā°\lem  \msCa\msNb\Ed; indo bhā° \msCb\msNa\oo 
                 °rūpī\lem  \msCa\msCb\msNa\msNb; °rūpi \Ed\unmetr\oo
                 °cchāditaḥ\lem  \msCa\msCb\msNa\msNbpcorr\Ed; °cchādi \msNbacorr}%

\nemslokac 
tatra sthāne sthito 'sau tribhuvananilayaḥ sarvabhūtādhivāsaḥ
\dontdisplaylinenum

\nemslokad 
ākāśād ūrdhvatattvasthitavikasakalāsa\.mhato muktabandhaḥ \veg\dontdisplaylinenum
            \var{\vd °sthita°\lem  \conj; °sita° \msCa\msCb\msNa\msNb\Ed\ \unmetr\oo 
                 °kalāsa\.mhato\lem  \Ed; °kasāsa\.mhato \msCa\msCb\msNa\msNb\oo
                 mukta°\lem  \conj; mukti° \msCa\msCb\msNa\msNb\Ed}%

\nemslokanormal


\ujvers\nemsloka 
etāni tattvāny akhilāni devi
\dontdisplaylinenum
            \var{\va akhilāni\lem  \msCa\msNa\msNb\Ed; akhikāti \msCb\oo 
                 devi\lem  \msCb\msNa\msNb\Ed; \uncl{de}{\il} \msCa}%

\nemslokab 
! sa\.mkṣepataḥ kīrtitaḥ pañcabhedaḥ \danda\dontdisplaylinenum

\nemslokac 
śrotu\.m kim anyad vijigīṣitārtham
\dontdisplaylinenum
            \var{\vc śrotu\.m kim\lem  \msCa\msCb\msNa\msNb; śrotakim \Ed\oo
                 vijigīṣitā°\lem  \msCa\msNa\msNb\Ed; vijigīṣatā° \msCb}%

\nemslokad 
sa\.msāramokṣeṇa ca tatparo 'sti \veg\dontdisplaylinenum

\vers

devy uvāca~{\dandab}\dontdisplaylinenum 

\nemsloka 
tuṣṭāsmi deva mama sa\.mśayam adya naṣṭam
\dontdisplaylinenum
            \var{\va tuṣṭā°\lem  \msCa\msNa\msNb; tu\uncl{ṣṭā}° \msCb, tuṣṭo \Ed}%

\nemslokab 
adya prasannaparameśvara īśvara tvam \danda\dontdisplaylinenum
            \var{\vb °parameśvara\lem  \msCa\msCb\msNa\Ed; °parameraśvara \msNb\oo
                 īśvara\lem  \msCa\msCb\msNa\msNb; īśvarama \Ed}%

\nemslokac 
adya śruta\.m tvayi ca puṇyaphalaprabhāvam
\dontdisplaylinenum
            \var{\vc \om\ \msCb}%

\nemslokad 
pūrṇāni cādya mama iṣṭamanorathāni \veg\dontdisplaylinenum
            \var{\vd iṣṭamanorathāni\lem  \msCb\msNa\msNb\Ed; \uncl{iṣṭa}{\il}{\il}{\il}thāni \msCa}%

\ujvers\nemsloka 
ajñānapaṅkaghanamadhyanilīyamānām
\dontdisplaylinenum
            \var{\va °nilīyamānām\lem  \msCa\msNa; °nilīyamānam \msCb\msNb\Ed}%

\nemslokab 
uttārayeśa sakalārtivināśanāya \danda\dontdisplaylinenum
            \var{\vb uttārayeśa\lem  \msCb\msNa\msNb\Ed; uttarāyeśa \msCaacorr, uttarayeśa \msCapcorr}%

\nemslokac 
sarveśa tattvaparamārtha namo namas te
\dontdisplaylinenum

\nemslokad 
adyāpi t\textsubring{r}ptir iha nāsti mamāpi śambho \veg\dontdisplaylinenum
            \var{\vd nāsti mamāpi\lem  \msCa\msCb\msNapcorr\msNb\Ed; nā pi \msNaacorr}%

\ujvers\nemsloka 
pītvām\textsubring{r}ta\.m cottamavaktrajātam
\dontdisplaylinenum
            \var{\va °vaktra°\lem  \msCa\msCb\msNb\Ed; °vacaktra° \msNaacorr, °caktra° \msNapcorr}%

\nemslokab 
ākhyāhi dāna\.m phaladharmasāram \danda\dontdisplaylinenum
            \paral{\textit{\vb {\normalfont After pāda b, \Ed\ inserts this line, which is absent in all
                        the sources consulted: } k\textsubring{r}pā\.m mayīśāna kuru prasīda }}

\nemslokad 
sa\.msārapāra\.m parama\.m nayasva \veg\dontdisplaylinenum
            \var{\vd parama\.m\lem  \msCa\msCb\msNa\Ed; parama \msNb\oo
                 nayasva\lem  \msCb\msNa\msNb\Ed; naya{\il} \msCa}%

\vers


\jump
\begin{center}
\ketdanda iti v\textsubring{r}ṣasārasa\.mgrahe 'dhyātmanirṇayo nāmādhyāyaḥ ṣoḍaśamaḥ\ketdanda
\end{center}
\dontdisplaylinenum\vers 
            \var{{\normalfont Colophon:} 'dhyātma°\lem  \corr; adhyātma° \msCa\msNa\msNb\Ed, ātma° \msCb\oo 
                        nāmādhyāyaḥ ṣoḍaśamaḥ\lem  \msCa\msCb\msNa\msNb; 
                             nāma ṣoḍaśo 'dhyāyaḥ \Ed}%

\vers
\bekveg\szamveg\vfill\phpspagebreak\szam\bek\versno=0\fejno=17
\thispagestyle{empty}



\alfejezet{\textbf{17 dānadharmaviśeṣaḥ}}\jump\jump
\vers

devy uvāca~{\dandab}\dontdisplaylinenum 

p\textsubring{r}thagdānasya icchāmi śrotu\.m mā\.m dātum arhasi\thinspace{\danda} \dontdisplaylinenum
            \var{\vb śrotu\.m mā\.m dātum arhasi\lem  \msCa; 
                māhātmya\.m vaktum arhasi \Ed}%

annavastrahiraṇyānā\.m gobhūmikanakasya ca \veg\dontdisplaylinenum

bhagavān uvāca~{\dandab}\dontdisplaylinenum 

\nemsloka 
! susa\.msk\textsubring{r}tam annam atipradadyāt
\dontdisplaylinenum

\nemslokab 
! gh\textsubring{r}taprabhūtam avada\.mśayuktam \danda\dontdisplaylinenum

\nemslokac 
gh\textsubring{r}taprapakva\.m suk\textsubring{r}ta\.m ca pūpa\.m
\dontdisplaylinenum
            \var{\vc suk\textsubring{r}ta\.m ca pūpa\.m\lem  \msCa; 
                        suk\textsubring{r}tammapūpa\.m \Ed}%

\nemslokad 
sitena khaṇḍena guḍena yuktam \veg\dontdisplaylinenum

\ujvers\nemsloka 
mārga\.m khagaś codakajaṅgamaś ca
\dontdisplaylinenum
            \var{\va mārga\.m\lem  \msCa; mārga° \Ed\ \unmetr\oo
                 khagaś\lem  \Ed; khañ \msCa\oo
                 °jaṅgala\.m ca\lem  \msCa; °jaṅgamaś ca \Ed}%

\nemslokab 
dadyād vaṭa\.m nāgarava\.mśamūlam \danda\dontdisplaylinenum
            \var{\vb vaṭa\.m\lem  \msCa; vaṭa \Ed\ \unmetr}%

\nemslokac 
śāka\.m phala\.m cāmlamadhūratiktam
\dontdisplaylinenum

\nemslokad 
pāna\.m payaḥ śītasugandhatoyam \veg\dontdisplaylinenum

\ujvers\nemsloka 
dadhi pradadyād guḍamiśrita\.m ca
\dontdisplaylinenum

\nemslokab 
m\textsubring{r}ṇālaśālūkavanālakā ca \danda\dontdisplaylinenum

\nemslokac 
sadakṣiṇālepapavitrapuṣpam
\dontdisplaylinenum

\nemslokad 
śraddhānvitaḥ satk\textsubring{r}tayā praṇamya \veg\dontdisplaylinenum
            \var{\vd satk\textsubring{r}tayā\lem  \msCa; saktatayā \Ed}%

\ujvers\nemsloka 
prayāti loka\.m jagadīśvarasya
\dontdisplaylinenum

\nemslokab 
vimānayānaiḥ sahito 'psarobhiḥ \danda\dontdisplaylinenum

\nemslokac 
ekaikasiṣṭasya sahasravarṣam
\dontdisplaylinenum

\nemslokad 
annaprado modati devaloke \veg\dontdisplaylinenum

\ujvers\nemsloka 
cyutaś ca martye sa bhaved dhanāḍhyaḥ
\dontdisplaylinenum

\nemslokab 
kulodgataḥ sarvaguṇopapannaḥ \danda\dontdisplaylinenum

\nemslokac 
yaśaḥ śriya\.m sarvakalajñatā ca
\dontdisplaylinenum

\nemslokad 
bhavet sa bhogī sakalatraputraḥ \veg\dontdisplaylinenum

\ujvers\nemsloka 
dadyād daridraḥ k\textsubring{r}paṇārtadīno
\dontdisplaylinenum

\nemslokab 
bālāgadatvāturamāgatānām \danda\dontdisplaylinenum

\nemslokac 
t\textsubring{r}ṣṇābubhukṣāgatikāgatānām
\dontdisplaylinenum

\nemslokad 
dattvā sadharmasya phala\.m kaniṣṭa \veg\dontdisplaylinenum

\ujvers\nemsloka 
vāṇijyadharmādiphalāśritānām
\dontdisplaylinenum

\nemslokab 
dharmo hi tasya na ca nirmalo 'sti \danda\dontdisplaylinenum

\nemslokac 
toya\.m ca dadyāl laghupūrṇakambham
\dontdisplaylinenum

\nemslokad 
śīta\.m sugandha\.m parivārita\.m ca \veg\dontdisplaylinenum

\ujvers\nemsloka 
sa yāti loka\.m salileśvarasya
\dontdisplaylinenum

\nemslokab 
na tasya janmānit\textsubring{r}ṣābhibhūtaḥ \danda\dontdisplaylinenum

\nemslokac 
upānaha\.m yo dadati dvijāya
\dontdisplaylinenum

\nemslokad 
suśobhana\.m tailasudī surapita\.m ca \veg\dontdisplaylinenum

\ujvers\nemsloka 
te yānti lokam amarādhipasya
\dontdisplaylinenum

\nemslokab 
yamālaya\.m kaṣṭapathāna yānti \danda\dontdisplaylinenum

\nemslokac 
prakṣīṇapuṇyā punar atra loke
\dontdisplaylinenum

\nemslokad 
jāto bhaved divyakulopapannaḥ \veg\dontdisplaylinenum

\ujvers\nemsloka 
dhanaiḥ sam\textsubring{r}ddhodhopatitvatāś ca
\dontdisplaylinenum

\nemslokab 
rathāś ca nāgā prabhavanti tasya \danda\dontdisplaylinenum

\nemslokac 
vastrapradānena bhavanti devi
\dontdisplaylinenum

\nemslokad 
rūpottamasarvakalajñatā\.m ca \veg\dontdisplaylinenum

\ujvers\nemsloka 
sam\textsubring{r}ddhisaubhāgyaguṇānvitāś ca
\dontdisplaylinenum

\nemslokab 
svargacyutās te puruṣā bhavanti \danda\dontdisplaylinenum

\nemslokac 
vastrapradānābhiratasya pu\.msaḥ
\dontdisplaylinenum

\nemslokad 
anyat pravakṣyāmi tataḥ praśastām \veg\dontdisplaylinenum

\ujvers\nemsloka 
vastra\.m tu lokeṣv atipūjanīyam
\dontdisplaylinenum

\nemslokab 
vastra\.m narāṇā\.m tv atimānanīyam \danda\dontdisplaylinenum

\nemslokac 
vastra\.m tu bhūyo na ca mānalābhaḥ
\dontdisplaylinenum

\nemslokad 
parābhavaś cāti jugupsanaś ca \veg\dontdisplaylinenum

\ujvers\nemsloka 
tasmād dhi vastra\.m satata\.m pradeyam
\dontdisplaylinenum

\nemslokab 
yaśaḥ śriyaḥ svargasamāntalābham \danda\dontdisplaylinenum

\nemslokac 
yāvanti sūtrāṇi bhavanti vastre
\dontdisplaylinenum

\nemslokad 
tāvad yuga\.m gacchanti somalokam \veg\dontdisplaylinenum

\ujvers\nemsloka 
puṇyakṣayāj jāyati m\textsubring{r}tyuloke
\dontdisplaylinenum

\nemslokab 
vastraprabhūte dhanadhānyakīrṇo ? \danda\dontdisplaylinenum

\nemslokac 
surūpasaubhāgyayaśaśivanaś ca
\dontdisplaylinenum

\nemslokad 
vidyādharo lokaprabhutvatāś ca \veg\dontdisplaylinenum

\ujvers\nemsloka 
dvijebhyac chatra\.m suk\textsubring{r}ta\.m pradadyāt
\dontdisplaylinenum

\nemslokab 
varṣātapatra\.m d\textsubring{r}ḍhaśobhana\.m ca \danda\dontdisplaylinenum

\nemslokac 
aṅgāravarṣatraṣu khaḍgamādyam
\dontdisplaylinenum

\nemslokad 
asa\.mśaya\.m trāyati yāmyamārge \veg\dontdisplaylinenum

\ujvers\nemsloka 
svarga\.m ca yānti grahanāyakaś ca
\dontdisplaylinenum

\nemslokab 
sa varṣakoṭyāyutam antakāle \danda\dontdisplaylinenum

\nemslokac 
jāyanti te mānuṣamartyaloke
\dontdisplaylinenum

\nemslokad 
g\textsubring{r}hottame bhogapatir bhavanti \veg\dontdisplaylinenum

\ujvers\nemsloka 
k\textsubring{r}tvā maṭha\.m śobhanavipradātā
\dontdisplaylinenum

\nemslokab 
dravyeṇa śuddhena tu pūjayitvā \danda\dontdisplaylinenum

\nemslokac 
sa yāti devendrasada\.m yatheṣṭam
\dontdisplaylinenum

\nemslokad 
savarṣakoṭiśatadivyasa\.mkhyaiḥ \veg\dontdisplaylinenum

\ujvers\nemsloka 
tadantakāle yadi mānuṣatvam
\dontdisplaylinenum

\nemslokab 
jāyanti te saptamahīprabhoktā \danda\dontdisplaylinenum

\nemslokac 
sa saptarathyatrayasamprayuktā
\dontdisplaylinenum

\nemslokad 
balādhiko yajñasahasrakartā \veg\dontdisplaylinenum

\ujvers\nemsloka 
bhūmipradātā dvijahīnadīnam
\dontdisplaylinenum

\nemslokab 
sa\.mm\textsubring{r}ddhasasyo jalasa\.mnik\textsubring{r}ṣta \danda\dontdisplaylinenum

\nemslokac 
sa yāti lokam amarādhipasya !
\dontdisplaylinenum

\nemslokad 
vimānayānena manohareṇa \veg\dontdisplaylinenum

\ujvers\nemsloka 
manvantara\.m yāvad abhuktabhogān
\dontdisplaylinenum

\nemslokab 
tadantakāle cyutamartyaloke \danda\dontdisplaylinenum

\nemslokac 
sa javamukhaṇḍādhipatir bhavet
\dontdisplaylinenum

\nemslokad 
vīryānvito rājasahasranāthaḥ \veg\dontdisplaylinenum

\ujvers\nemsloka 
sa cailaghaṇṭā\.m kanakāgraś\textsubring{r}ṅgām
\dontdisplaylinenum

\nemslokab 
dogdhī\.m savatsā\.m payasā\.m dvijānām \danda\dontdisplaylinenum

\nemslokac 
dattvā dvijebhyaḥ samalaṅk\textsubring{r}tānām
\dontdisplaylinenum

\nemslokad 
prayānti loka\.m surabhīsutānām \veg\dontdisplaylinenum

\ujvers\nemsloka 
yāvanti romāṇi bhavanti gāvaḥ
\dontdisplaylinenum
            \var{\va yāvanti\lem  \Ed; prayānti \msCa}%

\nemslokab 
tāvad yugānām anubhūyabhogān \danda\dontdisplaylinenum

\nemslokac 
tasmāc cyutā martyamahībhujās te
\dontdisplaylinenum

\nemslokad 
sahasrarājānugato mahātmā \veg\dontdisplaylinenum

\ujvers\nemsloka 
suvarṇakā\.msyāyasaraupyadātā
\dontdisplaylinenum

\nemslokab 
tāmrapravālāmaṇimauktikādyān \danda\dontdisplaylinenum

\nemslokac 
dattvā dvijebhyo vasusādhyaloke
\dontdisplaylinenum

\nemslokad 
prāpnoti varṣa\.m daśapañcakoṭyo !  \veg\dontdisplaylinenum

\ujvers\nemsloka 
bhuktvā yatheṣṭa\.m kramadevalokān
\dontdisplaylinenum

\nemslokab 
cyuta\.m ca martye sa bhaven narendraḥ \danda\dontdisplaylinenum

\nemslokac 
sudurjayaḥ śakrasahasrajetā
\dontdisplaylinenum

\nemslokad 
sudīrgham āyuś ca parākramaś ca \veg\dontdisplaylinenum

\ujvers\nemsloka 
yat prekṣaṇa\.m darśayitu\.m pradātā
\dontdisplaylinenum

\nemslokab 
surūpasaubhāgya phala\.m labheta \danda\dontdisplaylinenum

\nemslokac 
t\textsubring{r}ṇāśanāmūlaphalāśanena
\dontdisplaylinenum

\nemslokad 
labheta rājyāni kaṇṭakāni \veg\dontdisplaylinenum

\ujvers\nemsloka 
labhetaparṇāśanasvargavāsam
\dontdisplaylinenum

\nemslokab 
payaḥ prayogena ca devaloke \danda\dontdisplaylinenum

\nemslokac 
śuśrūṣaṇo yo gurave ca nityam
\dontdisplaylinenum

\nemslokad 
vidyādharo jāyati martyaloke \veg\dontdisplaylinenum

\ujvers\nemsloka 
dadyād gavā\.m dhāsat\textsubring{r}ṇasya muṣṭiḥ
\dontdisplaylinenum

\nemslokab 
gavāḍhyatā\.m jāyati martyaloke \danda\dontdisplaylinenum

\nemslokac 
śrāddha\.m ca dattvā prayato dvijāya
\dontdisplaylinenum

\nemslokad 
sam\textsubring{r}ddhasantāna bhaved yugānte \veg\dontdisplaylinenum

\ujvers\nemsloka 
ahi\.msako jāyati dīrgham āyuḥ
\dontdisplaylinenum

\nemslokab 
kulottama\.m jāyati dīkṣitena \danda\dontdisplaylinenum

\nemslokac 
kālatraya\.m snānak\textsubring{r}tena rājya\.m
\dontdisplaylinenum

\nemslokad 
pītvā ca vāyus tridaśādhipatvam \veg\dontdisplaylinenum

\ujvers\nemsloka 
anaśnatāyāḥ phalam īśaloke
\dontdisplaylinenum

\nemslokab 
t\textsubring{r}ptir bhavet toyapradānaśīlaḥ \danda\dontdisplaylinenum

\nemslokac 
annapradātā puruṣaḥ sam\textsubring{r}ddhaḥ
\dontdisplaylinenum

\nemslokad 
sa sarvakāmā labhatīha loke \veg\dontdisplaylinenum

\ujvers\nemsloka 
śraddhāmatir yaḥ praviśed dhutāsana\.m !
\dontdisplaylinenum

\nemslokab 
sa yāti loka\.m prapitāmahasya \danda\dontdisplaylinenum

\nemslokac 
satya\.m vaded yo 'pi ca dharmaśīlo
\dontdisplaylinenum

\nemslokad 
modaty asau devi sahāpsarobhiḥ \veg\dontdisplaylinenum

\ujvers\nemsloka 
rasās tu ṣaḍyo parivarjayanti
\dontdisplaylinenum

\nemslokab 
atīva saubhāgya labheta sādhvī \danda\dontdisplaylinenum

\nemslokac 
dānena bhogān atulya\.m labheta
\dontdisplaylinenum

\nemslokad 
cirāyutā\.m yāti hi brahmacaryāt \veg\dontdisplaylinenum

\ujvers\nemsloka 
dhanāḍhyatā\.m yānti hi puṇyakarmān
\dontdisplaylinenum

\nemslokab 
maunena - ājñā labhate alaṅghyām \danda\dontdisplaylinenum

\nemslokac 
prāpnoti kāma\.m tapasaḥ sutapta\.m
\dontdisplaylinenum

\nemslokad 
kīrtir yaśaḥ svargam anantabhogam \veg\dontdisplaylinenum

\ujvers\nemsloka 
āyuḥ śriyārogyadhanaprabhutva\.m
\dontdisplaylinenum

\nemslokab 
jñānādilābha\.m tapasā labheta \danda\dontdisplaylinenum

\ujvers\nemsloka 
trailokyādhipatitvaśakram agamat k\textsubring{r}tvā tapo duṣkaram
\dontdisplaylinenum

\nemslokab 
yakṣeśo 'pi tapaḥ prabhāvaguruṇā guhyādhipatva\.m mahat \danda\dontdisplaylinenum

\nemslokac 
rakṣeśo 'pi bibhīṣaṇas tv amaratā\.m prāptas tapasyaiva tu
\dontdisplaylinenum

\nemslokad 
rudrārādhanatatparās tapaphalāt nandīgaṇatva\.m gataḥ \veg\dontdisplaylinenum

\ujvers\nemsloka 
jñāna\.m dvijān tapaso āha viṣṇuḥ
\dontdisplaylinenum

\nemslokab 
kṣatra\.m taporakṣaṇam āha sūrya \danda\dontdisplaylinenum

\nemslokac 
vaiśya\.m tapaś cāñjanam āha vāyuḥ
\dontdisplaylinenum

\nemslokad 
śūdra\.m hi śilpa\.m tapa āha indraḥ \veg\dontdisplaylinenum

\ujvers\nemsloka 
raṇotsaha\.m kṣatriyayajñam iṣṭa\.m
\dontdisplaylinenum

\nemslokab 
vaiśya\.m havir yajñam udāharanti \danda\dontdisplaylinenum

\nemslokac 
śūdrasya yajñaḥ paricaryam iṣṭa\.m
\dontdisplaylinenum

\nemslokad 
yajña\.m dvijānā\.m japamuktamokṣam \veg\dontdisplaylinenum

\vers

devy uvāca~{\dandab}\dontdisplaylinenum 

svamā\.msarudhira\.m dāna\.m dāna\.m putrakalatrayoḥ\thinspace{\danda} \dontdisplaylinenum

ki\.m praśasya\.m mahādeva tattva\.m vaktum ihārhasi \veg\dontdisplaylinenum

maheśvara uvāca~{\dandab}\dontdisplaylinenum 

svamā\.msarudhira\.m dāna\.m praśa\.msanti manīṣiṇaḥ\thinspace{\danda} \dontdisplaylinenum

śrūyatā\.m pūrvav\textsubring{r}ttāni sa\.mkṣipya kathayāmy aham \veg\dontdisplaylinenum

uśīnaras tu rājarṣiḥ kayo ?tārthe svakāntantu? \thinspace{\dandab} \dontdisplaylinenum

tyaktvā svargam anuprāptaḥ parārthe paratatparaḥ \veg\dontdisplaylinenum

putramā\.msa\.m svaya\.m chitvā agnidatta\.m purānaghe\thinspace{\dandab} \dontdisplaylinenum

tena dānaprabhāvena alarkas tridiva\.m gataḥ \veg\dontdisplaylinenum

\ujvers\nemsloka 
svadānadānena mudā sa putra
\dontdisplaylinenum

\nemslokab 
aputrabhūtasya ca putra jātaḥ \danda\dontdisplaylinenum

\nemslokac 
svarge svaya\.m cokvaya bhogalābha\.m
\dontdisplaylinenum

\nemslokad 
prāpto mahaddānay?la prabhāvāt \veg\dontdisplaylinenum

\vers

yādavaś cārjano devi dattvā khaṇḍavabhājanam \veg\dontdisplaylinenum

tapanasya prasādena saptadvīpeśvaro bhavet\thinspace{\dandab} \dontdisplaylinenum

hariṇā ca śiro bhitvā datta\.m me rudhira\.m purā \veg\dontdisplaylinenum

pratīcchita\.m kapālena brahmasambhavajena me\thinspace{\dandab} \dontdisplaylinenum

divyavarṣasahasrāṇi dhārā tasya na chidyate \veg\dontdisplaylinenum

parituṣṭo 'smi tenāha\.m karmaṇānena sundari\thinspace{\dandab} \dontdisplaylinenum

vara\.m datta\.m mayā devi purāṇapuruṣo 'vyayaḥ \veg\dontdisplaylinenum

akṣaya\.m valamūrja\.m ca ajarāmaram eva ca\thinspace{\dandab} \dontdisplaylinenum

mamādhika\.m bhaved viṣṇur māma yitvam vijeṣyasi \veg\dontdisplaylinenum

evamādīny anekāni mayoktāni janārdane\thinspace{\dandab} \dontdisplaylinenum

niṣkampa niścalamanaḥ sthāṇubhūta iva sthitaḥ \veg\dontdisplaylinenum

da?ciḥ svatanu\.m dattvā vibudhānā\.m varānane\thinspace{\dandab} \dontdisplaylinenum

bhuktvā lokān kramāt sarvān śivaloke pratiṣṭhitaḥ \veg\dontdisplaylinenum

jāmadagnir mahī\.m dattvā kāśyapāya mahātmane\thinspace{\dandab} \dontdisplaylinenum

ihaiva sa yāla\.m bhoktā devarājyam avāpsyati \veg\dontdisplaylinenum 

dattvā go sakala\.m devi vyāsasyāmitatejasaḥ\thinspace{\dandab} \dontdisplaylinenum

yudhiṣṭhira mahīyāsa dehas tridivadbhataḥ \veg\dontdisplaylinenum ?

satyanāmaḥ ? (bhīmaḥ?) svaka\.m bhartā dattvā nārādasatk\textsubring{r}tam\thinspace{\dandab} \dontdisplaylinenum

dānasyāsya prabhāvena akṣaya\.m tridivadbhataḥ ? \veg\dontdisplaylinenum

catuḥṣaṣṭhisahastāṇi gavā\.m dattvā dvijanmane\thinspace{\dandab} \dontdisplaylinenum

duryodhanamahīyā?o gataḥ svargam anantakam \veg\dontdisplaylinenum

vāsukis sarparājendro dattvā viprasusa\.msk\textsubring{r}tam\thinspace{\dandab} \dontdisplaylinenum

ratkāruś ca ? sābhānyā sarve nāgavimokṣitāḥ \veg\dontdisplaylinenum

gobhūmikanakādīnā\.m dāna\.m kanyasam ucyate\thinspace{\dandab} \dontdisplaylinenum

bh\textsubring{r}tyaputrakalatrāṇā\.m dāna\.m madhyamam ucyate \veg\dontdisplaylinenum

svadeha\.m pisitādīnā\.m dānam uttamam ucyate\thinspace{\dandab} \dontdisplaylinenum

etat sarva\.m yadā dāna\.m tad dānam uttamottamam \veg\dontdisplaylinenum
            \var{\vd °ottamam\lem  \msCapcorr; °otta \msCaacorr}%

jāvaj janmasahasrāṇi bhoktā bhavati kanyasaḥ\thinspace{\dandab} \dontdisplaylinenum

śatajanmasahasrāṇi bhoktā bhavati madhyamaḥ \veg\dontdisplaylinenum

uttamaḥ palabhoktā (phala?) vi ? janmakoṭiśatatrayam\thinspace{\dandab} \dontdisplaylinenum 

parārdhadvayajanmānā\.m bhoktā vai cottamottamaḥ \veg\dontdisplaylinenum

bhūtānām anukampayā yadi dhana\.m dātā sadānvarṣine\thinspace{\dandab} \dontdisplaylinenum

dīnānvak\textsubring{r}yaṇeṣv anāthamalineśvānādini?? ca \veg\dontdisplaylinenum

yady eva kurute sadārtiharaṇa\.m śraddhānvitau bhaktimān\thinspace{\dandab} \dontdisplaylinenum

tasyānantayāla\.m vadanti vibudhā\.ms sa yasya sandarśanāt \veg\dontdisplaylinenum

\vers


\jump
\begin{center}
\ketdanda iti v\textsubring{r}ṣasārasa\.mgrahe dānadharmaviśeṣa\.m nāma saptādaśamo 'dhyāyaḥ\ketdanda
\end{center}
\dontdisplaylinenum\vers 
\bekveg\szamveg\vfill\phpspagebreak\szam\bek\versno=0\fejno=18
\thispagestyle{empty}



\alfejezet{\textbf{18 pūrvakarmavipākaḥ}}\jump\jump
devy uvāca~{\dandab}\dontdisplaylinenum 

\nemsloka 
bhuktvā tu bhogān sucira\.m yatheṣṭa\.m
\dontdisplaylinenum

\nemslokab 
puṇyakṣayān martyam upāgatānām \danda\dontdisplaylinenum

\nemslokac 
cihnāni teṣā\.m kathayasva me 'dya
\dontdisplaylinenum

\nemslokad 
yathākrama\.m karmaphala\.m viśeṣāt \veg\dontdisplaylinenum

\vers

maheśvara uvāca~{\dandab}\dontdisplaylinenum 

\nemsloka 
sadānnadātā k\textsubring{r}paṇārtidīnā\.m
\dontdisplaylinenum

\nemslokab 
sa varṣakoṭyāyutam īśaloke \danda\dontdisplaylinenum
            \var{\vb °yutam īśaloke\lem  \msCapcorr;
                °yutam īnaśaloke \msCaacorr}%

\nemslokac 
bhuktvā ca bhogān samam apsarobhiḥ
\dontdisplaylinenum

\nemslokad 
prakṣīṇapuṇyaḥ punar eti martyam \veg\dontdisplaylinenum

\ujvers\nemsloka 
jāyanti divyeṣu kuleṣu pu\.msaḥ
\dontdisplaylinenum

\nemslokab 
sastrīsam\textsubring{r}ddhe bahubh\textsubring{r}tya \danda\dontdisplaylinenum

\nemslokac 
pūrṇe gaurava? śvarannādi dhanā
\dontdisplaylinenum

\nemslokad 
kuleṣu \textsubring{r}ṣo ?jjvalakāntisamāyuta\.m ca \veg\dontdisplaylinenum 

\ujvers\nemsloka 
vastra\.m susatk\textsubring{r}tya dvijasya dānāt
\dontdisplaylinenum

\nemslokab 
svargeṣu modanti sa varṣakoṭyaḥ \danda\dontdisplaylinenum

\nemslokac 
punaś ca te martyam upāgatāś ca
\dontdisplaylinenum

\nemslokad 
cihna?āha?krīyavam āpnuvanti \veg\dontdisplaylinenum

\ujvers\nemsloka 
kūpaprayāpuṣkaraṇī pradātā
\dontdisplaylinenum

\nemslokab 
sa lokam āpnoti jaleśvarasya \danda\dontdisplaylinenum

\nemslokac 
tatas sa tasmāc cyutim āpya lokā
\dontdisplaylinenum

\nemslokad 
akhīsut\textsubring{r}pteṣu kuleṣu jāyet \veg\dontdisplaylinenum

\ujvers\nemsloka 
rannipramāṇād api hemadānāt
\dontdisplaylinenum

\nemslokab 
surendraloka\.m samavāpnuvanti \danda\dontdisplaylinenum

\nemslokac 
tasmāc cyuto martyam upāgatāna\.m
\dontdisplaylinenum

\nemslokad 
cihn?? (saja?) dvi? nadhānyalakṣyāḥ \veg\dontdisplaylinenum

\ujvers\nemsloka 
adūṣya bhūmīvaravipradānāt
\dontdisplaylinenum

\nemslokab 
sa lokam āpnoti sureśvarasya \danda\dontdisplaylinenum

\nemslokac 
bhuktvā tu bhogān cyuta martyaloke
\dontdisplaylinenum

\nemslokad 
cihna\.m labhed vai viṣayādhipatvam \veg\dontdisplaylinenum

\ujvers\nemsloka 
dvijasya satk\textsubring{r}tya tilapradātā sa
\dontdisplaylinenum

\nemslokab 
lokam āpnoti ca keśavasya \danda\dontdisplaylinenum

\nemslokac 
bhraṣṭas tato martyam upāgatas tu
\dontdisplaylinenum

\nemslokad 
cihna\.m labhed akṣayam arthalābham \veg\dontdisplaylinenum

\ujvers\nemsloka 
gadā ? sva?ayā\.m vidhivad dvijānām
\dontdisplaylinenum

\nemslokab 
dattvā ca gokolam avāpnuvanti \danda\dontdisplaylinenum

\nemslokac 
kaplāvasāne samupetya martye
\dontdisplaylinenum

\nemslokad 
cihnaṅsavāḍhya\.m śatagoyuta\.m ca \veg\dontdisplaylinenum

\ujvers\nemsloka 
svarga\.m satānā\.m puruṣasya cihna\.m
\dontdisplaylinenum

\nemslokab 
vanāḍhyatā śrī mukhabhogalābham \danda\dontdisplaylinenum

\nemslokac 
āyuryaśorūpakalatraputram
\dontdisplaylinenum

\nemslokad 
samyaṅ vibhūti kulakīrtim artham \veg\dontdisplaylinenum

\ujvers\nemsloka 
dānā?(ṣṭa?)bhūñco?ttamakīrtanante
\dontdisplaylinenum

\nemslokab 
cihna\.m ca loka\.m ca samāsato me \danda\dontdisplaylinenum

\nemslokac 
ś\textsubring{r}ṇotu devī nirayāgatānā\.m
\dontdisplaylinenum

\nemslokad 
cihna\.m ca karma\.m ca vipākatā\.m ca \veg\dontdisplaylinenum

\ujvers\nemsloka 
hatvā ca vipra\.m manasā ca vācā
\dontdisplaylinenum

\nemslokab 
sa yāti pāra\.m nirayasya ghoram \danda\dontdisplaylinenum

\nemslokac 
aśītikalpa\.m niraye krameṇa
\dontdisplaylinenum

\nemslokad 
bhuktvā punas tirya śatāyutānām \veg\dontdisplaylinenum

\ujvers\nemsloka 
jayanti te mānuṣahīnavidyā
\dontdisplaylinenum

\nemslokab 
pratyantavāmāḥ kulavittahīnāḥ \danda\dontdisplaylinenum

\nemslokac 
nitya\.m ca tasyākṣayarogapīḍā
\dontdisplaylinenum

\nemslokad 
idan tu cihna\.m dvijajīvahartuḥ \veg\dontdisplaylinenum

\ujvers\nemsloka 
pītvā ca madya\.m dvijaḥ ? kāmato vā
\dontdisplaylinenum

\nemslokab 
āghrāti gadhva\.m svamanīṣikeṇa \danda\dontdisplaylinenum

\nemslokac 
sa yāti ghora\.m narakam asahya\.m
\dontdisplaylinenum

\nemslokad 
yāvac ca kalpa\.m daśa atra bhuktvā \veg\dontdisplaylinenum

\ujvers\nemsloka 
tīrya\.m ca sarvam anubhūya??
\dontdisplaylinenum

\nemslokab 
sva\.m sa kaṣṭakaṣṭena manuṣyajanvā \danda\dontdisplaylinenum

\nemslokac 
caṇḍālaśaunaśvayacanvam eti
\dontdisplaylinenum

\nemslokad 
śyāma\.m ca tāla bhavatīha cihnam \veg\dontdisplaylinenum

\ujvers\nemsloka 
nindanti ye vedasasnūya jihvā
\dontdisplaylinenum

\nemslokab 
yaḥ kūṭasākṣī sa ca khalv alā?au \danda\dontdisplaylinenum

\nemslokac 
suh\textsubring{r}dvadhām\textsubring{r}tyuśata\.m hi garbhe
\dontdisplaylinenum

\nemslokad 
garhāśanocchiṣṭabhujo bhavanti \veg\dontdisplaylinenum

\ujvers\nemsloka 
stainyas tu yaiḥ kurvati pāpasattvam
\dontdisplaylinenum

\nemslokab 
te pāpadoṣān naraka\.m vrajanti \danda\dontdisplaylinenum

\nemslokac 
manvantarādīny anubhūyaduḥkham
\dontdisplaylinenum

\nemslokad 
punaś ca tiryak śataśo 'nubhūyāt \veg\dontdisplaylinenum

\ujvers\nemsloka 
mānuṣyajanmeṣu ca duḥkhabhāgī
\dontdisplaylinenum

\nemslokab 
steneyamāyāti punaś ca mūḍhaḥ \danda\dontdisplaylinenum

\nemslokac 
suvarṇacaurakunakhatvacihnam
\dontdisplaylinenum

\nemslokad 
viśīrṇagātro rajatāpahārī \veg\dontdisplaylinenum

\ujvers\nemsloka 
tāmrāpahāri sphaṭitāgrapāṇīr
\dontdisplaylinenum

\nemslokab 
lohāpahārī bhujacchedacihna\.m \danda\dontdisplaylinenum

\nemslokac 
kā\.msāpahārī karabhagnacihnam
\dontdisplaylinenum

\nemslokad 
h\textsubring{r}tvā carīti trapusīsakānām \veg\dontdisplaylinenum

\ujvers\nemsloka 
nāsauṣṭhakarṇaśravaṇasya chedaḥ
\dontdisplaylinenum

\nemslokab 
cihna\.m n\textsubring{r}ṇā\.m vastrahara\.m kucelaḥ \danda\dontdisplaylinenum

\nemslokac 
dhānyāpahārī bhavaty eṅgahīnaḥ
\dontdisplaylinenum

\nemslokad 
dīpopahārī bhavaty andhacihnam \veg\dontdisplaylinenum

\ujvers\nemsloka 
nirvāpahā kāṇa bhaveta cihnam
\dontdisplaylinenum

\nemslokab 
yaḥ strī haret so 'pi jitaḥ striyā syāt \danda\dontdisplaylinenum

\nemslokac 
sasyāpahārī bhavatennahīnaḥ
\dontdisplaylinenum

\nemslokad 
h\textsubring{r}tvāyudhayantrahatatvacihna\.m \veg\dontdisplaylinenum

\ujvers\nemsloka 
annāpahārī paradattabhoktā
\dontdisplaylinenum

\nemslokab 
h\textsubring{r}tvā tu gāvaḥ sa bhavet daridraḥ \danda\dontdisplaylinenum

\nemslokac 
hariharettaddhariṇā dahanti
\dontdisplaylinenum

\nemslokad 
h\textsubring{r}tvā tu meṣān ajagardabhaś ca \veg\dontdisplaylinenum

\ujvers\nemsloka 
sa bhārabh\textsubring{r}jjīvam udāharanti
\dontdisplaylinenum

\nemslokab 
ratnāpahārī anapatyatā ca \danda\dontdisplaylinenum

\nemslokac 
chatrāpahārī apavitratā ca
\dontdisplaylinenum

\nemslokad 
h\textsubring{r}tvā ca bīja\.m sa bhaved abījaḥ \veg\dontdisplaylinenum

\ujvers\nemsloka 
godhūmaśāliyavamudgamāṣān
\dontdisplaylinenum

\nemslokab 
h\textsubring{r}tvā masūra\.m vilaya\.m vrajanti \danda\dontdisplaylinenum

\nemslokac 
kāmāturo mātaramāt\textsubring{r}putrī
\dontdisplaylinenum

\nemslokad 
māt\textsubring{r}śvasāṅ gacchati mātulānīm \veg\dontdisplaylinenum

\ujvers\nemsloka 
rājāṅganā\.m putrasutā\.m snuṣā\.m ca
\dontdisplaylinenum

\nemslokab 
pravrājinī\.m brāhmaṇīmantyajā\.m ca \danda\dontdisplaylinenum 

\nemslokac 
ajāśvameṣasurabhīsutāś ca
\dontdisplaylinenum

\nemslokad 
yat kāmayet teṣu vimūḍhacetaḥ \veg\dontdisplaylinenum

\ujvers\nemsloka 
sa yāti k\textsubring{r}cchra\.m naraka\.m sughora\.m
\dontdisplaylinenum

\nemslokab 
sa varṣakoṭīśataśo bhramitvā \danda\dontdisplaylinenum

\nemslokac 
tīryañ ca bhūyaḥ śataśovyatītya
\dontdisplaylinenum

\nemslokad 
kaṣṭena vai jāyati mānuṣatvam \veg\dontdisplaylinenum

\ujvers\nemsloka 
hīnāṅgatādīnaśarīratāś ca
\dontdisplaylinenum

\nemslokab 
yo māt\textsubring{r}gāmī sa bhaved aliṅgaḥ \danda\dontdisplaylinenum

\nemslokac 
māt\textsubring{r}svasātalpagavānaliṅgā
\dontdisplaylinenum

\nemslokad 
liṅge 'parodhaḥ sutaputrikāmaḥ \veg\dontdisplaylinenum

\ujvers\nemsloka 
snuṣā\.m ca yaḥ sevati raktamehī
\dontdisplaylinenum

\nemslokab 
dauḥ carmatāś ca dvijasundarīṣu \danda\dontdisplaylinenum

\nemslokac 
rājāṅganāyāsu ca liṅgacchedaḥ
\dontdisplaylinenum

\nemslokad 
pravrājinī kāmukamūtrak\textsubring{r}cchram \veg\dontdisplaylinenum

\ujvers\nemsloka 
savyādhiliṅga labhatentyajāsu
\dontdisplaylinenum

\nemslokab 
vilīnaliṅgaḥ paśuyonigāmī \danda\dontdisplaylinenum

\nemslokac 
jāyanti te mūṣikadhānyacaurī
\dontdisplaylinenum

\nemslokad 
kṣīra\.m hared vāyasatā\.m prayāti \veg\dontdisplaylinenum

\ujvers\nemsloka 
ha\.msāpahārī sa bhaven niha\.msaḥ
\dontdisplaylinenum

\nemslokab 
śvānatvam āyāti rasāpahārī \danda\dontdisplaylinenum

\nemslokac 
h\textsubring{r}tvā ca sūcīn tu bhavet sa da\.mśaḥ
\dontdisplaylinenum

\nemslokad 
h\textsubring{r}tvā tu sarpir v\textsubring{r}ṣatā\.m prayāti \veg\dontdisplaylinenum

\ujvers\nemsloka 
mā\.msa\.m tu h\textsubring{r}tvā sa bhaveta g\textsubring{r}dhraḥ
\dontdisplaylinenum

\nemslokab 
tailāpahārī khagatā\.m prayāti \danda\dontdisplaylinenum

\nemslokac 
guḍa\.m ca h\textsubring{r}tvā guḍikā bhavanti
\dontdisplaylinenum

\nemslokad 
śākāpahārī sa bhaven mayūram \veg\dontdisplaylinenum

\ujvers\nemsloka 
h\textsubring{r}tvā paśu\.m paṅgurajāyatehaḥ
\dontdisplaylinenum

\nemslokab 
citratvam āyāti suvastrahārī \danda\dontdisplaylinenum

\nemslokac 
h\textsubring{r}tvā dukūla\.m sa ca sārasattva\.m
\dontdisplaylinenum

\nemslokad 
kṣauma\.m ca h\textsubring{r}tvā sa ca durbalatvam \veg\dontdisplaylinenum

\ujvers\nemsloka 
ūrnāni vastrāṇy apah\textsubring{r}tya meṣaḥ
\dontdisplaylinenum

\nemslokab 
chuchundarī jāyati gandhahārī \danda\dontdisplaylinenum

\nemslokac 
brahmasvam alpam apah\textsubring{r}tya bhoktā
\dontdisplaylinenum

\nemslokad 
sa g\textsubring{r}dhra ucchiṣṭabhujo bhavanti \veg\dontdisplaylinenum

\ujvers\nemsloka 
pādena yaḥ sparśayate dvijāṅghri\.m
\dontdisplaylinenum

\nemslokab 
tacchītarakta\.m caraṇau bhaveta \danda\dontdisplaylinenum

\nemslokac 
pādena yaḥ sparśayate ca gāvaḥ
\dontdisplaylinenum

\nemslokad 
sa pādarogān vividhā\.ml labheta \veg\dontdisplaylinenum

\ujvers\nemsloka 
yo mātaraḥ tāḍayate pādena
\dontdisplaylinenum

\nemslokab 
pāde tadīye k\textsubring{r}mayaḥ patanti \danda\dontdisplaylinenum

\nemslokac 
pādāt p\textsubring{r}śed yaḥ pitara\.m durātmā
\dontdisplaylinenum

\nemslokad 
sūnonnapādaḥ sa bhavet paratra \veg\dontdisplaylinenum

\ujvers\nemsloka 
padāt p\textsubring{r}śet toyam anādareṇa
\dontdisplaylinenum

\nemslokab 
saślīpadīpādayuge bhaveta \danda\dontdisplaylinenum

\nemslokac 
pādena ya sparśayate hutāśa\.m
\dontdisplaylinenum

\nemslokad 
sa cāgnipādaḥ satata\.m bhaveta \veg\dontdisplaylinenum

\ujvers\nemsloka 
pādena yaś cāryam upasp\textsubring{r}śeta
\dontdisplaylinenum

\nemslokab 
sa pādaccheda\.m bahuśo labheta \danda\dontdisplaylinenum

\nemslokac 
granthāpahārī sa bhaveta mūkaḥ
\dontdisplaylinenum

\nemslokad 
durgandhavaktraḥ parichidravādī \veg\dontdisplaylinenum

\ujvers\nemsloka 
paiśunyavādī sa ca pūtināsām
\dontdisplaylinenum

\nemslokab 
anamravaktras tv an\textsubring{r}tāpavādī \danda\dontdisplaylinenum

\nemslokac 
pāruṣyavaktā mukhapākarāgī
\dontdisplaylinenum

\nemslokad 
asat pralāpī sa ca dantarogaḥ \veg\dontdisplaylinenum

\ujvers\nemsloka 
stīkṣṇapradāyī sa ca vakranāsa
\dontdisplaylinenum

\nemslokab 
sambhinnavaktā sa ca kaṇṭharogī \danda\dontdisplaylinenum

\nemslokac 
kruddhekṣaṇaḥ paśyati yas tu vipra\.m
\dontdisplaylinenum

\nemslokad 
tīvrākṣirogī sa tu jāyate hi \veg\dontdisplaylinenum

\ujvers\nemsloka 
pradveṣayālokayate 'tithīn ya
\dontdisplaylinenum

\nemslokab 
utpāditākṣis sa bhavet paratra \danda\dontdisplaylinenum

\nemslokac 
vairūpya cakṣus tv atisūkṣmacakṣuḥ
\dontdisplaylinenum

\nemslokad 
sa jāyate kekarapiṅgayakṣuḥ \veg\dontdisplaylinenum

\ujvers\nemsloka 
gartākṣikādīni vipāṇḍurāṇi
\dontdisplaylinenum

\nemslokab 
netrāmayāny eva ca pāpadoṣāt \danda\dontdisplaylinenum

\nemslokac 
ś\textsubring{r}ṇvanti ye pāpakathā\.m praśastā\.m
\dontdisplaylinenum

\nemslokad 
tā\.m karṇasarpiḥ paripīḍiyeta \veg\dontdisplaylinenum

\ujvers\nemsloka 
ś\textsubring{r}ṇvanti nindā\.m hariśarvayor yaḥ
\dontdisplaylinenum

\nemslokab 
sa karṇaśūlena tu jīvatī vā \danda\dontdisplaylinenum

\nemslokac 
mātāpit\textsubring{\=r}ṇā\.m ś\textsubring{r}ṇute 'pavādaḥ
\dontdisplaylinenum

\nemslokad 
sa karṇasāphena vināśam eti \veg\dontdisplaylinenum

\ujvers\nemsloka 
ś\textsubring{r}ṇoti nindā\.m guruviprajā yaḥ
\dontdisplaylinenum

\nemslokab 
sa karṇapūya\.m sravate saraktam \danda\dontdisplaylinenum

\nemslokac 
virūpyadāridhrakulādhameṣu
\dontdisplaylinenum

\nemslokad 
aniṣṭakarmabh\textsubring{r}tijīvanāś ca \veg\dontdisplaylinenum

\ujvers\nemsloka 
akīrtana\.m darśanavarjana\.m ca
\dontdisplaylinenum

\nemslokab 
śvāpākato śvādiṣu jāyate saḥ \danda\dontdisplaylinenum

\nemslokac 
etāni cihna\.m nirayāgatānā\.m
\dontdisplaylinenum

\nemslokad 
mānuṣyaloke kuk\textsubring{r}tasya d\textsubring{r}ṣṭam \veg\dontdisplaylinenum

\nemslokab 
samāsataḥ kīrtita eva devi \danda\dontdisplaylinenum

\nemslokad 
yathaiva muktis tv iha karmabhaṅgaḥ \veg\dontdisplaylinenum

\ujvers\nemsloka 
mātāpitroghato yāsutaduhit\textsubring{r}vahā bhrāt\textsubring{r}gambhīravegā
\dontdisplaylinenum

\nemslokab 
bhāryāvartā vivartā kuṭilagativadhur bāndhavormītaraṅgā \danda\dontdisplaylinenum

\nemslokac 
kāmakrodhobhakūlā karimakarajhaṣā grāhakāmā bhayante
\dontdisplaylinenum

\nemslokad 
m\textsubring{r}tyor ākhyārṇave 'smin na śaraṇavivaśākālad\textsubring{r}ṣṭo prayāti \veg\dontdisplaylinenum

\ujvers\nemsloka 
nitya\.m yena vinā na yāti divasa\.m pañcatvam āpadyate
\dontdisplaylinenum

\nemslokab 
tyaktvā deha vanāntareṣu viṣame śvānaśrigālākule \danda\dontdisplaylinenum

\nemslokac 
bandhuḥ sarvanivartate gatadayā dharmaika tatra sthitaḥ
\dontdisplaylinenum

\nemslokad 
tasmād dharmaparo na cānyaḥ suh\textsubring{r}daḥ sevet paratrārthinaḥ \veg\dontdisplaylinenum

\vers


\jump
\begin{center}
\ketdanda iti v\textsubring{r}ṣasārasa\.mgrahe pūrvakarmavipākacihnāṣṭādaśo 'dhyāyaḥ\ketdanda
\end{center}
\dontdisplaylinenum\vers 

\vers
\bekveg\szamveg\vfill\phpspagebreak\szam\bek\versno=0\fejno=19
\thispagestyle{empty}



\alfejezet{\textbf{19 dānayajñaviśeṣaḥ}}\jump\jump
vigatarāga uvāca~{\dandab}\dontdisplaylinenum 

kriyāsūkṣmo mahādharmaḥ karmaṇā kena prāpyate\thinspace{\danda} \dontdisplaylinenum

alpopāya\.m narārthāya p\textsubring{r}cchāmi kathayasva me \veg\dontdisplaylinenum

anarthayajña uvāca~{\dandab}\dontdisplaylinenum 

alpopāya\.m mahādharma\.m kathayāmi dvijottama\thinspace{\danda} \dontdisplaylinenum

sukhena labhate svarga\.m karmaṇā yena tac ch\textsubring{r}ṇu \veg\dontdisplaylinenum

lokāna\.m mātaro gāvo gobhiḥ sarva\.m jagad dh\textsubring{r}tam\thinspace{\dandab} \dontdisplaylinenum

gomayam am\textsubring{r}ta\.m sarva\.m jāta\.m sarvaśivecchayā \veg\dontdisplaylinenum

sarvadevamayī gāvaḥ sarvadevamayo dvijaḥ\thinspace{\dandab} \dontdisplaylinenum

sarvadevamayo bhūmiḥ sarvadevamayaḥ śivaḥ \veg\dontdisplaylinenum

tasmād gāvaḥ sadā sevyā dharmamokṣārthasiddhidā\thinspace{\dandab} \dontdisplaylinenum

paricaryā yathāśaktyā grāsavāsajalādibhiḥ \veg\dontdisplaylinenum

tāḍayen nātivegena vācayen m\textsubring{r}dunācaret\thinspace{\dandab} \dontdisplaylinenum

pālayan tarpanād yeṣu bhagnodvigneṣu yatnataḥ \veg\dontdisplaylinenum

vyādhivanaparikleśa oṣadhopakramaś caret\thinspace{\dandab} \dontdisplaylinenum

kaṇḍūyana\.m ca kartavya\.m yathāsaukhya\.m bhaved gavām \veg\dontdisplaylinenum

gavā\.m pradakṣiṇa\.m k\textsubring{r}tvā śraddhābhaktisamanvitaḥ\thinspace{\dandab} \dontdisplaylinenum

sāgarāntā mahī sarvā n pradakṣiṇīk\textsubring{r}tā bhavet \veg\dontdisplaylinenum

p\textsubring{r}ṣṭasa\.msparśanād yañ ca śraddhayā yadi mānavaḥ\thinspace{\dandab} \dontdisplaylinenum

ahorātrak\textsubring{r}ta\.m pāpa\.m naśyate nātrasa\.mśayaḥ \veg\dontdisplaylinenum

lāṅgūlenoddh\textsubring{r}ta\.m toya\.m mūrddhnā g\textsubring{r}hṇāti yo naraḥ\thinspace{\dandab} \dontdisplaylinenum

yāvaj jīva k\textsubring{r}ta\.m pāpa\.m naśyate nātra sa\.mśayaḥ \veg\dontdisplaylinenum

vidhivat snāpayed gā\.mś ca mantrayuktena vāriṇā\thinspace{\dandab} \dontdisplaylinenum

tenāmbhasā svaya\.m snātvā sarvapāpakṣayo bhavet \veg\dontdisplaylinenum

vyādhivighnam alakṣmītva\.m naśyate sadya eva ca\thinspace{\dandab} \dontdisplaylinenum

m\textsubring{r}tāpatyāś ca gāvāś ca snānam eva praśasyate \veg\dontdisplaylinenum

gavā\.m ś\textsubring{r}ṅgodaka\.m g\textsubring{r}hya mūrdhni yo dhārayen naraḥ\thinspace{\dandab} \dontdisplaylinenum

sa sarvatīrthasnānasya phala\.m prāpnoti mānavaḥ \veg\dontdisplaylinenum

grāsamuṣṭipradānena goṣu bhaktisamanvitaḥ\thinspace{\dandab} \dontdisplaylinenum

agnihotra\.m huta\.m tena sarvadevāḥ sutarpitāḥ \veg\dontdisplaylinenum

catvāraḥ stanadhārās tu yas tu mūrdhnā pratīcchati\thinspace{\dandab} \dontdisplaylinenum

sa catuḥsāgara\.m gatvā snānapuṇyaphala\.m labhet \veg\dontdisplaylinenum

gavārtha\.m yas tyajet prāṇān gograheṣu dvijottama\thinspace{\dandab} \dontdisplaylinenum

kalpakoṭiśata\.m divya\.m śivaloke mahīyate \veg\dontdisplaylinenum

cyutabhagnādisa\.mskāra\.m sarva\.m yaḥ kurute naraḥ\thinspace{\dandab} \dontdisplaylinenum

bhāryākoṭiśata\.m dāna\.m yat phala\.m parikīrtitam \veg\dontdisplaylinenum 

tatphala\.m labhate martyaḥ śivaloka\.m ca gacchati\thinspace{\dandab} \dontdisplaylinenum

śivalokaparibhraṣṭaḥ p\textsubring{r}thivyām ekarāḍ bhavet \veg\dontdisplaylinenum

samāsataḥ samākhyāta\.m yathātattva\.m dvijottama\thinspace{\dandab} \dontdisplaylinenum

na śakya\.m vistarād vaktiu\.m gomahātmyasamuttamam \veg\dontdisplaylinenum

vigatarāga uvāca~{\dandab}\dontdisplaylinenum 

devāḥ r aṣṭavidhāḥ proktāḥ tiryak pañcavidhaḥ sm\textsubring{r}taḥ\thinspace{\danda} \dontdisplaylinenum

mānuṣyam ekam evāhuś cāturvarṇyaḥ katha\.m bhavet \veg\dontdisplaylinenum

anarthayajña uvāca~{\dandab}\dontdisplaylinenum 

pūrvakalpas\textsubring{r}jaty eṣa viṣṇunā prabhaviṣṇunā\thinspace{\danda} \dontdisplaylinenum

eva\.m varṇā dvijaś cāsīt sarvakalpāgram agrataḥ \veg\dontdisplaylinenum

sarvavedavido viprāḥ sarvavedavidas tathā\thinspace{\dandab} \dontdisplaylinenum

tathā viprasahasrāṇā\.m yajñotsāhamano bhavet \veg\dontdisplaylinenum

v\textsubring{r}ddhaviprasahasrāṇā\.m matam āśritya brāhmaṇaiḥ\thinspace{\dandab} \dontdisplaylinenum

kartu\.m karma samārabdhakarmaś cāpi vibhajyate \veg\dontdisplaylinenum

\textsubring{r}tvajatve sthitāḥ kecit kecit sa\.mrakṣaṇe sthitāḥ\thinspace{\dandab} \dontdisplaylinenum

arthopārjanayuktān ye anye śilpe niyojitāḥ \veg\dontdisplaylinenum

eva\.m yajñavidhānena kartum arebhire purā\thinspace{\dandab} \dontdisplaylinenum

yathoddiṣṭena karmeṇa yajñotsāham avartata \veg\dontdisplaylinenum

āgatā \textsubring{r}ṣayaḥ sarve devatāḥ pitaras tathā\thinspace{\dandab} \dontdisplaylinenum

anyonyam abruvan tatra devarṣipit\textsubring{r}devatāḥ \veg\dontdisplaylinenum

yajñārtam as\textsubring{r}jad varṇa\.m vidhinā pātuhetavaḥ\thinspace{\dandab} \dontdisplaylinenum

evam eva pravartantu bhavatir dvijasattamāḥ \veg\dontdisplaylinenum

ijyādhyādhyayanasampannā brahmaṇā yatra kalpitāḥ\thinspace{\dandab} \dontdisplaylinenum

suviprā vipratā\.m yāntu ṣaḍkarmāniratāḥ sadā \veg\dontdisplaylinenum

rakṣaṇārta\.m tu ye viprāḥ kalpitāḥ śastrapāṇayaḥ\thinspace{\dandab} \dontdisplaylinenum

k\textsubring{r}tatrāṇāya viprāṇā\.m nitya\.m kṣātravratodbhavāḥ \veg\dontdisplaylinenum

arthopārjanam uddiśya kalpitā ye dvijātayaḥ\thinspace{\dandab} \dontdisplaylinenum

te tu vaiśyatvam āyāntu vārto āpaṇatodbhavāḥ \danda\dontdisplaylinenum

vadhabandhanakarmeṣu śilpasthānavadheṣu ca \veg\dontdisplaylinenum

kalpitā ye dvijātīnā\.m sarve śūdrā bhavantu te\thinspace{\dandab} \dontdisplaylinenum

prājāpatya\.m brāhmaṇānām ījyādhyayanatatparām \veg\dontdisplaylinenum

sthānam aindra\.m kṣatriyāṇā\.m prajāpālanatatparam\thinspace{\dandab} \dontdisplaylinenum

vaiśyānā\.m vāsavasthāna\.m vāṇijya\.m k\textsubring{r}ṣijīvinām \veg\dontdisplaylinenum

śūdrāṇā\.m marutaḥ sthāna\.m śuśrūṣāniratātmanām\thinspace{\dandab} \dontdisplaylinenum

maharṣipit\textsubring{r}devānā\.m matam ājñāya niścitaḥ \danda\dontdisplaylinenum

eṣa sa\.mkalpito brahmā padmayoniḥ pitāmahaḥ \veg\dontdisplaylinenum

sa\.mkalpaprabhavāḥ sarve devadānavamānavāḥ\thinspace{\dandab} \dontdisplaylinenum

paśupakṣim\textsubring{r}gāmukhyā yāvanti jagasambhavāḥ \veg\dontdisplaylinenum

bhūtasa\.mkalpakartā ya kalpam āsīd dvijottama\thinspace{\dandab} \dontdisplaylinenum

kīrtitāni samāsena kim anyac chrotum icchasi \veg\dontdisplaylinenum

vigatarāga uvāca~{\dandab}\dontdisplaylinenum 

ki\.m tapaḥ sarvavarṇānā\.m v\textsubring{r}ttir vāpi tapodhana\thinspace{\danda} \dontdisplaylinenum

yajñāś caiva p\textsubring{r}thaktvena śrotum icchāmi tattvataḥ \veg\dontdisplaylinenum

anarthayajña uvāca~{\dandab}\dontdisplaylinenum 

brāhmaṇasya tapo yajñāḥ - tapaḥ kṣātrasya rakṣaṇam\thinspace{\danda} \dontdisplaylinenum

vaiśyaś ca tapa vāṇijya tapaḥ śūdrasya sevanam \veg\dontdisplaylinenum

pratigraha dhano vipraḥ kṣatriyasya dhanur dhanam\thinspace{\dandab} \dontdisplaylinenum

k\textsubring{r}ṣir dhana\.m tathā vaiśyaḥ śūdraḥ śuśrūṣaṇa\.m dhanam \veg\dontdisplaylinenum

ārambhayajñaḥ kṣatrasya havir yajño viśas tathā\thinspace{\dandab} \dontdisplaylinenum
            \paral{\textit{\vab {\normalfont  \kb\ MBh 12224061ab and 12230012ab } }}

śūdraḥ paricaro yajño japayajño dvijātayaḥ \veg\dontdisplaylinenum

satya tīrtha dvijātīnā\.m raṇa tīrtha\.m tu kṣatriyāḥ\thinspace{\dandab} \dontdisplaylinenum

āryā tīrtha\.m tu vaiśānā\.m ! śūdratīrtha\.m tu vai dvijāḥ \veg\dontdisplaylinenum

nāsti vidyāsamo mitro nāsti dānasamaḥ sakhā\thinspace{\dandab} \dontdisplaylinenum

nāsti jñānasamo bandur nāsti yajño japaḥ samaḥ \veg\dontdisplaylinenum

dharmahīno m\textsubring{r}tas tulyo devatulyo jitendriyaḥ\thinspace{\dandab} \dontdisplaylinenum

yajñatulyo 'bhaya\.m dātā śivatulyao manonmanaḥ \veg\dontdisplaylinenum

vigatarāga uvāca~{\dandab}\dontdisplaylinenum 

dāna yajñas tapas tīrtha\.m sa\.mnyāsa\.m yoga eva ca\thinspace{\danda} \dontdisplaylinenum

eteṣu katamaḥ śreṣṭhaḥ śrotum icchāmi kīrtaya \veg\dontdisplaylinenum

anarthayajña uvāca~{\dandab}\dontdisplaylinenum 

dānadharmasahasrebhyaḥ yajñayājī viśiṣyate\thinspace{\danda} \dontdisplaylinenum

yajñayājīsahasrebhyas tīrthayātrī viśiṣyate \veg\dontdisplaylinenum

tīrthayātrisahasrebhyas tapaniṣṭo viśiṣyate\thinspace{\dandab} \dontdisplaylinenum

tapaniṣṭhasahasrebhyaḥ śreṣṭhaḥ sa\.mnyāsikaḥ sm\textsubring{r}taḥ \veg\dontdisplaylinenum

sa\.mnyāsīnā\.m sahasrebhyaḥ śreṣṭho yac ya jitendriyaḥ\thinspace{\dandab} \dontdisplaylinenum

jitendriyasahasrebhyaḥ yogayukto viśiṣyate \veg\dontdisplaylinenum

yogayuktasahasrebhyaḥ śreṣṭho līnamanaḥ sm\textsubring{r}taḥ\thinspace{\dandab} \dontdisplaylinenum

tasmāt sarvaprayatnena ādau mana viśodhayet \veg\dontdisplaylinenum

nig\textsubring{r}hītendriyagrāmaḥ svargamokṣau tu sādhanam\thinspace{\dandab} \dontdisplaylinenum

viśiṣṭhe tv indriyagrāme tiryannarakasādhanam \veg\dontdisplaylinenum

vigatarāga uvāca~{\dandab}\dontdisplaylinenum 

carācarāṇā\.m bhūtānā\.m katamaḥ śreṣṭha ucyate\thinspace{\danda} \dontdisplaylinenum

kathayasva mamādya tva\.m chettum arhasi sa\.mśayam \veg\dontdisplaylinenum

anarthayajña uvāca~{\dandab}\dontdisplaylinenum 

carācarāṇā\.m bhūtānā\.m tatra śreṣṭho - carāḥ sm\textsubring{r}tāḥ\thinspace{\danda} \dontdisplaylinenum

carāṇā\.m caiva sarveṣā\.m buddhimān śreṣṭha ucyate \veg\dontdisplaylinenum

buddhimānṣu ! ca sarveṣu tataḥ śreṣṭha narāḥ sm\textsubring{r}tāḥ\thinspace{\dandab} \dontdisplaylinenum

narāṇā\.m caiva sarveṣā\.m brāhmaṇaḥ śreṣṭha ucyate \veg\dontdisplaylinenum

vidvarsv api ca sarveṣu k\textsubring{r}tabuddhir viśiṣyate\thinspace{\dandab} \dontdisplaylinenum

k\textsubring{r}tabuddhiṣu sarveṣu śreṣṭhaḥ kartā sa ucyate \veg\dontdisplaylinenum

kart\textsubring{r}ṣv api ca sarveṣu brahmavedī viśiṣyate\thinspace{\dandab} \dontdisplaylinenum

brahmavedi para\.m ! vipraḥ nānya\.m vedmi para\.mtapaḥ \danda\dontdisplaylinenum

sa vipraḥ sa tapasvī ca sa yogī sa śivaḥ sm\textsubring{r}taḥ \veg\dontdisplaylinenum


\jump
\begin{center}
\ketdanda iti v\textsubring{r}ṣasārasa\.mgrahe dānayajñaviśeṣo nāma unavi\.mśatitamo 'dhyāyaḥ\ketdanda
\end{center}
\dontdisplaylinenum\vers 
\bekveg\szamveg\vfill\phpspagebreak\szam\bek\versno=0\fejno=20
\thispagestyle{empty}



\alfejezet{\textbf{vi\.mśatimo 'dhyāyaḥ}}\jump\jump
\vers

vigatarāga uvāca~{\dandab}\dontdisplaylinenum 

pañcavi\.mśati yat tattva\.m jñātum icchāmi tattvataḥ\thinspace{\danda} \dontdisplaylinenum

kathayasva mamādya tva\.m chidyate yena sa\.mśayaḥ \veg\dontdisplaylinenum


\alalfejezet{tattvanirṇayam}
anarthayajña uvāca~{\dandab}\dontdisplaylinenum 

sarva\.m pratyakṣadarśitva\.m katha\.m mā\.m praṣṭum arhasi\thinspace{\danda} \dontdisplaylinenum
            \var{\va sarva\.m\lem  \msNa\Ed; sarva° \msCa\msCb\oo
                 °darśitva\.m\lem  \msCa\msNa\Ed; °darśītva\.m \msCb}%
            \var{\vb mā\.m\lem  \msCa\Ed; ma\.m \msNa}%

p\textsubring{r}ṣṭena kathanīyo 'smi eṣa me k\textsubring{r}taniścayaḥ \danda\dontdisplaylinenum
            \var{\vc 'smi\lem  \msCa\msNa\Ed; smī \msCb}%

ś\textsubring{r}ṇu te sampravakṣyāmi tattvasadbhāvam uttamam \veg\dontdisplaylinenum


\alalfejezet{puruṣaḥ-śivaḥ-brahmā (25)}
nādimadhya\.m na cānta\.m ca yan na vedya\.m surair api\thinspace{\dandab} \dontdisplaylinenum
            \var{\va cānta\.m\lem  \msCa\msCb\msNa; cāntaś \Ed}%
            \var{\vb surair api\lem  \msCa\msCb\msNa; surer api \msCb}%

atisūkṣmo hy atisthūlo nirālambo nirañjanaḥ \veg\dontdisplaylinenum

acintyaś cāprameyaś ca akṣarākṣaravarjitaḥ\thinspace{\dandab} \dontdisplaylinenum

sarvaḥ sarvagato vyāpī sarvam āv\textsubring{r}tya tiṣṭhati \veg\dontdisplaylinenum
            \var{\vcd \om\ \Ed}%

sarvendriyaguṇābhāsaḥ sarvendriyavivarjitaḥ\thinspace{\dandab} \dontdisplaylinenum
            \var{\vab \om\ \Ed}%

ajarāmarajaḥ śāntaḥ paramātmā śivo 'vyayaḥ \veg\dontdisplaylinenum
            \var{\vc °jaḥ\lem  \msCa\msCb\msNa; yaḥ \Ed}%

alakṣyalakṣaṇaḥ svastho brahmā puruṣasa\.mjñitaḥ\thinspace{\dandab} \dontdisplaylinenum
            \var{\vb brahmā\lem  \msCa\msCb\Ed; brahma \msNa}%

pañcavi\.mśaḥ sa vijñeyo janmam\textsubring{r}tyuharaḥ prabhuḥ \veg\dontdisplaylinenum
            \var{\vc °vi\.mśaḥ\lem  \msCa\msCb\Ed;
                         °vi\.mśat \msNaacorr, °vi\.mśa \msNapcorr\oo 
                 sa vijñeyo\lem  \msCa\msCb\msNaacorr\Ed; sarvajñeyo \msNapcorr}%

kalākalaṅkanirmukto vyomapañcāśavarjitaḥ\thinspace{\dandab} \dontdisplaylinenum

jalapakṣī yathā toyair na lipyeta jale caran \danda\dontdisplaylinenum
            \var{\vcd toyair na\lem  \msCa\Ed; toyī na \msCbacorr, toyer na \msCbpcorr, toyai na \msNa}%
            \var{\vd lipyeta\lem  \msCa\msCb\msNa; lipyate \Ed\oo
                 jale\lem  \msCa\msNa\Ed; jalai \msCb}%

tadvad doṣair na lipyeta pāpakarmaśatair api \veg\dontdisplaylinenum


\alalfejezet{prak\textsubring{r}tiḥ (24)}
caturvi\.mśati yat tattva\.m prak\textsubring{r}tir vidhiniścayaḥ\thinspace{\dandab} \dontdisplaylinenum
            \var{\va yat\lem  \msCa\msCb\msNa; yan \Ed}%

vik\textsubring{r}tiś ca sa vijñeyas tattvataḥ sa manīṣibhiḥ \veg\dontdisplaylinenum

prak\textsubring{r}tiprabhavāḥ sarve buddhyaha\.mkāra-ādayaḥ\thinspace{\dandab} \dontdisplaylinenum
            \var{\vb buddhyaha\.mkāra-ādayaḥ\lem  \msCa\msCb\msNa;
                         buddhyāha\.mkārakādayaḥ \Ed}%

vik\textsubring{r}ti\.m pratilīyante bhūmyādi kramaśas tu vai \veg\dontdisplaylinenum
            \var{\vd kramaśas tu\lem  \msCa\msCb\Ed; yaḥ kramas tu \msNa}%


\alalfejezet{matiḥ-buddhiḥ (23)}
matitattva trayovi\.mśa dharmādiguṇasa\.myutaḥ\thinspace{\dandab} \dontdisplaylinenum
            \var{\vb °yutaḥ\lem  \msCa\msNa\Ed; °yutam \msCb}%

sattvādhikasamutpannaboddhāra\.m viddhi dehinaḥ \veg\dontdisplaylinenum
            \var{\vc °samutpanna\lem  \msCa\msNa\Ed; °samutpanno \msCb}%
            \var{\vd °boddhāra\.m\lem  \eme; °boddhā ta \msCa\msCb\Ed, °boddhā ta\.m \msNa\oo
                 viddhi\lem  \eme; vidhi \msCa\msCb\msNa\Ed}%


\alalfejezet{aha\.mkāraḥ (22)}
dvāvi\.mśati aha\.mkāras tattvam ukta\.m manīṣibhiḥ\thinspace{\dandab} \dontdisplaylinenum
            \var{\vb ukta\.m\lem  \msCa\msCb\msNa; ukta \Ed}%

bhūtādi mama pañcāha rajādhikasamudbhavam \veg\dontdisplaylinenum
            \var{\vc bhūtādi mama pañcāha\lem  \msCa\msCb\msNa; bhūtādir nāma pañcāha \Ed}%
            \var{\vd rajā°\lem  \msCa\msCb\msNa; rajo° \Ed\oo
                 °dbhavam\lem  \msCa\msNa\Ed; °dbhavaḥ \msCb}%


\alalfejezet{ākāśaḥ (suṣiratva\.m) śabdaś ca (21-20)}
ekavi\.mśati yat tattva\.m suṣira\.m viddhi bho dvija\thinspace{\dandab} \dontdisplaylinenum
            \var{\vb suṣira\.m viddhi\lem  \eme; suśira\.m viddhi \msCa\msCb\msNa\Ed, susira v\textsubring{r}ddhi \msCc}%

śabdātīta\.m suṣiratva\.m saśabdaguṇalakṣaṇam \veg\dontdisplaylinenum
            \var{\vc suṣiratva\.m\lem  \eme; suśiratva\.m \msCa\msCb\msNa\Ed}%
            \var{\vd °lakṣaṇam\lem  \msCb\msNa\Ed; °\uncl{la}{\lost}ṇam \msCa}%


\alalalfejezet{śabdaḥ}

saptasvarās trayo grāmā mūrchanās tv ekavi\.mśatiḥ\thinspace{\dandab} \dontdisplaylinenum
            \var{\va grāmā\lem  \msCa\msCb\msNa; grāmāḥ \Ed}%
            \var{\vb °vi\.mśatiḥ\lem  \Ed; °vi\.mśati \msCa\msCb\msNa}%

tānā-m-ekonapañcāśac chabdabhedas tadādayaḥ \veg\dontdisplaylinenum
            \var{\vc °kona°\lem  \msCa\Ed; °kūna° \msCb\msNa}%

evam ādīny anekāni svarabhedā dvijottama\thinspace{\dandab} \dontdisplaylinenum
            \var{\vb °bhedā\lem  \msCa\msNa; °bhedān \Ed}%

gāndharvasvaratattvajñair munibhiḥ samudāh\textsubring{r}tam \veg\dontdisplaylinenum
            \var{\vc gāndharvasvara°\lem  \msNa\Ed; gāndharvāsura \msCa}%

veṇumurajatantrīṇā\.m dundubhīnā\.m svanāni ca\thinspace{\dandab} \dontdisplaylinenum
            \var{\vb svanāni\lem  \msNa\Ed; stanāni \msCa}%

śaṅkhakāhalakā\.msyānā\.m śabdāni vividhāni ca \veg\dontdisplaylinenum
            \var{\vcd kā\.msyānā\.m śabdāni\lem  \msNa\Ed; kā\.msyā\uncl{nā\.m} {\lost}{\lost}ni \msCa}%


\alalalfejezet{ākāśaḥ}

ākāśadhātu\.m viprendra ś\textsubring{r}ṇu vakṣyāmi te daśa\thinspace{\dandab} \dontdisplaylinenum
            \var{\va °dhātu\.m\lem  \msCa; °dhātu \msNa\Ed}%

pāyūpasthodaraḥ kaṇṭhaśaṅkhalau mukhanāsikau \veg\dontdisplaylinenum
            \var{\vc °daraḥ\lem  \Ed; °dara° \msCa\msNa}%
            \var{\vd śaṅkhalau\lem  \msCa\msNa; śrotau ca \Ed}%

h\textsubring{r}di\.m ca daśama\.m jñeya\.m deha ākāśasambhavaḥ\thinspace{\dandab} \dontdisplaylinenum
            \var{\va h\textsubring{r}di\.m\lem  \msCa\msNa; h\textsubring{r}diś \Ed}%

punar anyat pravakṣyāmi tac ch\textsubring{r}ṇuṣva dvijottama \veg\dontdisplaylinenum
            \var{\vc anyat\lem  \msCa\Ed; anya\.m \msNa}%

daśa dhātuguṇā jñeyāḥ pañcabhūtaḥ p\textsubring{r}thak p\textsubring{r}thak\thinspace{\dandab} \dontdisplaylinenum

ākāśasya guṇāḥ śabdo vyāpitva\.m chidratāpi ca \veg\dontdisplaylinenum
            \var{\vc ākāśasya\lem  \msNa\Ed; ākāśa{\lost} \msCa}%
            \paral{\textit{\vcd {\normalfont  \kb\ MBh 12.247.7ab: }ākāśasya guṇaḥ śabdo vyāpitva\.m chidratāpi ca}}

anāśrayanirālambam avyaktam avikāritā\thinspace{\dandab} \dontdisplaylinenum

apratīghātitā caiva bhūtatva\.m prak\textsubring{r}tāni ca \veg\dontdisplaylinenum
            \paral{\textit{\vcd {\normalfont  \kb\ MBh 12.247.7cd--8ab: }
                         anāśrayam anālambam avyaktam avikāritā{\thinspace\ketdanda}
                         apratīghātatā caiva bhūtatva\.m vik\textsubring{r}tāni ca{\thinspace\danda}}}


\alalfejezet{vāyuḥ sparśaś ca (19-18)}
ākāśadhāto[r?] viprendra tato vāyusamudbhavaḥ\thinspace{\dandab} \dontdisplaylinenum

śabdapūrvaguṇa\.m g\textsubring{r}hya vāyoḥ sparśaguṇaḥ sm\textsubring{r}taḥ \veg\dontdisplaylinenum

śabda pūrva\.m mayākhyāta\.m ś\textsubring{r}ṇu sparśa dvijottama\thinspace{\dandab} \dontdisplaylinenum
            \var{\vb dvijottama\lem  \msNa\Ed; {\lost}jottama \msCa}%

kaṭhinaś cikkaṇaḥ ślakṣo m\textsubring{r}dusnigdhakharadravāḥ \veg\dontdisplaylinenum
            \var{\vc cikkaṇaḥ\lem  \corr; cikkanaḥ \msCa\msNa; cikkaraḥ \Ed}%
            \paral{\textit{\vcd {\normalfont  \kb MBh 12.177.34ab: } kaṭhinaś cikkaṇaḥ ślakṣṇaḥ picchalo m\textsubring{r}dudāruṇaḥ}}

karkaśaḥ paruṣas tīkṣṇaḥ śītoṣṇa daśa ca dvayam\thinspace{\dandab} \dontdisplaylinenum
            \var{\va paruṣa°\lem  \msCa\Ed; \om\ \msNaacorr, paruṣā° \msNapcorr\oo
                 tīkṣṇaḥ\lem  \msNa; trīkṣṇaś \msCa; tīkṣṇa \Ed}%
            \var{\vb dvayam\lem  \msNa; dvaya \msCa\Ed}%

iṣṭāniṣṭadvayasparśa vapuṣā parig\textsubring{r}hyate \veg\dontdisplaylinenum


\alalalfejezet{prāṇāḥ}

prāṇopānaḥ samānaś ca udāno vyāna eva ca\thinspace{\dandab} \dontdisplaylinenum

nāgakūrmo 'tha k\textsubring{r}karo devadatto dhana\.mjayaḥ \veg\dontdisplaylinenum
            \var{\vc nāga°\lem  \msNa\Ed; nāma° \msCa\oo
                 k\textsubring{r}karo\lem  \msCa\msNa; k\textsubring{r}kalo \Ed}%
            \paral{\textit{\vo {\normalfont  The next XX verses are parallel to a passage in the B\textsubring{r}hatkālottara (NGMPP Reel No.\ B 29/59 Manuscript No.\ pra - 89): }
        prāṇopānaḥ samānaś ca udāno vyāna eva ca{\thinspace\ketdanda}
        nāgaḥ kurmodhva k\textsubring{r}karo devadattadhana\.myayau{\thinspace\danda}
        prāṇas tu prathamo vāyur daśānām api sa prabhuḥ{\thinspace\ketdanda}
        prāṇaḥ prāṇamayaḥ prāṇa visargāpūraṇa\.m prati{\thinspace\danda}
        nityam āpūrayaty eṣa prāṇinām urasi sthitaḥ{\thinspace\ketdanda}
        niśvāsocchvāsakāmais tu prāṇo jīvasamāśritaḥ{\thinspace\danda}
        prayāṇa\.m kurute yasmāt tasmāt prāṇa prakīrtitaḥ{\thinspace\ketdanda}
        apānasahāpānas tu āhāra\.m ca n\textsubring{r}ṇām adhaḥ{\thinspace\danda}
        mūtraśukravahovāyur apānas tena kīrtitaḥ{\thinspace\ketdanda}
        pīta\.m bhakṣitam āghrāta\.m raktapitakaphānila\.m{\thinspace\danda}
        sama\.m nayati mātreṣu samāno nāma mārutaḥ{\thinspace\ketdanda}
        spada\.myabhyadhara\.m vaktra\.m netragātra prakopanaḥ{\thinspace\danda}
        udvejayati marmāṇi udāto nāma mārutaḥ{\thinspace\ketdanda}
        vyāno vināmayatya\.mga\.m vyāno vyādhiprakopakaḥ{\thinspace\danda}
        prītecināsī kathito vāddhikyāt vyāna ucyate{\thinspace\ketdanda};
        {\normalfont  cf.\ also Sārdhatriśatikālottara, Agnipurāṇa and Dīpikā by Aghoraśivācārya on the M\textsubring{r}gendra } }}

daśavāyupradhānaite kīrtitā dvijasattama\thinspace{\dandab} \dontdisplaylinenum
            \var{\vb kīrtitā\lem  \msNa;
                         \uncl{kīrtti}tā \msCa, kīrtitāḥ \Ed}%

dhana\.mjayo bhaved ghoṣo devadatto vij\textsubring{r}mbhakaḥ \veg\dontdisplaylinenum
            \var{\vc ghoṣo\lem  \msCa\Ed; yoṣo \msNa}%

k\textsubring{r}karaḥ kṣudhak\textsubring{r}n nitya\.m kūrmonmīlitalocanaḥ\thinspace{\dandab} \dontdisplaylinenum
            \var{\va k\textsubring{r}karaḥ\lem  \msCa\msNa; k\textsubring{r}kara \Ed}%

nāga udghāṭana\.m puṣya\.m karoti satata\.m dvija \veg\dontdisplaylinenum

prāṇaḥ svasati bhūtānā\.m niśvasanti ca nityaśaḥ\thinspace{\dandab} \dontdisplaylinenum
            \var{\va prāṇaḥ\lem  \msCa\msNa; prāṇāḥ \Ed}%
            \var{\vb nityaśaḥ\lem  \msCa\msNa; nitya yaḥ \Ed}%

prayāṇa\.m kurute yasmāt tasmāt prāṇa iti sm\textsubring{r}taḥ \veg\dontdisplaylinenum
            \var{\vc prayāṇa\.m\lem  \msCa\msNa; prayāṇā \Ed}%

apanayaty apānas tu āhāra\.m manujāmadhaḥ\thinspace{\dandab} \dontdisplaylinenum
            \var{\va apanaya°\lem  \msNa\Ed; a\uncl{pa} {\lost} ya° \msCa}%
            \var{\vb °madhaḥ\lem  \msCa; °dhamaḥ \msNa, °pavaḥ \Ed}%

śukramūtravaho vāyur apānas tena kīrtitaḥ \veg\dontdisplaylinenum

pīta\.m bhakṣitam āghrāta\.m raktapittakaphānilam\thinspace{\dandab} \dontdisplaylinenum
            \var{\va pīta\.m\lem  \corr; pīta° \msCa\msNa\Ed\oo
                 °ghrāta\.m\lem  \msCa\Ed; °ghrāti \msNa}%

sama\.m nayati gātreṣu samāno nāma mārutaḥ \veg\dontdisplaylinenum

spandayaty adhara\.m vaktra\.m netragātraprakopanam\thinspace{\dandab} \dontdisplaylinenum
            \var{\va °dhara\.m\lem  \msCa\msNa; °dhara° \Ed\ \unmetr}%

udvejayati marmāṇi udāno nāma mārutaḥ \veg\dontdisplaylinenum
            \var{\vc marmāṇi\lem  \msCa\msNa; karmāṇi \Ed}%
            \var{\vd udāno nāma\lem  \msNa\Ed; \uncl{u}{\lost}{\lost}{\lost}{\lost} \msCa}%

vyāno vināmayaty aṅga\.m vyaṅgo vyādhiprakopanaḥ\thinspace{\dandab} \dontdisplaylinenum

prītivināśakathita\.m vārdhikya\.m vyāna ucyate \veg\dontdisplaylinenum
            \var{\vc prīti°\lem  \msCa\msNa; prītir \Ed}%

daśavāyuvibhāge ca kīrtito me dvijottama\thinspace{\dandab} \dontdisplaylinenum
            \var{\vb me\lem  \msCa\msNa; ye \Ed}%

daśavāyuguṇā\.mś cānyā\.m ch\textsubring{r}nu kīrtayato mama \veg\dontdisplaylinenum
            \var{\vc °vāyuguṇā\.mś\lem  \msCa\msNa; °dhātuguṇāś \Ed\oo
                 cānyā\.m\lem  \msCa\msNa; cānyac \Ed}%
            \var{\vd kīrtayato mama\lem  \msCa\msNa; kīrtaya me dvija \Ed}%

vāyor aniyamasparśo vātasthāne svatantratā\thinspace{\dandab} \dontdisplaylinenum
            \var{\vb vātasthāne\lem  \msNa\Ed; vāta{\lost}ne \msCa}%

bala\.m śīghra\.m ca mokṣa\.m ca ceṣṭākarmātmanā bhavaḥ \veg\dontdisplaylinenum


\alalfejezet{tejo rūpaś ca (17-16)}
vāyunāpi s\textsubring{r}jas tejas tadrūpaguṇam ucyate\thinspace{\dandab} \dontdisplaylinenum

śabdasparśasamajyotis triguṇa\.m samudāh\textsubring{r}tam \veg\dontdisplaylinenum

śabdaḥ sparśaḥ purā proktaḥ ś\textsubring{r}ṇu rūpaguṇa\.m tataḥ\thinspace{\dandab} \dontdisplaylinenum
            \var{\va śabdaḥ\lem  \msNa\Ed; śabda° \msCa\oo
                 proktaḥ\lem  \msCa\msNa; proktāḥ \Ed}%
            \var{\vb rūpaguṇa\.m\lem  \msCa\Ed; rūpa\.m guṇa\.m \msNa}%

hrasva\.m dīrgham aṇu sthūla\.m v\textsubring{r}ttamaṇḍalam eva ca \veg\dontdisplaylinenum
            \var{\vc hrasva\.m\lem  \msCa\msNa; hrasva° \Ed\oo
                 dīrgham aṇu\lem  \msCa\msNa; °dīrghalaghu \Ed}%

caturasra\.m dvirasra\.m ca tryasra\.m caiva ṣaḍasrakam\thinspace{\dandab} \dontdisplaylinenum
            \var{\va caturasra\.m dvirasra\.m\lem  \msNa;
                                 caturaśran dvi{\lost}śra\.m \msCa,
                                 caturasradvirasraś \Ed}%
            \var{\vb tryasra\.m\lem  \msCa\msNa; tisraś \Ed}%

śuklaḥ k\textsubring{r}ṣṇas tathā rakto nīlaḥ pīto 'ruṇas tathā \veg\dontdisplaylinenum
            \var{\vc śuklaḥ\lem  \msCa\msNa; śukla\.m \Ed}%
            \var{\vd nīlaḥ\lem  \msCa\msNa; nīla° \Ed}%

śyāmaḥ piṅgala babhruś ca nava raṅgāḥ prakīrtitāḥ\thinspace{\dandab} \dontdisplaylinenum
            \var{\va śyāmaḥ piṅgala babhruś ca\lem  \Ed; 
                        śyāmaḥ piṅgalo babhruś ca \msCa,
                       śyāmaś ca piṅgalo babhruś ca \msNaacorr,
                       śyāma piṅgalo bhruś ca \msNapcorr}%
            \var{\vb raṅgāḥ\lem  \msCa\msNa; raṅgaḥ \Ed}%

navadhā navaraṅgānām ekāśīti guṇāḥ sm\textsubring{r}tāḥ \veg\dontdisplaylinenum
            \var{\vd sm\textsubring{r}tāḥ\lem  \msCa\msNa; sm\textsubring{r}ta\.m \Ed}%

tejodhātu daśa brūmaḥ ś\textsubring{r}ṇuṣvāvahito bhava\thinspace{\dandab} \dontdisplaylinenum
            \var{\va tejodhātu daśa\lem  \msCa\msNa; 
                                tejodhātur daśa\.m \Ed}%

kāmas tejo kṣaṇaḥ krodho jaṭharāgniś ca pañcamaḥ \veg\dontdisplaylinenum
            \var{\vc tejokṣaṇaḥ\lem  \msCa; tejakṣaṇaḥ \msNa, tejekṣaṇaḥ \Ed}%
            \var{\vd jaṭharāgniś\lem  \msNa\Ed; jaṭha{\lost}gniś \msCa}%

jñāna\.m yogas tapo dhyāna\.m viśvāgnir daśamaḥ sm\textsubring{r}taḥ\thinspace{\dandab} \dontdisplaylinenum
            \var{\vb viśvāgnir\lem  \msCa\Ed; viśvāgni \msNa}%

daśa tejoguṇā\.mś cānyān pravakṣyāmi dvijottama \veg\dontdisplaylinenum
            \var{\vc daśa tejoguṇā\.mś\lem  \msCa\msNa; da\.mśatejoguṇāś \Ed}%

agner durdharṣavāpnoti tāpapākaprakāśanaḥ\thinspace{\dandab} \dontdisplaylinenum
            \var{\va agner durdharṣavāpnoti\lem  \Ed; 
                                agner durddhaṣatāpnoti \msCa,
                                agne durddhaṣatāpnoti \msNa}%

śauca\.m rāgo laghus taikṣṇa\.m daśamaś cordhabhāṣitam \veg\dontdisplaylinenum
            \var{\vc rāgo\lem  \msCa\msNa; gaṅgā \Ed}%
            \var{\vd daśama\.m cordhabhāṣitam\lem  \msNa; 
                 daśapañcorddhabhāṣitam\lem  \msCa;
                                daśamaś cordhabhāṣitam \Ed}%
            \paral{\textit{\vcd {\normalfont  \kb\ MBh 12.247.5cd: } 
                śauca\.m rāgo laghus taikṣṇya\.m daśama\.m cordhvabhāgitā}}


\alalfejezet{āpo rasaś ca (15-14)}
jyotiḥ so 'pi s\textsubring{r}jaś cāpi saraso guṇasa\.myutaḥ\thinspace{\dandab} \dontdisplaylinenum

caturguṇāḥ sm\textsubring{r}tā āpaḥ vijñeyā ca manīṣibhiḥ \veg\dontdisplaylinenum
            \var{\vd \om\ \msCa\msNa}%

śabdaḥ sparśaś ca rūpa\.m ca rasaś ca sa caturguṇaḥ\thinspace{\dandab} \dontdisplaylinenum
            \var{\va rūpa\.m\lem  \msCa\msNa; rūpaś \Ed}%
            \paral{\textit{\vab \kb\ {\normalfont  MBh 12.299.11ab: } 
                śabdaḥ sparśaś ca rūpa\.m ca raso gandhaś ca pañcamaḥ}}

rūpādiguṇa pūrvokta adhunātha rasa\.m ś\textsubring{r}ṇu \veg\dontdisplaylinenum
            \var{\vc pūrvokta\lem  \msCa\msNa; pūrvokta\.m \Ed}%

kaṭutiktakaṣāyāś ca lavaṇāmlas tathaiva ca\thinspace{\dandab} \dontdisplaylinenum
            \var{\va lavaṇāmlas\lem  \msCa\msNa; lavaṇāntas \Ed}%

madhuraś ca rasān ṣaḍ vai pravadanti manīṣiṇaḥ \veg\dontdisplaylinenum
            \var{\vc rasān\lem  \corr; rasā\.m \msCa, rasā \msNa\Ed}%

ṣaḍrasāḥ ṣaḍvibhedena ṣaḍtri\.mśaguṇa ucyate\thinspace{\dandab} \dontdisplaylinenum
            \var{\va °rasāḥ\lem  \msCa\msNa; °rasā \Ed\oo
                 ṣaḍvi°\lem  \msCa\Ed; ṣaḍbhir \msNa}%

āpadhātu daśa tv anyān ś\textsubring{r}ṇu kīrtayato mama \veg\dontdisplaylinenum
            \var{\va āpa°\lem  \msNa\Ed; \uncl{ā}pa° \msCa\oo
                 °nyān\lem  \corr; °nyā\.m \msCa\msNa, °nyā \Ed}%

lālā siṅghāṇikā śleṣmā raktaḥ pittaḥ kaphas tathā\thinspace{\dandab} \dontdisplaylinenum
            \var{\va siṅghāṇikā\lem  \corr;
                        sighānikā \msCa, si\.mghānikā \msNa\Ed\oo 
                 śleṣmā\lem  \msCa\msNa; śoṣmā \Ed}%
            \var{\vb raktaḥ\lem  \msCa\msNa; rakta° \Ed}%

svedam aśru rasaś caiva medaś ca daśamaḥ sm\textsubring{r}taḥ \veg\dontdisplaylinenum
            \var{\vc rasaś\lem  \msCa\Ed; rasa\.mś \msNa}%
            \var{\vd medaś\lem  \msCa\msNa; meda\.m \Ed}%

daśa āpaguṇāś cānyā kīrtayiṣyāmi tān ś\textsubring{r}ṇu\thinspace{\dandab} \dontdisplaylinenum
            \var{\va daśa āpa°\lem  \msCa\msNa; daśaś cāpa° \Ed\oo
                 cānyā\lem  \msNa\Ed; cānye \msCa}%
            \var{\vb tān\lem  \Ed; tā\.m \msCa\msNa}%

aṅgyaśaitya\.m rasakledo dravatva\.m snehasaumyatā \veg\dontdisplaylinenum
            \var{\vc aṅgyaśaitya\.m\lem  \msCa\msNa; agnyaśaitya° \Ed}%

jihvāviṣpandinī caiva bhaumān daśaguṇāñ ś\textsubring{r}ṇu\thinspace{\dandab} \dontdisplaylinenum
            \var{\va °viṣpandinī\lem  \msNa; °vi\uncl{ṣ}{\lost}{\lost}nī \msCa;
                         °niṣpandinī \Ed}%
            \var{\vb bhaumān daśaguṇāñ ś\textsubring{r}ṇu\lem  \Ed;
                         bhaumānyaśravaṇādhamaḥ \msCa\msNa}%


\alalfejezet{bhūmir gandhaś ca (13-12)}
āpaś ca bījyajā bhūmis tasyā gandhaguṇaḥ sm\textsubring{r}taḥ \veg\dontdisplaylinenum
            \var{\vc āpaś ca bījya°\lem  \Ed; āpaś cāpījya° \msCa\msNa}% 

caturāpaguṇān g\textsubring{r}hya bhūmer gandhaguṇaḥ sm\textsubring{r}taḥ\thinspace{\dandab} \dontdisplaylinenum
            \var{\va °guṇān\lem  \msCa\Ed; °guṇā \msNa}%

śabdaḥ sparśaś ca rūpa\.m ca raso gandhaś ca pañcamaḥ \veg\dontdisplaylinenum
            \var{\vc rūpa\.m\lem  \msCa\msNa; rūpaś \Ed}%
            \var{\vd pañcamaḥ\lem  \msCa\msNa; pañcama \Ed}%

āpaḥpūrvaguṇāḥ proktā bhūmer gandhaguṇa\.m ś\textsubring{r}ṇu\thinspace{\dandab} \dontdisplaylinenum
            \var{\va āpaḥ°\lem  \msCa; āpa° \Ed\oo
                 proktā\lem  \msCa; prokto \Ed}%
            \var{\vb bhūme°\lem  \msCa; bhūmi° \Ed\oo
                 ś\textsubring{r}ṇu\lem  \msCa; sm\textsubring{r}ta \Ed}%

iṣṭāniṣtadvayor gandhaḥ surabhir durabhis tathā \veg\dontdisplaylinenum
            \var{\vc dvayor gandhaḥ\lem  \Ed; dvayo{\lost}{\lost} \msCa}%

karpūraḥ kasturīka\.m ca candanāgarum eva ca\thinspace{\dandab} \dontdisplaylinenum
            \var{\va kasturīka\.m\lem  \msCa; kastūrīkaś \Ed\ \unmetr}%
            \var{\vb °garu°\lem  \msCa; °guru° \Ed}%

kuṅkumādisugandhāni ghrāṇam iṣṭa\.m prakīrtitam \veg\dontdisplaylinenum
            \var{\vd °kīrtitam\lem  \msCa; °kīrtitaḥ \Ed}%

viṅmūtrasvedagandhāni vaktragandha\.m ca duḥsaham\thinspace{\dandab} \dontdisplaylinenum
            \var{\vb °gandha\.m\lem  \msCa; °gandhaś \Ed}%

jīrṇasphoṭitagandhāni aniṣṭānīti kīrtitam \veg\dontdisplaylinenum
            \var{\vc °sphoṭita°\lem  \msCa; °sphoṭaka° \Ed}%

bhūmer dhātu daśa tv anyān kathayiṣyāmi tac ch\textsubring{r}ṇu\thinspace{\dandab} \dontdisplaylinenum
            \var{\va bhūmer\lem  \msCa; bhūme \Ed\oo
                 tv anyān\lem  \msCa; tv anyā \Ed}%
            \var{\vb tac ch\textsubring{r}ṇu\lem  \Ed; \uncl{ta}{\lost}ṇu \msCa}%

tvaca\.m mā\.msa\.m ca meda\.m ca snāyu majjā sirā tathā \veg\dontdisplaylinenum
            \var{\vc tvaca\.m mā\.msa\.m ca meda\.m ca\lem  \msCa; tvacā mā\.msaś ca medaś ca \Ed}%
            \var{\vd snāyu\lem  \Ed; śnāyu\.m \msCa\oo
                 sirā\lem  \eme;  śirās \msCa; śiras \Ed}%

nakhadantaruhāś caiva keśaś ca daśamas tathā\thinspace{\dandab} \dontdisplaylinenum
            \var{\vb keśa°\lem  \msCa; keśā° \Ed}%

daśa tv anyān pravakṣyāmi ś\textsubring{r}ṇu bhūmiguṇān dvija \veg\dontdisplaylinenum
            \var{\vc tv anyān\lem  \Ed; tv anyām \msCa}%

bhūmeḥ sthairya\.m rajastvañ ca kāṭhinya\.m prasavātmakam\thinspace{\dandab} \dontdisplaylinenum
            \var{\va rajastvañ ca\lem  \msCa; rajatvaś ca \Ed}%
            \var{\vb kāṭhinya\.m\lem  \msCa; kaṭhinya\.m \Ed}%

gandho guruś ca śaktiś ca nīhārasthāpanāk\textsubring{r}tiḥ \veg\dontdisplaylinenum

guṇa\.m dhātuviśeṣaś ca utpattiś ca dvijottama\thinspace{\dandab} \dontdisplaylinenum
            \var{\va guṇa\.m dhātu°\lem  \Ed; \uncl{guṇandhā}tu° \msCa}%

yathā śruta\.m mayā pūrva\.m kīrtita\.m nikhilena tu \veg\dontdisplaylinenum


\alalfejezet{buddhīndriyāṇi karmendriyāṇi ca (11-2)}
vaikārikam aha\.mkāra\.m sattvodriktāt tu sāttvikaḥ\thinspace{\dandab} \dontdisplaylinenum
            \var{\vab sattvodriktāt tu\lem  \corr;
                                 sattvod\textsubring{r}ktāt tu \msCa, sattvonuktānu \Ed}%
            \paral{\textit{\vab {\normalfont \kb\ Liṅgapurāṇa 1.70.38cd (= Sivapurāṇa 7.1.10.14cd): }
                vaikārikād aha\.mkārāt sattvodriktāt tu sāttvikāt}}

śrotra\.m tvak cakṣuṣī jihvā nāsikā caiva pañcamī \veg\dontdisplaylinenum

buddhīndriyāṇi pañcaiva kīrtitāni dvijottama\thinspace{\dandab} \dontdisplaylinenum

hastapādas tathā pāyur upastho vāk ca pañcamaḥ \veg\dontdisplaylinenum
            \var{\vc pāyu°\lem  \msCa; snāyu° \Ed}%
            \var{\vd °pastho vāk ca\lem  \Ed; °pa\uncl{stho vā}{\lost} \msCa\oo
                 pañcamaḥ\lem  \msCa; pañcamam \Ed}%


\alalalfejezet{śrotram (11)}

śrotreṇa g\textsubring{r}hyate śabdo vividhas tu dvijottama\thinspace{\dandab} \dontdisplaylinenum

veṇuvīṇāsvanānā\.m ca tantrīśabdam anekadhā \veg\dontdisplaylinenum

muraja saunda paṇavabherīpaṭahanisvanam\thinspace{\dandab} \dontdisplaylinenum
            \var{\va muraja\lem  \Ed; murava \msCa\oo
                 saunda\lem  \Ed; maunda \msCa}%

śaṅkhakāhalaśabda\.m ca śabda\.m ḍiṇḍimagomukham \veg\dontdisplaylinenum

kā\.msikāhalamiśra\.m ca gītāni vividhāni ca\thinspace{\dandab} \dontdisplaylinenum
            \var{\va °kāhala°\lem  \Ed; °kātāla° \msCa}%


\alalalfejezet{tvak (10)}

tvacayā g\textsubring{r}hyate sparśaḥ sukhaduḥkhasamanvitaḥ \veg\dontdisplaylinenum
            \var{\vc g\textsubring{r}hyate\lem  \Ed; g\textsubring{r}hya{\lost} \msCa}%

m\textsubring{r}dusūkṣmasukha\.m sparśaḥ vastraśayyāsanādayaḥ\thinspace{\dandab} \dontdisplaylinenum
            \var{\va °sukha\.m\lem  \Ed; °sukha° \msCa\ \unmetr}%

tīkṣṇaśastrajala\.m śaitya uṣṇataptakṣatekṣaraḥ \veg\dontdisplaylinenum
            \var{\vc śaitya\lem  \Ed; śaitye \msCa}%

evamādīny anekāni jñeyānīṣṭa\.m dvijottama\thinspace{\dandab} \dontdisplaylinenum


\alalalfejezet{cakṣuḥ (9)}

cakṣuṣā g\textsubring{r}hyate rūpa\.m sahasrāṇi śatāni ca \veg\dontdisplaylinenum

devarūpavikārāṇi nakṣatragrahatārakāḥ\thinspace{\dandab} \dontdisplaylinenum

mānuṣānā\.m vikārāṇi grāma\.m nagarapattanam \veg\dontdisplaylinenum

v\textsubring{r}kṣagulmalatānā\.m ca paśupakṣiśarīs\textsubring{r}pā\.m\thinspace{\dandab} \dontdisplaylinenum

k\textsubring{r}mikīṭapataṅgānā\.m jalajānām anekadhā \veg\dontdisplaylinenum

śailadāravaromāṇi rūpāṇi vividhāni ca\thinspace{\dandab} \dontdisplaylinenum
            \var{\va °romāṇi\lem  \Ed; °homāni \msCa}%

dhātudravyavikārāṇi rūpāṇi dvijasattama \veg\dontdisplaylinenum
            \var{\vd dvijasattama\lem  \Ed; dvija\uncl{sa}{\lost}{\lost} \msCa}%


\alalalfejezet{jihvā (8)}

jihvayā g\textsubring{r}hyate svādo h\textsubring{r}dyāh\textsubring{r}dyo dvijottama\thinspace{\dandab} \dontdisplaylinenum
            \var{\va jihvayā\lem  \Ed; {\lost}{\lost}yā \msCa}%

phalamūlāni śākāni kandāni piśitāni ca \veg\dontdisplaylinenum

pakvāpakva viśeṣāṇi dadhikṣīragh\textsubring{r}tāni ca\thinspace{\dandab} \dontdisplaylinenum

vrīhyoṣadhirasānā\.m ca miśrāmiśram anekadhā \veg\dontdisplaylinenum
            \var{\vc °ṣadhi°\lem  \Ed; °ṣadha° \msCa}%

ṣaṭkarmapratibhedena rasabhedaśata sm\textsubring{r}tam\thinspace{\dandab} \dontdisplaylinenum
            \var{\vb °śata\.m\lem  \msCa; °śata \Ed}%


\alalalfejezet{ghrāṇam (7)}

ghrāṇena g\textsubring{r}hyate gandha iṣṭāniṣṭo dvijarṣabhaḥ \veg\dontdisplaylinenum
            \var{\vcd g\textsubring{r}hyate gandha iṣṭā°\lem  \Ed; g\textsubring{r}\uncl{hyate ga}{\lost}{\lost}ṣṭā° \msCa}%
            \var{\vd °niṣṭo\lem  \msCa; °niṣṭā \Ed}%

guḍājya\.m guggulur bhaṣmacandanāgaruka\.m tathā\thinspace{\dandab} \dontdisplaylinenum
            \var{\va guḍājya\.m guggulur\lem  \msCa; guḍājyaguggulu \Ed}%
            \var{\vb °garuka\.m\lem  \msCa; °gurukas \Ed}%

kastūrikuṅkumādīnām iṣṭo gandho manoharaḥ \veg\dontdisplaylinenum
            \var{\vd gandho\lem  \msCa; gandha \Ed}%

vraṇamūtrapurīṣāṇā\.m mā\.msaparyuṣitāni ca\thinspace{\dandab} \dontdisplaylinenum
            \var{\vb mā\.msa°\lem  \Ed; māsa° \msCa}%

vātakarmādidurgandha aniṣṭaḥ samudāh\textsubring{r}taḥ \veg\dontdisplaylinenum


\alalalfejezet{hastakarma (6)}

hastena kurute karma vividhāni dvijottama\thinspace{\dandab} \dontdisplaylinenum
            \var{\va hastena\lem  \msCa; hastābhyā\.m \Ed}%

māhendra\.m vāruṇa\.m caiva vāyavyāgneyam eva ca \veg\dontdisplaylinenum
            \var{\vc māhendra\.m vāruṇa\.m\lem  \msCb; {\lost}{\lost}ndram vāruṇañ \msCa, mohendravāruṇa\.m \Ed}%

āgneyapavanādīni kā\.msyo lohas trapus tathā\thinspace{\dandab} \dontdisplaylinenum
            \var{\va °pavanā°\lem  \Ed; °pacanā° \msCa}%

agnikarmāṇy anekāni yajñahomakriyās tathā \veg\dontdisplaylinenum

sūryavyajanavātena mukhavātena vai tathā\thinspace{\dandab} \dontdisplaylinenum

camaracarmavātena vātayantra\.m ca vāyavam \veg\dontdisplaylinenum

vāruṇa\.m toyakarmāṇi kurute vividhāni ca\thinspace{\dandab} \dontdisplaylinenum
            \var{\vb kurute\lem  \Ed; kuru{\lost} \msCa}%

rasoparasakarmāṇi tasya poṣaṇakarma ca \veg\dontdisplaylinenum

snānācamanakarmāṇi vastraśaucādayas tathā\thinspace{\dandab} \dontdisplaylinenum

kāyaśauca\.m ca kurute t\textsubring{r}ṣānāśanam eva ca \veg\dontdisplaylinenum
            \var{\vd t\textsubring{r}ṣā°\lem  \Ed; t\textsubring{r}ṣa° \msCa}%

navamāni hy anekāni vāruṇa\.m karma ucyate\thinspace{\dandab} \dontdisplaylinenum

māhendra\.m pārthiva\.m karma anekāni dvijottama \veg\dontdisplaylinenum

kulālakarmabhūkarma karma pāṣāṇam eva ca\thinspace{\dandab} \dontdisplaylinenum
            \var{\va kulālakarma°\lem  \Ed; ku\uncl{la}{\lost}{\lost}rmma° \msCa}%
            \var{\vb karma\lem  \msCa; karma\.m \Ed}%

dārudantimaś\textsubring{r}ṅgādi karma pārthivam ucyate \veg\dontdisplaylinenum

catuṣkarma samāsena hastataḥ parikīrtitam\thinspace{\dandab} \dontdisplaylinenum


\alalalfejezet{pādakarma (5)}

pādābhyā\.m gamana\.m karma diśaś ca vidiśas tathā \veg\dontdisplaylinenum
            \var{\vd diśaś ca vidiśas\lem  \msCa; diśañ ca vidiśan \Ed}%

nimnonnatasame deśe śilāsa\.mkaṭakoṭare\thinspace{\dandab} \dontdisplaylinenum

toyakardamasa\.mghāte bahukaṇṭakasa\.mkule \veg\dontdisplaylinenum
            \var{\vd bahukaṇṭaka°\lem  \Ed; \uncl{bahu}{\lost}{\lost}ka° \msCa\oo
                 °kule\lem  \msCa; °yute \Ed}%


\alalalfejezet{pāyukarma (4)}

pāyukarma visarga\.m tu kaṭhinadravapicchilam\thinspace{\dandab} \dontdisplaylinenum
            \var{\va pāyu°\lem  \msCa; pāpa° \Ed}%

saraktaphenilādīni pāyuśakti pramuñcate \veg\dontdisplaylinenum
            \var{\vd pāyuśakti\lem  \Ed;  pāyucchakti \msCa\oo
                 °muñcati\lem  \msCa; °muñcate \Ed}%


\alalalfejezet{upasthakarma (3)}

upasthakarma ānanda\.m karoti janana\.m prajā\thinspace{\dandab} \dontdisplaylinenum
            \var{\va ānanda\.m\lem  \msCa; ānanda \Ed}%

strīpu\.mnapu\.msaka\.m caiva upastha\.m kurute dvija \veg\dontdisplaylinenum


\alalalfejezet{vākkarma (2)}

vācā tu kurute karma navadhā dvijapuṅgava\thinspace{\dandab} \dontdisplaylinenum

stutinindā praśa\.msā ca ākrośaḥ priya eva saḥ \veg\dontdisplaylinenum
            \var{\vd ākrośaḥ\lem  \Ed; {\lost}krośaḥ \msCa}%

praśno 'nujñā tathākhyānam āśīś ca vidhayo nava\thinspace{\dandab} \dontdisplaylinenum
            \var{\vb cā vidhayo nava\lem  \eme; ca vidhayo naya \msCa, cāvidhiyo nayaḥ \Ed}%

etā navavidhā vāṇī kīrtito me dvijottama \veg\dontdisplaylinenum


\alalfejezet{manaś conmanaś ca (1)}
adhunā kathayiṣyāmi manaso nava vai guṇān\thinspace{\dandab} \dontdisplaylinenum

calopapattiḥ sthaira\.m ca visargakalpanākṣamā \veg\dontdisplaylinenum
            \var{\vd visarga°\lem  \Ed; visarge \msCa\oo
                 °kṣamā\lem  \msCa; °samā \Ed}%

sad asac cāśutā caiva manaso nava vai guṇāḥ\thinspace{\dandab} \dontdisplaylinenum
            \var{\va cāśutā\lem  \Ed; cāśutāñ \msCa}%
            \paral{\textit{\vo {\normalfont  = MBh 12.247.9 } }} 

iṣṭāniṣṭavikalpaś ca vyavasāyaḥ samādhitā \veg\dontdisplaylinenum
            \var{\vc iṣṭā°\lem  \Ed; {\lost}ṣṭā° \msCa}%
            \var{\vd samādhitā\lem   \msCa; samādhinā \Ed}%
            \paral{\textit{\vcd {\normalfont  = MBh 12.247.10ab }}} 

manaso dvividha\.m rūpa\.m manaś conmana eva ca\thinspace{\dandab} \dontdisplaylinenum
            \var{\vb conmana\lem  \Ed; cotmana \msCa}%

manas tv indriyabhāvatve unmanastvam atīndriya \veg\dontdisplaylinenum
            \var{\vd °tīndriye\lem  \eme; °nīndriye \msCa, °tīndriya \Ed}%

nig\textsubring{r}hītā vis\textsubring{r}ṣta\.m ca bandhamokṣau tu sādhanam\thinspace{\dandab} \dontdisplaylinenum
            \var{\vab \om\ \Ed}%

nig\textsubring{r}hītendriyagrāmaḥ svargamokṣau tu sādhanam \veg\dontdisplaylinenum

vis\textsubring{r}ṣṭe indriyagrāme duḥkhasa\.msārasādhanam\thinspace{\dandab} \dontdisplaylinenum
            \var{\vb duḥkha°\lem  \corr; {\lost}kha° \msCa, duḥkha\.m \Ed}%

sakala\.m niṣkala\.m caiva mana eva vidur budhāḥ \danda\dontdisplaylinenum

sakala\.m mananānātve ekatve mananiṣkalam \veg\dontdisplaylinenum


\alalfejezet{unmanaḥ}
vigatarāga uvāca~{\dandab}\dontdisplaylinenum 

manaḥ svavedya\.m lokānām unmanas tu na vidyate\thinspace{\danda} \dontdisplaylinenum

unmanaḥ kathayāsmāka\.m kīd\textsubring{r}śa\.m lakṣaṇa\.m bhavet \veg\dontdisplaylinenum

anarthayajña uvāca~{\dandab}\dontdisplaylinenum 

unmanastva\.m gate vipra nibodha daśalakṣaṇam\thinspace{\danda} \dontdisplaylinenum
            \var{\vb nibodha\lem  \Ed; {\lost}bodha \msCa}%

na śabda\.m ś\textsubring{r}ṇute śrotra\.m śaṅkhabherīsvanād api \veg\dontdisplaylinenum
            \var{\vc śrotra\.m\lem  \msCa; śrotre \Ed}%

tvacaḥ sparśa\.m na jānāti śītoṣṇam api duḥsaham\thinspace{\dandab} \dontdisplaylinenum

rūpa\.m paśyati no cakṣuḥ parvatābhyadhiko 'pi vā \veg\dontdisplaylinenum

jihvā rasa\.m na vindeta madhurāmlavato 'pi vā\thinspace{\dandab} \dontdisplaylinenum
            \var{\vb °mlavato\lem  \Ed; °mlavano \msCa}%

gandha\.m jighrati na ghrāṇā tīkṣṇa\.m vāpy aśucīny api \veg\dontdisplaylinenum
            \var{\vc ghrāṇā\lem  \msCa; ghrāṇo \Ed}%

unmanas tava me khyāta\.m sarvadvaitavināśanam\thinspace{\dandab} \dontdisplaylinenum
            \var{\va unmanas tava me\lem  \Ed; {\lost}{\lost}{\lost}{\lost}{\lost}{\lost} \msCa}%

bhavapāragasuvyakta\.m niṣkala\.m śivam avyayam \veg\dontdisplaylinenum

sa śivaḥ sa paro brahmā sa viṣṇuḥ sa paro 'kṣaraḥ\thinspace{\dandab} \dontdisplaylinenum

sa sūkṣmaḥ sa paro ha\.msaḥ so 'kṣaraḥ kṣaravarjitaḥ \veg\dontdisplaylinenum

eṣa unmana jānīhi śivaś ca dvijapuṅgava\thinspace{\dandab} \dontdisplaylinenum

kīrtito 'smi samāsena kim anyat parip\textsubring{r}cchasi \veg\dontdisplaylinenum
            \var{\vd parip\textsubring{r}cchasi\lem  \Ed; pa\uncl{ri}{\lost}{\lost}{\lost} \msCa}%


\jump
\begin{center}
\ketdanda iti v\textsubring{r}ṣasārasa\.mgrahe pañcavi\.mśatitattvanirṇayo nāma vi\.mśatimo 'dhyāyaḥ\ketdanda
\end{center}
\dontdisplaylinenum\vers 
            \var{{\normalfont Colophon: }  vi\.mśatimo\lem  \msCa; vi\.mśatitamo \Ed}%
\bekveg\szamveg\vfill\phpspagebreak\szam\bek\versno=0\fejno=21
\thispagestyle{empty}


\vers


\alfejezet{\textbf{ekavi\.mśatimo 'dhyāyaḥ}}\jump\jump

\alalfejezet{kalpanirṇayaḥ}
vigatarāga uvāca~{\dandab}\dontdisplaylinenum 

aho matimatā\.m śreṣṭha aho dharmabh\textsubring{r}tā\.m vara\thinspace{\danda} \dontdisplaylinenum
            \var{\va matimatā\.m\lem  \mssCaCbCc\msNa; matimanā \Ed}%
            \var{\vb vara\lem  \msCa\msCc\msNa; varaḥ \msCb\Ed}%

aho dama śamaḥ satya aho yajña aho tapaḥ \veg\dontdisplaylinenum
            \var{\vc dama śamaḥ\lem  \msCa\msCb\msNa; damaḥ śamaḥ \msCc\Ed}%

anenām\textsubring{r}tavākyena vismayo me paro gataḥ\thinspace{\dandab} \dontdisplaylinenum

prīto 'smi ca tapādhārajñānādbhutarasena ca \veg\dontdisplaylinenum
            \var{\vc prīto 'smi ca\lem  \msCb\msCc\msNa\Ed; \uncl{pr}{\lost}{\lost} ca \msCa}%
            \var{\vd °bhuta°\lem  \mssCaCbCc\msNa; °bhūta° \Ed}%

ki\.m dadāmi vara\.m brūhi dātāsmi tava cepsitam\thinspace{\dandab} \dontdisplaylinenum
            \var{\vb cepsitam\lem  \msCa\msCc\msNa\Ed; cesmitam \msCb}%

etac chrutvā tatas tena pratyuvāca śubhā\.m giram \veg\dontdisplaylinenum
            \var{\vd śubhā\.m giram\lem  \mssCaCbCc\msNa; śubhāṅgirām \Ed}%

[anarthayajña uvāca~{\dandab}\dontdisplaylinenum ]

ko bhavān varadaśreṣṭha devadānavarākṣasāḥ\thinspace{\danda} \dontdisplaylinenum
            \var{\va bhavān\lem  \msCapcorr\msCb\msCc\msNa\Ed; bhagavān \msCaacorr\oo
                 varada śreṣṭha\lem  \mssCaCbCc\msNa; varadaḥ śreṣṭhaḥ \Ed}%
            \var{\vb °rākṣasāḥ\lem  \mssCaCbCc\msNa; °rākṣasaḥ \Ed}%

athavā bhagavān viṣṇur mama jijñāsur āgataḥ \veg\dontdisplaylinenum

vyakta\.m tvā\.m puruṣaśreṣṭha jānāmi puruṣottama\thinspace{\dandab} \dontdisplaylinenum
             \var{\va vyakta\.m tvā\.m\lem  \msCa\msCb\msNa; vyaktatva\.m \msCc\Ed\oo
                 °śreṣṭha\lem  \mssCaCbCc\msNa; °śreṣṭhaḥ \Ed}%
             \var{\vb puruṣottama\lem  \msCb\msNa;
                 \uncl{pu}{\lost}{\lost}ttama \msCa, puruṣotta{\lost} \msCc, puruṣottamaḥ \Ed}%

rūpa\.m darśaya govinda yady asti tapasaḥ phalam \veg\dontdisplaylinenum
            \var{\vc rūpa\.m darśaya govinda\lem  \msCa\msCb\msNa\Ed; 
                                                {\lost}{\lost}{\lost}{\lost}{\lost}{\lost}vinda \msCc}% 

tatas tu puṇḍarīkākṣo darśayāmāsa svā\.m tanum\thinspace{\dandab} \dontdisplaylinenum

śaṅkhacakragadāpāṇiḥ pītāmbaradharo hariḥ \veg\dontdisplaylinenum

anarthayajñas ta\.m d\textsubring{r}ṣṭvā vismaya\.m parama\.m gataḥ\thinspace{\dandab} \dontdisplaylinenum

praharṣam atula\.m labdhvā aśrupūrṇākulekṣaṇaḥ \veg\dontdisplaylinenum
            \var{\vc labdhvā\lem  \msCa\msCb\msNa\Ed; labba \msCc}%

vepamānasvareṇātra uvāca ca janārdanam\thinspace{\dandab} \dontdisplaylinenum
            \var{\vab vepamānasvareṇātra uvāca ca janārdanam\lem  \msCb\msNa; 
                vepamāna{\lost}{\lost}{\lost}{\lost}{\lost}{\lost}ca ca janārdanam \msCa,
                vepamāna{\lost}{\lost}{\lost}tra u{\lost}{\lost}{\lost}{\lost}{\lost}{\lost}{\lost} \msCc, 
                vepamānasvareṇārta uvāca ca janārdanam \Ed}%

adya me saphala\.m janma adya me saphala\.m tapaḥ \veg\dontdisplaylinenum
            \var{\vc adya me saphala\.m janma\lem  \msCa\msCb\msNa\Ed;
                                \uncl{adya}{\lost}{\lost}{\lost}{\lost}\uncl{janma} \msCc}%
                     \paral{\textit{\vcd {\normalfont  cf.\ MBh 5.113.5ab: } adya me saphala\.m janma tārita\.m cādya me kulam
                                {\normalfont  and MBh  13.14.179a: } adya jāto hy aha\.m deva adya me saphala\.m tapaḥ}}

\ujvers\nemsloka 
namo namas te 'stu janādisambhave
\dontdisplaylinenum
            \var{\vo \om\ \msCb}%

\nemslokab 
namo namas te 'stu ca viśvarūpiṇe \danda\dontdisplaylinenum
            \var{\vb namas te\lem  \mssCaCbCc\msNapcorr\Ed; namas tu \msNaacorr}%

\nemslokac 
namo namas te 'stu janābhisambhave
\dontdisplaylinenum
            \var{\vc \om\ \msNa}%

\nemslokad 
namo namas te 'stu pitāmahodbhave \veg\dontdisplaylinenum

\ujvers\nemsloka 
namo namas te 'stu sahasraśīrṣiṇe
\dontdisplaylinenum
            \var{\va \om\ \msCb}%
            \var{\va °śīrṣiṇe\lem  \msCa\msCb\msNa\Ed; °śīrṣaṇe \msCc}%

\nemslokab 
namo namas te 'stu sahasracakṣuṣe \danda\dontdisplaylinenum
            \var{\vb \om\ \msCb}%

\nemslokac 
namo namas te 'stu sahasraliṅgine
\dontdisplaylinenum
            \var{\vc \om\ \msCb}%

\nemslokad 
namo namas te 'stu sahasravakṣase \veg\dontdisplaylinenum
            \var{\vo \om\ \msCa}%

\ujvers\nemsloka 
namo namas te 'stu sahasramūrtaye
\dontdisplaylinenum

\nemslokab 
namo namas te 'stu sahasrabāhave \danda\dontdisplaylinenum

\nemslokac 
namo namas te 'stu sahasravaktriṇe
\dontdisplaylinenum
            \var{\vc °vaktriṇe\lem  \msCa\Ed; 
                        °cakriṇe \msCb\msCc, °vakriṇe \msNa}%

\nemslokad 
namo namas te 'stu sahasramāyine \veg\dontdisplaylinenum
            \var{\vo \om\ \msCa}%

\ujvers\nemsloka 
namo namas te 'stu varāharūpiṇe
\dontdisplaylinenum
            \var{\va \om\ \msCa}%

\nemslokab 
namo namas te 'stu mahīsamuddh\textsubring{r}te \danda\dontdisplaylinenum

\nemslokac 
namo namas te 'stu ca bhūtas\textsubring{r}ṣṭine
\dontdisplaylinenum
            \var{\vc °s\textsubring{r}ṣṭine\lem  \msCb\msCc\msNa\Ed; °s\textsubring{r}{\lost}{\lost} \msCa}%

\nemslokad 
namo namas te caturāśramāśraye \veg\dontdisplaylinenum
            \var{\vd namas te\lem  \msCa\msCb\msNapcorr\Ed; namas te stu \msCc\msNaacorr\oo
                 °śraye\lem  \msCa\msCc\msNa\Ed; °śrame \msCb}%

\ujvers\nemsloka 
namo namas te narasi\.mharūpiṇe
\dontdisplaylinenum

\nemslokab 
namo namas te ditijoradāriṇe \danda\dontdisplaylinenum
            \var{\vb namo namas te ditijoradāriṇe\lem  \mssCaCbCc\Ed; \om\ \msNa,
                         namo namas te 'ditijoradāraṇe \Ed}%

\nemslokac 
namo namas te 'suracakrasūdane
\dontdisplaylinenum
            \var{\vc °cakra°\lem  \conj; °śakra° \mssCaCbCc\msNa\Ed}%

\nemslokad 
namo namas te 'suradarpanāśane \veg\dontdisplaylinenum

\ujvers\nemsloka 
namo namas te ditiputradāmane
\dontdisplaylinenum
            \var{\va °dāmane\lem  \msCa\msCb\msNa; °vāmane \msCc\Ed}%

\nemslokab 
namo namas te baliyajñasūdane \danda\dontdisplaylinenum

\nemslokac 
namo namas te 'stu ṣaḍardhavikrame
\dontdisplaylinenum
            \var{\vc ṣaḍardhavikrame\lem  \msCb\msCc\msNa\Ed; 
                        \uncl{ṣa}{\lost}{\lost}{\lost}krame \msCa}%

\nemslokad 
namo namas te tridaśārtināśane \veg\dontdisplaylinenum

\ujvers\nemsloka 
namo namas te 'stu ananta acyute
\dontdisplaylinenum

\nemslokab 
namo namas te jagadartināśane \danda\dontdisplaylinenum
            \var{\vb jagada°\lem  \mssCaCbCc\msNa; jagadā° \Ed}%

\nemslokac 
namo namas te madhukaiṭanāśane
\dontdisplaylinenum
            \var{\vc °kaiṭa°\lem  \msCa\msCb\msNa; °kīṭa° \msCc\Ed}%

\nemslokad 
namo namas te 'stu trilokabāndhave \veg\dontdisplaylinenum

\ujvers\nemsloka 
namo namas te tridaśābhinandane
\dontdisplaylinenum

\nemslokab 
namo namas te 'stu ca divyacakṣuṣe \danda\dontdisplaylinenum

\nemslokac 
namo namas te 'stu bhavāntapārage
\dontdisplaylinenum

\nemslokad 
namo namas te 'stu trilokapūjite \veg\dontdisplaylinenum

\ujvers\nemsloka 
namo namas te 'stu gadāgrapāṇaye
\dontdisplaylinenum

\nemslokab 
namo namas te varacakrapāṇaye \danda\dontdisplaylinenum

\nemslokac 
namo namas te 'stu ca śaṅkhapāṇaye
\dontdisplaylinenum

\nemslokad 
namo namas te 'stu ca kambupāṇaye \veg\dontdisplaylinenum

\ujvers\nemsloka 
namo namas te 'stu jalaughaśāyine
\dontdisplaylinenum
            \var{\va jalaugha°\lem  \msCa\msCb\Ed; jalogha° \msCc\msNa}%

\nemslokab 
namo namas te haramardarūpiṇe \danda\dontdisplaylinenum
            \var{\vb namas te haramardarūpiṇe\lem  \msCb\msNa\Ed; 
                        nama{\lost}{\lost}{\lost}{\lost}rddarūpiṇe \msCa,
                        {\lost}{\lost}{\lost}{\lost}{\lost}{\lost}{\lost}marddarūpiṇe \msCc}%

\nemslokac 
namo namas te khagarājaketave
\dontdisplaylinenum
            \var{\vc °ketave\lem  \mssCaCbCc\msNa; °ketane \Ed}%

\nemslokad 
namo namas te śaśisūryalocane \veg\dontdisplaylinenum

\ujvers\nemsloka 
namo namas te uragārivāhane
\dontdisplaylinenum

\nemslokab 
namo namas te 'dbhutarūpadarśine \danda\dontdisplaylinenum

\nemslokac 
namo namas te 'yutasūryatejase
\dontdisplaylinenum
            \var{\vc 'yuta°\lem  \mssCaCbCc\msNa; 'stu ca \Ed\oo
                 °tejase\lem  \msCa\msCb\msNa\Ed; °te \msCc}%

\nemslokad 
namo namas te 'm\textsubring{r}tamanthanadhruve \veg\dontdisplaylinenum

\ujvers\nemsloka 
namo namas te 'maralokasa\.mstute
\dontdisplaylinenum
            \var{\va 'maralokasa\.mstute\lem  \mssCaCbCc;  maralokavandite \msNa,
                                 malalokasa\.mstute \Ed}%

\nemslokab 
namo namas te jagamaṇḍapāśraye \danda\dontdisplaylinenum
            \var{\vb namo namas te jagamaṇḍapāśraye\lem  \msCa\msCb\Ed;
                                                 {\lost}{\lost}{\lost}{\lost}{\lost}{\lost}{\lost}{\lost}{\lost}{\lost}\uncl{śraye} \msCc,
                                                \om\ \msNa}%

\nemslokac 
namo namas te jagadekavatsale
\dontdisplaylinenum
            \var{\vc jagadeka°\lem   \msCa\msCb\msNa\Ed; jaga\uncl{de}ka° \msCc\oo      
                °vatsale\lem  \mssCaCbCc\msNa; °vatsare \Ed}%

\nemslokad 
namo namas te śivasarvade namaḥ \veg\dontdisplaylinenum
            \paral{\textit{\vd {\normalfont Cf.\ B\textsubring{r}hatkālottara (NGMPP B 29/59) f.\ 87a: } 
                jñāna 2 śabda 2 sūkṣma 2 śivasarvada o\.m namaḥ śivāya{\thinspace\danda}}}

\ujvers\nemsloka 
kṣamasva govinda mamāparādham
\dontdisplaylinenum
            \var{\va mamā°\lem  \msCa\msCb\msNa\Ed; mama \msCc}%

\nemslokab 
atīva p\textsubring{r}ṣṭena durātmanena \danda\dontdisplaylinenum
            \var{\vb °tmanena\lem  \msCa\msCc\msNa\Ed; °tmane \msCb}%

\nemslokac 
mayeda sarva\.m kathita\.m smayena
\dontdisplaylinenum
            \var{\vc mayeda\lem  \mssCaCbCc\msNa; mayeda\.m \Ed\ \unmetr}%

\nemslokad 
dayā\.m kuru tva\.m tridaśeśvareṇa \veg\dontdisplaylinenum
            \var{\vd °śeśvareṇa\lem  \msCa\msCc\msNa\Ed; °śaiśvareṇa \msCb}%

\vers

vaiśampāyana uvāca~{\dandab}\dontdisplaylinenum 

stotreṇānena sa\.mtuṣṭaḥ keśavaḥ paravīrahā\thinspace{\danda} \dontdisplaylinenum
            \var{\vc keśavaḥ paravīrahā\lem  \msCa\msCb\msNa\Ed; 
                        keśava\uncl{ḥ paravīrahā} \msCc\ \toplost}%

pratyuvāca mahāseno girayā nirupasp\textsubring{r}hā \veg\dontdisplaylinenum
            \var{\vc pratyuvāca\lem  \msCa\msCb\msNa\Ed; \uncl{pratyuvāca} \msCc\ \toplost\oo
                 mahāseno\lem  \msCb\msCc\msNa\Ed; ma{\lost}{\lost}{\lost} \msCa}%
            \var{\vd girayā\lem  \msCa\msCb\msNa\Ed; giriyā \msCc\oo      
                 nirupa°\lem  \mssCaCbCc\msNa; nirūpa° \Ed}%

stotreṇānena me tāta tuṣṭo 'smi bh\textsubring{r}śam ejitaḥ\thinspace{\dandab} \dontdisplaylinenum
            \var{\va me tāta\lem  \msCa\msCc\Ed; mattāta \msCb, sa\.mtāta \msNa}%

durlabhāny api trailokye dadāmi varam īpsitam \veg\dontdisplaylinenum
            \var{\vc trailokye\lem  \msCa\msCb\msNa\Ed; trailokya \msCc}%

\ujvers\nemsloka 
anena mā\.m stauti nirāśritena
\dontdisplaylinenum
            \var{\va stauti\lem  \msCa\msCb\msNa\Ed; stoti \msCc}%

\nemslokab 
tvayoktavedārthamanohareṇa \danda\dontdisplaylinenum

\nemslokac 
yāvanti tatrākṣarasa\.mkhyam asti
\dontdisplaylinenum

\nemslokad 
tāvanti kalpān divi te vasanti \veg\dontdisplaylinenum
            \var{\vd kalpān\lem  \msCa\msNa; kalpa\.m \msCb\Ed, kalpa \msCc}%

\ujvers\nemsloka 
tva\.m cāpi me brūhi vara\.m yatheṣṭa\.m
\dontdisplaylinenum
            \var{\va tva\.m cāpi me brūhi\lem  \msCb\msCc\msNa\Ed;
                                        tvañc{\lost}{\lost}{\lost}{\lost}hi \msCa}%

\nemslokab 
trailokyarājyād api nirviśaṅkam \danda\dontdisplaylinenum
            \var{\vb °rājyā°\lem  \mssCaCbCc\Ed; °rā° \msNaacorr,
                        °rājā° \msNapcorr\oo
                 °śaṅkam\lem  \mssCaCbCc\msNa; °śaṅka \Ed}%

\nemslokac 
dadāmi ki\.m saptamahīśvaratvam
\dontdisplaylinenum
            \var{\vc ki\.m\lem  \msCa\msCc\msNa\Ed; ki \msCb\oo
                 °tvam\lem  \msCa\msCb\msNa\Ed; °tva\.mm \msCc}%

\nemslokad 
athārtharāśi\.m bahukanyakā\.m vā \veg\dontdisplaylinenum
            \var{\vd athārtharāśi\.m\lem  \msCa\msCb\msNa; athārtharāśi \msCc,
                                        athārtha\.m rāśī\.m \Ed\oo
                 °kanyakā\.m vā\lem  \msNa\Ed; °kanyakā vā \msCa\msCc,
                                                °kanyakā{\lost}{ā}\msCb}%

\ujvers\nemsloka 
śrutvaiva divya\.m varam acyutasya
\dontdisplaylinenum
            \var{\va śrutvaiva\lem  \Ed;
                         anarthayajña uvāca{\danda} śrutvaiva \msCa\msCb\msNa, 
                         vigatarāga uvāca{\danda} śrutvaiva \msCc}%

\nemslokab 
praṇamya pādadvayapaṅkaje tu \danda\dontdisplaylinenum
            \var{\vb °je tu\lem  \msCa\msCc\Ed; °hetu \msCb, °je nu \msNa}%

\nemslokac 
vijñāya viṣṇu\.m varada\.m vareṇya\.m
\dontdisplaylinenum
            \var{\vc \om\ \mssCaCbCc\msNa}%

\nemslokad 
? prah\textsubring{r} cetaḥ pukāncito 'to 'bravīt \veg\dontdisplaylinenum
            \var{\vd \om\ \mssCaCbCc\msNa}%

\vers

anarthayajña uvāca~{\dandab}\dontdisplaylinenum 
            \var{\vo anarthayajña uvāca\lem  \Ed; \om\ \mssCaCbCc\msNa}%

\nemsloka 
na kāmaye 'nyapravara\.m tu deva
\dontdisplaylinenum
            \var{\va 'nyapravara\.m tu\lem  \msCa\msCc\msNa; 
                                 nyaprabhavan tu \msCb, 
                                'nya\.m pravara\.m tu \Ed}%

\nemslokab 
asa\.mśaya\.m bandhanasāram ekam \danda\dontdisplaylinenum
            \var{\vb asa\.mśaya\.m\lem  \mssCaCbCc\msNa; asa\.mśaya \Ed\oo
                °sāram ekam\lem  \msCb\msCc\msNa\Ed; °sārame{\lost} \msCa}%

\nemslokac 
vimuktabandho bhavataḥ prasādād
\dontdisplaylinenum
            \var{\vc vimuktabandho\lem  \msCb\msCc\msNa\Ed; {\lost}{\lost}{\lost}{\lost}{\lost} \msCa\oo
                 prasādād\lem  \mssCaCbCc\msNa; pramādād \Ed}%

\nemslokad 
bhavāmi govinda rataś ca dharme \veg\dontdisplaylinenum
            \var{\vd rataś\lem  \mssCaCbCc\msNa; ratañ \Ed}%

\vers

bhagavān uvāca~{\dandab}\dontdisplaylinenum 

\nemsloka 
yathaiva citta\.m tava suprasanna\.m
\dontdisplaylinenum

\nemslokab 
maharṣi devair api naiva d\textsubring{r}ṣṭam \danda\dontdisplaylinenum

\nemslokac 
akalmaṣa\.m duḥkhavivarjitatvam
\dontdisplaylinenum
            \var{\vc akalmaṣa\.m\lem  \Ed; 
                      akalmaṣas tva\.m \msCa\msNa\ \unmetr,
                      akalmaṣatva\.m \msCb\ \unmetr,
                      akalmatva\.m \msCc\ \unmetr\oo
                 duḥkha°\lem  \msCapcorr\msCb\msCc\msNa\Ed; 
                                duḥ° \msCaacorr}%

\nemslokad 
bhavārṇavas tīrṇam asa\.mśayena \veg\dontdisplaylinenum

\ujvers\nemsloka 
gacchāma bho sāmprata śvetadvīpam
\dontdisplaylinenum
            \var{\va gacchāma bho\lem  \mssCaCbCc\msNa; gacchāmato \Ed\oo
                 sāmprata\lem  \mssCaCbCc\msNa; samprati \Ed}%

\nemslokab 
agamya devair api durnirīkṣyam \danda\dontdisplaylinenum
            \var{\vb durnirīkṣyam\lem  \msCb\msCc\msNa\Ed;
                                 durnirī\uncl{kṣ}{\lost} \msCa}%

\nemslokac 
madbhaktipūtamanasā prayāti
\dontdisplaylinenum
            \var{\vc madbhakti°\lem  \msCb\msCc\msNa\Ed; {\lost}{\lost}kti \msCa\oo
                °pūta\lem  \mssCaCbCc\msNa; °pūta\.m \Ed}%

\nemslokad 
ghorārṇave naiva punaś caranti \veg\dontdisplaylinenum

\vers

vaiśampāyana uvāca~{\dandab}\dontdisplaylinenum 
            \var{\vo vaiśampāyana uvāca\lem  \msCa\Ed; \om\ \msCb\msCc\msNaacorr,
                                 vaiśa\.m u \msNapcorr}%

evam uktvā haris tatra kare g\textsubring{r}hya tapodhanam\thinspace{\danda} \dontdisplaylinenum
            \var{\vb g\textsubring{r}hya tapodhanam\lem  \mssCaCbCc\Ed;
                         {\lost}{\lost}{\lost}{\lost}dha\uncl{na} \msNa}%

tataḥ so 'ntarhitas tatra tenaiva saha keśavaḥ \veg\dontdisplaylinenum
            \var{\vc so 'ntarhitas ta°\lem  \msCa\msCb;
                                 \uncl{so nta}rhitas ta° \msNa,
                                          te ntarhitās ta° \msCc,
                                          te karhitās ta° \Ed}%
            \var{\vd keśavaḥ\lem  \mssCaCbCc\msNa; keśava \Ed}%

\ujvers\nemsloka 
eva\.m hi dharmas tv adhikaprabhāvād
\dontdisplaylinenum
            \var{\va adhika°\lem  \msCa\msCb\msNa; adhika\.m \msCc\Ed}%

\nemslokab 
gataḥ sa loka\.m puruṣottamasya \danda\dontdisplaylinenum

\nemslokac 
aśeṣabhūtaprabhavāvyayasya
\dontdisplaylinenum

\nemslokad 
sanātana\.m śāśvatam akṣarasya \veg\dontdisplaylinenum
            \var{\vd sanātana\.m\lem  \Ed; sanātana \msCa\ \toplost\         
                                                \msCb\msCc\msNa\oo
                 °kṣarasya\lem  \msCb\msCc\msNa\Ed; {\lost}{\lost}{\lost} \msCa}%

\ujvers\nemsloka 
tvam eva bhakti\.m kuru keśavasya
\dontdisplaylinenum
            \var{\va tvam eva\lem  \msCb\msCc\msNa\Ed; {\lost}m eva \msCa}%

\nemslokab 
janārdanasyāmitavikramasya \danda\dontdisplaylinenum

\nemslokac 
yathā hi tasyaiva dvijarṣabhasya
\dontdisplaylinenum
            \var{\vc hi tasyaiva\lem  \msCa\msCb\Ed; jitasyaiva \msCc, 
                        \uncl{hi tasyava} \toplost\ \msNa}%

\nemslokad 
gati\.m labhasva puruṣottamasya \veg\dontdisplaylinenum

\ujvers\nemsloka 
kim anya bhūyaḥ kathayāmi rājan
\dontdisplaylinenum
            \var{\va kim anya bhū°\lem  \msCc\Ed; 
                        kim anyad bhū° \msCa\msCb\msNa\ \unmetr}%

\nemslokab 
yad asti kautūhalam anyaśeṣam \danda\dontdisplaylinenum

\nemslokac 
p\textsubring{r}cchasva mā\.m tāta yathepsita\.m te
\dontdisplaylinenum

\nemslokad 
bhaviṣyabhūta\.m bhavato yatheṣṭam \veg\dontdisplaylinenum
            \var{\vd yatheṣṭam\lem  \msCa\msCb\msNa\Ed; yatheṣṭa \msCc}%

\vers

janamejaya uvāca~{\dandab}\dontdisplaylinenum 
            \var{\vo janamejaya uvāca\lem  \mssCaCbCc\msNapcorr\Ed; \om\ \msNaacorr}%

\nemsloka 
kiyanti kalpāni gatāni pūrvam
\dontdisplaylinenum
            \var{\va kiyanti\lem  \mssCaCbCc\msNa; kiyanta \Ed}%

\nemslokab 
bhaviṣyakalpāni kiyanti vipra \danda\dontdisplaylinenum

\nemslokac 
ekaikakalpa\.m kiyad indram uktam
\dontdisplaylinenum
            \var{\vc °kalpa\.m\lem  \msCa\msCc\msNa\Ed; °kalpa \msCb\msNa}%

\nemslokad 
pravartamānād api kīrtayasva \veg\dontdisplaylinenum

\vers

vaiśampāyana uvāca~{\dandab}\dontdisplaylinenum 
            \var{\vo vaiśampāyana\lem  \msCa\msCc\msNa\Ed; 
                                        veśanampāyana \msCb}%

\nemsloka 
parārdhakalpa\.m gata pūrva rājyam
\dontdisplaylinenum

\nemslokab 
caturdaśaivendra narendra kalpam \danda\dontdisplaylinenum

\nemslokac 
tathaiva manvantara kalpam ekam
\dontdisplaylinenum
            \var{\vc manvantarakalpam ekam\lem  \mssCaCbCc\msNapcorr;
                        manvarakalpam ekam \msNaacorr,
                        manvantaram ekakalpam \Ed}%

\nemslokad 
bhaviṣyakalpa\.m ca parārdham eva \veg\dontdisplaylinenum
            \var{\vd kalpa\.m ca parārdham eva\lem  \msCb\msCc\msNa\Ed; 
                         ka{\lost}{\lost}{\lost}{\lost}{\lost}{\lost}{\lost} \msCa}%

\ujvers\nemsloka 
varāhakalpaḥ prathamo babhūva
\dontdisplaylinenum
            \var{\va babhūva\lem  \msCa\msCc\msNa\Ed; babhū \msCb}%

\nemslokab 
gatāś ca manvantara ṣaṭ narendra \danda\dontdisplaylinenum
            \var{\vb manvantaraṣaṭnarendra\lem  \msCa\msCb\msNapcorr;
                      manvaraṣaṭnarendra \msCc,
                      manvantaṣaṭnarendra \msNaacorr,
                      manvantaraṣaṭnarendraḥ \Ed}%

\nemslokac 
caturyuga\.m saptati ekayukta\.m
\dontdisplaylinenum
            \var{\vc °yuga\.m\lem  \mssCaCbCc\msNa; °yuga° \Ed\oo
                      ekayukta\.m\lem  \msCa\msCc\msNa\Ed; ekamukta\.m \msCb}%

\nemslokad 
manvantarā sa\.mkhyam udāharanti \veg\dontdisplaylinenum

\ujvers\nemsloka 
manvantarāṇā\.m ca caturdaśaiva
\dontdisplaylinenum
            \var{\vo \om\ \Ed}%
            \var{\va ca\lem  \msCa\msCb\msNa\Ed; tu \msCc}%

\nemslokab 
kalpasya sa\.mkhyā munayo vadanti \danda\dontdisplaylinenum

\nemslokac 
kalpāyutaś cāha pitāmahasya
\dontdisplaylinenum

\nemslokad 
tathā ca rātri\.m pravadanti tajjñāḥ \veg\dontdisplaylinenum

\ujvers\nemsloka 
ṣaḍlakṣakalpena tu māsam āhus
\dontdisplaylinenum
            \var{\va lakṣakalpena tu māsam āhus\lem  \msCb\msCc\msNa\Ed;
                       lakṣaka{\lost}{\lost}{\lost}{\lost}{\lost}m āhus \msCa}%

\nemslokad 
taddvādaśā varṣam udāharanti \veg\dontdisplaylinenum
            \var{\vb taddvādaśā va°\lem  \corr; tadvādaśā va° \msCa\msCb,
                         tatadvādaśā va° \msCc,
                         tadvādaśād va° \msNa, tvaddvādaśava° \Ed}%

\ujvers\nemsloka 
tenābdena parārdhakalpaguṇita\.m brahmāyur ity ucyate
\dontdisplaylinenum
            \var{\va °bdena\lem  \mssCaCbCc\msNa; °rdhena \Ed}%

\nemslokab 
trailokyādhipatiḥ pradhānapuruṣo brahmāpy anityaḥ sm\textsubring{r}taḥ \danda\dontdisplaylinenum
            \var{\vb °puruṣo\lem  \msCa\msCb\msNa\Ed; °puruṣā \msCc}% 

\nemslokac 
śeṣa\.m bhūtacaturvidhasya niyata\.m jīvasya ki\.m śocyate
\dontdisplaylinenum

\nemslokad 
tasmān nāsti jagatsusāravimala\.m muktvā śiva\.m śāśvatam \veg\dontdisplaylinenum
            \var{\vd °vimala\.m muktvā\lem  \msCc; °virala\.m muktvā \msCa\msCb\msNa,
                                 °viralamuktā \Ed}%

\vers


\jump
\begin{center}
\ketdanda iti v\textsubring{r}ṣasārasa\.mgrahe kalpanirṇayo nāmaikavi\.mśatimo 'dhyāyaḥ\ketdanda
\end{center}
\dontdisplaylinenum\vers 
            \var{{\normalfont Colophon: } °vi\.mśatimo\lem  \mssCaCbCc\msNa; °vi\.mśatitamo \Ed}%
\bekveg\szamveg\vfill\phpspagebreak\szam\bek\versno=0\fejno=22
\thispagestyle{empty}



\alfejezet{\textbf{dvāvi\.mśo 'dhyāyaḥ}}\jump\jump
\vers

janamejaya uvāca~{\dandab}\dontdisplaylinenum 

śruto 'thābjamukhād dharmasārasa\.mgraham uttamam\thinspace{\danda} \dontdisplaylinenum
            \var{\va śruto 'thābjamukhād dharma°\lem  \eme; 
        śruto vābjamukhād dharmaḥ \msCa, śruto vābjamukhod dharmaḥ \msCb, śruto vābjamukhā dharmaḥ \msCc, 
        śruto cābjamukhād dharmaḥ \msNa\msL, śruto cābdamukhā dharmaḥ \msNb, śrutvā vābjamukhād dharmaḥ \msNc,
                        śruto vā tvanmukhād dharmaḥ \Ed}%
            \paral{\textit{{\normalfont Testimonia for this chapter: \msCa\ ff.\thinspace 232r--234v, 
                                             \msCb\ ff.\thinspace 233v--235r, 
                                             \msCc\ ff.\thinspace 314r--317r,
                                             \msNa\ ff.\thinspace 39r--41v,
                                             \msNb\ ff.\thinspace 241v--243v, 
                                             \msNc\ ff.\thinspace 247v--250r;
                                                \mssCaCbCc\ = \msCa + \msCb + \msCc }}}

madhuraślakṣṇavāṇībhiḥ samyagvedārthasa\.myutam \veg\dontdisplaylinenum
            \var{\vc °ślakṣṇavāṇī°\lem  \msCb\msCc\msNa\msNb\msNc;
                       ślakṣṇaṇī° \msCa, °ślakṣyavānī° \msL, °ślakṣṇāvāṇī° \Ed}%

nyāyayukta\.m mahāsāra\.m guhyajñānam anuttaram\thinspace{\dandab} \dontdisplaylinenum
            \var{\va nyāyayukta\.m mahāsāra\.m\lem  \msCa\msCc\msNb\msNc\Ed; nyāyam ukta\.m mahat sāra\.m \msCb,
                                        nyāyayukta\.m mahat sāra\.m \msNa\msL}%
            \var{\vb guhya°\lem  \mssCaCbCc\msNa\msNb\msNc\msL; guhya\.m \Ed\oo
                 °nuttaram\lem  \msCa\msNa\msNb\msL; °nuttamam \msCb\msCc\msNc, °nantaram \Ed}%

t\textsubring{r}pto 'smīhām\textsubring{r}ta\.m pītvā janmam\textsubring{r}tyurujāpaham \veg\dontdisplaylinenum
            \var{\vcd pītvā janma°\lem  \msCb\msCc\msNa\msNb\msNc\msL\Ed; \uncl{pī}{\lost}{\lost}nma \msCa}%
            \var{\vd  °rujā°\lem  \msCa\msCc\msNa\msNb\msNc\msL\Ed; °mujā° \msCb}%

praśnam ekānya p\textsubring{r}cchāmi nāmahetu\.m tapodhana\thinspace{\dandab} \dontdisplaylinenum
            \var{\va praśna°\lem  \mssCaCbCc\msNa\msNc\msL\Ed; prasta° \msNb\oo
                 °kānya\lem  \mssCaCbCc\msNb\msNc; °kānyat \msNa\ \unmetr, 
                                   °kā\.mnyat \msL\ \unmetr, °konya \Ed}%
            \var{\vb nāma°\lem  \mssCaCbCc\msNa\msNb\msL\Ed; nāya° \msNc\oo
                 °hetu\.m\lem  \msCa\msCb\msNa\msL; °hetu \msCc\msNb\msNc\Ed\oo
                 °dhana\lem  \mssCaCbCc\msNb\msNc\Ed; °dhanam \msNa\msL}%

varṇagotrāśrama\.m tasmāc chrotum icchāmi te punaḥ \veg\dontdisplaylinenum
            \var{\vc varṇa°\lem  \mssCaCbCc\msNa\msNb\msNc\msBod\msL; varṇa\.m \Ed}%

vaiśampāyana uvāca~{\dandab}\dontdisplaylinenum 
            \var{\vo uvāca\lem  \mssCaCbCc\msNa\msNb\msL\Ed; {\lost}{\lost}{\lost} \msNc}%

ś\textsubring{r}ṇu rājann avahito yogendrasya mahātmanaḥ\thinspace{\danda} \dontdisplaylinenum
            \var{\va rājann a°\lem  \msCb\msCc\msNa\msNc\msL\Ed; rājan a° \msCa\msNb}%
            \var{\vab °vahito yogendrasya\lem  \mssCaCbCc\msNapcorr\msNb\msNc\Ed; °vahito yogendra \msNaacorr, 
                                                °hito yogandrasya \msL}%

āśrama\.m varṇajātīnā\.m vakṣyāmy eva narādhipa \veg\dontdisplaylinenum
            \var{\vd vakṣyāmy eva\lem  \msCa\msCc\msNa\msNb\Ed; vakṣyām eva \msCb\msNc\msL\oo
                 °pa\lem  \mssCaCbCc\msNa\msNb\msNc\msL; °paḥ \Ed}%

himavaddakṣiṇe pārśve m\textsubring{r}gendraśikhare n\textsubring{r}pa\thinspace{\dandab} \dontdisplaylinenum
            \var{\vb m\textsubring{r}gendra°\lem  \msCb\msCc\msNa\msNb\msNc\msL\Ed; \uncl{m\textsubring{r}}{\lost}ndra° \msCa\oo
                 n\textsubring{r}pa\lem  \mssCaCbCc\msNa\msNb; n\textsubring{r}paḥ \msNc\msL\Ed}%

mahendrapathagā nāma nadītīre narādhipa \veg\dontdisplaylinenum
            \var{\vc mahendra°\lem  \mssCaCbCc\msNa\msNc\Ed; m\textsubring{r}gendra° \msNb, mahindra° \msL}%
            \var{\vd °pa\lem  \mssCaCbCc\msNa\msNb\msNc\msL; °paḥ \Ed}%

tatrāśramapada\.m tasya puline sumanorame\thinspace{\dandab} \dontdisplaylinenum
            \var{\vb puline su°\lem  \msCa\msCb\msNa; pulineṣu \msCc\msNb\msNc\Ed, puline pu° \msL}%

vasati sma mahābhāgas tattvapāraganisp\textsubring{r}haḥ \veg\dontdisplaylinenum
            \var{\vc vasati\lem  \mssCaCbCc\msNa\msNb\msNc\Ed; vasanti \msL}%
            \var{\vd °pāraga°\lem  \msCa\msCc\msNa\msNb\msNc\msL\Ed; °pāra° \msCb\oo
                 °sp\textsubring{r}haḥ\lem  \mssCaCbCc\msNa\msNb\msNc\msL; °sp\textsubring{r}hāḥ \Ed}%

śīlaśaucasamācāro jitadvandvo jitaśramaḥ\thinspace{\dandab} \dontdisplaylinenum

jitamānabhayakrodho jitasarvaparigrahaḥ \veg\dontdisplaylinenum
            \var{\vd jita°\lem  \msCa\msCc\msNa\msNb\msNc\msL\Ed; jija° \msCb}%

somava\.mśaprasūtās te kṣatriyā dvijatā\.m gatāḥ\thinspace{\dandab} \dontdisplaylinenum
            \var{\va soma°\lem  \mssCaCbCc\msNa\msNb\msNc\Ed; soya° \msL\oo
                 prasūtās te\lem  \msCb\msCc\msNb\msNc\Ed; pra{\lost}{\lost}{\lost} \msCa, prasūtas te \msNa\msL}%
            \var{\vb kṣatriyā\lem  \mssCaCbCc\msNb; kṣatriyo \msNa\msNc\msL\Ed\oo
                 gatāḥ\lem  \mssCaCbCc\msNb\Ed; gataḥ \msNa\msNc\msL}% 

tapasā vinayācārair viṣṇunā dvijakalpitāḥ \veg\dontdisplaylinenum
            \var{\vc °cārair vi°\lem  \msCa\msCb\msNa\msNb\msNc\msL\Ed; °cārai vi° \msCc}%
            \var{\vd dvijakalpitāḥ\lem  \Ed; dvijaḥ kalpitaḥ \mssCaCbCc\msNc\ \unmetr,
                                                dvijakalpitaḥ \msNa\msNb\msL}%

ajitā nāma tat pūrva\.m kāmakrodhajitena tu\thinspace{\dandab} \dontdisplaylinenum
            \var{\va pūrva\.m\lem  \mssCaCbCc\msNb\msNc\Ed; pūrva \msNa\msL}%

sa\.mkalpas tasya rājendra kathayiṣyāmi tac ch\textsubring{r}ṇu \veg\dontdisplaylinenum
            \var{\vc sa\.mkalpas ta\lem  \mssCaCbCc\msNa\msNb\msNc\Ed; sa\.mkalpa ta \msL}%

adhyātmanagarasphītaḥ adhibhūtajanākulaḥ\thinspace{\dandab} \dontdisplaylinenum
            \var{\vab °sphītaḥ adhi°\lem  \msCb\msCc\msNa\msNb\msNc\msL\Ed; °sphītaradhi° \msCa}%

adhidaivatasā\.mnidhya\.m daśāyatana pañca ca \veg\dontdisplaylinenum
            \var{\vc °sā\.mnidhya\.m\lem  \msCa\Ed; sānaidhya\.m \msCb\msCc\msNa\msNb\msL, sānnaidhya\.m \msNc}%
            \var{\vd daśā°\lem  \mssCaCbCc\msNa\msNb\msNc\msL; deśā° \Ed}%
            \paral{\textit{\vo {\normalfont Cf.\ 4.72: } caturāyatana\.m vipra kathayiṣyāmi tac ch\textsubring{r}ṇu{\thinspace\danda}
                                 karuṇāmuditopekṣāmaitrī cāyātana\.m sm\textsubring{r}tam{\thinspace\ketdanda}}}

daśayajñavrata\.m cīrṇa\.m daśakāmaparājitaḥ\thinspace{\dandab} \dontdisplaylinenum
            \var{\va daśayajñavrata\.m cīrṇa\.m\lem  \msNa\msNb\msNc\msL; da\uncl{śayajña\.m} {\lost}{\lost}ñ cīrṇan \msCa, 
                        daśayajñavratacīrṇan \msCb\msCc, daśayajña\.m vrata\.m cīrṇa° \Ed}%
            \var{\vb °parājitaḥ\lem  \msCa\msCc\msNa\msNb\msNc\msL\Ed; °paparājitaḥ \msCb}%

niyamān daśa sa\.mśritya daśa vāyava \textsubring{r}tvijaḥ \veg\dontdisplaylinenum
            \var{\vc niyamān daśa\lem  \mssCaCbCc\msNa\msNb\msNc\Ed; nimāyā daśa \msLacorr, niyamā daśa \msLpcorr}%
            \paral{\textit{\vd {\normalfont cf.\ 11.17ab: } dhāraṇādhvaryuvat k\textsubring{r}tvā prāṇāyāmaś ca \textsubring{r}tvijaḥ}}

daśākṣareṇa mantreṇa daśadharmakriyāpadaḥ\thinspace{\dandab} \dontdisplaylinenum
            \var{\vb °dharmakriyāpadaḥ\lem  \msCa\msCb\msNa\msNb\msNc\msL\Ed; °dharmaḥ kripadaḥ \msCc}%

daśasa\.myamadīptāgnau jihvātejodaśendriyaḥ \veg\dontdisplaylinenum
            \var{\vc °sa\.myama°\lem  \mssCaCbCc\msNa\msNb\msNc\Ed; °sa\.mśaya° \msL\oo
                 °dīptā°\lem  \mssCaCbCc\msNa\msNc\msL; °dīpto \msNb, °dīpā° \Ed}%
            \var{\vd °daśe°\lem  \mssCaCbCc\msNa\msNb\msL; °jite° \msNc\Ed}%

daśayogāsanāsīno daśadhyānaparāyaṇaḥ\thinspace{\dandab} \dontdisplaylinenum
            \var{\va °sanāsīno\lem  \mssCaCbCc\msNa\msNb\msNc\msL; samāsīnā \Ed}%
            \var{\vb °yaṇaḥ\lem  \mssCaCbCc\msNb\msNc\Ed; °yaṇāḥ \msNa\msL}%

buddhir vedī mano yūpaḥ somapāno 'm\textsubring{r}tākṣaraḥ \veg\dontdisplaylinenum
            \var{\vc buddhir vedī\lem  \mssCaCbCc\msNa\msNb\msL; buddhi vedī \msNc, buddhir vedi \Ed}%
            \var{\vd °pāno 'm\textsubring{r}tākṣaraḥ\lem  \msCb\msNa\msNb\msNc\msL; {\lost}{\lost}{\lost}{\lost}{\lost}{\lost} \msCa, 
                                              °pānam\textsubring{r}tākṣaraḥ \msCc, °dānam\textsubring{r}tākṣaraḥ \Ed}%

dakṣiṇābhaya bhūtebhyaḥ paśubandha svaya\.mk\textsubring{r}taḥ\thinspace{\dandab} \dontdisplaylinenum
            \var{\va °bhaya\lem  \mssCaCbCc\msNa\msNb\msNc\msL; °gnaya \Ed}%

vinārtha\.m yajñam iṣṭvā tu kāla\.m ca kṣapayaty asau \danda\dontdisplaylinenum
            \var{\va °rtha\.m\lem  \msCa\msCb\Ed; °rtha° \msCc\msNa\msNb\msNc\msL}%
            \var{\vb kāla\.m\lem  \mssCaCbCc\msNa\msNb\msNc\msL; kālāñ \Ed\oo
                 kṣapayaty asau\lem  \mssCaCbCc\msNa\msNc\msL;
                        \uncl{kṣapayaty asau} \msNb, kṣapayaty asauḥ \Ed}%

anarthayajña\.m ta\.m prāhur munayas tattvadarśinaḥ \veg\dontdisplaylinenum
            \var{\vcd °yajña\.m ta\.m prāhur munayas ta°\lem  \msCa\msCb\msNb\msNc\Ed;
                °yajña ta\.m prāhu munayas ta° \msCc, °yajñan ta\.m prāhur munaya ta° \msNa,
                °yajña\.m prāhur munaya ta° \msL}%

janamejaya uvāca~{\dandab}\dontdisplaylinenum 

daśayajñam aha\.m śrotu\.m dehi mā\.m dvijasattama\thinspace{\danda} \dontdisplaylinenum
            \var{\va °yajñam aha\.m\lem  \mssCaCbCc\msNa\msNb\msNc\msLpcorr; °yajñam ida\.m \Ed}%
            \var{\vb mā\.m\lem  \msCa\msCb\msNa\msNb\msNc\msL\Ed; mā \msCc\oo
                 °ttama\lem  \mssCaCbCc\msNb\msNc\Ed; °ttamaḥ \msNa\msL}%

daśakāmadaśadhyāna\.m daśayogadaśākṣaram \veg\dontdisplaylinenum
            \var{\vc °daśadhyāna\.m\lem  \msCa\msCb\msNa\msNb\msNc; °daśadhyāna° \msCc\Ed, °datadhyānan \msL}%
            \var{\vd °kṣaram\lem  \msCb\msNb\msNc; °kṣara{\lost} \msCa, °kṣaraḥ \msCc\msNa\msL\Ed}%

vaiśampāyana uvāca~{\dandab}\dontdisplaylinenum 
            \var{\vo vaiśampāyana uvāca\lem  \msCb\msCc\msNa\msNb\msNc\msL\Ed; {\lost}{\lost}{\lost}{\lost}{\lost}{\lost}vāca \msCa}%

brahmadevapit\textsubring{r}yajño yajño bhūtātitheś ca ha\thinspace{\danda} \dontdisplaylinenum
            \var{\va °deva°\lem  \msCa\msCc\msNa\msNb\msNc\msL\Ed; °daiva° \msCb\oo
                 °yajño\lem  \msCa\msCb\msNa\msNb\Ed; °yojño \msNc, °yajña \msCc\msL}%
            \var{\vb yajño\lem  \msCa\msCb\msNa\msL; yajña° \msCc\msNb\msNc\Ed\oo
                 °titheś ca ha\lem  \msCb; °tithiś ca ha \msCa\msCc\msNa\msNb\msNc\msL, °tithiñ ca yaḥ \Ed}%
            \paral{\textit{\vb {\normalfont cf.\ Śatapathabrāhmana 11.5.6: } aharaharbhūtebhyo bali\.m haret tathaitam bhūtayajña\.m\oo
                {\normalfont Garuḍapurāṇa 1.50.71cd: } bhūtayajñaḥ sa vai jñeyo bhūtebhyo yastvaya\.m baliḥ}} 

japo yogas tapo dhyāna\.m svādhyāyaś ca daśa sm\textsubring{r}taḥ \veg\dontdisplaylinenum
            \var{\vc yogas tapo dhyāna\.m\lem  \mssCaCbCc\msNb\msNc\Ed; yoga{\lost}{\lost}\uncl{dhāna\.m} \msNa,
                                                yoga \gap\gap\ pāna\.m \msL}%
            \var{\vd svādhyāyaś ca\lem  \mssCaCbCc\msNb\msNc\Ed; \uncl{sādhyā}yaś ca \msNa, 
                                                sādhutapaś ca \msL}%

patnīputrapaśubh\textsubring{r}tyadhanadhānyayaśaḥśriyaḥ\thinspace{\dandab} \dontdisplaylinenum
            \var{\va °yaśaḥ°\lem  \msCa\msCb\msNa\msNb\msNc\msL; °yaśa° \msCc\Ed}%

māna bhoga daśa rājan daśakāma udāh\textsubring{r}taḥ \veg\dontdisplaylinenum
            \var{\vc °bhoga\lem  \mssCaCbCc\msNa\msNb\msNc\msL; °bhoga\.m \Ed}%
            \var{\vd °h\textsubring{r}taḥ\lem  \msCa\msCc\msNa\msNb\msNc\msL\Ed; °h\textsubring{r}tam \msCb}%

mānaso yaugapadyaś ca sa\.mkṣiptaś ca viśāmpate\thinspace{\dandab} \dontdisplaylinenum
            \var{\va yaugapadyaś ca\lem  \corr; yaugapadyañ ca \msCa\msCb\msNb,
                        yogapadya\.m ca \msCc\msNa\msNc\msL, yogapadyaś ca \Ed\
                 °kṣiptaś ca\lem  \Ed; °ksipta\.m ca \mssCaCbCc\msNa\msNb\msNc\msL}%
             \paral{\textit{\vo {\normalfont cf.\ Dharmaputrikā 1.56: } sa\.mkṣiptā prathamā jñeyā viśālā samanantaram{\thinspace\ketdanda}
                                        tato dvikaraṇī ceti trividho yoga ucyate{\thinspace\danda}}}

viśālā nāma yogaś ca tato dvikaraṇaḥ sm\textsubring{r}taḥ \veg\dontdisplaylinenum
            \var{\vc viśālā nāma yogaś ca\lem  \Ed; vi{\lost}{\lost}{\lost}{\lost} yogañ ca \msCa, 
                                        viśālā nāma yoga\.m ca \msCb\msCc\msNa\msNb\msNc\msL}%
            \var{\vd dvikaraṇaḥ\lem  \msCa\msCb\msNa\msL; vikaraṇaḥ \msCc\Ed, dvikaraṇī \msNb, dvikaraṇa \msNc}%

raviḥ somo hutāśaś ca sphaṭikāmbaram eva ca\thinspace{\dandab} \dontdisplaylinenum
            \var{\va raviḥ\lem  \msCa; ravi° \msCb\msCc\msNa\msNb\msNc\msL\Ed}%
            \var{\vb sphaṭikāmbara°\lem  \mssCaCbCc\msNb\msNc\Ed; sphaṭikā\.m{\lost}ra° \msNa, sphaṭikā\.msata° \msL}%
            \paral{\textit{\vab {\normalfont cf. Dharmaputikā 4:5cd: } sūryacandrahutāśārciḥsphāṭikāmbarasannibhāḥ}}

daśayogāsanāsīno nityam eva tapodhanaḥ \veg\dontdisplaylinenum
            \var{\vc daśayogāsanāsīno\lem  \msCa\msCc\msNa\msNb\msNc; daśayogasamāsīno \msCb,
                                       devayogāsatāsīno \msL, daśayogāsanāsīnau \Ed}%
            \var{\vd °dhanaḥ\lem  \msCa\msCb\msNa\msL; °dhana \msCc\msNb\msNc\Ed}%

anirodhamanāḥ sūkṣma\.m dhyāyed yogaḥ sa mānasaḥ\thinspace{\dandab} \dontdisplaylinenum
            \var{\va anirodha°\lem  \mssCaCbCc\msNa\msNb\msNc\msL; anilādha° \Ed\oo
                 °manāḥ\lem  \mssCaCbCc\msNa\msNc\msL\Ed; °manā \msNb}%
            \var{\vb dhyāyed yo°\lem  \msCa\msCb\msNa\msNb\msNc\msL; dhyāyo° \msCc, dhyāna\.m yo° \Ed}%
            \paral{\textit{\vab {\normalfont cf.\ Dharmaputrikā 1.54: } ak\textsubring{r}tvā prāṇasa\.mrodha\.m manasaikena kevalam{\thinspace\danda}
                                dhyāyeta parama\.m sūkṣma\.m sa yogo mānasaḥ sm\textsubring{r}taḥ{\thinspace\ketdanda}}}

prāṇāyāmair mano ruddhvā yaugapadyaḥ sa ucyate \veg\dontdisplaylinenum
            \var{\vc °yāmair ma°\lem  \msCa\msNa\msNb\msNcpcorr\msL\Ed;
                                 °yāmai ma° \msCb, °yāmai mma° \msCc, °yāmer ma° \msNcacorr\oo
                 ruddhvā\lem  \mssCaCbCc\msNa\msNb\msNc\msL; ruddhā \Ed}%
            \var{\vd yauga°\lem  \msCa\msCb\msNa\msNcpcorr\msL; yoga° \msCc\msNb\msNcacorr\Ed}%
            \paral{\textit{\vcd {\normalfont cf.\ Dharmaputrikā 1.55:} sa\.myamya manasā prāṇa\.m prāṇāyāmair manas tathā{\thinspace\danda}
                                        eva\.m dhyāyet para\.m sūkṣma\.m yaugapadyaḥ sa ucyate{\thinspace\ketdanda}}}

brahmādistambaparyanta\.m sarva\.m sthāvarajaṅgamam\thinspace{\dandab} \dontdisplaylinenum
            \var{\va °stamba°\lem  \mssCaCbCc\msNa\msNc\Ed; \om\ \msNb, °sta\.mbha° \msL\oo
                 °paryanta\.m\lem  \msCb\msCc\msNa\msL; °\uncl{dviya}{\lost}° \msCa, \om\ \msNb, °paryanta° \msNc\Ed}%
            \var{\vb sarva\.m\lem  \msCb\msNa; {\lost}{\lost} \msCa, sarva° \msCc\msNc\msL\Ed, \om\ \msNb}%
            \paral{\textit{\vab {\normalfont \kb\ Dharmaputrikā 1.57cd: } brahmādistambhaparyantāḥ sarve sthāvarajaṅgamāḥ}}

pralīyamāna\.m dhyāyeta kramāt sūkṣma\.m vicintayet \veg\dontdisplaylinenum
            \var{\vo \om\ \msNb}%
            \var{\vc pralīya°\lem  \mssCaCbCc\msNa\msNc\Ed; \om\ \msNb, praṇīya° \msL}%
            \var{\vd kramāt sū°\lem  \msCa\msCb\msNa\msNc\msL\Ed; kramā sū° \msCc, \om\ \msNb}%
            \paral{\textit{\vcd {\normalfont \kb\ Dharmaputrikā 1.59ab: } pralīyamānan dhyāyeta kramāc chūnya\.m bhavej jagat}}

sa\.mkṣipta eṣa ākhyāto viśālā\.m ch\textsubring{r}ṇu tattvataḥ\thinspace{\dandab} \dontdisplaylinenum
            \var{\va sa\.mkṣipta\lem  \mssCaCbCc\msNa\msNc\Ed; \om\ \msNb, sa\.mkṣiptaḥ \msL\oo
                 eṣa\lem  \mssCaCbCc\msNa\msNc\msL; \om\ \msNb, eva \Ed\oo
                 ākhyāto\lem  \msCb\msNc; ākhyātaḥ \msCa\msCc\msNa\msL\Ed, \om\ \msNb}%
            \paral{\textit{\vab {\normalfont cf.\ Dharmaputrikā 1.60ab: } eṣa yogavidhiḥ proktaḥ sa\.mkṣipto nāma nāmataḥ}}

brahmādisūkṣmaparyanta\.m cintayīta vicakṣaṇaḥ \veg\dontdisplaylinenum
            \var{\vo \om\ \msNb}%
            \var{\vc °sūkṣma°\lem  \mssCaCbCc\msNc\Ed; °sta\.mba° \msNa, \om\ \msNb, tava \msL\oo
                 °paryanta\.m\lem  \mssCaCbCc\msNa\msL; \om\ \msNb, °paryanta \msNc\Ed}%
            \var{\vd cintayīta\lem  \msCa\msCbpcorr\msCc\msNa\msNc\msL\Ed; \om\ \msNb, ciyīta \msCbacorr}%

sa\.mkṣiptā\.m ca viśālā\.m ca cintayīta parasparam\thinspace{\dandab} \dontdisplaylinenum
            \var{\va sa\.mkṣiptā\.m\lem  \msCb\msNc; sa\.mkṣiptā \msCapcorr\msCc\msNa\msL\Ed, \om\ \msCaacorr\msNb\oo
                 viśālā\.m\lem  \msCapcorr\msCb\msNc; \om\ \msCaacorr, viśālā \msCc\msNa\msL\Ed, \om\ \msNb}%

eṣā dvikaraṇī nāma yogasya vidhir ucyate \veg\dontdisplaylinenum
            \var{\vo \om\ \msNb}%
            \var{\vc dvi°\lem  \msCa\msCb\msNa\msNc\msL; vi° \msCc\Ed, \om\ \msNb}%
            \paral{\textit{\vo {\normalfont \kb\ Dharmaputrikā 1.62cd--63ab: } etau sa\.mhāravargau dvau pāramparyeṇa cintayet{\thinspace\ketdanda}
                                                       eṣā dvikaraṇī nāma yogasya vidhir iṣyate{\thinspace\danda}}}

dehamadhye h\textsubring{r}di jñeya\.m h\textsubring{r}dimadhye tu paṅkajam\thinspace{\dandab} \dontdisplaylinenum
            \var{\va jñeya\.m\lem  \msCa\msCb\msNa\Ed\msNc; jñeya \msCc\msL, jñe \msNbacorr; jñe{\lost} \msNbpcorr}%
            \var{\vb tu paṅkajam\lem  \msCb\msCc\msNa\msNb\msNc\msL\Ed; \uncl{tu} pa{\lost}{\lost} \msCa}%

paṅkajasya ca madhye tu karṇikā\.m viddhi gopate \veg\dontdisplaylinenum
            \var{\vc paṅkajasya ca\lem  \msCb\msCc\msNa\msNc\Ed; {\lost}ṅkajasya ca \msCa,
                                                        kaṅkasya tu \msNb, pankaja\.msya ca \msL}%
            \var{\vd karṇikā\.m viddhi gopate\lem  \msCa\msCb\msNa\msNb\msNc\msL; karṇiddhiddhi gopate \msCc, 
                                                               karṇikā\.m ca vi\.mśāpate \Ed}%

karṇikāyās tu madhye tu pañcabindu\.m vidur budhāḥ\thinspace{\dandab} \dontdisplaylinenum
            \var{\vb °bindu\.m\lem  \msCa\msNc; °bindu \msCb\msCc\msNa\msNb\msL\Ed}%

ravisomaśikhā\.m caiva sphaṭikāmbaram eva ca \veg\dontdisplaylinenum
            \var{\vc °śikhā\.m\lem  \msCa\msNa\msL; °śikhā \msCb\msCc\msNb\msNc\Ed}%
            \var{\vd sphaṭi°\lem  \msCa\msCc\msNa\msNb\msNc\msL\Ed; sphāṭi° \msCb}%
    \paral{\textit{\vcd {\normalfont cf.\ Dharmaputrikā 4.5cd: } sūryacandraprakā\-śārcisphāṭikāmbarasannibhāḥ}}

ravimaṇḍalamadhye tu bhāvayec candramaṇḍalam\thinspace{\dandab} \dontdisplaylinenum
            \var{\vb bhāvayec candramaṇḍalam\lem  \msCa\msCb\msNa\msNb\msNc\msL\Ed; bhāvaye candramaṇḍalaḥ \msCc}%

tasya madhye śikhā\.m dhyāyen nirdhūmajvalanaprabhām \veg\dontdisplaylinenum
            \var{\vc °śikhā\.m\lem  \msCa\msCb\msNa\msNb\msNc\msL; °śikhā \msCc\Ed}%

agnimadhye maṇi\.m dhyāyec chuddhadhārājalaprabham\thinspace{\dandab} \dontdisplaylinenum
            \var{\vab maṇi\.m dhyāyec chuddha°\lem  \msCb\msNa\msNb\msNc\msL\Ed; {\lost}{\lost}{\lost}{\lost}{\lost}{\lost} \msCa, 
                                                mani\.m dhyāyec chuddha° \msCc}%
            \var{\vb °dhārā°\lem  \msCa\msCb\msNa\msNb\msNc\msL; °dhāra° \msCc\Ed\oo
                  °prabham\lem  \msCc\msNa\msNb\msNc\msL\Ed; °prabhām \msCa\msCb}%

tasya madhye 'mbara\.m dhyāyet susūkṣma\.m śivam avyayam \veg\dontdisplaylinenum
            \var{\vc 'mbara\.m\lem  \msCa\msCb\msNa\msNb\msNc; 'mbara \msCc, bara\.m \msL, 'kṣara\.m \Ed\
             \vd susūkṣma\.m\lem  \msCc\msNa\msNc\msL; sūkṣma\.m \msCa, susūkṣma° \msCb, 
                                        \uncl{sva}sūkṣma° \msNb, sasūkṣma\.m \Ed}%

daśayogam ida\.m rājan kathita\.m ca mayā tava\thinspace{\dandab} \dontdisplaylinenum

daśadhyāna\.m samāsena kīrtita\.m ś\textsubring{r}ṇu tad yathā \veg\dontdisplaylinenum
            \var{\vc °dhyāna\.m\lem  \mssCaCbCc\msNa\msNc; °dhyāna \msNb\msL\Ed}%

ghoṣaṇī piṅgalā caiva vaidyutī candramālinī\thinspace{\dandab} \dontdisplaylinenum
            \var{\va ghoṣaṇī\lem  \mssCaCbCc\msNa\msNb\msNc\msL; ghoṣaṇā \Ed}%
            \var{\vb vaidyutī\lem  \msCa\msCb\msNa\msNb\msNc\msL\Ed; vidyuta \msCc, vidyutī \Ed}%

candrā mano'nugā caiva suk\textsubring{r}tā ca tathāparā \veg\dontdisplaylinenum
            \var{\vc candrā mano'nugā\lem  \msCb\msNa\msNb\msNc\msL; 
                                candrā manānugā \msCa, candramanonugā \msCc, candro mano'nugā \Ed}%
            \var{\vd suk\textsubring{r}tā ca tathāparā\lem  \msCa\msCc\msNa\msNc\msL; suk\textsubring{r}tā tathāparā \msCb, \om\ \msNb,
                                                                suk\textsubring{r}tā ca tathāpara \Ed}%

saumyā nirañjanā caiva nirālambā ca kīrtitā\thinspace{\dandab} \dontdisplaylinenum
            \var{\va saumyā nirañjanā caiva\lem  \msCb\msCc\msNa\msL\Ed; saumyā nirañjanā {\lost}{\lost} \msCa, \om\ \msNb,
                                                                saumyā ṇirañjanā caiva \msNc}%
            \var{\vb kīrtitā\lem  \mssCaCbCc\msNa\msNb\msNc\Ed; kīrtitāḥ \msL}%

supiṣitvāṅgulau śrotre dhvanim ākarṇayen naraḥ \veg\dontdisplaylinenum
            \var{\vc supiṣitvāṅgulau\lem  \msCa\msCb\msNa\msNb\msNc; su{\lost}{i}{}ṣicāṅgulau \msCc,
                                               supithitvāṅgulau \msL, suśiṣi cāṅgulau \Ed}%
            \var{\vd °karṇaye°\lem  \msNb; °karṣaye° \mssCaCbCc\msNa\msNc\Ed, °karṣaya° \msL}%

tat tad akṣaram ākarṇya am\textsubring{r}tatvāya kalpyate\thinspace{\dandab} \dontdisplaylinenum
            \var{\va °karṇya\lem  \mssCaCbCc\msNb\msNc\msL\Ed; °kaṇṇya \msNa}%

piṅgalā\.m tu śikhādhūmā\.m dhyāyen nityam atandritaḥ \veg\dontdisplaylinenum
            \var{\vc piṅgalā\.m tu śikhādhūmā\.m\lem  \msCa\msCb\msNb\msL; piṅgalā tu śikhādhūma\.m \msCc\Ed,
                                        piṅgalā\.mn tu śikhādhūmā\.m \msNa, piṅgalān tu śikhādhūmā \msNc}%
            \var{\vd °tandritaḥ\lem  \mssCaCbCc\msNa\msNb\msNc\Ed; °tendritaḥ \msL}%

vimuktaḥ sarvapāpebhyo nirdvandvapadam āpnuyāt\thinspace{\dandab} \dontdisplaylinenum
            \var{\va vimuktaḥ\lem  \msCa\msCb\msNa\msNb\msNc\msL\Ed; vimukta \msCc}%
            \var{\vb nirdvandva°\lem  \mssCaCbCc\msNc; nidvanda° \msNa\msNb\msL, nirdvanda° \Ed}%

vaidyutī tu niśāmadhye lakṣate 'jam anāmayam \veg\dontdisplaylinenum
            \var{\vc vaidyutī tu\lem  \mssCaCbCc\msNa\msNb\msNc\Ed; vaidyutīnta \msL}%
            \var{\vd lakṣate 'jam a°\lem  \msCc\Ed; lakṣye teja a° \msCa\msCb, lakṣyateja a° \msNa\msNb\msL,
                                                                                 lakṣateja a° \msNc}%

pañcamāsasadābhyāsād divyacakṣur bhaven naraḥ\thinspace{\dandab} \dontdisplaylinenum
            \var{\va pañcamāsasadā°\lem  \msCb\msNa\msNb\msL; \uncl{pa}{\lost}{\lost}sasadā° \msCa, pañcamāsassadā° \msCc, 
                                                      pañcamāsasamā° \Ed, pañcamāsa\.m sadā° \msNc}%
            \var{\vab °sād di\lem  \mssCaCbCc\msNa\msNb\msL\Ed; °sā di° \msNc}%
            \var{\vb °kṣur bhaven na°\lem  \msCa\msCb\msNa\Ed; °kṣur bhave na° \msCc,
                                        °kṣu bhaven na° \msNb\msL, °rkṣu bhaven na \msNc}%

bindumālā\.m tataḥ paśyet tarucchāyāsamāśritām \veg\dontdisplaylinenum
            \var{\vc tataḥ paśyet\lem  \mssCaCbCc\msNa\msNb\msNc\msL; tu yaḥ paśyen \Ed}%
            \var{\vd tarucchāyā°\lem  \mssCaCbCc\msNa\msNb\msNc\msL; naracchāyā\.m \Ed\oo
                 °śritām\lem  \mssCaCbCc\msNb; °śritāḥ \msNa\msL, °śritam \msNc\Ed}%

jātyasphaṭikasa\.mkāśa\.m d\textsubring{r}ṣṭvā mucyati bandhanaiḥ\thinspace{\dandab} \dontdisplaylinenum
            \var{\va °kasa\.mkāśa\.m\lem  \mssCaCbCc\msNa\msNb\msNc\msLpcorr\Ed; °sa\.mkakāśa\.m \msLpcorr}%
            \var{\vb bandhanaiḥ\lem  \msCa\msNa\msNc; bandhavaiḥ \msCb, bandhanāt \msCc\msNb\Ed,
                                                                                va\.mcanaiḥ \msL}%

dhyāyen mano'nugā nāma pakṣmīr āpīḍya locane \veg\dontdisplaylinenum
            \var{\vd pakṣmī°\lem  \mssCaCbCc\msNa\msL; yakṣmī \msNb, yakṣmo° \msNc, pakṣī° \Ed\oo
                 locane\lem  \msCa\msCb\msNa\msL; locanaḥ \msNb, locanaiḥ \msCc\Ed, locanai \msNc}%

śvetapītāruṇa\.m bindu\.m d\textsubring{r}ṣṭvā bhūyo na jāyate\thinspace{\dandab} \dontdisplaylinenum

mano'nugādi ṣaṭ tv ete dhyānam ukta\.m mayā tava \veg\dontdisplaylinenum
            \var{\vc °ṣaṭ tv ete\lem  \msCa\msNa\msNb\msNc\msL; °ṣaṭ tv etā \msCb, °ṣaṭkena \msCc\Ed}%
            \var{\vd °kta\.m mayā tava\lem  \msCc\msNa\msNc\msL\Ed; \uncl{ka}{\lost}{\lost} tava \msCa, °kta\.m samāsataḥ \msCb,
                                                                                °kta mayā tava \msNb}%


\alalfejezet{paramāṇuḥ}
adhunānyat pravakṣyāmi paramāṇu caturvidham\thinspace{\dandab} \dontdisplaylinenum
            \var{\vb °vidham\lem  \msCa\msCc\msNa\msNb\Ed; °vidhaḥ \msCb}%

pārthivādicaturbhūta\.m yair vyāpta\.m nikhila\.m jagat \danda\dontdisplaylinenum
            \var{\vcd °bhūta\.m yair vyāpta\.m\lem  \msNa; °bhūta\.m yair vyāptin \msCa, °bhūta\.m yai vyāpta\.m \msCb\msCc\msNb, °bhūtair yair vyāpta\.m \Ed}%

lakṣaṇa\.m tasya rājendra ś\textsubring{r}ṇu vakṣyāmi sāmpratam \veg\dontdisplaylinenum

pārthivordhvagatiḥ sūkṣmaḥ paramāṇu narādhipa\thinspace{\dandab} \dontdisplaylinenum
            \var{\va pārthivordhva°\lem  \mssCaCbCc\msNa\msNb; pārthivorddha° \Ed}%
            \var{\vb paramāṇu narādhipa\lem  \msCa\msCb\msNapcorr; paramāṇu narādhipaḥ \msCc,
                                                paramāṇu narādhinarādhipa \msNaacorr, paramānu narādhipa \msNb, paramāṇur narādhipa \Ed}%

pratyakṣadarśana\.m dhyāna\.m lakṣayen niyata\.m śuciḥ \veg\dontdisplaylinenum
            \var{\vc pratyakṣadarśana\.m\lem  \mssCaCbCc\msNb\Ed; pratyakṣa\.m darśana\.m \msNa}%
            \var{\vd lakṣayen niyata\.m\lem  \msCa\msNa\msNb; lakṣayen niyataḥ \msCb, lakṣayen niyata \msCc, lakṣayan niyataḥ \Ed}%

mucyate sarvapāpebhyo rāhunā candramā yathā\thinspace{\dandab} \dontdisplaylinenum
            \var{\va sarvapāpebhyo\lem  \msCb\msCc\msNa\msNb\Ed; \uncl{sarvapāpebhyo} \msCa}%
            \var{\vb rāhunā\lem  \msCb\msCc\msNa\msNb\Ed; {\il}{\il}nā \msCa}%

tena yo 'bhyasate nitya\.m sa yogī bhuvaneśvaraḥ \veg\dontdisplaylinenum
            \var{\vc 'bhyasate\lem  \msCa\msCc\msNa\msNb\Ed; labhyate \msCb}%
            \var{\vd °śvaraḥ\lem  \mssCaCbCc\msNa\msNb; °śvara \Ed}%

adhogati mahārāja paramāṇu jalodbhavaḥ\thinspace{\dandab} \dontdisplaylinenum

abhyased yad ida\.m rājan sarvapātakanāśanam \veg\dontdisplaylinenum
            \var{\vc abhyased yad ida\.m\lem  \msCa\msCb\msNa\msNb\Ed; abhyased ida\.m \msCc}%

āgneyaparamāṇūni tiryagūrdhvagatiḥ sm\textsubring{r}tā\thinspace{\dandab} \dontdisplaylinenum
            \var{\va āgneya°\lem  \mssCaCbCc\msNa\Ed; agneya° \msNb\oo
                 °paramāṇūni\lem  \mssCaCbCc\msNa; °paramānūni \msNb, paramāṇuś ca \Ed}%
            \var{\vb tiryagūrdhva°\lem  \mssCaCbCc\msNa\msNb; tiryagūrddha° \Ed\oo
                 °gatiḥ\lem  \mssCaCbCc\msNa\Ed; °mitiḥ \msNb\oo
                 sm\textsubring{r}tā\lem  \msCa\msNa; sm\textsubring{r}tāḥ \msCb\msCc\msNb\Ed}%

ya ida\.m dhyāyate nityam uttamā\.m gatim āpnuyāt \veg\dontdisplaylinenum
            \var{\vd gatim āpnu°\lem  \msCa\msCc\msNa\msNb\Ed; phalam āpnu° \msCb}%

vāyavyaparamāṇūni adhordhvatiryag āsm\textsubring{r}tā\thinspace{\dandab} \dontdisplaylinenum
            \var{\va vāyavyaparamāṇūni\lem  \msCb; vāya{\il}{\il}ramāṇūni \msCa, vāyavya\.m paramāṇūni \msCc\msNa,
                                                        vāyavyā paramāṇūni \msNb, vāyavya\.m paramāṇuś ca \Ed}%
            \var{\vb °rdhvatirya°\lem  \mssCaCbCc\msNb\Ed; °rdhvantirya° \msNa}%

na sa muhyati ta\.m d\textsubring{r}ṣṭvā vāyusambhava bhūpate \veg\dontdisplaylinenum

catvāra ete rājendra paramāṇu nirīkṣate\thinspace{\dandab} \dontdisplaylinenum
            \var{\vb paramāṇu nirīkṣate\lem  \msCa\msCc\msNa\msNb; paramāṇur rīkṣate \msCb, paramāṇur nirīkṣate \Ed}%

tena sarvamakhair iṣṭa\.m tena tapta\.m tapas tathā \veg\dontdisplaylinenum
            \var{\vc °makhair i°\lem  \msCa\msCb\msNa\msNb\Ed; mayair i° \msCc}%
            \var{\vd tapas tathā\lem  \msCa\msCb\msNa\msNb; tapan tathā \msCc, tapta\.m tathā \Ed}%

tena dattā mahī k\textsubring{r}tsnā saptasāgarasa\.mv\textsubring{r}tā\thinspace{\dandab} \dontdisplaylinenum

sarvatīrthābhiṣekaś ca sarvavratakriyā tathā \veg\dontdisplaylinenum
            \var{\vc °bhiṣekaś ca\lem  \mssCaCbCc\msNb\Ed; °bhiṣeka \msNaacorr, °bhiṣeka\.m ca \msNapcorr}%

anenaiva vidhānena daśadhyāna\.m narādhipa\thinspace{\dandab} \dontdisplaylinenum
            \var{\va anenaiva vidhānena\lem  \msCb\msCc\msNa\msNb\Ed; a{\il}{\lost}{\lost}{\lost}dhānena \msCa}%

kurute avyavacchinna\.m sarvakāmaphalapradam \veg\dontdisplaylinenum
            \var{\vc °cchinna\.m\lem  \mssCaCbCc\msNa\Ed; °cchinna \msNb}%


\alalfejezet{daśākṣaramantraḥ}
daśākṣaramahārāja yogīndrasya mahātmanaḥ\thinspace{\dandab} \dontdisplaylinenum

kathayāmi samāsena ś\textsubring{r}ṇuṣvāvahito bhava \veg\dontdisplaylinenum

praṇavādisvarā trīṇi dīrghabindusamāyutam\thinspace{\dandab} \dontdisplaylinenum

pañca pañca cavarge tu vāyubījam adhasthitam \veg\dontdisplaylinenum

trayodaśasvarāyukta\.m pañcama parikīrtitam\thinspace{\dandab} \dontdisplaylinenum

pañcavi\.mśatimaḥ ṣaṣṭha akṣaraḥ parikīrtitaḥ \veg\dontdisplaylinenum

yād\textsubring{r}śa\.m pañcamaḥ prokta\.m saptame ca prayojayet\thinspace{\dandab} \dontdisplaylinenum

akārasvarasa\.myukta\.m sarvapātakanāśanam \veg\dontdisplaylinenum

prathama\.m pañcame varge t\textsubring{r}tīyasvarayojitam\thinspace{\dandab} \dontdisplaylinenum

uktarekārasa\.myukta\.m navama\.m parikīrtitam \veg\dontdisplaylinenum

daśamaḥ punar o\.mkāraḥ mantraśreṣṭho daśākṣaraḥ\thinspace{\dandab} \dontdisplaylinenum

japato dhyāyate vāpi pārthivādi krameṇa tu \veg\dontdisplaylinenum

mucyate so 'pi sa\.msāre sa\.mśayo nāsti bhūpate\thinspace{\dandab} \dontdisplaylinenum

ācāramūlo dharmas tu dharmamūlo janārdanaḥ \danda\dontdisplaylinenum

tena sarvajagad vyāpta\.m trailokya\.m sa carācara\.m \veg\dontdisplaylinenum


\alalfejezet{ācāravidhiḥ}
\ujvers\nemsloka 
ācārāl labhatīha āyur atulam aiśvaryavitta\.m tathā
\dontdisplaylinenum

\nemslokab 
ācārāt sutam īpsita\.m ca labhate śrīkīrtiprajñāyaśaḥ \danda\dontdisplaylinenum

\nemslokac 
ācārāl labhate ca lakṣmim atula\.m khyāti\.m tathaivottamam
\dontdisplaylinenum

\nemslokad 
ācārād iha mantradharmaparama\.m prāpnoti niḥsa\.mśayam \veg\dontdisplaylinenum

\vers

janamejaya uvāca~{\dandab}\dontdisplaylinenum 

\nemsloka 
ācārāt prabhavānusaṅgakathita\.m suśliṣṭadharmākaram
\dontdisplaylinenum

\nemslokab 
ācārāt katidhāṅga kīrtaya punas t\textsubring{r}ptir na me jāyate \danda\dontdisplaylinenum

\nemslokac 
sarvajñaḥ tvam aha\.m ś\textsubring{r}ṇomi varada\.m kiñcin na me śāśvaram
\dontdisplaylinenum

\nemslokad 
tan me kīrtaya dharmasāraśubhadam ācāramūlāśrayam \veg\dontdisplaylinenum

\vers

vaiśampāyana uvāca~{\dandab}\dontdisplaylinenum 

\nemsloka 
nitya\.m namraśirodvijātiguruṣu śuśrūṣaṇa\.m daivatam
\dontdisplaylinenum

\nemslokab 
tiṣṭhenācamanena cāśanakara\.m vāmāsthimānādaram \danda\dontdisplaylinenum

\nemslokac 
sūryāgniśaśibandhur āryapurataḥ kuryān na cāvaśyakam
\dontdisplaylinenum

\nemslokad 
śasye bhasmani govrajedvijajala\.m kuryān na cārka\.m naraḥ \veg\dontdisplaylinenum

\ujvers\nemsloka 
pādenāgnijala\.m sp\textsubring{r}śen na ca guru\.m pādena pāda\.m tathā
\dontdisplaylinenum

\nemslokab 
śauca\.m kārya jalādinā ca niyata\.m nādho jala\.m kārayet \danda\dontdisplaylinenum

\nemslokac 
kuryān nityabhivādana\.m dvijaguror mātāpit\textsubring{r}r daivatam
\dontdisplaylinenum

\nemslokad 
etācāravidhiḥ samāsaniyamas tubhya\.m mayā kīrtitam \veg\dontdisplaylinenum


\alalfejezet{striyaḥ}
\vers

janamejaya uvāca~{\dandab}\dontdisplaylinenum 

\nemsloka 
strīṇā\.m ki\.m priyam asti tad vada vibho sa\.msārasārastriyām
\dontdisplaylinenum

\nemslokab 
ki\.m sadbhāva na vedmi tasya viṣaye ki\.m dveṣya ki\.m tātpriyam \danda\dontdisplaylinenum

\nemslokac 
paśyāmi na ca tasya garbhakalayā prāpnoti niḥsa\.mśayam
\dontdisplaylinenum

\nemslokad 
māyājālasahasragāpi yuvatī kurvanti bhartā priyam \veg\dontdisplaylinenum

\vers

vaiśampāyana uvāca~{\dandab}\dontdisplaylinenum 

\nemsloka 
rājan ki\.m priyam asti arthaparataḥ paśyāmi nānyan n\textsubring{r}pe
\dontdisplaylinenum

\nemslokab 
putrārthaikaprayojana\.m yuvatayaḥ svāyambhuvoktāmaraiḥ \danda\dontdisplaylinenum

\nemslokac 
kāntā nityakalā pravartanakarī dharmasakhāyā satī
\dontdisplaylinenum

\nemslokad 
māyā vāpi karoti sadya manujātyaktānya vā sevate \veg\dontdisplaylinenum

\ujvers\nemsloka 
strīsaṅga\.m parivarjayen narapate āyāsada\.m duḥkhadam
\dontdisplaylinenum

\nemslokab 
m\textsubring{r}tyudvārabhayākara\.m viṣag\textsubring{r}ham āpat sughorālayam \danda\dontdisplaylinenum

\nemslokac 
agnir mārutamattavāraṇasama tasyānugāmī sadā
\dontdisplaylinenum

\nemslokab 
strīhetor hatarāvaṇas tridaśapa indro 'pi visthāpitaḥ \danda\dontdisplaylinenum

\nemslokab 
strīhetor api candramāstribhuvane dhiktā\.m gataś cāmaro \danda\dontdisplaylinenum

\nemslokad 
daṇḍakṣo hatarāṣṭrapaurasahitaḥ ki\.m bhūya vakṣyāmy aham \veg\dontdisplaylinenum


\alalfejezet{vipramunibhikṣunirgranthiparivrājakarṣyādayaḥ}
\vers

janamejaya uvāca~{\dandab}\dontdisplaylinenum 

\nemsloka 
vipre kīd\textsubring{r}śalakṣaṇa\.m bhavati bho kīd\textsubring{r}g muniś cocyate
\dontdisplaylinenum

\nemslokab 
tenārthena bhaveta bhikṣu bhagavan nigranthi ko vā dvija \danda\dontdisplaylinenum

\nemslokac 
kenārthena bhaved dvijendra bhagavan jñeyaḥ parivrājakaḥ
\dontdisplaylinenum

\nemslokad 
! jñeyāḥ kim \textsubring{r}ṣayaś ca lakṣaṇa muner icchāmi jñātu\.m punaḥ \veg\dontdisplaylinenum

\vers

vaiśampāyana uvāca~{\dandab}\dontdisplaylinenum 

\nemsloka 
satya\.m śaucam ahi\.msatā damaśamau bhūtānukampī sadā
\dontdisplaylinenum

\nemslokab 
ātmārāmajito svadharmanirataḥ sattvastha nitya\.m manaḥ \danda\dontdisplaylinenum

\nemslokac 
kāmakrodhayamasvadāranirataḥ sa\.mtyajya lobhaḥ śanaiḥ
\dontdisplaylinenum

\nemslokad 
eva\.m yaḥ kurute dvijātisuvaraḥ śūdro 'pi yaḥ sa\.myamī \veg\dontdisplaylinenum

\ujvers\nemsloka 
tasmāc chadmakavarjitaḥ sa bhagavān sa\.msārabhībhidyakaḥ
\dontdisplaylinenum

\nemslokab 
yat tat sthānapara\.m vrajanti puruṣāḥ tasmāt parivrājakaḥ \danda\dontdisplaylinenum

\nemslokac 
granthidārasuta\.m dhana\.mś ca virati nirgranthika socyate
\dontdisplaylinenum

\nemslokad 
ramyante \textsubring{r}ṣir āśrame dh\textsubring{r}timanas tasmād \textsubring{r}ṣiḥ socyate \veg\dontdisplaylinenum

\ujvers\nemsloka 
kāyavāṅmanadaṇḍatatparataras te daṇḍikarūcyate
\dontdisplaylinenum

\nemslokab 
saddharmaśravaṇa\.m vadanti śravaṇaḥ saddharmabrahmākṣaraḥ \danda\dontdisplaylinenum

\nemslokac 
pāśaprakṣipata\.m paśutvasakala\.m pāśūpatās te sm\textsubring{r}tāḥ
\dontdisplaylinenum

\nemslokad 
vipre pāśupatādibhikṣusakala\.m p\textsubring{r}ṣṭo 'smy aha\.m lakṣaṇam \veg\dontdisplaylinenum

\ujvers\nemsloka 
sarva\.m tat kathito 'si lakṣaṇa mayā sandhiśvanirnāśanam
\dontdisplaylinenum

\nemslokab 
prajñāsa\.mgrahaśītavardhanapara\.m sa\.msāranirmūlanam \danda\dontdisplaylinenum

\nemslokac 
 
\dontdisplaylinenum

\nemslokad 
etaj jñānapara\.m prabodham atula\.m nitya\.m śiva\.m dhāryate \veg\dontdisplaylinenum

\vers


\jump
\begin{center}
\ketdanda iti v\textsubring{r}ṣasārasa\.mgrahe dvāvi\.mśatitamo 'dhyāyaḥ\ketdanda
\end{center}
\dontdisplaylinenum\vers 
\bekveg\szamveg\vfill\phpspagebreak\szam\bek\versno=0\fejno=23
\thispagestyle{empty}



\alfejezet{\textbf{23 nidrotpattiḥ}}\jump\jump
\vers

janamejaya uvāca~{\dandab}\dontdisplaylinenum 

devānā\.m dānavānā\.m ca uttarāraṇim eva ca\thinspace{\danda} \dontdisplaylinenum
            \var{\vab dānavānā\.m ca uttarāraṇim eva\lem  \msNa\Ed;
                               dā{\lost}{\lost}{\lost}{\lost}{\lost}{\lost}{\lost}ṇim eva \msCa}%

vidviṣanti ca te 'nyonya\.m kāraṇa\.m tasya kīrtaya \veg\dontdisplaylinenum
            \var{\vd tasya\lem  \msCa\Ed; ta\uncl{sya} \msNa}%

vaiśampāyana uvāca~{\dandab}\dontdisplaylinenum 

pāpapuṇyasvabhāvābhyā\.m devadaityasya bhūpate\thinspace{\danda} \dontdisplaylinenum

dharmapakṣasm\textsubring{r}to devo dānavo 'dharmapakṣataḥ \veg\dontdisplaylinenum
            \var{\vc dharmapakṣa°\lem  \msNa; dharme pakṣaḥ \msCa, 
                                         dharmapakṣaḥ \Ed\oo
                 devo\lem  \msCa\msNa; devā \Ed}%
            \var{\vd 'dharma°\lem  \Ed; darppa° \msCa, darpa° \msNa}%

hetunā tena rājendra anyonya\.m vidviṣanti te\thinspace{\dandab} \dontdisplaylinenum

devadveṣṭāsurāḥ sarve vibudhāś cāsuradviṣaḥ \veg\dontdisplaylinenum
            \var{\vc devadveṣṭāsurāḥ sarve\lem  \eme;
                      devadveṣṭāsuraḥ sarve \msNa\Ed,
                       \uncl{de}vadve\uncl{ṣṭā}suras {\lost}{\lost} \msCa}%
            \var{\vd vibudhāś\lem  \msNa\Ed; {\lost}{\lost}dhāś \msCa}%


\alalfejezet{dharmādharmavipakṣatā}
\ujvers\nemsloka 
dharmādharmavipakṣatā\.m ś\textsubring{r}ṇu parā\.m bhūtānukampodayām
\dontdisplaylinenum
            \var{\va °vipakṣatā\.m\lem  \Ed; °vivakṣatā\.m \msCa\msNa\oo
                 °kampodayām\lem  \msCa\msNa; °kampādayām \Ed}%

\nemslokab 
satya\.m śaucam ahi\.msatā damaśamo nirmānam īrṣyāruṣā \danda\dontdisplaylinenum
            \var{\vb īrṣā°\lem  \msCa\msNa; īrṣyā° \Ed}%

\nemslokac 
t\textsubring{r}ṣṇālobharatasya kāmaviṣayaḥ sarvendriyāṇā\.m jayaḥ
\dontdisplaylinenum

\nemslokad 
ādhyātmeṣu ratiḥ prasannamanaso nirdvandvasarvālayaḥ \veg\dontdisplaylinenum
            \var{\vd prasannamanaso nirdvandva°\lem  \msNa\Ed; 
                        prasanna{\lost}{\lost}{\lost}{\lost}{\lost} \msCa}%

\ujvers\nemsloka 
pāpopekṣaṇaśaśvapuṇyamudito dīneṣu kāruṇyatā
\dontdisplaylinenum
            \var{\va pāpo°\lem  \msCa\msNa; pāpā° \Ed\oo
                °śaśva°\lem  \msCa\msNa; °śaśca° \Ed}%

\nemslokab 
dāna\.m śīladh\textsubring{r}tikṣamājapatapaḥ svādhyāyamaune ratiḥ \danda\dontdisplaylinenum

\nemslokac 
yogābhyāsaratir divaukasagaṇe jñāne ca sā\.mkhye tathā
\dontdisplaylinenum
            \var{\vc yogābhyāsaratir divaukasa°\lem  \msCa; 
                       yogābhyāsaratidivaukasa° \msNa\ \unmetr, 
                 yogabhyāsaratidivaikasa° \Ed\ \unmetr}%

\nemslokad 
akrodhārjavatejayajñam abhaya\.m sa\.mtoṣa adrohatā \veg\dontdisplaylinenum
            \var{\vd °bhaya\.m\lem  \Ed; °bhayas \msCa, °bhayaḥ \msNa}%

\ujvers\nemsloka 
tyāgo mārdavahrīr acāpalaratir nyāsābhimāno grahāt
\dontdisplaylinenum
            \var{\va °hrīr acāpalaratinyāsā°\lem  \msNa\Ed;
                        \uncl{hrī}{\lost}{\lost}{\lost}{\lost}ratir nyāsā° \msCa}%

\nemslokab 
maitrībhāvasadārapaiśunamatir brāhmaṇyaśraddhānvitaḥ \danda\dontdisplaylinenum
            \var{\vb °nvitaḥ\lem  \msNa; °nvitā \msCa\Ed}%

\nemslokac 
etācāra sadā narendra vibudhāḥ sa\.mkṣepataḥ kīrtitāḥ
\dontdisplaylinenum
            \var{\vc kīrtitāḥ\lem  \msCa\msNa; kīrtitaḥ \Ed}%

\nemslokad 
daityānā\.m ś\textsubring{r}ṇu kīrtaye svavahito 'sambhāvya teṣā\.m nijam \veg\dontdisplaylinenum
            \var{\vd daityānā\.m\lem  \msNa\Ed; daityānā \msCa\oo
                 kīrtaye\lem  \msCa\Ed; kīrtaya \msNa\oo
                 svavahito\lem  \msCa; svavahisa\.m \msNa,
                                         tv avahito \Ed\oo
                 nijam\lem  \msCa\Ed; nijaḥ \msNa}%

\ujvers\nemsloka 
daityāḥ pāparatisvabhāvacapalā nirlajjadarpālasāḥ
\dontdisplaylinenum
            \var{\va daityāḥ\lem  \msCa; daityā \msNa\Ed}%

\nemslokab 
kāmakrodhavaśāḥ suduṣṭamanasas t\textsubring{r}ṣṇādhikā nirdayāḥ \danda\dontdisplaylinenum
            \var{\vb kāmakrodhavaśāḥ\lem  \msNa\Ed; {\il}{\lost}{\lost}{\lost}{\lost}śās \msCa}%

\nemslokac 
śaucācāravivarjitā gurugirānnānitya kuryuḥ kriyāḥ
\dontdisplaylinenum

\nemslokad 
jīvākarṣaṇajīvanaḥ pratidina\.m mohāndharāgānvitāḥ \veg\dontdisplaylinenum
            \var{\vd jīvākarṣaṇa°\lem  \msCa\msNa; naivākarṣaṇa° \Ed}%

\ujvers\nemsloka 
nidrā nitya divā prasaktam aśuciḥ sūryodaye supyate
\dontdisplaylinenum

\nemslokab 
āśāpāśaśatair nibaddhah\textsubring{r}dayo h\textsubring{r}tvā parasva\.m punaḥ \danda\dontdisplaylinenum
            \var{\vb h\textsubring{r}tvā parasva\.m punaḥ\lem  \msNa\Ed; 
                                \uncl{h\textsubring{r}}{\lost}{\lost}{\lost}{\lost}{\lost}naḥ \msCa}%

\nemslokac 
mātsaryāt parapākabhedanirato mūlasya duṣpūratā
\dontdisplaylinenum
            \var{\vc mātsaryā\lem  \msCa\msNa; mā\.msaryā° \Ed}%

\nemslokad 
! nāstīkatvaparāṅganāsvabhirata utkocakāmaḥ sadā \veg\dontdisplaylinenum
            \var{\vd °parāṅganāsvabhirata\lem  \msCa;
                                °parāṅganās tv abhirata \msNa,
                                °parāṅganāpy abhirato \Ed\oo
                 utkoca°\lem  \msCa\msNa; uktā ca \Ed}%

\ujvers\nemsloka 
devabrāhmaṇa vidviṣanti satata\.m lobhāc ca kāryakriyā
\dontdisplaylinenum

\nemslokab 
dharma\.m dūṣayate ca mūḍhamanasā ārya\.m ca tīrtha\.m tathā \danda\dontdisplaylinenum

\nemslokac 
hantavyāny ahatāś ca manyabahavo visphūrjitam adruvan
\dontdisplaylinenum
            \var{\vc °hatāś\lem  \msCa; °hatā\.mś \msNa, °hatā\.m \Ed\oo
                 manya°\lem  \msCa\msNa; yanya \Ed\oo
                 visphūrjitam adruvan\lem  \msNa;
                                        visphurjjite nakravat \Ed, 
                                        vi{\lost}{\lost}{\lost}{\lost}druvan \msCa}%

\nemslokad 
daityānā\.m kathita\.m ca cihna katicit sadbhāvataḥ kīrtitam \veg\dontdisplaylinenum
            \var{\vd kathita\.m\lem  \msCa\msNa; kathitaś \Ed}%

\ujvers\nemsloka 
martyeṣv eva narendra mānuṣam abhūd devāsurāṇā\.m n\textsubring{r}paḥ
\dontdisplaylinenum

\nemslokab 
yo ya\.m proktaḥ svabhāvatām ubhayato mānuṣyaloke tathā \danda\dontdisplaylinenum
            \var{\vb °loke\lem  \msCa\msNa; °lokan \Ed}%

\nemslokac 
yan me p\textsubring{r}cchitavān narendra kathita\.m yat tva\.m purā gopitam
\dontdisplaylinenum
            \var{\vc p\textsubring{r}cchitavān\lem  \msNa\Ed; p\textsubring{r}cchitavā \msCa}%

\nemslokad 
vidveṣobhayakāraṇa\.m narapate ki\.m bhūya vakṣyāmy aham \veg\dontdisplaylinenum
            \var{\vd vidveṣobhayakāraṇa\.m narapate ki\.m\lem  \msNa\Ed; 
                                vi\uncl{dveṣobhaya}{\lost}{\lost}{\lost}{\lost}{\lost}pate ki \msCa}%


\alalfejezet{nidrottpattiḥ}
\vers

janamejaya uvāca~{\dandab}\dontdisplaylinenum 

asti kautūhala\.m cānya\.m p\textsubring{r}cchāmi tvā\.m dvijottama\thinspace{\danda} \dontdisplaylinenum
            \var{\va kautūhala\.m\lem  \msCa\msNa; kautuhala\.mś \Ed}%

katha\.m nidrā samutpannā sarvabhūtavimohanī \veg\dontdisplaylinenum
            \var{\vd °mohanī\lem  \msCapcorr\msNa\Ed; °mohinī \msCaacorr}%

rātrau prajāyate kasmād divā kasmān na jāyate\thinspace{\dandab} \dontdisplaylinenum

kasmāc ca kurute jantor nidrā netrapramīlanam \danda\dontdisplaylinenum
            \var{\vc jantor\lem  \msCa\msNa; janto \Ed}%

etan me sa\.mśaya\.m chindhi sarvajño 'si dvijottama \veg\dontdisplaylinenum
            \var{\vf sarvajño 'si\lem  \msNa\Ed; {\lost}{\lost}{\lost}{\lost} \msCa}%

vaiśampāyana uvāca~{\dandab}\dontdisplaylinenum 

devī hy eṣā mahābhāgā nidrā netrāśrayā n\textsubring{r}ṇām\thinspace{\danda} \dontdisplaylinenum
            \var{\vb °śrayā\lem  \msCa\msNa; °śrayo \Ed}%

tasyā vaśa\.m gata\.m sarva\.m jagatsthāvarajaṅgamam \veg\dontdisplaylinenum

sadevadānavagaṇā gandharvoragarākṣasāḥ\thinspace{\dandab} \dontdisplaylinenum
            \var{\va °dānava°\lem  \msCa\Ed; °dānavā° \msNa}%

yakṣabhūtapiśācāś ca paśupakṣisarīs\textsubring{r}pāḥ \veg\dontdisplaylinenum
            \var{\vd °sarīs\textsubring{r}pāḥ\lem  \msCa\msNa; °śarīs\textsubring{r}paḥ \Ed}%

guhyakāś ca m\textsubring{r}gā nāgā ki\.mnarā jalajoragāḥ\thinspace{\dandab} \dontdisplaylinenum
            \var{\va guhyakāś ca\lem  \eme;
                                 guhyakaś ca \Ed,
                                 guhyavastra° \msCa\msNa\oo
                 nāgāḥ\lem  \msCa\msNa; nāgā \Ed}%
            \var{\vb ki\.mnarā jalajoragāḥ\lem  \eme;
                        ki\.mnarā jalajā nagāḥ \msNa\Ed;
                        kinna{\lost}{\lost}{\lost}{\lost}{\lost}gāḥ \msCa}%

nidrāvaśagatāḥ sarve pāpmanā tv abhilaṅghitāḥ \veg\dontdisplaylinenum

devadānavakarmānte tasminn am\textsubring{r}tasambhave\thinspace{\dandab} \dontdisplaylinenum
            \var{\va °karmānte\lem  \msCa\msNa; °karmāt te \Ed}%
            \var{\vb °m\textsubring{r}ta°\lem  \msCa\msNa; °n\textsubring{r}ta° \Ed}%

mandarotthāpane viṣṇur devāsurasamāgame \veg\dontdisplaylinenum
            \var{\vc °tthāpane\lem  \Ed; °tpādane \msCa\msNa}%

jāyate vigrahe tv eṣā k\textsubring{r}te hy am\textsubring{r}tamanthane\thinspace{\dandab} \dontdisplaylinenum

rajas tamaś cāsura\.m vai sattva\.m devak\textsubring{r}taiḥ śubhaiḥ \veg\dontdisplaylinenum

tataḥ sattvamayī devī rajastamanivāsinī\thinspace{\dandab} \dontdisplaylinenum
            \var{\vab sattvamayī devī rajas tamasi vāsinī\lem  \msNa;
                       sattvamayī \uncl{de}{\lost}{\lost}{\lost}{\lost}masi vāsinī \msCa,
                       sattvamayī devī rajas tamanivāsinī \Ed}%

krodhajā vai sthitā madhye devadānavapakṣayoḥ \veg\dontdisplaylinenum

tām adbhutamayī\.m d\textsubring{r}ṣṭvā vismitā devadānavāḥ\thinspace{\dandab} \dontdisplaylinenum
            \var{\va °bhuta°\lem  \msCa\msNa; °bhūta° \Ed}%

tasyāḥ prabhāvābhihatā dudruvas te diśo daśa \veg\dontdisplaylinenum
            \var{\vd daśa\lem  \msCa\msNa; daśaḥ \Ed}%

tatra pītāmbaradharo viṣṇur ekas tu tiṣṭhati\thinspace{\dandab} \dontdisplaylinenum
            \var{\va pītā°\lem  \msCa\msNa; pitā° \Ed}%

sābhigatvā viśālākṣī nārāyaṇam athābravīt \veg\dontdisplaylinenum
            \var{\vc sābhi°\lem  \msCa;\msNa sobhi° \Ed}%
            \var{\vd °bravīt\lem  \msNa\Ed; °\uncl{bra}{\lost} \msCa}%

devadānavanāthas tva\.m tvayi sarva\.m pratiṣṭhitam\thinspace{\dandab} \dontdisplaylinenum
            \var{\va deva°\lem  \msNa\Ed; {\lost}{\lost} \msCa}%
            \var{\vb sarva\.m\lem  \msCa\msNa; sarva° \Ed}%

dehi deva mamāvāsa\.m yatrāha\.m nivase sukham \veg\dontdisplaylinenum

tato nārāyaṇas tuṣṭas tā\.m devī\.m pratyabhāṣata\thinspace{\dandab} \dontdisplaylinenum

śarīre mama vastavya\.m viṣṇur enām athābravīt \veg\dontdisplaylinenum
            \var{\vc vastavya\.m\lem  \Ed; vāstavyam \msCa\msNa}%

tatas tā\.m vaiṣṇava\.m tejaḥ pāpmanā samatiṣṭhata\thinspace{\dandab} \dontdisplaylinenum
            \var{\va vaiṣṇava\.m\lem  \msCa\msNa; viṣṇuvat \Ed}%

tataḥ śete sa vaikuṇṭhaḥ pāpmanā tv abhilaṅghitaḥ \veg\dontdisplaylinenum
            \var{\vd pāpmanā tv abhilaṅghitaḥ\lem  \msNa;
                                pāpmanā tv abhilaṅghitāḥ \Ed,
                                pāpma{\lost}{\lost}{\lost}{\lost}ghitaḥ \msCa}%

tasmin śayāne vitrastā devāsuragaṇās tathā\thinspace{\dandab} \dontdisplaylinenum
            \var{\va tasmin\lem  \msCa\Ed; tasmi \msNa}%

ūcus te paramodvignāḥ śayāna\.m viṣṇum acyutam \veg\dontdisplaylinenum

trātāra\.m nābhigacchāma uttiṣṭhottiṣṭha keśava\thinspace{\dandab} \dontdisplaylinenum

tataḥ śaṅkhagadāpāṇir uttiṣṭhata mahābhujaḥ \veg\dontdisplaylinenum

utthitaś ca viśālākṣaḥ pāpmanā tasya p\textsubring{r}ṣṭhataḥ\thinspace{\dandab} \dontdisplaylinenum
            \var{\va utthita°\lem  \msCa\msNa; uttiṣṭha° \Ed\oo
                 viśālākṣaḥ\lem  \msCa\msNa; viśālākṣiḥ \Ed}%

tataḥ sā vigrahavatī sthitā nārāyaṇālaye \veg\dontdisplaylinenum
            \var{\vc tataḥ sā vigrahavatī\lem  \msNa\Ed;
                        tata{\lost}{\lost}{\lost}{\lost}\uncl{va}tī \msCa}%

viṣṇur devāsuragaṇān ida\.m vacanam abravīt\thinspace{\dandab} \dontdisplaylinenum
            \var{\va viṣṇur\lem  \msCa\Ed; viṣṇu \msNa\oo
                °gaṇān\lem  \msCa\Ed; °gaṇā \msNa}%

asmāka\.m vai śarīreṣu iya\.m pāpmā viniḥs\textsubring{r}tā \veg\dontdisplaylinenum
            \var{\vd viniḥs\textsubring{r}tā\lem  \eme; vinis\textsubring{r}tā \msCa\msNa\Ed\ \unmetr}%

eṣābhisattvārasatā satyena bhaginī mama\thinspace{\dandab} \dontdisplaylinenum
            \var{\va eṣābhisattvārasatā\lem  \msCa;
                        eṣātisatvānasatā \msNa,
                        eṣātisattvāmasatī \Ed}%

viśrutā\.m triṣu lokeṣu tā\.m pūjayatha mā\.m yathā \veg\dontdisplaylinenum
            \var{\vc °śrutā\.m\lem  \msCa; °śrutā \msNa, °śruto \Ed}%

tato devāsuragaṇāḥ saptalokāḥ samānuṣāḥ\thinspace{\dandab} \dontdisplaylinenum
            \var{\vb °lokāḥ samānuṣāḥ\lem  \msNa\Ed; °\uncl{lo}{\lost}{\lost}{\lost}nuṣāḥ \msCa}%

vibhaktā vaiṣṇavī pāpmā teṣu sarveṣu devatā \veg\dontdisplaylinenum

parvateṣv atha v\textsubring{r}kṣeṣu sāgareṣu saritsu ca\thinspace{\dandab} \dontdisplaylinenum

tato nidrāvaśagata\.m jagat sthāvarajaṅgamam \veg\dontdisplaylinenum

eṣotpattiś ca nidrāyā yathā vasati tac ch\textsubring{r}ṇu\thinspace{\dandab} \dontdisplaylinenum

trīṇi sthānāni yasyā vai śarīreṣu śarīriṇām \veg\dontdisplaylinenum

śleṣmapittānilasthāne trīṇi pakṣāṇi vāsinaḥ\thinspace{\dandab} \dontdisplaylinenum
            \var{\vab °nilasthāne trīṇi\lem  \Ed; °nilasthāna trīṇi \msNa, 
                                        ni{\lost}{\lost}{\lost}{\lost}ṇi \msCa}%
            \var{\vb pakṣāṇi\lem  \msCa; pakṣā ni° \msNa\Ed}%

tamaḥ śleṣmāśrayā nidrā rajonidrā tu vātikā \veg\dontdisplaylinenum
            \var{\vc tamaḥ\lem  \msCa\msNa; tama° \Ed}%
            \var{\vd nidrā tu\lem  \msCa\msNa; nidrāti° \Ed}%

pittāśrayā\.m sm\textsubring{r}tā\.m nidrā\.m sāttvikā\.m viddhi bhūpate\thinspace{\dandab} \dontdisplaylinenum
            \var{\va sm\textsubring{r}tā\.m\lem  \msCa\Ed; sm\textsubring{r}tā \msNa}%

ādityaprabhava\.m tejas tasmin sattva\.m pratiṣṭhati \veg\dontdisplaylinenum
            \var{\vd sattva\.m pratiṣṭhati\lem  \msCa\msNa; sarva pratiṣṭhita\.m \Ed}%

nidrā divā na bhavati tasmāt sattvaguṇātmikā\thinspace{\dandab} \dontdisplaylinenum

yasmāt somodbhavā nidrā tamā\.msi ca rajā\.msi ca \veg\dontdisplaylinenum
            \var{\vc yasmā°\lem  \msCa\msNa; tasmā° \Ed}%
            \var{\vd tamā\.msi ca rajā\.msi ca\lem  \msNa\Ed;
                        ta\uncl{mā\.m}si ca \uncl{ra}{\lost}{\lost}{\lost} \msCa}%

tasmād rātrau bhaven nidrā tāmasī harajātmikā\thinspace{\dandab} \dontdisplaylinenum
            \var{\va bhaven\lem  \msCa\Ed; bhavan \msNa}%

yadā hi sarvāṅgagatau śrotā\.msi pratipadyate \veg\dontdisplaylinenum
            \var{\vc sarvā°\lem  \msNa\Ed; satvā° \msCa}%

rajas tamaś ca niyatas tadā nidrā pravartate\thinspace{\dandab} \dontdisplaylinenum
            \var{\va niyatas\lem  \msCa\msNa; niyata\.ms \Ed}%

tamasy ūrdhvagataśroto hy akṣipakṣmāsamāśritā \veg\dontdisplaylinenum
            \var{\vc °gataśroto\lem  \msCa\msNa; °gate śrotro \Ed}%
            \var{\vd hy akṣi°\lem  \msCa\msNa; hy ākṣi° \Ed}%

tamaḥ pravartate jantos tatas tv akṣnor nimīlanam\thinspace{\dandab} \dontdisplaylinenum
            \var{\vab jantos tata°\lem  \msCa\msNa; janto tama° \Ed}%
            \var{\vb tv akṣṇor ni°\lem  \msCa; tv akṣṇo ni° \msNa\Ed}%

nāsākṣikarṇaśrotā\.msi prayujyante kaphena tu \veg\dontdisplaylinenum
            \var{\vc °śrotā\.msi\lem  \msNa\Ed; śro\uncl{tā}{\lost} \msCa}%
            \var{\vd prayujyante kaphena\lem  \msNa\Ed; {\lost}{\lost}{\lost}{\lost}{\lost}phena \msCa}%

h\textsubring{r}daya\.m muhyate cāpi tamasā cāv\textsubring{r}ta\.m manaḥ\thinspace{\dandab} \dontdisplaylinenum

sparśa\.m na vedayaty eva na ś\textsubring{r}ṇoti na paśyati \veg\dontdisplaylinenum

nocchvāsayati nāsābhyā\.m viv\textsubring{r}tākṣimukho naraḥ\thinspace{\dandab} \dontdisplaylinenum
            \var{\vb °mukho naraḥ\lem  \msCa\msNa; °mukhena ca \Ed}%

eṣā n\textsubring{r}ṇām antakarī nidrā vai tāmasī sm\textsubring{r}tā \veg\dontdisplaylinenum
            \var{\vc °ntakarī\lem  \msNa\Ed; °nakarī \msCa}%

akarmaṇy aprav\textsubring{r}ttiś ca m\textsubring{r}tavat svapate kṣitau\thinspace{\dandab} \dontdisplaylinenum

nidrotpatti\.m vikāra\.m ca kathito 'smi narādhipa \danda\dontdisplaylinenum
            \var{\va °tpatti\.m vikāra\.m ca\lem  \msNa;
                      °tpatti\.m vikāraś ca \Ed, 
                      °tpa\uncl{tti}{\lost}{\lost}{\lost}{\lost} \msCa}%

tasmān nidrā\.m na seveta tamomohapravardhanīm \veg\dontdisplaylinenum
            \var{\vd °vardhanīm\lem  \msCa; °vardhanī \msNa\Ed}%


\jump
\begin{center}
\ketdanda iti v\textsubring{r}ṣasārasa\.mgrahe nidrotpattis trayovi\.mśatimo 'dhyāyaḥ\ketdanda
\end{center}
\dontdisplaylinenum\vers 
            \var{{\normalfont Colophon: }  °vi\.mśatimo\lem  \msCa\msNa; °vi\.mśatitamo \Ed}%
\bekveg\szamveg\vfill\phpspagebreak\szam\bek\versno=0\fejno=24
\thispagestyle{empty}



\alfejezet{\textbf{24 śāstravarṇanam}}\jump\jump
\vers

janamejaya uvāca~{\dandab}\dontdisplaylinenum 

devānā\.m dānavānā\.m ca vaiṣamyāni śrutāni me\thinspace{\danda} \dontdisplaylinenum
            \var{\vb vaiṣamyāni\lem  \eme;  vaiśamyāni \msCa\msCb\Ed\oo
                 me\lem  \msCa\msCb; vai \Ed}%

nidrāsambhavam āścarya\.m tvatprasādena veditam \veg\dontdisplaylinenum
            \var{\vd tvatprasādena veditam\lem  \msCb\Ed; tvatpra{\lost}{\lost}{\lost}{\lost}{\lost}{\lost}tam \msCa}%

trailokyavistarāyāma\.m śrotum icchāmi bho dvija\thinspace{\dandab} \dontdisplaylinenum
            \var{\va °lokya°\lem  \msCa\msCb; °lokyā° \Ed}%
            \var{\vb bho\lem  \msCa\msCb; vai \Ed}%

kasmi\.mścin naraka\.m jñeya\.m pātāla\.m ca dvijottama \veg\dontdisplaylinenum
            \var{\vc kasmi\.mścin naraka\.m\lem  \eme; kasmi\.mścin narake \msCa\msCb, kasmiścin naraka\.m \Ed}%

saptadvīpa\.m samicchāmi saptasāgaram eva ca\thinspace{\dandab} \dontdisplaylinenum

merumūrdha\.m ca viprendra devālaya\.m nibodha mām \veg\dontdisplaylinenum
            \var{\vc °mūrdha\.m\lem  \msCa\msCb; °mūrdhaś \Ed}%
            \var{\vd devālaya\.m\lem  \corr; devālaya \msCa\msCb\Ed}%


\alalfejezet{trailokya\.m narakāṇi ca}
vaiśampāyana uvāca~{\dandab}\dontdisplaylinenum 

ś\textsubring{r}ṇu sa\.mkṣepato rājan trailokyāyāmavistaram\thinspace{\danda} \dontdisplaylinenum
            \var{\vb °vistaram\lem  \msCb\Ed; {\lost}{\lost}{\lost} \msCa}%

kālāgniḥ prathamo jñeyaḥ sarvādhastān nareśvara \veg\dontdisplaylinenum

tasyopari n\textsubring{r}paśreṣṭha jñeyā narakakoṭayaḥ\thinspace{\dandab} \dontdisplaylinenum
            \var{\va n\textsubring{r}pa°\lem  \msCa\Ed; n\textsubring{r}° \msCb}%

rauravādi avīcyanta\.m yātanāsthānam ucyate \veg\dontdisplaylinenum


\alalfejezet{sapta pātālāḥ}
upariṣṭāt tu vijñeyāḥ pātālāḥ sapta eva tu\thinspace{\dandab} \dontdisplaylinenum
            \paral{\textit{{\normalfont Niśv Kārikā 149:} upariṣṭāt tu deveśi pātālās sapta eva tu}}

ābhāsatālaḥ prathamaḥ svatālaś ca tataḥ param \veg\dontdisplaylinenum
            \var{\vd svatālaś ca\lem  \Ed; svalālañ ca \msCaacorr, svatālañ ca \msCapcorr, 
                                        sutālañ ca \msCb}%

śītalaś ca gabhastiś ca śarkaraś ca śilātalam\thinspace{\dandab} \dontdisplaylinenum
            \var{\va śītalaś ca\lem  \msCa\Ed; śrītalaś ca \msCb}%
            \var{\vb śarkaraś ca śilātalam\lem  \eme; {\lost}{\lost}{\lost}{\lost}{\lost}lātalam \msCa,
                                         śilātalam \msCb,
                                         śarkaraś ca śilāv\textsubring{r}tam \Ed}%

saptama\.m tu mahātāla\.m śeṣanāgak\textsubring{r}tālayaḥ \veg\dontdisplaylinenum
            \var{\vc saptama\.m\lem  \msCa\msCb; saptamas \Ed}%
            \var{\vd °layaḥ\lem  \msCa\Ed; °layam \msCb}%

baliś ca daityarājendro rākṣasaś ca viśa\.mkhaṇaḥ\thinspace{\dandab} \dontdisplaylinenum
            \var{\vb viśa\.mkhaṇaḥ\lem  \Ed; visa\.mśanaḥ \msCa, visa\.mśayaḥ \msCb}%

ity evam ādayaḥ sarve nāgadānavarākṣasāḥ \veg\dontdisplaylinenum


\alalfejezet{sapta dvīpāḥ priyavratasutāś ca}
sapta dvīpās tato jñeyāḥ saptasāgarasa\.mv\textsubring{r}tāḥ\thinspace{\dandab} \dontdisplaylinenum

priyavratasya putro 'bhūd daśa rājaparākramaḥ \veg\dontdisplaylinenum 
            \var{\vo \om\ \Ed}%
            \paral{\textit{{\normalfont For a similar enumeration of Priyavrata's ten sons and the seven islands,
                see, e.g., Vāyupurāṇa 33.1 ff.}}}

agnīdhraś cāgnibāhuś ca medhā medhātithir vasuḥ\thinspace{\dandab} \dontdisplaylinenum
            \var{\vab agnīdhraś cāgnibāhuś ca medhā medhātithir vasuḥ\lem  \corr;
                 agnīndhraś cāgnibāhuś ca medhā medhātithir vasuḥ \msCb,
                                agninvraścāgnivā{\il}{\lost} {\lost}{\lost}{\lost}dhātithir vvasuḥ \msCa, \om\ \Ed}%

jyotiṣmān dyutimān havyaḥ savanaḥ patra eva ca \veg\dontdisplaylinenum
            \var{\vcd havyaḥ savanaḥ patra eva ca\lem  \msCa\Ed; 
                                \uncl{havyaḥ savanaḥ patra eva ca} \msCb}%
            \paral{\textit{\vo {\normalfont \kb\ Brahmapurāṇa 5.9:}
                āgnīdhraś cāgnibāhuś ca medhyo medhātithir vasuḥ{\thinspace\danda}
                jyotiṣmān dyutimān havyaḥ savalaḥ putrasa\.mjñakaḥ{\thinspace\ketdanda}
                {\normalfont \kb\ Brahmāṇḍapurāṇa 1.13.104 and 1.14.9 
                \kb\ Padmapurāṇa 1.7.83 etc.}}}

agnibāhuś ca medhā ca patraś caiva trayo janāḥ\thinspace{\dandab} \dontdisplaylinenum
            \var{\va medhā ca\lem  \msCa\Ed; medhāś ca \msCb}%

sa\.msārabhayabhītena mokṣamārgasamāśritāḥ \veg\dontdisplaylinenum
            \var{\vd °mārga°\lem  \msCb\Ed; °mārga\.m \msCa}%

agnīdhra\.m prathamadvīpe abhyaṣiñcat priyavrataḥ\thinspace{\dandab} \dontdisplaylinenum
            \var{\va agnīdhra\.m\lem  \eme; agnindhra\.m \msCa, agnīndhra \msCb, agnindha\.m \Ed\oo
                 prathama°\lem  \Ed; prathama\.m \msCa\msCb}%
            \var{\vb abhyaṣiñcat\lem  \msCa\msCb; abhyaṣiñcata \Ed}%

plakṣadvīpeśvara\.m cakre nāmnā medhātithi\.m tathā \veg\dontdisplaylinenum
            \var{\vd medhātithi\.m tathā\lem  \msCa\Ed; medhātithitan tathā \msCb}%

vasuś ca śālmalīdvīpe abhiṣikto mahīpatiḥ\thinspace{\dandab} \dontdisplaylinenum
            \var{\va vasuś ca śālmalī°\lem  \msCb; {\lost}{\lost}{\lost}{\lost}{\lost}{\lost} \msCa, vasuñ ca śālmalī \Ed}%

jyotiṣmanta\.m kuśadvīpe rājānam abhiṣecayet \veg\dontdisplaylinenum

krauñcadvīpeśvara\.m cakre dyutimanta\.m nareśvara\thinspace{\dandab} \dontdisplaylinenum
            \var{\vb dyutimanta\.m nareśvara\lem  \msCa; dyutimantan nareśvaram \msCbpcorr, 
                        śvarañ cakre dyutimantan nareśvaram \msCbacorr,
                        dyutimanta\.m nareśvaraḥ \Ed}%

śākadvīpeśvara\.m havya\.m puṣkare savanaḥ sm\textsubring{r}taḥ \veg\dontdisplaylinenum
            \var{\vd savanaḥ\lem  \msCa\msCb; savana \Ed}%

madhye puṣkaradvīpasya parvato mānasottaraḥ\thinspace{\dandab} \dontdisplaylinenum

lokapālāḥ sthitās tatra caturbhiś caturo diśaḥ \veg\dontdisplaylinenum
            \var{\vd caturo diśaḥ\lem  \msCb\Ed; {\lost}{\lost}{\lost}{\lost}{\lost} \msCa}%

mahāvītaḥ sm\textsubring{r}to varṣo dhātakī ca narādhipa\thinspace{\dandab} \dontdisplaylinenum
            \var{\va mahāvītaḥ\lem  \msCa\msCb; mahānītaḥ \Ed\oo
                 sm\textsubring{r}to\lem  \msCa\Ed; sm\textsubring{r}tā \msCb}%

tasya bāhyaḥ samudro 'bhūt svādūdaka iti sm\textsubring{r}taḥ \veg\dontdisplaylinenum
            \var{\vo \om\ \Ed}%
            \var{\vc bāhyaḥ\lem  \msCa\Ed; bāhya \msCb}%
            \var{\vd °dūdaka\lem  \msCa\Ed; °dūka \msCb}%

catuḥṣaṣṭi sm\textsubring{r}to lakṣo yojanānā\.m narādhipa\thinspace{\dandab} \dontdisplaylinenum
            \var{\va catuḥ°\lem  \msCb; catu° \msCa\oo       
                 lakṣo\lem  \msCa; lakṣā \msCb, \om\ \Ed}%
            \var{\vb narādhipa\lem  \msCa; narādhipaḥ \msCb, \om\ \Ed}%

puṣkaradvīpam antaś ca kṣīrodo nāma sāgaraḥ \veg\dontdisplaylinenum

dvātri\.mśallakṣavistāraḥ śākadvīpabahirv\textsubring{r}taḥ\thinspace{\dandab} \dontdisplaylinenum
            \var{\va °vistāraḥ\lem  \msCa\msCb; °vistāraiḥ \Ed}%
            \var{\vb °bahirv\textsubring{r}taḥ\lem  \conj; °vahav\textsubring{r}ṇaḥ \msCa, °bahuv\textsubring{r}taḥ \msCb, °vahav\textsubring{r}ṇe \Ed}%

jaladaś ca kumāraś ca sukumāramaṇīcakaḥ \veg\dontdisplaylinenum
            \var{\vcd kumāraś ca sukumāramaṇīcakaḥ\lem  \msCb\Ed;
                                           kumā{\il}{\lost}{\lost}{\lost}{\lost}{\il}{\il}ṇīcakaḥ \msCa}%

kusumottaramodaś ca saptama\.m ca mahādrumam\thinspace{\dandab} \dontdisplaylinenum
            \var{\vb saptama\.m\lem  \msCa\msCb; saptamaś \Ed}%

havyaputrāḥ sm\textsubring{r}tāḥ sapta varṣanāma tathā sm\textsubring{r}taḥ \veg\dontdisplaylinenum

dvīpānta\.m dadhimaṇḍodakṣīrodārdha\.m vinirdiśet\thinspace{\dandab} \dontdisplaylinenum
            \var{\va °maṇḍoda°\lem  \msCb; °maṇḍādi° \msCa\Ed}%
            \var{\vb vinirdiśet\lem  \msCa\msCb; nirdiśet \Ed}%

krauñcadvīpasamudrānte sapta varṣās tu te sm\textsubring{r}tāḥ \veg\dontdisplaylinenum
            \var{\vc °dvīpa°\lem  \msCa\msCb; °dvīpe \Ed}%
            \var{\vd varṣās\lem  \msCa\msCb; varṣan \Ed}%

kuśalo manonugaś coṣṇaḥ yāvanaś cāndhakārakaḥ\thinspace{\dandab} \dontdisplaylinenum
            \var{\va kuśalo manonugaś coṣṇaḥ\lem  \msCb;
                                kuśalo manonugaś co\uncl{ṣṇaḥ} \msCa,
                                kuśalomnonugaś coṣṇaḥ \Ed}%
            \var{\vb yāvanaś cāndhakārakaḥ\lem  \msCb;
                                \uncl{yā}vana\uncl{ś cā}{\il}{\lost}{\lost}{\lost} \msCa,
                                yavanaś cāndhakārakaḥ \Ed}%

muniś ca dundubhiś caiva sutā dyutimatas tu vai \veg\dontdisplaylinenum
            \var{\vd sutā dyutimatas\lem  \msCa\msCb; sutadyutimanas \Ed}%

dadhyardhe gh\textsubring{r}tamaṇḍodaḥ kuśadvīpasamāv\textsubring{r}taḥ\thinspace{\dandab} \dontdisplaylinenum
            \var{\va gh\textsubring{r}ta°\lem  \msCa\msCb; dh\textsubring{r}ta° \Ed}%
            \var{\vb °dvīpa°\lem  \msCa\msCb; °dvīpaḥ \Ed}%

tatrāpi saptavarṣe ca nāmataḥ ś\textsubring{r}ṇu bhārata \veg\dontdisplaylinenum
            \var{\vc °varṣe\lem  \msCa\msCb; °varṣa\.m \Ed}%
            \var{\vd bhārata\lem  \msCa\Ed; bhārataḥ \msCb}%

udbhimān veṇumā\.mś caiva svairannālambano dh\textsubring{r}tiḥ\thinspace{\dandab} \dontdisplaylinenum
            \var{\va veṇumā\.mś caiva\lem  \msCa; veṇumā\.m va \msCb, dhenusāś caiva \Ed}%
            \var{\vb svaira°\lem  \msCa; svairā° \Ed}%

ṣaṣṭhaḥ prabhākaraś caiva kapilaḥ saptamaḥ sm\textsubring{r}taḥ \veg\dontdisplaylinenum
            \paral{\textit{{\normalfont Cf. Brahmapurāṇa 20.36--37ab: }
                 jyotiṣmataḥ kuśadvīpe ś\textsubring{r}ṇudhva\.m tasya putrakān{\thinspace\danda}
                 udbhido veṇumā\.mś caiva svairatho randhano dh\textsubring{r}tiḥ{\thinspace\ketdanda}
                 prabhākaro 'tha kapilas tannāmnā varṣapaddhatiḥ{\thinspace\danda}}}

gh\textsubring{r}tamaṇḍas tadardhena tasyānte madirodadhiḥ\thinspace{\dandab} \dontdisplaylinenum
            \var{\va °maṇḍas tadardhena\lem  \msCb; maṇḍotadardhena \msCa, maṇḍotardhena \Ed}%
            \var{\vb tasyānte madiro°\lem  \Ed;
                        \uncl{ta}{\lost}{\lost}{\lost}diro° \msCa, tasyāntemadhiro° \msCb}%

samantāc chālmalīdvīpo varṣāḥ saptaiva kīrtitāḥ \veg\dontdisplaylinenum
            \var{\vd varṣāḥ\lem  \msCb\Ed; varṣoḥ \msCa}%

śvetaś ca haritaś caiva jīmūto rohitas tathā\thinspace{\dandab} \dontdisplaylinenum
            \var{\vb rohita°\lem  \msCa\msCb; lohita° \Ed}%

vaidyuto mānasaś caiva suprabhaḥ saptamaḥ sm\textsubring{r}taḥ \veg\dontdisplaylinenum

madirodadhito 'rdhena jñeyas tv ikṣurasodadhiḥ\thinspace{\dandab} \dontdisplaylinenum
            \var{\va °dadhito\lem  \msCa\msCb; °dadhino \Ed}%
            \var{\vb jñeyas tv i°\lem  \msCa\msCb; jñeya tv i° \Ed}%

plakṣadvīpo v\textsubring{r}tas tena saptavarṣasamanvitaḥ \veg\dontdisplaylinenum

śāntaś ca śiśiraś caiva sukhadānanda eva ca\thinspace{\dandab} \dontdisplaylinenum
            \var{\va śāntaś ca śiśiraś\lem  \msCb\Ed; {\lost}{\lost}{\lost}{\lost}{\lost}raś \msCa}%

śivakṣemo dhruvaś caiva sapta medhātitheḥ sutāḥ \veg\dontdisplaylinenum
            \var{\vc śiva°\lem  \msCa; śivaśiva° \Ed}%

lavaṇodas tu tasyānte jambūdvīpasamāv\textsubring{r}taḥ\thinspace{\dandab} \dontdisplaylinenum
            \var{\va °das tu tasyānte\lem  \msCa\msCb; °dadhisyānte \Ed}%
            \var{\vb jambū°\lem  \msCa\Ed; ja\uncl{mbu}° \msCb\oo 
                 °dvīpa°\lem  \msCa\msCb; °dvīpā° \Ed\oo
                 °v\textsubring{r}taḥ\lem  \msCa\Ed; °v\textsubring{r}tāḥ \msCb}%

lakṣayojanavistāra upadvīpasamanvitaḥ \veg\dontdisplaylinenum
            \var{\vc °vistāra\lem  \msCa\msCb; °vistāro \Ed}%
            \var{\vd °dvīpa°\lem  \msCa\msCb; °dvipa° \Ed}%

aṅgadvīpo yavadvīpo malayadvīpa eva ca\thinspace{\dandab} \dontdisplaylinenum

śaṅkhadvīpakamudvīpo varāhadvīpa eva ca \veg\dontdisplaylinenum
            \var{\vo \om\ \msCb\ {\normalfont (eyeskip to 24.32cd)}}%
            \var{\vd eva ca\lem  \Ed; {\lost}{\lost} \msCa, \om\ \msCb}%

si\.mha barhiṇadvīpa\.m ca padmaś cakras tathaiva ca\thinspace{\dandab} \dontdisplaylinenum
            \var{\va si\.mha barhiṇa°\lem  \Ed; {\lost}{\lost}{\lost}rhiṇa° \msCa, \om\ \msCb}%
            \var{\vb padmaś cakra\lem  \msCa; \om\ \msCb, padmacakra° \Ed}%

vajraratnākaradvīpo ha\.msakaḥ kumudas tathā \veg\dontdisplaylinenum
            \var{\vo \om\ \msCb}%

lāṅgalo v\textsubring{r}ṣadvīpaś ca dvīpo bhadrākaras tathā\thinspace{\dandab} \dontdisplaylinenum

candradvīpaś ca sindhuś ca candanadvīpa eva ca \veg\dontdisplaylinenum
            \var{\vo \om\ \msCb}%
            \var{\vd candana°\lem  \msCa; \om\ \msCb, nandana° \Ed}%

upadvīpasahasrāṇi evamādīni kīrtitam\thinspace{\dandab} \dontdisplaylinenum
            \var{\vab \om\ \msCb}%


\alalfejezet{agnīdhraputrā jambudvīpe}
agnīdhro navavarṣeṣu navaputrān asiñcayat \veg\dontdisplaylinenum
            \var{\vc agnīdhro\lem  \eme; agnīndhra \msCa\msCb, agnīndhro \Ed}%
            \var{\vd °siñcayat\lem  \msCb; si{\il}{\lost} \msCa, °bhiṣiñcayat \Ed}%

nābhiḥ ki\.mpuruṣaś caiva harivarṣa ilāv\textsubring{r}taḥ\thinspace{\dandab} \dontdisplaylinenum
            \var{\va nābhiḥ\lem  \Ed; {\lost}{\lost} \msCa, nābhi \msCb}%

pañcama\.m ramyaka\.m varṣa\.m ṣaṣṭha\.m caiva hiraṇmayam \veg\dontdisplaylinenum
            \var{\vo \om\ \Ed}%

kuravaḥ saptamo jñeyo bhadrāśvaś cāṣṭamaḥ sm\textsubring{r}taḥ\thinspace{\dandab} \dontdisplaylinenum

navamaḥ ketumālo 'bhūn navavarṣāḥ prakīrtitāḥ \veg\dontdisplaylinenum
            \var{\vo \om\ \Ed}%
            \var{\vc °mālo\lem  \msCa; °māno \msCb, \om\ \Ed}%

himavaddakṣiṇe pārśve varṣo bhāratasa\.mjñitaḥ\thinspace{\dandab} \dontdisplaylinenum

atrāpi navabhedo 'bhūd bhāratātmajasambhavaḥ \veg\dontdisplaylinenum
            \var{\vcd 'bhūd bhāratātmaja°\lem  \msCb\Ed; {\lost}{\lost}{\lost}{\lost}{\lost}ja° \msCa}%

indradvīpaḥ kaśeruś ca tāmravarṇo gabhastimān\thinspace{\dandab} \dontdisplaylinenum

nāgadvīpas tathā saumyo gāndharvaś cātha vāruṇaḥ \veg\dontdisplaylinenum
            \var{\vc saumyo\lem  \msCb\Ed; saumyā \msCa}%
            \var{\vd gāndharva°\lem  \msCa\msCb; gandharva° \Ed}%

aya\.m ca navamo dvīpaḥ kumārīdvīpasa\.mjñitaḥ\thinspace{\dandab} \dontdisplaylinenum

dakṣiṇe hemakūṭasya varṣaḥ ki\.mpuruṣaḥ sm\textsubring{r}taḥ \veg\dontdisplaylinenum

niṣadho dakṣiṇapārśve harivarṣa iti sm\textsubring{r}taḥ\thinspace{\dandab} \dontdisplaylinenum

merumūle tu rājendra jñeyo varṣa ilāv\textsubring{r}taḥ \veg\dontdisplaylinenum

uttaraṇeṇa (uttareṇa?) tu nīlasya varṣa ramyaka ucyate\thinspace{\dandab} \dontdisplaylinenum

śveta-uttarato jñeyo varṣaramyahiraṇmayaḥ \veg\dontdisplaylinenum

tasya uttarato jñeyas triś\textsubring{r}ṅgavaraparvataḥ\thinspace{\dandab} \dontdisplaylinenum

tasya cottarapārśve tu varṣaḥ kuruvale sm\textsubring{r}taḥ \veg\dontdisplaylinenum

pūrva\.m bhadrāśvato jñeyaḥ ketumālas tu paścime\thinspace{\dandab} \dontdisplaylinenum

hima\.mvān hemakūṭaś ca niṣadho nīla eva ca \veg\dontdisplaylinenum

śvetaś ca ś\textsubring{r}ṅgavantaś ca ṣaḍ ete varṣaparvatāḥ\thinspace{\dandab} \dontdisplaylinenum

aśītinavatīlakṣaḥ - varṣaparvatam āyatam \veg\dontdisplaylinenum

himavān hemakūṭaś ca niṣadhaś ceti dakṣiṇa\thinspace{\dandab} \dontdisplaylinenum

śvetaś caivatriś\textsubring{r}ṅgaś ca nīlaś caiva tathottare \veg\dontdisplaylinenum

niṣadho nīlamadhye tu meruḥ śailamanoramaḥ\thinspace{\dandab} \dontdisplaylinenum

praviṣṭaṣoḍaśādhas tā\.m caturāśītim ucch\textsubring{r}taḥ \veg\dontdisplaylinenum

yojanānā\.m sahasrāṇi dvātri\.mśad ūrdha ! vist\textsubring{r}taḥ\thinspace{\dandab} \dontdisplaylinenum

brahmāmanovatī nāma pureva satimadhyame \veg\dontdisplaylinenum

devarājo 'marāvatyām agnis tejovatī pure \veg\dontdisplaylinenum

yamaḥ sa\.myamanī nāma nitya\.m vasati bhūpate\thinspace{\dandab} \dontdisplaylinenum

nai\textsubring{r}tir vasati nitya\.m ramye śuddhavatī pure \veg\dontdisplaylinenum

varuṇo bhogavatyā\.m tu vāyor gandhavatī purī\thinspace{\dandab} \dontdisplaylinenum

mahodayāpurī ramyā somasyālayara\.m sm\textsubring{r}tam \veg\dontdisplaylinenum

yaśovatī purī ramyānnityam āste triśūlinaḥ\thinspace{\dandab} \dontdisplaylinenum

tatragaṅgā catuḥbhinnā nipatantī mahītale \veg\dontdisplaylinenum

uttare paścime caiva pūrvadakṣiṇatas tathā\thinspace{\dandab} \dontdisplaylinenum

pūrva\.m gaṅgā sravatyāccālakānandā ca dakṣiṇe \veg\dontdisplaylinenum

śītā paścimagā gaṅgā bhadrasomā tathottare\thinspace{\dandab} \dontdisplaylinenum

ṣaṣṭiyojanasāhasra\.m nirālambā nipatya ca \veg\dontdisplaylinenum

bhadrāśva\.m plāvayitvā tu vanāny upavanāni ca\thinspace{\dandab} \dontdisplaylinenum

droṇasthalī girīṇā\.m ca atikramyārṇava\.m gatā \veg\dontdisplaylinenum

tathaivālakanandā ca gatāśailenanimnagā\thinspace{\dandab} \dontdisplaylinenum

gaṅgā bhāratavarṣa\.m ca praviṣṭālavaṇo dadhim \veg\dontdisplaylinenum

plāvayitvā sthalīn sarvān mānuṣākaluṣāpahā\thinspace{\dandab} \dontdisplaylinenum

paścimena gatāgaṅgā sītānāmā ca bhārataḥ \veg\dontdisplaylinenum

plāvayet ketumālā\.m ca kṣetraśaivavanasthalīm\thinspace{\dandab} \dontdisplaylinenum

atikramyārṇavagatā sthalīdroṇī ca nimnagā \veg\dontdisplaylinenum

bhadrasomanadīty eva\.m plāvayitvottara\.m kurun\thinspace{\dandab} \dontdisplaylinenum

sthalī prasravaṇadroṇīm atikramyārṇava\.m gatā \veg\dontdisplaylinenum

mero vai dakṣiṇe pārśve jambūv\textsubring{r}kṣaḥ sanātanaḥ\thinspace{\dandab} \dontdisplaylinenum

tena nāmāṅkito rājan jambūdvīpa iti śrutam \veg\dontdisplaylinenum

koṭīṣoḍaśabhiś caiva ayutāni trayodaśa\thinspace{\dandab} \dontdisplaylinenum

adhordhayāma rājendra kṣityāvaraṇam antataḥ \veg\dontdisplaylinenum

navalakṣādhika\.m rājan pañcakoṭī mahī sm\textsubring{r}tā\thinspace{\dandab} \dontdisplaylinenum

yojanānā\.m tu vijñeyaḥ p\textsubring{r}thivyāyām avistarāt \veg\dontdisplaylinenum

svādūdakasya ca bahir lokāloko mahāgiriḥ\thinspace{\dandab} \dontdisplaylinenum

kañcanidviguṇābhūmi tasmād giribahi sm\textsubring{r}taḥ \veg\dontdisplaylinenum

tasmād bāhyaḥ samudro bhūd garbhādeti samudrarāṭ\thinspace{\dandab} \dontdisplaylinenum

aṣṭāvi\.mśatika\.m lakṣa\.m śatalakṣāṇi vistaram \veg\dontdisplaylinenum

etad bhūrlokavistāro hy ata ūrdhva bhuvaḥ sm\textsubring{r}taḥ\thinspace{\dandab} \dontdisplaylinenum

svarlokāsyapareṇaiva maharlokam ataḥ param \veg\dontdisplaylinenum

janalokas tapaḥ satya\.m kramaśaḥ parikīrtitam\thinspace{\dandab} \dontdisplaylinenum

brahmalokaḥ sm\textsubring{r}taḥ satya\.m viṣṇulokam ataḥ param \veg\dontdisplaylinenum


\alalfejezet{śivalokaḥ}
tasmāt pareṇa bodhavya\.m divyadhyānapura\.m mahat\thinspace{\dandab} \dontdisplaylinenum

sahasrabhaumaprāsāda\.m vaidūryamaṇitoraṇam \veg\dontdisplaylinenum

nānāratnavicitrāṇi nānābhūtagaṇākulam\thinspace{\dandab} \dontdisplaylinenum

sarvakāmasam\textsubring{r}ddhāni pūrṇa\.m tatra manoharaiḥ \veg\dontdisplaylinenum

tatra si\.mhāsane divye sarvaratnavibhūṣite\thinspace{\dandab} \dontdisplaylinenum

tatrāste bhagavān rudraḥ somāṅkitajaṭādharaḥ \veg\dontdisplaylinenum

tryakṣatribhuvanaśreṣṭhas triśūlī tridaśādhipaḥ\thinspace{\dandab} \dontdisplaylinenum

devyā saha mahābhāgo gaṇaiś ca parivāritaḥ \veg\dontdisplaylinenum

skandanandipurogaś ca gaṇakoṭiśatākulaḥ\thinspace{\dandab} \dontdisplaylinenum

anekarudrakanyābhi rūpiṇībhir alaṅkitaḥ \veg\dontdisplaylinenum
            \var{\vc °kanyābhi°\lem  \corr; °kanyabhi° \Ed}%

tatra puṇyanadī sapta sarvapāpāpanodanī\thinspace{\dandab} \dontdisplaylinenum

suvarṇavālukādivyā ratnapāṣāṇaśobhitā \veg\dontdisplaylinenum

pāvanī ca vareṇyā ca varārhāvaradā varā\thinspace{\dandab} \dontdisplaylinenum

vareśāvarabhadrā ca suprasannā jalāśivā \veg\dontdisplaylinenum

anekakusumārāmā ratnapuṣpaphaladrumāḥ\thinspace{\dandab} \dontdisplaylinenum

anekaratnaprākārā yojanāyutam ucchritāḥ \veg\dontdisplaylinenum

ahi\.msāsatyaniratāḥ kāmakrodhavivarjitāḥ\thinspace{\dandab} \dontdisplaylinenum

dhyānayogaratānitya\.m tatra modanti te narāḥ \veg\dontdisplaylinenum 

tatra gomātaras sarvā nivasanti yatavratāḥ\thinspace{\dandab} \dontdisplaylinenum

golokaḥ śivalokaś ca eka eva vidhīyate \veg\dontdisplaylinenum

tasmād ūrdha\.m para\.m jñeya\.m sthānatrayam anuttamam\thinspace{\dandab} \dontdisplaylinenum

kandagaurī maheśāna\.m nityaśuddha\.m para\.m śivam \veg\dontdisplaylinenum

dinak\textsubring{r}t koṭisaṅkāsam anopamya\.m sanātanam\thinspace{\dandab} \dontdisplaylinenum

ādityāda ! śivāntaś ca dvistheṇordhvakramaiḥ m\textsubring{r}staḥ (sm\textsubring{r}taḥ) \veg\dontdisplaylinenum


\alalfejezet{śāstravarṇanā}
\ujvers\nemsloka 
abhyantare tat kathito 'dya sāra\.m
\dontdisplaylinenum
            \var{\va abhyantare tat ka°\lem  \msCa\msCb; atyantaretka° \Ed}%

\nemslokab 
kim anya rājan kathayāmi sāram \danda\dontdisplaylinenum
            \var{\vb kim anya rā°\lem  \msCa\Ed; kim anyad rā° \msCb}%

\nemslokac 
jñānārṇava\.m kīrtita dharmasāram
\dontdisplaylinenum
            \var{\vc jñānārṇava\.m kīrtita dharmasāram\lem  \msCb\Ed;
                jñānārṇṇa\uncl{ṅkīrti}{\lost}{\lost}{\lost}{\lost}{\lost} \msCa}%

\nemslokad 
purāṇavedopaniṣatsusāram \veg\dontdisplaylinenum

\ujvers\nemsloka 
yathā hi rājā parivāramadhye
\dontdisplaylinenum
            \var{\va °vāramadhye\lem  \msCa; °vāraṇai \msCb, °cāramadhye \Ed}%

\nemslokab 
yathāntavartī bahivartin eva \danda\dontdisplaylinenum
            \var{\vb yathāntava°\lem  \msCb\Ed; yathāntarvva \msCa\oo
                 °vartin eva\lem  \msCb\Ed; vartti\uncl{ne}va \msCa}%

\nemslokac 
bhuñjanti bhogān satatāntavartī
\dontdisplaylinenum
            \var{\vc bhuñjanti bhogān\lem  \msCb\Ed; \uncl{bhuñja} {\lost}{\lost}{\lost} \msCa\oo
                 satatāntavartī\lem  \msCa\Ed; satatānnavartī \msCb}%

\nemslokad 
kleśādhika\.m nitya bahiḥsthitānām \veg\dontdisplaylinenum
            \var{\vd bahiḥ°\lem  \msCa\Ed; bahi° \Ed}%

\ujvers\nemsloka 
yathaiva rājā kariṇo 'ntadantam
\dontdisplaylinenum
            \var{\va kariṇo 'ntadantam\lem  \msCb; kariṇo 'ntardantam \msCa, kariṇāntadantadattam \Ed}%

\nemslokab 
bhuñjanti bhogān satata\.m narendra \danda\dontdisplaylinenum
            \var{\vb bhuñjanti\lem  \msCb\Ed; bhujanti \msCa}%

\nemslokac 
yudhyeta rājā bahirdantabhogair
\dontdisplaylinenum
            \var{\vc rājā\lem  \msCa\Ed; rāja \msCb\oo
                 bahirdantabhogair\lem  \msCa\msCb; bahidattabhogair \Ed}%

\nemslokad 
yadantara\.m paśya samānajātam \veg\dontdisplaylinenum
            \var{\vd yadantara\.m paśya samānajātam\lem  \msCb; yadantare paśya samānajātam \Ed;
                                 yadanta\uncl{re} {\lost}{\lost}{\lost}{\lost}najātam \msCa}%

\ujvers\nemsloka 
na dānatulya\.m tv abhayapradasya
\dontdisplaylinenum

\nemslokab 
na yajñatulya\.m jita-indriyasya \danda\dontdisplaylinenum

\nemslokac 
na cārthatulya\.m jitakāminaś ca
\dontdisplaylinenum
            \var{\vc °kāminaś ca\lem  \Ed; kāmina{\lost}{\lost} \msCa}%

\nemslokad 
na dharmatulya\.m damakāmitasya \veg\dontdisplaylinenum
            \var{\vd na dharmatulya\.m\lem  \Ed; {\lost}{\lost}{\lost}{\lost}{\lost} \msCa, \om\ \msCb\oo
                 damakāmitasya\lem  \msCa; \om\ \msCb, damakāminasya \Ed}% 

\ujvers\nemsloka 
bahvantara\.m naiva hi dharmayoś ca
\dontdisplaylinenum

\nemslokab 
kleśādhika\.m bāhyaphalālpasāram \danda\dontdisplaylinenum

\nemslokac 
yad atra dharma\.m phalanaiṣṭhikasya
\dontdisplaylinenum
            \var{\vc °naiṣṭhikasya\lem  \Ed; °naiśikasya \msCb}%

\nemslokad 
na tulya koṭīśatayājināpi \veg\dontdisplaylinenum

\ujvers\nemsloka 
etat pavitra\.m parama\.m sadharmam
\dontdisplaylinenum

\nemslokab 
purā yathokta\.m parameśvareṇa \danda\dontdisplaylinenum

\nemslokac 
mayāpi tulya\.m kathita\.m yathāvat
\dontdisplaylinenum

\nemslokad 
purāṇavedopaniṣatsusāram \veg\dontdisplaylinenum

\ujvers\nemsloka 
sadojasaubhāgyam atīva medhā
\dontdisplaylinenum
            \var{\va sadoja°\lem  \Ed; sadojaḥ \msCb}%

\nemslokab 
nirutsukaḥ saumyam anuttama\.m ca \danda\dontdisplaylinenum
            \var{\vb nirutsukaḥ\lem  \Ed; nirutsuka° \msCb}%

\nemslokac 
suputrapautra\.m na vichinnagotram
\dontdisplaylinenum

\nemslokad 
bhavanti vidyādharalokapūjyam \veg\dontdisplaylinenum

\ujvers\nemsloka 
yaśaśriya\.m kīrtir atīva tejo
\dontdisplaylinenum
            \var{\va yaśaḥ°\lem  \msCb; yaśa° \Ed}%

\nemslokab 
janapriyo dhānyadhanāyuv\textsubring{r}ddhiḥ \danda\dontdisplaylinenum
            \var{\vb °v\textsubring{r}ddhim\lem  \msCb; °v\textsubring{r}ddhiḥ \Ed}%

\nemslokac 
prabodhaprajñārujadharmav\textsubring{r}ddhim
\dontdisplaylinenum
            \var{\vc °v\textsubring{r}ddhim\lem  \Ed; °v\textsubring{r}ddhi \msCb}%

\nemslokad 
bhavanti te śāstrasadābhiyogī \veg\dontdisplaylinenum
            \var{\vd ta\.m\lem  \msCb; te \Ed}%

\ujvers\nemsloka 
yaśasvinī āryasuvarṇaś\textsubring{r}ṅgī
\dontdisplaylinenum

\nemslokab 
vedāntavipradvijagāyaneṣu \danda\dontdisplaylinenum

\nemslokac 
dattvā phala\.m tīrtham anuttameṣu
\dontdisplaylinenum
            \var{\vc °nuttameṣu\lem  \Ed; °numeṣu \msCb}%

\nemslokad 
ś\textsubring{r}ṇvanti ye tasya bhavet sapuṇyam \veg\dontdisplaylinenum

\ujvers\nemsloka 
daśādhika\.m vācayituś ca puṇyam
\dontdisplaylinenum
            \var{\va vācayituś ca\lem  \msCb; vā ca catuś ca \Ed}%

\nemslokab 
śatādhika\.m yaḥ paṭhati prabhāte \danda\dontdisplaylinenum

\nemslokac 
sahasraśaḥ pustak\textsubring{r}tasya puṇyam
\dontdisplaylinenum

\nemslokad 
pare 'bhyaste kīrtayate 'yutāni \veg\dontdisplaylinenum
            \var{\vd pare\lem  \msCb; paro \Ed\oo
                 kīrta°\lem  \msCb\pcorr; kīrti° \msCb\acorr}%

\ujvers\nemsloka 
adhītya yasyoragata\.m suśāstram
\dontdisplaylinenum

\nemslokab 
samastamadhyāyam anukramena \danda\dontdisplaylinenum

\nemslokac 
daśāyutāṅgo dadatuś ca puṇyam
\dontdisplaylinenum
            \var{\vc daśāyutāṅgo dadatu°\lem  \Ed; daśāyu\uncl{taṅga de}datu° \msCb}%

\nemslokad 
labhaty asa\.mdigdhayathādinaika\.m \veg\dontdisplaylinenum
            \var{\vd °yathādinaika\.m\lem  \Ed; °ya\uncl{thādi}naika\.m \msCb}%

\ujvers\nemsloka 
yeneda\.m śāstrasāram avikalamanasā yo 'bhyaset tatprayatnāt
\dontdisplaylinenum
            \var{\va °sāram a°\lem  \msCa\msCb\msNa\Ed; °sāra\.mm a° \msCc
                 °vikala°\lem  \msCa\msNa\Ed; °kila° \msCb\oo
                 bhyaset tatpra°\lem  \mssCaCbCc\msNa;  bhyaseta pra° \Ed}%

\nemslokab 
vyakto 'sau siddhayogī bhavati ca niyata\.m yas tu cittaprasannaḥ \danda\dontdisplaylinenum
            \var{\vb sau\lem  \msCb\msCc\msNa\Ed; so \msCa\oo
                 citta°\lem  \mssCaCbCc\Ed; vinna° \msNa\oo
                 °prasannaḥ\lem  \msCa\msCc\msNa\Ed; °prayānnā \msCb}%

\nemslokac 
pitrya\.m yo gītapūrva\.m pratidina śataśa uddhriyante ca sarve
\dontdisplaylinenum
            \var{\vc pitrya\.m yo gītapūrva\.m\lem  \mssCaCbCc\msNa; nitya yo dhītayota pūrvva\.m \Ed\oo
                °dinaśataśa uddhriyante ca sarve\lem  \msNa;
                     \uncl{dina}{\lost}{\lost}{\lost}{\lost}{\lost}{\lost}{\lost}{\lost}{\lost}{\lost} \msCa,
                     °dinaśatasa udviyante ca sarve \msCb,
                     °dinaśatasa udriyante ca sarve \msCc,
               °dinaśataśo urddhi yante ca sarve \Ed}%

\nemslokad 
ātmāna\.m nirvikalpa\.m śivapadam asama\.m prāpnuvantīha sarve \veg\dontdisplaylinenum

\vers


\jump
\begin{center}
\ketdanda iti v\textsubring{r}ṣasārasa\.mgrahe śāstravarṇanā nāma caturvi\.mśatitamo 'dhyāyaḥ samāptaḥ\ketdanda
\end{center}
\dontdisplaylinenum\vers 
            \var{{\normalfont Colophon:}  °varṇanā\lem  \msCa\msCb; °varṇano \Ed\oo 
                           dhyāyaḥ samāptaḥ\lem  \msCa\msCb; dhyāyaḥ \Ed}%


\jump
\begin{center}
\ketdanda v\textsubring{r}ṣasārasa\.mgrahaḥ samāpta iti\ketdanda
\end{center}
\dontdisplaylinenum\vers 
            \var{v\textsubring{r}ṣasārasa\.mgrahaḥ samāpta iti\lem  \msCa; v\textsubring{r}ṣasārasa\.mgraha\.m samāpta iti \msCb, 
                                                                \om\ \Ed}%
